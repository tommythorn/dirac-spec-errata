\label{segol}

This section defines the signed interleaved exp-Golomb data format and the operation
of the $read\_sint()$ function.

The code for the signed interleaved exp-Golomb data format consists of the
unsigned interleaved exp-Golomb code for the magnitude, followed by a sign bit
for non-zero values (figure \ref{segolcodings}).

\begin{figure}[h]
\begin{tabular}{l|c}
Bit sequence & Decoded value \\
\hline\\
1                 &  0\\
0 0 1 0           &  1\\
0 0 1 1           &  -1\\
0 1 1 0            &  2\\
0 1 1 1            &  -2\\
0 0 0 0 1 0         &  3\\
0 0 0 0 1 1         &  -3\\
0 0 0 1 1 0         &  4\\
0 0 0 1 1 1         &  -4\\

\end{tabular}

\caption{Example conversions from signed interleaved exp-Golomb-coded values 
to signed integers \label{segolcodings}}
\end{figure}

The decoding operation is as follows.

\begin{pseudo}{read\_sint}{}
\bsCODE{value = read\_uint()}
\bsIF{read\_bool()}
  \bsCODE{value *= -1}
\bsEND
\bsRET{value}

\end{pseudo}
