%%%%%%%%%%%%%%%%%%%%%%%%%%%%%%%%%%%%%%%%%%%%%%%%%%
% - This chapter gives an informative overview - %
% - Dirac video coding concepts                - % 
%%%%%%%%%%%%%%%%%%%%%%%%%%%%%%%%%%%%%%%%%%%%%%%%%%

\begin{comment}
\subsection{A general introduction to video compression for beginners}
This is a simple guide to video coding. You can miss this subsection out
if you already know the basics.

Video signals are made up of a succession of still pictures, displayed
one after the other. Each picture is made up from a series of elementary
units called pixels, arranged in a raster of lines.

The raw total capacity required for the transmission of moving pictures
is the product of the number of pixels per picture, pictures per second,
colours in use and the quantising accuracy adopted. In nearly every
field of application, the resultant raw bandwidth required exceeds that
available by more than an order of magnitude.  If we keep the same size
pictures, then the only variable we have with the simple system is to
change the quantisation accuracy and lowering the accuracy rapidly leads
to poor quality pictures.

The solution which has been used from before 1970 is to make a
prediction of the value of each pixel using information that should be
available in the decoder. In general, if the prediction is good, then
the difference between reality and the prediction is small. The entropy
of the difference signal is low, and so we need less capacity to deliver
it. One simple solution was to recognise that the quantisation accuracy
required for the difference signal could be coarser than that required
for the original picture, and so fewer bits are needed to deliver the
difference signal than for the original.

This has formed the basis for many of the early video compression
systems.  The elements that have changed over the years have been

\begin{itemize}
    \item the algorithms used for making the prediction, and
    \item the formats used for delivering the difference signal
    efficiently to the receiver.
\end{itemize}

Many of the changes have been enabled by the availability of better
electronic circuitry. Real-time operation puts an upper limit on the
time available for processing.  We can only use more sophisticated
algorithms if we can carry out the calculations in the time available.
Fortunately, the improvement in the speed of hardware has almost matched
the development of algorithms.

Early algorithms made use of simple predictions predicting one pixel
from others nearby in the picture, or (spatial prediction in the
jargon).

Predictions were helped when field-stores became affordable (and smaller
than a small garden shed) and it became possible to use information from
the previous field (temporal prediction in the jargon). This process
works well when the picture is still, but is less effective when the
pictures depict a lot of movement.

This led to a whole raft of developments, seeking to identify motion in
the picture. Knowing how parts of the scene are moving (motion
estimation in the jargon) allowed much more accurate prediction than had
been possible hitherto. These developments led eventually to mature
products such as the MPEG 2 compression system.

Having created a good prediction, it is inevitable that it will not be
perfect. The error signal is the signal we wish to convey to the
decoder.  In any system, the raw error signal is at least as
bandwidth-hungry as the original. In most cases it is slightly more
hungry.

However, the error signal has some physical properties which help us to
reduce its bandwidth. In information theory terms, the error signal is
rarely totally random, real pictures have properties which distinguish
them from random noise.  We can therefore use some of these properties
to reduce the bit-rate further.

The spectrum of the error signal is usually heavily biased towards the
low frequency end. This is a direct consequence of the prediction
process usually being reasonably good. It also turns out that the eye is
less sensitive to small inaccuracies in the high frequency components of
the error signal.

Taken together, these observations provide the potential for coarser
quantisation or omission of some of the components of the error signal.

There are several methods available if we wish to translate the temporal
signal we started with into the frequency domain. Although many coding
systems use Discrete Cosine Transforms, we preferred to use the Discrete
Wavelet Transform. This approach divides the signal into higher and
lower frequency sub-bands. By quantising the different sub bands
appropriately, we achieve significant reduction in bandwidth.

Finally, when all the information for this system is packed together,
there is still a structure it is not statistically random. Information
theory says that we can use entropy coding to further increase the
randomness, and thereby reduce the bandwidth.  One of the more powerful
of these is arithmetic coding the system adopted in Dirac.  So we have
now outlined some of the key elements of a modern coder.


Fig XX. An outline of a typical modern video coder.

It is worth a quick look at the simple representation of the encoder in
Fig. XX as it leads us to understand the receiver topology.

The input signal $V_{in}$ is compared with the prediction $P$ to produce
an error signal $e$.  This is then compressed and passed (with the
various elements of compression metadata) to the arithmetic coder for
transmission.

The prediction is created from a local version of the signal for a very
good reason.  This is the signal that the decoder is able to recreate
with the information available to it. The signal delivered to the
receiver  $e_{TQ}$ allows the receiver to recreate a close copy of the
prediction error $e$. If we compare the two signals,

\begin{align*}
  V_{in} &= P + e \\
  V_{local}	&= P + e'
\end{align*}

As the difference between the two error signals $e$ and $e'$ is the
distortion introduced by the compression algorithm, the local signal is
a close approximation to the input signal. Looking carefully at Fig. XX,
we can see that we can create a decoder using a subset of the encoder.
This is shown in Fig. XXX.

It is quite clear that the main elements of the receiver are duplicated
in the encoder.  The encoder has to maintain a local decoder within it,
in part so that the result of the compression can be monitored at the
time of compression, but mainly because compressed pictures must be used
as reference frames for subsequent motion compensation else the encoder
and the decoder will not remain in synchronism.

The motion vectors are delivered from the encoder as metadata. This
avoids the need to analyse motion vectors in the receiver, and allows
the encoder considerable flexibility in the choice of appropriate motion
vectors.  We will now move on to consider the application of this
technology to Dirac.


Fig. YY An outline of the decoder
\end{comment}

\subsection{Outline of the compression methods used in Dirac}
The Dirac video codec uses four main techniques to compress the signal:

\begin{itemize}
    \item Prediction using motion compensation to exploit temporal redundancy
    \item Wavelet transformation to exploit spatial redundancy
    \item Quantisation of coefficients to form an approximate description of the picture data
    \item Entropy coding of the resulting quantised values to maximise efficiency
\end{itemize}


Initially, similarities between frames (temporal redundancy) are
exploited to predict one frame from another. The process is aided by
motion vectors - metadata detailing where a particular pixel in the
predicted frame might have been in the reference frame. It is a bit
wasteful to assign a motion vector to each pixel, so pixels are
aggregated in blocks. In Dirac, blocks are overlapping. This reduces
some of the artefacts found at the boundaries of blocks in some earlier
systems. The motion is calculated to sub pixel accuracy.

Once the prediction has been made, it is compared with the actual image
to be transmitted.  The resulting difference (error) is potentially
greater in range than the original signal. To reduce it the difference
signal is then transformed using the discrete wavelet transform. This
process, for the majority of video sequences, produces coefficients
which are largely zero or near zero, and most of the non-zero
coefficients are concentrated near at the lower frequency end of the
range. The properties of the eye allow us to coarsely quantise the high
frequency coefficients.

Both the motion vectors and the quantised coefficients are then further
compressed using arithmetic coding.

\subsection{Prediction using motion compensation}

The object of motion compensation is to try to predict the picture in
one image from others in the sequence. In television, the picture
usually contains moving objects, so a key part of the process is to
identify the moving objects and their the details of the motion.

In Dirac, the motion vector is merely an indication of which pixel in
the reference picture can be used as a good prediction for a particular
pixel in the current picture.

\subsubsection{Types of picture}

Three types of picture are defined.

Intra pictures are coded without reference to other pictures in the
sequence. These pictures form a useful point to start the decoding
process. They also have uses if low-delay coding is required - a
sequence of Intra pictures has mimimum delay, at the expense of
potentially greater bandwidth requirements. There is a third use for
Intra pictures: when the sequence is highly dynamic, prediction using
motion compensation may be impractical. In such instances, a coder may
choose to default to a sequence of Intra frames.

Inter pictures are coded with reference to other pictures in the
sequence, and are split into two types:
\begin{itemize}
	\item references for other pictures
	\item not references for other pictures
\end{itemize}

Within the receiver, we have to arrange that all the references for
pictures are available at the right time. This means that pictures are
often delivered in a different sequence from the display sequence, to
ensure that the buffer memory in the receiver has the necessary
information to decode pictures when it is needed.

\subsubsection{Blocks}

In theory, we could define a motion vector for every pixel in the image.
This would be extremely data intensive and of no practical value.

Instead, pixels are grouped into small regions or blocks, with a single
motion vector assigned to each block. Ideally these blocks are large -
thus minimising the amount of information we have to transmit. However,
the larger the block, the greater the chance that we have more than one
object or region, and so a single motion vector might be inappropriate.

Although there are several standard block sizes identified within the
specification, you can choose your own, and that could be as large as
the original picture. This would not be unreasonable for a picture which
is a static scene, with the camera being panned across it - although
there are other ways of identifying this in Dirac.

One of the hard tasks the coder has is to identify which is the optimum
size of block, and the best compromise for the motion vector in the
block.

When we have created the prediction, we then calculate the error. The
error is often greatest at the block boundaries, as this is where the
motion vector is often least accurate.

To overcome this, Dirac overlaps blocks. Each pixel in the overlap
regions uses a weighted prediction, incorporating information from all
the blocks it may lie in. This smooths the error signal, and making the
following wavelet transformation much more effective.

\subsubsection{Global motion parameters}

Although we said that it is of no practical value to specify a motion
vector for each pixel, there are exceptions.

Some types of motion are more likely to be features of the camera than
the scene. For example, a camera may pan, tilt, rotate, zoom, sheer, or
change of perspective through, say tracking.

If the coder can identify such motion, then Dirac permits the motion to
be signalled by a small set of parameters, which in effect assign a
motion vector to each pixel.

\subsubsection{Accuracy}
Often, we find that the motion vectors required to match a pixel in a
reference frame to the predicted frame are not integer values.

In Dirac, we can specify motion vectors to 1/8 pixel accuracy.

This means that the coder and decoder have to carry out a process that
is effectively an upconversion of the signal.

\subsubsection{Intra frames}
The Intra fields are not predicted using motion compensation.

This leaves us with a potential problem. The wavelet transform process
assumes that we have a signal which tends to a zero average.

The Intra field is processed to give it a zero mean by removing the DC
component (average value) of each block from the signal. The DC
components are signalled separately. In effect the DC components are a
local spatial prediction, rather than a temporal prediction as applied
in motion compensation.

\subsection{Wavelet tranforms}
The consequence of processing the Intra frames by removing the DC
values, and the Inter frames by removing the predicted values gives us a
difference signal. This difference signal is hopefully largely zero, but
can have some large peaks, of either polarity. The signal usually has a
large amount of low-frequency energy, but with occasional elements of
high frequency as the prediction process gets it wrong.

Conventional theory says that we can manipulate this signal in the
frequency domain to reduce the amount of information we need to
transmit. The properties of the eye are such that many of the higher
frequency components are less sensitive to coarse quantisation.

We can use wavelet transforms in Dirac. The wavelet transform
decorrelates the data in a roughly frequency-sensitive way, and
preserves the fine details of images better than the ubiquitous Discrete
Cosine Transform.

\subsubsection{Wavelet analysis}

Put simply, wavelet analysis splits a signal into a low and a high
frequency component, and then subsamples the two partial streams by a
factor of two.

The two filters used to split the signal are called the analysis
filters. It is not possible to use just any pair of half-band filters to
do this. There is an extensive mathematical theory of wavelet filter
banks. Appropriately chosen filters allow us to undo the aliasing
introduced by the critical sampling in the down conversion process and
perfectly reconstruct the signal.

Iterative use of a wavelet process allows us to decompose the
low-frequency component into successively lower components.

For pictures, we apply wavelet filters in both the horizontal and
vertical directions. This results in four so-called subbands, termed
Low-Low (LL), Low-High (LH) High-Low (HL), and High-High (HH). The LL
band can be iteratively decomposed to create a wavelet transformation of
many levels.


\subsubsection{Parent-child relationships}
In wavelet analysis, each subband represents a filtered and subsampled
version of the picture. Because all the subbands are derived from a
single source image, there is likely to be some form of relationship
between the images in the different subbands.

The coefficients of each subband relate to specific areas in the image.
We find that there is often a correlation between these specific areas
in the different subbands.

The subsampling structure means that a coefficient in the lowest level
corresponds to  2 by 2 block of coefficients in the next level, and so
on up the levels. In the jargon, the low-level component is referred to
as the parent and the higher-level component is referred to as the
child.

When coding picture features (edges on objects especially), significant
coefficients are found distributed across subbands, in position related
by the parent-child structure.

A coefficient of a child is more likely to be significant if its parent
is also significant. Children with zero or small parents seem to have
different statistics from children with large parents or ancestors.

These features allow us to entropy code the wavelet coefficients after
they have been quantised.

\subsection{Entropy coding}

Transmission of the raw motion vectors and wavelet coefficients is
inefficient. There are still many redundancies in the data, and the form
of the data is itself suboptimum.

Coding of the motion vectors is especially important for codecs with a
high level of motion accuracy (quarter or eighth pixel say). Motion
vector coding and decoding is quite complicated, since significant gains
in efficiency can be made by choosing a good prediction and entropy
coding structure.

The processes of removing the redundancies is probably one of the most
complicated part of the codec. Similar processes are used for coding the
motion vectors and the wavelet coefficients. In various ways they use a
combination of

\begin{itemize}
	\item prediction,
	\item binarisation,
	\item context modelling and
	\item adaptive arithmetic coding.
\end{itemize}

\subsubsection{Entropy coding of wavelets}

The entropy coding used by Dirac in wavelet subband coefficient coding
is based on three stages: binarisation, context modelling and adaptive
arithmetic coding.

Figure: Entropy coding block diagram

The purpose of the first stage is to provide a bitstream with easily
analysable statistics that can be encoded using arithmetic coding which
can adapt to those statistics, reflecting any local statistical
features.

Binarisation is the process of transforming the multi-valued coefficient
symbols into bits. The resulting bitstream can then be arithmetic coded.

Transform coefficients tend to have a roughly Laplacian distribution,
which decays exponentially with magnitude. This suits so-called unary
binarization. Unary codes are simple variable-length codes in which
every non-negative number $N$ is mapped to $N$ zeros followed by a 1:


\begin{verbatim}
U(0)    =   1
U(1)    =   0   1
U(2)    =   0   0   1
U(3)    =   0   0   0   1
U(4)    =   0   0   0   0   1
U(5)    =   0   0   0   0   0   1
U(6)    =   0   0   0   0   0   0   1
Bins:       1   2   3   4   5   6   7
\end{verbatim}

For Laplacian distributed values, the probability of $N$ occurring is
$2-(|N|+1)$, so the probability of a zero or a 1 occurring in any unary
bin is constant. So for an ideal only one context would be needed for
all the bins, leading to a very compact and reliable description of the
statistics. In practice, the coefficients do deviate from the Laplacian
ideal and so the lower bins are modelled separately and the larger bins
lumped into one context.

The process is best explained by example. Suppose one wished to encode
the sequence:

\begin{verbatim}
-3 0 1 0 -1
\end{verbatim}

When binarized, the sequence to be encoded is:

\begin{verbatim}
0 0 0 1 | 0 | 1 | 0 1 | 1 | 1 | 0 1 | 0
\end{verbatim}

The first 4 bits encode the magnitude, 3. The first bit is encoded using
the statistics for Bin1, the second using those for Bin 2 and so on.
When a 1 is detected, the magnitude is decoded and a sign bit is
expected. This is encoded using the sign context statistics; here it is
0 to signify a negative sign. The next bit must be a magnitude bit and
is encoded using the Bin 1 contexts; since it is 1 the value is 0 and
there is no need for a subsequent sign bit. And so on.

The context modelling in Dirac is based on the principle that whether a
coefficient is small (or zero, in particular) or not is well-predicted
by its neighbours and its parents. Therefore the codec conditions the
probabilities used by the arithmetic coder for coding bins 1 and 2 on
the size of the neighbouring coefficients and the parent coefficient.

The reason for this approach is that, whereas the wavelet transform
largely removes correlation between a coefficient and its neighbours,
they may not be statistically independent even if they are uncorrelated.
The main reason for this is that small and especially zero coefficients
in wavelet subbands tend to clump together, located at points
corresponding to smooth areas in the image, and as discussed elsewhere,
are grouped together across subbands in the parent-child relationship.

Conceptually, an arithmetic coder can be thought of a progressive way of
producing variable-length codes for entire sequences of symbols based on
the probabilities of their constituent symbols.

For example, if we know the probability of 0 and 1 in a binary sequence,
we also know the probability of the sequence itself occurring. So if

$P(0)=0.2, $

$P(1)=0.8$

then

$P(11101111111011110101)=(0.2)*3*(0.8)*17=1.8 * 10^{-4}$ (assuming
independent occurrences).

Information theory then says that optimal entropy coding of this
sequence requires $log_2 (\frac{1}{p})=12.4$ bits. Arithmetic coding
produces a code word very close to this optimal length, and
implementations can do so progressively, outputting bits when possible
as more arrive.

All arithmetic coding requires are estimates of the probabilities of
symbols as they occur, and this is where context modelling fits in.
Since arithmetic coding can, in effect, assign a fractional number of
bits to a symbol, it is very efficient for coding symbols with
probabilities very close to 1, without the additional complication of
run-length coding. The aim of context modelling within Dirac is to use
information about the symbol stream to be encoded to produce accurate
probabilities as close to 1 as possible.

Dirac computes these estimates for each context simply by counting their
occurrences. In order for the decoder to be in the same state as the
encoder, these statistics cannot be updated until after a binary symbol
has been encoded. This means that the contexts must be initialised with
a count for both 0 and 1, which is used for encoding the first symbol in
that context.

An additional source of redundancy lies in the local nature of the
statistics. If the contexts are not refreshed periodically then later
data has less influence in shaping the statistics than earlier data,
resulting in bias, and local statistics are not exploited. Dirac adopts
a simple way of refreshing the contexts by halving the the counts of 0
and 1 for that context at regular intervals. The effect is to maintain
the probabilities to a reasonable level of accuracy, but to keep the
influence of all coefficients roughly constant.

\subsubsection{Entropy coding of motion vectors}

The basic format of the coding the motion vectors is similar to the
coding of wavelet data: it consists of prediction, followed by
binarisation, context modelling and adaptive arithmetic coding.


Figure: motion vector entropy coding architecture

All the motion vector data are predicted from previously encoded data
from nearest neighbours.

\subsection{Bytestream}The bytestream of Dirac is the complete, compressed video stream, ready
for file storage or transmission.

The features are

\begin{itemize}
    \item A parse structure which gives ready access to the video, even
    in mid file. This identifies all the static features of the stream.
    Intelligent systems can use the redundancy in this structure to
    recover from errors in the stream.

    \item Blocks of data comprising the arithmetically-coded motion
    vector information

    \item Blocks of data comprising the arithmetically-coded wavelet
    coefficients
\end{itemize}

Because of the arithmetic coding, it is difficult to provide a simple
specification which gives meaningful details of the bytestream without
some reference to either the early coding processes, or the later
decoding process. The simple description of the bytestream would only
refer to compressed motion vectors, and compressed wavelet coefficients.
We have therefore had to provide extra information to explain how to
unpack this data, and how to use the resulting structured data.


The bytestream of Dirac is the complete, compressed video stream, ready
for file storage or transmission.

The features are

\begin{itemize}
    \item A parse structure which gives ready access to the video, even
    in mid file. This identifies all the static features of the stream.
    Intelligent systems can use the redundancy in this structure to
    recover from errors in the stream.

    \item Blocks of data comprising the arithmetically-coded motion
    vector information

    \item Blocks of data comprising the arithmetically-coded wavelet
    coefficients
\end{itemize}

Because of the arithmetic coding, it is difficult to provide a simple
specification which gives meaningful details of the bytestream without
some reference to either the early coding processes, or the later
decoding process. The simple description of the bytestream would only
refer to compressed motion vectors, and compressed wavelet coefficients.
We have therefore had to provide extra information to explain how to
unpack this data, and how to use the resulting structured data.

