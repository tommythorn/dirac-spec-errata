The arithmetic decoding process accesses data in a contiguous block of
length $block\_data\_length$. These bits are sufficient to allow for the
decoding of all coefficients. However, the specification of arithmetic
decoding operations in this section may cause further bits to be read,
even though they are not required for determining decoded values. For this
reason a read function $read\_bita()$ is defined which returns $0$ if the
bounds of this block of data have been exceeded:

\begin{pseudo}{read\_bita}{}
\bsIF{bytes\_left==0}
\bsRET{0}
\bsELSE
\bsRET{read\_bit()}
\bsEND
\end{pseudo}

\begin{informative}
The Dirac arithmetic decoding engine uses 16 bit words, and so no more than 16
additional bits can be read beyond the end of the block. Hence it is sufficient
to read in the entire block of data and pad the end with two zero bytes to
avoid a branch condition with each input bit.
\end{informative}
