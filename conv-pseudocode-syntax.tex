This description is a simple presentation of the syntax used in this
specification.

\textbf{Variables}

Variables are strongly typed - but the choice of type is implicit rather
than being explicitly defined.

Variables are usually local to the procedure in which they are defined.
Thus the same variable name may refer to different variables in
different scopes.

Global variables are identified in Appendix XX. There is no formal
declaration of global variables in the pseudocode as would normally be
the case in Python.

Variables are passed to procedures by reference. That means they may be
modified within the procedure and the modifications will be seen by
subsequent procedures and lines of code.

\textbf{Lists}

Lists in the pseudocode are one dimensional arrays, defined in the
manner: [1,2,3,4]. Elements of the list are counted from the left,
starting with zero as the index. E.g:

\begin{verbatim}
	x 		= [1,2,3,4]
	x[2] 	= 3
\end{verbatim}

\textbf{Dictionaries}

Dictionaries in the pseudocode are look-up tables defined in the manner:

$\lbrace a: \alpha , b: \beta, c: \gamma \rbrace$

The look-up parameters a, b and c return values of $\alpha$, $\beta$,
and $\gamma$  respectively.

\textbf{Flow}

The pseudocode comprises a series of statements, linked by functions and
flow control statements such as if, while, and for.

The statements do not have a termination character, unlike the ; in C
for example.  Blocks of statements are indicated by indentation:
indenting in begins a block, indenting out ends one.

Statements that expect a block (and hence a following indentation) end
in a colon.

\textbf{if}

The if control evaluates a function, and if true, passes the flow to the
block of following statement or block of statements. If the control
evaluates as false, then there is an option to include one or more elif
controls which offer alternative responses if some other condition is
true.  If none of the preceding controls evaluate to true, then there is
the option to include an else control. E.g.


\begin{verbatim}
if control:
    statement1
elif anothercontrol:
    statement2
else:
    statement3
\end{verbatim}

\textbf{for}

The for control repeats a loop for each instance of a variable in a list. E.g.

\begin{verbatim}

for variable in list:
    statement1
    statement2
\end{verbatim}

\textbf{while}

The while control repeats a loop until the switch variable is true. When
it is false, the loop breaks to the next statement outside the block.
E.g.

\begin{verbatim}
while (switch):
    statement1
    statement2        # changes value of switch to false
NextStatement
\end{verbatim}

\textbf{Functions}

Functions are identified by a function name, and the parameters that are
passed to the function when it is evaluated.

The function is defined using pseudocode in a block (which does not
require the def or return  statements which are required in Python. The
return  statement is optional, depending on functionality). E.g.

\begin{verbatim}
function (parameter):
    variable = 2*parameter
    return variable
\end{verbatim}

\textbf{Arithmetic}

The following operators are used for arithmetic:

$=$	equals

$+$	addition

$-$	subtraction

$*$	multiplication

$**$	raising preceding variable to the power of the following variable

$//$	truncating division

$/$	simple division

$\%$	indicates modulo division - the remainder after the preceding
variable has been divided by the following variable

\textbf{Logic}

The following operators are used in logic:

$= =$ 	test of equality of two variables

$<$	less than

$<=$	less than or equal to

$>$	greater than

$>=$	greater than or equal to

\textbf{Bitwise operations}

$>>$	shift all the elements in a bit sequence one step right

$<<$	shift all the elements in a bit sequence one step left

\textbf{Comments}

In many places, we have felt it useful to add comments to the code.
These are preceded with the \# symbol and do not form part of the main
code.
