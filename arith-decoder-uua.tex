\begin{comment}
Pseudo-code for unsigned unary arithmetic decoding uu\_arith\_decode() is
as follows:

VALUE=0

while ( !binary\_arith\_decode( choose\_context() ) )

    VALUE++

choose\_context() is a function that produces a context with which the
binary bit shall be decoded. The value it returns can depend on any
values known to the decoder at the time it is called, especially
including the binarisation bin (the bin number is equal to VALUE+1
according to the conventions of Appendix).
\end{comment}

%src: tim 0.9.1.48

reads and returns an unsigned integer encoded in the bytestream as an
arithmetic coded unary binarisation. ``context\_list'' is a list of
contexts for each bin. If the number of contexts in the list is less
than the bin number then the last context on the list is used. In other
words, a
common context is used for all the higher bins.

Read Arithmetic Coded Unsigned Integer
\begin{verbatim}
read_uua(context_list):

    value = 0
    context_index = 0 #Bin Number (numbered from zero)
    max_index = len(context_list) - 1
    # Read first bit
    context = context_list[context_index]
    if ( read_ba(context) ):
        more = False
    else:
        value += 1
        more = True
    context_list[context_index] = context
    # Read remaining bits
    while ( more ):
        if ( context_index < max_index):
            context_index += 1
        context = context_list[context_index]

        if ( read_ba(context) ):
            more = False
        else:
            value += 1
            more = True
        context_list[context_index] = context
    return value
\end{verbatim}
