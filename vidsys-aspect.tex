The PIXEL\_ASPECT\_RATIO value of an image is the ratio of the intended
spacing of horizontal samples (pixels) to the spacing of vertical
samples (picture lines) on the display device. Pixel aspect ratios are
fundamental properties of sampled images because they determine the
displayed shape of objects in the image. Failure to use the right value
of PIXEL\_ASPECT\_RATIO will result in distorted images – for example,
circles will be displayed as ellipses and so forth. 


Some HDTV standards and computer image formats are defined to have pixel
aspect ratios that are exactly 1:1.

The clean area is intended to define an area within which picture
information is subjectively uncontaminated by all edge transient (and
other) distortions. It may only be appropriate to display the clean area
rather than the whole picture, which may be distorted at the edge. 

The top-left corner of the clean area has coordinates (CLEAN\_TL\_X,
CLEAN\_TL\_Y) and dimensions CLEAN\_WIDTHxCLEAN\_HEIGHT.

The clean area and the pixel aspect ratio determine the
IMAGE\_ASPECT\_RATIO which is the ratio of the width of the intended
display area to the height of the intended display area. 

Given two separate sequences, with identical IMAGE\_ASPECT\_RATIO, if the
top left corner and bottom left corners of their clean apertures are
coincident when displayed, then the images as a whole should be exactly
coincident. This is regardless of the actual pixel dimensions of the
images or their clean areas. This allows sequences to be combined
together appropriately if they are appropriately scaled.


The defined pixel aspect ratios are designed to give standard image
aspect ratios for typical TV broadcasts. For example, for a 525 line
(American) 704 x 480 (clean area)  picture the image aspect ratio is
(704 x 10)/(480 x 11) which is exactly 4:3.


For 625 line systems the 59:54 pixel aspect ratio means (less
conveniently) that a 702.9x576 image would have an exact 4:3 image
aspect ratio. It might be argued that the pixel aspect ratio for 625
line systems should be such that a 702x576 image would have an exact 4:3
image aspect ratio. It could be said that this corresponds to the
analogue 625 line TV specification. This requirement would lead to a
pixel aspect ratio of 128:117. However, the tolerance of the analogue
line length is 702 3 pixels, which does not really seem to justify a
ratio of exactly 128:117.

The values specified here are generally agreed to be the
“correct” values. Then again not everyone agrees with this
consensus. These arise from the “industry standard” sampling
frequencies used for square pixels, which were originally designed for
digitising composite analogue video signals. These “industry
standard” sampling frequencies are 11+3/11 MHz for 525 line systems
and 14.75MHz for 625 line systems. The ratio of these frequencies to the
(standardised) 13.5MHz sampling frequency used for broadcasting yields
the pixel aspect ratios given in  and .

You are strongly advised to use one of the default pixel aspect ratios.
However, if you know what you are doing and don’t like the default
values you can define your own ratio. You should be aware that many
display devices may ignore your decision and may use different and
unsuitable values. 

