%%%%%%%%%%%%%%%%%%%%%%%%%%%%%%%%%%%%%%%%%%%%
% - This chapter defines how motion data - %
% - is decoded                           - % 
%%%%%%%%%%%%%%%%%%%%%%%%%%%%%%%%%%%%%%%%%%%%

\label{motiondec}

This section specifies the operation of the $block\_data()$ process for extracting
block motion data from the Dirac stream. 

Block data is aggregated into {\em superblocks}, consisting of a 4x4 array of blocks. The number 
of superblocks horizontally and vertically is determined so that there are sufficient superblocks
to cover the picture area. Superblocks may overlap the right and bottom edge of the picture.

\begin{informative}
Since superblocks may overlap the right and bottom edge of the picture, blocks in such superblocks
may also overlap the edges or even fall outside the picture area altogether. Motion data for 
blocks which fall outside the picture area is still decoded, but will not be used for motion compensation 
(Section \ref{motioncompensation}). 

Unlike macroblocks in MPEG standards, a superblock does not encapsulate all data within a 
given area of the picture. It is merely an aggregation device for motion data, and for this reason
a different nomenclature has been adopted.
\end{informative}

\subsection{Motion data conventions}

\label{motionconventions}

For the purposes of this specification, block motion data is stored in a two dimensional array 
$\BlockData$ of block data structures. A block motion data element ${\bf b}=\BlockData[j][i]$ consists of:

\begin{itemize}
\item A motion vector for reference 1, ${\bf b}[ref1]$, with integral horizontal and vertical components 
${\bf b}[ref1].x$ and ${\bf b}[ref1].y$
\item A motion vector for reference 2, ${\bf b}[ref2]$, with integral horizontal and vertical components 
${\bf b}[ref2].x$ and ${\bf b}[ref2].y$
\item A set of integral DC values, ${\bf b}[dc][Y]$, ${\bf b}[dc][U]$, and ${\bf b}[dc][V]$ for each component
\item A prediction mode, ${\bf b}[mode]$, consisting of two flags ${\bf b}[mode][1]$ and ${\bf b}[mode][2]$ 
indicating whether the corresponding reference is to be used for predicting block $(i,j)$
\end{itemize}

Four prediction modes shall be supported by the decoder: 

\begin{itemize}
\item \Intra - corresponding to ${\bf b}[mode][1]=\false$ and ${\bf b}[mode][2]=\false$, and using DC prediction only
\item \RefOneOnly - corresponding to ${\bf b}[mode][1]=\true$ and ${\bf b}[mode][2]=\false$, and using a prediction from Reference 1 only
\item \RefTwoOnly - corresponding to ${\bf b}[mode][1]=\false$ and ${\bf b}[mode][2]=\true$, and using a prediction from Reference 2 only
\item \RefOneAndTwo - corresponding to ${\bf b}[mode][1]=\true$ and ${\bf b}[mode][2]=\true$, and using a prediction from Reference 1 and Reference 2
\end{itemize}

In this way, Reference $X$ is used for prediction if and only if ${\bf b}[mode][X]=\true$. 

\subsection{Motion data decoding process}

This section specifies the overall operation of the $block\_data()$ process for extracting
block motion data elements: motion vectors and block prediction modes. This 
process is called by the $picture\_prediction()$ process (Section \ref{pictureprediction}) and
depends upon the parameters that have been extracted and set in the
$picture\_prediction\_parameters()$ process (Section \ref{picpredparams}).

Block motion data elements are all coded differentially with respect to a spatial prediction. The
spatial prediction processes for the block motion elements are specified in Section \ref{spatialprediction}

\subsubsection{Overall decoding loop}

\label{decodingloop}

The decoding loop for block data iterates over all superblocks in raster order:

\begin{pseudo}{block\_data}{}
\bsCODE{initialise\_motion\_data()}{\ref{motioninit}}
\bsITEM{length}{uint}{}
\bsCODE{byte\_align()}
\bsCODE{initialise\_arithmetic\_decoding(length)}{\ref{initarith}}
\bsCODE{superblock\_count = 0}
\bsFOR{v=0}{\SuperblocksY}
    \bsFOR{h=0}{\SuperblocksX}
        \bsCODE{superblock(4*v, 4*h)}
        \bsCODE{superblock\_count += 1}
        \bsIF{superblock\_count == 32}
            \bsCODE{superblock\_count = 0}
            \bsFOR{i=0}{len(\AContexts)-1}
                \bsCODE{rescale\_context(i)}{\ref{arithcontexts}}
            \bsEND
        \bsEND
    \bsEND
\bsEND
\bsCODE{flush\_arith()}{\ref{inputarith}}
\end{pseudo}

\begin{informative}
The specification for parsing and decoding block data is written indicating that a sequence of 
superblocks are read following the block data length. This is conceptually correct. However, 
in practice it may be more efficient to read the whole of the block data (excluding the block 
data length) into a buffer before parsing and decoding. To facilitate this the arithmetic coded 
superblock data is byte aligned, i.e. starts at the beginning of a byte boundary and occupies a 
whole number of bytes.

One reason for the improved efficiency afforded by first reading into a buffer is because the 
arithmetic coding engine may require (up to) two extra bytes of data for its output to 
converge. In this specification this feature is provided by testing that the amount of data 
requested has not exceeded $length$ bytes and, if it has, inputting a zero byte. By 
reading the data directly into a buffer of length  $length + 2$ this test can be 
avoided.
\end{informative}

\subsubsection{Motion data initialisation}

\label{motioninit}

This section specifies the operation of the $initialise\_motion\_data()$ process. It sets the dimension variables determining the
number of blocks and superblocks and hence the dimension of the $\BlockData$ array encapsulating block motion
data.

The number of superblocks horizontally and vertically is set by:

\begin{eqnarray*}
\SuperblocksX & = & \left\lceil \frac{\LumaWidth}{4*\LumaXBsep} \right\rceil  \\
\SuperblocksY & = & \left\lceil \frac{\LumaHeight}{4*\LumaYBsep} \right\rceil 
\end{eqnarray*}

The number of blocks horizontally and vertically is set by:

\begin{eqnarray*}
\BlocksX = 4*\SuperblocksX \\
\BlocksY = 4*\SuperblocksY
\end{eqnarray*}

The array $\BlockData$ is set to have horizontal dimension $\BlocksX$ and vertical dimension $\BlocksY$.

The array $\SBSplit$ is set to have horizontal dimension $\SuperblocksX$ and vertical dimension $\SuperblocksY$.

\subsubsection{Superblock decoding}

\label{superblockdecoding}

This section specifies the process $superblock(y, x)$ for decoding a superblock containing
the blocks with horizontal indices $x, x+1, x+2, x+3$ and vertical indices $y, y+1, y+2, y+3$.

Data for blocks within each superblock is preceded by a superblock header containing
a split mode.

There are three possible split modes: 0, 1 and 2. In split mode 0, a single set of block data is encoded, 
which is to be used for all blocks within
the superblock. In split mode 1, four sets of block data is encoded, to be used for each of the four
sets of 2x2 blocks within the superblock. In split mode 2, all sixteen sets of block data are encoded.

\begin{pseudo}{superblock}{y, x}
\bsCODE{sb\_split( y//4, x//4)}{\ref{superblocksplit}}
\bsCODE{block\_count = 2^{\SBSplit[y//4][x//4]}}
\bsCODE{step = 4//block\_count }
\bsFOR{q=0}{block\_count-1}
   \bsFOR{p=0}{block\_count-1}
        \bsCODE{block(y+q*step, x+p*step)}{\ref{blockdecoding}}
        \bsCODE{propagate\_data(y+q*step, x+p*step, step)}{\ref{propagatedata}}
   \bsEND
\bsEND
\end{pseudo}

\paragraph{Superblock splitting mode}
\label{superblocksplit}

The $sb\_split(y, x)$ decodes the superblock splitting mode at superblock coordinates ($x, y)$.

\begin{pseudo}{sb\_split}{ypos,xpos}
\bsCODE{sb\_split\_residual=read\_uinta(sb\_split\_contexts() ) }
\bsCODE{\SBSplit[ypos][xpos] = sp\_split\_residual+split\_prediction(ypos, xpos)}{\ref{splitprediction}}
\bsCODE{\SBSplit[ypos][xpos] \%= 3}
\end{pseudo}

\paragraph{Propagating data between blocks}
\label{propagatedata}

The $propagate\_data(s, r, k)$ copies decoded block data from the top-left-most block
of a set of $k\times k$ blocks:

\begin{pseudo}{propagate\_data}{s, r, k}
\bsFOR{j=s}{s+k-1}
    \bsFOR{i=r}{r+k-1}
        \bsCODE{\BlockData[j][i]=\BlockData[s][r]}
    \bsEND
\bsEND
\end{pseudo}

\subsubsection{Block decoding}
\label{blockdecoding}

The $block\_decode(y, x)$ process parses block motion data for the block at coordinates $(x, y)$. If $\PictureUsingGlobal$ is set
then non-intra blocks may use use global motion compensation according to a flag.
If the block is not intra, and not globally motion compensated, then motion vectors are decoded.

\begin{pseudo}{block\_decode}{y, x}
\bsCODE{\BlockData[y][x][mode]=block\_mode(y, x)}{\ref{blockmode}}
\bsIF{\BlockData[y][x][mode]==\Intra}
    \bsCODE{\BlockData[y][x][global]=\false}
    \bsCODE{block\_dc(y, x) }{\ref{blockdc}}
\bsELSE
    \bsIF{\PictureUsingGlobal==\true}
        \bsCODE{block\_global(y, x)}{\ref{blockglobal}}
    \bsELSE
        \bsCODE{\BlockData[y][x][global]=\false}
    \bsEND

    \bsIF{\BlockData[y][x][global]=\false}
        \bsCODE{block\_vectors(y,x)}{\ref{blockvectors}}
    \bsEND
\end{pseudo}

\begin{informative}
Note that if the picture is using global motion, block global motion vectors are never used: the
\end{informative}

\paragraph{Block mode \\}
\label{blockmode}

The $block\_mode(ypos,xpos)$ process parses the block prediction mode for the block at position $(xpos, ypos)$. Each
bit of the block prediction mode is decoded separately.

\begin{pseudo}{block\_mode}{ypos, xpos}
\bsCODE{pred\_mode\_residue = read\_boola(\PredModeOne)}
\bsCODE{\BlockData[ypos][xpos][mode][1] =pred\_mode\_residue}
\bsCODE{\BlockData[ypos][xpos][mode][1] \hat\quad =mode\_prediction(ypos, xpos, 1)}{\ref{modeprediction}}
\bsIF{\NumRefs == 2}
    \bsCODE{pred\_mode\_residue = read\_boola(\PredModeOne)}
    \bsCODE{\BlockData[ypos][xpos][mode][2] =pred\_mode\_residue}
    \bsCODE{\BlockData[ypos][xpos][mode][2]\hat\quad= mode\_prediction(ypos, xpos, 2)} {\ref{modeprediction}}
\bsEND
\end{pseudo}

\paragraph{Block global motion flag\\}
\label{blockglobal}

The $block\_global(ypos,xpos)$ process parses the block global motion flag for the block at position $(xpos, ypos)$.

\begin{pseudo}{block\_mode}{ypos, xpos}
\bsCODE{block\_global\_residue = read\_boola(\BlockGlobal)}
\bsCODE{\BlockData[ypos][xpos][global] = block\_global\_residue}
\bsCODE{\BlockData[ypos][xpos][global] \hat\quad= block\_global\_prediction(ypos, xpos)}{\ref{blockglobalprediction}} 
\end{pseudo}

\paragraph{DC values\\}
\label{blockdc}

The $block\_dc(ypos, xpos)$ process parses the block DC values for the three components for the block at position 
$(xpos, ypos)$.

\begin{pseudo}{block\_motion}{ypos, xpos}
\bsFOREACH{c}{Y,U,V}
    \bsCODE{dc\_residual = read\_sinta(dc\_contexts(c)) }{\ref{dcvaluecontexts}}
    \bsCODE{\BlockData[ypos][xpos][dc][c] = dc\_residual}
    \bsCODE{\BlockData[ypos][xpos][dc][c] +=dc\_prediction(ypos, xpos, c)}{\ref{dcprediction}}
\bsEND
\end{pseudo}

\paragraph{Block motion vectors \\}
\label{blockvectors}

The $block\_motion(ypos, xpos)$ process parses the block motion vectors for the block at position $(xpos, ypos)$.

\begin{pseudo}{block\_motion}{ypos, xpos}
\bsIF{\BlockData[ypos][xpos][mode][1] == \true}
    \bsCODE{mv\_residual.x = read\_sinta(ref1x\_contexts() ) }{\ref{mvcontexts}}
    \bsCODE{mv\_residual.y = read\_sinta(ref1y\_contexts() ) }{\ref{mvcontexts}}
    \bsCODE{\BlockData[ypos][xpos][ref1] = mv\_residual}
    \bsCODE{\BlockData[ypos][xpos][ref1] +=mv\_prediction(ypos, xpos, 1)}
\bsEND
\bsIF{\BlockData[ypos][xpos][mode][2] == \true}
    \bsCODE{mv\_residual.x = read\_sinta(ref2x\_contexts() ) }{\ref{mvcontexts}}
    \bsCODE{mv\_residual.y = read\_sinta(ref2y\_contexts() ) }{\ref{mvcontexts}}
    \bsCODE{\BlockData[ypos][xpos][ref2] = mv\_residual}
    \bsCODE{\BlockData[ypos][xpos][ref2] += mv\_prediction(ypos, xpos, 2)}
\bsEND
\end{pseudo}

\subsubsection{Spatial prediction of motion data elements}

\label{spatialprediction}

\paragraph{Prediction apertures}

A consistent convention for prediction apertures is used. The nominal prediction 
aperture for block motion data is defined to be the relevant data to the left, top
and top-left of the data element in question (Figure \ref{predaperture}). For the superblock split mode of 
the superblock with index $(i,j)$ this means the superblocks with indices $(i-1,j)$,
$(i,j-1)$ and $(i-1,j-1)$. For the block motion data itself, the same applies where these
indices are {\em block} indices. 

\setlength{\unitlength}{1em}
\begin{figure}[!h]
\centering
\begin{picture}(15,20)
\multiput(0,0)(8,0){3}%
  {\line(0,1){16}}
\multiput(0,0)(0,8){3}%
  {\line(1,0){16}}
  
%Shading  

\multiput(0,0)0.2,0){40}%  
{\multiput(8,0.1)(0,.2){40}%
  {\tiny.}
}

%Arrows
\put(4,12){\vector(1,-1){6}}
\put(12,12){\vector(0,-1){6}}
\put(4,4){\vector(1,0){6}}
\end{picture}
\caption{Basic prediction aperture}\label{predaperture}
\end{figure}

Note that this is the nominal prediction aperture. Not all data elements in this prediction
aperture may be available, either because they would require negative indices, or because
the data is not available - for example a block to the left of a block with mode \RefTwoOnly
may have mode \RefOneOnly and so can furnish no contribution for a prediction to the
Reference 2 motion vector.

Note also that when superblocks have split level 1 or 0, block data has been propagated
(Section \ref{propagatedata}) across 4 or 16 blocks so as to furnish a prediction. The
effect is illustrated for a variety of splitting modes in Figure \ref{splitapertures}.

\setlength{\unitlength}{.75em}
\begin{figure}[!h]
\centering
\begin{picture}(60,17.5)

\multiput(0,0)(20,0){3}%
{

%Main Grid
\multiput(0,0)(8,0){3}%
  {\line(0,1){16}}
\multiput(0,0)(0,8){3}%
  {\line(1,0){16}}
\multiput(0,0)0.2,0){40}%  

%Shading
\multiput(0,0)0.4,0){20}% 
{\multiput(8,0.2)(0,.4){20}%
  {\tiny.}
}

%Dotted Grid
\multiput(0,0)(0,2){3}%
{\multiput(0,2)(0.5,0){16}%
   {\line(1,0){.25}}
}
\multiput(0,0)(2,0){3}%
{\multiput(2,0)(0,0.5){16}%
   {\line(0,1){.25}}
}

\multiput(8,0)(0,8){2}%
{\multiput(0,4)(0.5,0){16}%
   {\line(1,0){.25}}
}

\multiput(12,0)(0,0.5){32}%
   {\line(0,1){.25}}
}
%Arrows
\put(4,12){\vector(1,-1){5}}
\put(7,7) {\vector(2,-1){1.5}}
\put(10,10){\vector(0,-1){3}}

\put(30,6){\vector(0,-1){3.5}}
\put(27,5) {\vector(1,-1){2}}
\put(27,3){\vector(2,-1){2}}

\put(50,2){\vector(1,0){3.5}}
\put(50,6) {\vector(1,-1){3.5}}
\put(54,6){\vector(0,-1){3.5}}

\end{picture}
\caption{Effect of splitting modes on spatial prediction}\label{splitapertures}
\end{figure}

\paragraph{Superblock split prediction \\}
\label{splitprediction}

$split\_prediction$ returns the mean of the the neighbouring split values:

\begin{pseudo}{split\_prediction}{ypos, xpos}
\bsIF{ xpos==0 \&\& ypos==0 }
    \bsRET{0}
\bsELSEIF{ypos==0}
    \bsRET{\SBSplit[ypos][xpos-1]}
\bsELSEIF{xpos==0}
    \bsRET{\SBSplit[ypos-1][xpos]}
\bsEND

\bsCODE{ 
\begin{array}{ll}
\text{return} & (\SBSplit[ypos-1][xpos-1] +\SBSplit[ypos][xpos-1]+  \\
                   &  \quad \SBSplit[ypos-1][xpos]+1)//3
\end{array}
}
\end{pseudo}

\paragraph{Block mode prediction \\}
\label{modeprediction}

$mode\_prediction$ returns a majority verdict for each of the references:

\begin{pseudo}{mode\_prediction}{ypos, xpos, n}
\bsIF{ xpos==0 \&\& ypos==0 }
    \bsRET{\false}
\bsELSEIF{ypos==0}
    \bsRET{\BlockData[ypos][xpos-1][mode][n]}
\bsELSEIF{xpos==0}
    \bsRET{\BlockData[ypos-1][xpos][mode][n]}
\bsEND
\bsCODE{
\begin{array}{ll}
\text{return} & \majority(\BlockData[ypos-1][xpos-1][mode][n], \\
& \quad\BlockData[ypos-1][xpos][mode][n], \\
& \quad\BlockData[ypos][xpos-1][mode][n] )
\end{array}
}

\end{pseudo}

\paragraph{Block global flag prediction \\}
\label{blockglobalprediction}

$block\_global\_prediction$ returns a majority verdict of the neighbouring blocks:

\begin{pseudo}{block\_global\_prediction}{ypos, xpos}
\bsIF{ xpos==0 \&\& ypos==0 }
    \bsRET{\false}
\bsELSEIF{ ypos==0 }
    \bsRET{\BlockData[ypos][xpos-1][global]}
\bsELSEIF{xpos==0}
    \bsRET{\BlockData[ypos-1][xpos][global]}
\bsEND

\bsCODE{
\begin{array}{ll}
\text{return} & \majority(\BlockData[ypos-1][xpos-1][global], \\
& \quad\BlockData[ypos-1][xpos][global],  \\
& \quad\BlockData[ypos][xpos-1][global]) 
\end{array}
}

\end{pseudo}

\paragraph{Motion vector prediction \\}
\label{mvprediction}

Motion vectors are predicted using the median of available block vectors in the aperture. A vector is available for
prediction if a) its block falls within the picture area and b) its prediction mode allows it to be defined and c) it
is not a global motion block. 

The process $mv\_prediction(ypos, xpos, ref)$ returns motion values according to
the following rules:

{\bf Case 1.}  If $xpos==0$ and $ypos==0$, there are no vectors in the prediction aperture and
the zero vector $(0, 0)$ is returned.

{\bf Case 2.} If $xpos>0$ and $ypos==0$ then:
\begin{enumerate}
   \item If 

$\BlockData[ypos][xpos-1][global]==\false$ 

and 

$\BlockData[ypos][xpos-1][mode][ref]==\true$ 

then $\BlockData[ypos][xpos-1][ref]$ is returned.
   \item Otherwise, $(0,0)$ is returned
\end{enumerate}

{\bf Case 3.} If $xpos==0$ and $ypos>0$ then:
\begin{enumerate}
   \item If 

$\BlockData[ypos-1][xpos][global]==\false$ 

and 

$\BlockData[ypos-1][xpos][mode][ref]==\true$

then $\BlockData[ypos-1][xpos][ref]$ is returned.
   \item Otherwise, $(0,0)$ is returned
\end{enumerate}

{\bf Case 4.} If both $xpos>0$ and $ypos>0$ then all 3 blocks in the prediction aperture may potentially
contribute to the prediction. Define sets $valuesx=\emptyset$ and $valuesy=\emptyset$. The prediction is the 
median vector has  horizontal and vertical components equal to the median of the horizontal and vertical
components of available vectors:

\begin{pseudo*}
\bsIF{\BlockData[ypos][xpos-1][global]==\true}
    \bsIF{\BlockData[ypos][xpos-1][mode][ref]==\false}
        \bsCODE{valuesx = valuesx\cup\{\BlockData[ypos][xpos-1][ref].x \} }
        \bsCODE{valuesy = valuesy\cup\{\BlockData[ypos][xpos-1][ref].y \} }
    \bsEND
\bsEND
\bsIF{\BlockData[ypos-1][xpos][global]==\true}
    \bsIF{\BlockData[ypos-1][xpos][mode][ref]==\false}
        \bsCODE{valuesx = valuesx\cup\{\BlockData[ypos-1][xpos][ref].x \} }
        \bsCODE{valuesy = valuesy\cup\{\BlockData[ypos-1][xpos][ref].y \} }
    \bsEND
\bsEND
\bsIF{\BlockData[ypos-1][xpos-1][global]==\false}
    \bsIF{\BlockData[ypos-1][xpos-1][mode][ref]==\true}
        \bsCODE{valuesx = valuesx\cup\{\BlockData[ypos-1][xpos-1][ref].x \} }
        \bsCODE{valuesy = valuesy\cup\{\BlockData[ypos-1][xpos-1][ref].y \} }
    \bsEND
\bsEND

\bsRET{(\median(valuesx), \median(valuesy) )}

\end{pseudo*}

(Note that the median of an empty set is zero.)

\paragraph{DC value prediction \\}
\label{dcprediction}

DC values are predicted using the unbiased mean available values in the aperture. The
process $dc\_prediction(ypos, xpos, comp)$ returns values according to
the following rules:

{\bf Case 1.}  If $xpos==0$ and $ypos==0$, there are no blocks in the prediction aperture and
the default prediction $2^{\VideoDepth-1}$ is returned.

{\bf Case 2.} If $xpos>0$ and $ypos==0$ then:
\begin{enumerate}
   \item If $\BlockData[ypos][xpos-1][mode]==\Intra$, $\BlockData[ypos][xpos-1][dc][comp]$ is returned
   \item Otherwise, $2^{\VideoDepth-1}$ is returned
\end{enumerate}

{\bf Case 3.} If $xpos==0$ and $ypos>0$ then:
\begin{enumerate}
   \item If $\BlockData[ypos-1][xpos][mode]==\Intra$, $\BlockData[ypos-1][xpos][dc][comp$ is returned
   \item Otherwise, $2^{\VideoDepth-1}$ is returned
\end{enumerate}

{\bf Case 4.} If both $xpos>0$ and $ypos>0$ then all 3 blocks in the prediction aperture may potentially
contribute to the prediction. Define a set $values=\emptyset$. The prediction is the 
unbiased mean of available values:

\begin{pseudo*}
\bsCODE{pred=(0, 0)}
\bsIF{\BlockData[ypos][xpos-1][mode]==\Intra}
    \bsCODE{values = values\cup\{\BlockData[ypos][xpos-1][dc][comp] \} }
\bsEND
\bsIF{\BlockData[ypos-1][xpos][mode]==\Intra}
    \bsCODE{values = valuesx\cup\{\BlockData[ypos-1][xpos][ref][dc][comp] \} }
\bsEND
\bsIF{\BlockData[ypos-1][xpos-1][mode]==\Intra}
    \bsCODE{values = values\cup\{\BlockData[ypos-1][xpos-1][ref][dc][comp] \} }
\bsEND

\bsIF{values!=\{\}}
    \bsRET{pred =(unbiased\_mean(values)}
\bsELSE
    \bsRET{2^{\VideoDepth-1}}
\bsEND
\end{pseudo*}


\subsubsection{Block motion data contexts}

\paragraph{Superblock splitting mode\\}

The $sb\_split\_contexts()$ function returns the following unsigned integer context set:

\begin{itemize}
\item Follow = [ \SBSplitFollowOne, \SBSplitFollowTwo ]
\item Data = \SBSplitData
\end{itemize}

\paragraph{Motion vectors\\}
\label{mvcontexts}

There are four motion vector context sets, which are signed integer context sets as follows.

$ref1x\_contexts$ returns the set with
\begin{itemize}
\item Follow = [ \RefOnexFollowOne, \RefOnexFollowTwo, \RefOnexFollowThree, \RefOnexFollowFour, \RefOnexFollowFivePlus ]
\item Data = \RefOnexData
\item Sign = \RefOnexSign
\end{itemize}

$ref1y\_contexts$ returns the set with
\begin{itemize}
\item Follow = [ \RefOneyFollowOne, \RefOneyFollowTwo, \RefOneyFollowThree, \RefOneyFollowFour, \RefOneyFollowFivePlus ]
\item Data = \RefOneyData
\item Sign = \RefOneySign
\end{itemize}

$ref2x\_contexts$ returns the set with
\begin{itemize}
\item Follow = [ \RefTwoxFollowOne, \RefTwoxFollowTwo, \RefTwoxFollowThree, \RefTwoxFollowFour, \RefTwoxFollowFivePlus ]
\item Data = \RefTwoxData
\item Sign = \RefTwoxSign
\end{itemize}

$ref1y\_contexts$ returns the set with
\begin{itemize}
\item Follow = [ \RefTwoyFollowOne, \RefTwoyFollowTwo, \RefTwoyFollowThree, \RefTwoyFollowFour, \RefTwoyFollowFivePlus ]
\item Data = \RefTwoyData
\item Sign = \RefTwoySign
\end{itemize}

\paragraph{DC values \\}
\label{dcvaluecontexts}

There are three DC value context sets, which are signed integer context sets for each component. 

$dc\_contexts(Y)$ returns the set:

\begin{itemize}
\item Follow = [ \YDCFollowOne, \YDCFollowTwoPlus ]
\item Data = \YDCData
\item Sign = \YDCSign
\end{itemize}

$dc\_contexts(U)$ returns the set:

\begin{itemize}
\item Follow = [ \UDCFollowOne, \UDCFollowTwoPlus ]
\item Data = \UDCData
\item Sign = \UDCSign
\end{itemize}

$dc\_contexts(V)$ returns the set:

\begin{itemize}
\item Follow = [ \VDCFollowOne, \VDCFollowTwoPlus ]
\item Data = \VDCData
\item Sign = \VDCSign
\end{itemize}