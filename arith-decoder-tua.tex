\begin{comment}
This section specifies the operation of the unsigned truncated unary
arithmetic decoding function ut\_arith\_decode() in terms of binary
arithmetic coding operations.

Pseudo-code for ut\_arith\_decode() is as follows, for values known to be
in the range :

\begin{verbatim}
VALUE=0

while ( !binary\_arith\_decode( choose\_context() ) && VALUE<N )

    VALUE++
\end{verbatim}

choose\_context() is a function that produces a context with which the
binary bit shall be decoded. The value it returns can depend on any
values known to the decoder at the time it is called, especially
including the binarisation bin (the bin number is equal to VALUE+1
according to the conventions of Appendix ).

\end{comment}

%src tim 0.9.1.48

reads and returns an unsigned integer encoded in the bytestream as an
arithmetic coded truncated unary binarisation. ``context\_list'' is a list
of contexts for each bin. If the number of contexts in the list is less
than the bin number then the content of the bin is assumed to be 1 (i.e.
the conditional probability of that bin, the context, is exactly 1).

Read Truncated Arithmetic Coded Unsigned Integer
\begin{verbatim}
read_uua(context_list):

    value = 0
    context_index = 0 #Bin Number (numbered from zero)
    max_index = len(context_list) - 1
    # Read first bit
    context = context_list[context_index]
    if ( read_ba(context) ):
        more = False
    else:
        value += 1
        more = True
    context_list[context_index] = context
    # Read remaining bits
    while ( (context_index < max_index) and more ):
        context_index += 1
        context = context_list[context_index]
        if ( read_ba(context) ):
            more = False
        else:
            value += 1
            more = True
        context_list[context_index] = context
    return value
\end{verbatim}
