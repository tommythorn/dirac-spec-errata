\label{colourmatrix}
Luma and chroma values EY, ECb, ECr should be used to derive ER, EG, EB
values by the following equations.



This follows by inverting the equations 



In the case of YCgCo coding, ER, EG, EB should be directly computed from
the integer Y, Cg and Co values by the following recipe, whereby integer
RGB IR, IG, IB values are decoded by

Y-=LUMA\_OFFSET

Cg-=CHROMA\_OFFSET

Co-=CHROMA\_OFFSET

TEMP=Y-(Cg>>1)

IG=TEMP+Cg

IB=TEMP-(Co>>1)

IR=IB+Co

These may be scaled down by dividing by (255<<ACC\_BITS) and clipped to
[0,1] to give ER, EG, EB. If the inverse transform has been correctly
applied prior to coding and lossless coding employed, then clipping will
be unnecessary.

Note that this matrix implies that the chroma range is twice as large as
the RGB range (and the luma range), since the chroma components involve
subtraction. Although logically knowing the signal range and scaling
signals is prior to performing matrixing, the matrix parameters are
coded first in the Display Parameters in order to allow the signal
ranges to be correctly determined in this case.


