This document specifies the Dirac decoder and stream structure in terms of
a layered model:

\begin{itemize}
    \item Stream data access
    \item Parsing and interpretation of the Dirac stream
    \item Picture decoding operations
\end{itemize}

Stream data access consists of the operations used to extract data values
(of boolean and integer type) from a raw Dirac bitstream. These include
data that has been encoded "literally" (i.e. according to conventional bit-wise
representations), variable-length codes, and data entropy coded using arithmetic
encoding.Stream data access methods are used both by parsing and decoding operations.

Parsing and interpretation defines the structure of Dirac streams, and defines
intermediate decoder data structures in which extracted data is stored, which 
encapsulate \em{both} meta-data used to control picture decoding processes (for 
example, motion compensation block sizes and overlaps, picture dimensions and
so forth) \em{and} the blocks of (arithmetically coded) data used as input to 
these processes.

Picture decoding operations produce decoded pictures from these populated
data structures by applying specified functions to them. 

It is unusual in a specification to separate these layers quite so distinctly, 
and our purpose in doing so is to provide much greater clarity. For implementors,
the decoupling of syntax ordering from semantics will (we hope) avoid imposing
implicit design decisions merely through the style of the specification. Many
other users of the specification will not be interested in the precise format
of syntax elements but in how the underlying algorithm works - or vice-versa.
It should be possible to construct a Dirac parsing engine, for example for
frame skipping in video playback applications, extremely simply and without
requiring comprehension of the entire specification.

It is worth observing, nevertheless, that this partition of functionality has 
some difficulties. In particular, the arithmetic decoding of wavelet subbands and 
has been assigned to layer 3, since arithmetic
decoding is defined as a basic data access method. However, these processes
do complicated things to manage contexts, perform predictions
and could have been considered to be part of picture decoding.

This layered structure is reflected in the structure of the specification,
which is divided into four parts. The first part introduces Dirac concepts,
both informatively, in terms of describing Dirac video coding processes, and
normatively, in terms of defining terms, concepts and conventions used 
throughout the specification.

The second part defines stream data access functions for data types. Part III
deals with accessing, parsing the Dirac bitstream and populating data
structures, including the wavelet coefficient and motion data.

Part IV defines high-level decoder operations and picture output, specifically the
inverse wavelet transform and motion compensation.

In addition to these parts, appendices deal with standard settings, ... TBC
