%%%%%%%%%%%%%%%%%%%%%%%%%%%%%%%%%%%%%%%%%%%%%%%%%%%%%
% - This chapter defines the arithmetic decoding  - %
% - engine                                        - % 
%%%%%%%%%%%%%%%%%%%%%%%%%%%%%%%%%%%%%%%%%%%%%%%%%%%%%

\label{arithdecoding}

This section describes the arithmetic decoding engine and
processes for using it to extract data from the Dirac stream.

The arithmetic decoding engine consists of two elements: 

\begin{itemize}
\item a collection of state variables representing the state of the arithmetic 
decoder (Section \ref{initarith})
\item a set of functions for extracting values from the decoder 
and updating the decoder state (Section \ref{extractarith})
\end{itemize}

\subsection{State and contexts}
\label{arithcontexts}

The arithmetic decoder state consists of the following decoder state variables:

\begin{itemize}
\item $\ALow$, an integer representing the beginning of the current coding interval
\item $\AHigh$, an integer representing the end of the current coding interval
\item $\ACode$, an integer within the interval from $\ALow$ to $\AHigh$, determined from the encoded bitstream
\item $\ABitsLeft$, a decrementing count of the number of bits yet to be read in
\item $\AContexts$, a map of all the contexts used in the Dirac decoder
\end{itemize}

A context $context$ is an integer array consisting of two positive values,
$context[0]$, and $context[1]$, representing counts of values $0$ and $1$
respectively. Contexts are accessed by decoding functions
via the indices defined in Section \ref{contextindices}.

\subsubsection{Rescaling contexts}
\label{rescalecontext}

An individual context is rescaled by halving the counts of $0$ and $1$ and ensuring that
these counts do not reach zero:

\begin{pseudo}{rescale\_context}{context}
\bsCODE{context[0] \gg= 1}
\bsCODE{context[0] += 1}
\bsCODE{context[1] \gg= 1}
\bsCODE{context[1] += 1}
\end{pseudo}

\subsection{Initialisation}
\label{initarith}

At the beginning of the decoding of any data unit, the arithmetic
decoding state is initialised as follows:

\begin{pseudo}{initialise\_arithmetic\_decoding}{block\_data\_length}
\bsCODE{\ABitsLeft=8*block\_data\_length}
\bsCODE{\ALow = \text{0x0000}}
\bsCODE{\AHigh =\text{0x0000}}
\bsCODE{\ACode =\text{ 0x0000}}
\bsCODE{init\_contexts()}
\end{pseudo}

Contexts are initialised by the $init\_contexts()$ function as follows:

\begin{pseudo}{init\_contexts}{}
\bsFOR{i=0}{length(\AContexts)-1}
  \bsCODE{\AContexts[i][0]=1}
  \bsCODE{\AContexts[i][1]=1}
\bsEND
\end{pseudo}

\subsection{Data input}
\label{inputarith}

The arithmetic decoding process accesses data in a contiguous block of bytes
whose size is set on initialisation (Section \ref{initarith}). The bits in this
block are sufficient to allow for the
decoding of all coefficients. However, the specification of arithmetic
decoding operations in this section may occasionally cause further bits to be read,
even though they are not required for determining decoded values. For this
reason a read function $read\_bita()$ is defined which returns $0$ if the
bounds of this block of data have been exceeded:

\begin{pseudo}{read\_bita}{}
\bsIF{\ABitsLeft==0}
  \bsRET{0}
\bsELSE
  \bsCODE{\ABitsLeft -= 1}
  \bsRET{read\_bit()}
\bsEND
\end{pseudo}

\begin{informative}
The Dirac arithmetic decoding engine uses 16 bit words, and so no more than 16
additional bits can be read beyond the end of the block. Hence it is sufficient
to read in the entire block of data and pad the end with two zero bytes to
avoid a branch condition with each input bit.
\end{informative}

\subsection{Decoder functions}
\label{extractarith}
The arithmetic decoding engine is a multi-context, adaptive binary
arithmetic decoder, performing binary renormalisation and producing
binary outputs. For each bit decoded, the semantics of the relevant
calling decoder function determine which contexts are passed to the
arithmetic decoding operations.

\subsubsection{Shifting bits in}

\label{arithshiftin}

This section defines the operation of the $shift\_bit\_in()$ 
and $shift\_all\_bits()$ functions
for reading bits into the arithmetic decoding state variables.

\begin{pseudo}{shift\_bit\_in}{}
\bsCODE{\AHigh \ll= 1}
\bsCODE{\AHigh \&= \text{0xFFFF}}
\bsCODE{\AHigh += 1}
\bsCODE{\ALow \ll= 1}
\bsCODE{\ALow \&= \text{0xFFFF}}
\bsCODE{\ACode \ll= 1}
\bsCODE{\ACode \&= \text{0xFFFF}}
\bsCODE{\ACode += read\_bita()}{\ref{inputarith}}
\end{pseudo}

$shift\_all\_bits()$ expands the interval between $\ALow$ and $\AHigh$
until the msbs (bit 15) differ and the interval no longer
straddles the half-way point 0x8000.

\begin{pseudo}{shift\_all\_bits}{}
\bsWHILE{ \AHigh\&\text{0x8000})==\text{0x0} \&\& (\ALow\&\text{0x8000})==\text{0x0}}
  \bsCODE{shift\_bit\_in()}
\bsEND
\bsWHILE{ (\AHigh\&\text{0x4000})==\text{0x0} \text{and} (\ALow\&\text{0x4000})==\text{0x4000} }
  \bsCODE{\ACode \hat{\text{ }}= \text{0x4000}}
  \bsCODE{\AHigh \hat{\text{ }}= \text{0x4000}}
  \bsCODE{\ALow \hat{\text{ }}= \text{0x4000}}
  \bsCODE{shift\_bit\_in()}
\bsEND
\end{pseudo}

\begin{informative}
Note that if 16-bit words (unsigned shorts) are used for decoder state variables $\ALow$,
 $\AHigh$ and $\ACode$ then there is no need for {\&}-ing with 0xFFFF. However, the 
operations specified here are defined in terms of integers, since intermediate calculations
 require higher dynamic range. In software, the efficiency of using short word lengths may
or may not be offset by the requirement to cast to other data types for these calculations.
\end{informative}

\subsubsection{Decoding boolean values}

\label{arithreadbool}

This section specifies the operation of the $read\_boola()$ function
for extracting a boolean value from the Dirac stream. Before extracting
any values, all possible bits are shifted in to ensure that the decoding
state has maximum information.

\begin{pseudo}{read\_boola}{context\_index}
\bsCODE{shift\_all\_bits()}{\ref{arithshiftin}}
\bsCODE{context=\AContexts[context\_index]}
\bsCODE{weight = context[0] + context[1]}
\bsCODE{scaler = (\text{0x10000}+weight//2)//weight}
\bsCODE{probability0 = context[0]*scaler}
\bsCODE{count = code-low+1}
\bsCODE{range = high-low+1}
\bsCODE{range\_times\_prob = (range * probability0)>>16}
\bsIF{ count > range\_times\_prob }
  \bsCODE{value = \true}
  \bsCODE{low = low + range\_times\_prob}
  \bsCODE{context[1] += 1}
\bsELSE
  \bsCODE{value = False}
  \bsCODE{high = low + range\_times\_prob - 1}
  \bsCODE{context[0] += 1}
\bsEND
\bsIF{ (context[0] + context[1]) > 255 }
  \bsCODE{rescale\_context(\AContexts[context\_index])}{\ref{rescalecontext}}
\bsEND
\bsRET{value}
\end{pseudo}

\begin{informative}
The function scales the probability of $0$ from the decoding context
so that a probability of $1$ is commensurate with the interval between
 $\ALow$ and $\AHigh$. If $\ACode$ is greater than this cut-off, then 1 ($\true$) has
been encoded, else 0 ($\false$) has.
\end{informative}

\subsubsection{Arithmetic decoding of integer values}

\label{arithreadint}

This section defines the operation of the $read\_sinta(context\_set)$ function
 for extracting integer values from a block of arithmetically coded data.

\paragraph{Binarisation and contexts \\}

Signed and unsigned integer values are binarised using interleaved exp-Golomb
 binarisation as per Section \ref{vlc}: the $read\_sinta()$ and $read\_uinta()$
processes are essentially identical to the 
$read\_sint()$ and $read\_uint()$ processes, except that instances of $read\_bool()$ are replaced
by instances of $read\_ba()$ (Section \ref{arithreadbool}) using suitable contextualisation. 

A choice of context depends upon whether the bit is a data bit, follow bit, or sign bit, and the 
position of the bit within the binarisation: $context\_set$ consists of three parts -
\begin{itemize}
\item an array of follow contexts, $context\_set[follow]$ (indexed from 0 to 
$\length(context\_set[follow])-1$)
\item a single data context $context\_set[data]$ 
\item a sign context $context\_set[sign]$ (ignored for unsigned integer decoding)
\end{itemize}

Each follow context is used for decoding the corresponding follow bit, with the
last follow context being used for all subsequent follow bits (if any) also. 
The follow context selection function $follow\_context()$ is defined by:

\begin{pseudo}{follow\_context}{index, context\_set}
\bsCODE{pos= max(index, length(context\_set[follow])-1 }
\bsCODE{ctx_index = context\_set[follow][pos]}
\bsRET{\Contexts[ctx\_index]}
\end{pseudo}

So the last follow context is used for all the remaining follow bits also.

\paragraph{Unsigned integer decoding \\}

Unsigned integers are decoded by:

\begin{pseudo}{read\_uinta}{context\_set}
\bsCODE{value = 1}
\bsCODE{index = 0}
\bsWHILE{read\_ba(follow\_context(index,context\_set) )== \false}
  \bsCODE{value \ll = 1}
  \bsIF{read\_ba(\text{state[contexts]}[context\_set[data])])}
    \bsCODE{value += 1}
  \bsEND
  \bsCODE{index += 1}
\bsEND
\bsCODE{value -= 1}
\bsRET{value}
\end{pseudo}

\paragraph{Signed integer decoding \\}

$read\_sinta()$ decodes first the magnitude then the sign, as necessary:

\begin{pseudo}{read\_sinta}{context\_set}
\bsCODE{value=read\_uinta(context_set)}
\bsIF{value != 0}
  \bsIF{read\_ba(\Contexts[context\_set[sign])]) == \true}
    \bsCODE{value=-value}
  \bsEND
\bsEND
\bsRET{value}
\end{pseudo}

\subsection{Context indices}
\label{contextindices}

The following is a list of all the context indices used in Dirac arithmetic decoding operations:

%\begin{tabular}{l}
\SignZero\\
\SignPos\\
\SignNeg\\
\ZPZNFollowOne\\
\ZPNNFollowOne\\
\ZPFollowTwo\\
\ZPFollowThree\\
\ZPFollowFour\\
\ZPFollowFive\\
\ZPFollowSixPlus\\
\NPZNFollowOne\\
\NPNNFollowOne\\
\NPFollowTwo\\
\NPFollowThree\\
\NPFollowFour\\
\NPFollowFive\\
\NPFollowSixPlus\\
\CoeffData\\
\ZeroCodeblock\\
\QOffsetFollow\\
\QOffsetInfo\\
\QOffsetSign\\
\SBSplitFollowOne\\
\SBSplitFollowTwo\\
\SBSplitData\\
\PredModeOne\\
\PredModeTwo\\
\BlockGlobal\\
\RefOnexFollowOne\\
\RefOnexFollowTwo\\
\RefOnexFollowThree\\
\RefOnexFollowFour\\
\RefOnexFollowFivePlus\\
\RefOnexData\\
\RefOnexSign\\
\RefOneyFollowOne\\
\RefOneyFollowTwo\\
\RefOneyFollowThree\\
\RefOneyFollowFour\\
\RefOneyFollowFivePlus\\
\RefOneyData\\
\RefOneySign\\
\RefTwoxFollowOne\\
\RefTwoxFollowTwo\\
\RefTwoxFollowThree\\
\RefTwoxFollowFour\\
\RefTwoxFollowFivePlus\\
\RefTwoxData\\
\RefTwoxSign\\
\RefTwoyFollowOne\\
\RefTwoyFollowTwo\\
\RefTwoyFollowThree\\
\RefTwoyFollowFour\\
\RefTwoyFollowFivePlus\\
\RefTwoyData\\
\RefTwoySign\\
\YDCFollowOne\\
\YDCFollowTwoPlus\\
\YDCData\\
\YDCSign\\
\UDCFollowOne\\
\UDCFollowTwoPlus\\
\UDCData\\
\UDCSign\\
\VDCFollowOne\\
\VDCFollowTwoPlus\\
\VDCData\\
\VDCSign
%\end{tabular}