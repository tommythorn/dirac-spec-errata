The consequence of processing the Intra frames by removing the DC
values, and the Inter frames by removing the predicted values gives us a
difference signal. This difference signal is hopefully largely zero, but
can have some large peaks, of either polarity. The signal usually has a
large amount of low-frequency energy, but with occasional elements of
high frequency as the prediction process gets it wrong.

Conventional theory says that we can manipulate this signal in the
frequency domain to reduce the amount of information we need to
transmit. The properties of the eye are such that many of the higher
frequency components are less sensitive to coarse quantisation.

We can use wavelet transforms in Dirac. The wavelet transform
decorrelates the data in a roughly frequency-sensitive way, and
preserves the fine details of images better than the ubiquitous Discrete
Cosine Transform.
