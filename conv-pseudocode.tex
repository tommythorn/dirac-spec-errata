The specification is written in a style of pseudocode. The intention is
to make it easy to understand, but at the same time to be reasonably
rigorous. We chose to base this loosely on the syntax used in Python. We
have to emphasise that this is pseudocode: it only resembles Python. In
some aspects the pseudocode differs from the conventional Python syntax.

The pseudocode is intended to achieve maximum clarity and hence it may
contain redundant code and other inefficiencies.

As we are doing little more than read some data and assign that data to
variables, only a small subset of the Python syntax is used. The
elements used are introduced in the next subsection for the benefit of
those who are not familiar with Python. The interested reader is
recommended to pick up one of the vast wealth of books on Python for
further information.

In the pseudocode, variable assignments and function calls that
represent data in the byte stream are emphasised by bold text.
