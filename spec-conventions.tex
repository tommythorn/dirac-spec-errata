%%%%%%%%%%%%%%%%%%%%%%%%%%%%%%%%%%%%%%%%%%%%%%%%%%
% - This chapter defines specification         - %
% - conventions                                - % 
%%%%%%%%%%%%%%%%%%%%%%%%%%%%%%%%%%%%%%%%%%%%%%%%%%

\subsection{State machine decoder representation}

This specification uses a state-machine model to express parsing and
decoding operations. The state of the decoder/parser is stored in the
global variable $\StateName$, and individual variable values are accessed
by means of named tokens, e.g. $\StateName[var\_name]$ (i.e. they
are {\em maps} as defined in Section \ref{datatypes}). All the individual
variables are therefore also globally accessible from all decoder functions.

A default state variable, $\SeqStateName$ is also defined, which is
initialised on accessing the stream. State values revert to the default
state values on beginning to parse each picture, as described in 
Section \ref{streamstructure}.

The parsing and decoding operations are specified in terms of modifying
the decoder state. The state variables may not directly correspond
to elements of the stream, but may be calculated from them taking into
account the decoder state as a whole. For example, a state variable value 
may be differentially encoded with respect to another value, with the
difference, not the variable itself, encoded in the stream. 

The Dirac stream syntax is illustrated in parse diagrams, which 
are presented in Appendix \ref{parsediagrams}. The parsing process
is defined by means of pseudocode and/or mathematical formulae. The
conventions for these elements are described in the following sections.
In the event of any conflict between the parse diagrams and the specified
parsing processes, the latter shall be deemed to be correct.

\subsection{Number formats}
\label{mathnotation}

Numbers without a prefix shall be interpreted as decimal numbers.

The prefix b indicates that the following value is to be interpreted as a binary
natural number (non-negative integer). 

{\bf Example} The value b1110100 is equal to the decimal value 116. 

The prefix 0x indicates the following value is to be interpreted as a hexadecimal (base 16)
natural number. 

{\bf Example} The value 0x7A is equal to the decimal value 122. 

\subsection{Data types}
\label{datatypes}

\subsubsection{Elementary data types}

Only three basic types are used in the pseudo code:
\begin{description}
\item[Boolean] - A Boolean variable that has only two possible values: $\true$ and $\false$.
\item[Integer] - A positive or negative whole number or zero.
\item[Unsigned] - A non-negative ($\geq 0$) whole number, of arbitrary size.
\end{description}

\subsubsection{Compound data types}

There are three types of compound data:

\begin{description}
\item[Set] A collection of variables or values with no particular indexing. 
A set containing members $A,B,C,\ldots$ is denoted $\{A,B,C,\ldots\}$. The usual
set-theoretic operations $\cup$ (union), $\cap$ (intersection), $\in$ (membership)
apply. Note that sets are used to define other entities in this specification, s
uch as mean and median values and other compound data types, and are used in the control structures of the pseudocode.

\item[Map]

A map is a set accessed by token names. For example
$p[Y]$, $p[C1]$, $p[C2]$ might give the value of the different video components
(Y, C1 and C2) of a pixel. The set of argument tokens of a map $m$ can be accessed by $\args(m)$,
so that $\args(p)=\{Y,C1,C2\}$.

\item[Array] 

An array is a set with an integer index or indices. All arrays are indexed from 0. 
Elements of a 1-dimensional array $a$ are accessed by $a[n]$ for $n$ in the
range 0 to $\length(a)-1$. Elements of a 2-dimensional array are accessed by
$a[n][m]$ for $0\leq m\leq\width(a)-1$ and $0\leq n\leq\height(a)-1$.

A 1-dimensional array can be explicitly defined by the syntax $a=[u, v, w, \hdots]$.
Then $a[0]=u$, 
$a[1]=v$ and so on.
\end{description}

The elements contained within any compound data type may be other compound data types.


\subsection{Numerical functions and operators}
\label{functionoperators}

This section defines the functions and operators used 
in the pseudo-code in this specification. Functions and operators
are similar but functions use the syntax, $(arg1, arg2,\ldots)$ 
whereas operators are simply placed before or between operands, 
e.g. $a+b$. The difference is purely syntactic and is to 
correspond with conventional mathematical notation.

All arithmetic defined by this specification is exact: the entire 
specification can be implemented using only integer and logical operations.
All operations are to be implemented with sufficiently large integers so 
that overflow cannot occur.

\subsubsection{Assignment}

The assignment operation  = applies to all variable types. After performing 
\[a=b\]
the value of $a$ shall become equal to that of $b$, and the value of $b$ shall remain unchanged.

\subsubsection{Boolean functions and operators}
\label{booleanops}
The following functions and operators are defined for one or more Boolean arguments:
\begin{description}
\item[not or !] 		(not a) or !a returns True for a boolean value a if and only if a is False
\item[and] 		(a and b) returns True if and only if a and b are both True. Operator "and" may be used in pseudo-code conditions to denote the logical AND between Boolean values, for example: if (condition1 and condition2): �etc.
\item[or] 		(a or b) returns True if either a or b are True, else it returns False.  Operator "or" may be used in pseudo-code conditions to denote the logical OR between Boolean values, for example: if (condition1 or condition2): � etc.
\item[xor] 		(a xor b) returns True if one, and only one, of a or b are True, else it returns False.  Operator "xor" may be used in pseudo-code conditions to denote the logical exclusive OR between Boolean values, for example: if (condition1 xor condition2): � etc.
\item[majority]		Given a set, S = {s0,�, sn-1} of Boolean values, majority(S) returns the majority condition. That is, if the number of True values is greater than or equal to the number of False values, majority(S) returns True, otherwise it returns False.
\end{description}

Boolean operations are to be distinguished from bitwise operations which operate on non-negative 
integer values, and are defined in Section \ref{integerops}.

\subsubsection{Integer functions and operators}

The following functions and operators are defined on integer values:

\begin{description}
\item[Absolute value] $|a|=\left\lbrace\begin{array}{l} a \text{ if $a\geq 0$}\\ 
                                                                                   -a \text{ otherwise} \end{array}\right.$.

\item[Sign] $\sign(a)$ is defined by
\[\sign(a)=
\left\{\begin{array}{l} 
1 \text{ if $a>0$} \\
-1 \text{ if $a<0$} \\
0 \text{ if $a=0$}
\end{array}\right.\]

\item[Addition] The sum of $a$ and $b$ is represented by $a+b$.

\item[Subtraction] $a$ minus $b$ is represented by $a-b$.

\item[Multiplication] $a$ times $b$ is represented, for clarity, by $a*b$.

\item[Real division] The real number value of $a$ divided by $b$ is represented by $a/b$ or $\frac{a}{b}$.

\item[Exponentiation] For integers $a, b$, $b>0$ $a^b$ is defined as $a*a*\hdots *a$ ($b$ times). $a^0$ is 1.

\item[Ceiling] $\lceil a \rceil$, the least integer greater than or equal to a real number $a$

\item[Floor] $\lfloor a \rfloor$, the greatest integer less than or equal to a real number $a$

\item[Maximum] $\max(a,b)$ returns the largest of $a$ and $b$.

\item[Minimum] $\min(a,b)$ returns the smallest of values $a$ and $b$.

\item[Clip] $\clip(a,b,t)$ clips the value $a$ to the range defined by $b$ (bottom)
and $t$ (top):
\[\clip(a,b,t)=\min(\max(a,b),t)\]

\item[Integer division] Integer division is defined for $a$ and $b$ integer values, $b>0$ by

\[a//b= \left\{ 
\begin{array}{l} 
\left\lfloor \dfrac{a}{b} \right\rfloor \text{ if } a>0 \\
\\
\left\lceil \dfrac{a}{b} \right\rceil \text{ if } a\leq 0
\end{array} \right.
\]

i.e. round towards zero as per C/C++ conventions. Note that explicit division of 
potentially negative numbers is avoided in this specification for clarity.

\item[Remainder] For integers $a,b$, with $b>0$, the remainder $a\%b$ is equal to 

\[a-\left\lfloor\dfrac{a}{\left| b \right|}\right\rfloor *\left| b\right| \]

 $a\%b$ always lies between 0 and $b-1$.

\item[Shift down] For integers $a,b$, with $b\geq 0$, $a\gg b$ is defined as 

\[\left\lfloor\dfrac{a}{2^b}\right\rfloor \]

Note that this is {\em not} equivalent to $a//2^b$, since shifting down rounds towards $-\infty$, not 0.

\item[Shift up] For integers $a,b$, with $b\geq 0$, $a\ll b$ is $a*2^b$.

\item[Rounded integer division] For integers $a,b$ with $b>0$, $\rounddivide(a,b)$ is the unbiased
integer division of $a$ by $b$, defined by

\[\left\lfloor\dfrac{a+(b//2)}{b}\right\rfloor\]

\end{description}

\begin{informative}
Note that $\rounddivide(a,2^b)=(a+2^{b-1})\gg b$. Note also that if $N$ is any sufficiently large integer 
(i.e. bigger than $\left\lceil\dfrac{a}{b}\right\rceil$), then

\[\rounddivide(a,b)=(a+b*N+(b//2))//b - N\]

which may be used to avoid a branch condition.
\end{informative}

\begin{description}
\item[Mean] Given a set  $S=\{s_0, s_1, \hdots, s_{n-1}\}$ of integer values, the integer unbiased mean, $\mean(S)$, is defined
to be

\[\rounddivide(s_0+s_1+\hdots +s_{n-1},n)\]

\item[Median] Given a set $S=\{s_0, s_1, \hdots, s_{n-1} \}$ of integer values the median, $\median(S)$, 
returns the middle value. If $t_0\leq t_1\leq \hdots \leq t_{n-1}$ are the values $s_i$ placed in ascending order, this
is 

$t_{(n-1)/2}$ 

if $n$ is odd and

$\mean(\{ t_{(n-2)/2},t_{n/2}\})$ if $n$ is even. If $S=\emptyset$, $\median(S)$ returns 0.
\end{description}
The following bitwise operations are defined on non-negative integer values:
\begin{description}
\item[\&] Logical AND is applied between the corresponding bits in the binary representation of two numbers, e.g.
$13\&6$ is $\text{b1101}\&\text{b110}$, which equals $\text{b100}$, or 4.

\item[${\mathbf |}$] Logical OR is applied between the corresponding bits in the binary representation of two numbers, e.g.
$13|6$ is $\text{b1101}\text{|}\text{b110}$, which equals $\text{b1111}$, or 15.

\item[${\mathbf \wedge}$] Logical XOR is applied between the corresponding bits in the binary representation of two numbers, e.g.
$13\wedge 6$ is $\text{b1101}\wedge\text{b110}$, which equals $\text{b1011}$, or 11.
\begin{comment}
\item[!] Logical NOT is applied to all bits in the representation of a number e.g. $!(b010)=b101$.
\end{comment}
\end{description}
These operations are also defined on boolean values, interpreted as single-bit integers, 
where 0 is interpreted as $\false$ and 1 as $\true$ and vice-versa. Logical NOT is 
not defined bitwise, to avoid ambiguity concerning leading zeroes).

\subsubsection{Set functions and operators}

The following functions and operators are defined for sets, maps and arrays. 
\begin{description}
\item[Indexing]		For an array $a$, $a[index]$ returns an element of $a$. If $a$ is a map the index shall be a token, else if $a$ is a map the index shall be an integer.
\item[Scalar Assignment]	Where the notation $a = 0$ is used for an array of integer values, it means "set all elements of the array to zero".
\item[Insertion]		a[index] = b inserts a copy of b into set a if the element does not already exist.
\item[Tokens]		for a map a, arg(a) returns the set of the indexing tokens.
\item[Length]		for a one dimensional array a, length(a) returns the number of elements in the array.
\item[Width]		for a two dimensional array a, width(a) returns the width the array. The width is the number of scalar elements corresponding to the right most array index.
\item[Height]					for a two dimensional array a, height(a) returns the height the array. The height is the number of one dimensional arrays in the two dimensional array and the "height" dimension corresponds to the left most array index.
\end{description}
\subsubsection{Precedence and associativity of operators}
\label{operatorprecedence}
To avoid any confusion over the order of operator precedence, every equation makes extensive use of the expression operators "(" and ")". All operations recursively execute the innermost expression(s) first until the calculation has been completed. In cases where the expression operators do not make clear the order of precedence, the following table defines the descending order of operator precedence and the associativity of each operator.
[Table tbc]
\begin{comment}
Operator Precedence	Associativity
( ) [ ]	left to right
* // %	left to right
+ -	left to right
<< >>	left to right
< <= > >=	left to right
== !=	left to right
! (not)	right to left
& (and)	left to right
^ (xor)	left to right
|	left to right
= += -= *= //= %= &= ^= |= <<= >>=	right to left
\end{comment}

\subsection{Pseudocode}
\label{pseudocode}

The bulk of the normative specification is defined by means of pseudocode. 
The syntax used is an amalgam of Python and BASIC. It is not intended
to be executable code, but rather both precise and descriptive. 
In particular, all the arithmetic functions and
operations defined in the preceding section may be applied to variables within
a pseudocode process.

\subsubsection{Processes}
\label{functionsprocesses}

Decoding and parsing operations are specified by means of processes
 -- a series of operations acting on input data and global variable data. 
A process can also be a function, which means it returns a value, but
it need not do so. So a process
taking in variables $in1$ and $in2$ looks like:

\begin{pseudo}{foo}{in1, in2}
\bsCODE{op1(in1)}
\bsCODE{op2(in2)}
\bsCODE{\hdots}
\end{pseudo}

whilst a function process looks like

\begin{pseudo}{bar}{in1, in2}
\bsCODE{op1(in1)}
\bsCODE{foo(in1,in2)}{\ref{functionsprocesses}}
\bsCODE{\hdots}
\bsRET{out1}
\end{pseudo}

The right-hand column in the pseudocode representation contains a cross-reference to the 
section in the specification containing the definition of other processes used at that line.

Note well: all input variables are deemed to be passed {\em by reference} in this
specification. This means that any modification to a variable value that
occurs within a process also applies to that variable within the calling process
{\em even if it has a different name} in the calling process. One way to understand
this is to envisage variable names as labels for pointers to workspace memory.

For example, if we define $foo$ and $bar$ by

\begin{pseudo}{foo}{}
\bsCODE{num=0}
\bsCODE{bar(num)}
\bsCODE{\StateName[var\_name]=num}
\end{pseudo}

and 

\begin{pseudo}{bar}{val}
\bsCODE{val=val+1}
\end{pseudo}

then at the end of $foo$, $\StateName[var\_name]$ has been set to 1.

The only global variables are the state variables encapsulated in $\StateName$
and $\SeqStateName$. If a variable is not declared as an input to
the process and is not a state variable, then it is local to the function.

If a process is particularly complex, it may be broken into a number of steps with 
intermediate discussion. This is signalled by appending  and prepending ``$\hdots$" to
the parts of the pseudocode specification:

\begin{pseudo}{foo}{}
\bsCODE{code}
\bsCODE{\hdots}
\end{pseudo}

[text]

\begin{pseudo*}
\bsCODE{more code}
\bsCODE{\hdots}
\end{pseudo*}

[text]

\begin{pseudo*}
\bsCODE{even more code}
\end{pseudo*}

Note that the intervening text may define or modify variables used in the succeeding
pseudocode, and must be considered as a normative part of the specification of the process.
This is done as it is sometimes much more clear to split up a long and complicated process
into a number of steps.


\paragraph{Assignment and equality}
$\ $\newline$\ $\newline
Assignment between variables $a$ and $b$ is denoted by $a=b$, and is a copy
operation between the data contained within $b$ to $a$. 

It is to be distinguished from
the boolean identity operator $a==b$ (Section \ref{booleanops}).

Occasionally the notation $a={\bf 0}$ will be used for an array of integer values:
it means set all elements of the array to 0.

For integer variables, assignment can be combined with arithmetic and bit-wise
operations in the usual programming manner: for example, 
\[x+=y\]
means $x=x+y$, and
\[x |= \text{0x4B}\]
means $x=x|\text{0x4B}$.

\subsubsection{Control flow}
\label{controlflow}

The pseudocode comprises a series of statements, linked by functions and
flow control statements such as {\bf if}, {\bf while}, and {\bf for}.

The statements do not have a termination character, unlike the ; in C
for example.  Blocks of statements are indicated by indentation:
indenting in begins a block, indenting out ends one.

Statements that expect a block (and hence a following indentation) end
in a colon.

\paragraph*{if}

The if control evaluates a boolean or boolean function, and if true, passes the 
flow to the block of following statement or block of statements. If the control
evaluates as false, then there is an option to include one or more else if
controls which offer alternative responses if some other condition is
true.  If none of the preceding controls evaluate to true, then there is
the option to include an else control which catches remaining cases.

\begin{pseudo*}
\bsIF{control1}
    \bsCODE{block1}
\bsELSEIF{control2}
    \bsCODE{block2}
\bsELSEIF{control3}
    \bsCODE{block3}
\bsELSE
    \bsCODE{block4}
\bsEND
\end{pseudo*}

The if and else if conditions are evaluated in the order in which they
are presented. In particular, if $control1$ or $control2$ is true in
the preceding example, $block3$ will not be executed
even if $control3$ is true; neither will $block4$.

\paragraph*{for}

The for control repeats a loop over an integer range of values. For example,

\begin{pseudo*}
\bsFOR{i=0}{n-1}
    \bsCODE{foo(i)}
\bsEND
\end{pseudo*}

calls $foo()$ with value $i$, as $i$ steps through from 0 to $n-1$ inclusive.


\paragraph*{for each} The for each control loops over the elements in
a list:

\begin{pseudo*}
\bsFOREACH{c}{Y,C1,C2}
    \bsCODE{block}
\bsEND
\end{pseudo*}

\paragraph*{for such that} The for such that control loops over elements in
a set which satisfy some condition:

\begin{pseudo*}
\bsFORSUCH{a\in A}{control}
    \bsCODE{block}
\bsEND
\end{pseudo*}

This may only be used when the order in which elements are processed is 
immaterial.

\paragraph*{while}

The while control repeats a loop so long as a switch variable is true. 
When it is false, the loop breaks to the next statement(s) outside the block.

\begin{pseudo*}
\bsWHILE{condition}
    \bsCODE{block1}
\bsEND
\bsCODE{block2}
\end{pseudo*}