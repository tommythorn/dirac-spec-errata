%%%%%%%%%%%%%%%%%%%%%%%%%%%%%%%%%%%%%%%%%%%%%%%%%%
% - This chapter defines specification         - %
% - conventions                                - % 
%%%%%%%%%%%%%%%%%%%%%%%%%%%%%%%%%%%%%%%%%%%%%%%%%%
\begin{comment}
This section is a bit of a jumble and needs reorganising
\end{comment}

\subsection{State machine decoder representation}

This specification uses a state-machine model to express parsing and
decoding operations. The state of the decoder/parser is stored in the
global variable $\StateName$, and individual variable values are accessed
by means of named tokens, e.g. $\StateName[var\_name]$ (i.e. they
are {\em maps} as defined in Section \ref{datatypes}). All the individual
variables are therefore also globally accessible from all decoder functions.

A default state variable, $\SeqStateName$ is also defined, which is
initialised on accessing the stream. State values revert to the default
state values on beginning to parse each picture, as described in 
Section \ref{streamparsing}.

The parsing and decoding operations are specified in terms of modifying
the decoder state. Decoder state variables may not directly correspond
to elements of the stream, but are calculated from them taking into
account the decoder state as a whole. For example, a state variable value 
may be differentially encoded with respect to another value, with the
difference, not the variable itself, encoded in the stream. 

The stream structure itself is summarised in parse diagrams, which 
are presented in Appendix \ref{parsediagrams}. The parsing process
is defined by means of pseudocode and/or mathematical formulae. The
conventions for these elements are described in the succeeding sections.
In the event of any conflict between the parse diagrams and the specified
parsing processes, the latter shall be deemed to be correct.

\subsection{Numbers and arithmetic operations}
\label{mathnotation}

\subsubsection{Numbers}
\label{numbers}

The prefix b indicates that the following value is to be interpreted as a binary
natural number (non-negative integer). 

{\bf Example} The value b1110100 is equal to the decimal value 116. 

The prefix 0x indicates the following value is to be interpreted as a hexadecimal (base 16)
natural number. 

{\bf Example} The value 0x7A is equal to the decimal value 122. 

\subsubsection{Arithmetic operations}
\label{arithoperations}

All arithmetic defined by this specification is exact: the entire 
specification can be implemented using only integer and logical operations.
All operations are to be implemented with sufficiently large integers so 
that overflow cannot occur.

The following arithmetic operators are defined on numerical values:

\paragraph*{Absolute value} $|a|=\left\lbrace\begin{array}{l} a \text{ if $a\geq 0$}\\ 
                                                                                   -a \text{ otherwise} \end{array}\right.$.

\paragraph*{Addition} The sum of $a$ and $b$ is represented by $a+b$.

\paragraph*{Subtraction} $a$ minus $b$ is represented by $a-b$.

\paragraph*{Multiplication} $a$ times $b$ is represented, for clarity, by $a*b$.

\paragraph*{Real division} The real number value of $a$ divided by $b$ is represented by $a/b$ or $\frac{a}{b}$.

\paragraph*{Exponentiation} For integers $a, b$, $b>0$ $a^b$ is defined as $a*a*\hdots *a$ ($b$ times). $a^0$ is 1.

\paragraph*{Ceiling} $\lceil a \rceil$, the smallest integer greater than or equal to a real number $a$

\paragraph*{Floor} $\lfloor a \rfloor$, the largest integer less than or equal to a real number $a$

\paragraph*{Maximum} $\max(a,b)$ returns the largest of $a$ and $b$.

\paragraph*{Minimum} $\min(a,b)$ returns the smallest of values $a$ and $b$.

\paragraph*{Clip} $\clip(a,b,t)$ clips the value $a$ to the range defined by $b$ and $t$:
\[\clip(a,b,t)=\min(\max(a,b),t)\]

\paragraph*{Integer division} Integer division is defined for $a$ and $b$ integer values, $b>0$ by

\[a//b= \left\lfloor \dfrac{a}{b} \right\rfloor\]

(i.e. always round down).

\paragraph*{Remainder} For integers $a,b$, with $b>0$, $a\%b$ is equal to $a-(a//b)*b$. $a\%b$ always lies between 0 and $b-1$.

\paragraph*{Shift down} For integers $a,b$, with $b\geq 0$, $a\gg b$ is $a//2^b$.

\paragraph*{Shift down} For integers $a,b$, with $b\geq 0$, $a\ll b$ is $a*2^b$.

\paragraph*{Mean} Given a set  $S=\{s_0, s_1, \hdots, s_{n-1}$ of integer values, the integer mean $\mean(S)$ is defined
to be

\[\left( s_0+s_1+\hdots +s_{n-1}+(n//2)\right //n)\]

\paragraph*{Median} Given a set $S=\{s_0, s_1, \hdots, s_{n-1}\}$ of integer values the median $\median(S)$ 
returns the middle value. If $t_0\leq t_1\leq \hdots \leq t_{n-1}$ are the values $s_i$ placed in ascending order, this
is 

$t_{(n-1)/2}$ 

if $n$ is odd and

$(t_{(n-2)/2}+t_{n/2})//2$ if $n$ is even. If $S=\emptyset$, $\median(S)$ returns 0.

The following bitwise operations are defined on non-negative integer values:
\paragraph*{\&} Logical AND is applied between the corresponding bits in the binary representation of two numbers, e.g.
$13\&6$ is $\text{b1101}\&\text{b110}$, which equals $\text{b100}$, or 4.

\paragraph*{$\mathbf{|}$} Logical OR is applied between the corresponding bits in the binary representation of two numbers, e.g.
$13|6$ is $\text{b1101}\text{|}\text{b110}$, which equals $\text{b1111}$, or 15.

\paragraph*{\Large$\hat\quad$} Logical XOR is applied between the corresponding bits in the binary representation of two numbers, e.g.
$13\hat\quad 6$ is $\text{b1101}\hat\quad\text{b110}$, which equals $\text{b1011}$, or 11.

These operations are also defined on boolean values, interpreted as single-bit integers, 
where 0 is interpreted as $\false$ and 1 as $\true$ and vice-versa. Logical NOT is 
not defined bitwise, to avoid ambiguity concerning leading zeroes).


\subsection{Pseudocode}
\label{pseudocode}

The bulk of the normative specification is defined by means of pseudocode. 
The syntax used is an amalgam of Python and Basic. It is not intended
to be executable code, but rather both precise and descriptive. 
In particular, all the arithmetic functions and
operations defined in the preceding section may be applied to variables within
a pseudocode process.

\subsubsection{Processes}
\label{functionsprocesses}

Decoding and parsing operations are specified by means of processes
 -- a series of operations acting on input data and global variable data. 
A process can also be a function, which means it returns a value, but
it need not do so. So a process
taking in variables $in1$ and $in2$ looks like:

\begin{pseudo}{foo}{in1, in2}
\bsCODE{op1(in1)}
\bsCODE{op2(in2)}
\bsCODE{\hdots}
\end{pseudo}

whilst a function process looks like

\begin{pseudo}{bar}{in1, in2}
\bsCODE{op1(in1)}
\bsCODE{foo(in1,in2)}{\ref{functionsprocesses}}
\bsCODE{\hdots}
\bsRET{out1}
\end{pseudo}

The right-hand column in the pseudocode representation contains a cross-reference to the 
section in the specification containing the definition of other processes used at that line.

Note well: all input variables are deemed to be passed {\em by reference} in this
specification. This means that any modification to a variable value that
occurs within a process also applies to that variable within the calling process
{\em even if it has a different name} in the calling process. One way to understand
this is to envisage variable names as labels for pointers to workspace memory.

For example, if we define $foo$ and $bar$ by

\begin{pseudo}{foo}{}
\bsCODE{num=0}
\bsCODE{bar(num)}
\bsCODE{\StateName[var\_name]=num}
\end{pseudo}

and 

\begin{pseudo}{bar}{val}
\bsCODE{val=val+1}
\end{pseudo}

then at the end of $foo$, $\StateName[var\_name]$ has been set to 1.

If a process is particularly complex, it may be broken into a number of steps with 
intermediate discussion. This is signalled by appending  and prepending ``$\hdots$" to
the parts of the pseudocode specification:

\begin{pseudo}{foo}{}
\bsCODE{code}
\bsCODE{\hdots}
\end{pseudo}

[text]

\begin{pseudo*}
\bsCODE{more code}
\bsCODE{\hdots}
\end{pseudo*}

[text]

\begin{pseudo*}
\bsCODE{even more code}
\end{pseudo*}

Note that the intervening text may define or modify variables used in the succeeding
pseudocode, and must be considered as a normative part of the specification of the process.
This is done as it is sometimes much more clear to split up a long and complicated process
into a number of steps.

\subsubsection{Variables and data types}
\label{datatypes}

The only global variables are the state variables encapsulated in $\StateName$
and $\SeqStateName$. If a variable is not declared as an input to
the process and is not a state variable, then it is local to the function.

The following basic types are defined:

\paragraph*{Boolean} A boolean variable has two possible states, $\true$ and $\false$.

\paragraph*{Unsigned integer} A non-negative ($\geq 0$) whole number, of arbitrary size.

\paragraph*{Integer} A whole number, of arbitrary size.

\paragraph*{Set} A collection of variables or values, with no particular indexing. The usual
set-theoretic operations such as $\cup$ (union), $\cap$ (intersection), $\in$ (membership)
and so on apply.

\paragraph*{Map}

A map is a set accessed by token names. For example
$p[Y]$, $p[U]$, $p[V]$ might give the value of the different video components
(Y, U and V) of a pixel. The set of argument tokens of a map $m$ can be accessed by $\args(m)$,
so that $\args(p)=\{Y,U,V\}$.

\paragraph*{Array} 

A list is a set with an integer index or indices. All arrays are indexed from 0. 
Elements of a 1-dimensional array $a$ are accessed by $a[n]$ for $n$ in the
range 0 to $\length(a)-1$. Elements of a 2-dimensional array are accessed by
$a[n][m]$ for $0\leq m\leq\width(a)-1$ and $0\leq n\leq\height(a)-1$.

A 1-dimensional array can be explicitly defined by the syntax $a=[u, v, w, \hdots]$.
Then $a[0]=u$, 
$a[1]=v$ and so on.

These basic variable types may be combined. For example, picture data may be considered
to be a map of arrays $pic$, where $pic[Y]$ is a 2-dimensional array storing luma data,
and $pic[U]$ and $pic[V]$ are two-dimensional arrays storing chroma data.

Variables within processes are not explicitly declared, and their type is determined from
context or defined in the surrounding description.

Assignment between variables $a$ and $b$ is denoted by $a=b$, and is a copy
operation between the data contained within $b$ to $a$. 

It is to be distinguished from
the boolean identity operator $a==b$ (Section \ref{booleanops}).

Occasionally the notation $a={\bf 0}$ will be used for an array of integer values:
it means set all elements of the array to 0.

For integer variables, assignment can be combined with arithmetic and bit-wise
operations in the usual programming manner: for example, 
\[x+=y\]
means $x=x+y$, and
\[x |= \text{0x4B}\]
means $x=x|0x4B$.

\subsubsection{Control flow}
\label{controlflow}

The pseudocode comprises a series of statements, linked by functions and
flow control statements such as if, while, and for.

The statements do not have a termination character, unlike the ; in C
for example.  Blocks of statements are indicated by indentation:
indenting in begins a block, indenting out ends one.

Statements that expect a block (and hence a following indentation) end
in a colon.

\paragraph*{if}

The if control evaluates a boolean or boolean function, and if true, passes the 
flow to the block of following statement or block of statements. If the control
evaluates as false, then there is an option to include one or more else if
controls which offer alternative responses if some other condition is
true.  If none of the preceding controls evaluate to true, then there is
the option to include an else control which catches remaining cases.

\begin{pseudo*}
\bsIF{control1}
    \bsCODE{block1}
\bsELSEIF{control2}
    \bsCODE{block2}
\bsELSEIF{control3}
    \bsCODE{block3}
\bsELSE
    \bsCODE{block4}
\bsEND
\end{pseudo*}

The if and else if conditions are evaluated in the order in which they
are presented. In particular, if $control1$ or $control2$ is true in
the preceding example, $block3$ will not be executed
even if $control3$ is true; neither will $block4$.

\paragraph*{for}

The for control repeats a loop over an integer range of values. For example,

\begin{pseudo*}
\bsFOR{i=0}{n-1}
    \bsCODE{foo(i)}
\bsEND
\end{pseudo*}

calls $foo()$ with value $i$, as $i$ steps through from 0 to $n-1$ inclusive.


\paragraph*{for each} The for each control loops over the elements in
a list:

\begin{pseudo*}
\bsFOREACH{c}{Y,U,V}
    \bsCODE{block}
\bsEND
\end{pseudo*}

\paragraph*{for such that} The for such that control loops over elements in
a set which satisfy some condition:

\begin{pseudo*}
\bsFORSUCH{a\in A}{control}
    \bsCODE{block}
\bsEND
\end{pseudo*}

This may only be used when the order in which elements are processed is 
immaterial.

\paragraph*{while}

The while control repeats a loop so long as a switch variable is true. 
When it is false, the loop breaks to the next statement(s) outside the block.

\begin{pseudo*}
\bsWHILE{condition}
    \bsCODE{block1}
\bsEND
\bsCODE{block2}
\end{pseudo*}

\subsubsection{Logical (boolean) operations}
\label{booleanops}

A logical operator takes a variable or pair of variables as arguments and
returns the boolean values $\true$ or $\false$.

The following logical operators are defined:

\paragraph*{$==$} Test of equality of two variables. $a==b$ is $\true$ if and
only if the value of $a$ equals the value of $b$. 

\paragraph*{$<$}	Less than

\paragraph*{$\leq$} Less than or equal to

\paragraph*{$>$}	Greater than

\paragraph*{$\geq$} Greater than or equal to

\paragraph*{$!$} Not. $!a$ is $\true$ for a boolean value $a$ if and only if $a$ is $\false$

\paragraph*{$!=$} not equal to. $a!=b$ is equivalent to $!(a==b)$

When used in pseudocode conditions, the words ``and" and ``or" are used to denote logical AND
and logical (inclusive) OR between boolean values, for example:

\begin{pseudo*}
\bsIF{condition1 \text{ and } condition2 }
    \bsCODE{\hdots}
\bsEND
\end{pseudo*}

\paragraph*{Majority} Given a set $S=s_0, \hdots, s_{n-1}$ of boolean values, $\majority(S)$ 
returns $\mean(S)$ where the elements of $S$ are intepreted as $0$ if $\false$ and 1 if $\true$.
So if the number of $\true$ values is greater than or equal to the number of $\false$ values, $\majority(S)$ returns
$\true$, otherwise it returns $\false$.
