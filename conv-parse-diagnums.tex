The bytestream can be considered as a process, in which various
groupings or structures of data are passed from encoder to decoder.
Within the bytestream, there is a natural association of data in
structures. When the parse diagram is written out in full, it has the
form of a tree and branch map.

In this text, parse diagrams are numbered in the form X.Y.

X is the level of the diagram, with the highest level of X being 0.

Y is the relevant branch, with all levels in the same branch having the
same value of Y. The same value of Y can only be found in the same
branch. Sub-branches lead to a new branch number.
