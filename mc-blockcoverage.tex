The predicted picture is divided into a regular overlapping blocks that
cover at least the entire picture area as shown in
figure~\ref{fig:blockcoverage}.  Each block (figure~\ref{fig:}) has a
width and height, $\xblen$ and $\yblen$ respectively.  A value of
$\xbsep$ and
$\ybsep$ repsectively define the horizontal and vertical distances between
blocks.

\begin{figure}[h]
\centering
\includegraphics[width=0.7\textwidth]{figs/block-coverage}
\caption{Block coverage of the predicted picture}
\label{fig:blockcoverage}
\end{figure}

Blocks will overspill the left and top edges by ($\xblen - \xbsep$) and
($\yblen - \ybsep$) pixels.  The overspill on the right hand and bottom edges
will be at least the amount on the left and top edges; in order to
maintain the overlapping property of Overlapped-block motion
compensation and satisfy bytestream requirements, the bottom and right
edge overspill may be multiple blocks.  Any predictions for pixels
outside the picture $(0 \leq x < \picWidth, 0 \leq y < \picHeight)$
are discarded.
