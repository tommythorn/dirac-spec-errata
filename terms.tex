This section defines the acronyms, terms and naming conventions used in the Dirac specification.

\subsection{Acronyms}

\begin{description}
\item[4CIF:] Four times CIF
\item[4CIF:] Four times SIF
\item[CIE:] Commission internationale de l'�clairage (International Commission on Illumination)
\item[CIF:] Common Image Format a.k.a. SIF 625
\item[DC:] Direct Current
\item[DCI:] Digital Cinema Initiatives
\item[DWT:] Discrete Wavelet Transform
\item[HD(TV):] High Definition (Television)
\item[HH:] High-High (subband)
\item[HL:] High-Low (subband)
\item[IDWT:] Inverse Discrete Wavelet Transform
\item[ITU:] International Telecommunications Union
\item[LH:] Low-High (subband)
\item[LL:] Low-Low (subband)
\item[NTSC:] National Television Systems Committee
\item[QCIF:] Quarter CIF
\item[QSIF:] Quarter SIF
\item[SD(TV):] Standard Definition (Television)
\item[SIF:] Source Input Format
\item[VC:] Video Codec
\item[VLC:] Variable Length Code
\end{description}

\subsection{Terms}
\begin{description}
\item[AC (sub)Band:] any signal band that is not the DC sub-band.
\item[Arithmetic coding:] a form of entropy coding used by Dirac, which is used in addition to exp-Golomb coding.
\item[Chroma:] a pair of color difference components. The term chroma is the direct equivalent to �luma� (see �luma� definition below). In this standard, the term chroma is not the same as that used in composite color television. In this standard the term chroma is used to cover both gamma-corrected and non-gamma-corrected signals.
\item[Codeblock:] a rectangular array of wavelet coefficients within a component subband.
\item[Codec:] a truncation of the terms "coder" and "decoder".
\item[DC subband:] the signal band that represents data composed from the lowest frequency band of a wavelet transform (0-LL).
\item[Discrete Wavelet Transform (DWT):] a means of transforming an array of values into space-frequency components through the use of a filter bank. 
\item[Entropy Coding:] a term for describing any mathematical process used to encode data in a lossless manner, intended to reduce the required bit rate.
\item[Exp-Golomb:] a form of variable-length code. This specification uses an interleaved variant.
\item[Intra DC Prediction:] the prediction of coefficients within the dc subband of intra pictures from neighbouring coefficients..
\item[Inverse Discrete Wavelet Transform (IDWT):] the inverse of the DWT that converts an array of space- frequency components back into an array of values.
\item[Inverse Quantisation:] a process whereby each sample of a sub-band has its signal range expanded by a defined value.
\item[Lifting:] the name given to reducing a DWT filtering operations into a number of elementary
filters, each operating on half the samples.
	(Note: see Bibliography item "Ripples in Mathematics", chapter 3, for more information. )
\item[Low Delay:] a term used to define the mode of the Dirac codec that can be used to compress video with a delay of less than one frame duration.
\item[Luma:] the weighted sum of RGB components of color video, which may or may not be gamma-corrected. (This term is used to prevent confusion with the term �luminance� that is created only from linear light levels as used in color science.)
\item[Parse Info header:] identifies the beginning of major Dirac syntax elements (sequence start, picture, sequence end, padding and auxiliary data) with defined parse code values.
\item[Parsing:] a process by which numerical and text strings within binary data are recognised and used to provide syntactic meaning.
\item[Picture:] a single frame or field of video.
\item[Quantisation:] a process whereby each sample of a sub-band has its signal range compressed by a defined value. 
\item[Quantiser:] The defined value used for the purposes of quantisation or inverse quantisation.
\item[Raster scan:] any 2-dimensional array of samples, whether as video samples or as wavelet transformed values, that is scanned in accordance with television systems; namely left to right, then top to bottom.
\item[Sequence:] the data contained in a Dirac sequence corresponds to a single video sequence with constant video parameters as defined in Section \ref{}. A Dirac sequence is preceded by a `Parse Info' header that indicates the beginning of the sequence with a unique parse code. A Dirac sequence can be extracted from a Dirac bit-stream and decoded entirely independently.
\item[Slice:] a component part of the low-delay syntax that provides for compression of small parts (slices) of a picture in order to reduce delay. 
\item[State:] the set of current decoder variable values.
\item[Stream:] a concatenation of Dirac sequences.
\item[Subband:] the signal band that represents data composed from a single space-frequency band of a wavelet transform.
\end{description}
