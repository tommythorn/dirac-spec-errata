

If a value lies in a range $0 \leq x \leq n$, then truncated unary
binarisation can be used: that is, for the last value the final 1 can be
omitted, since its presence is inevitable.

\begin{figure}[h]
\begin{tabular}{c|c}
Value & Binarisation \\
\hline\\
0     & 1 \\
1     & 01 \\
\dots & \dots \\
$n-1$ & $\underbrace{0\dots0}_{n-1} 1$ \\
$n$   & $\underbrace{0\dots0}_{n}$
\end{tabular}

\caption{Conversion from unsigned truncated unary to binary}
\end{figure}

