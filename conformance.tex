
Normative text is text that describes elements of 
the design that are indispensable or contains the 
conformance language keywords: `shall', `should', or `may'. 
Informative text is text that is potentially helpful to the user, 
but not indispensable, and can be removed, changed, or added 
editorially without affecting interoperability. Informative 
text does not contain any conformance keywords.

All text in this document is, by default, normative, 
except: the Introduction, any section explicitly labelled as `Informative' 
or individual paragraphs that are also indicated in this way.

The keywords `shall' and `shall not' indicate requirements 
strictly to be followed in order to conform to the document 
and from which no deviation is permitted

The keywords, `should' and `should not' indicate that, among 
several possibilities, one is recommended as particularly suitable, 
without mentioning or excluding others; or that a certain course 
of action is preferred but not necessarily required; or that 
(in the negative form) a certain possibility or course of action is deprecated but not prohibited.

The keywords `may' and `need not' indicate courses of action 
permissible within the limits of the document.

The keyword `reserved' indicates a provision that is not 
defined at this time, shall not be used, and may be defined 
in the future. The keyword `forbidden' indicates `reserved' and in
 addition indicates that the provision will never be defined in the future.

A conformant implementation according to this document is one that includes 
all mandatory provisions (`shall') and, if implemented, all recommended 
provisions (`should') as described. A conformant implementation need 
not implement optional provisions (`may') and need not implement them as described.

\subsection{Normative References}
Normative references are external documents referenced in normative 
text that are indispensable to the user. Bibliographic references 
are references made in informative text or are those otherwise not
 indispensable to the user.

The following standards contain provisions which, through
 reference in this text, constitute provisions of this standard. 
 At the time of publication, the editions indicated were valid. 
 All standards are subject to revision, and parties to agreements
  based on this standard are encouraged to investigate the 
  possibility of applying the most recent edition of the standards indicated below.

\begin{enumerate}  
\item	Proposed ISO/IEC MPEG and ITU-T VCEG [JVT] - YCoCg: A Color Space with RGB Reversibility and Low Dynamic Range.
\item	Recommendation ITU-R BT.709-5: Parameter values for the HDTV standards for production and international programme exchange, 2002. 
\item	ITU-BT.1361: Worldwide unified colorimetry and related characteristics of future television and imaging systems.
\item	ITU-BT.1700: Characteristics of composite video signals for conventional analogue television systems.
\item	SMPTE 170M-1994, for Television: Composite Analog Video Signal - NTSC for Studio Applications.
\item	EBU Tech 3213-1994: Standard for Chromaticity Tolerances for Studio Monitors.
\item	CIE Publication15:2004: Colorimetry.
\item	SMPTE 428.1: Digital Cinema Distribution Master - (DCDM) Image Characteristics.
\end{enumerate}