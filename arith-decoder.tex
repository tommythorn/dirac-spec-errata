The arithmetic decoding engine is a multi-context, adaptive binary
arithmetic decoder, performing binary renormalisation and producing
binary outputs. For each bit decoded, the semantics of the relevant
calling decoder function determine contexts to be passed to the
arithmetic decoding operation (Section \ref{arithcontexts}).

\subsubsection{Shifting bits in}

\label{arithshiftin}

This section defines the operation of the $shift\_bit\_in()$ 
and $shift\_all_bits()$ functions
for reading bits into the arithmetic decoding state variables.

\begin{pseudo}{shift\_bit\_in}{}
\bsCODE{\AHigh \ll= 1}
\bsCODE{\AHigh \&= \text{0xFFFF}}
\bsCODE{\AHigh += 1}
\bsCODE{\ALow \ll= 1}
\bsCODE{\ALow \&= \text{0xFFFF}}
\bsCODE{\ACode \ll= 1}
\bsCODE{\ACode \&= \text{0xFFFF}}
\bsCODE{\ACode += read\_bita()}{\ref{inputarith}}
\end{pseudo}

$shift\_all\_bits()$ expands the interval between \ALow and \AHigh
until the msbs (bit 15) differ and the interval no longer
straddles the half-way point $0x8000$.

\begin{pseudo}{shift\_all\_bits()}{}
\bsWHILE{ \AHigh\&0x8000)==0x0 \&\& (\ALow\&0x8000)==0x0}
  \bsCODE{shift\_bit\_in()}
\bsEND
\bsWHILE{ (\AHigh\&0x4000)==0x0 and (\ALow\&0x4000)==0x4000 }
  \bsCODE{\ACode \hat{\text{ }}= 0x4000}
  \bsCODE{\AHigh \hat{\text{ }}= 0x4000}
  \bsCODE{\ALow \hat{\text{ }}= 0x4000}
  \bsCODE{shift\_bit\_in()}
\bsEND
\end{pseudo}

\subsubsection{Decoding boolean values}

\label{arithreadbool}

This section specifies the operation of the $read\_boola()$ function
for extracting a boolean value from the Dirac stream. Before extracting
any values, all possible bits are shifted in to ensure that the decoding
state has maximum information.

\begin{pseudo}{read\_boola}{context\_index}
\bsCODE{shift\_all\_bits()}{\ref{arithshiftin}}
\bsCODE{context=\AContexts[context\_index]}
\bsCODE{weight = context[0] + context[1]}
\bsCODE{scaler = (0x10000+weight//2)//weight}
\bsCODE{probability0 = context[0]*scaler}
\bsCODE{count = code-low+1}
\bsCODE{range = high-low+1}
\bsCODE{range\_x\_prob = (range * probability0)>>16}
\bsIF{ count > range\_x\_prob }
  \bsCODE{value = \true}
  \bsCODE{low = low + range\_x\_prob}
  \bsCODE{context[1] += 1}
\bsELSE
  \bsCODE{value = False}
  \bsCODE{high = low + range\_x\_prob - 1}
  \bsCODE{context[0] += 1}
\bsEND
\bsIF{ (context[0] + context[1]) > 255 }
  \bsCODE{reset\_context(context)}
\bsEND
\bsRET{value}
\end{pseudo}

\begin{informative}
The function scales the probability of $0$ from the decoding context
so that a probability of $1$ is commensurate with the interval between
 \ALow and \AHigh. If \ACode is greater than this cut-off, then $1$ has
been encoded, else $0$ has.

\end{informative}
