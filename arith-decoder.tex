
%src: 0.9

The arithmetic decoding engine is a multi-context, adaptive binary
arithmetic decoder, performing binary renormalisation and producing
binary outputs. For each bit decoded, the semantics of the relevant
decoder function determine a statistical context (probability model) to
be used for the internal rescaling functions. Assuming this context, the
raw arithmetic decoding function is written

\begin{comment}
binary\_arith\_decode()

Its operation is defined in Section .

Signed and unsigned integer outputs can be derived by the use of
binarisation schemes which turn a symbol into a string of bits. There
are two basic binarisations employed for arithmetic coding: unary and
truncated unary binarisation. These are defined in Appendix , giving
rise to four decoder functions 

Unsigned unary arithmetic decoding: uu\_arith\_decode()

Signed unary arithmetic decoding:   su\_arith\_decode()

Unsigned truncated unary arithmetic decoding:   ut\_arith\_decode()

In addition a decoding process may invoke halve\_all\_counts() to reset
context statistics.

\begin{informative}
There is no signed truncated unary arithmetic decoding function in
Dirac, since truncated unary values have been derived from modulo
arithmetic and have no sign.
\end{informative}
\end{comment}
