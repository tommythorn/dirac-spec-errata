This section defines the video formats supported by the Dirac codec.

A selection of widely used video formats are defined in normative Annex 
\ref{videoformatdefaults}. These video formats are characterized by
their widespread use in television, cinema and multimedia applications.

This list is not exhaustive, however, and Dirac is a general purpose video 
codec. These predefined formats are base formats that may be modified element by
element to support a much larger range of possible video formats. Support
is provided by the sequence parameters of the bitstream (Section 
\ref{sequenceheader}) for signalling both the base video format and
any modifications for complete characterization of the video format metadata.


\subsection{Colour model}

Dirac supports any video format that codes the raw image colors in a luma 
(grey-level) component with two associated chroma (color difference) components.
 These components are referred to as $Y$, $C1$ and $C2$.

In ITU defined systems (including ITU-BT.709, ITU-R BT.1361 and ITU-BT.1700), 
the $Y$, $C1$ and $C2$ values shall relate to the $E’_Y$, $E’_U$ and $E’_V$ 
video components respectively. These video components are also widely referred 
to as $Y, U, V$ and $Y, C_B , C_R$.

In the ITU-T H.264 reversible color transform, the $Y$, $C1$ and $C2$ values 
shall correspond to the video components $Y, C_O, C_G$.

\begin{informative}
Coding using $Y, C_O, C_G$ provides a simple reversible conversion to and from
RGB components by using lossless integer transforms. The use of $Y, C_O, C_G$
supports lossless coding of RGB video and allows Dirac to be treated as an RGB
codec for applications that require this feature.
\end{informative}

\subsection{Interlace}

Dirac supports both interlace and progressive formats. Interlace formats may be
 either top field first or bottom field first.

Dirac codes pictures where a picture may be a frame or a field. Fields consist 
of sets of alternate lines of video frames (even and odd lines). A pair of 
fields constituting a frame may correspond to distinct time intervals (true
interlace scanning) or to the same time interval (progressive segmented frames).
 Hence the configuration of frame/field coding is independent of whether the 
video format is interlaced or progressive.

\subsection{Component sampling}

Chroma components $C1$ and $C2$ may be coded with the same dimensions as the Y
 component (4:4:4) sampling, or with half-width (4:2:2) or half-dimension 
(4:2:0) sampling.

$Y$, $C1$ and $C2$ picture components shall be sampled at the same temporal
instant.

\begin{informative}
All pictures are considered as individual entities whether or not all lines were
 sampled at the same instant. In video sequences that are not frame-based, such 
as 30fps interlaced video carrying 24fps progressive images in a 3:2 
pull-down sequence, the compression performance may not be optimum. In such 
cases, a pre-processor may provide the codec with a more easily compressed 
source such as the original 24fps source pictures. Such pre-processing does not form any part of this specification.
\end{informative}

\subsection{Bit resolution}

The bit depth of each component sample is, in principle, unrestricted. 
Application-specific codecs may restrict the supported bit depth to a single 
value or a limited range of values.

Video is represented internally within the decoder specification as a bipolar
 (signed integer) signal. Video is presented at the video interface as an 
unsigned integer value by addition of an offset to these values 
(Section \ref{videooutput}). Metadata concerning black level and white level 
is transmitted
 within the data stream (Section \ref{signalrange}), but is not enforced at the
decoder video interface: output video may undershoot or overshoot these values.

\subsection{Picture frame size and rate}

The frame size and frame rate is, in principle, unrestricted. 
Application-specific codecs may restrict the supported frame size and frame rate
 to a single value or a limited range of values, and compliance to a given level
 implies constraints on the values as specified in Annex \ref{profilelevel}.

