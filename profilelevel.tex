\label{profilelevel}

A compliant Dirac decoder may support a number of different profiles and levels, which determine 
which tools, syntax elements and structures are to be supported, and what decoder resources 
(computational and memory) are required. Profiles deal largely with the former and levels
with the latter, although in particular in a software implementation, the choice of supported tools
also affects decoder resources. Level constraints are expressed in terms of picture sizes and 
formats 

A particular profile will require that particular syntax/syntax elements be used and that decoder
variables or functions are set to particular values. Profiles also determine the size of the
decoded picture buffer $\DecodedBuffer$ and the stream buffer for data entering the decoder. [or is this levels?]
The operation of these buffers is described in Sections \ref{decodedbufferop} and \ref{streambufferop}.

\begin{comment}
Main=0, $is\_low\_delay()==false$, $using\_ac()==\true$, $using\_mc()==\true$
Simple=1, $is\_low\_delay()==false$, $using\_ac()==\false$, $using\_mc()==\false$
Link=2, $is\_low\_delay()==true$, $using\_ac()==\false$, $using\_mc()==\false$
Wireless=3, $is\_low\_delay()==true$, $using\_ac()==\true$, $using\_mc()==\true$

If we have Simple or Link then $using\_ac()$ is false and $using\_inter()$ is false.
\end{comment}

\subsection{Decoded picture buffer model operation}
\label{decodedbufferop}

In order to define compliance with level and profile requirements, a model of decoded picture buffer
operation is specified.

The decoded picture buffer (DPB) $\DecodedBuffer$ exists to allow for pictures to be presented in display, or 
picture number, order whilst pictures are coded out of order. No constraints are placed on the order
of pictures within a Dirac stream provided that reordering can be carried out within the given size of
the DPB. The DPB size is the minimum delay required by the decoder, ignoring decoding time and stream
buffering delays (Section \ref{streambufferop}).

Pictures are deemed to be output from the decoder and placed in the DPB instantaneously at multiples
of the (real number) picture sample interval $\PictureInterval$. This interval is defined as follows:

\begin{itemize}
\item if $\Interlaced$ is $\true$ and $\SequentialFields$ is $\true$ (i.e. pictures are fields) then 
\begin{equation*}
\PictureInterval=\dfrac{\FrameRateDenominator}{2*\FrameRateNumerator}
\end{equation*}
\item otherwise (i.e. pictures are frames) then 
\begin{equation*}
\PictureInterval=\dfrac{\FrameRateDenominator}{\FrameRateNumerator}
\end{equation*}
\end{itemize}

In modelling the display of a sequence, pictures are deemed not to be output from the DPB until after $\DPBSize$ pictures
have been placed in the buffer i.e. the first picture is output at the time instant with index $\DPBSize$, counting from 0. 
A picture is deemed to be output instantaneously and simultaneously with a picture being input to the buffer.

Hence if $\DPBSize$ is 0, a picture passes through the DPB with zero delay, and is produced for display at the same
time instant. 

Determining the picture to output at a given time is simple in practice but delicate to specify, since picture numbers
may wrap around through 0. Define the absolute difference between two pictures with picture numbers $n_1, n_2$
as

\[d(n_1,n_2)=\min\left((n_1-n_2) \% 2^{32}, (n_2-n_1) \% 2^{32} \right)\] 

An encoder must ensure that the absolute difference between the picture numbers of two pictures lying in the DPB is 
always positive and strictly less than $2^{31}$. This determines an unambiguous ordering
on pictures in the buffer: a picture with picture number $n_1$ is then earlier than a picture with a different picture number 
$n_2$ if and only if $n_2-n_1=d(n_1,n_2)>0$. The picture output at a given time instant is deemed to be the picture 
with the earliest picture number according to this metric.

An encoder is required to so order its pictures that the DPB shall produce pictures whose picture numbers comprise a
contiguous series, modulo $2^{32}$, when output according to this rule.

\begin{informative}
Note that, if the DPB is sufficiently large to allow it, a Dirac sequence may contain frames out of order
even if it is not strictly necessary: for example, a sequence containing only intra frames could be 
delivered out of order in Main profile, but not in Simple profile which has a zero-sized buffer.
\end{informative}

\subsection{Stream buffer model operation}
\label{streambufferop}

In order to define compliance with level and profile requirements, a model of decoded picture buffer
operation is specified.

The decoder stream buffer (DSB) $\StreamBuffer$ exists to allow for data to be transmitted at a constant bit
rate (CBR), yet consist of varying bit allocations to individual pictures. It is well-known that input buffers
operate in mirror-image to output buffers, and so the specification of the ISB implicitly defines
how an identically-sized encoder stream buffer (ESB) would operate to ensure correct CBR operation, although
in practice some encoder-side headroom may be required.

Buffer operation is complicated by random access, which implies that the state of the DSB is not
wholly known at the encoder. At best, an upper bound on the decoder buffer occupancy is  
 

\subsection{Supported profiles}