\label{profilelevel}

A Dirac decoder shall support one or more different profiles and levels. 
Profiles and levels determine which tools, syntax elements and structures 
shall be supported, and what decoder resources (computational and memory) are required.

\subsection{Profiles}
\label{profiles}
A given profile requires that particular syntax/syntax elements shall 
be used and that decoder variables or functions shall be set to particular values.

Dirac defines four profiles, Main (Long GOP), Main (Intra), Simple and Low Delay, 
corresponding to different picture types. The Main (Intra), Simple and Low Delay
profiles shall correspond to the Main, Simple and Low Delay profiles of VC-2. 

The profiles shall satisfy the following conditions:
\begin{itemize}
\item A Low Delay profile Dirac sequence shall set $\Profile$ equal to a value of 
$0$ in the parse parameters (section \ref{parseparameters}) for each sequence header 
in the sequence.
\item A Simple profile sequence shall set $\Profile$ equal to a value of $1$ 
in the parse parameters for each sequence header in the sequence.
\item A Main (Intra) profile sequence shall set $\Profile$ equal to a value of $2$ 
in the parse parameters for each sequence header in the sequence.
\item A Main (Long GOP) profile sequence shall set $\Profile$ equal to a value of $8$ 
in the parse parameters for each sequence header in the sequence.
\end{itemize}

Further VC-2 compatible profiles may be added in future revisions of this 
specification with profile number less than 8; other profiles may be added with
profile number greater than 8.

A Dirac sequence shall comply with one of the supported profiles.

\subsubsection{Low Delay profile}

A Low Delay profile sequence shall contain only those data units whose 
parse codes are listed in Table \ref{table:ldprofile}.

\begin{table}[!ht]
\centering
\begin{tabular}{|c|c|l|c|}
\hline
\rowcolor[gray]{0.75}
\ParseCode &  {\bf Bits} & {\bf Description} & \begin{tabular}{c} {\bf Number of}\\ {\bf Reference}\\{\bf Pictures}\end{tabular}\\
\hline
0x00 & 0000 0000 & Sequence header &--\\
\hline
0x10 & 0001 0000 & End of Sequence & -- \\
\hline
0x20 & 0010 0000 & Auxiliary data & -- \\
\hline
0x30 & 0011 0000 & Padding data & -- \\
\hline
0xC8 & 1100 1000 & Intra Non Reference Picture & 0\\
\hline
\end{tabular}
\caption{Parse code values for Low Delay profile sequences}\label{table:ldprofile}
\end{table}

\subsubsection{Simple profile}

A Simple profile sequence shall contain only those data units whose 
parse codes are listed in Table \ref{table:simpleprofile}.

\begin{table}[!ht]
\centering
\begin{tabular}{|c|c|l|c|}
\hline
\rowcolor[gray]{0.75}
\ParseCode &  {\bf Bits} & {\bf Description} & \begin{tabular}{c} {\bf Number of}\\ {\bf Reference}\\{\bf Pictures}\end{tabular}\\
\hline
0x00 & 0000 0000 & Sequence header &--\\
\hline
0x10 & 0001 0000 & End of Sequence & -- \\
\hline
0x20 & 0010 0000 & Auxiliary data & -- \\
\hline
0x30 & 0011 0000 & Padding data & -- \\
\hline
0x48 & 0100 1000 & Intra Non Reference Picture (no arithmetic coding) & 0\\
\hline
\end{tabular}
\caption{Parse code values for Simple profile sequences}\label{table:simpleprofile}
\end{table}

\subsubsection{Main (Intra) profile}

A Main (Intra) profile sequence shall contain only those data units whose 
parse codes are listed in Table \ref{table:mainintraprofile}.

\begin{table}[!ht]
\centering
\begin{tabular}{|c|c|l|c|}
\hline
\rowcolor[gray]{0.75}
\ParseCode &  {\bf Bits} & {\bf Description} & \begin{tabular}{c} {\bf Number of}\\ {\bf Reference}\\{\bf Pictures}\end{tabular}\\
\hline
0x00 & 0000 0000 & Sequence header &--\\
\hline
0x10 & 0001 0000 & End of Sequence & -- \\
\hline
0x20 & 0010 0000 & Auxiliary data & -- \\
\hline
0x30 & 0011 0000 & Padding data & -- \\
\hline
0x08 & 0000 1000 & Intra Non Reference Picture (arithmetic coding) & 0\\
\hline
\end{tabular}
\caption{Parse code values for Main (Intra)  profile sequences}\label{table:mainintraprofile}
\end{table}

\subsubsection{Main (Long GOP) profile [TBD]}

A Main (Long GOP) profile sequence shall contain only those data units whose 
parse codes are listed in Table \ref{table:mainlonggopprofile}.

\begin{table}[!ht]
\centering
\begin{tabular}{|c|c|l|c|}
\hline
\rowcolor[gray]{0.75}
\ParseCode &  {\bf Bits} & {\bf Description} & \begin{tabular}{c} {\bf Number of}\\ {\bf Reference}\\{\bf Pictures}\end{tabular}\\
\hline
0x00 & 0000 0000 & Sequence header &--\\
\hline
0x10 & 0001 0000 & End of Sequence & -- \\
\hline
0x20 & 0010 0000 & Auxiliary data & -- \\
\hline
0x30 & 0011 0000 & Padding data & -- \\
\hline
0x0C & 0000 1100 & Intra Reference Picture (arithmetic coding) & 0\\
\hline
0x08 & 0000 1000 & Intra Non Reference Picture (arithmetic coding) & 0\\
\hline
0x0D & 0000 1101 & Inter Reference Picture (arithmetic coding) & 1\\
\hline
0x0E & 0000 1110 & Inter Reference Picture (arithmetic coding) & 2\\
\hline
0x09 & 0000 1001 & Inter Non Reference Picture (arithmetic coding)& 1\\
\hline
0x0A & 0000 1010 & Inter Non Reference Picture (arithmetic coding) & 2\\
\hline
\end{tabular}
\caption{Parse code values for Main (Long GOP)  profile sequences}
\label{table:mainlonggopprofile}
\end{table}

Low delay syntax pictures shall not be present in a Main (Long GOP) profile
sequence.

\subsection{Levels}
\label{level}

A given value of level shall define constraints on the decoder 
resources required to decode a compliant sequence. The values that are 
constrained are defined individually for each level.

A particular application domain may impose additional constraints 
on a decoder, for example the presence or absence of a suitable video 
interface. Such additional constraints are not covered by this specification.

This specification defines levels 1 and 128. Other levels are application 
specific and will be defined in future revisions of this specification. VC-2
compatible levels will have level number less than 128 and other levels
will have level number greater than 128.

The value 0 shall be RESERVED and shall not be used for any defined level. 
Sequences may use the value 0 for streams that do not conform to any defined level. 

For level 1, the video parameters shall correspond 
to one of the base video formats as defined in annex \ref{videoformatdefaults} 
and the video parameters
of the base video format shall not be overridden. Level 1 shall be available
for Low Delay, Simple and Main (Intra) (VC-2 compatible) profiles only.

For level 128, the video parameters shall correspond to one of the base video
formats as defined in annex \ref{videoformatdefaults}, 
except that the signal range is restricted
to 8 bit ranges and the chroma sampling format and frame dimensions may be
overridden by suitable values. Level 2 shall be available for Main (Long GOP)
profile only.

A sequence compliant with a given level $n$ shall set $\Level$ equal to $n$.

\subsubsection{Decoder data buffers[DRAFT-TBC]}

A reference picture buffer $\RefBuffer$ is defined for storing 
decoded reference pictures (if any). In addition, levels may define
two additional buffers, and applicable parameters for them:

\begin{itemize}
\item a bit stream buffer $\StreamBuffer$ for buffering 
stream data prior to decoding
\item a decoded picture buffer $\DecodedBuffer$ for storing 
decoded pictures (reference or non-reference) for the purposes
of picture reordering
\end{itemize}

For the purposes of this specification, the reference picture 
buffer shall be deemed to be additional to the
decoded picture buffer (i.e. reference picture storage will be
duplicated in the decoded picture buffer).

\paragraph{Bit stream buffer operation[TBC]}$\ $\newline

\paragraph{Picture reordering and decoded picture buffer[TBC]}$\ $\newline
\label{picturereordering}

Two parameters shall be defined for constraining picture reordering.
The first is the size, $\DPBSize$, of the decoded picture buffer.

The second is the reordering depth applicable to a Dirac sequence. This shall
be defined as the maximum number of picture data units that may occur 
between a picture with picture number $N$ and a picture with picture number
$N+1$ or $N-1$.

\subsubsection{Buffer models [TBC]}


\subsubsection{Level 1: VC-2 default level}

This level is intended to provide minimal constraints on VC-2 compatible
streams encoding one of the base video formats 
(annex \ref{videoformatdefaults}). Data rates in 
this level are not bounded. This level is not intended to provide
guarantees of real time decoding.

This level shall only be available if the profile is set to Low Delay, Simple, or
Main (Intra).

\paragraph{Sequence header parameters}$\ $\newline

The following constraints shall apply to the sequence header parameters
(section \ref{sequenceheader}):
\begin{itemize}
\item$base\_video\_format$ shall be between 1 and 20 inclusive
\item $custom\_dimensions\_flag$ shall be $\false$
\item $custom\_chroma\_format\_flag$ shall be $\false$
\item$custom\_scan\_format\_flag$ shall be $\false$
\item $custom\_frame\_rate\_flag$ shall be $\false$
\item $custom\_pixel\_aspect\_ratio\_flag$ be $\false$
\item $custom\_clean\_area\_flag$ shall be $\false$
\item $custom\_signal\_range\_flag$ shall be $\false$
\item $custom\_color\_spec\_flag$ shall be $\false$
\item $picture\_coding\_mode$ shall be as per section \ref{picturecodingmode}
\end{itemize}

\paragraph{Picture header parameters}$\ $\newline

The following constraints shall apply to picture header parameters
(section \ref{pictureheader}):
\begin{itemize}
\item $\WaveletIndex$ shall be as table \ref{wltfilterpresets}
\item $\TransformDepth$ shall be between 0 and 4 inclusive
\item for core syntax pictures:
    \begin{itemize}
    \item $\CodeblocksX$ and $\CodeblocksY$ shall be such that there is 
    at least one coefficient in each codeblock.
    \end{itemize}
\item for low delay syntax pictures:
    \begin{itemize}
    \item $\SlicesX$ and $\SlicesY$ shall be such that there is at least one 
    DC (0-LL) coefficient per slice.
    \item values in the quantizer matrix shall lie between 0 and 127 inclusive
    \end{itemize}
\end{itemize}

\paragraph{Transform data}$\ $\newline

The following constraints shall apply to transform data elements 
(section \ref{wavelettransform}):
\begin{itemize}
\item quantized and inverse quantized wavelet coefficients shall lie 
between $-2^{19}$ and $2^{19}$ inclusive
\item for core syntax pictures:
    \begin{itemize}
    \item quantization indices for subbands and codeblocks shall lie between 
    0 and 68 inclusive 
    \item quantization offsets encoded in codeblocks shall lie between 
    -68 and 68 inclusive
    \end{itemize} 
\end{itemize} 

\subsubsection{Level 128: Long-GOP default level [DRAFT-TBD]}

This level is intended to provide minimal constraints on long GOP
streams encoding simple variants of the base video formats 
(annex \ref{videoformatdefaults}). 
Data rates in this level are not bounded. This level is not intended to provide
guarantees of real time decoding.

This level shall only be available if the profile is set to Main (Long GOP).

\paragraph{Sequence header parameters}$\ $\newline

The following constraints shall apply to the sequence header parameters
(section \ref{sequenceheader}):
\begin{itemize}
\item$base\_video\_format$ shall be between 1 and 20 inclusive
\item $custom\_dimensions\_flag$ may be $\false$ or $\true$; if $\true$ then
the frame width and frame height shall be less than the values set in the
base video format
\item $custom\_chroma\_format\_flag$ may be $\false$ or $\true$
\item$custom\_scan\_format\_flag$ may be $\false$ or $\true$
\item $custom\_frame\_rate\_flag$ shall be $\false$
\item $custom\_pixel\_aspect\_ratio\_flag$ be $\false$
\item $custom\_clean\_area\_flag$ shall be $\false$
\item $custom\_signal\_range\_flag$ shall be $\true$ and the signal range 
parameters set to 8 bit SDI ranges (index 2 in table \ref{table:signalrangevalues})
\item $custom\_color\_spec\_flag$ shall be $\false$
\item $picture\_coding\_mode$ shall be as per section \ref{picturecodingmode}
\end{itemize}

\paragraph{Picture header parameters}$\ $\newline

The following constraints shall apply to picture header parameters
(section \ref{pictureheader}):
\begin{itemize}
\item $\WaveletIndex$ shall be between 0 and 4 inclusive
\item $\TransformDepth$ shall be between 0 and 4 inclusive
\item $\CodeblocksX$ and $\CodeblocksY$ shall be such that there is 
    at least one coefficient in each codeblock
\end{itemize}

\paragraph{Transform data}$\ $\newline

The following constraints shall apply to transform data elements 
(section \ref{wavelettransform}):
\begin{itemize}
\item quantized and inverse quantized wavelet coefficients shall lie 
between $-2^{15}$ and $2^{15}$ inclusive
\item quantization indices for subbands and codeblocks shall lie between 
    0 and 63 inclusive 
\item quantization offsets encoded in codeblocks shall lie between 
    -63 and 63 inclusive
\end{itemize} 

\paragraph{Reordering and reference buffers [DRAFT-TBD]}$\ $\newline


A decoder must maintain a reference picture buffer for the 
purposes of inter picture prediction, and (separately) a decoded 
picture buffer for the purposes of picture reordering, in addition to 
storage provided for decoding the current picture. The decoded picture
buffer shall contain non-reference pictures only. 

A bit stream buffer is not required for this level.

The following constraints shall apply:
\begin{itemize}
\item The reference buffer size $\MaxRefBufferSize$ shall be 3.
\item The decoded picture buffer size shall be 2 in the case
of frame coding or 4 in the case of field coding.
\item The picture reordering depth shall be 5 in the case of frame coding
or 11 in the case of field coding
\end{itemize}














