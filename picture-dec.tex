%%%%%%%%%%%%%%%%%%%%%%%%%%%%%%%%%%%%%%%%%%%%%%%%%
% - This chapter defines the overall process  - %
% - for decoding a picture                    - % 
%%%%%%%%%%%%%%%%%%%%%%%%%%%%%%%%%%%%%%%%%%%%%%%%%

\label{picturedec}

This section specifies the process for decoding pictures from the Dirac stream. Picture decoding depends upon
correctly parsing the Dirac bitstream, and decoding operations are dependent upon the parsing operations
set out in Sections \ref{streamstructure}, \ref{motiondec} and \ref{transformdec}.

This section does not specify how pictures are encoded, nor how pictures are reordered and presented for 
display, which is described in Section \ref{profilelevel}. 

\subsection{Introduction}

Dirac supports both intra and inter picture coding, with forward and backward prediction. This means that
pictures may be encoded in the stream in non-display order: reordering pictures will be required in order
to display them correctly, and  decoded picture buffer will be necessary to store pictures while temporally 
prior pictures are decoded. Note that the core Dirac specification does not encompass the operation of the
decoded picture buffer: this is specified in conjunction with the level and profile values extracted from
the stream (Section \ref{parseparameters}), in Appendix \ref{profilelevel}. 

Decoded pictures may, however, be reference pictures, used for the prediction of subsequent pictures
in the Dirac stream. Reference pictures are stored in a reference picture buffer $\RefBuffer$. The operation
of $\RefBuffer$ does form part of the core Dirac specification, and the rules for management of the
buffer are set out in Section \ref{refbuffer}.

\subsection{Decoding sequences}
\label{sequencedecoding}

The process for decoding a picture sequence (in coded order) is as follows:

\begin{pseudo}{decode\_sequence}{}
   \bsCODE{\SeqStateName = \{\}}
   \bsCODE{video\_params = \{\}}
   \bsCODE{decoded\_pictures = \{\}}
   \bsCODE{\RefBuffer=\{\}}
   \bsCODE{\RetiredPictureList=\{\}}
   \bsCODE{parse\_info()}{\ref{parseinfo}}
   \bsWHILE{is\_end\_of\_sequence() == \false}
      \bsIF{is\_AU()==\true}
         \bsCODE{video\_params = access\_unit\_header()}{\ref{auheader}}
      \bsELSEIF{is\_picture()==\true}
         \bsCODE{picture\_parse()}{\ref{pictureparse}}
         \bsCODE{decoded\_pictures[\PictureNumber] = picture\_decode()}{\ref{picturedecprocess}}
         \bsCODE{ref\_buffer\_remove()}{\ref{refbuffer}}
         \bsIF{is\_ref()}
           \bsCODE{ref\_buffer\_add()}{\ref{refbuffer}}
         \bsEND
         \bsCODE{offset\_output\_data(decoded\_pictures[\PictureNumber])}{\ref{videooutput}}
      \bsEND
      \bsCODE{parse\_info()}{\ref{parseinfo}}
   \bsEND
   \bsRET{\{video\_params, decoded\_pictures\}}
\end{pseudo}

The process returns the video parameters, consisting of the essential metadata required for 
display and interpretation of the video data, and the array of decoded pictures. Each decoded
picture contains the three video component data arrays together with a picture number.

The pseudocode describes the decoding process. Decoding starts by clearing the decoder state 
and the decoder output. Thus video sequences may be decoded as independent entities. The 
first data extracted from the Dirac stream is parse information. Parse Info indicates what type of 
Data Unit follows, and this information is stored in the decoder state. The decoder continues to read
 pairs of Data Unit and Parse Info until the end of the sequence is reached. The end of sequence 
is indicated by data in the final Parse Info header. If a Data Unit is an Access Unit Header the 
decoded video parameters are updated with the information contained in the header. If the 
Data Unit is a Picture then:
\begin{itemize}
\item the picture is parsed, then decoded
\item the picture is placed in the correct position in the output array
\item the reference picture buffer is managed by deleting obsolete pictures and, if the current picture
is a reference, adding it to the reference buffer
\end{itemize}

Note that for clarity this code ignores the presence of Auxiliary Data and Padding Data in the sequence. 
Nor does it illustrate providing the picture numbers, which are coded in the stream, nor details of the 
coding parameters, which may be required by some applications.

Note also that various operations, such as editing, may result in a discontinuity of picture number 
values between sequences within a Dirac stream, even taking into account picture re-ordering.

\subsection{Reference picture buffer management}
\label{refbuffer}

This section specifies how the Dirac stream data is used to manage the reference 
picture buffer $\RefBuffer$. The reference picture buffer has a maximum size of
$\RefBufferSize$ elements, as set in the applicable level (Appendix \ref{profilelevel}).

The $ref\_picture\_remove()$ process operates as
follows:

\begin{pseudo}{ref\_picture\_remove}{}
\bsFOR{i=0}{\length(\RetiredPictureList)-1}
    \bsCODE{n=\RetiredPictureList[i]}
    \bsFOR{k=0}{\RefBufferSize-1}
       \bsIF{\RefBuffer[k][pic\_number]==n}
            \bsFOR{j=k}{\RefBufferSize-1}
                \bsCODE{\RefBuffer[j]=\RefBuffer[j+1]}
            \bsEND
        \bsEND
    \bsEND
\bsEND
\bsCODE{\RetiredPictureList=\emptyset}
\end{pseudo}

The $get\_ref(n)$ function returns the (first) reference picture in the buffer with 
picture number $n$.  

The $ref\_picture\_add()$ process for adding pictures to the reference picture
buffer proceeds according to the following rules:

{\bf Case 1.} If the reference picture buffer is not full i.e. has fewer than $\RefBufferSize$ elements,
then add the $\CurrentPicture$ data to the end of the buffer. 

{\bf Case 2.} If the reference picture is full i.e. it has $\RefBufferSize$ elements, then remove the
first (i.e. oldest) element of the buffer, $\RefBuffer[0]$, set
\[\RefBuffer[i] = \RefBuffer[i+1] \]
for $i=0$ to $\RefBufferSize-2$, and set the last element $\RefBuffer[\RefBufferSize-1]$ equal to
a copy of $\CurrentPicture$.

\subsection{Video output ranges}
\label{videooutput}

Video output data ranges are deemed to be non-negative, so that the offset and excursion values 
may be applied by subsequent processing. Since decoded video data is bipolar, it must be 
suitably offset before output:

\begin{pseudo}{offset\_output\_data}{picture\_data}
\bsFOREACH{c}{Y, C1, C2}
    \bsIF{c==Y}
        \bsCODE{\BitDepth=\LumaDepth}
    \bsELSE
        \bsCODE{\BitDepth=\ChromaDepth}
    \bsEND
    \bsCODE{comp=picture\_data[c]}
    \bsFOR{y=0}{\height{comp}-1}
        \bsFOR{x=0}{\width{comp}-1}
            \bsCODE{comp[y][x]+=2^{\BitDepth-1}}
        \bsEND
    \bsEND
\bsEND
\end{pseudo}

\subsection{Random access}
\label{randomaccess}

Many applications involve random access of a Dirac sequence. In the context of this specification, this
means beginning reading and parsing data from the beginning of an Access Unit header, located
at some point not at the beginning of a sequence. In such circumstances, the process of Section 
\ref{sequencedecoding} is followed, except that parsing begins from the AU header and pictures
that cannot be decoded, since they use as references pictures not previously accessible or decoded, are discarded.
Clearly, what is decodeable depends upon the point at which the sequence has been accessed.

Compliant Dirac streams shall be so constructed that after parsing and partially decoding one whole
access unit, all pictures in subsequent access units may be fully decoded. Hence the placing of
an AU header within the stream constitutes a guarantee of ultimate decodeability.

\begin{informative*}
\subsection{Non-sequential picture decoding (Informative)}

The ability to decode pictures in a non-sequential manner is important for many applications, 
such as video editing. Non-sequential access means decoding a stream in any manner other 
than decoding pictures sequentially from the beginning of the stream to the end: this may
include decoding only intra pictures, decoding backwards, or decoding pictures in random
parts of the stream. Non-sequential 
picture access is outside the scope of this specification. Nevertheless the Dirac stream has 
been designed to support this feature. This section provides informative notes on this aspect 
of the Dirac stream specification.

Stream navigation, including non-sequential access is supported by the information in the Parse 
Info headers in the stream. Details of Parse Info headers are defined in Section \ref{parseinfo}. However, 
in order to discuss stream navigation, it is necessary to indicate the information contained within 
a Parse Info header. A Parse Info headers contains four components: a Parse Info Prefix, a 
Parse Code, a Next Parse Offset and a Previous Parse Offset.  The Parse Info Prefix is a fixed 
4 byte sequence. The Parse Code specifies the type of the following Data Unit. The Next and 
Previous Parse Offsets specify the number of bytes until the next, or from the previous, Parse 
Info header.

In order to start decoding, other than at the start of a stream, the decoder must first synchronize 
to the stream. The Parse Info Prefix is present to support such synchronization. A decoder would 
first search for the Parse Info Prefix to locate the start of a Parse Info header. The Parse Info 
Prefix is not guaranteed to occur uniquely within Parse Info headers (the entropy coding used in 
Dirac precludes this), the Parse Info Prefix may, by chance, occur within a Data Unit. The decoder 
may read the Next or Previous Parse Info Offsets to confirm that an occurrence of the Parse Info 
Prefix corresponds to a Parse Info header.

When the decoder finds a Parse Info Prefix it may skip forward or back by the value of the appropriate offset
 and check whether the next four bytes are again those of the Parse Info Prefix. If so the 
decoder can be reasonably certain that it has found a Parse Info Header. The probability of a spurious 
prefix occuring is low: 1 in $2^{32}$, since the prefix is 4 bytes long. The probability of finding 
two spurious Prefix sequences separated by the value of Next Parse Offset is 1 in $2^{64}$. The test outlined is, therefore, 
more than adequate in practice. 

Having synchronized with the stream the decoder now needs to locate an Access Unit Header in 
order to find parameters needed to decode pictures. This may be done by moving forward (or backward) 
through the stream, between successive Parse Info block, using Next (or Previous) Parse Offsets, 
until a Parse Info header is found containing the Parse Code for an Access Unit Header. Decoding 
may now commence, as if from the beginning of the stream.

The Dirac stream also supports seeking to a particular picture number. The first four bytes of either 
an Access Unit Header or a Picture contain a 4 byte picture number. Picture numbers provide an
 identifier for each picture with a sequence. So, to find a particular picture, the decoder may 
move forward or backward in the stream, using the offsets in the Parse Info headers, until the 
correct picture number is reached. A picture may be decoded once the parameters within an 
Access Unit Header, in the same sequence, have been read. Picture decoding depends upon decoding
all references upon which a picture depends, and all references upon which they {\em they} depend
and so on. The requirements of Section \ref{randomaccess} imply that this will be at most the
contents of two access units, and with MPEG-style Group of Picture structuring will normally
be much less.
\end{informative*}

\subsection{Overall picture decoding process}
\label{overallpicturedec}

\subsubsection{Picture data initialisation}
\label{picdataconventions}

Picture data from the current picture being decoded is stored in the $\CurrentPicture$ state
variable, which is a structure with indices $pic\_num$, $Y$, $C1$ and $C2$ representing
luma and chroma data (typically Y, Cb and Cr, although other formats are supported too -- see
Appendix \ref{vidsys}).


The $init\_picture\_data()$ initialises the current picture data so that:
\begin{itemize}
\item $\CurrentPicture[pic\_num]=\PictureNumber$
\item $\CurrentPicture[Y]$ is a 2-dimensional array of width $\LumaWidth$ and height $\LumaHeight$, 
all values $\CurrentPicture[Y][y][x]$ set to 0
\item $\CurrentPicture[C1]$ and $\CurrentPicture[C2]$ are 2-dimensional arrays of width $\ChromaWidth$ and height $\ChromaHeight$, 
all values $\CurrentPicture[C1][y][x]$ and $\CurrentPicture[C2][y][x]$ set to 0
\end{itemize}

\subsubsection{Decoding process}
\label{picturedecprocess}

The process for decoding a picture within a Dirac sequence can commence 
when: 
\begin{itemize}
\item an Access Unit header has been located, and parsed, and the default parameters set
\item a picture data unit has been located and parsed, overriding default parameters
\item any reference pictures for the current picture have been decoded
\end{itemize}

At this point the reference picture buffer shall be initialised with no 
reference pictures and picture data initialised as per Section \ref{picdataconventions}.

This initialisation process need only occur once within a sequence, since
apart from the AU picture number, all AU headers within a sequence are
identical. 

After this initialisation process, decoding a picture with picture number 
$n$ in a Dirac sequence consists of:

\begin{pseudo}{picture\_decode}{}
\bsCODE{init\_picture\_data()}{\ref{picdataconventions}}
\bsIF{\ZeroResidual==\false}
    \bsCODE{\CurrentPicture[Y]=idwt(\YTransform)}{\ref{idwt}}
    \bsCODE{\CurrentPicture[C1]=idwt(\COneTransform)}{\ref{idwt}}
    \bsCODE{\CurrentPicture[C2]=idwt(\CTwoTransform)}{\ref{idwt}}
\bsEND
\bsIF{is\_inter()}
    \bsCODE{ref1=get\_ref(\RefOneNum)}
    \bsIF{num\_refs()==2}{\ref{parseinfo}}
        \bsCODE{ref2=get\_ref(\RefTwoNum)}
    \bsEND
    \bsCODE{motion\_compensate(ref1[Y], ref2[Y],  \CurrentPicture[Y], c)}{\ref{motioncompensate}}
    \bsCODE{motion\_compensate(ref1[C1], ref2[C1],  \CurrentPicture[C1], c)}{\ref{motioncompensate}}
    \bsCODE{motion\_compensate(ref1[C2], ref2[C2],  \CurrentPicture[C2], c)}{\ref{motioncompensate}}
\bsEND
\bsCODE{clip\_picture()}{\ref{pictureclip}}
\bsRET{\CurrentPicture}
\end{pseudo}

When randomly accessing a sequence, a picture may not be decodeable because reference pictures
may not be available in the buffer. In this case the current picture may be discarded, although some
decoders may be designed to produce an output. 

A Dirac sequence shall be so constructed so that if
pictures are decoding commences from the beginning of the stream and pictures are decoded in 
stream order, there shall be no undecodeable pictures i.e. the reference pictures associated with
any picture in the sequence shall have occurred prior to that picture in the sequence.

Picture numbers within the stream may not be in numerical order, and subsequent reordering may be
required: the size of the decoded picture buffer required to perform any such reordering is specified
as part of the application profile and level (Appendix \ref{profilelevel}).

\subsection{Inverse discrete wavelet transform}
%%%%%%%%%%%%%%%%%%%%%%%%%%%%%%%%%%%%%%%%%%%%%%%%%%%%%
% - This chapter defines how the inverse discrete - %
% - wavelet transform is done                     - % 
%%%%%%%%%%%%%%%%%%%%%%%%%%%%%%%%%%%%%%%%%%%%%%%%%%%%%

\label{idwt}


This section defines the process iwt() for reconstructing component data
from decoded subbands using the Inverse Wavelet Transform (IWT). iwt()
is invoked in decoding Transform Data units which encapsulate component
data, if and only if ZERO\_COEFFS is FALSE.

The iwt() process consists of two sub-processes:

1. iwt\_synthesis()

2. iwt\_pad\_removal()

The output of this process is a two-dimensional array DATA[][] of pixel
data representing Y, U or V component data.


\subsection{IDWT synthesis operation}
This section defines the process iwt\_synthesis() invoked by iwt().

This is an iterative procedure operating on four subbands at each
iteration stage to produce a new subband. In pseudocode the procedure
is:

\begin{verbatim}
LL=SUBBAND[NUM\_SUBBANDS]
for (n=0 , k=NUM\_SUBBANDS-1; n<TRANSFORM\_DEPTH ; ++n , k-=3)
{
    LL=vh\_synthesis( LL , SUBBAND[k-2] ,  SUBBAND[k-1] , SUBBAND[k] )
}
\end{verbatim}

The decoded component data values will comprise the subband coefficients
LL[][], which are identified with the two-dimensional array DATA[][].



\subsection{Removal of IDWT pad values}This section defines the decoding process iwt\_pad\_removal().

When carrying out the forward transform in the coder, it is desirable
that the data is in blocks which permit the iterative subsampling which
is required as part of the lifting process. For some pictures, the
natural size of the image may not give rise to an appropriate set of
values, so padding is inserted. The padding has no useful data and so
should be discarded.

This process is invoked after iwt\_synthesis(). In this process values
not used subsequently are discarded.

Values WIDTH and HEIGHT are defined as follows. For Intra frame
components, if the component is the luma component, the values are set
as

WIDTH = LUMA\_WIDTH

HEIGHT = LUMA\_HEIGHT

If the component is a chroma component, then

WIDTH = CHROMA\_WIDTH

HEIGHT = CHROMA\_HEIGHT

For Inter frame component data, the values are set as follows. For luma
components,

WIDTH = MC\_LUMA\_WIDTH

HEIGHT = MC\_LUMA\_HEIGHT

If the component is a chroma component, then

WIDTH = MC\_CHROMA\_WIDTH

HEIGHT = MC\_CHROMA\_HEIGHT

All component data with horizontal index greater than or equal to WIDTH
or with vertical index greater than or equal to HEIGHT is discarded
(Figure ??).

[Figure ?? Discarded data values]


\subsection{Vertical and horizontal synthesis}This section specifies the operation of the vertical and horizontal
synthesis process vh\_synthesis().

vh\_synthesis( LL , HL , LH , HH ) is repeatedly invoked by
iwt\_synthesis(). It operates on four subbands labelled LL, HL, LH and HH
to produce a new subband SYNTH\_BAND, which is returned as the result of
the process. The subbands LL, HL, LH and HH shall all have identical
dimensions.

First, the data from the four subbands are interleaved to form a single
two-dimensional array A[][] whose vertical and horizontal dimensions are
twice that of each of the original subbands, using the interleave()
process specified in Section :

A=interleave( LL , HL, LH , HH )

Vertical synthesis is performed second. For each column of coefficients
in the array A[][], the 1d\_synthesis() procedure is applied.

Horizontal synthesis is performed third. For each row of coefficients in
the array A[][], the 1d\_synthesis procedure is applied.

The new subband SYNTH\_BAND comprises the data in the processed 2D array
A[][].



\subsection{Interleaving}This section specifies the interleaving process interleave( LL , HL , LH
, HH ) by which the data elements belonging to four subband data arrays
are combined into a single data array A. The elements of A are defined
by:

A[2*j][2*i] = LL[j][i]

A[2*j][2*i+1] = HL[j][i]

A[2*j+1][2*i] = LH[j][i]

A[2*j+1][2*i+1] = HH[j][i]

for all $i$ such that $0 \leq i < width(LL)$ and all $j$ such that
$0 \leq j < height(LL)$.

\begin{informative}
Note that the interleaving process always creates an array with even
dimensions.
\end{informative}


\subsection{One-dimensional synthesis}This section specifies the one-dimensional synthesis process
1d\_synthesis() applied either to rows or columns of the matrix A[][].

1d\_synthesis() applies filtering to a one dimensional array C[] of
coefficients.

Where 1d\_synthesis() is applied to a row of data, then a value C[k]
shall be identified with a value A[j][k] where j is the index of the row
being processed.

Where 1d\_synthesis() is applied to a column of data, then a value C[k]
shall be identified with a value A[k][i] where i is the index of the
column being processed.

The one-dimensional synthesis process comprises the application of a
number of reversible integer lifting filter operations lift1() to
liftN().

The number of lifting operations and their definition depends upon the
choice of wavelet filter derived from the WAVELET\_FILTER\_INDEX value
defined for the frame data unit. The definition of lifting and of
lifting operations for particular wavelet filters are given in
subsequent sections.


\subsection{Integer lifting}This section defines the general operation of a single integer lifting
operation.

Each lifting process applies either to odd or to even coefficients. If
it applies to even coefficients, then the even coefficients are modified
using the values of odd coefficients only. If it applies to odd
coefficients, then the odd coefficients are modified using the values of
even coefficients only. Specifically, a single lifting filtering
operation is of the form:

\begin{displaymath}
  C(2n) = C(2n) + \big( \sum^M_{i=-N} t_i C(2(n+i) - 1) \big) >> s
\end{displaymath}

if it operates on even coefficients and of the form

\begin{displaymath}
  C(2n+1) = C(2n+1) + \big( \sum^M_{i=-N} t_i C(2(n+i)) \big) >> s
\end{displaymath}

if it operates on odd coefficients. The values ti are the lifting filter
tap values; the value s is the lifting filter scale factor.

The lifting operation is applied for all the even, or all the odd,
coefficients in the row or column array C. A decoder may perform the
individual coefficient filtering operations in any order, as this does
not affect the result.

The filtering process uses edge extension.  Where filtering requires
values which fall out of the range of values of the array C, then the
value selected is determined as follows:

- C[2k] is identified with C[0] if $2k < 0$
- C[2k] is identified with C[length(C)-2] if $2k > length(C)$
- C[2k+1] is identified with C[1] if $2k+1 < 0$
- C[2k+1] is identified with C[length(C)-1] if $2k + 1 > length(C)$

\begin{informative}
This specification defines the lifting process on the basis of lifting
procedures applied to an entire row or column consecutively. It is
possible to implement lifting filtering operations so that a filtering
operation associated with one lifting filter is followed by a filtering
operation associated with another lifting filter. I.e. the order of
iteration is changed. In this case, the order in which filtering is
applied to coefficients does affect the outcome of the process as even
lifting operations may be followed by odd ones, and care must be taken
that values are not modified in the wrong order. Nevertheless such an
implementation may be more efficient, and complies with this
specification if it produces identical results.
\end{informative}


\subsection{avaliable filters}     Daubechies (9,7) lifting filters

This section specifies the lifting filters that must be used if
WAVELET\_FILTER\_INDEX=0.

There are four lifting filters lift1, lift2, lift3, and lift4. Their
filtering operations are given in .


Filter

Parity

Filter equation

lift1

Even

C[2n] -= ( 1817 * ( C[2n-1] + C[2n+1] ) )>>12

lift2

Odd

C[2n+1] -= ( 3616 * ( C[2n] + C[2n+2] ) )>>12

lift3

Even

C[2n] += ( 217 * ( C[2n-1] + C[2n+1] ) )>>12

lift4

Odd

C[2n+1] += ( 6497 * ( C[2n] + C[2n+2] ) )>>12


Table   Lifting filters for WAVELET\_FILTER\_INDEX=0



    Approximate Daubechies (9,7) lifting filters

This section specifies the lifting filters that must be used if
WAVELET\_FILTER\_INDEX=1.

There are two lifting filters lift1 and lift2. Their filtering
operations are given in .


Filter

Parity

Filter equation

lift1

Even

C[2n] -=  ( C[2n-1] + C[2n+1] )>>2

lift2

Odd

C[2n+1] += ( 9*( C[2n] + C[2n+2] ) - ( C[2n-2]+C[2n+4] ) )>>4


Table   Lifting filters for WAVELET\_FILTER\_INDEX=1


    (5,3) lifting filters

This section specifies the lifting filters that must be used if
WAVELET\_FILTER\_INDEX=2.

There are two lifting filters lift1 and lift2. Their filtering
operations are given in .


Filter

Parity

Filter equation

lift1

Even

C[2n] -=  ( C[2n-1] + C[2n+1] )>>2

lift2

Odd

C[2n+1] += ( C[2n] + C[2n+2]  )>>1


Table   Lifting filters for WAVELET\_FILTER\_INDEX=2


    (13,5) lifting filters

This section specifies the lifting filters that must be used if
WAVELET\_FILTER\_INDEX=3.

There are two lifting filters lift1 and lift2. Their filtering
operations are given in .


Filter

Parity

Filter equation

lift1

Even

C[2n] -= ( 9*( C[2n-1] + C[2n+1] ) - ( C[2n-3]+C[2n+3] ) )>>4

lift2

Odd

C[2n+1] += ( 9*( C[2n] + C[2n+2] ) - ( C[2n-2]+C[2n+4] ) )>>5


Table   Lifting filters for WAVELET\_FILTER\_INDEX=3





\subsection{Motion compensation}
\label{motioncompensate}

This section defines the operation of the process
$motion\_compensate(ref1, ref2,  pic, c)$ for motion-compensating a
picture component array  $pic$ of type $c=Y, U$ or $V$ from reference 
component arrays $ref1$ and $ref2$ of the same type.

This process shall be invoked for each component in a picture, subsequent to the 
decoding of coefficient data, specified in Section \ref{transformdec}, and the Inverse Wavelet Transform (IWT), specified in Section \ref{idwt}. 

Motion compensation shall use the motion block data $\BlockData$ and optionally may use the
global motion parameters $\GlobalParams$.

\begin{informative*}
\subsubsection{Overlapped Block Motion Compensation (OBMC) (Informative)}

Motion compensated prediction methods provide methods for determining 
predictions for pixels in the current picture by using motion vectors to 
define offsets from those pixels to pixels in previously decoded
pictures. Motion compensation techniques vary in how those pixels are grouped
together, and how a prediction is formed for pixels in a given group. In 
conventional  block motion compensation, as used in MPEG2, H.264 and many other
codecs, the picture is divided into {\em disjoint} rectangular blocks and the
motion vector or vectors associated with that block defines the offset(s) into
the reference pictures.

In OBMC, by contrast, the predicted picture is divided into a regular overlapping 
blocks of dimensions $xblen$ by $yblen$ that cover at least the entire picture 
area as shown in figure \ref{fig:blockcoverage}.  Overlapping is ensured by starting
each block at a horizontal separation $xbsep$ and a vertical separation $ybsep$ 
from its neighbours, where these values are less than the corresponding block dimensions.
\end{informative*}

\begin{figure}[!ht]
\centering
\includegraphics[width=0.7\textwidth]{figs/block-coverage.eps}
\caption{Block coverage of the predicted picture}
\label{fig:blockcoverage}
\end{figure}

\begin{informative*}
The overlap (offset) between blocks horizontally is $xblen - xbsep$ and vertically is
$yblen - ybsep$. As a result pixels in the overlapping areas lie in more than
one block, and so more than one motion vector set (and set of associated predictions)
applies to them. Indeed, a pixel may have up to eight predictions, as it may belong to
up to four blocks, each of which may have up to two motion vectors. These are combined
into a single prediction by using weights, which are so constructed so as to sum to 1. In the
 Dirac integer implementation, fractional weights are achieved by insisting that weights sum 
to a power of 2, which is then shifted out once all contributions have been summed.

In Dirac blocks are positioned so that blocks will overspill the left and top edges by 
($xoffset$) and ($yoffset$) pixels.  The number of blocks has been
determined (Section \ref{}) so that the picture area is wholly covered, and the overspill
 on the right hand and bottom edges will be at least the amount on the left and top edges. 
Indeed, the number of blocks has been set so that the blocks divide into whole superblocks
(sets of 4x4 blocks), which mean that some blocks may fall entirely out of the picture area. 
 Any predictions for pixels outside the actual picture area are discarded.

\end{informative*}

\subsubsection{Overall motion compensation process}
\label{mcprocess}

The motion compensation process shall form an integer prediction for each pixel in 
the predicted picture component $pic$, which shall be added to the pixel value, and
 then clipped to keep it in range.

The $motion\_compensate()$ process is defined by means of a temporary data
array $mc\_tmp$ for storing the motion-compensated prediction for the 
current picture. 

The $motion\_compensate()$ shall be defined as follows:

\begin{pseudo}{motion\_compensate}{ref1, ref2,  pic, c}
\bsIF{c==Y}
    \bsCODE{bit\_depth=\LumaDepth}
\bsELSE
    \bsCODE{bit\_depth=\ChromaDepth}
\bsEND
\bsCODE{init\_dimensions(c)}{\ref{mcdimensions}}
\bsCODE{mc\_tmp=init\_temp\_array()}{\ref{mctemparray}}
%\bsCODE{total\_wt\_bits=set\_mc\_wt()}{\ref{wtbits}}
\bsFOR{j=0}{\BlocksY-1}
    \bsFOR{i=0}{\BlocksX-1}
        \bsCODE{block\_mc(mc\_tmp,i,j)}{\ref{blockmc}}
    \bsEND
\bsEND
\bsFOR{y=0}{\LenY-1}
    \bsFOR{x=0}{\LenX-1}
%        \bsCODE{pic[y][x] += (mc\_tmp[y][x]+2^{total\_wt\_bits-1})\gg total\_wt\_bits}
       \bsCODE{pic[y][x] += (mc\_tmp[y][x]+32)\gg 6}
        \bsCODE{pic[y][x] = \clip(pic[y][x], -2^{bit\_depth-1}, 2^{bit\_depth-1}-1)}
    \bsEND
\bsEND
\end{pseudo}

\begin{informative}
Six bits are used for the overlapped-block weighting matrix. This ensures that 10-bit
data may normally be motion compensated using 16-bit words as per Section \ref{blockmc}.
\end{informative}

\subsubsection{Dimensions}
\label{mcdimensions}
Since motion compensation shall apply to both luma and (potentially subsampled)
chroma data, for simplicity a number of variables are defined by the 
$init\_dimensions()$ function, which is as follows:

\begin{pseudo}{init\_dimensions}{c}
\bsIF{c==Y}
   \bsCODE{\LenX=\LumaWidth}
   \bsCODE{\LenY=\LumaHeight}
   \bsCODE{\XBlen=\LumaXBlen}
   \bsCODE{\YBlen=\LumaYBlen}
   \bsCODE{\XBsep=\LumaXBsep}
   \bsCODE{\YBsep=\LumaYBsep}
\bsELSE
   \bsCODE{\LenX=\ChromaWidth}
   \bsCODE{\LenY=\ChromaHeight}
   \bsCODE{\XBlen=\ChromaXBlen}
   \bsCODE{\YBlen=\ChromaYBlen}
   \bsCODE{\XBsep=\ChromaXBsep}
   \bsCODE{\YBsep=\ChromaYBsep}
\bsEND
\bsCODE{\XOffset = (\XBlen-XBsep)//2}
\bsCODE{\YOffset = (\YBlen-YBsep)//2}
\end{pseudo}

\begin{informative}
The subband data that makes up the IWT coefficients is padded in order that the IWT
may function correctly. For simplicity, in this specification, padding data is removed
after the IWT has been performed so that the picture data and reference data arrays have
the same dimensions for motion compensation. However, it may be more efficient to 
perform all operations prior to the output of pictures using padded data, i.e. to discard
 padding values subsequent to motion compensation. Such a course of action is equivalent,
 so long as it is realised that blocks must be regarded as edge blocks if they overlap the
 actual picture area, not the larger area produced by padding.
\end{informative}

\subsubsection{Initialising the motion compensated data array}
\label{mctemparray}

The $init\_temp\_array()$ function shall return a two-dimensional data array with
horizontal size $\LenX$ and vertical size $\LenY$, such that each element of the two dimensional array shall be set to zero.

\begin{comment}
\subsubsection{Weighting bits}
\label{wtbits}


[Can omit this section if spatial weights are just set to 6 bits.]

The function $set\_mc\_wts()$ shall set the total number of extra bits of resolution
that shall be added to motion compensation data as a result of applying weighting
matrices and picture weights. It shall be defined as follows:

\begin{pseudo}{set\_mc\_wts}{}
\bsCODE{hbits  = \intlog2(\XOffset)+2}
\bsCODE{vbits  = \intlog2(\YOffset)+2}
\bsRET{hbits+vbits+\RefsWeightPrecision}
\end{pseudo}

\begin{informative}
This is the number of bits added to pixel values in order to perform OBMC 
reversibly with integer arithmetic using the spatial matrix specified in Sections 
\ref{mcspatialweights} and the reference weights extracted in
parsing the picture prediction header data (Section \ref{refpicweights}).

Note that  the number of motion compensation bits depends upon the 
block sizes -- specifically
the block overlaps - selected. A Dirac decoder level (Section \ref{profilelevel}) 
specifies the maximum block overlaps allowable, and hence 
the word widths necessary for processing OBMC. If we assume that 
the picture weights are complementary (i.e. the weights
for reference 1 and reference 2 sum to $2^\RefsWeightPrecision$, 
then the number of bits required for performing motion compensation 
calculations is \[bit\_depth+total\_wt\_bits+\RefsWeightPrecision\]
unsigned bits. 8 bit video data encoded with block overlaps of 4 
luminance pixels and the standard picture weights therefore
requires 8+3+3+1=15 unsigned bits. The additional bit within 
a 16 bit word could be used to provide additional reference 
weighting.
\end{informative}
\end{comment}

\subsubsection{Motion compensation of a block}
\label{blockmc}

This section defines the $block\_mc()$ process for motion-compensating a single
block.

Each block shall be motion-compensated by applying a weighting matrix to a block prediction and adding the weighted prediction into the motion-compensated 
prediction array. 

The $block\_mc()$ process shall be defined as follows:

\begin{pseudo}{block\_mc}{mc\_pred,i,j}
\bsCODE{xstart = i*\XBsep-\XOffset}
\bsCODE{ystart = j*\XBsep-\XOffset}
\bsCODE{xstop = (i+1)*\XBsep+\XOffset}
\bsCODE{ystop = (j+1)*\YBsep+\YOffset}
\bsCODE{mode=\BlockData[j][i][mode]}
\bsCODE{W=spatial\_wt(i,j)}{\ref{pixelprediction}}
\bsFOR{y=\max(ystart,0)}{\min(ystop,\LenY)-1}
    \bsFOR{x=\max(xstart,0)}{\min(xstop,\LenX)-1}
        \bsCODE{p=x-xstart}
        \bsCODE{q=y-ystart}
        \bsIF{mode==\Intra}
            \bsCODE{val=\BlockData[j][i][dc][c]}
        \bsELSEIF{mode==\RefOneOnly}
            \bsCODE{val=pixel\_pred(ref1, 1, i, j, x, y, c)}{\ref{pixelprediction}}
            \bsCODE{val*=\RefOneWeight+\RefTwoWeight}
            \bsCODE{val=(val+2^{\RefsWeightPrecision-1})\gg\RefsWeightPrecision}
        \bsELSEIF{mode==\RefTwoOnly}
            \bsCODE{val=pixel\_pred(ref2, 2, i, j, x, y, c)}{\ref{pixelprediction}}
            \bsCODE{val*=\RefOneWeight+\RefTwoWeight}
            \bsCODE{val=(val+2^{\RefsWeightPrecision-1})\gg\RefsWeightPrecision}          
        \bsELSEIF{mode==\RefOneAndTwo}
            \bsCODE{val1=pixel\_pred(ref1, 1, i, j, x, y, c)}{\ref{pixelprediction}}
            \bsCODE{val1*=\RefOneWeight}
            \bsCODE{val2=pixel\_pred(ref2, 2, i, j, x, y, c)}{\ref{pixelprediction}}
            \bsCODE{val2*=\RefTwoWeight}
            \bsCODE{val=(val1+val2+2^{\RefsWeightPrecision-1})\gg\RefsWeightPrecision}
        \bsEND
        \bsCODE{val *= W[q][p]}
        \bsCODE{mc\_tmp[y][x]+=val}
    \bsEND
\bsEND
\end{pseudo}

\begin{informative}
Note that if the two reference weights are 1 and $\RefsWeightPrecision$ is 1, then
reference weighting is transparent and
\begin{pseudo*}
\bsCODE{val=pixel\_pred(ref1, 1, i, j, x, y, c)}
\bsCODE{val*=\RefOneWeight+\RefTwoWeight}
\bsCODE{val=(val+2^{\RefsWeightPrecision-1})\gg\RefsWeightPrecision}
\bsCODE{\ldots}
\end{pseudo*}

reduces to

\begin{pseudo*}
\bsCODE{val=pixel\_pred(ref1, 1, i, j, x, y, c)}
\bsCODE{\ldots}
\end{pseudo*}

In this case, therefore, the normal reference weighting produces no additional dynamic
range for internal processing and 10 bit video can be motion compensated with 16 bit
unsigned internal values.

In general, however, the worst case internal bit widths consist of the video bit depth plus the maximum of: 6 (the spatial matrix bit width) and the value of $\RefsWeightPrecision$. 6 bits
should be sufficient for most fading compensation applications, and so 16 bit internals will
suffice for all practical motion compensation scenarios for 8 and 10 bit video.
\end{informative}


\subsubsection{Spatial weighting matrix}

\label{mcspatialweights}

This section specifies the function $spatial\_wt(i,j)$ for deriving the 6-bit spatial weighting 
matrix that shall be applied to the block with coordinates  $(i,j)$. 

Note that other weights shall be applied to the prediction as a result of the 
weights applied to each reference.

The same weighting matrix shall be returned for all blocks within the interior
of the picture component array. Suitably modified weighting matrices shall
be returned for blocks at the edges of the picture component data array.

The function shall return a two-dimensional spatial weighting matrix. This
shall apply a linear roll-off in both horizontal and vertical directions.

The spatial matrix returned shall be the product of a horizontal and a vertical
weighting matrix. It shall be defined as follows:

\begin{pseudo}{spatial\_wt}{i,j}
\bsFOR{y==0}{\YBlen-1}
    \bsFOR{x==0}{\XBlen-1}
        \bsCODE{W=h\_wt(i)[x]*v\_wt(j)[y]}
    \bsEND
\bsEND
\bsRET{W}
\end{pseudo}

The horizontal weighting function shall be defined as follows:

\begin{pseudo}{h\_wt}{i}
\bsFOR{x=0}{2*\XOffset-1}
    \bsCODE{hwt[x]=1+(6*x+\XOffset-1)//(2*\XOffset-1)}
    \bsCODE{hwt[x+\XBsep]=8-hwt[x]}
\bsEND
\bsFOR{2*\XOffset}{\XBsep-1}
    \bsCODE{hwt[x]=8}
\bsEND
\bsIF{i==0}
    \bsFOR{x=0}{2*\XOffset-1}
        \bsCODE{hwt[x]=8}
    \bsEND
\bsELSEIF{i==\BlocksX-1}
    \bsFOR{x=0}{2*\XOffset-1}
        \bsCODE{hwt[x+\XBsep]=8}
    \bsEND
\bsEND
\bsRET{hwt}
\end{pseudo}

The vertical weighting function  shall be defined as follows:

\begin{pseudo}{v\_wt}{j}
\bsFOR{y=0}{2*\YOffset-1}
    \bsCODE{vwt[y]=1+(6*y+\XOffset-1)//(2*\YOffset-1)}
    \bsCODE{vwt[y+\XBsep]=8-vwt[y]}
\bsEND
\bsFOR{2*\YOffset}{\YBsep-1}
    \bsCODE{xwt[y]=8}
\bsEND
\bsIF{j==0}
    \bsFOR{y=0}{2*\YOffset-1}
        \bsCODE{vwt[y]=8}
    \bsEND
\bsELSEIF{j==\BlocksY-1}
    \bsFOR{y=0}{2*\YOffset-1}
        \bsCODE{vwt[x+\YBsep]=8}
    \bsEND
\bsEND
\bsRET{vwt}
\end{pseudo}

\begin{informative}
The horizontal and vertical weighting arrays satisfy the perfect reconstruction property across block overlaps by construction:
\begin{eqnarray*}
hwt[x+\XBsep] & = & 8 - hwt[x] \\ 
vwt[y+\YBsep] & = & 8 - vwt[y]  
\end{eqnarray*}

In addition, it can be shown they are always symmetric (except at picture edges), or
equivalently the leading edges have skew-symmetry about the half-way point:
\begin{eqnarray*}
hwt[\XBlen-1-x] & =  & hwt[x] \\
vwt[\YBlen-1-y] & = & vwt[y] 
\end{eqnarray*}

The horizontal and vertical weighting matrix components for various block
 overlaps are shown in Table \ref{table:leadingedges}. 
These encompass all the default values listed 
in Table \ref{} for both luma and chroma.
\end{informative}
\begin{table}[!ht]
\centering
\begin{tabular}{|c|c|c|}
\hline
\rowcolor[gray]{0.75}\bf{Overlap}  & \bf{Offset} & \bf{Leading edge} \\
\rowcolor[gray]{0.75}\bf{(length-separation)} & & \\
\hline
2 & 1 & 1,7\\
\hline
4 & 2 & 1,3,5,7\\
\hline
8 & 4 & 1,2,3,4,4,5,6,7\\
\hline
16 & 8 & 1,1,2,2,3,3,3,4,4,5,5,5,6,6,7,7 \\
\hline
\end{tabular}
\caption{Leading and trailing edge values for different block overlaps}
\label{table:leadingedges}
\end{table}

\begin{comment}
The profile of the matrix 
for interior blocks is illustrated in Figure \ref{fig:weightprofile}.

\begin{figure}[!ht]
\centering
\includegraphics[width=0.7\textwidth]{figs/obmc-profile}
\caption{Profile of overlapped-block motion compensation matrix}
\label{fig:weightprofile}
\end{figure}

\begin{informative*}
\subsubsection{Reference weights and fade prediction (Informative)}

The reference prediction weights used for each prediction mode for 
block prediction (Section \ref{blockmc}) may appear 
confusing. It is helpful
to think of two cases for using reference picture weighting. The first is interpolative 
prediction, where the picture being predicted is, for example, a cross-fade and is
closely approximated by some mixture of the reference pictures:
 $P\backsimeq\delta R_1+(1-\delta)R_2$. Here the weights we'd like to
use for each frame prediction add up to 1 (or $2^\RefsWeightPrecision$ 
for integer weights). 
The second case is scaling prediction, where 
the weights we'd like to use for the frame predictions don't add up to 1: for example,
a fade to or from black
$P\backsimeq\delta_1 R_1$ and $P\backsimeq\delta_2 R_2$. It is not possible to choose 
weights for each prediction mode which will be optimal both cases. The weighting
factors chosen will give work with interpolative prediction (which is more common) 
but are not perfect for scaling prediction. It would have been possible to create a variety of
prediction modes to cover all cases, however the potential savings do not justify the
additional complexity.

For interpolative prediction, all data in the current picture will be of commensurate scale to
that of the references. In forming the bi-directional prediction, a value 
$W_1 p_1 + W_2 p2_2$ is 
formed, so the prediction has "scale" $W_1+W_2$. $W_1+W_2$ is 
therefore the weighting value used to scale unidirectional prediction, in order to provide
predictions of commensurate order. The unity weighting value 
$2^\RefsWeightPrecision$ is used
for DC blocks as this gives the best prediction, and in the interpolative case 
this equals $W_1+W_2$
so all predictions are of the same order.

The weighting factors we would like to use for unidirectionally 
redicted blocks in the scaling case
are $2W_1$ and $2W_2$ - the factor 2 takes into account that 
we're only adding in one prediction
value as against two for bidirectional prediction. These factors differ f
rom $W_1+W_2$, and hence
unidirectional prediction is incorrect when there are two references. 
Note, however, that we can
still perform prediction with the correct scaling values when we 
only have a single reference. Note
also that the value of $W_1+W_2$ was selected instead of 
$2^\RefsWeightPrecision$, which
would be equivalent in the interpolative case, as it gives a 
better approximation when the
weights do not sum to $2^\RefsWeightPrecision$.
\end{informative*}
\end{comment}

\subsubsection{Pixel prediction}
\label{pixelprediction}

This section defines the operation of the $pixel\_pred(ref, ref\_num, i, j, x, y, c)$ 
process which shall be used for forming the prediction for a pixel 
with coordinates $(x,y)$ in component $c$, belonging to the block with coordinates $(i,j)$.

The pixel prediction process shall consist of two stages. In the first stage, a motion vector
 to be applied to pixel $(x,y)$ shall be derived. For block motion, this shall be a block
 motionvector that shall apply to all pixels in a block. For global motion the motion
vector shall be computed from the global motion parameters and may vary pixel-by-pixel.

In the second stage, the motion vector shall be used to derive coordinates in an reference picture.

\begin{pseudo}{pixel\_pred}{ref,ref\_num,i,j,x,y,c}
\bsIF{\BlockData[j][i][global]==\false}
  \bsCODE{mv= \BlockData[j][i][ref]}
\bsELSE
  \bsCODE{mv=global\_mv(ref, ref\_num, x, y, c)}{\ref{globalmv}}
\bsEND
\bsIF{c!=Y}
  \bsCODE{mv = chroma\_mv\_scale(mv)}{\ref{chromamvscale}}
\bsEND
\bsCODE{px = (x\ll \MotionVectorPrecision)+mv[0]}
\bsCODE{py = (y\ll \MotionVectorPrecision)+mv[1]}
\bsIF{\MotionVectorPrecision>0}
  \bsRET{subpel\_predict(ref, c, px, py))}{\ref{upconvert}}
\bsELSE
  \bsRET{ref[\clip(py,0,\height(ref)-1)][\clip(px,0,\width(ref)-1)]}
\bsEND
\end{pseudo}

\subsubsection{Global motion vector field generation}
\label{globalmv}

This section specifies the operation of the $global\_mv(ref, ref\_num, x,y, c)$ process
for deriving a global motion vector for a pixel at location $(x,y)$, in a component of 
type $c$ from a reference $ref$.

The function shall be defined as follows:

\begin{pseudo}{global\_mv}{ref, ref\_num, x,y, c}
\bsCODE{ez  =  \GlobalParams[ref\_num][ZRS\_exp]}
\bsCODE{ep  =  \GlobalParams[ref\_num][perspective\_exp]}
\bsCODE{b=\GlobalParams[ref\_num][pan\_tilt]}
\bsCODE{A=\GlobalParams[ref\_num][ZRS]}
\bsCODE{m=2^{ep}-(c[0]*x+c[1]*y)}
\bsCODE{v[0]=m*((A[0][0]*x+A[0][1]*y)+2^{ez}*b[0])}
\bsCODE{v[1]=m*((A[1][0]*x+A[1][1]*y)+2^{ez}*b[0])}
\bsCODE{v[0] = (v[0]+(1\ll(ez+ep)) )\gg (ez+ep)}
\bsCODE{v[1] = (v[1]+(1\ll(ez+ep)) )\gg (ez+ep)}
\bsRET{v}
\end{pseudo}

\begin{informative}
Write ${\bf x}=\left( \begin{array}{c} x\\y \end{array}\right)$. 
Mathematically, we wish the global motion vector ${\bf v}$ to be defined by:
\[{\bf v}=\dfrac{{\bf Ax}+{\bf b}}{1+{\bf c}^T{\bf x}}\]
where: ${\bf A}$ is a matrix describing the degree of zoom, rotation or shear; ${\bf b}$
is a translation vector; and ${\bf c}$ is a perspective vector which expresses the
degree to which the global motion is not orthogonal to the axis of view.

In Dirac, this formula is adjusted in two ways in order to get an implementable result.
Firstly, the perspective element is adjusted to remove a division, changing the 
formula to:
\[{\bf v}=(1-{\bf c}^T{\bf x})({\bf Ax}+{\bf b})\]
which is valid for small ${\bf c}$. Secondly, the formula is re-cast in terms of integer
arithmetic by giving the matrix element an accuracy factor $\alpha$ and the perspective
element an accuracy factor $\beta$:
\[{\bf v}=(1-2^{-\beta}{\bf c}^T{\bf x})(2^{-\alpha}{\bf Ax}+{\bf b})\]
where the parameters ${\bf A}, {\bf b},{\bf c}$ are now integral. (No accuracy bits are required for the translation, since it must be an integral number of sub-pixels.) 

This reduces to
\[2^{\alpha+\beta}{\bf v}=(2^\beta-{\bf c}^T{\bf x})({\bf Ax}+2^\alpha{\bf b})\]
and this formula is used for the computation of values.
\end{informative}

\subsubsection{Chroma subsampling}
\label{chromamvscale}

When motion compensating chroma components, motion vectors shall be scaled by the
$chroma\_mv\_scale()$ function. This produces chroma vectors in units of 
$\MotionVectorPrecision$ with respect to the chroma samples, as follows:

\begin{pseudo}{chroma\_mv\_scale}{v}
\bsCODE{sv[0] = v[0]//chroma\_h\_ratio()}{\ref{picturedimensions}}
\bsCODE{sv[1] = v[1]//chroma\_v\_ratio()}{\ref{picturedimensions}}
\bsRET{sv}
\end{pseudo}.

\begin{informative}
Recall that division in this specification rounds towards -infinity. This division can be achieved by a bit-shift in C/C++ as chroma dimension ratios are 1 or 2.
\end{informative}


\subsubsection{Sub-pixel prediction}
\label{upconvert}

This section defines the operation of the $subpel\_predict(ref, c, u, v)$ function
for producing a sub-pixel accurate value at location $(u,v)$ from an upconverted picture reference component of type $c$ (Y, C1 or C2). 

Upconversion shall be defined by means of a half-pixel interpolated reference array
$upref$.  $upref$ shall have dimensions $(2W-1)$x$(2H-1)$ where the original reference 
picture component array has dimensions $W$x$H$, as per Section \ref{halfpel}. 

Motion vectors shall be permitted to extend beyond the edges of reference picture data,
 where values lying outside shall be determined by edge extension. 

If $\MotionVectorPrecision==2$, upconverted values shall be derived directly from the
the half-pixel interpolated array $upref$, which shall be calculated as per Section \ref{halfpel}.

If $\MotionVectorPrecision==2$ or $\MotionVectorPrecision==3$, upconverted values shall be
derived by linear interpolation from the half-pixel interpolated array.

The sub-pixel prediction process shall be defined as follows:

\begin{pseudo}{subpel\_predict}{ref,c,u,v}
\bsCODE{upref=interp2by2(ref,c)}{\ref{halfpel}}
\bsIF{\MotionVectorPrecision==1}
    \bsCODE{xpos=\clip(u,0,\width(upref)-1)}
    \bsCODE{ypos=\clip(v,0,\height(upref)-1)}
    \bsRET{upref[ypos][xpos]}
\bsELSE
    \bsCODE{hu = u \gg (\MotionVectorPrecision-1)}
    \bsCODE{hv = v \gg (\MotionVectorPrecision-1)} 
    \bsCODE{ru = u-(hu\ll (\MotionVectorPrecision-1))}
    \bsCODE{rv = v-(hv\ll (\MotionVectorPrecision-1))}
    \bsCODE{w00 = (2^{\MotionVectorPrecision-1}-rv)*(2^{\MotionVectorPrecision-1}-ru)}
    \bsCODE{w01 = (2^{\MotionVectorPrecision-1}-rv)*ru}
    \bsCODE{w10 = rv*(2^{\MotionVectorPrecision-1}-ru)}
    \bsCODE{w11 = rv*ru}
    \bsCODE{xpos = \clip(hu, 0, \width(upref)-1)}
    \bsCODE{xpos1 = \clip(hu+1, 0,\width(upref)-1)}
    \bsCODE{ypos = \clip(hv, 0, \height(upref)-1)}
    \bsCODE{ypos1 = \clip(hv+1, 0, \height(upref)-1)}
    \bsCODE{\begin{array}{ll} val = & w00*upref[ypos][xpos]+w01*upref[ypos][xpos1]+ \\
                                & w10*upref[ypos1][xpos]+w11*upref[ypos1][xpos1]
            \end{array}}
    \bsRET{(val+2^{\MotionVectorPrecision-1})\gg\MotionVectorPrecision}
\bsEND
\end{pseudo} 

\begin{informative}
$hu$ and $hv$ represent the half-pixel part of the sub-pixel position $(u,v)$.

$ru$ and $rv$ represent the remaining sub-pixel component of the position.
$ru$ and $rv$ satisfy \[0\leq ru,rv <2^{\MotionVectorPrecision-1}\] 

The four weights $w00,w01,w10$ and $w11$ sum to $2^{\MotionVectorPrecision}$, and
hence the upconverted value is returned to pixel ranges in the pseudocode above.

Note that the remainder values $ru$ and $rv$, and hence the four weight values, 
only depend on the motion vectors. This is because
$u$ and $v$ have been computed by scaling the picture coordinates by
$2^{\MotionVectorPrecision}$ and adding the motion vector.

In particular constant linear interpolation weights are applied throughout a 
block when block motion is used. Likewise, the necessity of clipping the ranges of
$xpos$, $ypos$ etc can be determined in advance for each block by checking whether any 
corner of the reference block will fall outside of the reference picture area. In most
cases it will not and clipping will not be required for motion compensating most blocks. 
\end{informative}


\subsubsection{Half-pixel interpolation}
\label{halfpel}

This section defines the $interp2by2(ref,c)$ process for generating
an upconverted reference array $upref$ representing a half-pixel interpolation of 
the reference array $ref$ for component $c$ (Y, C1, or C2). 

$upref$ shall be created in two stages. The first stage shall upconvert vertically. The second stage shall upconvert horizontally. 

$upref$ shall have width $2*\width(ref)-1$ and height $2*\height(ref)-1$, so that all
edge values shall be copied from the original array and not interpolated.

The interpolation filter shall be the 8-tap symmetric filter with taps as defined in Figure \ref{upfilter}.

\begin{figure}[h!]
\begin{centering}
\begin{tabular}{l|ccccc}
Tap & $t[0]$ & $t[1]$ & $t[2]$ & $t[3]$\\
\hline
Value & 21 & -7 & 3 & -1
\end{tabular}
\caption{Interpolation filter coefficients \label{upfilter}}
\end{centering}
\end{figure}

Where coefficients used in the filtering process fall outside the bounds of the 
reference array, values shall be supplied by edge extension. 

The overall process shall be defined as follows:

\begin{pseudo}{interp2by2}{ref,c}
\bsIF{c==Y}
    \bsCODE{bit\_depth=\LumaDepth}
\bsELSE
    \bsCODE{bit\_depth=\ChromaDepth}
\bsEND
\bsFOR{q=0}{2*\height(ref)-2}
    \bsIF{q\%2==0}
        \bsFOR{p=0}{\width(ref)-1}
            \bsCODE{ref2[q][p]=ref[q//2][p]}
        \bsEND
    \bsELSE
        \bsFOR{p=0}{\width(ref)-1}
            \bsCODE{ref2[q][p]=16}
            \bsFOR{i=0}{3}
                \bsCODE{ypos=(q-1)//2-i}
                \bsCODE{ref2[q][p]+=t[i]*ref[\clip(ypos,0,\height(ref)-1)][p]}
                \bsCODE{ypos=(q+1)//2+i}
                \bsCODE{ref2[q][p]+=t[i]*ref[\clip(ypos,0,\height(ref)-1)][p]}
            \bsEND
            \bsCODE{ref2[q][p] \gg=5}
            \bsCODE{ref2[q][p] = \clip(ref2[q][p], -2^{bit\_depth-1}, 2^{bit\_depth-1}-1)}
        \bsEND    
    \bsEND
\bsEND
\bsFOR{q=0}{2*\height(ref)-2}
    \bsFOR{p=0}{2*\width(ref)-2}
        \bsIF{p\%2==0}
            \bsCODE{upref[q][p]=ref2[q][p//2]}
        \bsELSE
            \bsCODE{upref[q][p]=16}
            \bsFOR{i=0}{3}
                \bsCODE{xpos=(p-1)//2-i}
                \bsCODE{upref[q][p]+=t[i]*ref2[q][\clip(xpos,0,\height(ref)-1)]}
                \bsCODE{xpos=(p+1)//2+i}
                \bsCODE{upref[q][p]+=t[i]*ref2[q][\clip(xpos,0,\height(ref)-1)]}
            \bsEND
            \bsCODE{upref[q][p] \gg=5}
            \bsCODE{upref[q][p] = \clip(upref[q][p], -2^{bit\_depth-1}, 2^{bit\_depth-1}-1)}
        \bsEND
    \bsEND
\bsEND
\end{pseudo}

\begin{informative}
While this filter may appear to be variable separable, the integer rounding and 
clipping processes prevent this being so. Note also that the clipping process for
filtering terms implies that the upconversion uses edge-extension at the array
edges, consistent with the edge-extension used in motion-compensation itself.
\end{informative}


\subsection{Clipping}
\label{pictureclip}

Picture data must be clipped prior to being output or being
used as a reference:

\begin{pseudo}{clip\_picture}{}
\bsFOREACH{c}{Y,C1,C2}
    \bsCODE{clip\_component(\CurrentPicture[c])}
\bsEND
\end{pseudo}


\begin{pseudo}{clip\_component}{comp\_data,c}
\bsIF{c==Y}
    \bsCODE{\BitDepth=\LumaDepth}
\bsELSE
    \bsCODE{\BitDepth=\ChromaDepth}
\bsEND
\bsFOR{y=0}{\height(comp\_data)-1}
    \bsFOR{x=0}{\width(comp\_data)-1}
        \bsCODE{data = \clip(comp\_data[y][x], -2^{\BitDepth-1}, 2^{\BitDepth-1}-1)}
     \bsEND
\bsEND
\end{pseudo}

\begin{informative}
Note that clipping is incorporated into motion compensation, so that strictly speaking additional
clipping is only required for intra pictures.
\end{informative}