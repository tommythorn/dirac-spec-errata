%%%%%%%%%%%%%%%%%%%%%%%%%%%%%%%%%%%%%%%%%%%%%%%%%
% - This chapter defines the overall process  - %
% - for decoding a picture                    - % 
%%%%%%%%%%%%%%%%%%%%%%%%%%%%%%%%%%%%%%%%%%%%%%%%%

\label{picturedec}

This section specifies the process for decoding a picture from the Dirac stream. Picture decoding depends upon
correctly parsing the Dirac bitstream, and decoding operations are dependent upon the parsing operations
set out in Sections \ref{streamstructure}, \ref{motiondec} and \ref{transformdec}.

This section does not specify how pictures are encoded, nor how pictures are reordered and presented for display, 
which is described in Section \ref{profilelevel}. 

\subsection{Introduction}

Dirac supports both intra and inter picture coding, with forward and backward prediction. This means that
pictures may be encoded in the stream in non-display order: reordering pictures will be required in order
to display them correctly, and  decoded picture buffer will be necessary to store pictures while temporally 
prior pictures are decoded. Note that the core Dirac specification does not encompass the operation of the
decoded picture buffer: this is specified in conjunction with the level and profile values extracted from
the stream (Section \ref{parseparameters}), in Appendix \ref{profilelevel}. 

Decoded pictures may, however, be reference pictures, used for the prediction of subsequent pictures
in the Dirac stream. Reference pictures are stored in a reference picture buffer $\RefBuffer$. The operation
of $\RefBuffer$ does form part of the core Dirac specification, and the rules for management of the
buffer are set out in Section \ref{refbuffer}.

\subsection{Overall picture decoding process}
\label{overallpicturedec}

\subsubsection{Picture data initialisation}
\label{picdataconventions}

Picture data from the current picture being decoded is stored in the $\CurrentPicture$ state
variable, which is a structure with indices $pic\_num$, $Y$, $U$ and $V$.


The $init\_picture\_data()$ initialises the current picture data so that:
\begin{itemize}
\item $\CurrentPicture[pic\_num]=\PictureNumber$
\item $\CurrentPicture[Y]$ is a 2-dimensional array of width $\LumaWidth$ and height$\LumaHeight$, 
all values $\CurrentPicture[Y][y][x]$ set to 0
\item $\CurrentPicture[U]$ and $CurrentPicture[V]$ are 2-dimensional arrays of width $\ChromaWidth$ and height $\ChromaHeight$, 
all values $\CurrentPicture[U][y][x]$ and $\CurrentPicture[V][y][x]$ set to 0
\end{itemize}

\subsubsection{Initialisation}
\label{picturedecodinginit}

The process for decoding pictures within a Dirac sequence can commence 
once an Access Unit header has been
located and parsed, and the default parameters set. This is achieved by 
searching for a Parse Info header within the sequence for which
$is\_au()$ returns $\true$ and then initialising the default state parameters
 by parsing the subsequent Access Unit header as per Section \ref{auheader}.

At this point the reference picture buffer shall be initialised with no reference pictures.

This initialisation process need only occur once within a sequence, since
apart from the AU picture number, all AU headers within a sequence are
identical. 

\subsubsection{Decoding process}
\label{picturedecprocess}

The process for decoding a picture with picture number 
$n$ in a Dirac sequence consists of:

\begin{pseudo}{decode\_pic}{n}
\bsCODE{seek\_picture\_parse(n)}{\ref{seeking}}
\bsCODE{picture()}{\ref{picture}}
\bsCODE{init\_picture\_data()}{\ref{picturedecodinginit}}
\bsIF{\ZeroResidual==\false}
    \bsCODE{\CurrentPicture[Y]=idwt(\YTransform)}{\ref{idwt}}
    \bsCODE{\CurrentPicture[U]=idwt(\UTransform)}{\ref{idwt}}
    \bsCODE{\CurrentPicture[V]=idwt(\VTransform)}{\ref{idwt}}
\bsEND
\bsIF{is\_inter()}
    \bsCODE{ref1=get\_ref(\RefOneNum)}
    \bsIF{num\_refs()==2}{\ref{parseinfo}}
        \bsCODE{ref2=get\_ref(\RefTwoNum)}
    \bsEND
    \bsCODE{motion\_compensate(ref1[Y], ref2[Y],  \CurrentPicture[Y], c)}{\ref{motioncompensate}}
    \bsCODE{motion\_compensate(ref1[U], ref2[U],  \CurrentPicture[U], c)}{\ref{motioncompensate}}
    \bsCODE{motion\_compensate(ref1[V], ref2[V],  \CurrentPicture[V], c)}{\ref{motioncompensate}}
\bsEND
\bsCODE{\clip\_picture()}{\ref{pictureclip}}
\bsCODE{ref\_buffer\_remove()}{\ref{refbuffer}}
\bsIF{is\_ref()}
    \bsCODE{ref\_buffer\_add()}{\ref{refbuffer}}
\bsEND
\bsCODE{output\_current\_picture()}{\ref{picoutput}}
\end{pseudo}

When randomly accessing a sequence, a picture may not be decodeable because reference pictures
may not be available in the buffer. In this case the current picture may be discarded, although some
decoders may be designed to produce an output. 

A Dirac sequence shall be so constructed so that if
pictures are decoding commences from the beginning of the stream and pictures are decoded in 
stream order, there shall be no undecodeable pictures i.e. the reference pictures associated with
any picture in the sequence stall have occurred prior to that picture in the sequence.

Picture numbers within the stream may not be in numerical, and subsequent reordering may be
required: the size of the decoded picture buffer required to perform any such reordering is specified
as part of the application profile and level (Appendix \ref{profilelevel}).

\subsection{Seeking in the Dirac stream}
\label{seeking}

The $seek\_picture\_parse(n)$ locates and parses a Parse Info header within the Dirac sequence for
which $is\_picture()$ is $\true$ and $\PictureNumber==n$. If pictures are being decoded sequentially
from the stream, there is no need to search for parse codes. The parse offset values may be used
to make the seeking process greatly more efficient, so that only a single parse code need be located
initially (Section \ref{parsecodes})

\subsection{Reference picture buffer management}
\label{refbuffer}

This section specifies how the Dirac stream data is used to manage the reference 
picture buffer $\RefBuffer$. The reference picture buffer has a maximum size of
$\RefBufferSize$ elements, as set in the applicable level (Appendix \ref{profilelevel}).

The $ref\_picture\_remove()$ process operates as
follows:

\begin{pseudo}{ref\_picture\_remove}{}
\bsFOR{i=0}{\length(\RetiredPictureList)-1}
    \bsCODE{n=\RetiredPictureList[i]}
    \bsFOR{k=0}{\RefBufferSize-1}
       \bsIF{\RefBuffer[k][pic\_number]==n}
            \bsFOR{j=k}{\RefBufferSize-1}
                \bsCODE{\RefBuffer[j]=\RefBuffer[j+1]}
            \bsEND
        \bsEND
    \bsEND
\bsEND
\bsCODE{\RetiredPictureList=\emptyset}
\end{pseudo}

The $get\_ref(n)$ function returns the (first) reference picture in the buffer with 
picture number $n$.  

The $ref\_picture\_add()$ process for adding pictures to the reference picture
buffer proceeds according to the following rules:

{\bf Case 1.} If the reference picture buffer is not full i.e. has fewer than $\RefBufferSize$ elements,
then add the $\CurrentPicture$ data to the end of the buffer. 

{\bf Case 2.} If the reference picture is full i.e. it has $\RefBufferSize$ elements, then remove the
first (i.e. oldest) element of the buffer, $\RefBuffer[0]$, set
\[\RefBuffer[i] = \RefBuffer[i+1] \]
for $i=0$ to $\RefBufferSize-2$, and set the last element $\RefBuffer[\RefBufferSize-1]$ equal to
a copy of $\CurrentPicture$.
 
\subsection{Clipping}
\label{pictureclip}

Picture data must be clipped to the video range prior to being output or being
used as a reference:

\begin{pseudo}{clip\_picture}{}
\bsFOREACH{c}{Y,U,V}
    \bsCODE{clip\_comp(\CurrentPicture[c])}
\bsEND
\end{pseudo}

\begin{pseudo}{clip\_component}{comp\_data}
\bsFOR{y=0}{\height(comp\_data)-1}
    \bsFOR{x=0}{\width(comp\_data)-1}
        \bsCODE{data = \clip(comp\_data[y][x], 0, 2^\VideoDepth-1)}
     \bsEND
\bsEND
\end{pseudo}

\subsection{Picture output}
\label{picoutput}

Once decoded, a copy of the current picture is added to the end of the decoded picture buffer
$\DecodedBuffer$ for display purposes.
