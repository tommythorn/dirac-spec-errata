%%%%%%%%%%%%%%%%%%%%%%%%%%%%%%%%%%%%%%%%%%%%%%%%%
% - This chapter defines the overall process  - %
% - for decoding a picture                    - % 
%%%%%%%%%%%%%%%%%%%%%%%%%%%%%%%%%%%%%%%%%%%%%%%%%

\label{picturedec}

This section defines the processes for decoding a picture from a Dirac stream. 
Picture decoding depends upon correctly parsing the stream, and decoding operations
 are dependent on decoding the sequence header and picture metadata 
(Section \ref{sequenceheader} and \ref{picturesyntax}) and unpacking the coefficient 
and motion data (Sections \ref{wltunpacking} and \ref{motiondec}).

\subsection{Overall picture decoding process}
\label{overallpicturedec}

Picture data from the current picture being decoded is stored in the $\CurrentPicture$ state
variable, which is a map with labels $\PicNum$, and $Y$, $C1$ and $C2$ representing
luma and chroma data.

After decoding the decoded picture is returned to the decoding application.

The $picture\_decode()$ process shall be invoked after parsing the $picture\_parse()$ process and shall be defined as follows:

\begin{pseudo}{picture\_decode}{}
\bsCODE{\CurrentPicture=\{\}}
\bsCODE{\CurrentPicture[\PicNum]=\PictureNumber}
\bsIF{is\_ref(()}
    \bsCODE{ref\_buffer\_remove()}{\ref{refbuffer}}
\bsEND
\bsIF{\ZeroResidual==\false}
    \bsCODE{inverse\_wavelet\_transform()}{\ref{idwt}}
\bsELSE
    \bsCODE{\CurrentPicture[Y]=\bf{0}}
    \bsCODE{\CurrentPicture[C1]=\bf{0}}
    \bsCODE{\CurrentPicture[C2]=\bf{0}}
\bsEND
\bsIF{is\_inter()}
    \bsCODE{ref1=get\_ref(\RefOneNum)}{\ref{refbuffer}}
    \bsIF{num\_refs()==2}{\ref{parsecodevalues}}
        \bsCODE{ref2=get\_ref(\RefTwoNum)}{\ref{refbuffer}}
    \bsEND
    \bsCODE{motion\_compensate(ref1[Y], ref2[Y],  \CurrentPicture[Y], c)}{\ref{motioncompensate}}
    \bsCODE{motion\_compensate(ref1[C1], ref2[C1],  \CurrentPicture[C1], c)}{\ref{motioncompensate}}
    \bsCODE{motion\_compensate(ref1[C2], ref2[C2],  \CurrentPicture[C2], c)}{\ref{motioncompensate}}
\bsEND
\bsCODE{clip\_picture()}{\ref{pictureclip}}
\bsIF{is\_ref()}
    \bsCODE{ref\_buffer\_add()}{\ref{refbuffer}}
\bsEND
\bsCODE{offset\_output\_picture(\CurrentPicture)}{\ref{videooutput}}
\bsRET{\CurrentPicture}
\end{pseudo}


\subsection{Picture re-ordering}
\label{picturereordering}
Picture numbers within the stream may not be in numerical order, and 
subsequent reordering may be required: the size (reordering depth) 
of the decoded picture buffer required to perform any such reordering may be 
specified as part of the application profile and level (Annex \ref{profilelevel}).
 
\subsection{Random access}
\label{randomaccess}
Sequence headers represent safe entry points for decoding a sequence. 

An accessible picture (with reference to a given sequence header)
shall be defined as a picture decodeable without dependence on 
to data prior to the sequence header in coded order.

Accessibility should normally imply that each accessible picture has
no reference picture prior to the sequence header, and no chain of 
references leading to a reference picture prior to the sequence header. 
A given level may allow this condition to
be relaxed (for example, in P-only coding where unavailable references
may be substituted for by zero pictures), but where no specific provision
to the contrary is specified in an applicable level or profile, it shall 
apply.

The first picture data unit after a sequence header shall be called the 
access picture and shall be accessible with respect to the sequence header.
It should normally be an intra picture. If the sequence contains inter pictures
it should normally be an intra reference picture.

All picture data units subsequent to the sequence header in coded order 
shall also be accessible with respect to the sequence header
if their picture numbers are greater than 
or equal to that of the access picture. The access picture therefore represents 
a temporal access point into the sequence.

\begin{informative}

If a sequence satisfies a maximum reordering depth constraint 
(Section \ref{picturereordering})
of size $N$ all pictures more than $N$ pictures later than the sequence header will
have larger picture numbers than the first picture after the sequence header, and hence will be accessible. A reordering depth constraint thus implies that after a 
sequence header at most $N$ pictures will need to be discarded before all pictures are decodeable.
\end{informative}
\subsection{Reference picture buffer management}
\label{refbuffer}

This section specifies how the Dirac stream data shall be used to manage the reference 
picture buffer $\RefBuffer$. The reference picture buffer has a maximum size of
$\RefBufferSize$ elements, as set in the applicable level (Annex \ref{profilelevel}).

The $ref\_picture\_remove()$ process shall be defined as
follows:

\begin{pseudo}{ref\_picture\_remove}{}
\bsCODE{n=\RetiredPicture}
\bsFOR{k=0}{\RefBufferSize-1}
   \bsIF{\RefBuffer[k][\PicNum]==n}
        \bsFOR{j=k}{\RefBufferSize-2}
            \bsCODE{\RefBuffer[j]=\RefBuffer[j+1]}
        \bsEND
        \bsCODE{\RefBufferSize -= 1}
    \bsEND
\bsEND
\end{pseudo}

The $get\_ref(n)$ function shall returns the reference picture in the buffer with 
picture number $n$.  If there is no such picture it shall return an all-zero picture.

The $ref\_picture\_add()$ process for adding pictures to the reference picture
buffer shall proceed according to the following rules:

{\bf Case 1.} If the reference picture buffer is not full i.e. has fewer than $\MaxRefBufferSize$ elements,
then add $\CurrentPicture$ to the end of the buffer. 

{\bf Case 2.} If the reference picture is full i.e. it has $\MaxRefBufferSize$ elements, then remove the
first (i.e. oldest) element of the buffer, $\RefBuffer[0]$, set
\[\RefBuffer[i] = \RefBuffer[i+1] \]
for $i=0$ to $\RefBufferSize-2$, and set the last element $\RefBuffer[\RefBufferSize-1]$ equal to
a copy of $\CurrentPicture$.

%%%%%%%%%%%%%%%%%%%%%%%%%%%%%%%%%%%%%%%%%%%%%%%%%%%%%
% - This chapter defines how the inverse discrete - %
% - wavelet transform is done                     - %
%%%%%%%%%%%%%%%%%%%%%%%%%%%%%%%%%%%%%%%%%%%%%%%%%%%%%


\subsection{Picture IDWT}

The inverse discrete wavelet transform process shall consist of transforming the 
wavelet coefficients for each of the video components. It shall be defined as follows:

\begin{pseudo}{inverse\_wavelet\_transform}{}
\bsCODE{\CurrentPicture[Y]=idwt(\YTransform)}{\ref{idwt}}
\bsCODE{\CurrentPicture[C1]=idwt(\COneTransform)}{\ref{idwt}}
\bsCODE{\CurrentPicture[C2]=idwt(\CTwoTransform)}{\ref{idwt}}
\bsFOREACH{c}{Y,C1,C2}
    \bsCODE{idwt\_pad\_removal(\CurrentPicture[c],c)}{\ref{padremoval}}
\end{pseudo}

\subsection{Component IDWT}
\label{idwt}

This section defines the $idwt(coeff\_data)$ process for reconstructing picture 
component data from decoded subband data $coeff\_data$ using the inverse discrete wavelet transform (IDWT). The IDWT shall be invoked in the picture decoding process only after successful unpacking of the subband coefficient data (Section \ref{transformdec}).

The IDWT process shall return a pixel array from the subband wavelet coefficients representing a reconstructed video component (Y, C1 or C2) for a single picture.


Since wavelet filtering operates on both rows and columns of two-dimensional arrays
 independently it is useful to define operators $\row(a,k)$ and $\column(a,k)$ for 
extracting rows and columns with index $k$ from a 2-dimensional array $a$:

If $b=\row(a,k)$ then $b[r]$ is a {\em reference} to the value of $a[k][r]$. This means that modifying the
value of $b[r]$ modifies the value of $a[k][r]$.

If $b=\column(a,k)$ then $b[r]$ is a {\em reference} the value of $a[r][k]$. This means that modifying the
value of $b[r]$ modifies the value of $a[r][k]$.

The $idwt()$ process shall be an iterative procedure operating on four subbands 
(\LL, \HL,\LH and \HH) at each iteration stage to produce a new subband. The procedure
shall be as follows

\begin{pseudo}{idwt}{coeff\_data}
\bsCODE{LL\_band = coeff\_data[0][\LL]}
\bsFOR{n=1}{\TransformDepth}
   \bsCODE{ 
                   new\_LL= vh\_synth(LL\_band, coeff\_data[n][\HL], coeff\_data[n][\LH], 
                   coeff\_data[n][\HH])
                   }{\ref{vhsynth}}
   \bsCODE{LL\_band=new\_LL}
\bsEND
\bsRET{LL\_band}
\end{pseudo}

Note that at each stage, the input dimensions of the input $LL\_band$ will be the same 
as those of the other input bands, whereas the output dimensions are double those of the input bands.

\subsubsection{Vertical and horizontal synthesis}
\label{vhsynth}

This section specifies the operation of the vertical and horizontal
synthesis process:

$vh\_synth(LL\_data, HL\_data, LH\_data, HH\_data)$

$vh\_synth$ shall return an array of twice the dimensions of each of the input
argument arrays. 

$vh\_synth$ is repeatedly invoked by the IDWT synthesis process and operates on four subband data arrays of identical dimensions to produce a new array $synth$, 
which shall be returned as the result of the process.

{\bf Step 1.} $synth$ is a temporary two-dimensional array that shall be initialised so that:
\begin{eqnarray*}
\width(synth) & = & 2*\width(LL\_data) \\
\height(synth) & = & 2*\height(LL\_data)
\end{eqnarray*}

{\bf Step 2.} The data from the four arrays shall be interleaved as follows:

\begin{pseudo*}
\bsFOR{y=0}{(\height(synth)//2)-1}
    \bsFOR{x=0}{(\width(synth)//2)-1}
        \bsCODE{synth[2*y][2*x] = LL\_data[y][x]}
        \bsCODE{synth[2*y][2*x+1] = HL\_data[y][x]}
        \bsCODE{synth[2*y+1][2*x] = LH\_data[y][x]}
        \bsCODE{synth[2*y+1][2*x+1] = HH\_data[y][x]}
    \bsEND
\bsEND
\bsCODE{\hdots}
\end{pseudo*}

Note: This enables in-place calculation during the inverse filter process.

{\bf Step 3.} Data shall be synthesised vertically by operating on each column
of data using a one-dimensional filter, and then horizontally by operating
on each row. The one-dimensional filters used shall be determined by
the value of $\WaveletIndex$ according to Tables \ref{filtertable0}--\ref{filtertable6}. 
The process shall be as follows:

\begin{pseudo*}
\bsFOR{x=0}{\width(synth)-1}
    \bsCODE{1d\_synthesis(\column(synth, x) )}{\ref{onedsynth}}
\bsEND
\bsFOR{y=0}{\height(synth)-1}
    \bsCODE{1d\_synthesis(\row(synth, y) )}{\ref{onedsynth}}
\bsEND
\bsCODE{\hdots}
\end{pseudo*}

{\bf Step 4.} Finally, the synthesised subband data shall implement a bitshift to
remove any accuracy bits. The bit shift value $filtershift()$ used shall be determined 
by the value of  $\WaveletIndex$ according to Tables \ref{filtertable0}--\ref{filtertable6}. 
The process shall be as follows:

\begin{pseudo*}
\bsCODE{shift = filtershift()}
\bsIF{shift>0}
    \bsFOR{y=0}{\height(synth)-1}
        \bsFOR{x=0}{\width(synth)-1}
            \bsCODE{synth[y][x] = (synth[y][x] + (1<<(shift-1)))\gg shift}
         \bsEND
    \bsEND
\bsEND
\end{pseudo*}

\begin{informative}
Accuracy bits are added in the encoder by shifting up all coefficients in the LL band 
prior to applying any filtering (this includes an initial shift of all values in the component
 data). Adding a small shift before each decomposition stage is the most efficient way of
 providing additional resolution mitigating aliasing through non-linear rounding effects.
\end{informative}

\subsubsection{One-dimensional synthesis}
\label{onedsynth}

This section specifies the one-dimensional synthesis process
$1d\_synthesis()$, which shall apply to a 1-dimensional array of coefficients 
of even length, consisting of either a row or a column of a 2-dimensional integral data array.

The one-dimensional synthesis process shall comprise the application of a
number of reversible integer lifting filter operations. 

Lifting filtering operations shall be one of four types, Type 1, Type 2, Type 3 and 
Type 4. Each type shall be characterised by four elements:
\begin{itemize}
\item a filter length value $L$
\item a filter offset value $D$
\item an array of taps of length $L$: $taps[0],\ldots,taps[L-1]$
\item a scale factor $S$
\end{itemize}

The four types of lifting operations shall be defined by the functions:
\begin{description}
\item[] $lift1(A,L,D,taps,S)$,
\item[] $lift2(A,L,D,taps,S)$,
\item[] $lift3(A,L,D,taps,S)$, and
\item[] $lift4(A,L,D,taps,S)$.
\end{description}
respectively and shall act upon the values in a one-dimensional array $A$.

The Type 1 lifting process $lift1(A,L,D,taps,S)$ shall be defined as follows:

\begin{pseudo}{lift1}{A,L,D,taps,S}
\bsFOR{n=0}{(\length(A)//2)-1}
    \bsCODE{sum=0}
    \bsFOR{i=D}{L+D-1}
        \bsCODE{pos=2*(n+i)-1}
        \bsCODE{pos=\min(pos, \length(A)-1)}
        \bsCODE{pos=\max(pos, 1)}
        \bsCODE{sum+=taps[i-D]*A[pos]}
    \bsEND
    \bsIF{S>0}
        \bsCODE{sum+=(1\ll (S-1))}
    \bsEND
    \bsCODE{A[2*n]+=(sum\gg S)}
\bsEND
\end{pseudo}

The Type 2 lifting process $lift2(A,L,D,taps,S)$ shall be defined as follows:

\begin{pseudo}{lift2}{A,L,D,taps,S}
\bsFOR{n=0}{(\length(A)//2)-1}
    \bsCODE{sum=0}
    \bsFOR{i=D}{L+D-1}
        \bsCODE{pos=2*(n+i)-1}
        \bsCODE{pos=\min(pos, \length(A)-1)}
        \bsCODE{pos=\max(pos, 1)}
        \bsCODE{sum+=taps[i-D]*A[pos]}
    \bsEND
    \bsIF{S>0}
        \bsCODE{sum+=(1\ll (S-1))}
    \bsEND
    \bsCODE{A[2*n]-=(sum\gg S)}
\bsEND
\end{pseudo}

The Type 3 lifting process $lift3(A,L,D,taps,S)$ shall be defined as follows:

\begin{pseudo}{lift3}{A,L,D,taps,S}
\bsFOR{n=0}{(\length(A)//2)-1}
    \bsCODE{sum=0}
    \bsFOR{i=D}{L+D-1}
        \bsCODE{pos=2*(n+i)}
        \bsCODE{pos=\min(pos, \length(A)-2)}
        \bsCODE{pos=\max(pos, 0)}
        \bsCODE{sum+=taps[i-D]*A[pos]}
    \bsEND
    \bsIF{S>0}
        \bsCODE{sum+=(1\ll (S-1))}
    \bsEND
    \bsCODE{A[2*n+1]+=(sum\gg S)}
\bsEND
\end{pseudo}

The Type 4 lifting process $lift4(A,L,D,taps,S)$ shall be defined as follows:

\begin{pseudo}{lift4}{A,L,D,taps,S}
\bsFOR{n=0}{(\length(A)//2)-1}
    \bsCODE{sum=0}
    \bsFOR{i=D}{L+D-1}
        \bsCODE{pos=2*(n+i)}
        \bsCODE{pos=\min(pos, \length(A)-2)}
        \bsCODE{pos=\max(pos, 0)}
        \bsCODE{sum+=taps[i-D]*A[pos]}
    \bsEND
    \bsIF{S>0}
        \bsCODE{sum+=(1\ll (S-1))}
    \bsEND
    \bsCODE{A[2*n+1]-=(sum\gg S)}
\bsEND
\end{pseudo}

$1d\_synthesis$ shall apply the sequence of lifting filters specified in Section \ref{wltfilters}
corresponding to the value of $\WaveletIndex$,and shall invoke the corresponding lifting processes with the parameters defined.

\paragraph{Mathematical formulation of lifting processes (Informative)}
$\ $\newline

The lifting processes defined in the previous section are extremely similar, and 
careful attention should be paid to the detail of their operation in any implementation. 
The four different variants arise from two factors: the �phase� (odd or
even) of the lifting operation, and their implementation using integer-only 
operations, which introduces rounding errors and makes addition and subtraction 
subtly different. 

A lifting operation either modifies the odd coefficients by a linear combination of the 
even coefficients, or vice-versa. Mathematically, the four types of filter may be 
described as follows.

Type 1 and Type 2 lifting filtering operations modify the even coefficients
by the odd coefficients:
\begin{eqnarray*}
  A[2*n]& +=& \left( \sum^M_{i=-N} t_i *A[2*(n+i) + 1] +(1\ll (s-1))\right) \gg s \mbox{ (Type 1)} \\
  A[2*n]& -=& \left( \sum^M_{i=-N} t_i *A[2*(n+i) + 1] +(1\ll (s-1))\right) \gg s \mbox{ (Type 2)}
\end{eqnarray*}

Type 3 and Type 4 lifting filtering operation modify the odd coefficients
 by the even coefficients:
\begin{eqnarray*}
  A[2*n+1]& +=&  \left( \sum^M_{i=-N} t_i A[2*(n+i)]+(1\ll (s-1)) \right) \gg s \mbox{ (Type 3)} \\
  A[2*n+1]& -=&  \left( \sum^M_{i=-N} t_i A[2*(n+i)] +(1\ll (s-1))\right) \gg s \mbox{ (Type 4)} \\
\end{eqnarray*}

The distinctions between Type 1 and Type 2 and between 
Type 3 and Type 4 filters are necessary
because integer division (bit-shifting) is being used, and so the filters are non-linear:
a Type 1 or Type 3 filter with taps $t_i$ is {\em not} equivalent to 
an Type 2 or Type 4 filter with taps $-t_i$.

Edge extension is used where the filter would otherwise extend beyond the 
boundaries of the array. This is slightly different between Types 1 and 2 on the 
one hand and Types 3 and 4 on the other. This is because even values and
odd values must be extended separately to maintain the correct phase (and hence invertibility of the filter). For example, a Type 1 filter must use the values 1 and $\length(A)-1$ at the edges because (as the length is even) these are the odd values nearest the edges.

Further information on wavelet filtering and lifting is provided in Annex 
\ref{transformandlifting}.

\subsubsection{Lifting filter parameters}
\label{wltfilters}

The lifting filters and filter bit-shift operations that apply for each value
$\WaveletIndex$ shall be as specified in Tables  \ref{filtertable0} to \ref{filtertable6}
below.

\begin{table}[h]
\begin{tabular}{|l|}

\hline
Lifting steps:  \\
\begin{tabular}{l}
1. Type 2, $L=2, D=0, taps=[1,1], S=2$ i.e. \\
\quad $ A[2*n]     -=  (A[2*n-1]+A[2*n+1]+2)\gg 2$ \\
2. Type 3, $L=4, D=-1, taps=[-1,9,9,-1], S=4$ i.e. \\
\quad $ A[2*n+1]  +=  (-A[2*n-2]+9*A[2*n]+9*A[2*n+2]-A[2*n+4]+8)\gg 4$
\end{tabular}
\\
\\
$filtershift()$ returns 1\\
\hline

\end{tabular}
\caption{$\WaveletIndex==0$: Deslauriers-Dubuc (9,7) lifting stages and shift values}\label{filtertable0}
\end{table}

\begin{table}[h]

\begin{tabular}{|l|}
\hline
Lifting steps: \\

\begin{tabular}{l}
1. Type 2, $L=2, D=0, taps=[1,1], S=2$ i.e. \\
\quad $ A[2*n]  -= (A[2*n-1]+A[2*n+1]+2)\gg 2$ \\
2. Type 3, $L=2, D=0, taps=[1,1], S=1$ i.e. \\
\quad $ A[2*n+1]  += (A[2*n]+A[2*n+2]+1)\gg 1$
\end{tabular}
\\
\\
$filtershift()$ returns 1\\
\hline

\end{tabular}
\caption{$\WaveletIndex==1$: LeGall (5,3) lifting stages and shift values}
\end{table}

\begin{table}[h]
\begin{tabular}{|l|}

\hline
Lifting steps: \\
\begin{tabular}{l}
1. Type 2, $L=4, D=-1, taps=[-1,9,9,-1], S=5$ i.e. \\
\quad $ A[2*n]  -= (-A[2*n-3]+9*A[2*n-1]+9*A[2*n+1]-A[2*n+3]+16)\gg 5$ \\
2. Type 3, $L=4, D=-1, taps=[-1,9,9,-1], S=4$ i.e. \\
\quad $ A[2*n+1]  += (-A[2*n-2]+9*A[2*n]+9*A[2*n+2]-A[2*n+4]+8)\gg 4$
\end{tabular}
\\
\\
$filtershift()$ returns 1\\
\hline

\end{tabular}
\caption{$\WaveletIndex==2$: Deslauriers-Dubuc (13,7) lifting stages and shift values}
\end{table}

\begin{table}[h]
\begin{tabular}{|l|}

\hline
Lifting steps:\\
\begin{tabular}{l}
1. Type 2, $L=1, D=1, taps=[1], S=1$ i.e. \\
\quad $A[2*n] -=  (A[2*n+1]+1)\gg 1$ \\
2. Type 3, $L=1, D=0, taps=[1], S=0$ i.e. \\
\quad $ A[2*n+1] += A[2*n]$
\end{tabular}
\\
\\
$filtershift()$ returns 0\\
\hline

\end{tabular}
\caption{$\WaveletIndex==3$: Haar filter with no shift}
\end{table}

\begin{table}[h]
\begin{tabular}{|l|}

\hline
Lifting steps:\\
\begin{tabular}{l}
1. Type 2, $L=1, D=1, taps=[1], S=1$ i.e. \\
\quad $A[2*n] -=  (A[2*n+1]+1)\gg 1$ \\
2. Type 3, $L=1, D=0, taps=[1], S=0$ i.e. \\
\quad $ A[2*n+1] += A[2*n]$
\end{tabular}
\\
\\
$filtershift()$ returns 1\\
\hline

\end{tabular}
\caption{$\WaveletIndex==4$: Haar filter with single shift}
\end{table}

\begin{table}[h]
\begin{tabular}{|l|}
\hline
Lifting steps:\\
\begin{tabular}{l}
1. Type 3, $L=8, D=-3, taps=[]-2,10,-25,81,81,-25,10,-2], S=8$ i.e. \\
$ \begin{array}{rcl} A[2*n+1] & += & (-2*(A[2*n-6]+A[2*n+8])+10*(A[2*n-4]+A[2*n+6])\\
                                             & &       -25*(A[2*n-2]+A[2*n+4])+81*(A[2*n]+A[2*n+2])+128)\gg 8 \end{array}$ \\
1. Type 2, $L=8, D=-3, taps=[-8,21,-46,161,161,-46,21,-8], S=8$ i.e. \\
 $ \begin{array}{rcl}A[2*n]  & -= & (-8*(A[2*n-7]+A[2*n+7])+21*(A[2*n-5]+A[2*n+5]) \\
                                          & &      -46*(A[2*n-3]+A[2*n+3])+161*(A[2*n-1]+A[2*n+1])+128)\gg 8 \end{array}$ 
\end{tabular}
\\
\\
$filtershift()$ returns 0\\
\hline

\end{tabular}
\caption{$\WaveletIndex==5$: Fidelity filter for improved downconversion and anti-aliasing}
\end{table}

\begin{table}[h]
\begin{tabular}{|l|}

\hline
Lifting steps:\\
\begin{tabular}{l}
1. Type 2, $L=2, D=0, taps=[1817,1817], S=12$ i.e. \\
\quad $A[2*n] -=  (1817*A[2*n-1]+1817*A[2*n+1]+2048)\gg 12$ \\
2. Type 4, $L=2, D=0, taps=[3616,3616], S=12$ i.e. \\
\quad $A[2*n+1] -= (3616*A[2*n]+3616*A[2*n+2]+2048)\gg 12$ \\
3. Type 1, $L=2, D=0, taps=[217,217], S=12$ i.e. \\
\quad $A[2*n] +=  (217*A[2*n-1]+217*A[2*n+1]+2048)\gg 12$ \\
4. Type 3, $L=2, D=0, taps=[6497,6497], S=12$ i.e. \\
\quad $ A[2*n+1] += (6497*A[2*n]+6497*A[2*n+2]+2048)\gg 12$
\end{tabular}
\\
\\
$filtershift()$ returns 1\\
\hline

\end{tabular}
\caption{$\WaveletIndex==6$: Integer lifting approximation to Daubechies (9,7)}\label{filtertable6}
\end{table}
\clearpage
\subsection{Removal of IDWT pad values}
\label{padremoval}

This section defines the decoding process $idwt\_pad\_removal(pic, c)$
for removing extraneous values after performing the IDWT.

Section \ref{wltinit} requires that subband coefficient data arrays are padded to ensure that 
the reconstructed data array $pic$ has dimensions divisible by $2^\TransformDepth$.

Values $width$ and $height$ are defined to be the appropriate dimensions
of the component data:

\begin{itemize}
\item If $c=Y$, then
\begin{eqnarray*}
width & =& \LumaWidth \\
height & =& \LumaHeight
\end{eqnarray*}
\item else if $c=C1$ or $c=C2$,
\begin{eqnarray*}
width & =& \ChromaWidth \\
height & =& \ChromaHeight
\end{eqnarray*}
\end{itemize}

All component data $pic[j][i]$ with

\begin{itemize}
\item $i\geq width$, or
\item $j\geq height$
\end{itemize}

shall be discarded and $pic$ shall be resized to have width $width$ and height $height$.



\subsection{Motion compensation}
\label{motioncompensate}

This section defines the operation of the process
$motion\_compensate(ref1, ref2,  pic, c)$ for motion-compensating a
picture component array  $pic$ of type $c=Y, U$ or $V$ from reference 
component arrays $ref1$ and $ref2$ of the same type.

This process is invoked for each component in a picture, subsequent to the 
decoding of coefficient data, specified in Section \ref{transformdec}, and the Inverse Wavelet 
Transform (IWT), specified in Section \ref{idwt}. 

\subsection{Definitions and conventions}

Motion compensation uses the motion block data $\BlockData$ and (optionally) the
global motion parameters $\GlobalParams$.

Since $motion\_compensate()$ applies to both luma and (potentially subsampled)
chroma data, for simplicity a number of local variables are defined. If $c=Y$ then:
\begin{eqnarray*}
lenX & = & \LumaWidth \\
lenY & = & \LumaHeight \\
xblen & = & \LumaXBlen \\
yblen & = & \LumaYBlen \\
xbsep & = & \LumaXBsep \\
ybsep & = & \LumaYBsep
\end{eqnarray*}

If $c=U$ or $c=V$, then likewise:
\begin{eqnarray*}
lenX & = & \ChromaWidth \\
lenY & = & \ChromaHeight \\
xblen & = & \ChromaXBlen \\
yblen & = & \ChromaYBlen \\
xbsep & = & \ChromaXBsep \\
ybsep & = & \ChromaYBsep
\end{eqnarray*}

Define the offsets $xoffset, yoffset$ by
\begin{eqnarray*}
xoffset & = & (xblen-xbsep)//2 \\
yoffset & = & (yblen-ybsep)//2
\end{eqnarray*}

\begin{informative}
The subband data that makes up the IWT coefficients is padded in order that the IWT
may function correctly. For simplicity, in this specification, padding data is removed
after the IWT has been performed so that the picture data and reference data arrays have
the same dimensions for motion compensation. However, it may be more efficient to perform
all operations prior to the output of pictures using padded data, i.e. to discard padding values
subsequent to motion compensation. Such a course of action is equivalent, so long as it is realised
that blocks must be regarded as edge blocks if they overlap the actual picture area, not the
larger area produced by padding. The specification of this section fully supports such an 
interpretation.
\end{informative}

Throughout this Section, the following conventions are used.

\begin{itemize}
\item $x,y$ are co-ordinates in the predicted picture component
\item $u,v$ are co-ordinates in a potentially upconverted reference picture component
\end{itemize}

\begin{comment}
[Are they??]
\end{comment}

\begin{informative*}
\subsection{Overlapped Block Motion Compensation (OBMC) (Informative)}

Motion compensated prediction methods provide methods for determining 
predictions for pixels in the current picture by using motion vectors to 
define offsets from those pixels to pixels in previously decoded
pictures. Motion compensation techniques vary in how those pixels are grouped
together, and how a prediction is formed for pixels in a given group. In 
conventional  block motion compensation, as used in MPEG2, H.264 and many other
codecs, the picture is divided into {\em disjoint} rectangular blocks and the
motion vector or vectors associated with that block defines the offset(s) into
the reference pictures.

In OBMC, by contrast, the predicted picture is divided into a regular overlapping 
blocks of dimensions $xblen$ by $yblen$ that cover at least the entire picture 
area as shown in figure \ref{fig:blockcoverage}.  Overlapping is ensured by starting
each block at a horizontal separation $xbsep$ and a vertical separation $ybsep$ 
from its neighbours, where these values are less than the corresponding block dimensions.
\end{informative*}

\begin{figure}[!h]
\centering
\includegraphics[width=0.7\textwidth]{figs/block-coverage.eps}
\caption{Block coverage of the predicted picture}
\label{fig:blockcoverage}
\end{figure}

\begin{informative*}
The overlap between blocks horizontally is $xblen - xbsep$ and vertically is
$yblen - ybsep$. As a result pixels in the overlapping areas lie in more than
one block, and so more than one motion vector set (and set of associated predictions)
applies to them. Indeed, a pixel may have up to eight predictions, as it may belong to
up to four blocks, each of which may have up to two motion vectors. These are combined
into a single prediction by using weights, which are so constructed so as to sum to 1. In the
 Dirac integer implementation, fractional weights are achieved by insisting that weights sum 
to a power of 2, which is then shifted out once all contributions have been summed.

In Dirac blocks are positioned so that blocks will overspill the left and top edges by 
($xoffset$) and ($yoffset$) pixels.  The number of blocks has been
determined (Section ??) so that the picture area is wholly covered, and the overspill
 on the right hand and bottom edges will be at least the amount on the left and top edges. 
Indeed, the number of blocks has been set so that the blocks divide into whole superblocks
(sets of 4x4 blocks), which mean that some blocks may fall entirely out of the picture area. 
 Any predictions for pixels outside the picture area defined by $0 \leq x < lenX, 0 \leq y <lenY$
are discarded.

\end{informative*}

\subsection{Overall motion compensation process}
\label{mcprocess}

The motion compensation process forms an integer prediction $p[y][x]$ for each pixel in the predicted
picture component $pic$, and adds it to the component data. This is then clipped to keep it in range.
Note that this clipping is {\em in addition} to clipping performed on the output picture after motion
 compensation and/or the inverse wavelet transform (Section \ref{pictureclip}).

\begin{pseudo*}
\bsFOR{y=0}{lenY-1}
    \bsFOR{x=0}{lenX-1}
        \bsCODE{pic[y][x] += pixel\_predict(y, x, pic, ref1, ref2, c)}{\ref{pixelpredict}}
        \bsCODE{pic[y][x] = \clip(pic[y][x], 0, 2^\VideoDepth-1)}
    \bsEND
\bsEND
\end{pseudo*}

\subsection{Pixel prediction}
\label{pixelpredict}

In order to specify the $pixel\_predict()$ process, some definitions are required. For block indices $(i,j)$, 
define the set of elements $B(i,j)$ in the corresponding
block by:
\begin{eqnarray*}
xstart & = & \max(i*xbsep-xoffset, 0) \\
ystart & = & \max(j*xbsep-xoffset, 0) \\
xstop & = & \min\left( xstart+xblen, lenX\right)= \min\left( (i+1)*xbsep+xoffset, lenX\right)\\
ystop & = & \min\left( ystart+yblen, lenY\right)= \min\left( (j+1)*ybsep+yoffset, lenY\right)\\
B(i,j) & = & \{(x,y): xstart\leq x<xstop, ystart\leq y<ystop\}
\end{eqnarray*}

Define the total weight resolution $total\_wt\_bits$ as follows: 
\begin{eqnarray*}
hbits  & = & \log_2(xblen-xbsep)+1 = \log_2(xoffset)+2 \\
vbits  & = & \log_2(yblen-ybsep)+1 = \log_2(yoffset)+2 \\
total\_wt\_bits & = & hbits+vbits+\RefsWeightPrecision
\end{eqnarray*}

This is the number of bits added to pixel values in order to perform OBMC reversibly with integer arithmetic
using the spatial specified in Sections \ref{mcspatialweights} and the reference weights extracted in
parsing the picture prediction header data (Section \ref{refpicweights}).

The $pixel\_predict(y, x, ref1, ref2, c)$ function forms a prediction by adding together weighted predictions
for all blocks containing the pixel $(x,y)$. Weight contributions come both from a spatial matrix and from the
weights assigned to references:

\begin{pseudo}{pixel\_pred}{y, x, pic, ref1, ref2, c}
\bsCODE{p=0}
\bsFORSUCH{(i,j)}{(x,y)\in B(i,j)}
    \bsCODE{m=\BlockData[j][i][mode]}
    \bsIF{m==\Intra}
        \bsCODE{val=\BlockData[j][i][dc][c]}
        \bsCODE{val=val*2^\RefsWeightPrecision}
    \bsELSEIF{m==\RefOneOnly}
        \bsCODE{val=block\_pred(ref1, 1, i, j, x, y, c)}
        \bsCODE{val=val*(\RefOneWeight+\RefTwoWeight)}
    \bsELSEIF{m==\RefTwoOnly}
        \bsCODE{val=block\_pred(ref2, 2, i, j, x, y, c)}
        \bsCODE{val=val*(\RefOneWeight+\RefTwoWeight)}
   \bsELSE
        \bsCODE{val1=block\_pred(ref1, 1, i, j, x, y, c)}
        \bsCODE{val1=val1*\RefOneWeight}
        \bsCODE{val2=block\_pred(ref2, 2, i, j, x, y, c)}
        \bsCODE{val2=va2l*\RefTwoWeight}
    \bsEND{}
    \bsCODE{val=val*spatial\_wt(i,j,x,y)}{\ref{mcspatialweights}}
    \bsCODE{p=p+val}
\bsEND
\bsCODE{p=(p+2^{total\_wt\_bits-1})\gg total\_wt\_bits}
\bsRET{p}
\end{pseudo}


\begin{informative}

{\bf 1.} Note that the number of bits $total\_wt\_bits$ used for the OBMC weighting matrix depends upon the block sizes - specifically
the block overlaps - selected. A Dirac decoder level (\ref{levels}) specifies the maximum block overlaps allowable, and hence 
the word widths necessary for processing OBMC. If we assume that the picture weights are complementary (i.e. the weights
for reference 1 and reference 2 sum to $2^\RefsWeightPrecision$, then the number of bits required for performing 
motion compensation 
calculations is
\[\VideoDepth+total\_wt\_bits+\RefsWeightPrecision\]
unsigned bits. 8 bit video data encoded with block overlaps of 4 luminance pixels and the standard picture weights therefore
requires 8+3+3+1=15 unsigned bits. The additional bit within a 16 bit word could be used to provide additional reference 
weighting.


{\bf 2.} The motion compensation process has been presented as double loop: first over all 
pixels in a given component and second over all blocks of which the pixel is a member. However, if an intermediate
buffer is allowed, the loop order can be reversed. In this case one sets a picture buffer, consisting of 
a component of data (or a strip of component data lines) and add in the weighted predictions for each block. As the
blocks overlap, the contributions sum and form a prediction for every pixel. This is the most natural implementation
strategy:

\begin{pseudo*}
\bsCODE{b[\quad][\quad]=0}
\bsFOR{j=0}{\BlocksY-1}
    \bsFOR{i=0}{\BlocksX-1}
        \bsCODE{m=\BlockData[j][i][mode]}
        \bsFOREACH{(x,y)}{B(i,j)}
            \bsIF{m==\Intra}
                \bsCODE{wt=2^\RefsWeightPrecision * spatial\_wt(i,j,x,y)}
                \bsCODE{b[y][x]+=\BlockData[j][i][dc][c]*wt}
            \bsELSEIF{m==\RefOneOnly}
                \bsCODE{wt=(\RefOneWeight+\RefTwoWeight) * spatial\_wt(i,j,x,y)}
                \bsCODE{b[y][x]+=block\_pred(ref1, 1, i, j, x, y, c)*wt}
            \bsELSEIF{m==\RefTwoOnly}
                \bsCODE{wt=(\RefOneWeight+\RefTwoWeight) * spatial\_wt(i,j,x,y)}
                \bsCODE{b[y][x]+=block\_pred(ref2, 2, i, j, x, y, c)*wt}
            \bsELSE
                \bsCODE{wt=\RefOneWeight * spatial\_wt(i,j,x,y)}
                \bsCODE{b[y][x]+=block\_pred(ref1, 1, i, j, x, y, c)*wt}
                \bsCODE{wt=\RefTwoWeight * spatial\_wt(i,j,x,y)}
                \bsCODE{b[y][x]+=block\_pred(ref2, 2, i, j, x, y, c)*wt}
            \bsEND
        \bsEND
    \bsEND
\bsEND
\bsFOR{y=0}{lenY-1}
    \bsFOR{x=0}{lenX-1}
        \bsCODE{pic[y][x] += (b[y][x]+2^{total\_wt\_bits-1})\gg total\_wt\_bits}
        \bsCODE{pic[y][x] = \clip(pic[y][x], 0, 2^\VideoDepth-1)}
    \bsEND
\bsEND
\end{pseudo*}

The double multiplication by a spatial and by a reference weight can be avoided by using a set of spatial weighting matrices
pre-multiplied by the applicable reference weights according to prediction mode. In the default case where the reference
picture weights are one, as is the picture weight precision, this means a double-size spatial matrix for all modes other than
$\RefOneAndTwo$.

{\bf 3.} The reference prediction weights used for each prediction mode may appear confusing. It is helpful
to think of two cases for using reference picture weighting. The first is interpolative 
prediction, where the picture being predicted is, for example, a cross-fade and is
closely approximated by some mixture of the reference pictures:
 $P\backsimeq\delta R_1+(1-\delta)R_2$. Here the weights we'd like to
use for each frame prediction add up to 1 (or $2^\RefsWeightPrecision$ for integer weights). 
The second case is scaling prediction, where 
the weights we'd like to use for the frame predictions don't add up to 1: for example,
a fade to or from black
$P\backsimeq\delta_1 R_1$ and $P\backsimeq\delta_2 R_2$. It is not possible to choose 
weights for each prediction mode which will be optimal both cases. The weighting
factors chosen will give work with interpolative prediction (which is more common) 
but are not perfect for scaling prediction. It would have been possible to create a variety of
prediction modes to cover all cases, however the potential savings do not justify the
additional complexity.

For interpolative prediction, all data in the current picture will be of commensurate scale to
that of the references. In forming the bi-directional prediction, a value $W_1 p_1 + W_2 p2_2$ is 
formed, so the prediction has "scale" $W_1+W_2$. $W_1+W_2$ is 
therefore the weighting value used to scale unidirectional prediction, in order to provide
predictions of commensurate order. The unity weighting value $2^\RefsWeightPrecision$ is used
for DC blocks as this gives the best prediction, and in the interpolative case this equals $W_1+W_2$
so all predictions are of the same order.

The weighting factors we would like to use for unidirectionally predicted blocks in the scaling case
are $2W_1$ and $2W_2$ - the factor 2 takes into account that we're only adding in one prediction
value as against two for bidirectional prediction. These factors differ from $W_1+W_2$, and hence
unidirectional prediction is incorrect when there are two references. Note, however, that we can
still perform prediction with the correct scaling values when we only have a single reference. Note
also that the value of $W_1+W_2$ was selected instead of $2^\RefsWeightPrecision$, which
would be equivalent in the interpolative case, as it gives a better approximation when the
weights do not sum to $2^\RefsWeightPrecision$.
\end{informative}

\subsection{Spatial weighting matrix}

\label{mcspatialweights}

This section specifies the process $wt(i,j,x,y)$ for deriving a spatial weighting value for a pixel with
coordinates $(x,y)$ in the block with coordinates $(i,j)$. Note that other weights are applied
to the prediction as a result of the weights applied to each reference.

The two-dimensional spatial weighting matrix $W$ applies a linear roll-off in both horizontal and vertical directions
based on the position of the pixel $(x,y)$ within the block $B(i,j)$. Define $xpos, ypos$ as the relative
pixel coordinates from the top-left corner of the block:

\begin{eqnarray*}
xpos & = & x-(i*xbsep-xoffset) \\
ypos & = & x-(i*ybsep-yoffset) 
\end{eqnarray*}

Define a horizontal weighting array $WH$ by the recipe:

\providecommand{\abs}[1]{\left\lvert#1\right\rvert}
\begin{pseudo*}
\bsCODE{max\_x=2(xblen-xbsep)}
\bsIF{i==0 \text{ and } xpos< xblen//2}
    \bsCODE{WH[xpos]=max\_x}
\bsELSEIF{i==\BlocksX \text{ and } xpos\geq xblen//2}
    \bsCODE{WH[xpos]=max\_x}
\bsELSE
    \bsCODE{WH[xpos]  =  \clip(xblen-2\abs{xpos-\dfrac{(xblen-1)}{2}}, max\_x ) }
\bsEND
\end{pseudo*}

Likewise define $WV$ by

\begin{pseudo*}
\bsCODE{max\_y=2(yblen-ybsep)}
\bsIF{j==0 \text{ and } xpos< yblen//2}
    \bsCODE{WH[ypos]=max\_y}
\bsELSEIF{j==\BlocksY \text{ and } ypos\geq yblen//2}
    \bsCODE{WH[ypos]=max\_y}
\bsELSE
    \bsCODE{WV[ypos]  =  \clip(yblen - 2\abs{ypos-\dfrac{(yblen-1)}{2}},0, max\_y) }
\bsEND
\end{pseudo*}

The overall spatial weighting matrix $W$ is given by

\begin{equation*}
W[ypos][xpos] = WH[xpos][WV[ypos]
\end{equation*}

and this is the value returned.

Note that blocks at the extremities of the block set receive maximum weight around their outward-facing edges.
This is to compensate for the lack of blocks making weight contributions on these edges, and ensures that
the total contribution for the pixels in the blocks is $2^alpha$. In section ($i=0$), the profile of the matrix 
for interior blocks is:

\begin{figure}[!h]
\centering
\includegraphics[width=0.7\textwidth]{figs/obmc-profile}
\caption{Profile of overlapped-block motion compensation matrix}
\label{fig:weightprofile}
\end{figure}

\subsection{Block prediction}
\label{blockprediction}

This section specifies the operation of the $block\_pred(ref, ref\_num, i, j, x, y, c)$ 
process for forming a prediction for a pixel 
with coordinates $(x,y)$ in component $c$, belonging to the block with coordinates $(i,j)$.

{\bf Case 1: $\BlockData[j][i][global]=\false$}. In this case, the block motion vectors are used to form a prediction.
Motion vectors for chroma components must be scaled according to the chroma scale factors. If $c=Y$, set

\begin{equation*}
mv= \BlockData[j][i][ref]
\end{equation*}

whereas if $c=U$ or $c=V$, set

\begin{eqnarray*}
mv.x & = & \rounddivide(\BlockData[j][i][ref].x, chroma\_h\_factor() ) \\
mv.y & = & \rounddivide(\BlockData[j][i][ref].y, chroma\_v\_factor() )
\end{eqnarray*}

(chroma subsampling factors are as specified in Section \ref{chromaformats}.)

{\bf Case 2: $\BlockData[j][i][global]=\true$}. In this case, a motion vector is determined from the global
motion parameters as per Section \ref{globalmv}:
\begin{equation*}
mv=global\_mv(ref, ref\_num, x, y, c)
\end{equation*}

In both cases the value 

$upconvert(ref, (x\ll \MotionVectorPrecision)+mv.x, (y\ll \MotionVectorPrecision)+mv.y)$ 

where $upconvert$ is defined in Section \ref{upconvert} is returned.
 
\subsection{Global motion vector field generation}
\label{globalmv}

This section specifies the operation of the $global\_pred(ref, ref\_num, x,y, c)$ process
for deriving a global motion vector for a pixel at location $(x,y)$, in a component of 
type $c$ from a reference $ref$.

\subsubsection{Chroma scaling}
\label{chromascaling}

The global motion parameters are extracted from the state data. If the component is a chroma
component, the parameters must be scaled appropriately. Set:

\begin{itemize}
\item $\tilde{\bf A}=\GlobalParams[ref\_num].{\bf A}$
\item $\tilde{\bf b}=\GlobalParams[ref\_num].{\bf b}$
\item $\tilde{\bf c}=\GlobalParams[ref\_num].{\bf c}$
\end{itemize}

If $c=Y$, set ${\bf A}=\tilde{\bf A}$, ${\bf b}=\tilde{\bf b}$ and ${\bf c}=\tilde{\bf c}$.

If $c=U$ or $c=V$, ${\bf A}$, ${\bf b}$ and ${\bf c}$ are defined as follows. Scale 
${\bf b}$ according to the chroma scale factors:
\begin{eqnarray*}
b_0 & = & \left(\tilde{b}_0+chroma\_h\_factor()//2\right)//chroma\_h\_factor() \\
b_1 & = & \left(\tilde{b}_1+chroma\_h\_factor()//2\right)//chroma\_h\_factor()
\end{eqnarray*}

Scale ${\bf A}$, taking into account vertical and horizontal factors:
\begin{eqnarray*}
A_{0,0} & = & \tilde{A}_{0,0}\\
A_{1,1} & = & \tilde{A}_{1,1}\\
A_{0,1} & = & \left(\tilde{A}_{0,1}*chroma\_h\_factor()+chroma\_v\_factor()//2 \right) // chroma\_v\_factor() \\
A_{1,0} & = & \left(\tilde{A}_{1,0}*chroma\_v\_factor()+chroma\_h\_factor()//2\right)//chroma\_h\_factor()
\end{eqnarray*}

and give ${\bf c}$ the inverse scaling to ${\bf b}$:
\begin{eqnarray*}
c_0 & = & \tilde{c}_0*chroma\_h\_factor \\
c_1 & = & \tilde{c}_1*chroma\_h\_factor
\end{eqnarray*}

\subsubsection{Field generation}

Set $\alpha = \GlobalParams[ref\_num][ZRS\_exp]$ and 
$\beta =\GlobalParams[ref\_num][perspective\_exp]$.

Writing ${\bf x}=\left( \begin{array}{c} x\\y \end{array}\right)$, set ${\bf y}$ to be
the integer vector defined by:
\begin{equation*}
{\bf y}=\left(2^{\alpha+\beta}-2^\alpha * {\bf c}^T {\bf x}\right)\left(2^\beta*{\bf Ax}+2^{\alpha+\beta}*{\bf b}\right)
\end{equation*}

Set
\begin{eqnarray*}
mv.x & = & y_0\gg (\alpha+\beta) \\
mv.x & = & y_1\gg (\alpha+\beta)
\end{eqnarray*}

and return the motion vector $mv$.

\subsection{Upconversion}
\label{upconvert}

This section specifies the operation of the $upconvert(ref, u, v)$ function
for producing a value from an upconverted picture reference. This
allows for sub-pixel precision in motion compensation.

Motion vectors are allowed to extend beyond the edges of the 
upconverted reference picture component and values lying outside the range
of the component are determined by edge extension, using the values:
\begin{eqnarray*}
cu & = &\clip(u, 0, 2^\MotionVectorPrecision*width(ref)) \\
cv & = &\clip(v, 0, 2^\MotionVectorPrecision*height(ref))
\end{eqnarray*}

There are four cases, depending upon the motion vector precision selected.

\subsubsection{Pixel-accurate motion vectors}

If $\MotionVectorPrecision==0$, no upconversion is actually required and the value
$ref[cv][cu]$ is returned.

\subsubsection{Half-pixel accurate motion vectors}
\label{halfpel}

If $\MotionVectorPrecision==1$ then the reference picture component $ref$ is
upconverted by a factor of 2 in each dimension to create an
array $upref$. The value returned is $upref[cv][cu]$.

$upref$ is created in two stages, first upconverting vertically by
a factor of 2, then horizontally. Define the interpolation filter $h$
to be the 10-tap symmetric filter with taps as defined in figure \ref{upfilter}.

\begin{figure}[h!]
\begin{centering}
\begin{tabular}{l|ccccc}
Tap & $t_0$ & $t_1$ & $t_2$ & $t_3$ & $t_4$\\
\hline
Value & 167 & -56 & 25 & -11 & 3
\end{tabular}
\caption{Interpolation filter coefficients \label{upfilter}}
\end{centering}
\end{figure}

Define an array $ref2$ of height $2*height(ref)$ and width $width(ref)$ by
the recipe, for $0\leq p<width(ref)$ and $0\leq q<2*height(ref)$:

{\bf Case 1.} If $q\%2==0$, set
\begin{equation*}
ref2[q][p]=ref2[q//2][p]
\end{equation*}

{\bf Case 2.} If $q\%2!=0$, $ref2[q][p]$ is set by

\begin{pseudo*}
\bsCODE{ref2[q][p]=  \sum_{i=0}^{4} t_i *(ref[\clip( (q-1)//2-i, 0, height(ref)-1)][p]+
ref[\clip((q+1)//2+i, 0,height(ref)-1)) ][p])}
\bsCODE{ref2[q][p] = (ref2[q][p]+128)\gg 8}
\bsCODE{ref2[q][p] = \clip(ref2[q][p], 0, 2^\VideoDepth-1)}
\end{pseudo*}

The full upconverted array is constructed from $ref2$ in the same way.
 For $0\leq p<2*width(ref)$ and $0\leq q<2*height(ref)$ we have:

{\bf Case 1.} If $p\%2==0$, set
\begin{equation*}
upref[q][p]=ref2[q][p//2]
\end{equation*}

{\bf Case 2.} If $p\%2!=0$, $upref[q][p]$ is set by

\begin{pseudo*}
\bsCODE{upref[q][p]=  \sum_{i=0}^{4} t_i *\left(ref2[q][\clip( (p-1)//2-i, 0, width(ref2)-1)]+
ref2[q][\clip((p+1)//2+i, 0, width(ref2)-1)]\right)}
\bsCODE{upref[q][p] = (upref[q][p]+128)\gg 8}
\bsCODE{upref[q][p] = \clip(upref[q][p], 0, 2^\VideoDepth-1)}
\end{pseudo*}

\begin{informative}
While this filter may appear to be variable separable, the integer rounding and 
clipping processes prevent this being so. Note also that the clipping process for
filtering terms implies that the upconversion uses edge-extension at the array
edges, consistent with the edge-extension used in motion-compensation itself.
\end{informative}

\subsubsection{Quarter- and eighth-pixel accurate motion vectors}

If $\MotionVectorPrecision==2$ or $\MotionVectorPrecision==3$, upconverted values are derived by linear
interpolation from the half-pixel interpolation values $upref$, which is calculated as 
per Section \ref{halfpel}. Given coordinates $(u,v)$, their half-pixel part
is extracted by:

\begin{eqnarray*}
hu & = & u \gg (\MotionVectorPrecision-1) \\
hv & = & v \gg (\MotionVectorPrecision-1) 
\end{eqnarray*}

and their remainder (giving the residual subpixel accuracy) by

\begin{eqnarray*}
ru & = & u-(hu\ll (\MotionVectorPrecision-1)) \\
rv & = & v-(hv\ll (\MotionVectorPrecision-1)) 
\end{eqnarray*}

$ru$ and $rv$ satisfy $0\leq ru,rv <2^{\MotionVectorPrecision-1}$. Then define four weighting
values by:

\begin{eqnarray*}
w00 & = & (2^{\MotionVectorPrecision-1}-rv)*(2^{\MotionVectorPrecision-1}-ru)\\
w01 & = & (2^{\MotionVectorPrecision-1}-rv)*ru\\
w10 & = & rv*(2^{\MotionVectorPrecision-1}-ru)\\
w11 & = & rv*ru
\end{eqnarray*}

and also define the clipped coordinates that we shall use for interpolation by:

\begin{eqnarray*}
cu & = \clip(hu, 0, width(upref)-1) \\
cu1 & = \clip(hu+1, 0, width(upref)-1) \\
cv & = \clip(hv, 0, width(upref)-1) \\
cv1 & = \clip(hv+1, 0, width(upref)-1)
\end{eqnarray*}

The value returned is:
\[
\left(
\begin{array}{l}
w00*upref[cv][cu]+\\
w01*upref[cv][cu1]+\\
w10*upref[cv1][cu]+\\
w11*upref[cv1][cu1]+2^{\MotionVectorPrecision-2}
\end{array}
\right)\gg(\MotionVectorPrecision-1)\]

\begin{informative}
Note that the remainder values $ru$, $rv$ are not determined from the clipped
half-pixel values $cu$, $cv$, $cu1$, and $cv1$. This ensures the remainder
values depend only on the motion vector, and hence are constant across
each block, and allows a block-wise implementation. If the clipped values 
had been used, blocks whose reference
block straddled the edge of a picture would use different remainders in
different parts of the block. See Section \ref{mcimplementation}.
\end{informative}

\begin{informative*}
\subsection{Implementation}
\label{mcimplementation}
[TBC]

The motion compensation process is defined in this specification in
terms of the prediction determined for each pixel. Typically, an
implementation will motion compensate all pixels within a prediction
unit or a block together, as they share motion parameters and hence will
have a contiguous prediction data set in the reference frames.


%The weighting matrix is then applied directly
%to the prediction block as a whole.
%
%It is not necessary for the prediction buffer to consist of the whole video
%component -- it could be a single row of blocks, with values being overwritten
%successively.
%
%The process for a single video component is effectively:
%
%-- Starting with an empty prediction buffer,
%pred(x,y) = 0, forall x,y
%
%for r = 0 to numrefs:
%    -- upconvert the reference frame and clip the results
%    ref'_r(x,y) = clip(upconvert(ref_r,x,y),0,2^{video_depth})
%
%    -- perform block prediction, OBMC weighting and reference weighting,
%    -- accumumulating the results into the prediction buffer.
%    foreach b in blocks:
%        b'pred(x,y) = block_predict(b,x,y)
%
%        -- choosereference weight contribution
%        W <- ...
%
%        for i = 0 to yblen:
%            for j = 0 to xblen:
%                -- accumulate predicted pixels into prediction buffer
%                pred
%
%%\begin{lstlisting}
%%    upconvertB_1d(f,r,x) = let $x' = x mod f$
%%                           in $%
%%    \lfloor\frac{(16-fx')r_{\frac{x-x'}{f}} + fx'r_{\frac{x+f-x'}{f}} + 8}{16}\rfloor%
%%    $
%%\end{lstlisting}
%
%
%\annotate{df}{Contrasts to the rest of the
%section that uses absolute pixel position (not vector)}
%\begin{informative}
%A note about global motion calculation.
%
%It is possible to calculate the entire global motion field using
%approximately four additions per pixel.
%
%The global transformation may be expanded as:
%
%\begin{equation*}
%\pmb{z}_{ref} = -\pmb{A}\left(\pmb{z}\pmb{z}^\text{T}\right)\pmb{c}
%                + \left(\pmb{A} - \pmb{b}\pmb{c}^\text{T}\right)\pmb{z}
%                + \pmb{b}
%\end{equation*}
%
%Then $\pmb{z}_{ref} - \pmb{z}$ may be further expanded to give the individual
%components of the motion vector:
%
%\begin{align*}
%  v_x &= \alpha{}x^2 + \beta{}xy + \gamma{}y^2 + (\delta - 1)x + \epsilon{}y + b_1 \\
%  v_y &= \zeta{}x^2 + \eta{}xy + \theta{}y^2 + \kappa{}x + (\mu - 1)y + b_2 \\
%\end{align*}
%where,
%\begin{align*}
%\alpha &= -a_{11}c_1 &
%\beta &= -(a_{11}c_2 + a_{12}c_1) &
%\gamma &= -a_{12}c_2 \\
%\delta &= a_{11} - b_1c_1 &
%\epsilon &= a_{12} - b_1c_2 &
%\zeta &= -a_{21}c_1 \\
%\eta &= -(a_{21}c_2 + a_{22}c_1) &
%\theta &= -a_{22}c_2 &
%\kappa &= a_{21} - b_2c_1 &
%\mu &= a_{22} - b_2c_2
%\end{align*}
%
%That is the components of the global motion vector are given by
%two-dimensional quadratic functions.
%
%Given a two dimensional quadratic function:
%
%\begin{equation*}
%z(x,y) = px^2 + qxy + ry^2 + sx + ty + u
%\end{equation*}
%
%We can calculate z(y,x) for successive values of $x$ and $y$ by merely
%incremeting the values for $z(y, x-1)$ or $z(y-1, x)$.  The following
%algorithm for generating a two dimensional array containing the quadratic
%values $z(y,x)$.
%
%\begin{verbatim}
%quadratic(p,q,r,s,t,u):
%    zed = array2D(ylength, xlength)
%    g = 2*p
%    h0 = s - p - q
%    k = 2*r
%    m0 = t - r
%    m = m0 - k
%    z0 = u - m0
%    for y in range(ylength):
%        h0 += q
%        h = h0 - g
%        m += k
%        z0 += m
%        z = z0 - h0
%        for x in range(xlength):
%            h += g
%            z += h
%            zed[y][x] = z
%    return zed
%\end{verbatim}
%

\end{informative*}

\subsection{Clipping}
\label{pictureclip}

Picture data must be clipped prior to being output or being
used as a reference:

\begin{pseudo}{clip\_picture}{}
\bsFOREACH{c}{Y,C1,C2}
    \bsCODE{clip\_component(\CurrentPicture[c])}
\bsEND
\end{pseudo}


\begin{pseudo}{clip\_component}{comp\_data,c}
\bsIF{c==Y}
    \bsCODE{bit\_depth=\LumaDepth}
\bsELSE
    \bsCODE{bit\_depth=\ChromaDepth}
\bsEND
\bsFOR{y=0}{\height(comp\_data)-1}
    \bsFOR{x=0}{\width(comp\_data)-1}
        \bsCODE{data = \clip(comp\_data[y][x], -2^{bit\_depth-1}, 2^{bit\_depth-1}-1)}
     \bsEND
\bsEND
\end{pseudo}

\begin{informative}
Note that clipping is incorporated into motion compensation, so that strictly speaking additional
clipping is only required for intra pictures.
\end{informative}

\subsection{Video output ranges}
\label{videooutput}

Video output data ranges are deemed to be non-negative, so that the offset and excursion
 values may be applied by subsequent processing. Since decoded video data is bipolar, it must be suitably offset before output:

\begin{pseudo}{offset\_output\_data}{picture\_data}
\bsFOREACH{c}{Y, C1, C2}
    \bsIF{c==Y}
        \bsCODE{bit\_depth=\LumaDepth}
    \bsELSE
        \bsCODE{bit\_depth=\ChromaDepth}
    \bsEND
    \bsCODE{comp=picture\_data[c]}
    \bsFOR{y=0}{\height{comp}-1}
        \bsFOR{x=0}{\width{comp}-1}
            \bsCODE{comp[y][x]+=2^{bit\_depth-1}}
        \bsEND
    \bsEND
\bsEND
\end{pseudo}
