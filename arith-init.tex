\label{initarith}

At the beginning of the decoding of any data unit, the arithmetic
decoding state is initialised as follows:

\begin{pseudo}{initialise\_arithmetic\_decoding}{(block\_data\_length}
\bsCODE{\ABitsLeft=8*block\_data\_length}
\bsCODE{\ALow = 0x0000}
\bsCODE{\AHigh = 0x0000}
\bsCODE{\ACode = 0x0000}
\bsCODE{init\_contexts()}
\end{pseudo}

Contexts are initialised by the $init\_contexts()$ function as follows:
\begin{pseudo}{init\_contexts}{}
\bsFOR{i=0}{len(\AContexts)-1}
  \bsCODE{\AContexts[i][0]=1}
  \bsCODE{\AContexts[i][1]=1}
\bsEND
\end{pseudo}

\begin{informative}
[Move to a general section on spec methodology??]
The use of global variables for implementation code is
considered poor programming practice. It should be remembered that this
document is a specification and not implementation code. Global
variables are used in this specification for two reasons. Firstly they
avoid the need to pass the arithmetic coding state to each function that
reads arithmetic coded data, thereby (we hope) clarifying the specification.
Secondly a specific implementation is likely to associate these
variables with some sort of output stream object. For example in a
software implementation written in C++ these variables may be associated
with an output stream in which output is implemented using custom stream
manipulators. Data output is highly implementation specific. The use of
global variables is a way of specifying output in a generic way. Their
use should not be taken to indicate that they should (or should not) be
used in any specific implementation.
\end{informative}
