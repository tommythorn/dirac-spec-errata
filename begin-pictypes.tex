Three types of picture are defined.

Intra pictures are coded without reference to other pictures in the
sequence. These pictures form a useful point to start the decoding
process. They also have uses if low-delay coding is required - a
sequence of Intra pictures has mimimum delay, at the expense of
potentially greater bandwidth requirements. There is a third use for
Intra pictures: when the sequence is highly dynamic, prediction using
motion compensation may be impractical. In such instances, a coder may
choose to default to a sequence of Intra frames.

Inter pictures are coded with reference to other pictures in the
sequence, and are split into two types:
\begin{itemize}
	\item references for other pictures
	\item not references for other pictures
\end{itemize}

Within the receiver, we have to arrange that all the references for
pictures are available at the right time. This means that pictures are
often delivered in a different sequence from the display sequence, to
ensure that the buffer memory in the receiver has the necessary
information to decode pictures when it is needed.

