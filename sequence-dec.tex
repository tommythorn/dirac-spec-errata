\label{sequencedecoding}

There is no one unique way of describing a Dirac decoder. However the pseudocode 
below is illustrative code for a sample decoder. It emphasizes which parts of the decoding
 process generate decoded output data. Note that the potential presence of padding or auxiliary data is ignored for clarity.

\begin{pseudo}{decode\_sequence}{}
   \bsCODE{\StateName=\{\}}
   \bsCODE{decoded\_pictures = \{\}}
   \bsCODE{\RefBuffer=\{\}}
   \bsCODE{parse\_info()}{\ref{parseinfoheader}}
   \bsCODE{\VideoParams = sequence\_header()}{\ref{sequenceheader}}
   \bsCODE{parse\_info()}{\ref{parseinfoheader}}
   \bsWHILE{is\_end\_of\_sequence() == \false}{\ref{parsecodevalues}}
      \bsIF{is\_seq\_header()==\true}{\ref{parsecodevalues}}
         \bsCODE{\VideoParams = sequence\_header()}{\ref{sequenceheader}}
      \bsELSEIF{is\_picture()==\true}{\ref{parsecodevalues}}
         \bsCODE{picture\_parse()}{\ref{pictureparse}}
         \bsCODE{decoded\_pictures[\PictureNumber] = picture\_decode()}{\ref{picturedec}}
      \bsEND
      \bsCODE{parse\_info()}{\ref{parseinfoheader}}
   \bsEND
   \bsRET{\{\VideoParams, decoded\_pictures\}}
\end{pseudo}

The process returns the video parameters, consisting of the essential metadata required for 
display and interpretation of the video data, and the array of decoded pictures. Each decoded picture contains the three video component data arrays together with a picture number.

The pseudocode describes the decoding process. Decoding starts by clearing the 
decoder state and the decoder output. Thus video sequences may be decoded as independent entities. The first data extracted from the Dirac stream is parse information. The parse info header indicates what type of data unit follows, and this information is stored in the decoder state. The decoder continues to read pairs of data unit and parse info 
headers until the end of the sequence is reached. The end of sequence 
is indicated by data in the final parse info header. If a data unit is a sequence header the 
decoded video parameters are updated with the information contained in the header. If the 
data unit is a picture then:
\begin{itemize}
\item the picture is parsed, then decoded
\item the picture is placed in the correct position in the output array
\end{itemize}

Note that since Dirac supports inter as well as intra picture coding, picture numbers
within the stream may not be sequential, and the decoded output pictures will not be
placed in the output buffer in order. Annex \ref{profilelevel} defines the constraints
which may be placed upon re-ordering depth.

Sequences need not be decoded from the start: decoding can start from any sequence header
according to the provisions of section \ref{randomaccess}, although some pictures might
not be (completely) decodeable due to a chain of references reaching back earlier in the stream
than the sequence header, introducing dependencies on unavailable pictures. 
The behaviour of a decoder when confronted with such pictures is application-specific.

\subsection{Non-sequential picture decoding}
\label{nonsequential}
The ability to decode pictures in a non-sequential manner is important for many
 applications, such as video editing. Non-sequential access means decoding a 
stream in any manner other than decoding pictures sequentially from the beginning 
of the stream to the end: this may include decoding only intra pictures, decoding backwards, or decoding pictures in random parts of the stream. 

Stream navigation, including non-sequential access is supported by the information 
in the parse info headers in the stream (section \ref{parseinfoheader}). 

In order to start decoding, other than at the start of a sequence, the decoder 
must first synchronize to the stream. The parse info prefix is present to support such synchronization. A decoder would first search for the parse info prefix to locate 
the start of a parse info header. The parse info prefix is not guaranteed to occur
 uniquely within parse info headers (the entropy coding used in 
Dirac precludes this). However, the probability of a spurious 
prefix occuring is low: 1 in $2^{32}$, since the prefix is 4 bytes long. The probability of finding 
two spurious prefix sequences separated by the value of the next (or previous) parse 
offset is 1 in $2^{64}$.  

Having synchronized with the stream the decoder now needs to locate a sequence header 
in order to find parameters needed to decode pictures. This may be done by moving back (or forward) through the stream, using the parse offsets.

The Dirac stream also supports seeking to a particular picture number, since this
is contained in each picture header.

