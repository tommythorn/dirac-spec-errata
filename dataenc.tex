%%%%%%%%%%%%%%%%%%%%%%%%%%%%%%%%%%%%%%%%%%%%%%%%
% - This chapter defines how raw and VLC data  - %
% - is extracted                               - % 
%%%%%%%%%%%%%%%%%%%%%%%%%%%%%%%%%%%%%%%%%%%%%%%%

\label{dataenc}

Data shall be encoded in the Dirac bitstream using four basic methods: 
\begin{itemize}
\item fixed-length bit-wise codings,
\item fixed-length byte-wise codings,
\item variable-length codes,
\item arithmetic encoding.
\end{itemize}

This annex defines how data shall be encoded in the Dirac stream and how 
sequences of bits shall be extracted as values of various types using the 
aforementioned fundamental data coding types. The extraction of arithmetically
encoded data shall require the use of the arithmetic decoding engine defined in 
Annex \ref{arithengine}.

\subsection{Bit-packing and data input}
\label{bitpacking}

This section defines the operation of the $read\_bit()$, $read\_byte()$ 
and $byte\_align()$ functions used for direct access to the Dirac stream.

The stream data shall be accessed byte by byte, and a decoder is deemed to 
maintain a copy of the current byte, $\CurrentByte$, and an index to the next 
bit (in the byte) to be read, $\NextBit$. $\NextBit$ shall be an integer from 0
 (least-significant bit) to 7 (most-significant bit). Bits within bytes shall be
 accessed from the msb first to the lsb.

\subsubsection{Reading a byte}

The $read\_byte()$ function shall perform the following steps:
\begin{enumerate}
\item Set $\NextBit=7$ 
\item Set $\CurrentByte$ to the next unread byte in the stream
\end{enumerate}

\subsubsection{Reading a bit}\label{readbit}

The $read\_bit()$ function shall be defined as follows:

\begin{pseudo}{read\_bit}{}
\bsCODE{bit = ( \CurrentByte \gg \NextBit ) \& 1 }
\bsCODE{\NextBit -= 1}
\bsIF{\NextBit<0}
    \bsCODE{\NextBit = 7}
    \bsCODE{read\_byte()}
\bsEND
\bsRET{bit}
\end{pseudo}

\subsubsection{Byte alignment}\label{bytealign}

The $byte\_align()$ function shall be used to discard data in the current byte 
and begin data access at the next byte, unless input is already at the beginning
 of a byte. It shall be defined as follows: 

\begin{pseudo}{byte\_align}{}
\bsIF{\NextBit != 7}
    \bsCODE{read\_byte()}
\bsEND
\end{pseudo}

\subsection{Parsing of fixed-length data}

Dirac defines three fixed length data encodings as follows:

\subsubsection{Boolean}

The $read\_bool()$ function shall return $\true$ if 1 is read from the stream and 
$\false$ otherwise. The $read\_bool()$ function shall be defined as follows:

\begin{pseudo}{read\_bool}{}
\bsIF{read\_bit()==1}
    \bsRET{\true}
\bsELSE
    \bsRET{\false}
\bsEND
\end{pseudo}

\subsubsection{n-bit literal}\label{readnbits}

An $n$-bit number in literal format shall be decoded by extracting $n$ bits
in order, using the $read\_bit()$ function (Section \ref{readbit})
 and placing the first bit in the leftmost position, the second
bit in the next position and so on. The resulting value shall be
interpreted as an unsigned integer.

The $read\_nbits()$ function shall be defined as follows:

\begin{pseudo}{read\_nbits}{n}
\bsCODE{val=0}
\bsFOR{i=0}{n-1}
    \bsCODE{val \ll = 1}
    \bsCODE{val += read\_bit()}{\ref{readbit}}
\bsEND
\bsRET{val}
\end{pseudo}

\subsubsection{$n$-byte unsigned integer literal}

The $read\_uint\_lit()$ function shall be defined as follows:

\begin{pseudo}{read\_uint\_lit}{n}
\bsCODE{byte\_align()}{\ref{bytealign}}
\bsRET{read\_nbits(8*n)}{\ref{readnbits}}
\end{pseudo}

\subsection{Variable-length codes}
\label{vlc}
Variable-length codes shall be used in three ways in the Dirac stream:
\begin{enumerate}
\item The first use shall be for direct encoding of header values into the stream. 
\item The second use shall be for entropy coding of motion data and coefficients, where arithmetic coding
is used.
\item The third use shall be for binarisation in the arithmetic encoding/decoding process so that integer 
values may be coded and decoded using a binary arithmetic coding engine. This is
defined in Annex \ref{arithdecoding}.
\end{enumerate}

When used for coding motion data and coefficients, VLCs shall be employed within
a data block of known length. It is possible to gain additional compression by early termination:
maintaining a count of remaining bits, and returning default values when this length
is exceeded. This shall be achieved by use of the $read\_bitb()$, $read\_boolb()$, 
$read\_uintb()$ and $read\_sintb()$ for reading values from data blocks. 

\begin{informative}
A similar early termination facility is a used for arithmetic decoding.
\end{informative}

\subsubsection{Data input for bounded block operation}
\label{blockreadbit}

This section specifies the operation of the $read\_bitb()$ process for reading bits from
a block of known size, and the $flush\_inputb()$ process for discarding the remainder of a block of 
data. 

These processes shall use $\ABitsLeft$ to determine the number of bits left to the 
end of the block.

The $read\_bitb()$ function shall be defined as follows:

\begin{pseudo}{read\_bitb}{}
\bsIF{\ABitsLeft==0}
  \bsRET{1}
\bsELSE
  \bsCODE{\ABitsLeft -= 1}
  \bsRET{read\_bit()}
\bsEND
\end{pseudo}

When all bits in the block have been read, then $read\_bitb()$ shall return 1 by default. 

It is possible that not all data in a block is exhausted after a sequence of read operations.
At the end of a sequence of read operations, the decoder shall flush the block. 

The $flush\_inputb()$ process shall be defined as follows:

\begin{pseudo}{flush\_inputb}{}
\bsWHILE{\ABitsLeft>0}
    \bsCODE{read\_bit()}
    \bsCODE{\ABitsLeft-=1}
\bsEND
\end{pseudo}

\subsubsection{Unsigned interleaved exp-Golomb codes}
This section defines the unsigned interleaved exp-Golomb data format and the operation
of the $read\_uint()$ and the $read\_uintb()$ functions. 

Unsigned interleaved exp-Golomb data shall be decoded to produce unsigned
 integer values.The format shall consist of two interleaved parts, 
and each code shall be an odd number, $2K+1$ bits in length.

The $K+1$ bits in the even positions (counting from zero) shall be the ``follow" bits, and 
the $K$ bits in the odd positions shall be the ``data" bits $b_i$ that are used to construct
the decoded value itself. A follow bit value of $0$ shall indicate a subsequent data bit,
whereas a follow bit value of $1$ shall terminate the code, a typical sequence being
\begin{equation*}
0\quad x_{K-1}\quad 0\quad x_{K-2}\hdots 0\quad x_{0}\quad 1
\end{equation*}

The data bits $x_{K-1}, x_{K-2}, \hdots, x_0$ shall form the binary representation 
 of the first $K$ bits of the $(K+1)$-bit number 
$N+1$, where $N$ is the number to be decoded, i.e.:
\begin{equation*}
N+1=1 x_{K-1} x_{K-2}\hdots x_0 \text{ (base $2$) }=2^K+\sum_{i=0}^{K-1} 2^i * x_i
\end{equation*}

Table \ref{table:uegolcodings} shows encodings of the values 0--9.

\begin{table}[!ht]
\centering
\begin{tabular}{|l|c|}
\hline
\rowcolor[gray]{0.75}Bit sequence & Decoded value \\
\hline
1                 &  0\\
0 0 1             &  1\\
0 1 1             &  2\\
0 0 0 0 1         &  3\\
0 0 0 1 1         &  4\\
0 1 0 0 1         &  5\\
0 1 0 1 1         &  6\\
0 0 0 0 0 0 1     &  7\\
0 0 0 0 0 1 1     &  8\\
0 0 0 1 0 0 1     &  9\\
\hline
\end{tabular}

\caption{Example conversions from unsigned interleaved exp-Golomb-coded 
values to unsigned integers \label{table:uegolcodings}}
\end{table}

Although apparently complex, the interleaving ensures that the code has a very simple decoding loop. 

The $read\_uint()$ function shall return an unsigned integer value and shall be defined 
as follows:

\begin{pseudo}{read\_uint}{}
\bsCODE{value = 1}
\bsWHILE{read\_bit() == 0}
  \bsCODE{value \ll = 1}
  \bsIF{read\_bit()==1}
    \bsCODE{value += 1}
  \bsEND
\bsEND
\bsCODE{value -= 1}
\bsRET{value}
\end{pseudo}

\begin{informative}
Conventional exp-Golomb coding places all follow bits at the beginning as a prefix. This is
easier to read, but requires that a count of the prefix length be maintained. Values can only
be decoded in two loops -- the prefix followed by the data bits. Interleaved exp-Golomb 
coding allows values to be decoded in a single loop, without the need for a length count.
\end{informative}

The $read\_uintb()$ function is identical to $read\_uint()$ except that the block-bounded read
operation is employed, and shall be defined as follows:

\begin{pseudo}{read\_uintb}{}
\bsCODE{value = 1}
\bsWHILE{read\_bitb() == 0}
  \bsCODE{value \ll = 1}
  \bsIF{read\_bitb()==1}
    \bsCODE{value += 1}
  \bsEND
\bsEND
\bsCODE{value -= 1}
\bsRET{value}
\end{pseudo}

\begin{informative}
When $\ABitsLeft==0$, all subsequent values read by $read\_uintb()$ will be 0.
\end{informative}

\subsubsection{Signed interleaved exp-Golomb}
\label{segol}

This section defines the signed interleaved exp-Golomb data format and the operation
of the $read\_sint()$ and $read\_sintb()$ functions.

The code for the signed interleaved exp-Golomb data format consists of the
unsigned interleaved exp-Golomb code for the magnitude, followed by a sign bit
for non-zero values, as shown in the table below:

%\begin{table}[!ht]
{
\centering
\begin{tabular}{|l|c|}
\hline
\rowcolor[gray]{0.75}Bit sequence & Decoded value \\
\hline
0 0 0 1 1 1         &  -4\\
0 0 0 0 1 1         &  -3\\
0 1 1 1            &  -2\\
0 0 1 1           &  -1\\
1                 &  0\\
0 0 1 0           &  1\\
0 1 1 0            &  2\\
0 0 0 0 1 0         &  3\\
0 0 0 1 1 0         &  4\\
\hline
\end{tabular}

%\caption{Example conversions from signed interleaved exp-Golomb-coded values 
%to signed integers \label{table:segolcodings}}
%\end{table}
}

The $read\_sint()$ function shall be defined as follows.

\begin{pseudo}{read\_sint}{}
\bsCODE{value = read\_uint()}
\bsIF{ value!= 0}
  \bsIF{read\_bit()==1}
    \bsCODE{value = -value}
  \bsEND
\bsEND
\bsRET{value}
\end{pseudo}

The $read\_sintb()$ function is identical to $read\_sint()$ except that the block-bounded read
operation is employed, and shall be defined as follows:

\begin{pseudo}{read\_sintb}{}
\bsCODE{value = read\_uintb()}
\bsIF{value != 0}
  \bsIF{read\_bitb()==1}
    \bsCODE{value = -value}
  \bsEND
\bsEND
\bsRET{value}
\end{pseudo}

\begin{informative} When $\ABitsLeft==0$, all subsequent values read by $read\_sintb()$ will be 0.\end{informative}

\subsection{Parsing of arithmetic-coded data}

\label{arithdecoding}

This section defines the operations for reading arithmetic-coded
data. These operations shall  make use of the elementary arithmetic coding functions 
defined by the arithmetic decoding engine defined in Annex \ref{arithengine}.

Arithmetically-coded data is present in the Dirac stream in data blocks which shall consist of
a whole number of bytes and which shall be byte aligned. Where arithmetic coding is used, each such
block shall be preceded by data which includes a length code $length$, which shall be equal to the length in
bytes of the data block. 

The function $initialise\_arithmetic\_decoding(length)$
(Section \ref{initarith}) shall then initialises the arithmetic decoder. Once the arithmetic
decoder is initialised, boolean and integer values may be extracted.

After all values in a particular arithmetic coded block have been parsed, any remaining data
shall be flushed using the $flush\_inputb()$ process (Section \ref{blockreadbit}).

\subsubsection{Context probabilities}
\label{contextprobs}

Values shall be extracted by using binary context probabilities. A context is a 
decoding state, representing the set of all data decoded so far.

A context probability shall be a 16 bit unsigned integer value representing the 
probability of a bit being 0 in a given context, where zero probability is 
represented by 0x0, and equal likelihood by 0x8000. The process for initializing
 and updating context probabilities shall be as defined in annex 
\ref{initarith} and \ref{contextupdate}.

\begin{informative}
Probability 1, or certainty, would be represented by the 17-bit number 0x10000. 
This value, and probability 0 (0x0), can never be attained due to the operation 
of the probability update process (annex \ref{contextupdate}).
\end{informative}

Different context probabilities shall be employed for extracting binary values, 
based on the values of previously decoded data. Each context probability shall 
be updated by the arithmetic decoding engine to track statistics after it
has been used to extract a value.

The set of contexts probabilities shall be defined by $\AContexts$, and an 
individual context shall be accessed via a keyword label i.e. 
$\AContexts$ is a map and the context value shall be $\AContexts[l]$ for a 
label $l$.

The array of context probability labels to be used in arithmetic decoding shall 
be passed to the arithmetic decoding engine at initialization (annex 
\ref{initarith}).

\subsubsection{Arithmetic decoding of boolean values}

Given a context probability lable $l$, the arithmetic decoding engine shall support a function
$read\_boola(l)$, specified in Section \ref{arithreadbool}, which shall returna boolean value.

\subsubsection{Arithmetic decoding of integer values}

\label{arithreadint}

This section defines the operation of the $read\_sinta(context\_prob\_set)$ and
$read\_uinta(context\_prob\_set)$ functions
 for extracting integer values from a block of arithmetically coded data.

\paragraph{Binarisation and contexts}\label{binandcontext}
$\ $\newline
Signed and unsigned integer values shall be coded by first converting to binary form by 
using interleaved exp-Golomb binarisation as per Section \ref{vlc}. The 
$read\_sinta()$ and $read\_uinta()$ processes shall be identical to the 
$read\_sint()$ and $read\_uint()$ processes, except that instances of $read\_bit()$ shall be replaced
by instances of $read\_boola()$ (Section \ref{arithreadbool}) using a suitable context
for each bit. 

$read\_sinta()$ and $read\_uinta()$ shall be provided with a map $context\_prob\_set$,
which shall consist of three parts:
\begin{enumerate}
\item an array of follow context indices, $context\_prob\_set[FOLLOW]$
\item a single data context index, $context\_prob\_set[DATA]$ 
\item a sign context index, $context\_prob\_set[SIGN]$ (ignored for unsigned integer decoding)
\end{enumerate}

Each follow context shall be used for decoding the corresponding follow bit, with the
last follow context being used for all subsequent follow bits also (if any).
 
The follow context selection function $follow\_context()$ shall be defined as follows:

\begin{pseudo}{follow\_context}{index, context\_prob\_set}
\bsCODE{pos= \min(index, length(context\_prob\_set[FOLLOW])-1) }
\bsCODE{ctx\_label = context\_prob\_set[FOLLOW][pos]}
\bsRET{ctx\_label}
\end{pseudo}

\paragraph{Unsigned integer decoding}
$\ $\newline

The $read\_uinta()$ function shall be defined as follows:

\begin{pseudo}{read\_uinta}{context\_prob\_set}
\bsCODE{value = 1}
\bsCODE{index = 0}
\bsWHILE{read\_boola(follow\_context(index,context\_prob\_set) )== \false}{\ref{binandcontext}}
  \bsCODE{value \ll = 1}
  \bsIF{read\_boola(context\_prob\_set[DATA]) == \true}
    \bsCODE{value += 1}
  \bsEND
  \bsCODE{index += 1}
\bsEND
\bsCODE{value -= 1}
\bsRET{value}
\end{pseudo}

\paragraph{Signed integer decoding}
$\ $\newline
$read\_sinta()$ decodes first the magnitude then the sign, as necessary:

\begin{pseudo}{read\_sinta}{context\_prob\_set}
\bsCODE{value=read\_uinta(context\_prob\_set)}
\bsIF{value != 0}
  \bsIF{read\_boola(context\_prob\_set[SIGN]) == \true}
    \bsCODE{value=-value}
  \bsEND
\bsEND
\bsRET{value}
\end{pseudo}
