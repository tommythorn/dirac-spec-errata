This section specifies the process for decoding the Block Motion data
unit. This process is invoked for Inter frames when decoding the Frame
Prediction data unit. The Block Motion data unit is byte aligned and
occupies a whole number of bytes, padded with zero bits as necessary.
Its size in bytes must be equal to BLOCK\_DATA\_LENGTH.

Inputs to this process are: global motion field GLOBAL[][][];
GLOBAL\_MOTION\_FLAG, X\_NUM\_MB, Y\_NUM\_MB, X\_NUM\_BLOCKS, Y\_NUM\_BLOCKS,
NUM\_REFS, 

Outputs from this process are: a value of MB\_SPLIT and MB\_COMMON for
each MB; a value of PRED\_MODE for each prediction unit in each MB; a
motion vector MV1 for each prediction unit with PRED\_MODE equal to
REF1ONLY or REF1AND2; a motion vector MV2 for each prediction unit with
PRED\_MODE equal to REF2ONLY or REF1AND2.

The Block Motion data unit is a single block of arithmetically coded
binary data. This section specifies the decoding operations to be used
in conjunction with the arithmetic decoding engine specified in Section
with the contexts and initialisation defined in Section .

Block motion data is used in predicting Inter frames using either global
or block motion or both. When block motion is used it encodes the motion
vectors to be used. When two reference frames are used it encodes motion
vectors for both references. With two references it also encodes which
reference, or both references, are to be used and which blocks are to be
coded Intra (i.e. without using motion compensated prediction). The
Block Motion data also encodes any other information that is needed to
perform motion compensated prediction, such as macroblock  splitting.

Motion vector data is organised into macroblock s, which are 4x4 arrays
of blocks. The motion data is decoded by decoding the data in each MB,
scanning in raster order from the top-left corner. 

Numbering

For the purposes of this specification, macroblocks  are numbered by x-
and y- coordinates in raster order, from the top-left MB. Blocks are
also numbered from by x- and y-coordinates in raster order from the
top-left block. The indices of blocks within a macroblock with
coordinates (x,y) therefore run from (4x,4y) (top-left) through to
(4x+3,4y+3) (bottom right). 

Overall decoding process

The decoding process iterates across all macroblocks, first decoding the
macroblock data and then decoding the prediction unit data pertaining to
each macroblock. The macroblock data constrains how much block data
there is and how it is interpreted. The decoder maintains a value
MB\_COUNT which is used to reset statistics periodically.

In pseudocode the decoding process is:

MB\_COUNT=0

for (y=0;  y<Y\_NUM\_MB;  y++)

{

    for (x=0;  x<X\_NUM\_MB;  x++)

    {

          decode\_MB\_data (x,y)

          decode\_MB\_block\_data(x,y)

          MB\_COUNT++

          if (MB\_COUNT>32)

              halve\_all\_counts()

    }

}

The arithmetic decoding engine function halve\_all\_counts() is specified
in Section .


