This section specifies the number and dimensions of code blocks used in
decoding subband data.

Inputs to this process are: PARTITION\_INDEX, IS\_INTRA, SUBBAND\_NUM,
MAX\_XBLOCKS, MAX\_YBLOCKS

Outputs to this process are: XNUM\_CODEBLOCKS, YNUM\_CODEBLOCKS

If SPATIAL\_PARTITION is FALSE, then

XNUM\_CODEBLOCKS=1
YNUM\_CODEBLOCKS=1

If SPATIAL\_PARTITION\_FLAG is TRUE then XNUM\_CODEBLOCKS and
YNUM\_CODEBLOCKS are derived from the values of PARTITION\_INDEX and
SUBBAND\_NUM as follows.

Default

If PARTITION\_INDEX is 1 then the default spatial partition is used. The
minimum dimension of a code block is set by

MIN\_BLOCK\_DIM=4

The partition is derived from the subband number and whether the frame
is intra or not. Initial values of XNUM\_CODEBLOCKS and YNUM\_CODEBLOCKS
are derived from .




IS\_INTRA



TRUE

FALSE

SUBBAND\_NUM

<NUM\_SUBBANDS-6

XNUM\_CODEBLOCKS=4

YNUM\_CODEBLOCKS=3

XNUM\_CODEBLOCKS=12

YNUM\_CODEBLOCKS=8



≥NUM\_SUBBANDS-6

<NUM\_SUBBANDS-3

XNUM\_CODEBLOCKS=1

YNUM\_CODEBLOCKS=1

XNUM\_CODEBLOCKS=8

YNUM\_CODEBLOCKS=6



≥NUM\_SUBBANDS-3

XNUM\_CODEBLOCKS=1

YNUM\_CODEBLOCKS=1

XNUM\_CODEBLOCKS=1

YNUM\_CODEBLOCKS=1


Table   Matrix of code block numbers

XNUM\_CODEBLOCKS and YNUM\_CODEBLOCKS are then adjusted to ensure that
minimum codeblock sizes are respected:

XNUM\_CODEBLOCKS=min(XNUM\_CODEBLOCKS , SUBBAND\_WIDTH//MIN\_BLOCK\_DIM)

YNUM\_CODEBLOCKS=min(YNUM\_CODEBLOCKS , SUBBAND\_HEIGHT//MIN\_BLOCK\_DIM)

Custom

If PREDICTION\_INDEX is 0 then the number of codeblocks is configurable
on a frame basis by setting MAX\_XBLOCKS and MAX\_YBLOCKS.
XNUM\_CODEBLOCKS. The minimum dimension of a code block is set by

MIN\_BLOCK\_DIM=4

and

XNUM\_CODEBLOCKS=min(MAX\_XBLOCKS , SUBBAND\_WIDTH//MIN\_BLOCK\_DIM)

YNUM\_CODEBLOCKS=min(MAX\_YBLOCKS , SUBBAND\_HEIGHT//MIN\_BLOCK\_DIM)


