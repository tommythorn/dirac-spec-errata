This document only specifies the decoding of arithmetic coded data. 
However whilst it is clearly vital that an encoding process matches the decoding
process, it is not straightforward to derive an implementation of the
encoder by only looking only at the decoder specification. Therefore this
informative section describes a possible implementation for an
arithmetic encoding engine that will produce output decodeable by
the Dirac arithmetic decoding engine.

\subsection{Encoder variables}

An arithmetic encoder would require the following integer variables:

\begin{itemize}
\item $\text{low}$, a value indicating the bottom of the encoding interval
\item $\text{high}$, a value indicating the top of the encoding interval
\item $\text{underflow}$, a value tracking the number of unresolved ``straddle" conditions 
(described below)
\end{itemize}

A Dirac binary arithmetic encoder implementation will code a set of data in three stages:

\begin{enumerate}
\item Initialisation
\item Processing of all values
\item Flushing
\end{enumerate}

\subsubsection{Initialisation}

Initialisation of the arithmetic encoder is very simple -- the internal variables are
set as:

\begin{eqnarray*}
\text{low}&=&\text{0x0} \\
\text{high}&=&\text{0xFFFF} \\
\text{underflow}&=&0
\end{eqnarray*}

(Note that although the arithmetic decoder initialises $\AHigh$ to be 0x0000 in decoding the first
value, $shift\_all\_bits()$ is called which will set $\AHigh$ to 0xFFFF and $\ACode$ to the first
16 bits in the stream.)

\subsubsection{Processing of binary values}
\begin{comment}
The arithmetic coding processBits are encoded one at a time based on an estimated probability
embodied in a ``context''. As with decoding, the context simply relates to
counts of the number of zeros or ones that have been previously encoded.

A possible algorithm for encoding a single bit, given a context, is:

Encode Binary Arithmetic Coded Bit
\begin{verbatim}
write_ba(symbol, context):
    while ( ((high&0x4000)==0x0) and
           ((low&0x4000)==0x4000) ):
        code ^= 0x4000
        high ^= 0x4000
        low ^= 0x4000
        shift_bit_out()
    weight = context[0] + context[1]
    scaler = (0x10000+weight//2)//weight

    probability0 = context[0]*scaler
    range = high-low+1
    range_x_prob = (range * probability0)>>16
    if ( symbol )
        context[1] += 1
        low = low + range_x_prob
    else
        context[0] += 1
        high = low  + range_x_prob - 1
    if ( (context[0] + context[1]) > 255 ):
        #Halve counts in the context
        context[0] >> 1
        context[0] += 1
        context[1] >> 1
        context[1] += 1
    while (((high&0x8000)==0x0) and ((low&0x8000)==0x0)):
        output_bits()
        shift_bit_out()
\end{verbatim}

Shift Bit Out
\begin{verbatim}
shift_bit_out():
    high << 1
    high &= 0xFFFF
    high += 1
    low << 1
    low &= 0xFFFF
\end{verbatim}

Output Bits
\begin{verbatim}
output_bits():
    write_bit( (high&0x8000)==0x8000 )
    while ( underflow > 0 ):
        write_bit( high&0x8000)==0x0 )
        underflow -= 1
\end{verbatim}

Flush Encoder
\begin{verbatim}
flush_encoder():

    write_bit( (high&0x4000)==0x4000 )
    while ( underflow >= 0 ):
        write_bit( high&0x4000)==0x0 )
        underflow -= 1
\end{verbatim}

Where the ``write\_bit(bit)'' function outputs a single bit.
\end{comment}