This appendix provides three things:
\begin{itemize}
\item a description of the principles of arithmetic
coding
\item a specification of the arithmetic decoding
engine used in Dirac
\item a description of a compatible arithmetic encoder
\end{itemize}

\begin{informative*}
\subsection{Arithmetic coding principles (INFORMATIVE)}

This section provides an introduction to the principles underlying arithmetic
coding. Since arithmetic coding is very extensively described in published literature,
this section is necessarily brief: for more information, Alasdair Moffat's
article ''Arithmetic coding revisited'' is recommended. 

Arithmetic coding is extremely powerful and can approach the Shannon information 
limit for given data. Arithmetic encoding consists of an asynchronous state machine, 
in which data
is fed to an arithmetic encoding engine, together with an estimate of its
probability, and the encoder outputs bits. It is asynchronous because data
input does not trigger any output directly, but changes the state of the 
engine. When a certain state is reached a variable amount of output is produced.

This asynchronous nature makes arithmetic coding trickier to implement 
than Variable-Length Codes (VLCs) but is essential to its optimal nature. Consider
a binary symbol $b$, with $p(b=0)=p_0$ and $p(b=1)=1-p_0$. The entropy of $b$
is the expected number of bits required to encode $b$, and is equal to
\[e(p_0)=p_0\log_2(1/p_0)+(1-p_0)\log(1/(1-p_0))\]

If $e(p_0)$ is plotted against $p_0$, it can be seen that if $p_0$ is not equal
to 0.5 exactly, $e(p_0)<1$. This means that an optimal binary entropy encoder
that operates symbol by symbol, cannot produce an output for every symbol - i.e.
it must operate asynchronously.

\subsubsection{Interval division and scaling}
The fundamental idea of arithmetic coding is interval division and scaling. An
arithmetic code can be thought of as a single number lying in an interval 
determined by the sequence of values being coded. For simplicity, this discussion
describes binary arithmetic coding, but larger symbol alphabets can be treated
in an analogous manner.
[TBC]

\subsubsection{Finite precision arithmetic}

\subsection{Arithmetic encoding (INFORMATIVE)}
\end{informative*}

This document only specifies the decoding of arithmetic coded data. 
However whilst it is clearly vital that an encoding process matches the decoding
process, it is not straightforward to derive an implementation of the
encoder by only looking only at the decoder specification. Therefore this
informative section describes a possible implementation for an
arithmetic encoding engine that will produce output decodeable by
the Dirac arithmetic decoding engine.

\subsubsection{Encoder variables}

An arithmetic encoder would require the following integer variables:

\begin{itemize}
\item $\text{low}$, a value indicating the bottom of the encoding interval
\item $\text{high}$, a value indicating the top of the encoding interval
\item $\text{underflow}$, a value tracking the number of unresolved ``straddle" conditions 
(described below)
\end{itemize}

A Dirac binary arithmetic encoder implementation will code a set of data in three stages:

\begin{enumerate}
\item Initialisation
\item Processing of all values
\item Flushing
\end{enumerate}

\subsubsection{Initialisation}

Initialisation of the arithmetic encoder is very simple -- the internal variables are
set as:

\begin{eqnarray*}
\text{low}&=&\text{0x0} \\
\text{high}&=&\text{0xFFFF} \\
\text{underflow}&=&0
\end{eqnarray*}

(Note that although the arithmetic decoder initialises $\AHigh$ to be 0x0000 in decoding the first
value, $shift\_all\_bits()$ is called which will set $\AHigh$ to 0xFFFF and $\ACode$ to the first
16 bits in the stream.)

\subsubsection{Processing of binary values}
\begin{comment}
The arithmetic coding processBits are encoded one at a time based on an estimated probability
embodied in a ``context''. As with decoding, the context simply relates to
counts of the number of zeros or ones that have been previously encoded.

A possible algorithm for encoding a single bit, given a context, is:

Encode Binary Arithmetic Coded Bit
\begin{verbatim}
write_ba(symbol, context):
    while ( ((high&0x4000)==0x0) and
           ((low&0x4000)==0x4000) ):
        code ^= 0x4000
        high ^= 0x4000
        low ^= 0x4000
        shift_bit_out()
    weight = context[0] + context[1]
    scaler = (0x10000+weight//2)//weight

    probability0 = context[0]*scaler
    range = high-low+1
    range_x_prob = (range * probability0)>>16
    if ( symbol )
        context[1] += 1
        low = low + range_x_prob
    else
        context[0] += 1
        high = low  + range_x_prob - 1
    if ( (context[0] + context[1]) > 255 ):
        #Halve counts in the context
        context[0] >> 1
        context[0] += 1
        context[1] >> 1
        context[1] += 1
    while (((high&0x8000)==0x0) and ((low&0x8000)==0x0)):
        output_bits()
        shift_bit_out()
\end{verbatim}

Shift Bit Out
\begin{verbatim}
shift_bit_out():
    high << 1
    high &= 0xFFFF
    high += 1
    low << 1
    low &= 0xFFFF
\end{verbatim}

Output Bits
\begin{verbatim}
output_bits():
    write_bit( (high&0x8000)==0x8000 )
    while ( underflow > 0 ):
        write_bit( high&0x8000)==0x0 )
        underflow -= 1
\end{verbatim}

Flush Encoder
\begin{verbatim}
flush_encoder():

    write_bit( (high&0x4000)==0x4000 )
    while ( underflow >= 0 ):
        write_bit( high&0x4000)==0x0 )
        underflow -= 1
\end{verbatim}

Where the ``write\_bit(bit)'' function outputs a single bit.
\end{comment}


\subsection{Arithmetic decoding engine}

This section is a normative specification of the operation of the arithmetic
decoding engine.

\label{arithdecodingengine}

This section specifies the arithmetic decoding engine and
processes for using it to extract data from coded streams.

The arithmetic decoding engine consists of two elements: 

\begin{itemize}
\item a collection of state variables representing the state of the arithmetic 
decoder (Section \ref{initarith})
\item a function for extracting binary values from the decoder 
and updating the decoder state (Section \ref{extractarith})
\end{itemize}

\subsubsection{State and contexts}
\label{arithcontexts}

The arithmetic decoder state consists of the following decoder state variables:

\begin{itemize}
\item $\ALow$, an integer representing the beginning of the current coding interval
\item $\ARange$, an integer representing the size of the current coding interval
\item $\ACode$, an integer within the interval from $\ALow$ to $\AHigh$, determined from the encoded bitstream
\item $\ABitsLeft$, a decrementing count of the number of bits yet to be read in
\item $\AContexts$, a map of all the contexts used in the Dirac decoder
\end{itemize}

A context $context$ is an integer array with a single value which encapsulates
the probability of zero in that context represented as a 16 bit number, such that
\[0<context[prob0]<\text{0xFFFF}\]

Contexts are accessed by decoding functions via the indices defined in Section \ref{contextindices}. 

\subsubsection{Initialisation}
\label{initarith}

At the beginning of the decoding of any data unit, the arithmetic
decoding state is initialised as follows:

\begin{pseudo}{initialise\_arithmetic\_decoding}{block\_data\_length}
\bsCODE{\ABitsLeft=8*block\_data\_length}
\bsCODE{\ALow = \text{0x0}}
\bsCODE{\ARange =\text{0x10000}}
\bsCODE{\ACode =\text{ 0x0}}
\bsFOR{i=0}{15}
    \bsCODE{\ACode <<= 1}
    \bsCODE{\ACode+= read\_bitb()} 
\bsEND
\bsCODE{init\_contexts()}
\end{pseudo}

Contexts are initialised by the $init\_contexts()$ function as follows:

\begin{pseudo}{init\_contexts}{}
\bsFOR{i=0}{length(\AContexts)-1}
  \bsCODE{\AContexts[i][prob0]=0x8000}
\bsEND
\end{pseudo}

\subsubsection{Data input}
\label{inputarith}

The arithmetic decoding process accesses data in a contiguous block of bytes
whose size is set on initialisation (Section \ref{initarith}). The bits in this
block are sufficient to allow for the
decoding of all coefficients. However, the specification of arithmetic
decoding operations in this section may occasionally cause further bits to be read,
even though they are not required for determining decoded values. For this
reason the bounded-block read function $read\_bitb()$ (Section \ref{blockreadbit}) is 
used for data access.

Since the length of arithmetically coded data elements is given in bytes within the Dirac
stream, there may be bits left unread when all values have been extracted. These
are flushed as desribed in Section \ref{blockreadbit}. Since arithmetically coded blocks
are byte-aligned and a whole number of bytes, this aligns data input with the beginning of the byte 
after the arithmetically coded data i.e. at the end of the
data chunk. $flush\_inputb()$ is always called at the end of decoding an arithmetically encoded
data element.

\begin{informative}
The Dirac arithmetic decoding engine uses 16 bit words, and so if all coefficients are
coded no more than 16 additional bits should be read beyond the end of the block. Hence it 
would seem sufficient to read in the entire block of data and pad the end with two bytes of value 0xFF,
to avoid a branch condition on inputting data
However, an arithmetically encoded block may end with a string of 1s, which an encoder could
conceivably strip out to save bits, in the knowledge that $read\_bitb()$ will re-insert them. Terminating
strings of 1s can occur (but not exclusively) in coding many zero wavelet subband coefficients at the end
of a subband. So a larger number of pad bytes may be required in practice.
\end{informative}

\subsubsection{Decoding boolean values}
\label{arithreadbool}

The arithmetic decoding engine is a multi-context, adaptive binary
arithmetic decoder, performing binary renormalisation and producing
binary outputs. For each bit decoded, the semantics of the relevant
calling decoder function determine which contexts are passed to the
arithmetic decoding operations. 

This section specifies the operation of the $read\_boola()$ function
for extracting a boolean value from the Dirac stream. The overall decoding
process for extracting a symbol is as defined by the following
pseudocode:

\begin{pseudo}{read\_boola}{context\_index}
\bsCODE{context=\AContexts[context\_index]}
\bsCODE{count = \ACode-\ALow+1}
\bsCODE{range\_times\_prob = (\ARange * context[prob0])\gg 16}
\bsIF{ count > range\_times\_prob }
  \bsCODE{value = \true}
  \bsCODE{\ALow +=  range\_times\_prob}
  \bsCODE{\ARange -= range\_time\_prob}
\bsELSE
  \bsCODE{value = \false}
  \bsCODE{\ARange = range\_times\_prob}
\bsEND
\bsCODE{update\_context(\AContexts[context\_index],value)}
\bsWHILE{\ARange<=\text{0x4000}}
    \bsCODE{renormalise()}{\ref{renormalisation}}
\bsEND
\bsRET(value)
\end{pseudo}

\begin{informative}
The function scales the probability of the symbol $0$ from the decoding context
so that if this probability were $1$, then the interval would equal that between
 $\ALow$ and 
 \[high=\ALow+\ARange-1\]
and $count$ is set to the normalised cut-off between 0 and 1 within this range.
\end{informative}

\subsubsection{Renormalisation}
\label{renormalisation}

Renormalisation is applied to stop the arithmetic decoding 
engine from losing accuracy: the range must not get too small,
 in order that 0 and 1 may be distinguished. Renormalisation is
 applied while the range is less than or equal to a quarter of 
 the total available 16-bit range ($\text{0x4000}$). 

For convenience let $low=\ALow$ and $high=\ALow+\ARange-1$ 
represent the upper and lower bounds of the interval. If the
range is $<=\text{0x4000}$ then
one of three possibilities must obtain:
\begin{enumerate}
\item the msbs of $low$ and $high$ are both 0
\item the msbs of $low$ and $high$ are both 1
\item $low=b01...$, $high=b10....$,  and the interval straddles the half-way point 0x8000. 
\end{enumerate}

Renormalisation doubles the interval and reads a bit into the codeword
as follows:

\begin{pseudo}{renormalise}{}
\bsIF{(\ALow+\ARange-1)^\ALow>=\text{0x8000}}
    \bsCODE{\ACode ^= \text{0x4000}}
    \bsCODE{\ALow ^= \text{0x4000}}
\bsEND
\bsCODE{\ALow  <<=  1}
\bsCODE{\ARange  <<=  1}
\bsCODE{\ALow \&= \text{0xFFFF}}
\bsCODE{\ACode <<= 1}
\bsCODE{\ACode+= read\_bitb()}
\bsCODE{\ACode \&= \text{0xFFFF}}
\end{pseudo}

The second bit is flipped if there is a straddle condition (case 3). The renormalisation
process has the effect that: in case 1, the interval $[low,high]$ is doubled from 0 ($x\mapsto 2*x$); 
in case 2 it is doubled from 1 ($x\mapsto 2*x-1$); and in case 3 it is doubled from 1/2 ($x\mapsto 2x-0.5$).
 
\begin{informative}
Note that if 16-bit words (unsigned shorts) are used for decoder state variables $\ALow$,
 and $\ACode$ then there is no need for {\&}-ing with 0xFFFF. However, the 
operations specified here are defined in terms of integers, since intermediate calculations
 require higher dynamic range. In software, the efficiency of using short word lengths may
or may not be offset by the requirement to cast to other data types for these calculations.
\end{informative}

\begin{comment}
\subsection{Alternative arithmetic decoding engines}

to
16 consisting of three positive values:
\begin{itemize}
\item $context[count0]$ is a count of the number of zeroes
\item $context[count1]$ is a count of the number of ones
\item $context[prob0]$ is an estimate of the probability of zero to 16 bit accuracy
\end{itemize}

Contexts are accessed by decoding functions
via the indices defined in Section \ref{contextindices}. 

Although counts are updated with each symbol decoded, the probability is only updated occasionally, as it is computationally
expensive (see Section \ref{rescalecontext} below).

\subsubsection{Rescaling contexts and probabilities}
\label{rescalecontext}

Contexts maintain counts to 8 bit accuracy, and contexts are rescaled when the total count (count0+count1) reaches 256. In addtion, prob0 is recalculated every
8 symbols in each context. A context is rescaled by halving the counts of $0$ and $1$.

\begin{pseudo}{update\_context}{context,value}
\bsIF{value==\true}
    \bsCODE{context[count1] += 1}
\bsELSE
    \bsCODE{context[count0] += 1}
\bsEND
\bsIF( (context[count0]+context[count1])\%8==0)
    \bsIF{context[0]+context[1]== 256}
      \bsCODE{context[0] += 1}
      \bsCODE{context[0] \gg= 1}
      \bsCODE{context[1] += 1}
      \bsCODE{context[1] \gg= 1}
    \bsEND
    \bsCODE{calc\_prob0(context)}
\bsEND
\end{pseudo}

The probability of zero is recalculated to approximate
\[ \frac{context[count0]*2^{16}}{context[count0]+context[count1]}\]
to 16 bit accuracy:

\begin{pseudo}{calc\_prob0}{context}
\bsCODE{weight = context[count0]+context[count1]}
\bsCODE{inverse\_weight=(2^{16}+(weight//2))//weight}
\bsCODE{context[prob0]=context[count0]*inverse\_weight}
\end{pseudo}

\begin{informative}
Note that since $context[count0]<weight$, $context[prob0]$ is always a 16 bit unsigned quantity.
The inverse weight may easily be stored within a look-up table.
\end{informative}

\subsection{Initialisation}
\label{initarith}

At the beginning of the decoding of any data unit, the arithmetic
decoding state is initialised as follows:

\begin{pseudo}{initialise\_arithmetic\_decoding}{block\_data\_length}
\bsCODE{\ABitsLeft=8*block\_data\_length}
\bsCODE{\ALow = \text{0x0}}
\bsCODE{\ARange =\text{0x10000}}
\bsCODE{\ACode =\text{ 0x0}}
\bsFOR{i=0}{15}
    \bsCODE{\ACode <<= 1}
    \bsCODE{\ACode+= read\_bitb()} 
\bsEND
\bsCODE{init\_contexts()}
\end{pseudo}

Contexts are initialised by the $init\_contexts()$ function as follows:

\begin{pseudo}{init\_contexts}{}
\bsFOR{i=0}{length(\AContexts)-1}
  \bsCODE{\AContexts[i][count0]=1}
  \bsCODE{\AContexts[i][count1]=1}
  \bsCODE{\AContexts[i][prob0]=0x8000}
\bsEND
\end{pseudo}

\subsection{Data input}
\label{inputarith}

The arithmetic decoding process accesses data in a contiguous block of bytes
whose size is set on initialisation (Section \ref{initarith}). The bits in this
block are sufficient to allow for the
decoding of all coefficients. However, the specification of arithmetic
decoding operations in this section may occasionally cause further bits to be read,
even though they are not required for determining decoded values. For this
reason the bounded-block read function $read\_bitb()$ (Section \ref{blockreadbit}) is 
used for data access.

Since the length of arithmetically coded data elements is given in bytes within the Dirac
stream, there may be bits left unread when all values have been extracted. These
are flushed as desribed in Section \ref{blockreadbit}. Since arithmetically coded blocks
are byte-aligned and a whole number of bytes, this aligns data input with the beginning of the byte 
after the arithmetically coded data i.e. at the end of the
data chunk. $flush\_inputb()$ is always called at the end of decoding an arithmetically encoded
data element.

\begin{informative}
The Dirac arithmetic decoding engine uses 16 bit words, and so if all coefficients are
coded no more than 16 additional bits should be read beyond the end of the block. Hence it 
would seem sufficient to read in the entire block of data and pad the end with two bytes of value 0xFF,
to avoid a branch condition on inputting data
However, an arithmetically encoded block may end with a string of 1s, which an encoder could
conceivably strip out to save bits, in the knowledge that $read\_bitb()$ will re-insert them. Terminating
strings of 1s can occur (but not exclusively) in coding many zero wavelet subband coefficients at the end
of a subband. So a much larger number of pad bytes may be required in practice.
\end{informative}

\subsection{Decoder functions}
\label{extractarith}
The arithmetic decoding engine is a multi-context, adaptive binary
arithmetic decoder, performing binary renormalisation and producing
binary outputs. For each bit decoded, the semantics of the relevant
calling decoder function determine which contexts are passed to the
arithmetic decoding operations.

\subsubsection{Decoding boolean values}

\label{arithreadbool}

This section specifies the operation of the $read\_boola()$ function
for extracting a boolean value from the Dirac stream. The overall decoding
process is defined for extracting a symbol is as defined by the following
pseudocode:

\begin{pseudo}{read\_boola}{context\_index}
\bsCODE{context=\AContexts[context\_index]}
\bsCODE{count = \ACode-\ALow+1}
\bsCODE{range\_times\_prob = (\ARange * context[prob0])\gg 16}
\bsIF{ count > range\_times\_prob }
  \bsCODE{value = \true}
  \bsCODE{\ALow +=  range\_times\_prob}
  \bsCODE{\ARange -= range\_time\_prob}
\bsELSE
  \bsCODE{value = \false}
  \bsCODE{\ARange = range\_times\_prob}
\bsEND
\bsCODE{update\_context(\AContexts[context\_index],value)}
\bsCODE{renormalise()}{\ref{renormalisation}}
\bsEND
\bsRET(value)
\end{pseudo}

\begin{informative}
[Describe what's going on here]
\end{informative}

\subsubsection{Renormalisation}
\label{renormalisation}

Renormalisation is applied to stop the arithmetic decoding engine from losing accuracy: the range
must not get too small to allow 0 and 1 to be distinguished. Renormalisation is applied while the
range is less than or equal to a quarter of the total available 16-bit  range:

\begin{pseudo}{renormalise}{}
\bsWHILE{\ARange<=\text{0x4000}}
    \bsIF{(\ALow+\ARange-1)^\ALow>=\text{0x8000}}
        \bsCODE{resolve\_straddle()}
    \bsEND
    \bsCODE{shift\_bits()}
\bsEND
\end{pseudo}

\begin{informative}
Let the bottom of the arithmetic coding interval is represented by $low=\ALow$ and the top by $high=\ALow+\ARange-1$.
When the range is one quarter or less of the original range ($2^{16}$), then one of three possibilities occurs:
\begin{enumerate}
\item the top bits of $low$ and $high$ are both 0
\item the top bits of $low$ and $high$ are both 1
\item $low=b01...$, $high=b10....$,  and the interval straddles the half-way point 0x8000. 
\end{enumerate}

In all of these cases the interval can be doubled in size to increase accuracy. In the first case, it is doubled from zero ($x\mapsto 2*x$); 
in the second it is doubled from 1 ($x\mapsto 2*x-1$); and in the third it is doubled from 1/2 ($x\mapsto 2x-0.5$).
 
\end{informative}

\begin{pseudo}{resolve\_staddle}{}
\bsCODE{\ACode ^= \text{0x4000}}
\bsCODE{\ALow ^= \text{0x4000}}
\end{pseudo}

\begin{pseudo}{shift\_bits}{}
\bsCODE{\ALow  <<=  1}
\bsCODE{\ARange  <<=  1}
\bsCODE{\ALow \&= \text{0xFFFF}}
\bsCODE{\ACode <<= 1}
\bsCODE{\ACode+= read\_bitb()}
\bsCODE{\ACode \&= \text{0xFFFF}}
\end{pseudo}

\begin{comment}

\begin{informative}
The function scales the probability of the symbol $0$ from the decoding context
so that if this probability were $1$, then the interval would equal that between
 $\ALow$ and $\AHigh$. If $\ACode$ is greater than this cut-off, then 1 ($\true$) has
been encoded, else 0 ($\false$) has.
\end{informative}

\subsubsection{Shifting bits in}

\label{arithshiftin}

This section defines the operation of the $shift\_bit\_in()$ 
and $shift\_all\_bits()$ functions
for reading bits into the arithmetic decoding state variables.

\begin{pseudo}{shift\_bit\_in}{}
\bsCODE{\AHigh \ll= 1}
\bsCODE{\AHigh \&= \text{0xFFFF}}
\bsCODE{\AHigh += 1}
\bsCODE{\ALow \ll= 1}
\bsCODE{\ALow \&= \text{0xFFFF}}
\bsCODE{\ACode \ll= 1}
\bsCODE{\ACode \&= \text{0xFFFF}}
\bsCODE{\ACode += read\_bitb()}{\ref{blockreadbit}}
\end{pseudo}

$shift\_all\_bits()$ expands the interval between $\ALow$ and $\AHigh$
until the msbs (bit 15) differ and the interval no longer
straddles the half-way point 0x8000.

\begin{pseudo}{shift\_all\_bits}{}
\bsWHILE{ \AHigh\&\text{0x8000})==\text{0x0} \text{ and } (\ALow\&\text{0x8000})==\text{0x0}}
  \bsCODE{shift\_bit\_in()}
\bsEND
\bsWHILE{ (\AHigh\&\text{0x4000})==\text{0x0} \text{ and } (\ALow\&\text{0x4000})==\text{0x4000} }
  \bsCODE{\ACode \wedge= \text{0x4000}}
  \bsCODE{\AHigh \wedge= \text{0x4000}}
  \bsCODE{\ALow \wedge= \text{0x4000}}
  \bsCODE{shift\_bit\_in()}
\bsEND
\end{pseudo}

\begin{informative}
Note that if 16-bit words (unsigned shorts) are used for decoder state variables $\ALow$,
 $\AHigh$ and $\ACode$ then there is no need for {\&}-ing with 0xFFFF. However, the 
operations specified here are defined in terms of integers, since intermediate calculations
 require higher dynamic range. In software, the efficiency of using short word lengths may
or may not be offset by the requirement to cast to other data types for these calculations.
\end{informative}



\end{comment}