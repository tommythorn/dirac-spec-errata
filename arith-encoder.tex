
%src: tim-0.9.1.48

This document only specifies the decoding of arithmetic coded data.
However it is important that the encoding process matches the decoding
process. It is not straightforward to derive an implementation of the
encoder by only looking at the decoder specification. Therefore this
informative section describes a possible implementation for the
arithmetic encoder.

An arithmetic encoder would require the following variables.

\begin{verbatim}
Global Variables used for arithmetic encoding
low        #integer
high       #integer
underflow  #integer
\end{verbatim}

In an encoder implementation the following operations must be performed.

\begin{verbatim}
Arirhmetic Encoding Process
initialise_arithmetic_encoding()
.
.
perform arithmetic encoding
.
.
flush_encoder()
\end{verbatim}

Initialisation of arithmetic encoding is straightforward.

\begin{verbatim}
Initialise Arithmetic Encoding Engine

low       = 0x0000
high      = 0xFFFF
underflow = 0x0000
\end{verbatim}

Bits are encoded one at a time based on an estimated probability
embodied in a ``context''. As with decoding, the context simply relates to
counts of the number of zeros or ones that have been previously encoded.

A possible algorithm for encoding a single bit, given a context, is:

Encode Binary Arithmetic Coded Bit
\begin{verbatim}
write_ba(symbol, context):
    while ( ((high&0x4000)==0x0) and
           ((low&0x4000)==0x4000) ):
        code ^= 0x4000
        high ^= 0x4000
        low ^= 0x4000
        shift_bit_out()
    weight = context[0] + context[1]
    scaler = (0x10000+weight//2)//weight

    probability0 = context[0]*scaler
    range = high-low+1
    range_x_prob = (range * probability0)>>16
    if ( symbol )
        context[1] += 1
        low = low + range_x_prob
    else
        context[0] += 1
        high = low  + range_x_prob - 1
    if ( (context[0] + context[1]) > 255 ):
        #Halve counts in the context
        context[0] >> 1
        context[0] += 1
        context[1] >> 1
        context[1] += 1
    while (((high&0x8000)==0x0) and ((low&0x8000)==0x0)):
        output_bits()
        shift_bit_out()
\end{verbatim}

Shift Bit Out
\begin{verbatim}
shift_bit_out():
    high << 1
    high &= 0xFFFF
    high += 1
    low << 1
    low &= 0xFFFF
\end{verbatim}

Output Bits
\begin{verbatim}
output_bits():
    write_bit( (high&0x8000)==0x8000 )
    while ( underflow > 0 ):
        write_bit( high&0x8000)==0x0 )
        underflow -= 1
\end{verbatim}

Flush Encoder
\begin{verbatim}
flush_encoder():

    write_bit( (high&0x4000)==0x4000 )
    while ( underflow >= 0 ):
        write_bit( high&0x4000)==0x0 )
        underflow -= 1
\end{verbatim}

Where the ``write\_bit(bit)'' function outputs a single bit.
