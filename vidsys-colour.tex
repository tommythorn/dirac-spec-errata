All current video systems use the following model for YUV coding of the
RGB values (computer systems often omit coding to and from YUV). 

The R, G and B are tristimulus values (e.g. candelas/meter2). Their
relationship to CIE XYZ tristimulus values can be derived from the set
of primaries and white point defined in the colour primaries part of the
colour specification below using the method described in SMPTE RP
177-1993. In this document the RGB values are normalised to the range
[0,1], so that RGB=1,1,1 represents the peak white of the display device
and RGB=0,0,0 represents black.

The ER, EG, EB values, are related to the RGB values by non-linear
transfer functions labelled “f()” and “g()” in the
diagram. Normally, these values also fall in the range [0,1], but in the
case of extended gamut, negative values may be allowed also. The
transfer function “f()” is typically performed in the camera and
is specified in the “Transfer Characteristic” part of the
“Colour Specification”. For aesthetic and psychovisual reasons
the transfer function “g()” is not quite the inverse of
“f()”. In fact the combined effect of “f()” and
“g()” is such that



where γ is the “rendering intent” or end to end gamma of the
system, which may vary between about 1.1 and 1.6 depending on viewing
conditions. The rationale for this is given in [Digital Video and HDTV,
Charles Poynton 2003, Morgan Kaufmann Publishers, ISBN 1-55860-792-7].

The non-linear ER, EG, EB values are subject to a matrix operation
(known as “non-constant luminance coding”), which transforms
them into luma (EY) and chroma (normally ECb and ECr, but sometimes ECg
and ECo). EY is normally limited to the range [0,1] and the chroma
values to the range [-0.5, 0.5]. This is “YUV” coding and
sometimes the chroma components are subsampled, either horizontally or
both horizontally and vertically. UV sampling is specified by the
CHROMA\_FORMAT value. 

The EY, ECb, ECr (or EY, ECg, ECo) values are mapped to a range of
integers Y, Cb, Cr (Y, Cg, Co). Typically they are mapped to an 8 bit
range [0, 255]. The way this mapping occurs is defined by the signal
range parameters. It is these integer values that are actually output
from the decoder.In order to display video, the inverse to the above
operations must be performed to convert this data to EY, ECb, ECr, then
to ER, EG, EB and thence to R, G and B.  

The E values can be viewed as something of a mathematical abstraction.
For example in digital display devices, R, G and B values are specified
in terms of integer levels which are derived from the integral luma and
chroma values by direct operations subsuming and approximating all the
real-number operations described here. Generally, these approximations
cause loss through quantisation of intermediate values, and the
restriction of values to particular ranges also restricts the colour
gamut. In the case of YCgCo coding, a lossless direct integer transform
is used, so that in this mode (together with 4:4:4 sampling and lossless
compression), Dirac supports lossless RGB coding.



