\label{wltdecodeconventions}

\subsubsection{Wavelet data initialisation}

\label{wltinit}

The $transform\_data()$ process begins with an initialisation process, 
$initialise\_data()$, which returns a structure which will
contain the decoded wavelet coefficients for the component. 

For the purposes of this specification, this is a four-dimensional array $data$,
where individual subbands are two-dimensional arrays accessed by level and depth:

$band = data[level][orientation]$

Valid levels are integers from in the range 0 to \TransformDepth inclusive. 
Level 0 consists of a single subband with orientation \LL. 
All other levels consist of 3 subbands of orientation \LH, \HL, 
and \HH.

Individual subband coefficients are signed integers accessed by vertical and 
horizontal coordinates within the subband:

$c = band[y][x], x\in\{0, ... , subband\_width(level)-1\}, 
y\in\{0, ... , subband\_height(level)-1\}$

where the dimensions $subband\_width(level)$ and $subband\_height(level)$ of the subband
are as defined in Section 
\ref{subbandwidthheight}. These dimensions correspond to a wavelet transform
being performed on a copy of the component data which has been padded (if necessary) so that its
dimensions are a multiple of $2^{\TransformDepth}$.


\subsubsection{Dimensions of wavelet subbands}
\label{subbandwidthheight}

This section defines the values of the $subband\_width(level)$ and $subband\_height(level)$
functions, giving the width and height of subbands at a given level, and hence the range
of subband vertical and horizontal indices. 

Define the padded dimensions of the component by

\begin{eqnarray*}
ph = 2^{\TransformDepth}*\lceil\frac{\ComponentHeight}{2^{\TransformDepth}}\rceil \\
pw = 2^{\TransformDepth}*\lceil\frac{\ComponentWidth}{2^{\TransformDepth}}\rceil
\end{eqnarray*}

If $level==0$,

\begin{eqnarray*}
subband\_height(level)=ph//2^{\TransformDepth} \\
=\lceil\frac{\ComponentHeight}{2^{\TransformDepth}}\rceil \\
subband\_width(level)=pw//2^{\TransformDepth} \\
=\lceil\frac{\ComponentWidth}{2^{\TransformDepth}}\rceil
\end{eqnarray*}

If $level>0$
\begin{eqnarray*}
subband\_height(level)=ph//2^{\TransformDepth-level+1} \\
=2^{level-1}*\lceil\frac{\ComponentHeight}{2^{\TransformDepth}}\rceil \\
subband\_width(level)=pw//2^{\TransformDepth-level+1} \\
=2^{level-1}*\lceil \frac{\ComponentWidth}{2^{\TransformDepth}}\rceil
\end{eqnarray*}

\begin{informative}
In encoding, these padded dimensions may be achieved by padding the 
component data up to the padded dimensions and applying the forward
Discrete Wavelet Transform (the inverse of the operations specified in
Section \ref{idwt}). Any values may be used for the padded data, although
the choice will affect wavelet coefficients at the right and bottom 
edges of the subbands. Good results, in compression terms, may be obtained
 by using edge extension for intra pictures and zero extension for inter 
pictures. A poor choice of padding may cause visible artefacts near the
bottom and right edges at high levels of compression.
\end{informative}
