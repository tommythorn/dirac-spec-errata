
The \verb|parse_info_prefix| is the sequence of bytes
``\verb|0x42, 0x42, 0x43, 0x44|''\annotate{df}{tim-77 has the last byte incorrectly as
0x43}, which are the ASCII codes for ``\verb|BBCD|''.

The \verb|parse_code| is a bit field:
bits 7, 6 \& 5 (MSBs) unused, always zero.
bits 4 \& 3 indicate parse unit type (i.e. Access Unit Header, Picture
or End of Sequence), (value 11 undefined).
bits 2, 1 \& 0(LSBs) specify picture properties, (zero for non
pictures).\annotate{df}{do we want to specify these as zero?}
bit 2 indicatesa reference picture (else non reference picture).
bits 1 \& 0 indicate the number of reference pictures (value 11(binary)
undefined).

\verb|video_depth| signifies the number of bits used to code the
uncompressed input signal, typically eight or ten bits.  It would be
possible, for example, to have an 8 bit signal represented in a 10 bit
word (in which case either the upper two, or lower two bits of the word
would always be zero).  The meaning of the bits defined in the Signal
Range part of the Source Parameters.  Video Depth relates to how the
coded.  The Signal Range relates to what the numbers mean and how the
video should be displayed.

The \verb|field_dominance_flag| if asserted means that you do not use
the default field dominance.  If we have an interlaced source, the field
lines can either be interleaved line by line (pseudo-progressive
formate, the default for Dirac), or interleaved field by field
(sequential fields, required for low delay and low resource coding).

The \verb|field_interleaving_flag| indicates non-default field
interleaving, and the \verb|sequential_fields| parameter indicates
whether the fields are interleaved as pseudo-progressive or sequential
fields.  With field sequential coding the picture sequence is a sequence
of fields rather than frames. So for example, for interlaced 625 line
video we would have picture size 720x288, frame rate 25Hz and
\verb|sequential_fields| true.

Field parity is the same as picture parity.  Each picture has a unique
picture number.  If the picture number is even and the picture is a
field, then that field has even parity, i.e. it is the first field of a
pair of fields in a frame.\annotate{df}{Parity, dominance,
topfieldfirst, so many ways to say the same thing}

\verb|aspect_ratio| refers to the pixel aspect ratio, not the image
aspect ratio.\annotate{df}{then lets rename it to
pixel\_aspect\_ratio!}


