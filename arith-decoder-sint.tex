This section defines the operation of the $read\_sinta(context\_set)$ function
 for extracting integer values from a block of arithmetically coded data.

Integer values are binarised using interleaved exp-Golomb binarisation as per
Section ??. The set $context\_set$ of contexts therefore consists of three parts: 
an array of follow contexts, $context\_set[follow]$ (indexed from 0 to 
$len(context\_set[follow])$, a single data context 
$context\_set[data]$ and a sign context $context\_set[sign]$.

The $read\_sinta()$ process is identical to the $read\_sint()$ process
specified in Section ??, except that instances of $read\_bool()$ are replaced
by instances of $read\_ba()$ using suitable contextualisation. It uses $read\_uinta()$
as for decoding the magnitude:


\begin{pseudo}{read\_uinta}{context\_set}
\bsCODE{value = 1}
\bsCODE{index = 0}
\bsWHILE{read\_ba(follow\_context(index,context\_set) )== \false}
  \bsCODE{value \ll = 1}
  \bsIF{read\_ba(\text{state[contexts]}[context\_set[data])])}
    \bsCODE{value += 1}
  \bsEND
  \bsCODE{index += 1}
\bsEND
\bsCODE{value -= 1}
\bsRET{value}
\end{pseudo}

Each follow context is used for decoding the corresponding follow bit, with the
last follow context being used for all subsequent follow bits (if any) also. 
The follow context selection function $follow\_context()$ is defined by:

\begin{pseudo}{follow\_context}{index, context\_set}
\bsCODE{pos= max(index, len(context\_set[follow])-1 }
\bsCODE{ctx_index = context\_set[follow][pos]}
\bsRET{\Contexts[ctx_index]}
\end{pseudo}

$read\_sinta()$ decodes first the magnitude then the sign, as necessary:

\begin{pseudo}{read\_sinta}{context\_set}
\bsCODE{value=read\_uinta(context_set)}
\bsIF{value != 0}
  \bsIF{read\_ba(\Contexts[context\_set[sign])]) == \true}
    \bsCODE{value=-value}
  \bsEND
\bsEND
\end{pseudo}
