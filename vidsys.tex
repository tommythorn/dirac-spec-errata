\label{vidsys}

The interpretation of sequence source parameters (Section \ref{sourceparameters})
by a display mechanism interfacing with a compliant decoder is non-normative. However, it
should where possible follow the recommendations and interpretations
specified in this section. Likewise, encoders should ensure that
accurate display parameter information is encoded to maximise the
quality of displayed video.

[Include discussion of YCgCo here. Different matrixing requirements]

\begin{informative*}
\subsection{Colour}
All current video systems use the following model for YUV coding of the
RGB values (computer systems often omit coding to and from YUV). 

The R, G and B are tristimulus values (e.g. candelas/meter2). Their
relationship to CIE XYZ tristimulus values can be derived from the set
of primaries and white point defined in the colour primaries part of the
colour specification below using the method described in SMPTE RP
177-1993. In this discussion the RGB values are normalised to the range
[0,1], so that RGB=[1,1,1] represents the peak white of the display device
and RGB=0,0,0 represents black.

Values $E_R$, $E_G$ and $E_B$ are are defined which are related to the RGB 
values by non-linear forward and inverse transfer functions $f()$ and $g()$ respectively
(Figure \ref{fig:transferchar}).

[figure \label{fig:transferchar}]

Normally, $E_R$, $E_G$ and $E_B$ also fall in the range $[0,1]$, but in the
case of extended gamut, negative values may be allowed also. The
transfer function $f()$ is typically performed in the camera and
is specified in the Transfer Characteristic part of the
Colour Specification. For aesthetic and psychovisual reasons
the transfer function $g()$ is not quite the inverse of
$f()$. In fact the combined effect of $f()$ and
$g()$ is such that:
\[g(f(x))=x^\gamma\]

where $\gamma$ is the ''rendering intent"' or end to end gamma of the
system, which may vary between about 1.1 and 1.6 depending on viewing
conditions. The rationale for this is given in ''Digital Video and HDTV"',
Charles Poynton 2003, Morgan Kaufmann Publishers, ISBN 1-55860-792-7.

The non-linear $E_R$, $E_G$ and $E_B$ values are subject to a matrix operation
(known as ''non-constant luminance coding"), which transforms
them into luma ($E_Y$) and chroma (normally $E_{Cb}$ and $E_{Cr}$) values. 
$E_Y$ is normally limited to the 
range $[0,1]$ and the chroma
values to the range $[-0.5, 0.5]$. This is YUV coding and
sometimes the chroma components are subsampled, either horizontally or
both horizontally and vertically. UV sampling is specified by the
$\ChromaFormat$ value. 

\subsubsection{Conventional YUV coding}
In conventional YUV coding, the $E_Y$, $E_{Cb}$ and $E_{Cr}$ values are
mapped to a range of integers denoted $Y$, $C1$ and $C2$ within this
specification. In this way, $C1$ typically corresponds to $Cb$ and
$C2$ to $Cr$. The way this mapping occurs is defined by the signal
range parameters (Section \ref{signalranges}). It is these integer values 
that are actually output from the decoder. In order to display video, the inverse to the above
operations must be performed to convert this data to $E_Y$, $E_{Cb}$, $E_Cr$, then
to $E_R$, $E_G$, $E_B$ and thence to R, G and B.  

\subsubsection{YCgCo coding}
The E values can be viewed as something of a mathematical abstraction.
For example in digital display devices, R, G and B values are specified
in terms of integer levels which are derived from the integral luma and
chroma values by direct operations subsuming and approximating all the
real-number operations described here. Generally, these approximations
cause loss through quantisation of intermediate values, and the
restriction of values to particular ranges also restricts the colour
gamut. 

In the case of YCgCo coding, the $E_R$, $E_G$ and $E_B$ values are directly
linearly scaled to integer ranges $[0,2^\VideoDepth-1]$ before a lossless 
direct integer transform is applied to convert this data to $Y$, $C1$ (representing
Cg) and $C2$ (representing $Co$) data. This transform is described in Section
\ref{colourmatrix}. This supports efficient lossless RGB coding.

\subsection{Signal range}

\subsection{Colour primaries}

\subsection{Colour matrix}

\subsection{Transfer charactaristics}

\subsection{Frame rate}

\subsection{Aspect ratios and clean area}

\end{informative*}
\subsection{Source parameter presets}\label{sourceparamspresets}

Source parameters are signalled using a range of preset indices into the following
tables, as specified in Section \ref{auheader}. 

\begin{table}[!h]
\centering
\begin{tabular}{|c|c|c|}
\hline
$index$ & $\SFrameRateNumerator$ & $\SFrameRateDenominator$ \\
\hline
1 & 24000 & 1001 \\
\hline
2 & 24 & 1 \\
\hline
3 & 25 & 1 \\
\hline
4 & 30000 & 1001 \\
\hline
5 & 30 & 1 \\
\hline
6 & 50 & 1 \\
\hline
7 & 60000 & 1001 \\
\hline
8 & 60 & 1 \\
\hline
\end{tabular}
\caption{Available preset frame rate values}\label{frameratevalues}
\end{table}

\begin{table}[!h]
\centering
\begin{tabular}{|c|c|c|}
\hline
$index$ & $\SAspectRatioNumerator$ & $\SAspectRatioDenominator$ \\
\hline
1 (Square Pixels) & 1 & 1 \\
\hline
2 (525-line systems) & 10 & 11 \\
\hline
3 (625-line systems) & 12 & 11 \\
\hline
\end{tabular}
\caption{Available preset aspect ratio values}\label{aspectratiovalues}
\end{table}

\begin{table}[!h]
\centering
\begin{tabular}{|c|c|c|}
\hline
$index$ & $\SLumaOffset$ & $\SLumaExcursion$ \\
\hline
1 (8 Bit Full Range) & 0 & 255 \\
\hline
2 (8 Bit Video) & 16 & 235 \\
\hline
3 (10 Bit Video) & 64 & 876 \\
\hline
\end{tabular}
\caption{Luma signal range available presets}\label{lumasignalrangevalues}
\end{table}

\begin{table}[!h]
\centering
\begin{tabular}{|c|c|c|}
\hline
$index$ & $\SChromaOffset$ & $\SChromaExcursion$ \\
\hline
1 (8 Bit Full Range) & 128 & 255 \\
\hline
2 (8 Bit Video) & 128 & 224 \\
\hline
3 (10 Bit Video) & 512 & 896 \\
\hline
\end{tabular}
\caption{Chroma signal range available presets}\label{chromasignalrangevalues}
\end{table}

\begin{informative}
The only presets available cover a full 8-bit range or 8- or 10-bit SDI video ranges.
If other video depths have been selected, custom signal range parameters
should be signalled, or the resulting video may have an unintended appearance that
affects video quality adversely on a display device.
\end{informative}

\begin{table}[!h]
\centering
\begin{tabular}{|c|c|l|l|l|}
\hline
$index$ & {\bf Description}           & {\bf Primaries}       & {\bf Matrix}  & {\bf Transfer function}\\
\hline
0 & Custom, HDTV, PC \& Internet & 0 - ITU709 \& sRGB & 0 - HDTV/PC & 0 - TV \\ 
\hline
1 & NTSC                                    & 1 - SMPTE C           & 1 - SDTV      & 0 - TV \\
\hline
2 & PAL                                       & 2 - EBU Tech 3213   & 1 - SDTV      & 0 - TV \\
\hline
3 & D-Cinema                              & 3 - CIE XYZ             & 2 - YCgCo     & 3 - DCI \\
\hline
\end{tabular}
\caption{Colour specification presets}\label{colourspecvalues}
\end{table}

\begin{table}[!h]
\centering
\begin{tabular}{|c|l|}
\hline
$index$ &  {\bf Primaries}      \\
\hline
0          &  ITU709 \& sRGB  \\ 
\hline
1          &  SMPTE C (as used for NTSC)          \\
\hline
2          &  EBU Tech 3213, as used for PAL   \\
\hline
3          & CIE XYZ              \\
\hline
\end{tabular}
\caption{Colour primaries presets}\label{primariesvalues}
\end{table}

\begin{table}[!h]
\centering
\begin{tabular}{|c|l|}
\hline
$index$ &  {\bf Matrix}\\
\hline
0 & HDTV, PC \& Internet: $K_R=0.2126$, $K_B=0.0722$ \\ 
\hline
1 & SDTV: $K_R=0.299$, $K_B=0.114$ \\
\hline
2 & Reversible: YCgCo \\
\hline
\end{tabular}
\caption{Colour matrix presets}\label{matrixvalues}
\end{table}

\begin{table}[!h]
\centering
\begin{tabular}{|c|l|}
\hline
$index$ & {\bf Transfer function}\\
\hline
0 & TV \\ 
\hline
1 & Extended Gamut \\
\hline
2 & Linear\\
\hline
3 & DCI Gamma\\
\hline
\end{tabular}
\caption{Transfer function presets}\label{transfervalues}
\end{table}

[Must explain all these presets]


