\label{vidsys}

The interpretation of sequence source parameters (Section \ref{sourceparameters})
by a display mechanism interfacing with a compliant decoder is non-normative. However, it
should where possible follow the recommendations and interpretations
specified in this section. Likewise, encoders should ensure that
accurate display parameter information is encoded to maximise the
quality of displayed video.

[Include discussion of YCgCo here. Different matrixing requirements]

\begin{informative*}
\subsection{Colour}
All current video systems use the following model for YUV coding of the
RGB values (computer systems often omit coding to and from YUV). 

The R, G and B are tristimulus values (e.g. candelas/meter2). Their
relationship to CIE XYZ tristimulus values can be derived from the set
of primaries and white point defined in the colour primaries part of the
colour specification below using the method described in SMPTE RP
177-1993. In this discussion the RGB values are normalised to the range
[0,1], so that RGB=[1,1,1] represents the peak white of the display device
and RGB=0,0,0 represents black.

Values $E_R$, $E_G$ and $E_B$ are are defined which are related to the RGB 
values by non-linear forward and inverse transfer functions $f()$ and $g()$ respectively
(Figure \ref{fig:transferchar}).

[figure \label{fig:transferchar}]

Normally, $E_R$, $E_G$ and $E_B$ also fall in the range $[0,1]$, but in the
case of extended gamut, negative values may be allowed also. The
transfer function $f()$ is typically performed in the camera and
is specified in the Transfer Characteristic part of the
Colour Specification. For aesthetic and psychovisual reasons
the transfer function $g()$ is not quite the inverse of
$f()$. In fact the combined effect of $f()$ and
$g()$ is such that:
\[g(f(x))=x^\gamma\]

where $\gamma$ is the ''rendering intent"' or end to end gamma of the
system, which may vary between about 1.1 and 1.6 depending on viewing
conditions. The rationale for this is given in ''Digital Video and HDTV"',
Charles Poynton 2003, Morgan Kaufmann Publishers, ISBN 1-55860-792-7.

The non-linear $E_R$, $E_G$ and $E_B$ values are subject to a matrix operation
(known as ''non-constant luminance coding"), which transforms
them into luma ($E_Y$) and chroma (normally $E_{Cb}$ and $E_{Cr}$) values. 
$E_Y$ is normally limited to the 
range $[0,1]$ and the chroma
values to the range $[-0.5, 0.5]$. This is YUV coding and
sometimes the chroma components are subsampled, either horizontally or
both horizontally and vertically. UV sampling is specified by the
$\ChromaFormat$ value. 

\subsubsection{Conventional YUV coding}
In conventional YUV coding, the $E_Y$, $E_{Cb}$ and $E_{Cr}$ values are
mapped to a range of integers denoted $Y$, $C1$ and $C2$ within this
specification. In this way, $C1$ typically corresponds to $Cb$ and
$C2$ to $Cr$. The way this mapping occurs is defined by the signal
range parameters (Section \ref{signalranges}). It is these integer values 
that are actually output from the decoder. In order to display video, the inverse to the above
operations must be performed to convert this data to $E_Y$, $E_{Cb}$, $E_Cr$, then
to $E_R$, $E_G$, $E_B$ and thence to R, G and B.  

\subsubsection{YCgCo coding}
The E values can be viewed as something of a mathematical abstraction.
For example in digital display devices, R, G and B values are specified
in terms of integer levels which are derived from the integral luma and
chroma values by direct operations subsuming and approximating all the
real-number operations described here. Generally, these approximations
cause loss through quantisation of intermediate values, and the
restriction of values to particular ranges also restricts the colour
gamut. 

In the case of YCgCo coding, the $E_R$, $E_G$ and $E_B$ values are directly
linearly scaled to integer ranges $[0,2^\VideoDepth-1]$ before a lossless 
direct integer transform is applied to convert this data to $Y$, $C1$ (representing
Cg) and $C2$ (representing $Co$) data. This transform is described in Section
\ref{colourmatrix}. This supports efficient lossless RGB coding.

\subsection{Signal range}The offset and excursion values should be used to convert the
integer-valued decoded luma and chroma data Y, Cb, Cr to intermediate
values EY, ECr, and ECb by the recipe



EY, is normally clipped to the range [0,1], and ECr, and ECb to the
range [-0.5,0.5]. This effectively clips Y to 

[LUMA\_OFFSET, LUMA\_OFFSET+LUMA\_EXCURSION]

and Cb, Cr to

[CHROMA\_OFFSET-LUMA\_EXCURSION/2, LUMA\_OFFSET+LUMA\_EXCURSION/2]

However, maintaining an extended RGB gamut may mean that either such
clipping is not done, or non-standard offset and excursion values are
used to extract the extended gamut from the non-negative decoded Y, Cr,
and Cb values.

Non-default offset and excursion values cannot be coded if the chroma
format is YCgCo: default parameters should be used. However, even in
this case, EY, ECg, and ECo should not be calculated. Instead, direct
integer conversion to RGB should be done as described in Section . (In
fact, excursion values will be ignored in this integer conversion.)


\subsection{Colour primaries}The colour primaries allow device dependent linear RGB colour
co-ordinates to be mapped to device independent linear CIE XYZ space.
The primaries specified below are the CIE (1931) XYZ chromaticity
co-ordinates of the primaries and the white point of the device. The
maths required to convert between RGB and XYZ is reproduced below.



The colour primary specification therefore allows exact colour
reproduction of decoded RGB values on different displays with different
display primaries. It has to be said that often conversion between
encoded primaries and display primaries is not done.


\subsection{Colour matrix}\label{colourmatrix}
Luma and chroma values EY, ECb, ECr should be used to derive ER, EG, EB
values by the following equations.



This follows by inverting the equations 



In the case of YCgCo coding, ER, EG, EB should be directly computed from
the integer Y, Cg and Co values by the following recipe, whereby integer
RGB IR, IG, IB values are decoded by

Y-=LUMA\_OFFSET

Cg-=CHROMA\_OFFSET

Co-=CHROMA\_OFFSET

TEMP=Y-(Cg>>1)

IG=TEMP+Cg

IB=TEMP-(Co>>1)

IR=IB+Co

These may be scaled down by dividing by (255<<ACC\_BITS) and clipped to
[0,1] to give ER, EG, EB. If the inverse transform has been correctly
applied prior to coding and lossless coding employed, then clipping will
be unnecessary.

Note that this matrix implies that the chroma range is twice as large as
the RGB range (and the luma range), since the chroma components involve
subtraction. Although logically knowing the signal range and scaling
signals is prior to performing matrixing, the matrix parameters are
coded first in the Display Parameters in order to allow the signal
ranges to be correctly determined in this case.



\subsection{Transfer charactaristics}TV transfer characteristic

Denoting R or G or B as “L” (light) and ER, EG, EB as
“E” then E=f(L) such is that;



All modern TV systems use this transfer characteristic at present. ITU-R
BT 470 (Conventional Television systems PAL, NTSC and SECAM) specifies
an “assumed gamma value of the receiver for which the primary
signals are pre-corrected” as 2.2 for NTSC and 2.8 for PAL. This
specification is incomplete, incorrect and obsolete and modern PAL and
NTSC systems use the “TV” transfer characteristic above.


Extended Colour Gamut

ITU-R BT 1361, Worldwide unified colorimetry of future TV systems
defines a transfer characteristic for systems with an extended colour
gamut as follows.

Denoting R or G or B as “L” (light) and ER, EG, EB as
“E” then E=f(L) such that;



This transfer characteristic is intended to be used with systems using
an extended colour gamut.

Linear

A linear transfer characteristic has f(x)=x. 


\subsection{Frame rate}The FRAME\_RATE value encodes the intended rate at which frames should be
displayed subsequent to decoding. If INTERLACE is TRUE, then fields are
displayed at double the frame rate, in the order specified by the
TOP\_FIELD\_FIRST flag.


\subsection{Aspect ratios and clean area}The PIXEL\_ASPECT\_RATIO value of an image is the ratio of the intended
spacing of horizontal samples (pixels) to the spacing of vertical
samples (picture lines) on the display device. Pixel aspect ratios are
fundamental properties of sampled images because they determine the
displayed shape of objects in the image. Failure to use the right value
of PIXEL\_ASPECT\_RATIO will result in distorted images – for example,
circles will be displayed as ellipses and so forth. 


Some HDTV standards and computer image formats are defined to have pixel
aspect ratios that are exactly 1:1.

The clean area is intended to define an area within which picture
information is subjectively uncontaminated by all edge transient (and
other) distortions. It may only be appropriate to display the clean area
rather than the whole picture, which may be distorted at the edge. 

The top-left corner of the clean area has coordinates (CLEAN\_TL\_X,
CLEAN\_TL\_Y) and dimensions CLEAN\_WIDTHxCLEAN\_HEIGHT.

The clean area and the pixel aspect ratio determine the
IMAGE\_ASPECT\_RATIO which is the ratio of the width of the intended
display area to the height of the intended display area. 

Given two separate sequences, with identical IMAGE\_ASPECT\_RATIO, if the
top left corner and bottom left corners of their clean apertures are
coincident when displayed, then the images as a whole should be exactly
coincident. This is regardless of the actual pixel dimensions of the
images or their clean areas. This allows sequences to be combined
together appropriately if they are appropriately scaled.


The defined pixel aspect ratios are designed to give standard image
aspect ratios for typical TV broadcasts. For example, for a 525 line
(American) 704 x 480 (clean area)  picture the image aspect ratio is
(704 x 10)/(480 x 11) which is exactly 4:3.


For 625 line systems the 59:54 pixel aspect ratio means (less
conveniently) that a 702.9x576 image would have an exact 4:3 image
aspect ratio. It might be argued that the pixel aspect ratio for 625
line systems should be such that a 702x576 image would have an exact 4:3
image aspect ratio. It could be said that this corresponds to the
analogue 625 line TV specification. This requirement would lead to a
pixel aspect ratio of 128:117. However, the tolerance of the analogue
line length is 702 3 pixels, which does not really seem to justify a
ratio of exactly 128:117.

The values specified here are generally agreed to be the
“correct” values. Then again not everyone agrees with this
consensus. These arise from the “industry standard” sampling
frequencies used for square pixels, which were originally designed for
digitising composite analogue video signals. These “industry
standard” sampling frequencies are 11+3/11 MHz for 525 line systems
and 14.75MHz for 625 line systems. The ratio of these frequencies to the
(standardised) 13.5MHz sampling frequency used for broadcasting yields
the pixel aspect ratios given in  and .

You are strongly advised to use one of the default pixel aspect ratios.
However, if you know what you are doing and don’t like the default
values you can define your own ratio. You should be aware that many
display devices may ignore your decision and may use different and
unsuitable values. 


\end{informative*}
\subsection{Source parameter presets}\label{sourceparamspresets}

Source parameters are signalled using a range of preset indices into the following
tables, as specified in Section \ref{auheader}. 

\begin{table}[!ht]
\centering
\begin{tabular}{|c|c|c|}
\hline
$index$ & $\SFrameRateNumerator$ & $\SFrameRateDenominator$ \\
\hline
1 & 24000 & 1001 \\
\hline
2 & 24 & 1 \\
\hline
3 & 25 & 1 \\
\hline
4 & 30000 & 1001 \\
\hline
5 & 30 & 1 \\
\hline
6 & 50 & 1 \\
\hline
7 & 60000 & 1001 \\
\hline
8 & 60 & 1 \\
\hline
\end{tabular}
\caption{Available preset frame rate values}\label{table:frameratevalues}
\end{table}

\begin{table}[!ht]
\centering
\begin{tabular}{|c|c|c|}
\hline
$index$ & $\SAspectRatioNumerator$ & $\SAspectRatioDenominator$ \\
\hline
1 (Square Pixels) & 1 & 1 \\
\hline
2 (525-line systems) & 10 & 11 \\
\hline
3 (625-line systems) & 12 & 11 \\
\hline
\end{tabular}
\caption{Available preset aspect ratio values}\label{table:aspectratiovalues}
\end{table}

\begin{table}[!ht]
\centering
\begin{tabular}{|c|c|c|}
\hline
$index$ & $\SLumaOffset$ & $\SLumaExcursion$ \\
\hline
1 (8 Bit Full Range) & 0 & 255 \\
\hline
2 (8 Bit Video) & 16 & 235 \\
\hline
3 (10 Bit Video) & 64 & 876 \\
\hline
\end{tabular}
\caption{Luma signal range available presets}\label{table:lumasignalrangevalues}
\end{table}

\begin{table}[!ht]
\centering
\begin{tabular}{|c|c|c|}
\hline
$index$ & $\SChromaOffset$ & $\SChromaExcursion$ \\
\hline
1 (8 Bit Full Range) & 128 & 255 \\
\hline
2 (8 Bit Video) & 128 & 224 \\
\hline
3 (10 Bit Video) & 512 & 896 \\
\hline
\end{tabular}
\caption{Chroma signal range available presets}\label{table:chromasignalrangevalues}
\end{table}

\begin{informative}
The only presets available cover a full 8-bit range or 8- or 10-bit SDI video ranges.
If other video depths have been selected, custom signal range parameters
should be signalled, or the resulting video may have an unintended appearance that
affects video quality adversely on a display device.
\end{informative}

\begin{table}[!ht]
\centering
\begin{tabular}{|c|c|l|l|l|}
\hline
$index$ & {\bf Description}           & {\bf Primaries}       & {\bf Matrix}  & {\bf Transfer function}\\
\hline
0 & Custom, HDTV, PC \& Internet & 0 - ITU709 \& sRGB & 0 - HDTV/PC & 0 - TV \\ 
\hline
1 & NTSC                                    & 1 - SMPTE C           & 1 - SDTV      & 0 - TV \\
\hline
2 & PAL                                       & 2 - EBU Tech 3213   & 1 - SDTV      & 0 - TV \\
\hline
3 & D-Cinema                              & 3 - CIE XYZ             & 2 - YCgCo     & 3 - DCI \\
\hline
\end{tabular}
\caption{Colour specification presets}\label{table:colourspecvalues}
\end{table}

\begin{table}[!ht]
\centering
\begin{tabular}{|c|l|}
\hline
$index$ &  {\bf Primaries}      \\
\hline
0          &  ITU709 \& sRGB  \\ 
\hline
1          &  SMPTE C (as used for NTSC)          \\
\hline
2          &  EBU Tech 3213, as used for PAL   \\
\hline
3          & CIE XYZ              \\
\hline
\end{tabular}
\caption{Colour primaries presets}\label{table:primariesvalues}
\end{table}

\begin{table}[!ht]
\centering
\begin{tabular}{|c|l|}
\hline
$index$ &  {\bf Matrix}\\
\hline
0 & HDTV, PC \& Internet: $K_R=0.2126$, $K_B=0.0722$ \\ 
\hline
1 & SDTV: $K_R=0.299$, $K_B=0.114$ \\
\hline
2 & Reversible: YCgCo \\
\hline
\end{tabular}
\caption{Colour matrix presets}\label{table:matrixvalues}
\end{table}

\begin{table}[!ht]
\centering
\begin{tabular}{|c|l|}
\hline
$index$ & {\bf Transfer function}\\
\hline
0 & TV \\ 
\hline
1 & Extended Gamut \\
\hline
2 & Linear\\
\hline
3 & DCI Gamma\\
\hline
\end{tabular}
\caption{Transfer function presets}\label{table:transfervalues}
\end{table}

[Must explain all these presets]


