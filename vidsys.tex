\label{vidsys}

\begin{informative*}
\subsection{Video systems model for interpreting source parameters (Informative)}
\label{vidsysmodel}

The interpretation of sequence source parameters (Section \ref{sourceparameters})
by a display mechanism interfacing with a compliant decoder is non-normative. However, it
should where possible follow the recommendations and interpretations
specified in this section. Likewise, encoders should ensure that
accurate display parameter information is encoded to maximise the
quality of displayed video.
\subsubsection{Colour}
All current video systems use the following model for luminance/chrominance 
(`YUV')coding of RGB values (computer systems often omit coding to and from YUV). 

The R, G and B are tristimulus values (e.g. candelas/$m^2$). Their
relationship to CIE XYZ tristimulus values can be derived from the set
of primaries and white point defined in the colour primaries part of the
colour specification below using the method described in SMPTE RP
177-1993. In this discussion the RGB values are normalised to the range
[0,1], so that RGB=[1,1,1] represents the peak white of the display device
and RGB=0,0,0 represents black.

Values $E_R$, $E_G$ and $E_B$ are are defined which are related to the RGB 
values by non-linear forward and inverse transfer functions $f()$ and $g()$ respectively
(Figure \ref{fig:transferchar}).

[figure \label{fig:transferchar}]

Normally, $E_R$, $E_G$ and $E_B$ also fall in the range $[0,1]$, but in the
case of extended gamut, negative values may be allowed also. The
transfer function $f()$ is typically performed in the camera and
is specified in the Transfer Characteristic part of the
Colour Specification. For aesthetic and psychovisual reasons
the transfer function $g()$ is not quite the inverse of
$f()$. In fact the combined effect of $f()$ and
$g()$ is such that:
\[g(f(x))=x^\gamma\]

where $\gamma$ is the `rendering intent' or end to end gamma of the
system, which may vary between about 1.1 and 1.6 depending on viewing
conditions. The rationale for this is given in `Digital Video and HDTV',
Charles Poynton 2003, Morgan Kaufmann Publishers, ISBN 1-55860-792-7.

The non-linear $E_R$, $E_G$ and $E_B$ values are subject to a matrix operation
(known as `non-constant luminance coding'), which transforms
them into luma ($E_Y$) and chroma (normally $E_{Cb}$ and $E_{Cr}$) values. 
$E_Y$ is normally limited to the 
range $[0,1]$ and the chroma
values to the range $[-0.5, 0.5]$. This is YUV coding and
sometimes the chroma components are subsampled, either horizontally or
both horizontally and vertically. UV sampling is specified by the
$\ChromaFormat$ value. 

\paragraph{Conventional YUV coding}
$\ $\newline
In conventional YUV coding, the $E_Y$, $E_{Cb}$ and $E_{Cr}$ values are
mapped to a range of integers denoted $Y$, $C1$ and $C2$ within this
specification. In this way, $C1$ typically corresponds to $Cb$ and
$C2$ to $Cr$. The way this mapping occurs is defined by the signal
range parameters (Section \ref{signalranges}). It is these integer values 
that are actually output from the decoder. In order to display video, the inverse to the above
operations must be performed to convert this data to $E_Y$, $E_{Cb}$, $E_Cr$, then
to $E_R$, $E_G$, $E_B$ and thence to R, G and B.  

\paragraph{YCoCg coding}
$\ $\newline
The E values can be viewed as something of a mathematical abstraction.
For example in digital display devices, R, G and B values are specified
in terms of integer levels which are derived from the integral luma and
chroma values by direct operations subsuming and approximating all the
real-number operations described here. Generally, these approximations
cause loss through quantisation of intermediate values, and the
restriction of values to particular ranges also restricts the colour
gamut. 

In the case of YCoCg coding, the $E_R$, $E_G$ and $E_B$ values are directly
linearly scaled to integer ranges $[0,2^\VideoDepth-1]$ before a lossless 
direct integer transform is applied to convert this data to $Y$, $C1$ (representing
Co) and $C2$ (representing $Cg$) data. This transform is described in Section
\ref{colourmatrix}. This supports efficient lossless RGB coding.

\subsubsection{Signal range}
\label{signalranges}

When $YCoCg$ coding is not being applied, offset and excursion values 
$\LumaOffset$, $\LumaExcursion$, $\ChromaOffset$ and
$\ChromaExcursion$ should be used to convert the
integer-valued decoded luma and chroma data $Y$, $C1$ ($Cb$) and $C2$ ($Cr$) 
to intermediate values $E_Y$, $E_{Cb}$, and $E_{Cr}$ by the recipe
\begin{eqnarray*}
E_Y & = & \dfrac{Y-\LumaOffset}{\LumaExcursion} \\
E_{Cb} & = & \dfrac{C1-\ChromaOffset}{\ChromaExcursion} \\
E_{Cr} & = & \dfrac{C2-\ChromaOffset}{\ChromaExcursion}
\end{eqnarray*}

$E_Y$ is normally clipped to the range $[0,1]$ and $E_{Cb}$, $E_{Cr}$
to the range $[-0.5,0.5]$. This effectively clips integer $Y$ values to 
the interval
\[ [\LumaOffset,\LumaOffset+\LumaExcursion] \]
and $C1$, $C2$ values to
\[ [\ChromaOffset-\ChromaExcursion/2,\ChromaOffset+\ChromaExcursion/2] \]

However, maintaining an extended RGB gamut may mean that either such
clipping is not done, or non-standard offset and excursion values are
used to extract the extended gamut from the non-negative decoded $Y$, $C1$,
and $C2$ values.

Signal range values are not used in the case of $YCoCg$ coding.[True??]

\begin{comment}
Non-default offset and excursion values cannot be coded if the chroma
format is YCoCg: default parameters should be used. However, even in
this case, EY, ECo, and ECg should not be calculated. Instead, direct
integer conversion to RGB should be done as described in Section . (In
fact, excursion values will be ignored in this integer conversion.)
\end{comment}

\subsubsection{Primaries}
\label{primaries}
The colour primaries allow device dependent linear RGB colour
co-ordinates to be mapped to device independent linear CIE XYZ space.
The primaries specified below are the CIE (1931) XYZ chromaticity
co-ordinates of the primaries and the white point of the device. The
maths required to convert between RGB and XYZ is reproduced below. $V_X$,
$V_Y$ and $V_Z$ are the XYZ coordinates of value $V$, for $V$ equal to
the device-dependent red, green, blue or white value.

\begin{eqnarray*}
F & = &
\left(
    \begin{array}{ccc}
    \dfrac{R_X}{R_Y} & \dfrac{G_X}{G_Y} & \dfrac{B_X}{B_Y} \\
    1 & 1 & 1 \\
    \dfrac{1-R_X-R_Y}{R_Y} & \dfrac{1-G_X-G_Y}{G_Y} & \dfrac{1-B_X-B_Y}{B_Y}
    \end{array}
\right)
\\
\left(
    \begin{array}{c}
    s_r \\
    s_g \\
    s_b
    \end{array}
\right ) & = & F^{-1}
\left(
    \begin{array}{c}
    \dfrac{W_X}{W_Y} \\
    1 \\
    \dfrac{1-W_X-W_Y}{W_Y}
    \end{array}
\right) 
\\
\left(
    \begin{array}{c}
    X \\
    Y \\
    Z
    \end{array}
\right) & = & 
\left(
    \begin{array}{c}
    s_r*R \\
    s_g*G \\
    s_b*B
    \end{array}
\right)
\end{eqnarray*}

The colour primary specification therefore allows exact colour
reproduction of decoded RGB values on different displays with different
display primaries. (Although it has to be said that often conversion between
encoded primaries and display primaries is not done.)

\subsubsection{Matrix}
\label{matrix}
\paragraph{Conventional YUV coding}
$\ $\newline
Unit-scale luma and chroma values $E_Y$, $E_{Cb}$ and $E_{Cr}$ should be
derived from decoded $Y$, $C1$ and $C2$ values using the signal range parameters
as per Section \ref{signalranges}. Given these values, $E_R$, $E_G$ and $E_B$ are
determined as follows:
\begin{eqnarray*}
E_R & = & E_Y + 2*(1-K_R)*E_{Cr} \\
E_G & = & E_Y - \dfrac{2*K_R*(1-K_R)*E_{Cr}}{K_G}-\dfrac{2*K_B*(1-K_B)*E_{Cb}}{K_G} \\
E_B & = & E_Y + 2*(1-K_R)*E_{Cb} 
\end{eqnarray*}
where $K_G=1-K_R-K_B$.
This follows by inverting the equations 
\begin{eqnarray*}
K_R+K_G+K_B & = & 1 \\
E_Y & = & K_R*E_R+K_G*E_G+K_B*E_B \\
E_{Cb} & = & \dfrac{E_B - E_Y}{2*(1-K_B)} \\
E_{Cr} & = & \dfrac{E_R - E_Y}{2*(1-K_R)} \\
\end{eqnarray*}

\paragraph{YCoCg coding}
$\ $\newline
In the case of YCoCg coding, integer $I_R$, $I_G$, $I_B$ should be directly computed from
the decoded $Y$, $C1$ ($Co$) and $C2$ ($Cg$) values by
\begin{eqnarray*}
Y & -= & \LumaOffset \\
Co=C1 & -= & \ChromaOffset \\
Cg=C2 & -= & \ChromaOffset \\
t & = & Y-(Cg\gg1) \\
I_G & = & t+Cg \\
I_B & = & t-(Co\gg1) \\
I_R & = & I_B+Co
\end{eqnarray*}
The integer values are converted to unit-scale $E_R$, $E_G$, $E_B$ by dividing by 
$2^\VideoDepth$ and clipping to $[0,1]$.
If the inverse transform has been correctly
applied prior to coding and lossless coding employed, then clipping will
be unnecessary, and reversing the above operations will reproduce $Y$, $Co$ and $Cg$
losslessly from $I_R$, $I_G$ and $I_R$ yielding a transparent RGB to RGB coding system:
\begin{eqnarray*}
Co & = & I_R-I_B \\
t & = & I_B+(I_R-I_B)\gg1 \approx (I_R+I_B)/2\\
Cg & = & I_G-t = \approx I_G-(I_R+I_B)/2\\
Y & = & t+(Cg\gg1) \approx I_G/2-(I_R+I_B)/4+(I_R+I_B)/2=I_R/4+I_G/2+I_B/4
\end{eqnarray*}

Note that these matrix operations give that the chroma ranges are twice the
size of the luma range, due to the subtractions used to create chroma components. 
So for 8-bit RGB ($I_R$, $I_G$, $I_B$) values, $Y$ will be 8 bits and $Co$ and
$Cg$ will be 9 bits. 

Hence $\VideoDepth$ should be set to 1 plus the original RGB resolution.

Note also that $\LumaExcursion$ and $\ChromaExcursion$ are not required for
these scaling operations. They should be set to $2^\VideoDepth-1$ and $2^\VideoDepth$
respectively..

\subsubsection{Transfer characteristics}
\paragraph{TV transfer characteristic}
$\ $\newline

Denoting R or G or B as `$L$' (light) and $E_R$, $E_G$, $E_B$ as
`$E$' then $E=f(L)$ such that
\[
E=\left\{
        \begin{array}{ll}
        4.5L & 0\leq L<0.18\\
        1.099*L^{0.45} &0.18\leq L \leq 1
        \end{array}
  \right.
\]

All modern TV systems use this transfer characteristic at present. ITU-R
BT 470 (`Conventional Television systems PAL, NTSC and SECAM') specifies
an ``assumed gamma value of the receiver for which the primary
signals are pre-corrected'' as 2.2 for NTSC and 2.8 for PAL. This
specification is incomplete, incorrect and obsolete and modern PAL and
NTSC systems use the `TV' transfer characteristic above.

\paragraph{Extended Colour Gamut}
$\ $\newline

ITU-R BT 1361, `Worldwide unified colorimetry of future TV systems'
defines a transfer characteristic for systems with an extended colour
gamut as follows.

Denoting R or G or B as `$L$' (light) and $E_R$, $E_G$, $E_B$ as
`$E$' then $E=f(L)$ such that
\[
E=\left\{
        \begin{array}{ll}
        -\dfrac{1.099*(-4*L)^{0.45}-0.099}{4} & -0.25\leq L<-0.0045\\
        4.5L & -0.0045\leq L<0.18\\
        1.099*L^{0.45} &0.18\leq L \leq 1.33
        
        
        \end{array}
  \right.
\]

This transfer characteristic is intended to be used with systems using
an extended colour gamut.

\paragraph{Linear}
$\ $\newline
A linear transfer characteristic has $f(x)=x$. 

\subsubsection{Frame rate}
The ratio of the frame rate values $\FrameRateNumerator$ and $\FrameRateDenominator$
 encodes the intended rate at which frames should be
displayed subsequent to decoding. If $\Interlaced$ is $\true$, then fields are
displayed at double the frame rate, in the order specified by the
$\TopFieldFirst$ flag.

\subsubsection{Aspect ratios and clean area}
The aspect ratio of an image is the ratio of the intended
spacing of horizontal samples (pixels) to the spacing of vertical
samples (picture lines) on the display device. Pixel aspect ratios are
fundamental properties of sampled images because they determine the
displayed shape of objects in the image. Failure to use the right value
of will result in distorted images for example,
circles will be displayed as ellipses and so forth. HDTV standards and 
computer image formats are generally defined to have pixel
aspect ratios that are exactly 1:1.

The clean area defines an area of pixels within the picture which
should be displayed -- other pixels outside the area should not be
displayed. In particular, the clean area can be used to suppress artefacts 
near the picture edges: at high levels of compression it may only be appropriate
 to display the clean area rather than the whole picture. 

The top-left corner of the clean area has coordinates
\[(\LeftOffset,\TopOffset)\]
counting from the top-left corner of the picture data, and
dimensions $\CleanWidth$ by $\CleanHeight$.

Note that these dimensions refer to pixels within a picture, not a frame,
so a change from interlaced to progressive picture coding will
necessitate a change of clean area if a custom clean area is used.

The clean area and the pixel aspect ratio together determine the
aspect ratio of the displayed image which is the ratio of the width of the intended
display area to the height of the intended display area:
\[\dfrac{\CleanWidth*\AspectRatioNumerator}{\CleanHeight*\AspectRatioDenominator}\]

Given two separate sequences, with identical image aspect ratio, if the
top left corner and bottom right corners of their clean apertures are
coincident when displayed, then the images as a whole should be exactly
coincident. This is regardless of the actual pixel dimensions of the
images or their clean areas. This allows sequences to be combined
together appropriately if they are appropriately scaled.

The defined pixel aspect ratios are designed to give standard image
aspect ratios for typical TV broadcasts. For example, for a 525 line
(American) 704 x 480 (clean area)  picture the image aspect ratio is
(704 x 10)/(480 x 11) which is exactly 4:3.

For 625 line systems the 12:11 pixel aspect ratio means (less
conveniently) that a 704x576 image would have an exact 4:3 image
aspect ratio. It might be argued that the pixel aspect ratio for 625
line systems should be such that a 702x576 image would have an exact 4:3
image aspect ratio, corresponding to the nominal sample rate for 625 line TV.
This requirement would lead to a
pixel aspect ratio of 128:117. However, the tolerance of the analogue
line length is $702\pm 3$ pixels, which does not really seem to justify a
ratio of exactly 128:117.

The values specified here are generally agreed to be the
`correct' values. Then again not everyone agrees with this
consensus. These arise from the `industry standard' sampling
frequencies used for square pixels, which were originally designed for
digitising composite analogue video signals. These `industry
standard' sampling frequencies are 11+3/11 MHz for 525 line systems
and 14.75MHz for 625 line systems. The ratio of these frequencies to the
(standardised) 13.5MHz sampling frequency used for broadcasting (approximately) 
yields the pixel aspect ratios given.

You are strongly advised to use one of the default pixel aspect ratios.
However, if you know what you are doing and don't like the default
values you can define your own ratio. You should be aware that many
display devices may ignore your decision and may use different and
unsuitable values. 

\end{informative*}

\subsection{Source parameter presets}
\label{sourceparamspresets}

Source parameters are signalled using a range of preset indices into the following
tables, as specified in Section \ref{sourceparameters}. Their correct
interpretation by a display device is described in Appendix \ref{vidsysmodel}.

\begin{table}[!ht]
\centering
\begin{tabular}{|c|c|c|}
\hline
$index$ & $\SFrameRateNumerator$ & $\SFrameRateDenominator$ \\
\hline
1 & 24000 & 1001 \\
\hline
2 & 24 & 1 \\
\hline
3 & 25 & 1 \\
\hline
4 & 30000 & 1001 \\
\hline
5 & 30 & 1 \\
\hline
6 & 50 & 1 \\
\hline
7 & 60000 & 1001 \\
\hline
8 & 60 & 1 \\
\hline
\end{tabular}
\caption{Available preset frame rate values}\label{table:frameratevalues}
\end{table}

\begin{table}[!ht]
\centering
\begin{tabular}{|c|c|c|}
\hline
$index$ & $\SAspectRatioNumerator$ & $\SAspectRatioDenominator$ \\
\hline
1 (Square Pixels) & 1 & 1 \\
\hline
2 (525-line systems) & 10 & 11 \\
\hline
3 (625-line systems) & 12 & 11 \\
\hline
\end{tabular}
\caption{Available preset aspect ratio values}\label{table:aspectratiovalues}
\end{table}

\begin{table}[!ht]
\centering
\begin{tabular}{|c|c|c|}
\hline
$index$ & $\SLumaOffset$ & $\SLumaExcursion$ \\
\hline
1 (8 Bit Full Range) & 0 & 255 \\
\hline
2 (8 Bit Video) & 16 & 235 \\
\hline
3 (10 Bit Video) & 64 & 876 \\
\hline
\end{tabular}
\caption{Luma signal range available presets}\label{table:lumasignalrangevalues}
\end{table}

\begin{table}[!ht]
\centering
\begin{tabular}{|c|c|c|}
\hline
$index$ & $\SChromaOffset$ & $\SChromaExcursion$ \\
\hline
1 (8 Bit Full Range) & 128 & 255 \\
\hline
2 (8 Bit Video) & 128 & 224 \\
\hline
3 (10 Bit Video) & 512 & 896 \\
\hline
\end{tabular}
\caption{Chroma signal range available presets}\label{table:chromasignalrangevalues}
\end{table}

\begin{informative}
The only presets available cover a full 8-bit range or 8- or 10-bit SDI video ranges.
If other video depths have been selected, custom signal range parameters
should be signalled, or the resulting video may have an unintended appearance that
affects video quality adversely on a display device.
\end{informative}

\begin{table}[!ht]
\centering
\begin{tabular}{|c|c|l|l|l|}
\hline
$index$ & {\bf Description}           & {\bf Primaries}       & {\bf Matrix}  & {\bf Transfer function}\\
\hline
0 & Custom, HDTV, PC \& Internet & 0 - ITU709 \& sRGB & 0 - HDTV/PC & 0 - TV \\ 
\hline
1 & NTSC                                    & 1 - SMPTE C           & 1 - SDTV      & 0 - TV \\
\hline
2 & PAL                                       & 2 - EBU Tech 3213   & 1 - SDTV      & 0 - TV \\
\hline
3 & D-Cinema                              & 3 - CIE XYZ             & 2 - YCoCg     & 3 - DCI \\
\hline
\end{tabular}
\caption{Colour specification presets}\label{table:colourspecvalues}
\end{table}

\begin{table}[!ht]
\centering
\begin{tabular}{|c|l|}
\hline
$index$ &  {\bf Primaries}      \\
\hline
0          &  ITU709 \& sRGB  \\ 
\hline
1          &  SMPTE C (as used for NTSC)          \\
\hline
2          &  EBU Tech 3213, as used for PAL   \\
\hline
3          & CIE XYZ              \\
\hline
\end{tabular}
\caption{Colour primaries presets}\label{table:primariesvalues}
\end{table}

\begin{table}[!ht]
\centering
\begin{tabular}{|c|l|}
\hline
$index$ &  {\bf Matrix}\\
\hline
0 & HDTV, PC \& Internet: $K_R=0.2126$, $K_B=0.0722$ \\ 
\hline
1 & SDTV: $K_R=0.299$, $K_B=0.114$ \\
\hline
2 & Reversible: YCgCo \\
\hline
\end{tabular}
\caption{Colour matrix presets}\label{table:matrixvalues}
\end{table}

\begin{table}[!ht]
\centering
\begin{tabular}{|c|l|}
\hline
$index$ & {\bf Transfer function}\\
\hline
0 & TV \\ 
\hline
1 & Extended Gamut \\
\hline
2 & Linear\\
\hline
3 & DCI Gamma\\
\hline
\end{tabular}
\caption{Transfer function presets}\label{table:transfervalues}
\end{table}



