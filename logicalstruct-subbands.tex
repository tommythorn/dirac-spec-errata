Using the subband numbering in figure~\ref{fig:dwt}, subbands are
transmitted from 10 downto 1, ie from the lowest frequency to the
highest frequency.  Since the wavelet transform and its inverse are
lossless, coding gains are made by quantizing the subbands for
transmission.

Further coding gains may be made by exploting the correlation in
frequency components, whereby each pixel in a subband is predicted from
some of its neighbours.  Correlation accross different frequency bands
is also exploited.

An extension avaliable to encoders is to use multiple quantizers per
subband.  This is achieved by subdividing a subband into a number of
codeblocks, each having its own quantizer.  As an aside, this means that
we can identify parts of the picture which require most accurate coding
(for example we often take most notice of errors in the region of a
person's eyes when looking at images of faces) and enhance the coding
accuracy in this area. This process is sometimes referred to as region
of interest coding.
