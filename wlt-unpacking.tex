\label{wltunpacking}

This section defines the process for unpacking (parsing, entropy decoding and 
inverse quantizing) wavelet transform coefficient data from the Dirac stream. 
Wavelet coefficients shall be signed integer values and shall be extracted 
using the integer VLC and (optionally) arithmetic decoding functions defined in annex A. 
The use of arithmetic coding within a picture shall be as signaled by the parse codes
defined in Section \ref{parseinfoheader}. The result of this process shall be a set of fully
 populated wavelet subband data arrays, as defined in
Section \ref{wltdecodeconventions}.

Wavelet coefficients shall be packed in the bitstream in one of two possible formats: 
\begin{enumerate}
\item In the core syntax, coefficients shall be grouped within individual subbands,
 representing a range of spatial frequencies, from the lowest to the highest. A full set of
 subbands shall be encoded for each video component in turn.  
\item  In the low delay syntax, coefficients shall be grouped into slices that represent
coefficients pertaining to an area of the picture. Each slice shall contain data for all video
components and spatial frequency bands. Unpacking a slice allows an area of picture to
be extracted without extracting (or even receiving) the remaining picture data.
\end{enumerate}

The overall process for unpacking transform data shall be as follows:

\begin{pseudo}{transform\_data}{}
\bsIF{is\_low\_delay()==\false}
    \bsCODE{\YTransform = core\_transform\_data(Y)}{\ref{transformdec}}
    \bsCODE{\COneTransform = core\_transform\_data(C1)}{\ref{transformdec}}
    \bsCODE{\CTwoTransform = core\_transform\_data(C2)}{\ref{transformdec}}
\bsELSE
    \bsCODE{low\_delay\_transform\_data()}{\ref{ldpicture}}
\bsEND
\end{pseudo}

Unpacked wavelet coefficient data shall be stored in the state variables $\YTransform$,
 $COneTransform$ and $CTwoTransform$ for the IDWT process (Section \ref{idwt}).

\subsection{Subband data structures}
\label{wltdecodeconventions}

Subband data shall be ordered by level (0, 1, 2, 3 etc) and orientation (LL, HL, LH and HH). In level 0, only the LL orientation shall be available (known also as the DC band); in the other levels only the HL, LH and HH orientations shall be available.

The 0-LL subband shall be presented first in each slice (low delay coding) or component (core syntax coding). Within each subband depth level, the orientation order shall be x-HL, x-LH, x-HH (where �x� is the transform depth level).

The subbands partition the spatial frequencies by orientation and level so that a four-level subband array is as illustrated in Figure \ref{fig:orientlevel}. 


\setlength{\unitlength}{1em}
\begin{figure}[!ht]
\centering
\begin{picture}(25,35)

%Main Grid
\multiput(0,0)(16,0){3}%
  {\line(0,1){32}}
\multiput(0,0)(0,16){3}%
  {\line(1,0){32}}

%Second Grid
\put(0,24){\line(1,0){16}}
\put(8,16){\line(0,1){16}}

%Third Grid
\put(0,28){\line(1,0){8}}
\put(4,24){\line(0,1){8}}

%Fourth Grid
\put(0,30){\line(1,0){4}}
\put(2,28){\line(0,1){4}}


%Put Levels

\put(24,34){Level 4}
\put(23,36){\vector(0,-1){3}}

\put(13,34){Level 3}
\put(12,36){\vector(0,-1){3}}

\put(7,34){2}
\put(6,36){\vector(0,-1){3}}

\put(4,34){1}
\put(3,36){\vector(0,-1){3}}

%Put decomposition numbers
\put(23,24){4-HL}
\put(7,8){4-LH}
\put(23,8){4-HH}
\put(11,28){3-HL}
\put(3,20){3-LH}
\put(11,20){3-HH}
\put(5,30){2-HL}
\put(1,26){2-LH}
\put(5,26){2-HH}
\put(2.2,31){\tiny 1-HL}
\put(0.2,29){\tiny 1-LH}
\put(2.2,29){\tiny 1-HH}
\put(0.2,31){\tiny 0-LL}

  
\end{picture}
\caption{Subband decomposition of the spatial frequency domain showing subband 
numbering, for a 4-level wavelet decomposition}\label{fig:orientlevel}

\end{figure}


\subsubsection{Wavelet data initialisation}

\label{wltinit}

This section defines the $initialise\_wavelet\_data(comp)$ process, which returns a structure which will
contain the wavelet coefficients for the component (Y, C1 or C2) indicated by $comp$. 

The coefficient data shall comprise the four dimensional array $coeff\_data$, where
 individual subbands shall be two-dimensional arrays accessed by level $level$ and orientation $orient$: e.g. 
\[band = coeff\_data[level][orient] \]

Valid levels shall be integer values in the range 0 to $\TransformDepth$ inclusive. 

Level 0 shall consists of a single subband with orientation \LL. 

All other levels shall consist of 3 subbands of orientation \HL, \LH 
and \HH in that order within the Dirac stream.

The orientations correspond to either low- or high-pass filtering
horizontally and vertically: so e.g. the \LH band consists of coefficients derived
from horizontal low-pass filtering and vertical high-pass filtering. 

Each subband array shall be initialised so that:
\begin{eqnarray*}
\width(coeff\_data[level][orient]) & = & subband\_width(level,comp) \\
\height(coeff\_data[level][orient]) & = & subband\_height(level,comp)
\end{eqnarray*}

as specified in Section \ref{subbandwidthheight}. These dimensions correspond 
to a wavelet transform being performed on a copy of the component data which 
has been padded (if necessary) so that its
dimensions are a multiple of $2^{\TransformDepth}$.

Individual subband coefficients shall be signed integers accessed by vertical and 
horizontal coordinates within the subband, e.g.:
\[c = coeff\_data[level][orient][y][x]\]

for coordinates $(x,y)$ such that
\begin{eqnarray*} 
0\leq x<subband\_width(level,comp) \\
0\leq y<subband\_height(level,comp)
\end{eqnarray*}

\subsubsection{Subband dimensions}
\label{subbandwidthheight}

This section defines the values of the $subband\_width(level, comp)$ and $subband\_height(level,comp)$
functions, giving the width and height of subbands at a given level for a given component, and hence the range
of subband vertical and horizontal indices. 

If $comp==Y$, set
\begin{eqnarray*}
w & = & \LumaWidth \\
h & = & \LumaHeight
\end{eqnarray*}

Otherwise, set
\begin{eqnarray*}
w & = & \ChromaWidth \\
h & = & \ChromaHeight
\end{eqnarray*}

The padded dimensions of the component shall be defined by:
\begin{eqnarray*}
scale & = & 2^\TransformDepth \\
pw & = & scale*((w+scale-1)//scale)\\ 
ph & = & scale*((h+scale-1)//scale)
\end{eqnarray*}

If $level==0$,
\begin{eqnarray*}
subband\_width(level) & = & pw//2^{\TransformDepth} \\
subband\_height(level) & = & ph//2^{\TransformDepth}
\end{eqnarray*}

If $level>0$
\begin{eqnarray*}
subband\_width(level) & = & pw//2^{\TransformDepth-level+1}\\
subband\_height(level) & = & ph//2^{\TransformDepth-level+1}
\end{eqnarray*}

\begin{informative}
In encoding, these padded dimensions may be achieved by padding the 
component data up to the padded dimensions and applying the forward
Discrete Wavelet Transform (the inverse of the operations specified in
Section \ref{idwt}). Any values may be used for the padded data, although
the choice will affect wavelet coefficients at the right and bottom 
edges of the subbands. Good results, in compression terms, may be obtained
 by using edge extension for intra pictures and zero extension for inter 
pictures. A poor choice of padding may cause visible artefacts near the
bottom and right edges at high levels of compression.
\end{informative}

\subsection{Inverse quantisation}
\label{invquant}

This section defines the operation of inverse quantisation, which scales the
dynamic range of unpacked wavelet coefficients according to a pre-determined factor.
The inverse quantisation operation is common to both the low-delay and core syntax.

The $inverse\_quant()$ function shall be defined as follows:

\begin{pseudo}{inverse\_quant}{quantised\_coeff, quant\_index}
\bsCODE{magnitude = |{quantised\_coeff}|}
\bsIF{magnitude!=0}
  \bsCODE{magnitude *= quant\_factor(quant\_index)}{\ref{quantfacs}}
  \bsCODE{magnitude += quant\_offset(quant\_index)}{\ref{quantfacs}}
  \bsCODE{magnitude += 2}
  \bsCODE{magnitude = magnitude//4}
\bsEND
\bsRET{\sign( quantised\_coeff )*magnitude}
\end{pseudo}

\begin{informative}
\begin{enumerate}
\item Dirac quantisation is an integer approximation of dead-zone quantisation,
in which a value is quantised as
\[\left| x\right| // qf\]
for $x\geq 0$ or 
\[-\left| x\right| // qf\]
Since this process involves rounding down, the inverse quantisation process adds
an offset to reconstructed values after multiplying by $qf$. This produces a 
value on average closer to the original value.
\item The pseudocode description separates inverse quantisation from coefficient unpacking. However, 
since dead-zone quantisation is used, the $inverse\_quant()$ function must compute
the magnitude. Hence it is more efficient to first extract the coefficient magnitude,
then inverse quantise, and then extract the coefficient sign. 
\item In the low delay syntax, the quantisation index is limited to 6 bits,
i.e. a maximum value of 63. In the core syntax this limit will also suffice 
for 8 bit data and a 4-level transform, but the maximum value will in general
depend upon the video bit depth, the type of wavelet filter and the transform
depth. A value of 127 will account for most practical situations.
\end{enumerate}
\end{informative}

\subsubsection{Quantisation factors and offsets}
\label{quantfacs}

This section defines the operation of the $quant\_factor()$ and 
$quant\_offset()$ functions for performing inversion quantisation.

Quantisation factors shall be determined as follows:

\begin{pseudo}{quant\_factor}{index}
\bsCODE{base = 2^{index//4}}
\bsIF{ (index\%4)==0 }
  \bsRET{4*base}
\bsELSEIF{ (index\%4)==1 }
  \bsRET{(503829*base+52958)//105917}
\bsELSEIF{ (index\%4)==2 }
  \bsRET{(665857*base+58854)//117708}
\bsELSEIF{ (index\%4)==3 }
  \bsRET{(440253*base+32722)//65444}
\bsEND
\end{pseudo}

For intra pictures, offsets are approximately $1/2$ of the 
quantisation factors, and for inter pictures they are $3/8$ - these
mark the reconstruction point within the quantisation interval:

\begin{pseudo}{quant\_offset}{index}
\bsIF{index==0}
  \bsCODE{offset = 1}
\bsELSE
  \bsIF{ is\_intra() }
    \bsIF{index==1}
      \bsCODE{offset = 2}
    \bsELSE
      \bsCODE{ offset=(quant\_factor(index)+1)//2 }
    \bsEND
  \bsELSE
    \bsCODE{ offset=(quant\_factor(index)*3+4)//8 }
  \bsEND
\bsEND
\bsRET{offset}
\end{pseudo}

The value of $index$ passed to both functions shall be greater than or equal 0. 

\begin{informative}
The quantisation offsets have been selected so as to make inverse quantisation
and re-quantisation by the same quantisation factor transparent. This requires that
\[3\leq quant\_offset+2<quant\_factor\] -- hence the special conditions for 
quantisation indexes 0 and 1.
\end{informative}

\subsection{Intra DC subband prediction}
\label{intradcprediction}
$\ $\newline
This section defines the operation of the $intra\_dc\_prediction(band)$ function
for reconstructing values within Intra picture DC subbands using spatial prediction.

This function may be applied once all coefficients within the DC
band have been unpacked, although it may be applied progressively to each coefficient
as soon as it has been unpacked.

Intra DC values shall be derived by spatial prediction using the mean of the
three values to the left, top-left and above a coefficient (where available).

The Intra DC subband prediction process shall be defined as follows:

\begin{pseudo}{intra\_dc\_prediction}{band}
\bsCODE{prediction = 0 }
\bsFOR{v=0}{\height(band)-1}
  \bsFOR{h=0}{\width(band)-1}
      \bsIF{h>0 \text{ \bf and } v>0}
            \bsCODE{prediction=\mean(band[v][h-1],band[v-1][h-1],band[v-1][h])}
      \bsELSEIF{h>0 \text{ \bf and } v==0}
            \bsCODE{prediction=band[0][h-1]}
      \bsELSEIF{h==0 \text{ \bf and } v>0}
            \bsCODE{prediction=band[v-1][0]}
      \bsELSE
            \bsCODE{prediction = 0}
      \bsEND
      \bsCODE{band[v][h] += prediction}
  \bsEND
\bsEND
\end{pseudo}

\subsection{Core syntax wavelet coefficient unpacking}

\label{transformdec}

This section defines the overall operation of the $core\_transform\_data(comp)$ process
for unpacking the set of coefficient subbands corresponding
to a video component (Y, C1 or C2) of a picture in the core Dirac syntax, 
according to the conventions set out in Section \ref{wltdecodeconventions}.


In the Dirac core syntax, subband data shall be entropy coded. It shall be arranged by 
level and orientation, from level 0 up to level \TransformDepth. Coefficients may
be VLC or arithmetic coded. Where arithmetic coding is used, the unpacking process
 for each subband is contingent on data from subbands of the same orientation in 
 the next lower level. This is the {\em parent} subband; the subband of the same orientation in the next
higher level is the {\em child} subband. 

Unpacking an individual subband therefore requires prior unpacking of the parent subband,
and of its parent, and so on until level 1 is reached (unpacking level 1 subbands does not depend upon the single level 0 DC band).

\begin{informative}
The data for each subband consists of a subband header and a block of 
coded coefficient data. The subband header contains a length code giving the number of
bytes of the block of coded data. The transform data can therefore be
parsed without invoking entropy decoding at all, since the length codes allow a 
parser to skip from one subband header to the next.
\end{informative}


\subsubsection{Overall process}
The overall $core\_transform\_data()$ process shall be defined as follows:

\begin{pseudo}{core\_transform\_data}{comp}
\bsCODE{coeff\_data=initialise\_wavelet\_data(comp)}{\ref{wltinit}}
\bsCODE{byte\_align()}
\bsCODE{subband(coeff\_data, 0, \LL)}{\ref{subbanddecodeprocess}}    
\bsFOR{level=1}{\TransformDepth}
  \bsFOREACH{orient}{\HL,\LH,\HH}
    \bsCODE{byte\_align()}
    \bsCODE{subband(coeff\_data, level, orient)}{\ref{subbanddecodeprocess}}
  \bsEND
\bsEND
\bsRET{coeff\_data}
\end{pseudo}

\subsubsection{Subbands}

\label{subbanddecodeprocess}

This section defines the process for unpacking coefficients of a specified level
and orientation $orient$. 

The overall process shall consist of reading a byte-aligned header for each
subband, including a length code for the subsequent arithmetically-coded data.
Subband data shall be initialised to 0. If the length code is 0, the subband shall be 
skipped and all data within it shall remain set to zero.

Intra DC bands are predicted, and so must additionally be reconstructed.

The subband unpacking process shall be defined as follows:

\begin{pseudo}{subband}{coeff\_data, level, orient}
\bsITEM{length}{uint}
\bsCODE{zero\_subband\_data(coeff\_data[level][orient])}{\ref{zerosubband}}
\bsIF{length == 0}
  \bsCODE{byte\_align()}
\bsELSE
    \bsCODE{quant\_index = read\_uint()}
    \bsCODE{byte\_align()}
    \bsCODE{subband\_coeffs(coeff\_data,level,orient,length,quant\_index)}{\ref{subbandcoeffs}}
\bsEND 
\bsIF{is\_intra() \text{ and } level==0}
    \bsCODE{intra\_dc\_prediction(coeff\_data[level][orient])}{\ref{intradcprediction}}
\bsEND
\end{pseudo}

\paragraph{Zero subband}
\label{zerosubband}
$\ $\newline$\ $\newline
The $zero\_subband()$ process shall sets all coefficients in a given subband to 0.

 It shall be defined as follows:

\begin{pseudo}{zero\_subband\_data}{band}
\bsFOR{y=0}{\height(band)-1}
  \bsFOR{x=0}{\width(band)-1}
    \bsCODE{band[y][x] = 0}
  \bsEND
\bsEND
\end{pseudo}

\paragraph{Non-skipped subbands}
\label{subbandcoeffs}
$\ $\newline$\ $\newline
Data within subbands may be split into one or more rectangular codeblocks (Figure \ref{codeblocks}). Codeblocks shall be scanned in raster order across the subband
 and coefficients shall be scanned in raster order within each codeblock. 

\begin{figure}
[Include codeblock figure \label{codeblocks}]
\end{figure}

The  $subband\_coeffs()$ process shall be defined as follows:

\begin{pseudo}{subband\_coeffs}{coeff\_data,level,orient,length,quant\_index}
\bsCODE{\ABitsLeft= 8*length}
\bsIF{using\_ac()==\true}
    \bsCODE{ctx\_labels=
\begin{array}{l} [\SignZero, \SignPos, \SignNeg, \ZPZNFollowOne,\\
                    \ZPNNFollowOne, \ZPFollowTwo, \ZPFollowThree, 
                      \ZPFollowFour, \ZPFollowFive, \ZPFollowSixPlus,\\
                    \NPZNFollowOne, \NPNNFollowOne, \NPFollowTwo, 
                      \NPFollowThree, \NPFollowFour,\\
                    \NPFollowFive, \NPFollowSixPlus, \CoeffData, \ZeroCodeblock,\\
                    \QOffsetFollow, \QOffsetData, \\
                     \QOffsetSign]
\end{array}
}
    \bsCODE{initialise\_arithmetic\_decoding(ctx\_labels)}{\ref{initarith}}
\bsEND
\bsFOR{y=0}{\CodeblocksY[level]-1}
    \bsFOR{x=0}{\CodeblocksX[level]-1}
        \bsCODE{codeblock(coeff\_data,level,orient,x,y,quant\_index)}{\ref{codeblocks}}
    \bsEND
\bsEND
\bsCODE{flush\_inputb()}{\ref{blockreadbit}}
\end{pseudo}

The key to the context labels is explained in Section \ref{selectcoeffcontext}.

\subsubsection{Subband codeblocks}
\label{codeblocks}

This section defines the operation of the process: 

$codeblock(band,parent,level, orient,cx,cy,quant\_index)$

This process shall unpack coefficients within the codeblock at position $(cx,cy)$.
The dimensions of the codeblock shall be as defined in section \ref{codeblockdimensions}.
The process for unpacking coefficients within the codeblock given these dimensions
shall be as defined in section \ref{codeblockloop}.

\paragraph{Codeblock dimensions}
$\ $\newline
\label{codeblockdimensions}

Each codeblock shall cover coefficients in the horizontal range $cb\_left$ to 
$cb\_right-1$ 
and in the vertical range $cb\_top$ to $cb\_bottom-1$ where these values shall be
 defined by the functions: 

\begin{eqnarray*}
  cb\_left(x,band,level) & = & (\width(band)*x)//\CodeblocksX[level]\\
  cb\_right(x,band,level) & = & (\width(band)*(x+1))//\CodeblocksX[level] \\
  cb\_top(y,band,level) & = & (\height(band)*y)//\CodeblocksY[level] \\
  cb\_bottom(y,band,level) & = & (\height(band)*(y+1))//\CodeblocksY[level]
\end{eqnarray*}

where $x$ and $y$ are the codeblock coordinates within the subband.

\paragraph{Codeblock unpacking loop}
$\ $\newline
\label{codeblockloop}

The codeblock unpacking process shall be defined as follows: 

\begin{pseudo}{codeblock}{coeff\_data,level,orient,cx,cy,quant\_index}
\bsCODE{skipped=zero\_flag(level)}{\ref{zeroblockflag}}
\bsIF{skipped==\false}
  \bsCODE{band = coeff\_data[level][orient]}
  \bsCODE{quant\_index += codeblock\_quant\_offset()}{\ref{blockquantidx}}
  \bsFOR{y=cb\_top(cy,band,level)}{cb\_bottom(cy,band,level)-1}{\ref{codeblockdimensions}}
    \bsFOR{x=cb\_left(cx,band,level)}{cb\_right(cx,band,level)-1}{\ref{codeblockdimensions}}
          \bsCODE{coeff\_unpack(coeff\_data,level,orient,x,y,quant\_index)}{\ref{wltcoeff}}
      \bsEND
    \bsEND
  \bsEND
\bsEND

\end{pseudo}

If the codeblock is skipped, then coefficients within that codeblock
shall remain zero.

The function $codeblock\_quant\_offset()$ returns a signed value, but the
quantiser offset values coded in the stream shall be constrained so that the
reconstructed value of $quant\_index$ shall be non-negative.

\begin{informative}
Codeblock quantisers are encoded differentially in the stream, and the 
value of $quant\_index$ is modified by this function (all variables)
are passed by reference). A decoder ought to check that the reconstructed value
of $quant\_index$ falls within the bounds it supports.
\end{informative}

\paragraph{Skipped codeblock flag}
\label{zeroblockflag}
$\ $\newline$\ $\newline
The skipped codeblock flag process shall be as follows:

\begin{pseudo}{zero\_flag}{level}
\bsCODE{num\_blocks=\CodeblocksX[level]* \CodeblocksY[level] }
\bsIF{num\_blocks==1}
    \bsRET{\false}
\bsELSEIF{using\_ac()==\true}
    \bsRET{ read\_boola(\ZeroCodeblock) }
\bsELSE
    \bsRET{read\_boolb()}
\bsEND
\end{pseudo}

If the number of codeblocks is 1, then $zero\_flag()$ shall return  $\false$.

If arithmetic coding is employed, then the zero flag shall be decoded using
the context probability indicated by the $\ZeroCodeblock$ label, as defined
in annex \ref{arithreadbool}.

\paragraph{Codeblock quantiser offset}
$\ $\newline$\ $\newline
\label{blockquantidx}
The $codeblock\_quant\_offset()$ process shall be defined as follows:

\begin{pseudo}{codeblock\_quant\_offset}{}
\bsIF{\CodeblockMode==0}
    \bsRET{0}
\bsELSEIF{using\_ac()==\true}
    \bsRET{ read\_sinta(quant\_context\_probs()) }
\bsELSE
    \bsRET{read\_sintb()}
\bsEND
\end{pseudo}

where $quant\_context\_probs()$ shall return the context probability label set:
\[\{[\QOffsetFollow],\QOffsetData,\QOffsetSign\}\]

\subsubsection{Subband coefficients}

\label{wltcoeff}

This section describes the operation of the 
$coeff\_unpack(coeff\_data,level,orient,quant\_index,x,y)$ process
for unpacking an individual coefficient in position $(x,y)$ 
in the subband $coeff\_data[level][orient]$.

Unpacking a coefficient shall make use of entropy decoding, inverse quantisation
and, in the case of DC (level 0) bands of Intra pictures, neighbourhood prediction.

Arithmetic coding uses a highly compact set of contexts, 
with magnitudes contextualised on whether parent values
and neighbouring values are zero or non-zero. See Annex \ref{arithdecoding}
for a definition of the Dirac arithmetic decoder. 

The process for coefficient unpacking shall comprise up to four stages:
\begin{enumerate}
\item	(for arithmetic coding only) determination of the magnitude context, based on whether the parent or neighbouring values are zero,
\item	 (for arithmetic coding only) determination of the sign context, based on the predicted sign value,
\item	 entropy decoding of the quantized coefficient value, and
\item	 inverse quantization of the quantized value.
\end{enumerate}

The $coeff\_unpack()$ process shall be defined as follows:

\begin{pseudo}{coeff\_unpack}{coeff\_data,level,orient,quant\_index,x,y}
\bsIF{using\_ac()==\true}
    \bsCODE{parent = zero\_parent(coeff\_data,level,orient, x, y)}{\ref{zeroparent}}
    \bsCODE{nhood = zero\_nhood(coeff\_data[level][orient],x,y)}{\ref{zeronhood}}
    \bsCODE{sign\_pred = sign\_predict(coeff\_data[level][orient],orient,x,y)}{\ref{signpredict}}
    \bsCODE{context\_prob\_set = select\_coeff\_ctxs(nhood, parent, sign\_pred)}{\ref{selectcoeffcontext}}
    \bsCODE{quant\_coeff = read\_sinta( context\_prob\_set )}{}
\bsELSE
    \bsCODE{quant\_coeff = read\_sintb()}{}
\bsEND
\bsCODE{coeff\_data[level][orient][y][x] = inverse\_quant( quant\_coeff, quant\_index )}{\ref{invquant}}
\end{pseudo}

\paragraph{Zero parent}
\label{zeroparent}
$\ $\newline
The function $zero\_parent(coeff\_data,level,orient,v, h)$ shall return a boolean flag indicating
whether the parent value of a coefficient in a subband is zero. The parent coefficient
shall be the co-located coefficient in the parent subband, if there is one. There is deemed to be a parent if $level\geq 2$. If a parent coefficient does not exist, $\true$ shall be
returned.

Note: Levels 0 and 1 have the same number of coefficients. Thus the first level that can be used as a parent is level 1, with level 2 as its child.

The parent value shall be determined as follows:

\begin{pseudo}{zero\_parent}{data,level,orient,x,y}
\bsIF{level>=2}
    \bsCODE{parent = data[level-1][orient][y//2][x//2]}
\bsELSE
    \bsCODE{parent = 0}
\bsEND
\bsRET{parent==0}
\end{pseudo}

\paragraph{Zero neighbourhood}
\label{zeronhood}
$\ $\newline
The $zero\_nhood()$ function shall return a boolean flag indicating whether the
neighbouring values of a given subband coefficient are all zero.

The zero neighbourhood value shall be determined as follows:

\begin{pseudo}{zero\_nhood}{band,x,y}
\bsIF{y>0 \text{ and } x>0}
  \bsIF{( band[y-1][x-1]!=0 \text{ or } band[y][x-1]!=0 )\text{ or } band[y-1][x]!=0}
    \bsRET{\false}
  \bsEND
\bsELSEIF{y>0 \text{ and } x==0}
    \bsIF{band[y-1][0]!=0}
      \bsRET{\false}
    \bsEND
\bsELSEIF{y==0 \text{ and } x>0}
    \bsIF{ band[0][x-1] !=0}
      \bsRET{\false}
    \bsEND
\bsEND
\bsRET{\true}
\end{pseudo}

\paragraph{Sign prediction}
\label{signpredict}
$\ $\newline
The $sign\_predict()$ function shall return a prediction for the sign of the 
current pixel. 

Correlation within subbands depends upon orientation,
and so this is taken into account in forming the prediction.

For vertically-oriented (HL) bands, the predictor shall be the sign of the
coefficient above the current coefficient; for horizontally-oriented (LH)
bands, the predictor shall be the sign of the coefficient to the left. 

The predictions shall be used only for the conditioning of the sign contexts.

The sign prediction value shall be determined as follows:

\begin{pseudo}{sign\_predict}{band,orient,x,y}
\bsIF{orient==HL \text{ and } y==0}
    \bsRET{0}
\bsELSEIF{orient==HL \text{ and } y>0}
    \bsRET{\sign(band[y-1][x])}
\bsELSEIF{orient==LH \text{ and } x==0}
    \bsRET{0}
\bsELSEIF{orient==LH \text{ and } x>0}
    \bsRET{\sign(band[y][x-1])}
\bsELSE
  \bsRET{0}
\bsEND{}
\end{pseudo}

\paragraph{Coefficient context selection}
\label{selectcoeffcontext}
$\ $\newline
This section defines the coefficient context selection function which shall return a map
 of context probability labels for decoding a coefficient value. Context probabilities 
shall be as defined in annex \ref{contextprobs}.

The map $m$ returned shall comprise three elements, accessed by the 
labels FOLLOW, DATA, and SIGN, where:
\begin{itemize}
\item $m$[FOLLOW] is an array of labels
\item $m$[DATA] is a label
\item $m$[SIGN] is a label 
\end{itemize}

The elements of the map returned by 
\[select\_coeff\_contexts(zero\_nhood,parent,sign\_pred)\]

shall be as defined in Table \ref{contexttable}, for values of 
$zero\_parent$, $zero\_nhood$ and $sign\_pred$, where the 
context labels (e.g. \ZPZNFollowOne, \ZPFollowTwo, \CoeffData and 
\SignZero) correspond to context probabilities (i.e. 16 bit unsigned
integers) stored in the decoded state as defined in annex 
\ref{contextprobs}.

The three columns on the left of table \ref{contexttable} represent
the three inputs to the coefficient context selection function. The
output of the function is a map with three elements. These elements
are accessed by the labels FOLLOW, DATA and SIGN. The values of the
three elements of the map are defined in the rightmost column for
each set of inputs.

\begin{informative}
The follow context probability sets are arrays indexed from zero as
per annex \ref{binandcontext}. Note also that the parent values affect the context
of all follow bits, and that neighbour values only affect the context of
the first follow bit. A common data context probability is used for all
coefficients.
\end{informative}

Key to interpretation of the label names:

\begin{description}
\item[ZP] zero parent
\item[NP] non-zero parent
\item[ZN] zero neighbour
\item[NN] non-zerro neighbour
\item[Fn] follow bit, bin N (n+ means bin n and higher)
\end{description}

%% Table of context sets for signed coefficient extraction %%
\begin{table}[!ht]
\begin{tabular}{|c|c|c|l|l|}
\hline
\rowcolor[gray]{0.75} $zero\_parent$ & $zero\_nhood$ & $sign\_pred$ & \multicolumn{2}{c|}{\cellcolor[gray]{0.75}\bf{Context map}} \\

% Zero parent, zero neighbour, zero sign prediction
\hline
\true & \true & 0 &  FOLLOW & [\ZPZNFollowOne,\ZPFollowTwo,\ZPFollowThree,
                            \ZPFollowFour,\ZPFollowFive,\ZPFollowSixPlus] \\ \cline{4-5}
  &   &   &  DATA & \CoeffData \\ \cline{4-5}
  &   &   &  SIGN & \SignZero \\

% Zero parent, zero neighbour, -ve sign prediction
\hline
\true & \true & $<0$ &  FOLLOW & [\ZPZNFollowOne,\ZPFollowTwo,\ZPFollowThree,
                               \ZPFollowFour,\ZPFollowFive,\ZPFollowSixPlus] \\ \cline{4-5}
  &   &    &  DATA & \CoeffData \\ \cline{4-5}
  &   &    &  SIGN & \SignNeg \\

% Zero parent, zero neighbour, +ve sign  prediction
\hline
\true & \true & $>0$ &  FOLLOW & [\ZPZNFollowOne,\ZPFollowTwo,\ZPFollowThree,
                               \ZPFollowFour,\ZPFollowFive,\ZPFollowSixPlus] \\ \cline{4-5}
  &   &    &  DATA & \CoeffData \\ \cline{4-5}
  &   &    &  SIGN & \SignPos \\

% Zero parent, non-zero neighbour, zero sign prediction
\hline
\true & \false & 0 &  FOLLOW & [\ZPNNFollowOne,\ZPFollowTwo,\ZPFollowThree,
                             \ZPFollowFour,\ZPFollowFive,\ZPFollowSixPlus] \\ \cline{4-5}
  &   &   &  DATA & \CoeffData \\ \cline{4-5}
  &   &   &  SIGN & \SignZero \\

% Zero parent, non-zero neighbour, -ve sign prediction
\hline
\true & \false & $<0$ &  FOLLOW & [\ZPNNFollowOne,\ZPFollowTwo,\ZPFollowThree,
                                \ZPFollowFour,\ZPFollowFive,\ZPFollowSixPlus] \\ \cline{4-5}
  &        &    &  DATA & \CoeffData \\ \cline{4-5}
  &        &    &  SIGN & \SignNeg \\

% Zero parent, non-zero neighbour, +ve sign prediction
\hline
\true & \false & $>0$ &  FOLLOW & [\ZPNNFollowOne,\ZPFollowTwo,\ZPFollowThree,
                                \ZPFollowFour,\ZPFollowFive,\ZPFollowSixPlus] \\ \cline{4-5}
  &        &      &  DATA & \CoeffData \\ \cline{4-5}
  &        &      &  SIGN & \SignPos \\

% Non-zero parent, zero neighbour, zero sign prediction
\hline
\false &  \true & 0 & FOLLOW & [\NPZNFollowOne,\NPFollowTwo,\NPFollowThree,
                                    \NPFollowFour,\NPFollowFive,\NPFollowSixPlus] \\ \cline{4-5}
& &      &  DATA & \CoeffData \\ \cline{4-5}
& &      &  SIGN & \SignZero \\

% Non-zero parent, zero neighbour, -ve sign prediction
\hline
\false & \true & $<0$ &  FOLLOW & [\NPZNFollowOne,\NPFollowTwo,\NPFollowThree,
                                      \NPFollowFour,\NPFollowFive,\NPFollowSixPlus] \\ \cline{4-5}
& &      &  DATA & \CoeffData \\ \cline{4-5}
& &      &  SIGN & \SignNeg \\

% Non-zero parent, zero neighbour, +ve sign prediction
\hline
\false & \true & $>0$ &  FOLLOW & [\NPZNFollowOne,\NPFollowTwo,\NPFollowThree,
                                      \NPFollowFour,\NPFollowFive,\NPFollowSixPlus] \\ \cline{4-5}
& &      &  DATA & \CoeffData \\ \cline{4-5}
& &      &  SIGN & \SignPos \\

% Non-zero parent, non-zero neighbour, zero sign prediction
\hline
\false & \false & 0 &  FOLLOW & [\NPNNFollowOne,\NPFollowTwo,\NPFollowThree,
                                    \NPFollowFour,\NPFollowFive,\NPFollowSixPlus] \\ \cline{4-5}
& &      &  DATA & \CoeffData \\ \cline{4-5}
& &      &  SIGN & \SignZero \\

% Zero parent, non-zero neighbour, -ve sign prediction
\hline
\false & \false & $<0$ &  FOLLOW & [\NPNNFollowOne,\NPFollowTwo,\NPFollowThree,
                                       \NPFollowFour,\NPFollowFive,\NPFollowSixPlus] \\ \cline{4-5}
& &      &  DATA & \CoeffData \\ \cline{4-5}
& &      &  SIGN & \SignNeg \\

% Zero parent, non-zero neighbour, +ve sign prediction
\hline
\false & \false  & $>0$ &  FOLLOW & [\NPNNFollowOne,\NPFollowTwo,\NPFollowThree,
                                        \NPFollowFour,\NPFollowFive,\NPFollowSixPlus] \\ \cline{4-5}
& &      &  DATA & \CoeffData \\ \cline{4-5}
& &      &  SIGN & \SignPos \\
\hline

\end{tabular}
\caption{Subband coefficient context sets}\label{contexttable}
\end{table}

\clearpage

\subsection{Low delay wavelet coefficient unpacking}
\label{lowdelayparsing}

This section defines the stream syntax that shall be used for low delay profiles. T
his section defines the syntax and parsing operations only; picture decoding operations
 are defined in Section \ref{picturedec}.

In low delay operation, the Dirac syntax shall partition the wavelet coefficients into a
 number of slices, from all subbands, corresponding to localized areas of the picture 
\begin{comment}(see Figure \ref{})\end{comment}.

A slice shall meet the following requirements:
\begin{enumerate}
\item A single quantizer, weighted for each subband by a quantization matrix, shall be used for quantization of the coefficients in each slice.
\item All wavelet coefficients shall be entropy-coded using variable-length coding alone. Arithmetic coding shall not be used.
\item The number of bytes used per slice shall be the same, to within one byte, for each slice in a picture.
\item Each picture may change the slice parameters within the picture by setting the relevant wavelet transform parameters (Section \ref{sliceparams}).
\end{enumerate}

\begin{informative}
\begin{enumerate}
\item The slice structure implies that in practice incremental picture decoding can be easily achieved without accumulating an entire picture data set, yielding a decoding delay
 proportional to the height of the slices. (The actual achievable delay may be more than one slice height due to vertical filtering delay).
\item Using a fixed number of bits per slice does impact on compression efficiency but
 simplifies both encoder and decoder hardware, and assists a chain of multiple encoders
 and decoders using the same slice parameters in producing identical coding decisions
 and hence no cascading loss. These factors are of great significance in a
professional environment.
\end{enumerate}
\end{informative}

\subsubsection{Overall process}
\label{ldpicture}

The low delay transform data process shall be defined as follows:

\begin{pseudo}{low\_delay\_transform\_data()}{}
\bsCODE{\YTransform = initialise\_wavelet\_data(Y)}{\ref{wltinit}}
\bsCODE{\COneTransform = initialise\_wavelet\_data(C1)}{\ref{wltinit}}
\bsCODE{\CTwoTransform = initialise\_wavelet\_data(C2)}{\ref{wltinit}}
\bsFOR{sy=0}{\SlicesY-1}
  \bsFOR{sx=0}{\SlicesX-1}
    \bsCODE{slice(sx, sy)}{\ref{sliceparsing}}
  \bsEND
\bsEND
\bsCODE{intra\_dc\_prediction(\YTransform[0][\LL])}{\ref{intradcprediction}}
\bsCODE{intra\_dc\_prediction(\COneTransform[0][\LL])}{\ref{intradcprediction}}
\bsCODE{intra\_dc\_prediction(\CTwoTransform[0][\LL])}{\ref{intradcprediction}}
\end{pseudo}

DC values at the top and left edges of a slice shall be predicted from DC values in
 previously decoded slices, which must therefore be retained.

\subsubsection{Slices}
\label{sliceparsing}

This section defines the operation of the $slice(sx,sy)$ process for unpacking
 coefficients within the slice with coordinates $(sx,sy)$. 

Each slice shall contain the relevant coefficients from all subbands and for all 
components. Luma data shall be unpacked rst, and shall be followed by the chroma
 data, in which the chroma component coefficients shall be interleaved. A length code
 shall allow the luma and chroma coefficients each to be terminated early, with remaining
values set to zero by means of bounded read operations.

The overall slice unpacking process shall be defined as follows:

\begin{pseudo}{slice}{sx,sy}
\bsCODE{slice\_bits\_left = 8*slice\_bytes(sx,sy)}{\ref{slicebytes}}
\bsCODE{qindex=read\_nbits(7)}
\bsCODE{slice\_bits\_left -= 7}
\bsCODE{slice\_quantisers(qindex)}{\ref{slicequantisers}}
\bsCODE{length\_bits = \intlog2(8*slice\_bytes(sx,sy)-7)}
\bsCODE{slice\_y\_length=read\_nbits(length\_bits)}
\bsCODE{slice\_bits\_left -= length\_bits}
\bsCODE{\ABitsLeft=slice\_y\_length}
\bsCODE{luma\_slice\_band(0,\LL,sx,sy)}{\ref{lumasliceband}}
\bsFOR{level=1}{\TransformDepth}
  \bsFOREACH{orient}{\HL,\LH,\HH}
    \bsCODE{luma\_slice\_band(level,orient,sx,sy)}{\ref{lumasliceband}}
  \bsEND
\bsEND
\bsCODE{flush\_inputb()}{\ref{blockreadbit}}
\bsCODE{slice\_bits\_left -= slice\_y\_length}
\bsCODE{\ABitsLeft=slice\_bits\_left}
\bsCODE{chroma\_slice\_band(0,\LL,sx,sy)}{\ref{chromasliceband}}
\bsFOR{level=1}{\TransformDepth}
  \bsFOREACH{orient}{\HL,\LH,\HH}
    \bsCODE{chroma\_slice\_band(level,orient,sx,sy)}{\ref{chromasliceband}}
  \bsEND
\bsEND
\bsCODE{flush\_inputb()}{\ref{blockreadbit}}
\end{pseudo}

$slice\_y\_length$ shall satisfy the following condition:
\[slice\_y\_length = 8*slice\_bytes(sx, sy)-7-length\_bits\]

\begin{informative}
Slice decoding makes use of bounded read functions, which return 1 when $\ABitsLeft$
 is zero. This means that remaining coefficients are set to 0, since a solitary 1 is the VLC for 0. The logic of slice decoding applies this twice in each slice: once for luma coefficients, initializing the bits left to $slice\_y\_length$, and a second time to
the chroma coefficients, initializing to the number of bits remaining.
\end{informative}

\subsubsection{Determining the number of bytes in a slice}
\label{slicebytes}

The $slice\_bytes(sx,sy)$ shall be defined as follows:

\begin{pseudo}{slice\_bytes}{sx,sy}
\bsCODE{slice\_num=sy*\SlicesX+sx}
\bsCODE{\begin{array}{rcl}
bytes & = & ((slice\_num+1)*\SliceBytesNum)// \\
& & \SliceBytesDenom
\end{array}}
\bsCODE{\begin{array}{rcl}
bytes &-= & ((slice\_num)*\SliceBytesNum)// \\
& & \SliceBytesDenom
\end{array}}
\bsRET{bytes}
\end{pseudo}

\begin{informative}
This function produces an integer value which will on average be the ratio
of $\SliceBytesNum$ to $\SliceBytesDenom$. In many applications this ratio
will not be an integer number, and the number of bytes will vary from
slice to slice by one byte from time to time. This allows the low delay
syntax to support any compression ratio exactly.
\end{informative}

\subsubsection{Setting slice quantisers}
\label{slicequantisers}

This section defines how quantisers for individual subbands are 
determined from the quantisation matrix and the quantisation index.
The $slice\_quantisers()$ function shall be defined as follows:

\begin{pseudo}{slice\_quantisers}{qindex}
\bsCODE{\Quant[0][\LL]=\max(qindex-\QuantMatrix[0][\LL],0)}
\bsFOR{level=1}{\TransformDepth}
    \bsFOREACH{orient}{HL, LH, HH}
        \bsCODE{qval=\max(qindex-\QuantMatrix[level][orient],0)}
        \bsCODE{\Quant[level][orient]=qval}
    \bsEND
\bsEND
\end{pseudo}

\begin{informative}
The non-negative quantisation matrix values are subtracted from the slice
quantiser value, and so a higher value in the quantisation matrix represents
a lower quantisation index and thus a lower degree of quantisation. The
quantisation index value is also clipped to 0 so that it is non-negative.
This ensures that as many subbands as possible within the slice can be 
coded losslessly. The quantisation matrix values are set as part of decoding
the transform parameters (section \ref{quantmatrix}).
\end{informative}

\subsubsection{Slice subbands}
\label{sliceband}

This section defines the operation of the $luma\_slice\_band( level,orient,sx,sy)$ for 
unpacking individual luma slice subbands, and $chroma\_slice\_band( level,orient,sx,sy)$. for 
unpacking individual chroma slice subbands.

\paragraph{Slice subband dimensions}
$\ $\newline\label{slicedimensions}
The rectangular set of coefficients covered by a slice component ($Y$, $C1$ and $C2$) 
is demarcated by the values $slice\_left, slice\_right, slice\_top, slice\_bottom$, defined as fractions
of the subband dimensions (Section \ref{subbandwidthheight}) by the following functions:
\begin{eqnarray*}
  slice\_left(sx,level,c) & = & (subband\_width(level,c)*sx)//\SlicesX \\
  slice\_right(sx,level,c) & = & (subband\_width(level,c)*(sx+1))//\SlicesX \\
  slice\_top(sy,level,c) & = & (subband\_height(level,c)*sy)//\SlicesY \\
  slice\_bottom(sy,level,c) & = & (subband\_height(level,c)*(sy+1))//\SlicesY
\end{eqnarray*}

where $c=Y$ for luma coefficients, and $C1$ or $C2$ for chroma coefficients.

\begin{informative}
\begin{itemize}
\item The slice subband area formulae correspond to the codeblock area 
formulae for the core syntax (Section \ref{codeblockdimensions}).
\item Slice subbands may change dimension by 1 from one slice to another 
if $\SlicesX$ or $\SlicesY$ do not divide the horizontal or vertical dimensions 
exactly.
\item Chroma slice subbands might not have exactly the scaled dimensions of 
the luma slice subband, since the $\SlicesX$ and $SlicesY$ values may exactly
 divide luma dimensions but not chroma dimensions; and chroma components 
may receive different padding, depending on the transform depth.
\item These issues may be easily avoided in particular applications by choosing 
suitable values for the transform depth and $\SlicesX$ and $SlicesY$.
\end{itemize}
\end{informative}

\paragraph{Luma slice subband data}
\label{lumasliceband}
$\ $\newline
The process for unpacking luma slice coefficients shall be defined as follows:

\begin{pseudo}{luma\_slice\_band}{level, orient, sx, sy}
\bsFOR{y=slice\_top(sy,level,Y)}{slice\_bottom(sy,level,Y)-1}{\ref{slicedimensions}}
  \bsFOR{x=slice\_left(sx,level,Y)}{slice\_right(sx,level,Y)-1}{\ref{slicedimensions}}
    \bsITEM{val}{sintb}{\ref{segol}}
    \bsCODE{q=\Quant[Y][level][orient]}
    \bsCODE{\YTransform[level][orient][y][x]=inverse\_quant(val,q)}
  \bsEND
\bsEND
\end{pseudo}

\paragraph{Chroma slice subband data}
\label{chromasliceband}
$\ $\newline
Chroma slice subband coefficients shall follow luma coefficients within each slice. 
The two chroma components shall be interleaved coefficient-by-coefficient. The 
process for unpacking chroma slice coefficients shall be defined as
follows:

\begin{pseudo}{luma\_slice\_band}{level, orient, sx, sy}
\bsFOR{y=slice\_top(sy,level,C1)}{slice\_bottom(sy,level,C1)-1}{\ref{slicedimensions}}
  \bsFOR{x=slice\_left(sx,level,C1)}{slice\_right(sx,level,C1)-1}{\ref{slicedimensions}}
    \bsCODE{q=\Quant[level][orient]}
    \bsITEM{val}{sintb}{\ref{segol}}
    \bsCODE{\COneTransform[level][orient]=inverse\_quant(val,q)}{\ref{invquant}}
    \bsITEM{val}{sintb}{\ref{segol}}
    \bsCODE{\CTwoTransform[level][orient]=inverse\_quant(val,q2)}{\ref{invquant}}
  \bsEND
\bsEND
\end{pseudo}

\begin{comment}
\begin{informative*}
\subsection{Data adaption and application requirements}
\label{apps}
\end{informative*}
\end{comment}
