In wavelet analysis, each subband represents a filtered and subsampled
version of the picture. Because all the subbands are derived from a
single source image, there is likely to be some form of relationship
between the images in the different subbands.

The coefficients of each subband relate to specific areas in the image.
We find that there is often a correlation between these specific areas
in the different subbands.

The subsampling structure means that a coefficient in the lowest level
corresponds to  2 by 2 block of coefficients in the next level, and so
on up the levels. In the jargon, the low-level component is referred to
as the parent and the higher-level component is referred to as the
child.

When coding picture features (edges on objects especially), significant
coefficients are found distributed across subbands, in position related
by the parent-child structure.

A coefficient of a child is more likely to be significant if its parent
is also significant. Children with zero or small parents seem to have
different statistics from children with large parents or ancestors.

These features allow us to entropy code the wavelet coefficients after
they have been quantised.
