Frames are the fundamental components of a television image. In Dirac,
each transmitted frame comprises picture data in the form of wavelet
transformed residuals and optional motion vector data.  Each frame may
be Inter of Intra coded, the former using a motion compensated
prediction based upon one or two reference frames, the latter not.

A frame may be stored for later use as a reference frame.  The choice
depends on the properties of the frame type: whether it is an Inter or
Intra frame,  the reference status (Reference/NonReference) and the
number of reference frames used in the prediction.

Reference frames are stored in the decoder until they are either
explicitly signalled as expired by the encoder, or implicitly retired
according to the level and profile settings.
