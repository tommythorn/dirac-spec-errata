\begin{comment}
\section{Dirac Pro profiles}

Dirac Pro consists of a set of profiles of Dirac for professional, high-quality, high-bitrate use, 
specifically in television and digital cinema production. Dirac Pro applications cover:
\begin{itemize}
\item Editing and post production
\item Archiving
\item Low-delay links and digital distribution
\item Wireless cameras 
\end{itemize}


The first two application areas do not require a modified syntax, and are defined in Section \ref{editingarchive}.

 The remaining areas require a different
syntax suitable for low delay. In particular, Dirac Pro may be used as a serial-link replacement, and in
that case, parse info, access unit and picture headers may be omitted from the stream. In such a 
configuration the system may not be used in a networked fashion but in point-to-point links
where encoders and decoders are configured pair-wise to use the same video format and other
parameters. 

The syntax in these cases is defined in Section \ref{lowdelay}. Although different from the baseline
Dirac syntax defined in preceding sections, heavy use is made of the individual syntax elements
and decoder functions specified for the baseline profile. In particular the stream access functions
defined in Part \ref{streamaccess} are used for low-level access to the stream.

\section{Editing and archive profiles}
\label{editingarchive}

\subsection{Dirac Edit}
\label{editing}

\subsection{Dirac Archive}
\label{archive}

\section{Low-delay profiles}
\label{lowdelay}

The primary feature of the low-delay profile stream syntax is to group both wavelet coefficients and 
motion vectors into small chunks so that pictures may be decoded progressively. Both the wavelet
transform and motion compensation are specified exactly (i.e. in terms of operations over
the whole picture) as for the baseline profile, and the decoding process is equivalent to obtaining
all motion vectors and all wavelet coefficients for a picture and proceeding as specified in Sections
\ref{idwt} and \ref{mc}. 

However since all the data relating to small areas of the picture is presented together, both the wavelet
transform and motion compensation may be performed bit-by-bit as data arrives. Note that the data is
chunked into slices in the {\em compressed} domain, which may mean that a number of subsequent slices
may also need to be parsed in order to correctly decode the current slice -- due to long wavelet filters
and block overlaps.


\subsection{Dirac Link}
\label{diraclink}

\subsection{Dirac Wireless}
\label{diraclink}

\end{comment}

\section{Low-delay stream parsing}
\label{lowdelayparsing}

This section specifies the stream syntax to be used for low-delay profiles (Link profile and Wireless profile).
The overall format of the low-delay Dirac stream is as specified in Section \ref{streamparsing}. This section
specifies the picture data elements which are used in low-delay operation.

In low-delay operation, the Dirac syntax partitions the wavelet coefficients into a number of sets,
from all subbands, corresponding to a localised area of the picture (Figure \ref{fig:waveletslice}).
 Note that it is the coefficients that are divided,
not the picture. These sets are called slices. A single quantiser, weighted by a quantisation matrix, is used for each slice, 
a slice is  of fixed dimensions within a picture and is also of fixed length in bytes. Each slice interleaves all three components, and hence
the transform parameters for the components are placed in the picture header.

\begin{informative}
Not allowing slice bit allocations to vary within a picture does impact on
compression efficiency, but on the other hand vastly simplifies both encoder and decoder hardware. Furthermore, by
not allowing load-balancing within a buffer, it assists a chain of multiple encoders and decoders using the same
slice parameters in producing identical coding decisions and hence no cascading loss. These are all factors of great 
significance in a professional environment.
\end{informative}

Although this specification is not concerned with network adaption and carriage over data interfaces, low-delay
applications may have specific requirements for headerless and synchronous operation which may imply that
data carried ``on the wire" should differ from a straight-forward encapsulation of the stream. How this should
be done in particular cases is described in Section \ref{apps}.

\subsection{Picture header elements}
\label{ldpictureheader}

This section specifies the operation of the $low\_delay\_header()$ process for parsing low-delay elements
of the picture header. The low delay elements consist of the wavelet transform parameters and the picture
prediction parameters (both placed in the picture header), the slice dimensions 
(width, height, byte allocation) and the quantisation matrix parameters:

\begin{pseudo}{low\_delay\_header}{}
    \bsCODE{transform\_parameters()}{\ref{transformparameters}}
    \bsCODE{picture\_prediction\_parameters()}{\ref{picpredparams}}
    \bsCODE{slice\_parameters()}{\ref{slicedims}}
    \bsCODE{quant\_matrix()}{\ref{quantmatrix}}
\end{pseudo}

\subsubsection{Slice dimensions}
\label{slicedims}

This section specifies the $slice\_dimensions()$ process for parsing the width, height and byte allocation of 
the slices. Default slice dimensions are defined by the video format parameters in the AU header, as per Appendix 
\ref{videoformatdefaults}. They may be overridden as follows:

\begin{pseudo}{slice\_dimensions}{}
\bsITEM{non\_default\_slice\_dims}{bool}
\bsIF{non\_default\_slice\_dims==\true}
    \bsITEM{width\_exp}{uint}
    \bsCODE{\SliceWidth=2^{width\_exp}}
    \bsITEM{height\_exp}{uint}
    \bsCODE{\SliceHeight=2^{height\_exp}}
\bsEND
\bsITEM{non\_default\_slice\_size}{bool}
\bsIF{non\_default\_slice\_size==\true}
    \bsITEM{\SliceBytes}{uint}
\bsEND
\end{pseudo}

If $is\_low\_delay==\true$, the following constraints are placed upon the slice dimensions, the wavelet depth
 and motion compensation block sizes:

\begin{itemize}
\item $\SliceWidth$ and $\SliceHeight$ shall both be divisible by $2^\WaveletDepth$
\item If $is\_inter()==\true$, $\SliceWidth$ shall be divisible by $4*\LumaXBsep$ and $\SliceHeight$ shall be divisible by $4*\LumaYBsep$
\item Chroma and luma subbands must contain corresponding levels of padding, i.e. for all orientations $orient$ and levels $l$,
\begin{eqnarray*}
\width(\YTransform[orient][l]) & = & chroma\_h\_factor*\width(\UTransform[orient][l]) \\
& = & chroma\_h\_factor*\width(\VTransform[orient][l]) \\
\height(\YTransform[orient][l]) & = & chroma\_h\_factor*\height(\UTransform[orient][l]) \\
& = & chroma\_h\_factor*\height(\VTransform[orient][l]) \\
\end{eqnarray*}
\end{itemize}

These restrictions ensure that both wavelet coefficients and (if present) motion data superblocks may be partitioned
correctly by the slices. For example, the third condition restricts the transform depth to be at most 3 on $720\times 576$ 4:2:2 video
since a depth of 4 would lead to padding for chroma coefficients but not for luma, as $2^4=16$ divides 720 but not 360.

\subsubsection{Quantisation matrices}
\label{quantmatrix}

This section specifies the operation of the $quant\_matrix()$ process for setting the $\QuantMatrix$ parameters.

\begin{pseudo}{quant\_matrix}{}
\bsITEM{quant\_matrix\_index}{uint}
\bsIF{quant\_matrix\_index==0}
    \bsITEM{\QuantMatrix[0][\LL]}{uint}
    \bsFOR{level=1}{\WaveletDepth}
        \bsITEM{\QuantMatrix[level][\HL]}{uint}
        \bsITEM{\QuantMatrix[level][\LH]}{uint}
        \bsITEM{\QuantMatrix[level][\HH]}{uint}
    \bsEND
\bsELSE
    \bsCODE{set\_quant\_matrix(quant\_matrix\_index)}
\bsEND
\end{pseudo}

Default quantisation matrices are defined only if $\WaveletDepth$ is 4. If $\WaveletDepth!=4$ then $quant\_matrix\_index$ shall be 0 (custom).
The function $set\_quant\_matrix()$ sets the quantisation matrix based on the wavelet filter and the value of $quant\_matrix\_index$, as per Table
\ref{presetqmatrices}. For $quant\_matrix\_index==1$, this is an ``unweighted" matrix, whose values merely compensate for the differential 
power gain of the different subband filters. Higher values of $quant\_matrix\_index$ imply greater degrees of perceptual weighting.

\begin{table}[!h]
\centering
\begin{tabular}{|c|c|l|}
\hline
\WaveletIndex & quant\_matrix\_indices & {\bf Matrix values} \\
\hline
\end{tabular}
\caption{Preset quantisation matrices}\label{presetqmatrices}
\end{table}
  
\subsection{Low-delay picture parsing}
\label{ldpicture}

This section specifies the $low\_delay\_picture()$ process for parsing pictures in the low-delay syntax. Each picture
is merely a concatenation of slices. The number of slices in a picture is determined from the padded picture dimensions
-- the padding being that required for the application of the wavelet transform to depth $\WaveletDepth$.

\begin{pseudo}{low\_delay\_picture}{}
\bsCODE{initialise\_wavelet\_data()}{\ref{wltinit}}
\bsCODE{ph=2^{\TransformDepth}*\left\lceil\frac{\LumaHeight}{2^{\TransformDepth}}\right\rceil}
\bsCODE{pw= 2^{\TransformDepth}*\left\lceil\frac{\LumaWidth}{2^{\TransformDepth}}\right\rceil}
\bsCODE{\SlicesX=ph//\SliceWidth}
\bsCODE{\SlicesY=ph//\SliceHeight}
\bsFOR{sy=0}{\SlicesY-1}
  \bsFOR{sx=0}{\SlicesX-1}
    \bsCODE{slice(sy, sx)}
  \bsEND
\bsEND
\end{pseudo}

Note that the padded dimensions are the padded luma dimensions. Since these are divisible by $2^\TransformDepth$ they
are {\em a fortiori} divisible by the slice dimensions by the constraints of Section \ref{slicedims}. 
\begin{comment}
\subsubsection{Slice parsing}
\label{sliceparsing}

This section specifies the operation of the $slice(sy,sx)$ process for decoding the slice with coordinates $(sx,sy)$. Each slice
constains coefficients from all subbands, and  the overall slice decoding process is as follows:

\begin{pseudo}{slice}{sy,sx}
\bsFOREACH{c}{Y,U,V}
  \bsCODE{\SliceCompLength=read\_nbits(8)}{\text{NB:shd be variable}}
  \bsCODE{qindex=read\_nbits(7)}{\text{NB:shd be variable}}
  \bsCODE{\BitsLeft=0}
  \bsCODE{q\_idx=\max(qindex-\QuantMatrix[0][\LL],0)}
  \bsCODE{slice\_band(0,\LL,sy,sx,q_idx)}{\ref{sliceband}}
  \bsFOR{level=1}{\WaveletDepth}
    \bsFOREACH{orient}{\HL,\LH,\HH}
      \bsCODE{q\_idx=\max(qindex-\QuantMatrix[level][orient],0)}
      \bsCODE{slice\_band(level,orient,sy,sx,q_idx)}{\ref{sliceband}}
    \bsEND
  \bsEND
\bsEND
\end{pseudo}

\subsubsection{Slice subband parsing}
\label{sliceband}



[Want to change the way the wavelet stuff is done in wlt-dec.tex so that I pass in which component it is, since we're doing all
components here. Move wavelet data initialisation into bs-spec.tex. Interface to wltdecoding will be wlt\_decode(banddata, component)]
\end{comment}
\begin{informative*}
\subsection{Data adaption and application requirements}


\end{informative*}