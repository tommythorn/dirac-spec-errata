
\textbf{Coefficient Context}

The wavelet coefficients are decoded using a probability model derived
from previously decoded coefficients. The probability model determines
the arithmetic decoding contexts that are used. So, before the quantised
coefficient can be read, it is first necessary to select the right
context to use, based on previously decoded coefficients. This is done
by using the Coefficient Context

After the quantised coefficient has been read the contexts, must be
updated to reflect the new probabilities.

\textbf{Codeblock Mode}

A flag which indicates the functionality of the quantisers.

XXX Needs better description.

\textbf{Chroma Height}

The height of the chrominance blocks input to the wavelet transform.

\textbf{Chroma Transform Data}

The Chroma Transform Data is the data after applying the wavelet
transform to the residual.

XXX is it before or after arithmetic coding?

\textbf{Chroma Width}

The width of the chrominance blocks input to the wavelet transform.

\textbf{Horizontal Codeblocks}

The number of Codeblocks across the width of the image. This allows us
to calculate the width of the Codeblocks in pixels.

\textbf{Luma Height}

The height of the luminance blocks input to the wavelet transform.

\textbf{Luma Transform Data}

The Luma Transform Data is the data after applying the wavelet transform
to the residual.

XXX is it before or after arithmetic coding?

\textbf{Luma Width}

The width of the luminance blocks input to the wavelet transform.

\textbf{Quantiser Codeblock}

XX Is this the definition of the codeblocks subject to the spatial
partition??????

\textbf{Spatial Partition}

The Spatial Partition enables us to chose whether to use separate
quantisers for each code block in a subband or a single quantiser to be
used for the whole subband.

These different spatial partition modes can be used to support
region-of-interest coding.



\textbf{Subband}

The subband names, LL, LH, HL, HH, correspond to the frequencies in the
subband.

Wavelet analysis results in a filtered and subsampled version of the
picture. When the picture is subject to wavelet analysis in two
directions this results in the four so-called subbands termed Low-Low
(LL), Low-High (LH), High-Low(HL) and High-High (HH). The first
descriptor refers to horizontal frequencies, the second to vertical
frequencies. L represents low frequencies and H represents high
frequencies. The order of these letters (horizontal, vertical) is
consistent with names used in the literature.

However this is the opposite order from with the array indices used in
this document.  XX Do we need to say the so what factor.

When several levels of wavelet are used, the LL band (only) is
iteratively decomposed. As a consequence, there are Low-High (LH),
High-Low(HL) and High-High subbands for each iteration, plus an extra
subband for the outstanding Low-Low information. This latter is
effectively a low frequency, sub-sampled version of the original.

In the processing, Subband is the metadata describing the transformed
data, which component it is, i.e. which level and which band.

\textbf{Transform Data}

Transform Data is the data after applying the wavelet transform to the
residual.

XXX is it before or after arithmetic coding?

\textbf{Transform Depth}

Same as Wavelet Depth: the total number of wavelet levels.

XX Do not understand why there are two names for it.

\textbf{Transform Parameters}

The transform parameters are the framework used for delivering the
Transform Data. The Transform Parameters describe the Wavelet Filter,
the Wavelet Depth and the Spatial Partition.

See also Wavelet Filter, Wavelet Depth and Spatial Partition

\textbf{Vertical Codeblocks}

The number of codeblocks across the height of the image. This allows us
to calculate the height of the Codeblocks in pixels.

\textbf{Wavelet Coefficients}

The Wavelet Coefficients are decoded using a probability model derived
from previously decoded coefficients. The probability model determines
the contexts that are use. Different sets of contexts are used for the
XXX"follow" bits, but a single context is used for all the XXX"data".

The magnitude and the sign of the coefficient are modelled separately.

In coding the magnitude, different sets of contexts are used for the
"follow" bits, but a single context is used for all the "data". The
probability distribution of coefficient magnitude is modelled as
depending on whether the parent coefficient was zero or non zero. The
probability distribution is also assumed to depend on the sum of
(previously decoded) neighbouring coefficients. Because the probability
distribution is usually peaked around zero it is only the context that
models the probability of zero/non-zero coefficient that changes with
the neighbourhood sum.

The probability distribution of the sign of the coefficient is assumed
to depend on an appropriate neighbouring coefficient.

\textbf{Wavelet Depth }

Wavelet Depth is the total number of wavelet levels. Usually this is
four, i.e. we do 4 splits both horizontally and vertically.

\textbf{Wavelet Filter}

XXX [filter pairs??] The definition of the filter used in the wavelet
analysis (and this, as a consequence, defines the filter used in the
synthesis process too).

\textbf{Wavelet index}

The Wavelet index is a parameter which allows us to identify which of
the wavelet filters we are using for the lifting filters.

\textbf{Zero Residual}

XXX Zero Residual is a flag which signals that there is no Residual to
decode.
