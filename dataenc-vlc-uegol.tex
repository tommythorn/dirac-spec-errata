Unsigned exp-Golomb data is decoded to produce unsigned integer values.
The format consists of two parts. A prefix part, consisting of n zeroes
followed by a one, indicates how many further bits to read. A suffix
part, consisting of n bits, is then used to determine the value. The
decoding procedure for extracting a value VALUE is mathematically
equivalent to:

\begin{verbatim}
COUNT= 0
while( !read\_bits(1) ) {
    COUNT++
}
VALUE = (1<<COUNT) -1 + read\_bits( COUNT )
\end{verbatim}

where the value returned by read\_bits( COUNT ) is interpeted as a binary
representation of an unsigned integer with most-significant bit first.
The bit sequences corresponding to some values are shown in  .

\begin{figure}[h]
\begin{tabular}{c|c}
Bit sequence & Decoded value \\
\hline\\
1       &  0\\
010     &  1\\
011     &  2\\
00100   &  3\\
00101   &  4\\
00110   &  5\\
00111   &  6\\
0001000 &  7\\
0001001 &  8\\
0001010 &  9\\
0001011 & 10
\end{tabular}

\caption{Example conversions from uegol-coded values to binary}
\end{figure}



