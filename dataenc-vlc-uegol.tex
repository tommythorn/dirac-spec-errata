This section defines the unsigned interleaved exp-Golomb data format and the operation
of the $read\_uint()$ function. 

Unsigned interleaved exp-Golomb data is decoded to produce unsigned
 integer values.The format consists of two interleaved parts, which are interleaved
in the encoding, and each code is an odd number, $2K+1$ bits, in length.

The $K+1$ bits in the even positions (counting from zero) are the "follow" bits, and 
the $K$ bits in the odd positions are the "data" bits which are used to construct
the decoded value itself. A follow bit value of $0$ indicates a subsequent data bit,
whereas a follow bit value of $1$ terminates the code:

$0 d_{K-1} 0 d_{K-2} ... 0 d_{0} 1$

The data bits themselves are the binary representation 
$d_{K-1} d_{K-2}... d_0$ of the first $K$ bits of the $K+1$-bit number 
$N+1=d_{K-1} d_{K-2}... d_0$, where $N$ is the number to
be decoded. A table of encodings of the first 10 values is shown in figure
\ref{uegolcodings}.

\begin{figure}[h]
\begin{tabular}{l|c}
Bit sequence & Decoded value \\
\hline\\
1                 &  0\\
0 0 1             &  1\\
0 1 1             &  2\\
0 0 0 0 1         &  3\\
0 0 0 1 1         &  4\\
0 1 0 0 1         &  5\\
0 1 0 1 1         &  6\\
0 0 0 0 0 0 1     &  7\\
0 0 0 0 0 1 1     &  8\\
0 0 0 1 0 0 1     &  9\\

\end{tabular}

\caption{Example conversions from unsigned interleaved exp-Golomb-coded 
values to unsigned integers \label{uegolcodings}}
\end{figure}

Although apparently complex, this code has a very simple decoding loop. The 
$read\_uint()$ function returns an unsigned integer value and is defined by the recipe:

\begin{pseudo}{read\_uint}{}
\bsCODE{value = 1}
\bsWHILE{read\_bool()== \false}
  \bsCODE{value \ll = 1}
  \bsIF{read\_bool()}
    \bsCODE{value += 1}
  \bsEND
\bsEND
\bsCODE{value -= 1}
\bsRET{value}

\end{pseudo}


\begin{informative}
Conventional exp-Golomb coding places all "follow" bits at the beginning as a prefix. This is
easier to read, but requires that a count of the prefix length be maintained. Values can only
be decoded in two loops - the prefix followed by the data bits. Interleaved exp-golomb 
coding allows values to be decoded in a single loop, without the need for a length count.
\end{informative}
