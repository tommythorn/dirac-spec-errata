This section specifies the $transform\_data()$ process for
unpacking (parsing and inverse
quantisation) of wavelet coefficient data
from the Dirac stream. The result of this process is a set
of fully-populated wavelet subband data arrays, as defined in
Section \ref{wltdecodeconventions}, containing coefficient
data with full dynamic range.

Wavelet coefficients are packed in one of two possible formats.
In the core syntax, coefficients are grouped within individual
subbands, representing a range of spatial frequencies, from the
lowest to the highest. A full set of subbands is encoded for each
video component in turn. 

In the low-delay syntax, coefficients
are groups into `slices' which represent coefficient pertaining
to an area of the picture. Each slice contains both luma and chroma
data, and all spatial frequency bands. Unpacking a slice therefore
allows an area of picture to be extracted without extracting (or even
receiving) the remaining picture data.

The overall process is as follows:

\begin{pseudo}{transform\_data}{}
\bsIF{is\_low\_delay()==\false}
    \bsCODE{\YTransform = core\_transform\_data(Y)}{\ref{transformdec}}
    \bsCODE{\COneTransform = core\_transform\_data(C1)}{\ref{transformdec}}
    \bsCODE{\CTwoTransform = core\_transform\_data(C2)}{\ref{transformdec}}
\bsELSE
    \bsCODE{low\_delay\_transform\_data()}{\ref{ldpicture}}
\bsEND
\end{pseudo}

\subsection{Subband data structures}
\label{wltdecodeconventions}

\subsubsection{Wavelet data initialisation}

\label{wltinit}

This section specifies the $initialise\_wavelet\_data(comp)$ process, which returns a structure which will
contain the wavelet coefficients for the component (Y, C1 or C2) indicated by $comp$. 

For the purposes of this specification, this is a four-dimensional array $data$,
where individual subbands are two-dimensional arrays accessed by level and orientation:

$band = data[level][orient]$

Valid levels are integers from in the range 0 to $\TransformDepth$ inclusive. 
Level 0 consists of a single subband with orientation \LL. 
All other levels consist of 3 subbands of orientation \LH, \HL, 
and \HH. The orientations correspond to either low- or high-pass filtering
horizontally and vertically: so the \LH band consists of coefficients derived
from horizontal low-pass filtering and vertical high-pass filtering. The subbands
partition the spatial frequency domain by orientation and level as illustrated
in Figure \ref{fig:orientlevel}.

Each subband array is initialised so that:
\begin{eqnarray*}
\width(data[level][orient]) & = & subband\_width(level,comp) \\
\height(data[level][orient]) & = & subband\_height(level,comp)
\end{eqnarray*}

as specified in Section \ref{subbandwidthheight}. These dimensions correspond 
to a wavelet transform being performed on a copy of the component data which 
has been padded (if necessary) so that its
dimensions are a multiple of $2^{\TransformDepth}$.

Individual subband coefficients are signed integers accessed by vertical and 
horizontal coordinates within the subband
\[c = data[level][orient][y][x]\]

where the range of allowable coordinates for a subband coefficient is 
$0\leq x<subband\_width(level,comp)$ and
$0\leq y<subband\_height(level,comp)$.

\setlength{\unitlength}{1em}
\begin{figure}[!ht]
\centering
\begin{picture}(25,35)

%Main Grid
\multiput(0,0)(16,0){3}%
  {\line(0,1){32}}
\multiput(0,0)(0,16){3}%
  {\line(1,0){32}}

%Second Grid
\put(0,24){\line(1,0){16}}
\put(8,16){\line(0,1){16}}

%Third Grid
\put(0,28){\line(1,0){8}}
\put(4,24){\line(0,1){8}}

%Fourth Grid
\put(0,30){\line(1,0){4}}
\put(2,28){\line(0,1){4}}


%Put Levels

\put(24,34){Level 4}
\put(23,36){\vector(0,-1){3}}

\put(13,34){Level 3}
\put(12,36){\vector(0,-1){3}}

\put(7,34){2}
\put(6,36){\vector(0,-1){3}}

\put(4,34){1}
\put(3,36){\vector(0,-1){3}}

%Put decomposition numbers
\put(23,24){4-HL}
\put(7,8){4-LH}
\put(23,8){4-HH}
\put(11,28){3-HL}
\put(3,20){3-LH}
\put(11,20){3-HH}
\put(5,30){2-HL}
\put(1,26){2-LH}
\put(5,26){2-HH}
\put(2.2,31){\tiny 1-HL}
\put(0.2,29){\tiny 1-LH}
\put(2.2,29){\tiny 1-HH}
\put(0.2,31){\tiny 0-LL}

  
\end{picture}
\caption{Subband decomposition of the spatial frequency domain showing subband 
numbering, for a 4-level wavelet decomposition}\label{fig:orientlevel}

\end{figure}

\subsubsection{Dimensions of wavelet subbands}
\label{subbandwidthheight}

This section defines the values of the $subband\_width(level, comp)$ and $subband\_height(level,comp)$
functions, giving the width and height of subbands at a given level for a given component, and hence the range
of subband vertical and horizontal indices. 

If $comp==Y$, set
\begin{eqnarray*}
w & = & \LumaWidth \\
h & = & \LumaHeight
\end{eqnarray*}

Otherwise, set
\begin{eqnarray*}
w & = & \ChromaWidth \\
h & = & \ChromaHeight
\end{eqnarray*}

Define the padded dimensions of the component by
\begin{eqnarray*}
pw = 2^{\TransformDepth}*\left\lceil\frac{w}{2^{\TransformDepth}}\right\rceil\\ 
ph = 2^{\TransformDepth}*\left\lceil\frac{h}{2^{\TransformDepth}}\right\rceil
\end{eqnarray*}

If $level==0$,
\begin{eqnarray*}
subband\_width(level) & = & pw//2^{\TransformDepth} \\
& = & \left\lceil\frac{w}{2^{\TransformDepth}}\right\rceil \\
subband\_height(level) & = & ph//2^{\TransformDepth} \\
& = & \left\lceil\frac{h}{2^{\TransformDepth}}\right\rceil
\end{eqnarray*}

If $level>0$
\begin{eqnarray*}
subband\_width(level) & = & pw//2^{\TransformDepth-level+1} \\
& = & 2^{level-1}*\left\lceil \frac{w}{2^{\TransformDepth}}\right\rceil \\
subband\_height(level) & = & ph//2^{\TransformDepth-level+1} \\
& = & 2^{level-1}*\left\lceil\frac{h}{2^{\TransformDepth}}\right\rceil
\end{eqnarray*}

\begin{informative}
In encoding, these padded dimensions may be achieved by padding the 
component data up to the padded dimensions and applying the forward
Discrete Wavelet Transform (the inverse of the operations specified in
Section \ref{idwt}). Any values may be used for the padded data, although
the choice will affect wavelet coefficients at the right and bottom 
edges of the subbands. Good results, in compression terms, may be obtained
 by using edge extension for intra pictures and zero extension for inter 
pictures. A poor choice of padding may cause visible artefacts near the
bottom and right edges at high levels of compression.
\end{informative}

\subsection{Inverse quantisation}
\label{invquant}

This section specifies the operation of inverse quantisation, which scales the
dynamic range of unpacked wavelet coefficients according to a pre-determined factor.
The inverse quantisation operation is common to both the low-delay and core syntax.

The $inverse\_quant()$ function is defined by:

\begin{pseudo}{inverse\_quant}{quantised\_coeff, quant\_index}
\bsCODE{magnitude = |{quantised\_coeff}|}
\bsCODE{magnitude *= quant\_factor(quant\_index)}{\ref{quantfacs}}
\bsIF{magnitude!=0}
  \bsCODE{magnitude += quant\_offset(quant\_index)}{\ref{quantfacs}}
  \bsCODE{magnitude += 2}
  \bsCODE{magnitude = magnitude//4}
\bsEND
\bsRET{\sign( quantised\_coeff )*magnitude}
\end{pseudo}

\begin{informative}
The pseudocode description separates inverse quantisation from coefficient unpacking. However, 
since dead-zone quantisation is used, the $inverse\_quant()$ function must compute
the magnitude. Hence it is more efficient to first extract the coefficient magnitude,
then inverse quantise, and then extract the coefficient sign. 

Note that an unbiased division by 4 is applied: the additional 2 may be absorbed into
the quantisation offset, but is separated in this specification.
\end{informative}

\subsubsection{Quantisation factors and offsets}
\label{quantfacs}

This section specifies the operation of the $quant\_factor()$ and 
$quant\_offset()$ functions for performing inversion quantisation.

Quantisation factors represent an integer approximation of quarter-bit values 
with two bits of accuracy i.e. of $(2^{\frac{index}{4}+2})$. Note that 64 bits 
of accuracy is required to compute these factors correctly up to $index=128$.

Quantisation factors are set as follows:

\begin{pseudo}{quant\_factor}{index}
\bsCODE{q=\max(index,0)}
\bsCODE{base = 2^{q//4}}
\bsIF{ (q\%4)==0 }
  \bsRET{4*base}
\bsELSEIF{ (q\%4)==1 }
  \bsRET{(503829*base+52958)//105917}
\bsELSEIF{ (q\%4)==2 }
  \bsRET{(665857*base+58854)//117708}
\bsELSEIF{ (q\%4)==3 }
  \bsRET{(440253*base+32722)//65444}
\bsEND
\end{pseudo}

For intra pictures, offsets are approximately $1/2$ of the 
quantisation factors, and for inter pictures they are $3/8$ - these
mark the reconstruction point within the quantisation interval:

\begin{pseudo}{quant\_offset}{index}
\bsIF{index==0}
  \bsCODE{offset = 1}
\bsELSE
  \bsIF{ is\_intra() }
    \bsIF{index==1}
      \bsCODE{offset = 2}
    \bsELSE
      \bsCODE{ offset=(quant\_factor(index)+1)//2 }
    \bsEND
  \bsELSE
    \bsCODE{ offset=(quant\_factor(index)*3+4)//8 }
  \bsEND
\bsEND
\bsRET{offset}
\end{pseudo}

\begin{informative}
The quantisation offsets have been selected so as to make inverse quantisation
and re-quantisation by the same quantisation factor transparent. This requires that
\[3\leq quant\_offset+2<quant\_factor\] -- hence the special conditions for 
quantisation indexes 0 and 1.
\end{informative}
