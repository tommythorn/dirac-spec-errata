This section defines what the Bytestream. As the Bytestream contains a
lot of metadata which is in arithmetically-coded format, there is a vast
volume of data which cannot be instantly be assigned to particular data
structures. This data will require processing (as described in Sections
XXXXXX) before the Bytestream can be fully decoded.

A Stream is a concatenation of Sequences, which are Video Sequences
that have constant Source Parameters (eg, picture size, aspect ratio,
etc).

The parameters specified in the Access Unit Header reamain the same
throughout a Sequence.  That is, the parameters in later Access Unit
Headers simply repeat those in earlier headers and are provided to
provide entry points to start decoding the byte stream.

Presentation order picture numbers within a sequence must be contiguous.

If the parameters need to change the only way to do this is to signal
the end of sequence and start a new sequence.

The process of editing two sequences together would introduce
discontiguous presentation order picture numbers.  This is accommodated
by introducing End of Sequence parse codes before a cut so that the
decoder would restart after a cut.
\annotate{shas}{May need to explain that the stream is presented as a
series of bytes. When being read in a bitwise fashion, bytes are
presented with the most significant bit first/last?}

