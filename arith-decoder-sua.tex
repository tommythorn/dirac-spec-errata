\begin{comment}
Signed unary arithmetic decoding su\_arith\_decode() has the following
pseudocode representation:

VALUE= uu\_arith\_decode()

if ( VALUE!=0 )

{

    if ( !binary\_arith\_decode( choose\_context() ) )

    VALUE=-VALUE

}

\end{comment}

%src: tim-0.9.1.48

reads and returns a signed integer encoded in the bytestream as an
arithmetic coded truncated unary binarisation.  ``context\_list'' is a two
element list. The second element is the context for the sign bit. The
first element is a  context list for reading the magitude of the signed
integer.  The magnitude context list is a list of contexts for each bin.
If the number of contexts in the magnitude context list is less than the
bin number then the last context on the list is used. That is a common
context is used for all the higher bins.

Read Arithmetic Coded Signed Integer
\begin{verbatim}
read_sua(context_list):
    #Read magnitude
    magnitude_context_list = context_list[0]
    magnitude = read_uua(magnitude_context_list)
    context_list[0] = magnitude_context_list
    if ( magnitude==0 ):
        value = 0
    else:
        #Read sign
        sign_context = context_list[1]
        sign = read_ba(sign_context)
        context_list[1] = sign_context
        #Determine value
        if ( sign = False):
            value = magnitude
        else:
            value = -magnitude
    return value
\end{verbatim}
