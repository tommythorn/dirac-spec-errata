This Section specifies the $transform_data()$ process for decoding wavelet subband
coefficients.
elements of the bitstream. This process is invoked in decoding Subband
Data elements.

Inputs to this process are: IS\_INTRA, SUBBAND\_NUM, NUM\_SUBBANDS,
SPATIAL\_PARTITION, PARTITION\_INDEX, MULTI\_QUANT, SUBBAND\_WIDTH,
SUBBAND\_HEIGHT; previously decoded subband data arrays within the same
Transform Data element.

Outputs of this process are: A two-dimensional array SUBBAND[][] of
decoded subband coefficients.

 The Subband Coefficient Data unit is only present is ZERO\_SUBBAND\_FLAG is
FALSE. It is byte-aligned and occupies a whole number of bytes, padded
with zero bits as necessary. It consists of a byte-aligned block of raw
arithmetically-coded bytes of length SUBBAND\_LENGTH.

Subband decoding conventions are specified in Section  and the overall
subband decoding process is specified in Section .

\begin{verbatim}
transform_data():
    subband(0,LL)
    for level in range(1,transform_depth+1):
        for band in [LH, HL, HH]:
            subband(level,band)
\end{verbatim}

\begin{verbatim}
subband(level,band):
    subband_data_length[level][band] = read_uegol()
    if (subband_data_length[level][band] != 0):
        quantiser_index[level][band] = read_uegol()
        BYTEALIGN()
        compressed_subband_data[level][band] = read_chunk(subband_data_length[level][band])
\end{verbatim}


