\documentclass[a4paper,9pt]{extarticle}
\usepackage{array}
\usepackage{amsmath}
\usepackage{amssymb}
\usepackage{graphicx}
\usepackage{subfigure}
\usepackage{verbatim}
\usepackage{framed}
\usepackage{color}
\usepackage{fancyhdr}
\usepackage{layout}

\DeclareFontShape{OT1}{cmtt}{bx}{n}{<->cmbtt10}{}

\setlength\voffset{-0.5in}
\addtolength\textheight{1.5in}
\setlength\oddsidemargin{0in}
\setlength\marginparwidth{1.25in}
\setlength\textwidth{\paperwidth}
\addtolength\textwidth{-1.25in}
\addtolength\textwidth{-\oddsidemargin}
\addtolength\textwidth{-\marginparsep}
\addtolength\textwidth{-\marginparwidth}

\linespread{1.2}
\setlength\parskip{6pt}
\setlength\parindent{0pt}

\pagestyle{fancy}
\fancyhf{}
\fancyhead[EL,OR]{\bfseries\thepage}
\fancyhead[ER]{\bfseries\leftmark}
\fancyhead[OL]{\bfseries\rightmark}
\fancyfoot[EC,OC]{\input{.version-date}}
\renewcommand{\headrulewidth}{0.5pt}
\renewcommand{\footrulewidth}{0.5pt}
\addtolength{\headheight}{0.5pt}
\addtolength{\textheight}{-0.5pt}
\fancypagestyle{plain}{% no headers on plain pages
    \fancyhead{}%
    \renewcommand{\headrulewidth}{0pt}%
}

%\bibliographystyle{IEEEtran}
%\renewcommand\bibname{References}

\DeclareMathOperator{\clip}{clip}

\definecolor{shadecolour}{gray}{0.90}
\newenvironment{informative}[0]%
{\def\FrameCommand{\colorbox{shadecolour}}%
 \MakeFramed {\setlength\hsize{0.8\textwidth}}{\bf Informative:}}%
{\endMakeFramed}

\definecolor{commentcolour}{rgb}{1,0,0}
\begin{document}
\newcommand{\annotate}[2]{\marginpar{\color{commentcolour} #2 {\bf --- #1}}}

\newcounter{indent}
\newlength{\indentx}
\setlength{\indentx}{1em}
\newenvironment{pseudo}[2]
    {\newcommand{\dfindent}{\\\hline\hspace{\value{indent}\indentx}}
     \newcommand{\bsIF}[1]{\dfindent\textrm{if (\textrm{##1}):}\stepcounter{indent} & &}
     \newcommand{\bsEND}{\addtocounter{indent}{-1}}
     \newcommand{\bsELSE}{\addtocounter{indent}{-1}\dfindent\textrm{else:}\stepcounter{indent} & &}
     \newcommand{\bsELSEIF}[1]{\addtocounter{indent}{-1}\dfindent\textrm{else if (\textrm{##1}):}\stepcounter{indent} & &} 
     \newcommand{\bsWHILE}[1]{\dfindent\textrm{while (\textrm{##1}):}\stepcounter{indent} & &}
     \newcommand{\bsFOREACH}[2]{\dfindent\textrm{foreach \textrm{##1} in \textrm{##2}:}\stepcounter{indent} & &}
     \newcommand{\bsFOR}[2]{\dfindent\textrm{for \textrm{##1} to \textrm{##2}:}\stepcounter{indent} & &}
     \newcommand{\bsRET}[1]{\dfindent\textrm{return \textrm{##1}}\addtocounter{indent}{-1} & &}
     \newcommand{\bsCODE}[1]{\dfindent\textrm{##1} & &}
     %\newcommand{\bsITEM}[3]{\dfindent\textbf{\textrm{##1}} & ##2 & ##3}
     \newcommand{\bsITEM}[3]{\dfindent\textbf{\textrm{##1 = read\_##2()}} & & ##3}
     \setcounter{indent}{1}
     \hspace{0.5in}
%     \begin{tabular}{|m{3.75in}|m{0.6in}|m{0.25in}|}
     \begin{tabular}{|m{4.35in}m{0.0in}|m{0.25in}|}
         % firstline is function definition
         \hline
%         \textrm{#1(#2)} : & \textbf{Type} & \textbf{Ref}
         \textrm{#1(#2)} : &  & \textbf{Ref}
    }
    {    % last line is endof function
         \\\hline
         \end{tabular}
         }

\tableofcontents
\clearpage
%%%%%%%%%%%%%%%%%%%%%%%%%%%%%%%%%
% -       Introductions       - %
%%%%%%%%%%%%%%%%%%%%%%%%%%%%%%%%%

\section{Introduction}%%%%%%%%%%%%%%%%%%%%%%%%%%%%%%%%%%%%%%%%%%%%%%%%%%%%%%
% - This part relates to informative and normative - %
% - concepts for Dirac parsing and decoding        - % 
%%%%%%%%%%%%%%%%%%%%%%%%%%%%%%%%%%%%%%%%%%%%%%%%%%%%%%

%\section{Overview of Dirac video coding (Informative)}\input{dirac-overview}

%\clearpage
%\section{Logical constructs used in Dirac}\input{logicalstructs}

\clearpage
\section{The conventions used in the specification}\input{spec-conventions}

%\clearpage
%\section{Semantics}\input{semantics}


\subsection{Purpose}
\label{intropurpose}
Dirac was developed to address the growing complexity and cost of current video
compression technologies, which provide greater compression efficiency
at the expense of implementing a very large number of tools. Dirac is
a powerful and flexible compression system, yet uses only a small number
of core tools. A key element of its flexibility is its use of the wavelet
multi-resolution transform for compressing pictures and motion-compensated 
residuals, which allows Dirac to be used across a very wide range of resolutions
without enlarging the toolset.

Dirac is an Open Source software project, and reference implementations
of the decoder and encoder are available at \underline{http://sourceforge.net/projects/dirac}.
A high-performance implementation, called Schrodinger, is also available
open source at \underline{http://schrodinger.sourceforge.net}.

\subsection{Scope}
\label{introscope}

This document specifies normative decoder operations and 
stream syntax. The stream syntax is primarily specified by means of
pseudocode, the conventions of which are described in Section \ref{pseudocode}.
The decoder operations are specified by means of a mixture of pseudocode
and conventional mathematical symbolism.

A number of other elements are also included for informative purposes.
The specification is not an implementation guide, and in the interests
of clarity many of the operations are specified in a way that would not be 
efficient to implement. However, we have attempted
to indicate where this is so, and to suggest ways in which an efficient implementation
may be achieved, but these are by no means exhaustive. An optimised Open Source
software Dirac encoder and decoder system, named Schr\"odinger, is available at
\underline{http://sourceforge.net/projects/schrodinger}, and may be studied to aid implementation.

In addition, we are well aware that many users of this document may wish
to make both encoders and decoders. There are many sources of information
on how to design efficient compression algorithms, for example for entropy coding,
motion estimation, frame-dropping, rate control, motion estimation and 
rate-distortion optimisation. This document does not attempt to address these
issues in detail, but to provide supplementary information where appropriate
to allow those reasonably `skilled in the art' to develop a Dirac encoder
rapidly and accurately, and approach design compromises knowledgably.

\subsection{Status}
\label{introstatus}

This is version $\SpecVersion$ of the Dirac specification. The document includes
a full description of the core Dirac stream syntax and decoder operations, together
with compatible extensions to support low-delay operation. It does
not yet contain a full specification of profiles and levels supported by Dirac.

\subsection{Document structure}
\label{introdocstruct}

This document specifies the Dirac decoder and stream structure in terms of
a layered model:

\begin{enumerate}
    \item Stream data access
    \item Parsing and interpretation of the Dirac stream metadata
    \item Unpacking of Dirac data -- coefficients and motion data
    \item Picture decoding operations
\end{enumerate}

Stream data access consists of the operations used to extract data values
(of boolean and integer type) from a raw Dirac bitstream. These include
data that has been encoded `literally' (i.e. according to conventional bit-wise
representations), variable-length codes, and data entropy coded using arithmetic
encoding. Stream data access methods are used both by parsing and unpacking operations.

Parsing and interpretation defines the structure of Dirac streams, including
random access points and navigation. 

Unpacking comprises parsing transform and motion data, and performing
inverse quantisation to reconstruct wavelet coefficients, and defines
intermediate decoder data structures in which extracted data is stored. This
step may be seen either as a stream parsing operation or as a decoding
operation, but it is separated here for clarity, and also because it
deals with intermediate data structures (transform coefficients and motion vectors), 
neither directly present in the bitstream nor output by the decoder.

Picture decoding operations produce decoded pictures from these populated
data structures by applying specified functions to them. These operations
are logically distinct from those for navigating the stream, reading the stream data
or reconstructing coefficients.

Note in particular that the distinction between parsing and picture decoding is
{\em not} exactly that between syntax and semantics: complex semantics are
required for correct parsing of the stream as well as for decoding pictures. 

It is perhaps unusual in a specification to separate these layers quite so distinctly, 
and our purpose in doing so is to provide much greater clarity. For implementors,
we hope that the decoupling of the stream structure from the (computationally intensive)
picture decoding processes will avoid imposing
implicit design decisions merely through the style of the specification. Many
other users of the specification will not be interested in the precise format
of stream elements but in how the underlying algorithm works - or vice-versa.
It should be possible to construct a Dirac parsing engine, for example for
frame skipping in video playback applications, extremely simply and without
requiring comprehension of the entire specification.

\begin{comment}
This layered structure is reflected in the structure of the specification,
which, after defining conventions used, is divided into three
corresponding parts: stream data access, defining functions for data types; 
accessing and parsing the Dirac bitstream and populating data
structures (including the wavelet coefficients and motion data); and 
high-level decoder operations and picture output, specifically the
inverse wavelet transform and motion compensation.

In addition to these parts, appendices deal with standard settings, parameter
presets and levels and profiles.
\end{comment}

%%%%%%%%%%%%%%%%%%%%%%%%%%%%%%%%%
% - Main body of the document - %
%%%%%%%%%%%%%%%%%%%%%%%%%%%%%%%%%
%\clearpage
%\part{Dirac concepts}
\clearpage
\section{Conformance notation}
Normative text is text that describes elements of 
the design that are indispensable or contains the 
conformance language keywords: `shall', `should', or `may'. 
Informative text is text that is potentially helpful to the user, 
but not indispensable, and can be removed, changed, or added 
editorially without affecting interoperability. Informative 
text does not contain any conformance keywords.

All text in this document is, by default, normative, 
except: the Introduction, any section explicitly labelled as `Informative' 
or individual paragraphs that are also indicated in this way.

The keywords `shall' and `shall not' indicate requirements 
strictly to be followed in order to conform to the document 
and from which no deviation is permitted

The keywords, `should' and `should not' indicate that, among 
several possibilities, one is recommended as particularly suitable, 
without mentioning or excluding others; or that a certain course 
of action is preferred but not necessarily required; or that 
(in the negative form) a certain possibility or course of action is deprecated but not prohibited.

The keywords `may' and `need not' indicate courses of action 
permissible within the limits of the document.

The keyword `reserved' indicates a provision that is not 
defined at this time, shall not be used, and may be defined 
in the future. The keyword `forbidden' indicates `reserved' and in
 addition indicates that the provision will never be defined in the future.

A conformant implementation according to this document is one that includes 
all mandatory provisions (`shall') and, if implemented, all recommended 
provisions (`should') as described. A conformant implementation need 
not implement optional provisions (`may') and need not implement them as described.

\subsection{Normative References}
Normative references are external documents referenced in normative 
text that are indispensable to the user. Bibliographic references 
are references made in informative text or are those otherwise not
 indispensable to the user.

The following standards contain provisions which, through
 reference in this text, constitute provisions of this standard. 
 At the time of publication, the editions indicated were valid. 
 All standards are subject to revision, and parties to agreements
  based on this standard are encouraged to investigate the 
  possibility of applying the most recent edition of the standards indicated below.

\begin{enumerate}  
\item	Proposed ISO/IEC MPEG and ITU-T VCEG [JVT] - YCoCg: A Color Space with RGB Reversibility and Low Dynamic Range.
\item	Recommendation ITU-R BT.709-5: Parameter values for the HDTV standards for production and international programme exchange, 2002. 
\item	ITU-BT.1361: Worldwide unified colorimetry and related characteristics of future television and imaging systems.
\item	ITU-BT.1700: Characteristics of composite video signals for conventional analogue television systems.
\item	SMPTE 170M-1994, for Television: Composite Analog Video Signal - NTSC for Studio Applications.
\item	EBU Tech 3213-1994: Standard for Chromaticity Tolerances for Studio Monitors.
\item	CIE Publication15:2004: Colorimetry.
\item	SMPTE 428.1: Digital Cinema Distribution Master - (DCDM) Image Characteristics.
\end{enumerate}

\clearpage
\section{Definition of acronyms, terms and naming conventions}This section defines the acronyms, terms and naming conventions used in the VC-2 specification.

\subsection{Acronyms}

\begin{description}
\item[4CIF:] Four times Common Image Format
\item[AU:] Access Unit
\item[CIE:] Commission internationale de l'�clairage (International Commission on Illumination)
\item[CIF:] Common Image Format
\item[DC:] Direct Current
\item[DCI:] Digital Cinema Initiatives
\item[DWT:] Discrete Wavelet Transform
\item[EBU:] European Broadcasting Union
\item[FIR:] Finite Impulse Response
\item[HD:] High Definition
\item[HH:] High-High (subband)
\item[HL:] High-Low (subband)
\item[HPF:] High Pass Filter
\item[IDWT:] Inverse Discrete Wavelet Transform
\item[ITU:] International Telecommunications Union
\item[LH:] Low-High
\item[LL:] Low-Low
\item[LPF:] Low Pass Filter
\item[NTSC:] National Television Systems Committee
\item[PAL:] Phase Alternating Line
\item[SD:] Standard Definition
\item[TV:] Television
\item[QCIF:] Quarter Common Image Format
\item[VC:] Video Codec
\end{description}

\subsection{Terms}
\begin{description}
\item[AC (sub)Band:] any signal band that is not the DC sub-band.
\item[Access Unit:] a unit of the Dirac bit-stream that provides points at which the stream may be randomly accessed.
\item[Codec:] a truncation of the terms "coder" and "decoder".
\item[DC (sub)Band:] the signal band that represents data composed from the lowest frequency band of a wavelet transform (0-LL).
\item[Discrete Wavelet Transform (DWT):] a means of transforming an array of values into space-frequency components through the use of a filter bank. 
(Note: see http://en.wikipedia.org/wiki/Discrete\_wavelet\_transform for a fuller description).
\item[Entropy Coding:] a term for describing any mathematical process used to encode data in a lossless manner, intended to reduce the required bit rate.
\item[Exp-Golomb:] a form of variable-length code. This specification uses an interleaved variant.
\item[Intra-Prediction:] the sample prediction of coefficient data within the dc sub-band.
\item[Inverse Discrete Wavelet Transform (IDWT):] the inverse of the DWT that converts an array of space- frequency components back into an array of values.
\item[Inverse Quantisation:] a process whereby each sample of a sub-band has its signal range expanded by a defined value.
\item[Lifting:] the name given to reducing a DWT filtering operations into a number of elementary
filters, each operating on half the samples.
	(Note: see Bibliography item "Ripples in Mathematics", chapter 3, for more information. It is also instructive to view the Wikipedia definition at http://en.wikipedia.org/wiki/Lifting\_scheme).
\item[Low Delay:] a term used to define the mode of the Dirac codec that can be used to compress video with a delay of less than one frame duration.
\item[Parse Info header:] identifies the beginning of major Dirac syntax elements (sequence start, picture, sequence end, padding and auxiliary data) with defined parse code values.
\item[Parsing:] a process by which numerical and text strings within binary data are recognised and used to provide syntactic meaning.
\item[Picture:] a single frame or field of video or a still image.
\item[Quantisation:] a process whereby each sample of a sub-band has its signal range compressed by a defined value. 
\item[Quantiser:] The defined value used for the purposes of quantisation or inverse quantisation.
\item[Raster scan:] any 2-dimensional array of samples, whether as video samples or as wavelet transformed values, that is scanned in accordance with television systems; namely left to right, then top to bottom.
\item[Sequence:] the data contained in a Dirac sequence corresponds to a single video sequence with constant video parameters as defined in xxx. A Dirac sequence is preceded by a `Parse Info' header that indicates the beginning of the sequence with a unique parse code. A Dirac sequence can be extracted from a Dirac bit-stream and decoded entirely independently.
\item[Slice:] a component part of the low-delay syntax that provides for compression of small parts (slices) of a picture in order to reduce delay. It is similar in meaning to the term of the same name in MPEG-2 coding.
\item[State Machine:] a defined behaviour model for data that is composed of a finite number of states, transitions between those states, and actions. A state stores information about the past, i.e. it reflects the input changes from the system start to the present moment.
(Note: see http://en.wikipedia.org/wiki/State\_machine for a fuller description).
\item[Stream:] a concatenation of Dirac sequences
\item[Subband:] the signal band that represents data composed from a single space-frequency band of a wavelet transform.
\end{description}

\begin{comment}
\subsection{Naming Conventions}
Syntax names and textual names are widely used in this document. A syntax name is expressed as a single text string in a monotype font (such as Courier) with any word spacing using the underscore character. A textual name is expressed with capitalised words and a regular text space between words. Thus VC-2\_syntax is a syntax name used in the syntax definition and VC-2 Syntax is the same entity expressed as a textual name used in the text body.
\end{comment}

\clearpage
\section{Conventions}%%%%%%%%%%%%%%%%%%%%%%%%%%%%%%%%%%%%%%%%%%%%%%%%%%
% - This chapter defines specification         - %
% - conventions                                - % 
%%%%%%%%%%%%%%%%%%%%%%%%%%%%%%%%%%%%%%%%%%%%%%%%%%
\label{spec-conventions}
\subsection{State representation}

This standard uses a state model to express parsing and decoding operations. 
The state of the decoder/parser shall be stored in the variable state. Individual elements of the decoder $\StateName$ (state variables) may be accessed
 by means of 
named labels, e.g. $\StateName[\text{VAR\_NAME}]$ (i.e. state is a map, as defined in Section \ref{datatypes}). 

The decoder state shall be globally accessible within the decoder. Other 
variables, not declared as inputs to a process, shall be local to that process.
The parsing and decoding operations are defined in terms of modifying the 
decoder state. The state variables need not directly correspond to elements 
of the stream, but may be calculated from them taking into account the decoder 
state as a whole. For example, a state variable value may be differentially 
encoded with respect to another value, with the difference, not the variable 
itself, encoded in the stream. 

\begin{comment}
The Dirac stream syntax structure is illustrated with informative parse diagrams,
contained in Annex \ref{parsediagrams},
 that complement the normative stream syntax definitions.
\end{comment}
 The parsing process 
is defined by means of pseudocode and/or mathematical formulae. The 
conventions for these elements are described in the following sections.

\subsection{Number formats}
\label{mathnotation}

Numbers without a prefix shall be interpreted as decimal numbers.

The prefix b indicates that the following value shall be interpreted as a binary
natural number (non-negative integer). 

{\bf Example} The value b1110100 is equal to the decimal value 116. 

The prefix 0x shall indicate that the following value is to be interpreted as a hexadecimal (base 16) natural number. 

{\bf Example} The value 0x7A is equal to the decimal value 122. 

\subsection{Data types}
\label{datatypes}

\subsubsection{Elementary data types}

Three basic types are used in the pseudo code:
\begin{description}
\item[Boolean] - A Boolean variable that has only two possible values: $\true$ and $\false$.
\item[Integer] - A positive or negative whole number or zero.
\item[Label] - a unique immutable value used in control structures and to 
access maps (see below).
\end{description}

\subsubsection{Compound data types}

Elementary and compound data types may be aggregated into a single 
compound data type.
There are three compound data type:

\begin{description}
\item[Set]
 A collection of data types. A set is indicated by enclosing the elements within 
curly braces, for example $\{a,b,c\}$ represents a set containing the values $a,b$
 and $c$. An empty set may be indicated by $\{\}$. The usual set-theoretic 
operations such as: $\cup$ (union), $\cap$ (intersection), $\in$ (membership) 
apply to sets and the other compound data types. 

\item[Map] A set of data types whose elements are accessed by their 
corresponding label. For example, $p[Y] ,p[C1] ,p[C2] $ might be the values 
of the different video components of a pixel. The set of  labels corresponding to 
the elements of a map $m$ can be accessed by $\args(m)$, so that, for example, $args(p)$ returns$\{Y,C1,C2\}$.
\item[Array]  A collection of data types accessed by a non-negative integer index.
 This compound data type is typically used to represent an array of variables. Elements of a 1-dimensional array $a$ are accessed by $a[n]$ for $n$
 in the range 0 to $\length(a) - 1$. 
\end{description}

A compound data type may contain other compound data types. For example, 
a two dimensional array is an array of one dimensional arrays. Elements of a 2-dimensional array are accessed by $a[n][m]$ for $0\leq m\leq(\width(a)-1)$, 
and $0\leq n\leq (\height(a)-1)$. Compound data types may be more complex. 
For example, picture data, pic, may be considered to be a map of arrays, where $pic[Y]$ is a 2-dimensional array storing luma data, and $pic[C1]$ and $pic[C2]$
 are two-dimensional arrays storing chroma data.

Elements may be added to a map or array by assignment using the appropriate
 index (label or integer). For example, $a[7]=2$, adds element 7 to the array $a$,
 if a does not already contain element 7, then this element is assigned the value 2.

\subsection{Functions and operators}
\label{functionoperators}

This section defines the functions and operators used 
in the pseudocode in this specification. Functions and operators
are similar but functions use the syntax, $(arg1, arg2,\ldots)$ 
whereas operators are simply placed before or between operands, 
e.g. $a+b$. The difference is purely syntactic and is to 
correspond with conventional mathematical notation.

\subsubsection{Assignment}

The assignment operation  = applies to all variable types. After performing 
\[a=b\]
the value of $a$ shall become equal to that of $b$, and the value of $b$ shall remain unchanged. For a set (or map or array) this constitutes an element-wise copy
 i.e.
\[a[x]=b[x]\]
for all valid values of $x$.

\subsubsection{Boolean functions and operators}
\label{booleanops}
The following functions and operators are defined for one or more Boolean arguments:
\begin{description}
\item[not] 		(not a) or returns $\true$ for a boolean value $a$ if and only if 
$a$ is $\false$
\item[and] 		(a and b) returns $\true$ if and only if a and b are both $\true$. Operator "and" may be used in pseudo-code conditions to denote the logical AND between Boolean values, for example: if (condition1 and condition2): �etc.
\item[or] 		(a or b) returns True if either a or b are True, else it returns False.  Operator "or" may be used in pseudo-code conditions to denote the logical OR between Boolean values, for example: if (condition1 or condition2): � etc.
\item[majority]		Given a set, $S =\{s_0,�, s_{n-1}\}$ of Boolean values, $\majority(S)$ returns the majority condition. That is, if the number of $\true$ 
values is greater than or equal to the number of $\false$ values, $majority(S)$
 returns $\true$, otherwise it returns $\false$.
\end{description}

Boolean operations are to be distinguished from bitwise operations which operate on non-negative 
integer values, and are defined in Section \ref{integerops}.

\subsubsection{Integer functions and operators}
\label{integerops}
The following functions and operators are defined on integer values:

\begin{description}
\item[Absolute value] $|a|=\left\lbrace\begin{array}{l} a \text{ if $a\geq 0$}\\ 
                                                                                   -a \text{ otherwise} \end{array}\right.$.

\item[Sign] $\sign(a)$ is defined by
\[\sign(a)=
\left\{\begin{array}{l} 
1 \text{ if $a>0$} \\
0 \text{ if $a==0$}\\
-1 \text{ if $a<0$} 
\end{array}\right.\]

\item[Addition] The sum of $a$ and $b$ is represented by $a+b$.

\item[Subtraction] $a$ minus $b$ is represented by $a-b$.

\item[Multiplication] $a$ times $b$ is represented, for clarity, by $a*b$.

\item[Integer division] Integer division is defined for integer values $a,b$ with 
$b>0$ where: $n=a//b$ is defined to be the largest integer $n$ such that
\[n*b\leq a\] 

i.e. numbers are rounded towards -infinity. N.B. this differs from C/C++ conventions
of round towards 0.

\item[Remainder] For integers $a,b$, with $b>0$, the remainder $a\%b$ is 
equal to 

\[a\%b = a-(a//b)*b \]

 $a\%b$ always lies between 0 and $b-1$.

\item[Exponentiation] For integers $a, b$, $b>0$ $a^b$ is defined as $a*a*\hdots *a$ ($b$ times). $a^0$ is 1.

\item[Maximum] $\max(a,b)$ returns the largest of $a$ and $b$.

\item[Minimum] $\min(a,b)$ returns the smallest of values $a$ and $b$.

\item[Clip] $\clip(a,b,t)$ clips the value $a$ to the range defined by $b$ (bottom)
and $t$ (top):
\[\clip(a,b,t)=\min(\max(a,b),t)\]

\item[Shift down] For integers $a,b$, with $b\geq 0$, $a\gg b$ is defined as 
$a//2^b$.

\item[Shift up] For integers $a,b$, with $b\geq 0$, $a\ll b$ is defined as $a*2^b$.

\item[Integer logarithm] $m=\intlog2(n)$, for $n>0$, $m$ is the integer such that
$2^{m-1}<n\leq 2^m$.


\item[Mean] Given a set  $S=\{s_0, s_1, \hdots, s_{n-1}\}$ of integer values, the integer unbiased mean, $\mean(S)$, is defined
to be

\[(s_0+s_1+\hdots +s_{n-1}+(n//2))//n\]

\item[Median] Given a set $S=\{s_0, s_1, \hdots, s_{n-1} \}$ of integer values the median, $\median(S)$, 
returns the middle value. If $t_0\leq t_1\leq \hdots \leq t_{n-1}$ are the values $s_i$ placed in ascending order, this
is 

$t_{(n-1)/2}$ 

if $n$ is odd and

$\mean(\{ t_{(n-2)/2},t_{n/2}\})$ if $n$ is even. If $S=\emptyset$, $\median(S)$ returns 0.
\end{description}
The following bitwise operations are defined on non-negative integer values:
\begin{description}
\item[\&] Logical AND is applied between the corresponding bits in the binary representation of two numbers, e.g.
$13\&6$ is $\text{b1101}\&\text{b110}$, which equals $\text{b100}$, or 4.

\item[${\mathbf |}$] Logical OR is applied between the corresponding bits in the binary representation of two numbers, e.g.
$13|6$ is $\text{b1101}\text{|}\text{b110}$, which equals $\text{b1111}$, or 15.

\item[${\mathbf \wedge}$] Logical XOR is applied between the corresponding bits in the binary representation of two numbers, e.g.
$13\wedge 6$ is $\text{b1101}\wedge\text{b110}$, which equals $\text{b1011}$, or 11.

\item[$\mathbf{\&=}$]	 $a \&= b$ is equivalent to $a = (a \& b)$.
\item[$\mathbf{|=}$]	 $a |= b$ is equivalent to $a = (a | b)$.
\item[$\mathbf{\wedge=}$]	$a\wedge^= b$ is equivalent to $a = (a \wedge b)$.
\end{description}

Bitwise-not is not defined for integers to avoid ambiguity concerning leading 
zeroes

The following logical operators are defined for integer and boolean arguments:
\begin{description}
\item[==] Test of equality of two variables. $a == b$ is $\true$ if and only if the 
value of a equals the value of b.
\item[!=] Not equal to. $a != b$ is equivalent to not $(a == b)$
\end{description}

The following logical operators are defined for integer arguments only:
\begin{description}
\item[$\mathbf{<}$] 	Less than
\item[$\mathbf{<=}$]	Less than or equal to
\item[$\mathbf{>}$]	Greater than
\item[$\mathbf{>=}$]	Greater than or equal to

\end{description}

The following combined assignment operators are defined for integer 
arguments:
Operators $+, -, *, //, \%, \gg, \ll, \&, |, \wedge$, may be combined with the assignment operator (as for the Boolean operators $\&$, $|$, and $\wedge$
 above). For example $a += b$ is equivalent to $a = (a + b)$.

\subsubsection{Array and map functions and operators}

The following functions and operators are defined for sets, maps and arrays. 
\begin{description}
\item[Indexing]		For an array $a$, $a[index]$ returns an element of $a$. If $a$ is a map the index shall be a label, else if $a$ is an array the index shall be an integer.
\item[Scalar Assignment]	Where the notation $a = 0$ is used for an array of integer values, it means "set all elements of the array to zero".
\item[Insertion]		$a[index] = b$ inserts a copy of $b$ into set $a$ 
if the element does not already exist.
\item[Tokens]		for a map $a$, $\args(a)$ returns the set of the indexing tokens.
\item[Length]		for a one dimensional array $a$, $\length(a)$ 
returns the number of elements in the array.
\item[Width]		for a two dimensional array $a$, $\width(a)$ returns the 
width the array. The width is the number of scalar elements corresponding to the right most array index.
\item[Height]					for a two dimensional array $a$, $\height(a)$ returns 
the height the array. The height is the number of one dimensional arrays in the two dimensional array and the "height" dimension corresponds to the left most array index.
\end{description}
\subsubsection{Precedence and associativity of operators}
\label{operatorprecedence}
To avoid any confusion over the order of operator precedence, every equation makes extensive use of the expression operators "(" and ")". All operations recursively execute the innermost expression(s) first until the calculation has been completed. In cases where the expression operators do not make clear the order of precedence, the following table defines the descending order of operator precedence and the associativity of each operator.
[Table tbc]
\begin{comment}
Operator Precedence	Associativity
( ) [ ]	left to right
* // %	left to right
+ -	left to right
<< >>	left to right
< <= > >=	left to right
== !=	left to right
! (not)	right to left
& (and)	left to right
^ (xor)	left to right
|	left to right
= += -= *= //= %= &= ^= |= <<= >>=	right to left
\end{comment}

\subsection{Pseudocode}
\label{pseudocode}

Most of the normative specification is defined by means of pseudocode. 
The syntax is intended to be both precise and descriptive; the pseudocode is 
not intended to form the basis for the implementation of a Dirac decoder.

All processing defined by this standard is precise and the entire specification
can be implemented using only the data types, functions and operators 
defined herein. That is, no operations on "real" or "floating point" numbers
 are required. All operations shall be implemented with sufficiently large 
integers so that overflow cannot occur.

The type of variables in the pseudocode is not explicitly declared. 
A variable assumes a type when it is assigned a value, which shall always 
have a defined type.

\subsubsection{Processes and functions}
\label{functionsprocesses}

Decoding and parsing operations are specified by means of processes
 -- a series of operations acting on input data and global variable data. 
A process can also be a function, which means it returns a value, but
it need not do so. So a process
taking in variables $in1$ and $in2$ looks like:

\begin{pseudo}{foo}{in1, in2}
\bsCODE{op1(in1)}
\bsCODE{op2(in2)}
\bsCODE{\hdots}
\end{pseudo}

Whilst a function process looks like:

\begin{pseudo}{bar}{in1, in2}
\bsCODE{op1(in1)}
\bsCODE{foo(in1,in2)}{\ref{functionsprocesses}}
\bsCODE{\hdots}
\bsRET{out1}
\end{pseudo}

The right-hand column in the pseudocode representation contains a cross-reference to the 
section in the specification containing the definition of other processes used at that line.

\subsubsection{Variables}

All input variables are deemed to be passed {\em by reference} in this
specification. This means that any modification to a variable value that
occurs within a process also applies to that variable within the calling process
{\em even if it has a different name} in the calling process. One way to understand
this is to envisage variable names as pointers to workspace memory.

For example, if we define $foo$ and $bar$ by

\begin{pseudo}{foo}{}
\bsCODE{num=0}
\bsCODE{bar(num)}
\bsCODE{\StateName[var\_name]=num}
\end{pseudo}

and 

\begin{pseudo}{bar}{val}
\bsCODE{val=val+1}
\end{pseudo}

then at the end of $foo$, $\StateName[var\_name]$ has been set to 1.

The only global variables are the state variables encapsulated in $\StateName$.
 If a variable is not declared as an input to
the process and is not a state variable, then it is local to the function.

If a process is particularly complex, it may be broken into a number of steps with 
intermediate discussion. This is signalled by appending  and prepending ``$\hdots$" to
the parts of the pseudocode specification:

\begin{pseudo}{foo}{}
\bsCODE{code}
\bsCODE{\hdots}
\end{pseudo}

[text]

\begin{pseudo*}
\bsCODE{more code}
\bsCODE{\hdots}
\end{pseudo*}

[text]

\begin{pseudo*}
\bsCODE{even more code}
\end{pseudo*}

The intervening text may define or modify variables used in the succeeding
pseudocode, and must be considered as a normative part of the specification of the process.
This is done as it is sometimes much more clear to split up a long and complicated process
into a number of steps.

\subsubsection{Control flow}
\label{controlflow}

The pseudocode comprises a series of statements, linked by functions and
flow control statements such as {\bf if}, {\bf while}, and {\bf for}.

The statements do not have a termination character, unlike the ; in C
for example.  Blocks of statements are indicated by indentation:
indenting in begins a block, indenting out ends one.

Statements that expect a block (and hence a following indentation) end
in a colon.

\paragraph*{if}

The if control evaluates a boolean or boolean function, and if true, passes the 
flow to the block of following statement or block of statements. If the control
evaluates as false, then there is an option to include one or more else if
controls which offer alternative responses if some other condition is
true.  If none of the preceding controls evaluate to true, then there is
the option to include an else control which catches remaining cases.

\begin{pseudo*}
\bsIF{control1}
    \bsCODE{block1}
\bsELSEIF{control2}
    \bsCODE{block2}
\bsELSEIF{control3}
    \bsCODE{block3}
\bsELSE
    \bsCODE{block4}
\bsEND
\end{pseudo*}

The if and else if conditions are evaluated in the order in which they
are presented. In particular, if $control1$ or $control2$ is true in
the preceding example, $block3$ will not be executed
even if $control3$ is true; neither will $block4$.

\paragraph*{for}

The for control repeats a loop over an integer range of values. For example,

\begin{pseudo*}
\bsFOR{i=0}{n-1}
    \bsCODE{foo(i)}
\bsEND
\end{pseudo*}

calls $foo()$ with value $i$, as $i$ steps through from 0 to $n-1$ inclusive.


\paragraph*{for each} The for each control loops over the elements in
a list:

\begin{pseudo*}
\bsFOREACH{c}{Y,C1,C2}
    \bsCODE{block}
\bsEND
\end{pseudo*}

\paragraph*{for such that} The for such that control loops over elements in
a set which satisfy some condition:

\begin{pseudo*}
\bsFORSUCH{a\in A}{control}
    \bsCODE{block}
\bsEND
\end{pseudo*}

This may only be used when the order in which elements are processed is 
immaterial.

\paragraph*{while}

The while control repeats a loop so long as a switch variable is true. 
When it is false, the loop breaks to the next statement(s) outside the block.

\begin{pseudo*}
\bsWHILE{condition}
    \bsCODE{block1}
\bsEND
\bsCODE{block2}
\end{pseudo*}


\clearpage
\section{Overall specification}

\clearpage
\section{Video interface}This section defines the video formats supported by the Dirac codec.

A selection of widely used video formats are defined in normative Appendix 
\ref{videoformatdefaults}. These video formats are characterized by
their widespread use in television, cinema and multimedia applications.

This list is not exhaustive, however, and Dirac is a general purpose video 
codec. These predefined formats are base formats that may be modified element by
element to support a much larger range of possible video formats. Support
is provided by the sequence parameters of the bitstream (Section 
\ref{sequenceheader}) for signalling both the base video format and
any modifications for complete characterization of the video format metadata.


\subsection{Colour model}

Dirac supports any video format that codes the raw image colors in a luma 
(grey-level) component with two associated chroma (color difference) components.
 These components are referred to as $Y$, $C1$ and $C2$.

In ITU defined systems (including ITU-BT.709, ITU-R BT.1361 and ITU-BT.1700), 
the $Y$, $C1$ and $C2$ values shall relate to the $E’_Y$, $E’_U$ and $E’_V$ 
video components respectively. These video components are also widely referred 
to as $Y, U, V$ and $Y, C_B , C_R$.

In the ITU-T H.264 reversible color transform, the $Y$, $C1$ and $C2$ values 
shall correspond to the video components $Y, C_O, C_G$.

\begin{informative}
Coding using $Y, C_O, C_G$ provides a simple reversible conversion to and from
RGB components by using lossless integer transforms. The use of $Y, C_O, C_G$
supports lossless coding of RGB video and allows Dirac to be treated as an RGB
codec for applications that require this feature.
\end{informative}

\subsection{Interlace}

Dirac supports both interlace and progressive formats. Interlace formats may be
 either top field first or bottom field first.

Dirac codes pictures where a picture may be a frame or a field. Fields consist 
of sets of alternate lines of video frames (even and odd lines). A pair of 
fields constituting a frame may correspond to distinct time intervals (true
interlace scanning) or to the same time interval (progressive segmented frames).
 Hence the configuration of frame/field coding is independent of whether the 
video format is interlaced or progressive.

\subsection{Component sampling}

Chroma components $C1$ and $C2$ may be coded with the same dimensions as the Y
 component (4:4:4) sampling, or with half-width (4:2:2) or half-dimension 
(4:2:0) sampling.

$Y$, $C1$ and $C2$ picture components shall be sampled at the same temporal
instant.

\begin{informative}
All pictures are considered as individual entities whether or not all lines were
 sampled at the same instant. In video sequences that are not frame-based, such 
as 30fps interlaced video carrying 24fps progressive images in a 3:2 
pull-down sequence, the compression performance may not be optimum. In such 
cases, a pre-processor may provide the codec with a more easily compressed 
source such as the original 24fps source pictures. Such pre-processing does not form any part of this specification.
\end{informative}

\subsection{Bit resolution}

The bit depth of each component sample is, in principle, unrestricted. 
Application-specific codecs may restrict the supported bit depth to a single 
value or a limited range of values.

Video is represented internally within the decoder specification as a bipolar
 (signed integer) signal. Video is presented at the video interface as an 
unsigned integer value by addition of an offset to these values 
(Section \ref{videooutput}). Metadata concerning black level and white level 
is transmitted
 within the data stream (Section \ref{signalrange}), but is not enforced at the
decoder video interface: output video may undershoot or overshoot these values.

\subsection{Picture frame size and rate}

The frame size and frame rate is, in principle, unrestricted. 
Application-specific codecs may restrict the supported frame size and frame rate
 to a single value or a limited range of values, and compliance to a given level
 implies constraints on the values as specified in Appendix \ref{profilelevel}.



\clearpage
\section{Stream syntax specification}%%%%%%%%%%%%%%%%%%%%%%%%%%%%%%%%%%%%%%%%%%%%%%%%%%%%%
% - This chapter defines the bytestream structure - %
%%%%%%%%%%%%%%%%%%%%%%%%%%%%%%%%%%%%%%%%%%%%%%%%%%%%%

\label{streamstructure}

This section specifies the overall structure of Dirac streams. 
Subsequent sections define the processes for 
parsing pictures, and Section \ref{picturedec} specifies how pictures 
are decoded.

\subsection{Pseudocode}
The parsing process is normatively defined using pseudocode and/or mathematical formulae.
 The definitions of stream syntax operations and pseudocode shall be as defined in 
section \ref{spec-conventions}. 

The Dirac stream syntax uses a state model to express the stream in a way that can 
be parsed and used for decoding operations. The parsing and decoding operations are 
specified in terms of modifying the decoder state according to the data extracted 
from the Dirac stream. The state of the decoder is stored in the global variable $\StateName$.
 This is a map (Section \ref{datatypes}) and individual elements are accessed by 
means of named labels, e.g. $\StateName[\text{VAR\_NAME}]$. 
The state variables comprise the parameters that shall be used in parsing and decoding a picture. 
The variable state is a global variables and shall be accessible to all decoder functions and processes. 
All other variables shall be local to the function or process in which they are defined.

Decoder state variables (that is elements of state) may not directly correspond 
to parts of the stream, but may be calculated from them taking into account the 
decoder state as a whole. For example, a state variable value may be differentially 
encoded with respect to another value, with the difference, not the variable itself, 
encoded in the stream. Some parameters are encoded in the stream as indices
 to tables of values. The indices are coded as variable length integers. 
This allows the tables to be extended to contain new entries, in future versions 
of this specification, without changing the syntax.


\subsection{Stream}
\label{stream}

A stream is a concatenation of Dirac sequences. The process for parsing a stream 
is to parse all sequences it contains. A Dirac sequence shall be decoded as a separate entity.

\subsection{Sequence}

The data contained in a Dirac sequence corresponds to a single video sequence with
constant video parameters as defined in Sections \ref{sourceparameters}. A sequence
 A Dirac sequence can be excised from a Dirac stream and decoded entirely
independently.

A Dirac sequence shall comprise an alternating sequence of parse info headers and 
data units. The first data unit shall be a sequence header, and further sequence 
headers may be inserted at any data unit point in the sequence . 
The process for parsing a Dirac sequence shall be as defined below:

\begin{pseudo}{video\_sequence}{}
\bsCODE{\RefBuffer=\{\}}
\bsCODE{\DecodedBuffer=\{\}}
\bsCODE{parse\_info()}{\ref{parseinfo}}
\bsCODE{\VideoParams=sequence\_header()}{\ref{sequenceheader}}
\bsCODE{parse\_info()}{\ref{parseinfo}}
\bsWHILE {is\_end\_of\_sequence()==\false}{\ref{parseinfo}}
    \bsIF{is\_seq\_header()==\true}{\ref{parseinfo}}
        \bsCODE{\VideoParams=sequence\_header()}{\ref{sequenceheader}}
    \bsELSEIF{is\_picture()}{\ref{parseinfo}}
        \bsCODE{picture\_parse()}{\ref{pictureparse}}
    \bsELSEIF{is\_auxiliary\_data()}{\ref{parseinfo}}
        \bsCODE{auxiliary\_data()}{\ref{auxdata}}
    \bsELSEIF{is\_padding()}{\ref{parseinfo}}
        \bsCODE{padding()}{\ref{paddingdata}}
    \bsEND
    \bsCODE{parse\_info()}{\ref{parseinfo}}
\bsEND
\end{pseudo}

Each Dirac sequence shall start and end with a parse info header. 

\subsection{Parse Info headers}

Parse info headers shall contain a 32 bit code so that the decoder can 
be synchronized with the stream. They are defined in section \ref{parseinfo}.
The parse info headers support navigating through the stream without the 
need to decode any data units. Each parse info header contains pointers 
to the location of the next and previous parse info headers within the stream. 
The stream may thus be thought of as a doubly linked list of data units.
Each parse info headers contains a code that identifies the type of data held 
in the following data unit. This is the only information contained within
the parse info headers that is needed to decode the sequence.

\subsection{Data units}

Data units may be one of:
\begin{itemize}
\item a sequence header, 
\item a picture, 
\item auxiliary data 
\item padding data. 
\end{itemize}

A sequence header shall contain metadata describing the coded sequence 
and metadata needed to decode the stream. The sequence header is defined 
in Section \ref{sequenceheader}. The first data unit in a sequence shall be a 
sequence header. To support reverse-parsing applications, the last data unit in 
a sequence should also be a sequence header.

Each sequence shall contain at least one picture and at least one sequence header.
The first picture after each sequence header (if there is one) shall be an intra picture.

If a sequence contains more than one sequence header, the data 
in each sequence header shall be the same (byte-for-byte identical) within the sequence.

Each picture, whether a frame or field, may be coded with a dependency 
on prior pictures in the stream (reference pictures).

A picture data unit shall contain sufficient data to decode a single picture 
(frame or field of video), subject to having parsed a sequence header within 
the sequence and decoded any reference pictures.

Pictures within a sequence shall either all be fields or all be frames. Where pictures 
are fields, a sequence shall contain an even number of pictures, comprising a whole number of frames.

Auxiliary data and padding data do not contribute to the decoding 
process and so may be discarded. 


Auxiliary and padding data units comprise undefined data for the purposes
 of this standard. These data units (together with the correct preceding 
parse info header) may be interposed at any point in the stream, but 
may safely be skipped by a compliant decoder. For the purposes of 
subsequent parts of this standard, the potential presence of auxiliary 
and padding data shall be ignored.

Padding data units shall not be used for any form of auxiliary data 
service or content. They may be used by an encoder, where required, 
to insert additional data to assist in complying with constant or constrained bit rate requirements.

\subsubsection{Auxiliary data}\label{auxdata}

The $auxiliary\_data()$ process for reading auxiliary data shall be as follows:

\begin{pseudo}{auxiliary\_data}{}
\bsCODE{byte\_align()}
\bsFOR{i=1}{\NextParseOffset-13}
    \bsCODE{read\_byte}
\bsEND
\end{pseudo}

\subsubsection{Padding data}\label{paddingdata}

The $padding()$ process for reading padding data shall be as follows:

\begin{pseudo}{padding}{}
\bsCODE{byte\_align()}
\bsFOR{i=1}{\NextParseOffset-13}
    \bsCODE{read\_byte}
\bsEND
\end{pseudo}

\subsection{Parse Info header syntax}
\label{parseinfo}

This section specifies the operation of the $parse\_info()$ process for parsing
parse info header data. This header shall consists of 13 bytes of data, byte-aligned 
within the sequence. Thus it shall ensure that succeeding data elements
are  byte aligned. It shall occur:
\begin{itemize}
\item at the beginning of a sequence
\item at the end of a sequence
\item before each data unit
\end{itemize}

The value of the parse code, which is a component of the parse info, shall be used to determine the type and format of the subsequent data structures, in particular indicating whether a picture is Intra or Inter coded, and if Inter how many references it has.

\begin{pseudo}{parse\_info}{}
\bsCODE{byte\_align()}
\bsCODE{\ParseInfoPrefix=read\_uint\_lit(4)}
\bsCODE{\ParseCode=read\_byte()}
\bsCODE{\NextParseOffset=read\_uint\_lit(4)}
\bsCODE{\PrevParseOffset=read\_uint\_lit(4)}
\end{pseudo}

The Parse Info parameters shall satisfy the following constraints:

\begin{itemize}
\item $\ParseInfoPrefix$ shall be set to be 0x42 0x42 0x43 0x44, which is the character string ``BBCD'' as expressed by ISO/IEC 646.
\item $\ParseCode$ shall be one of the supported values set out in Table \ref{parsecodes}
\item $\NextParseOffset$ shall be the number of bytes from the first byte of the current
Parse Info header to the first byte of the next Parse Info header, if there is one. If there
is no subsequent Parse Info header, it shall be be zero.
\item $\PrevParseOffset$ shall be the number of bytes from the first byte of the current
Parse Info header to the first byte of the previous Parse Info header, if there is one. If there
is no subsequent Parse Info header, it shall be be zero.
\end{itemize}

Consequently, the previous parse offset value of the current parse info header shall equal the next parse offset value of the previous parse info header, if there is one.

Parse codes shall be divided into three sets: generic, core syntax and low delay
syntax.

The value of parse codes allowed within the Dirac sequence shall be as shown in Table \ref{parsecodes}

\begin{table}[!ht]
\centering
\begin{tabular}{|c|c|l|c|}
\hline
\rowcolor[gray]{0.75}
\ParseCode &  {\bf Bits} & {\bf Description} & \begin{tabular}{c} {\bf Number of}\\ {\bf Reference}\\{\bf Pictures}\end{tabular}\\
\hline
\multicolumn{4}{|c|}{\cellcolor[gray]{0.85}\bf Generic}\\
\hline
0x00 & 0000 0000 & Sequence header &--\\
\hline
0x10 & 0001 0000 & End of Sequence & -- \\
\hline
0x20 & 0010 0000 & Auxiliary data & -- \\
\hline
0x30 & 0011 0000 & Padding data & -- \\
\hline
\multicolumn{4}{|c|}{\cellcolor[gray]{0.85}\bf Core syntax}\\
\hline
0x0C & 0000 1100 & Intra Reference Picture (arithmetic coding) & 0\\
\hline
0x08 & 0000 1000 & Intra Non Reference Picture (arithmetic coding) & 0\\
\hline
0x4C & 0100 1100 & Intra Reference Picture (no arithmetic coding) & 0\\
\hline
0x48 & 0100 1000 & Intra Non Reference Picture (no arithmetic coding) & 0\\
\hline
0x0D & 0000 1101 & Inter Reference Picture (arithmetic coding) & 1\\
%\hline
%0x4D & 0100 1101 & Inter Reference Picture (no arithmetic coding) & 1\\
\hline
0x0E & 0000 1110 & Inter Reference Picture (arithmetic coding) & 2\\
%\hline
%0x4E & 0100 1110 & Inter Reference Picture (no arithmetic coding) & 2\\
\hline
0x09 & 0000 1001 & Inter Non Reference Picture (arithmetic coding)& 1\\
%\hline
%0x49 & 0100 1001 & Inter Non Reference Picture (no arithmetic coding) & 1\\
\hline
0x0A & 0000 1010 & Inter Non Reference Picture (arithmetic coding) & 2\\
%\hline
%0x4A & 0100 1010 & Inter Non Reference Picture (no arithmetic coding)& 2\\
\hline
\multicolumn{4}{|c|}{\cellcolor[gray]{0.85}\bf Low-delay syntax}\\
\hline
0xCC & 1100 1100 & Intra Reference Picture & 0\\
\hline
0xC8 & 1100 1000 & Intra Non Reference Picture & 0\\
\hline
\end{tabular}
\caption{Parse codes}\label{parsecodes}
\end{table}

Future versions of this specification may introduce new parse codes. In order that 
decoders complying with this version of the specification may decode future 
versions of the coded stream, the decoder shall discard data units that immediately 
follow parse info blocks containing unknown parse codes. 

The parse codes shall be associated with a group of functions, listed below, which shall determine the type of subsequent data and the parsing and decoding processes which 
shall be used. All functions shall return a
boolean, except for $num\_refs()$ which shall returns an integer:

\begin{pseudo}{is\_seq\_header}{}
\bsRET{\ParseCode==\text{0x00}}
\end{pseudo}

\begin{pseudo}{is\_end\_of\_sequence}{}
\bsRET{\ParseCode==\text{0x10}}
\end{pseudo}

\begin{pseudo}{is\_auxiliary\_data}{}
\bsRET{(\ParseCode \& \text{0xF8})==\text{0x20}}
\end{pseudo}

\begin{pseudo}{is\_padding}{}
\bsRET{(\ParseCode \& \text{0xF8})==\text{0x30}}
\end{pseudo}

\begin{pseudo}{is\_picture}{}
\bsRET{((\ParseCode \&\text{0x08})==\text{0x08})}
\end{pseudo}

\begin{pseudo}{is\_low\_delay}{}
\bsRET{((\ParseCode \&\text{0x88})==\text{0x88})}
\end{pseudo}

\begin{pseudo}{is\_core\_syntax}{}
\bsRET{((\ParseCode \&\text{0x88})==\text{0x08})}
\end{pseudo}

\begin{pseudo}{using\_ac}{}
\bsRET{((\ParseCode \&\text{0x48})==\text{0x08})}
\end{pseudo}

\begin{pseudo}{is\_reference}{}
\bsRET{((\ParseCode \&\text{0x0C})==\text{0x0C})}
\end{pseudo}

\begin{pseudo}{is\_non\_reference}{}
\bsRET{((\ParseCode \&\text{0x0C})==\text{0x08})}
\end{pseudo}

\begin{pseudo}{num\_refs}{}
\bsRET{(\ParseCode \&\text{0x03})}
\end{pseudo}

\begin{pseudo}{is\_intra}{}
\bsRET{is\_picture() \text{ and } (num\_refs()==0)}
\end{pseudo}

\begin{pseudo}{is\_inter}{}
\bsRET{is\_picture() \text{ and }(num\_refs()>0)}
\end{pseudo}



\begin{informative*}
\subsubsection{Parse code value rationale (Informative)}

The rationale for the parse code values in Table \ref{parsecodes} is as follows:
\begin{itemize}
\item The MS bit (bit 7) is used to indicate the picture syntax (core or low delay syntax) 
and only applies to pictures. Core syntax codes whole frames rather than slices. Low delay syntax codes slices not frames. 
\item The second MS bit (bit 6) is used to indicate whether arithmetic coding is used and only applies to pictures. Core syntax may optionally use arithmetic coding. Low delay syntax does not use arithmetic coding. The permutation of the two bits which might indicate low delay syntax with arithmetic coding is reserved. Arithmetic coding is not supported on
Inter pictures.
\item The next three MS bits (bits 5, 4 and 3) indicate the type of data unit following the parse info unit. Bit 3 indicates whether it is a picture or non-picture data unit. Bits 5 and 4 indicate the 4 other parse codes.
\item The three LS bits (bits 2, 1 and 0) indicate picture types. Bit 2 indicates whether a picture
is a reference picture or not. Bits 0 and 1 indicate the number of references a picture has
for motion compensation purposes: if these are both 0, the picture is an Intra picture.
\end{itemize}

\end{informative*}

\section{Sequence header}
\label{sequenceheader}

This section defines the structure of the sequence header syntax. 
The sequence header shall be byte aligned. 
Parsing this header consists of reading the sequence parameters 
(parse parameters, base video format, source parameters and 
picture coding mode) and initializing the decoder parameters. The decoder 
parameters are initialized in the $coding_parameters()$ process (below).

The sequence header shall remain byte identical throughout a sequence.

The process for parsing the sequence header shall be as follows:

\begin{pseudo}{sequence\_header}{}
\bsCODE{byte\_align()}
\bsCODE{parse\_parameters()}{\ref{parseparameters}}
\bsITEM{base\_video\_format}{uint}{\ref{videoformat}}
\bsCODE{\VideoParams=source\_parameters(base\_video\_format)}{\ref{sourceparameters}}
\bsITEM{picture\_coding\_mode}{uint}{\ref{interlacedcoding}}
\bsCODE{set\_coding\_parameters(\VideoParams,picture\_coding\_mode)}{\ref{interlacecoding}}
\bsRET{\VideoParams}
\end{pseudo}

Parse parameters contain information a decoder may use to determine whether 
it is able to parse or decode the stream. Parse parameters are not used to decode the stream.

The base video format is a numerical index denoting a default set of parameters 
that describe the video source. For many common video formats the predefined 
values indicated by the base video format and defined in Appendix \ref{videoformatdefaults}, will be sufficient 
without the need for further metadata to be present in the stream. However, to provide 
flexibility, source parameters may override the parameters indicated 
by the base video format (with the exception of the top field first flag).

Source parameters are parameters that describe the source video, not all of 
which are required to decode the stream. The source parameters are needed by 
applications that use the decoded video and so should be made available to them.

Field coding indicates whether the video has been coded as a sequence of frames or fields.

Once the base video format, source parameters and frame/field coding mode have 
been read from the stream the information they contain may be decoded to 
provide the parameters used for decoding pictures. It is the purpose of the 
$coding\_parameters()$ process to initialize these parameters.

\begin{informative}
Note that video parameters indicate whether the video sequence is interlaced or progressive.
In particular a change from interlaced to progressive video, or vice-versa, necessitates that
the Dirac sequence be terminated and a new sequence begun. The coding mode indicates whether
the pictures within a Dirac sequence are fields or frames. Note that progressive video may
still be encoded as fields, to provide backward compatibility with pseudo-progressive frame (PSF)
video transmission.

The video parameters are not used by the Dirac decoder. Video parameter values should
be made available using appropriate interfaces and standards to any downstream video
processing device or display, but their use and interpretation by other devices is not specified in this standard. 
Neverthless, Appendix \ref{vidsys} specifies the video systems model that should be used for the interpretation
of video parameters.
\end{informative}

%%%%%%%%%%%%%%%%%%%%%%%%%%
\subsection{Parse parameters}
\label{parseparameters}

This section specifes the structure of the parse parameters, which is as follows:

\begin{pseudo}{parse\_parameters}{}
\bsITEM{\VersionMajor}{uint}{}
\bsITEM{\VersionMinor}{uint}{}
\bsITEM{\Profile}{uint}{}
\bsITEM{\Level}{uint}{}
\end{pseudo}

Parse parameter data shall be constant
(byte-for-byte identical) for all instances of the sequence header within a Dirac sequence.

\subsubsection{Version number}

The major version number shall define the version of the syntax with which the stream complies. A decoder complies with a major version number if it can parse all bit streams that comply with that version number. Decoders that comply with a major version of the specification may not be able to parse the bit stream corresponding to a later specification.

Depending on the profile and level defined, a decoder compliant with a given major version number may still not be able to decode fully all parts of a stream.

All minor versions of the specification shall be functionally compatible with earlier minor versions with the same major version number. Later minor versions may contain corrections, clarifications, and removal of ambiguities. Later minor version numbers shall not contain new features or new normative provisions.

The major version number of a stream compliant with this version of the Dirac standard shall be 1.

The minor version number of a stream compliant with this version of the Dirac standard shall be 0.


\subsubsection{Profiles and levels}

A profile shall define the toolset that is sufficient to decode a sequence. 

A level shall determine decoder resources (picture and data buffers; computational resources) sufficient
to decode a sequence, including the sizes $\RefBufferSize$ and $\DPBSize$ of 
the reference picture and decoded picture buffers. 

Applicable values of profile and level and the variables they set are specified in Appendix
\ref{profilelevel}.

\subsection{Base video format}
\label{videoformat}

The value of $base\_video\_format$ decoded in parsing the sequence header shall be 
an index into Table \ref{table:videoformats}. For each entry in the table parameters are defined, in 
Appendix \ref{videoformatdefaults}, indicating parameters corresponding to one of a set of 
predefined formats. These base parameters may be modified by subsequent metadata present
 in the stream as defined in subsequent sections below, with the exception of the top field first
 parameter which shall only be set by the base video format (see Section \ref{scanformat}).

\begin{table}[!ht]
\centering
\begin{tabular}{|c|c|}
\hline
\rowcolor[gray]{0.75}Video format index	& Video format description \\
\hline
0	& Custom Format\\
\hline
1	&	QSIF525\\
\hline
2	&	QCIF\\
\hline
3	&	SIF525\\
\hline
4	&	CIF\\
\hline
5	&	4SIF525\\
\hline
6	&	4CIF\\
\hline
7	&	SD 480I-60 (525 Line 60 Field/s Standard Definition)\\
\hline
8	&	SD 576I-50 (625 Line 50 Field/s Standard Definition)\\
\hline
9	&	HD 720P-60 (720 Line 60 Frame/s High Definition)\\
\hline
10 &	HD 720P-50 (720 Line 50 Frame/s High Definition)\\
\hline
11 &	HD 1080I-60 (1080 Line 60 Field/s High Definition)\\
\hline
12	&	HD 1080I-50 (1080 Line 50 Field/s High Definition)\\
\hline
13	&	HD 1080P-60 (1080 Line 60 Frame/s High Definition)\\
\hline
14	&	HD 1080P50 (1080 Line 50 Frame/s High Definition)\\
\hline
15	&	DC 2K-24 (2K D-Cinema, 24fps)\\
\hline
16	&	DC 4K-24 (2K D-Cinema, 24fps)\\
\hline
\end{tabular}
\caption{Dirac predefined video formats}
\label{table:videoformats}
\end{table}

Custom format is intended for use when no other suitable base video format is available from
 the table. Video format defaults will still be set as per Appendix \ref{videoformatdefaults},
 but these are token values which are expected to be almost wholly overridden by the subsequent source parameters. 

\subsection{Source parameters}
\label{sourceparameters}

The source parameters are intended to indicate the format of the video that was 
originally encoded. They provide metadata that indicates how the decoded video should be displayed. 

The source parameters shall comprise frame size, chroma sampling format, scan format, frame rate,
 pixel aspect ratio, clean area, signal range and color specification. The frame size, chroma sampling
 format, scanning format and the signal range are required to decode the video. Display and 
downstream processing falls outside the scope of this specification, hence the interpretation of the other parameters (not required to decode the video) is not normatively defined, with the exception of frame rate (Section \ref{framerate}). The frame rate imposes requirements on
 compliant decoders for a given level and profile (Appendix \ref{profilelevel}).

Source parameter data shall remain constant throughout a VC-2 sequence. 

Default values for the source parameters shall be derived from the video format, as defined in annex C. These default values shall be the source parameters unless they are overridden with alternative values encoded as part of the Source Parameters part of the stream. 

The $source\_parameters()$ process shall return a structure defining the video source parameters. It shall be defined as follows:

\begin{pseudo}{source\_parameters}{base\_video\_format}
\bsCODE{\VideoParams = set\_source\_defaults(base\_video\_format)}{\ref{setsourcedefaults}}
\bsCODE{frame\_size(\VideoParams)}{\ref{framedimensions}}
\bsCODE{chroma\_sampling\_format(\VideoParams)}{\ref{chromaformat}}
\bsCODE{scan\_format(\VideoParams)}{\ref{scanformat}}
\bsCODE{frame\_rate(\VideoParams)}{\ref{framerate}}
\bsCODE{pixel\_aspect\_ratio(\VideoParams)}{\ref{aspectratio}}
\bsCODE{clean\_area(\VideoParams)}{\ref{cleanarea}}
\bsCODE{signal\_range(\VideoParams)}{\ref{signalrange}}
\bsCODE{colour\_spec(\VideoParams)}{\ref{colourspec}}
\bsRET{\VideoParams}
\end{pseudo}

Note: Although some source parameters are not used by the Dirac decoder all source parameters should, nevertheless, be made available any downstream video processing device or display to allow the proper interpretation of the decoded video. Their use and interpretation by other devices is not specified in this standard. 

\subsubsection{Setting source defaults}
\label{setsourcedefaults}

The function that sets the default values of the source video parameters 
shall take the video format index as an argument. That is, the signature of this function is: $set\_source\_defaults(video\_format\_index)$ where $video\_format\_index$ is an unsigned integer. The function returns a map of source video parameters.

The source video parameters shall be set, based on the video format index, as defined in 
Appendix \ref{videoformatdefaults}. The parameters set by this function shall be: Frame Size, 
Sampling Format (4:4:4, 4:2:2 or 4:2:0), Scan Format (Progressive or Interlace), Frame Rate, Pixel Aspect Ratio, Clean Area, Signal Range, Color Specification. The tokens used to access the map returned by the function shall be as defined in the subsequent sections that specify how to override the default video source parameters.

Note: as an example, if $video\_format\_index  == 4$, CIF defaults are set, with picture size equal to 252x288, 4:2:0 chroma format, amongst other parameters. A frame width of 360 pixels may be encoded by overriding the CIF format default value of 352. 

\subsubsection{Frame dimensions}
\label{framedimensions}

If a flag is set, the image dimensions specified by the video format defaults may
be overridden:

\begin{pseudo}{frame\_size}{\VideoParams}
\bsITEM{custom\_dimensions\_flag}{bool}{}
\bsIF{custom\_dimensions\_flag==\true}
    \bsITEM{\SFrameWidth}{uint}{}
    \bsITEM{\SFrameHeight}{uint}{}
\bsEND
\end{pseudo}

\subsubsection{Chroma sampling formats}
\label{chromaformat}

If a flag is set, the chroma format specified by the video format defaults is overridden.

\begin{pseudo}{chroma\_sampling\_format}{\VideoParams}
\bsITEM{custom\_chroma\_format\_flag}{bool}{}
\bsIF{custom\_chroma\_format\_flag==\true}
     \bsITEM{\SChromaFormatIndex}{uint}{}
\bsEND
\end{pseudo}

The supported chroma sampling formats are specified in Table \ref{tab:chromaformats}:

\begin{table}[!ht]
\centering
\begin{tabular}{|c|c|}
\hline
\rowcolor[gray]{0.75}\SChromaFormatIndex & {\bf Chroma format} \\
\hline
0 & 4:4:4 \\
\hline
1 & 4:2:2 \\
\hline
2 & 4:2:0 \\
\hline
\end{tabular}
\caption{Supported chroma sampling formats}\label{tab:chromaformats}
\end{table}

The chroma sampling format shall be used to determine the width and height of the 
chroma components of the coded video as described in Section \ref{picturedimensions} below.

\subsubsection{Scan format}
\label{scanformat}

The scan format parameter shall indicate whether the source video represents progressive 
frames or interlaced fields.

The process for parsing the scan format parameters shall be as follows:

\begin{pseudo}{scan\_format}{\VideoParams}
\bsITEM{custom\_scan\_format\_flag}{bool}{}
\bsIF{custom\_scan\_format\_flag==\true}
    \bsITEM{\SSourceSampling}{uint}{}
\bsEND
\end{pseudo}

If the custom scan format flag is set to $\true$, the source sampling parameter defined by 
the base video format values shall be overridden by the new value.

If $\SSourceSampling$ is set to 0, then the source video shall be progressively sampled. 
If it is 1, then the source video shall be interlaced. Values greater than 1 shall be reserved.

If the source video is interlaced then $\STopFieldFirst$ shall be $\true$ 
if the top line of the frame is in the earlier field, else $\STopFieldFirst$
 shall be $\false$. This shall be set only by the base video format and cannot be overridden 
in the source parameters.

Both interlaced and progressive video may be coded as fields or frames.

\subsubsection{Frame rate}
\label{framerate}

The frame rate value (in frames per second) shall be 
$\SFrameRateNumer$ divided by $\SFrameRateDenom$

The process for decoding the frame rate parameters shall be as follows:

\begin{pseudo}{frame\_rate}{\VideoParams}
\bsITEM{frame\_rate\_flag}{bool}{}
\bsIF{frame\_rate\_flag==\true}
    \bsITEM{index}{uint}{}
    \bsIF{index == 0}
        \bsITEM{\SFrameRateNumer}{uint}{}
        \bsITEM{\SFrameRateDenom}{uint}{}
    \bsELSE
        \bsCODE{preset\_frame\_rate(\VideoParams,index)}
     \bsEND
\bsEND
\end{pseudo}

The decoded value of $index$ shall fall in the range 0 to 10.

For $index>0$, $preset\_frame\_rate(\VideoParams,index)$ shall set frame rate
elements of $\VideoParams$ 
as specified in Table \ref{table:frameratevalues}.

\begin{table}[!ht]
\centering
\begin{tabular}{|c|c|c|}
\hline
\rowcolor[gray]{0.75}$index$ & Numerator & Denominator \\
\hline
1 & 24000 & 1001 \\
\hline
2 & 24 & 1 \\
\hline
3 & 25 & 1 \\
\hline
4 & 30000 & 1001 \\
\hline
5 & 30 & 1 \\
\hline
6 & 50 & 1 \\
\hline
7 & 60000 & 1001 \\
\hline
8 & 60 & 1 \\
\hline
9 & 15000 & 1001 \\
\hline
10 & 25 & 2 \\
\hline
\end{tabular}
\caption{Available preset frame rate values}\label{table:frameratevalues}
\end{table}

Note that what is encoded is frame rate, not picture rate. If the video is coded
as fields, then picture rate is twice the encoded frame rate.

\subsubsection{Pixel aspect ratio}
\label{aspectratio}

The process for decoding the pixel aspect ratio parameters shall be defined as follows:

\begin{pseudo}{pixel\_aspect\_ratio}{\VideoParams}
\bsITEM{custom\_pixel\_aspect\_ratio\_flag}{bool}{}
\bsIF{custom\_pixel\_aspect\_ratio\_flag==\true}
    \bsITEM{index}{uint}{}
    \bsIF{index == 0}
        \bsITEM{\SAspectRatioNumer}{uint}{}
        \bsITEM{\SAspectRatioDenom}{uint}{}
    \bsELSE
        \bsCODE{preset\_pixel\_aspect\_ratio(\VideoParams,index)}
    \bsEND
\bsEND
\end{pseudo}

The decoded value of $index$ shall fall in the range 0 to 3.

For $index>0$, $preset\_pixel\_aspect\_ratio(\VideoParams, index)$ shall set the pixel aspect ratio value elements of $\VideoParams$ as specified in Table \ref{table:aspectratiovalues}.

\begin{table}[!ht]
\centering
\begin{tabular}{|c|c|c|}
\hline
\rowcolor[gray]{0.75}$index$ & Numerator & Denominator \\
\hline
1 (Square Pixels) & 1 & 1 \\
\hline
2 (525-line systems) & 10 & 11 \\
\hline
3 (625-line systems) & 12 & 11 \\
\hline
4 (16:9 525-line systems) & 40 & 33 \\
\hline
5 (16:9 625-line systems) & 16 & 11 \\
\hline
6 (reduced horizontal resolution) & 4 & 3 \\
\hline
\end{tabular}
\caption{Available preset pixel aspect ratio values}\label{table:aspectratiovalues}
\end{table}

\begin{informative}
The pixel aspect ratio is defined as the ratio of the aspect ratio parameters:
		\[\SAspectRatioNumer : \SAspectRatioDenom\]
The pixel aspect ratio (PAR) value defines the intended pixel aspect ratio of the pixels such that the viewed picture has no geometric distortion. The pixel aspect ratio of an image is the PARs are fundamental properties of sampled images because they determine the displayed shape of objects in the image. Failure to use the right PAR will result in distorted images, for example circles will be displayed as ellipses etc. 
Some video processing tools require an image aspect ratio. This may be derived from the pixel aspect ratio by multiplying the ratio of horizontal to vertical pixels by the pixel aspect ratio. So, for example, for a 704 x 480 line picture, with a pixel aspect ratio of 10:11 the image aspect ratio is (704 x 10)/(480 x 11) which is exactly 4:3.
\end{informative}

\subsubsection{Clean area}
\label{cleanarea}

The process for decoding the clean area parameters shall be as follows:

\begin{pseudo}{clean\_area}{\VideoParams}
\bsITEM{custom\_clean\_area\_flag}{bool}{}
\bsIF{custom\_clean\_area\_flag==\true}
    \bsITEM{\SCleanWidth}{uint}{}
    \bsITEM{\SCleanHeight}{uint}{}
    \bsITEM{\SLeftOffset}{uint}{}
    \bsITEM{\STopOffset}{uint}{}
\bsEND
\end{pseudo}

The following restrictions shall apply:

\begin{itemize}
\item $\SCleanWidth+\SLeftOffset
\leq \SFrameWidth$
\item $\SCleanHeight+\STopOffset
\leq \SFrameHeight$
\end{itemize}

\subsubsection{Signal range}
\label{signalrange}

The signal range parameters indicate how the signal range of the picture component data, decoded by the Dirac decoder, should be adjusted prior to the color matrixing operations (described in informative Appendix \ref{signalranges}). 

The signal range parameters shall also be used to determine the luma depth and chroma depth parameters (Section \ref{videodepth}) and the resulting clipping levels applied to the decoded video (Section \ref{pictureclip}).

Note that picture component data processed by the Dirac decoder is bipolar, and
is adjusted to positive ranges when pictures are output.

The process for decoding the signal range parameters is as follows:

\begin{pseudo}{signal\_range}{\VideoParams}
\bsITEM{custom\_signal\_range\_flag}{bool}{}
\bsIF{custom\_signal\_range\_flag==\true}
    \bsITEM{index}{uint}{}
    \bsIF{index == 0}
        \bsITEM{\SLumaOffset}{uint}{}
        \bsITEM{\SLumaExcursion}{uint}{}
        \bsITEM{\SChromaOffset}{uint}{}
        \bsITEM{\SChromaExcursion}{uint}{}
    \bsELSE
        \bsCODE{preset\_signal\_ranges(\VideoParams,index)}
    \bsEND
\bsEND
\end{pseudo}

The decoded value of $index$ shall fall in the range 0 to 3.

If $index>0$ the $preset\_signal\_ranges(\VideoParams,index)$ shall set the signal range elements of $\VideoParams$ as specified in Tables \ref{table:signalrangevalues}.

\begin{table}[!ht]
\centering
\begin{tabular}{|c|c|c|c|c|}
\hline
\rowcolor[gray]{0.75}$index$ & Luma offset & Luma excursion & Chroma offset & Chroma excursion\\
\hline
1 (8 Bit Full Range) & 0 & 255 & 128 & 255\\
\hline
2 (8 Bit Video) & 16 & 219 & 128 & 224\\
\hline
3 (10 Bit Video) & 64 & 876 & 512 &  896\\
\hline
4 (12 Bit Video) & 256 & 3504 & 2048 & 3584\\
\hline
\end{tabular}
\caption{Available signal range presets}\label{table:signalrangevalues}
\end{table}

Note that decoded video is represented within the decoder specification as bi-polar 
signals. An offset is added when video is output so that it is represented by unsigned 
integer values.

\subsubsection{Colour specification}
\label{colourspec}

The color specification shall consist of three component parts:
\begin{itemize}
\item Color primaries
\item Color matrix 
\item Transfer function
\end{itemize}

Defaults are available for all three parts collectively and individually.

The process for decoding the color specification parameters shall be follows: 

\begin{pseudo}{colour\_spec}{\VideoParams}
\bsITEM{custom\_colour\_spec\_flag}{bool}{}
\bsIF{custom\_colour\_spec\_flag==\true}
    \bsITEM{index}{uint}{}
    \bsCODE{preset\_colour\_specs(\VideoParams,index)}
    \bsIF{index == 0}
        \bsCODE{colour\_primaries(\VideoParams)}{\ref{colourprimaries}}
        \bsCODE{colour\_matrix(\VideoParams)}{\ref{colourmatrix}}
        \bsCODE{transfer\_function(\VideoParams)}{\ref{transferfunction}}
    \bsEND
\bsEND
\end{pseudo}

The decoded value of $index$ shall fall in the range 0 to 4.

$preset\_colour\_spec(index)$ shall set the colour primaries, matrix and transfer function elements of $\VideoParams$ as specified 
in Table \ref{table:colourspecvalues}. If the value of $index$ is 0, these values may be overridden as defined in the succeeding sections. 

\begin{table}[!ht]
\centering
\begin{tabular}{|c|c|c|c|c|}
\hline
\rowcolor[gray]{0.75}$index$ & {\bf Description}           & {\bf Primaries}       & {\bf Matrix}  & {\bf Transfer function}\\
\hline
0 & Custom & HDTV & HDTV & TV gamma \\ 
\hline
1 & SDTV 525 & SDTV 525 & SDTV & TV gamma \\
\hline
2 & SDTV 625 & SDTV 625 & SDTV & TV gamma \\
\hline
3 & HDTV & HDTV & HDTV & TV gamma \\
\hline
4 & Cinema & CIE XYZ & HDTV & DCI Gamma\\
\hline
\end{tabular}
\caption{Colour specification presets}\label{table:colourspecvalues}
\end{table}

\paragraph{Colour primaries}
\label{colourprimaries}

The color primaries decoding process shall be defined as follows:

\begin{pseudo}{colour\_primaries}{\VideoParams}
\bsITEM{custom\_colour\_primaries\_flag}{bool}{}
\bsIF{custom\_colour\_primaries\_flag==\true}
    \bsITEM{index}{uint}{}
    \bsCODE{preset\_colour\_primaries(\VideoParams,index)}
\bsEND
\end{pseudo}

The decoded value of $index$ shall fall in the range 0 to 3.

 $preset\_colour\_primaries(\VideoParams,index)$ shall set the colour primaries 
element of $\VideoParams$ as specified
in Table \ref{table:primariesvalues}.

\begin{table}[!ht]
\centering
\begin{tabular}{|c|c|c|c|}
\hline
\rowcolor[gray]{0.75}$index$ &  {\bf Description} & {\bf Specification} & {\bf Comment}      \\
\hline
0       &  HDTV & ITU-R BT.709 & Also Computer, Web, sRGB \\ 
\hline
1       &  SDTV 525 & ITU-R BT.601 & 525 primaries          \\
\hline
2       &  SDTV 625 & ITU-R BT.601 & 625 primaries  \\
\hline
3       &  Cinema & SMPTE 428.1 & CIE XYZ              \\
\hline
\end{tabular}
\caption{Colour primaries presets}\label{table:primariesvalues}
\end{table}

\paragraph{Colour matrix}
\label{colourmatrix}
$\ $\newline
The color matrix decoding process shall be defined as follows:

\begin{pseudo}{colour\_matrix}{}
\bsITEM{colour\_matrix\_flag}{bool}{}
\bsIF{colour\_matrix\_flag==\true}
    \bsITEM{index}{uint}{}
    \bsCODE{preset\_colour\_matrices(index)}
\bsEND
\end{pseudo}

The decoded value of $index$ shall fall in the range 0 to 2. 

The $preset\_colour\_matrices(\VideoParams,index)$ process shall set the colour 
matrix element in $\VideoParams$ as specified
in Table \ref{table:matrixvalues}.

\begin{table}[!ht]
\centering
\begin{tabular}{|c|c|c|c|c|}
\hline
\rowcolor[gray]{0.75}$index$ &  {\bf Description} & {\bf Specification} & {\bf Colour matrix} & {\bf Comment}\\
\hline
0 & HDTV & ITU-R BT.709 & $K_R=0.2126$, $K_B=0.0722$ & Also computer and web\\ 
\hline
1 & SDTV & ITU-R BT.601 & $K_R=0.299$, $K_B=0.114$ & \\
\hline
2 & Reversible & ITU-T H.264 & YCgCo & \\
\hline
\end{tabular}
\caption{Colour matrix presets}\label{table:matrixvalues}
\end{table}

\paragraph{Transfer function}
\label{transferfunction}
$\ $\newline
The transfer function decoding process shall be defined as follows:

\begin{pseudo}{transfer\_function}{\VideoParams}
\bsITEM{custom\_transfer\_function\_flag}{bool}{}
\bsIF{custom\_transfer\_function\_flag==\true}
    \bsITEM{index}{uint}{}
    \bsCODE{preset\_transfer\_function(\VideoParams,index)}
\bsEND
\end{pseudo}

$index$ shall fall in the range 0 to 3. The $preset\_transfer\_function(\VideoParams,index)$ process shall set the transfer function
element of $\VideoParams$ as specified
in Table \ref{table:transfervalues}.

\begin{table}[!ht]
\centering
\begin{tabular}{|c|c|c|}
\hline
\rowcolor[gray]{0.75}$index$ & {\bf Description} & {\bf Specification}\\
\hline
0 & TV gamm & ITU-R BT.1361\\ 
\hline
1 & Extended Gamut & ITU-R BT.1361 1998 Annex 1\\
\hline
2 & Linear & Linear\\
\hline
3 & DCI Gamma & SMPTE 428.1\\
\hline
\end{tabular}
\caption{Transfer function presets}\label{table:transfervalues}
\end{table}

\subsection{Picture coding mode}
\label{interlacecoding}

The picture coding mode value in the sequence header shall determine 
whether source video is coded as frames or fields. 

If the picture coding mode value is $1$ then pictures shall correspond to fields. If it is $0$ 
then pictures shall correspond to frames. Other picture coding mode values shall be
reserved for future extensions.

If video is coded as fields then the earliest field in each frame shall have an even picture number (Section \ref{pictureheader}). That is the LSB of the frame number, expressed as a binary number, indicates field parity.

With field coding each frame shall be split into two fields as indicated by the scan format (Section \ref{scanformat}).

An effect of field coding shall be to halve the vertical dimensions of coded pictures. Hence, once the frame/field coding mode is known, the picture dimensions, which shall be stored as part of the global state variable, shall be set (Section \ref{picturedimensions}).

\begin{informative}
It is possible to code progressive video as fields. In this case, the assignment of 
frame lines to fields will be determined by the value of the top field first parameter in the base 
video format (Appendix \ref{videoformatdefaults}). Note that, according to Section \ref{scanformat},
 this base format default cannot be overridden for progressive video, as to do so would be artificial.
\end{informative}

\subsection{Initializing coding parameters}
\label{codingparameters}

The $set\_coding\_parameters()$ process shall initialize the dimensions of the coded picture (frame or field), and the video depth (the maximum number of bits in a decoded video sample), which are needed to decode pictures. 

Picture dimensions and video depth shall remain constant throughout a Dirac sequence. 

Initialization of the coding parameters shall be as defined in the table below:

\begin{pseudo}{set\_coding\_parameters}{\VideoParams, picture\_coding\_mode}
\bsCODE{picture\_dimensions(\VideoParams, picture\_coding\_mode)}{\ref{picturedimensions}}
\bsCODE{video\_depth(\VideoParams)}{\ref{videodepth}}
\end{pseudo}

\subsubsection{Picture dimensions}
\label{picturedimensions}
The picture dimensions process, which determines the size of coded pictures, shall be defined as follows:

\begin{pseudo}{picture\_dimensions}{\VideoParams, picture\_coding\_mode}
\bsCODE{\LumaWidth = \SFrameWidth}
\bsCODE{\LumaHeight = \SFrameHeight}
\bsCODE{\ChromaWidth = \LumaWidth}
\bsCODE{\ChromaHeight = \LumaHeight}
\bsCODE{chroma\_format\_index = \SChromaFormatIndex]}
\bsIF{chroma\_format\_index == 1}
  \bsCODE{\ChromaWidth //= 2}
\bsELSEIF{chroma\_format\_index == 2}
  \bsCODE{\ChromaWidth //= 2}
  \bsCODE{\ChromaHeight //= 2}
\bsEND
\bsIF{picture\_coding\_mode==1}
  \bsCODE{\LumaHeight //=2}
  \bsCODE{\ChromaHeight //=2}
\bsEND
\end{pseudo}

The parameter $\SFrameHeight$  refers to the height of a frame. The parameter $\LumaHeight$ refers to the height of a picture.  A picture may be either a frame or a field depending on whether it is being coded in an interlaced or progressive mode.

Frame height shall be an integer multiple of picture chroma height.

For convenience, the following utility functions shall be defined:

\begin{pseudo}{chroma\_h\_ratio}{}
\bsRET{\LumaWidth//\ChromaWidth}
\end{pseudo}
 
\begin{pseudo}{chroma\_v\_ratio}{}
\bsRET{\LumaHeight//\ChromaHeight}
\end{pseudo}

\subsubsection{Video depth}
\label{videodepth}
The $video\_depth()$ process, which determines the maximum number of bits required to represent a sample of the decoded video, shall be defined as follows:

\begin{pseudo}{video\_depth}{\VideoParams}
\bsCODE{\LumaDepth =\intlog2(\SLumaExcursion+1)}
\bsCODE{\ChromaDepth =\intlog2(\SChromaExcursion+1)}
\end{pseudo}

Note that for YCoCg format the luma and chroma depths are different.

\section{Picture syntax}
\label{picturesyntax}
This section specifies the structure of Dirac picture data units.

\subsection{Picture parsing}
\label{picture}
\label{pictureparse}

This section specifies the operation of the $picture\_parse()$ process. The process for
decoding and outputting pictures is specified in Section \ref{picturedec}.

Picture data may be successfully parsed after parsing a sequence header within the same VC-2 sequence. The picture parsing process shall be defined as follows:

\begin{pseudo}{picture\_parse}{}
\bsCODE{byte\_align()}
\bsCODE{picture\_header()}{\ref{pictureheader}}
\bsIF{is\_inter()}{\ref{parseinfo}}
    \bsCODE{byte\_align()}
    \bsCODE{picture\_prediction()}{\ref{pictureprediction}}
\bsEND
\bsCODE{byte\_align()}
\bsCODE{wavelet\_transform()}{\ref{wavelettransform}}
\end{pseudo}

\subsubsection{Picture header}
\label{pictureheader}

The picture header shall immediately follow a parse info header with a picture parse 
code (Section \ref{parseinfo}). The picture header parsing process shall be defined as follows:

\begin{pseudo}{picture\_header}{}
\bsCODE{\PictureNumber=read\_byte\_lit(4)}
\bsIF{is\_inter()}{\ref{parseinfo}}
    \bsCODE{\RefOneNum=(\PictureNumber+read\_sint())\%2^{32}}
    \bsIF{num\_refs() == 2}{\ref{parseinfo}}
        \bsCODE{\RefTwoNum=(\PictureNumber+read\_sint())\%2^{32}}
    \bsEND\bsEND
\bsIF{is\_ref()}{\ref{parseinfo}}
    \bsCODE{\RetiredPicture=(\PictureNumber+read\_sint())\% 2^{32}}
\bsEND
\end{pseudo}

Picture numbers shall be unique within a sequence and the set of all picture numbers
within a sequence shall form a contiguous block of numbers.

Reference picture numbers shall encoded differentially with respect to the
picture number.

The pictures corresponding to the reference picture numbers of a given picture
shall occur before the given picture in the sequence.

The retired picture shall be a pictures which shall be removed from 
the reference picture buffer before the current picture is decoded
(Sections \ref{picturedecprocess}and \ref{refbuffer}). The rules for the 
use of the reference picture buffer shall be as defined in Section \ref{refbuffer}.

%%%%%%%%%%%%%%%%%%%%%%%%%%
\subsection{Picture prediction data}
\label{pictureprediction}

This section defines the picture prediction process that shall be used for decoding 
picture prediction parameters and motion vector fields for motion compensation.

The picture prediction process shall be defined as follows:

\begin{pseudo}{picture\_prediction}{}
\bsCODE{picture\_prediction\_parameters()}{\ref{picpredparams}}
\bsCODE{byte\_align()}
\bsCODE{block\_data()}{\ref{motiondec}}
\end{pseudo}

The decoding and generation of block motion vector fields shall be as defined in Section \ref{motiondec}. 

\subsubsection{Picture prediction parameters}
\label{picpredparams}

Picture prediction parameters consist of metadata required for successful parsing of the
motion data and for performing motion compensation (Section \ref{motioncompensate}).

The picture prediction parameters shall be defined as follows:

\begin{pseudo}{picture\_prediction\_parameters}{}
\bsCODE{block\_parameters()}{\ref{blockparameters}}
\bsCODE{motion\_vector\_precision()}{\ref{mvprecision}}
\bsCODE{global\_motion()}{\ref{globalmotion}}
\bsCODE{picture\_prediction\_mode()}{\ref{picpredmode}}
\bsCODE{reference\_picture\_weights()}{\ref{refpicweights}}
\end{pseudo}

\subsubsection{Block parameters}
\label{blockparameters}

This section specifies the operation of the process for
setting motion compensation block parameters, which shall consist of the state variables
$\LumaXBlen$, $\LumaYBlen$, $\LumaXBsep$, and $\LumaYBsep$
defining luma blocks, and $\ChromaXBlen$, $\ChromaYBlen$, $\ChromaXBsep$,
and $\ChromaYBsep$ defining chroma blocks. 

\begin{pseudo}{block\_parameters}{}
\bsITEM{index}{uint}
\bsIF{index == 0}
    \bsITEM{\LumaXBLen}{uint}{}
    \bsITEM{\LumaYBLen}{uint}{}
    \bsITEM{\LumaXBSep}{uint}{}
    \bsITEM{\LumaYBSep}{uint}{}
\bsELSE
       \bsCODE{preset\_block\_params(index)}
\bsEND
\bsCODE{chroma\_block\_params()}{\ref{chromablockparams}}
\end{pseudo}

$index$ shall lie in the range 0 to 4. 

The $preset\_block\_params(index)$ shall set the block parameters as specified
in Table \ref{blockparamsvalues}.

Chroma block parameter values shall be determined from luma values
as defined in Section \ref{chromablockparams}).

\begin{table}[!ht]
\centering
\begin{tabular}{|c|c|c|c|c|}
\hline
\rowcolor[gray]{0.75}$index$  & \LumaXBlen & \LumaYBlen & \LumaXBsep & \LumaYBsep \\
\hline
1 & 8 & 8 & 4 & 4 \\
\hline
2 & 12 & 12 & 8 & 8\\
\hline
3 & 16 & 16 & 12 & 12\\
\hline
4 & 24 & 24 & 16 & 16\\
\hline
\end{tabular}
\caption{Luma block parameter presets}\label{blockparamsvalues}
\end{table}

Block parameters shall satisfy the following constraints:

\begin{enumerate}
\item $\LumaXBlen$, $\LumaYBlen$, $\LumaXBsep$, and $\LumaYBsep$ shall all be positive
multiples of 4
\item $\LumaXBlen\geq\LumaXBsep$ and $\LumaYBlen\geq\LumaYBsep$
\item $\LumaXBlen\leq 2*\LumaXBsep$ and $\LumaYBlen\leq 2*\LumaYBsep$
\end{enumerate}

\begin{informative}
Note that these requirements do not preclude length from equalling separation, i.e.
motion compensation blocks that are not overlapped. 
\end{informative}

\subsubsection{Setting chroma block parameters}
\label{chromablockparams}

This section defines how chroma block parameters shall be derived from luma block dimensions. 

Chroma block parameters shall be equal to the corresponding luma block parameters scaled according to the chroma vertical and horizontal subsampling ratios. In this way chroma blocks and luma blocks are co-located in the video picture.

\begin{pseudo}{chroma\_block\_params}{}
\bsCODE{\ChromaXBlen=\LumaXBlen//chroma\_h\_ratio()}{\ref{picturedimensions}}
\bsCODE{\ChromaYBlen=\LumaYBlen//chroma\_v\_ratio()}{\ref{picturedimensions}}
\bsCODE{\ChromaXBsep=\LumaXBsep//chroma\_h\_ratio()}{\ref{picturedimensions}}
\bsCODE{\ChromaYBsep=\LumaYBsep//chroma\_v\_ratio()}{\ref{picturedimensions}}
\end{pseudo}

\subsubsection{Motion vector precision}
\label{mvprecision}

The motion vector precision process shall be as follows:

\begin{pseudo}{motion\_vector\_precision}{}
\bsITEM{\MotionVectorPrecision}{uint}
\end{pseudo}

$\MotionVectorPrecision$ shall lie in the range 0 (pixel-accurate) to 3 (1/8th-pixel accurate).

\subsubsection{Global motion}
\label{globalmotion}

Global motion parameters shall be encoded if the $\PictureUsingGlobal$ flag is set
to $\true$. Up to two sets shall be encoded,
depending upon the number of references.

The global motion process shall be as follows:

\begin{pseudo}{global\_motion}{}
\bsITEM{\PictureUsingGlobal}{bool}{}
\bsIF{\PictureUsingGlobal==\true}
    \bsCODE{global\_motion\_parameters(\GlobalParams[1])}
    \bsIF{num\_refs() == 2}
        \bsCODE{global\_motion\_parameters(\GlobalParams[2])}
    \bsEND
\bsEND
\end{pseudo}

Each of the global motion parameters shall consist of three elements: 

\begin{itemize}
\item an integer pan/tilt vector $\GlobalParams[n][pan\_tilt]$
\item an integer 2x2 matrix element $\GlobalParams[n][ZRS]$
capturing zoom, rotation and shear, together with a scaling exponent 
$\GlobalParams[n][ZRS\_exp]$
\item an integer perspective vector $\GlobalParams[n][pespective]$
capturing the effect of non-orthogonal projection onto the image plane, together 
with a scaling exponent $\GlobalParams[n][pespective\_exp]$
\end{itemize}

Their interpretation and the process for generating a global motion vector field 
shall be as defined in Section \ref{globalmv}. 

The global motion parameters process shall be defined as follows:

\begin{pseudo}{global\_motion\_parameters}{gparams}
\bsCODE{pan\_tilt(gparams)}
\bsCODE{zoom\_rotate\_shear(gparams)}
\bsCODE{perspective(gparams)}
\end{pseudo}

The $pan\_tilt()$ process shall extracts horizontal and vertical translation elements
and shall be defined as follows:

\begin{pseudo}{pan\_tilt}{gparams}
\bsCODE{gparams[pan\_tilt]={\mathbf{0}} }
\bsITEM{nonzero\_pan\_tilt\_flag}{bool}{}
\bsIF{nonzero\_pan\_tilt\_flag==\true}
    \bsITEM{gparams[pan\_tilt][0]}{sint}{}
    \bsITEM{gparams[pan\_tilt][1]}{sint}{}
\bsEND
\end{pseudo}

The $zoom\_rotate\_shear()$ process shall extract a linear matrix element and shall be
as defined as follows:

\begin{pseudo}{zoom\_rotation\_shear}{gparams}
\bsITEM{nontrivial\_zrs\_flag}{bool}{}
\bsIF{nontrivial\_zrs\_flag==\true}
    \bsITEM{gparams[ZRS\_exp]}{uint}{}
    \bsITEM{gparams[ZRS][0][0]}{sint}{}
    \bsITEM{gparams[ZRS][0][1]}{sint}{}
    \bsITEM{gparams[ZRS][1][0]}{sint}{}
    \bsITEM{gparams[ZRS][1][1]}{sint}{}
\bsELSE
    \bsCODE{gparams[ZRS\_exp]=0}
    \bsCODE{gparams[ZRS][0][0]=1}
    \bsCODE{gparams[ZRS][0][1]=0}
    \bsCODE{gparams[ZRS][1][0]=0}
    \bsCODE{gparams[ZRS][1][1]=1}
\bsEND
\end{pseudo}

The $perspective()$ process shall extract horizontal and vertical perspective
elements and shall be defined as follows:

\begin{pseudo}{perspective}{gparams}
\bsITEM{nonzero\_perspective\_flag}{bool}{}
\bsIF{nonzero\_perspective\_flag==\true}
    \bsITEM{gparams[perspective\_exp]}{uint}{}
    \bsITEM{gparams[perspective][0]}{sint}{}
    \bsITEM{gparams[perspective][1]}{sint}{}
\bsELSE
    \bsCODE{gparams[perspective\_exp]=0}
    \bsCODE{gparams[perspective]={\mathbf{0}} }
\bsEND
\end{pseudo}

\subsubsection{Picture prediction mode}
\label{picpredmode}

The picture prediction mode encodes alternative methods of motion compensation
and is present to support future extensions of this specification.

It shall be defined as follows:

\begin{pseudo}{picture\_prediction\_mode}{}
\bsITEM{\PicturePredictionModeIndex}{uint}
\end{pseudo}

In this specification, $\PicturePredictionModeIndex$ shall be 0.

\subsubsection{Reference picture weight values}
\label{refpicweights}

Reference picture weight values shall be determined as follows:

\begin{pseudo}{reference\_picture\_weights}{}
\bsCODE{\RefsWeightPrecision=1}
\bsCODE{\RefOneWeight=1}
\bsCODE{\RefTwoWeight=1}
\bsITEM{custom\_weights\_flag}{bool}
\bsIF{custom\_weights\_flag==\true}
    \bsITEM{\RefsWeightPrecision}{uint}
    \bsITEM{\RefOneWeight}{sint}
    \bsIF{num\_refs() == 2}
        \bsITEM{\RefTwoWeight}{sint}
    \bsEND
\bsEND
\end{pseudo}

\begin{informative}
For bi-directional prediction modes, reference 1 data will be weighted by 

$\dfrac{\RefOneWeight}{2^\RefsWeightPrecision}$

and reference 2 data by

$\dfrac{\RefTwoWeight}{2^\RefsWeightPrecision}$ 

(see Section \ref{pixelpredict}).

The picture weights are signed integers and may be negative. In
addition, they may not sum to $2^\RefsWeightPrecision$, to accomodate fade
prediction.
\end{informative}

%%%%%%%%%%%%%%%%%%%%%%%%%%
\subsection{Wavelet transform data}
\label{wavelettransform}

The wavelet transform syntax shall provide metadata determining the wavelet transform
 parameters (including filter type, transform depth, and codeblock or slice structures) together with the transformed wavelet coefficients. 

The wavelet transform process for parsing transform metadata and coefficients shall be defined as follows:

\begin{pseudo}{wavelet\_transform}{}
\bsCODE{\ZeroResidual = \false}
\bsIF{is\_inter()}{\ref{parseinfo}}
    \bsITEM{\ZeroResidual}{bool}
\bsEND
\bsIF{\ZeroResidual == \false}
    \bsCODE{transform\_parameters()}{\ref{transformparameters}}
    \bsCODE{byte\_align()}
    \bsCODE{transform\_data()}{\ref{wltunpacking}}
\bsEND
\end{pseudo}

Parsing (unpacking) the wavelet transform data shall be as defined in Section \ref{wltunpacking}. 

Decoding the transformed wavelet transform data to produce decoded pictures shall be
 as defined in Section \ref{picturedec}.

If $\ZeroResidual=\true$ then all component pixels shall be 
set to zero (Section \ref{picturedecprocess}).

\subsubsection{Transform parameters}
\label{transformparameters}

The wavelet transform parameters shall define the metadata required to configure the inverse wavelet transform for both the low delay and core syntax. 

The $transform\_parameters()$ process shall be defined as follows:

\begin{pseudo}{transform\_parameters}{}
\bsITEM{\WaveletIndex}{uint}{\ref{wltfilter}}
\bsITEM{\TransformDepth}{uint}{\ref{wltdepth}}
\bsIF{is\_low\_delay()==\false}
    \bsCODE{codeblock\_parameters()}{\ref{codeblockparams}}
\bsELSE
    \bsCODE{slice\_parameters()}{\ref{sliceparams}}
    \bsCODE{quant\_matrix()}{\ref{quantmatrix}}
\bsEND
\end{pseudo}

\paragraph{Wavelet filters}
\label{wltfilter}
$\ $\newline

The wavelet filter parameter shall define the wavelet filter used by the Dirac stream. T
The value of $\WaveletIndex$ shall lie in the range 0 to 6 with values as 
defined in Table \ref{wltfilterpresets}: 

\begin{table}[!ht]
\centering
\begin{tabular}{|c|c|}
\hline
\rowcolor[gray]{0.75}\WaveletIndex & {\bf Filter} \\
\hline
0 & Deslauriers-Debuc (9,7) \\
\hline
1 & LeGall (5,3) \\
\hline
2 & Deslauriers-Debuc (13,7) \\
\hline
3 & Haar with no shift \\
\hline
4 & Haar with single shift per level\\
\hline
5 & Fidelity filter \\
\hline
6 & Daubechies (9,7) integer approximation \\
\hline
\end{tabular}
\caption{Wavelet filter presets}\label{wltfilterpresets}
\end{table}


The implementation of the chosen wavelet filter shall be as defined 
 in Section \ref{wltfilters}. 


\begin{informative}
For consistency, the filter nomenclature $(m, n)$ refers to the length of the analysis low-pass
and high-pass filters in the conventional prefiltering (i.e. before subsampling) 
model of wavelet filtering. They do not reflect the length of lifting filters, which
operate in the subsampled domain: see Section \ref{wltfilters}. Deslauriers-Debuc
filters are normally referred to in terms of the number of vanishing moments of their
synthesis filters, so the (9,7) and (13,7) filters may be referred to in the literature
as (2,2) and (4,2) filters respectively.
\end{informative}

\subsubsection{Transform depth}
\label{wltdepth}

The transform depth parameter shall determine the number of stages in the wavelet transform.that the vertical and horizontal wavelet filters are applied. 

Note: The transform depth determines the number of 
subbands and the the dimensions of the subband data array (Section \ref{wltinit}).

\subsubsection{Codeblock parameters (core syntax only)}
\label{spatialpartition}

In the core syntax only, each subband may be partitioned into a number of code blocks. 

The process for extracting codeblock parameters shall be as follows:

\begin{pseudo}{codeblock\_parameters}{}
\bsCODE{\CodeblockMode=0}
\bsFOR{level=0}{\TransformDepth}
    \bsCODE{\CodeblocksX[level]=1}
    \bsCODE{\CodeblocksY[level]=1}
\bsEND
\bsITEM{spatial\_partition\_flag}{bool}
\bsIF{spatial\_partition\_flag==\true}
    \bsFOR{level=0}{\TransformDepth}
        \bsITEM{\CodeblocksX[level]}{uint}
        \bsITEM{\CodeblocksY[level]}{uint}
    \bsEND
    \bsITEM{\CodeblockMode}{uint}
\bsEND
\end{pseudo}

The presence of codeblocks in subbands shall be indicated by setting $spatial\_partition\_flag$ to $\true$; otherwise it shall be $\false$.

The number of codeblocks to be used for subbands at each transform depth level shall be encoded in $\CodeblocksY[level]$ and $\CodeblocksX[level]$ for vertical and horizontal axes respectively.

The codeblock mode is encoded in $\CodeblockMode$, which shall have value 0 or 1, with meanings as defined in Table \ref{codeblockmodes}. 

\begin{table}[!ht]
\centering
\begin{tabular}{|c|c|}
\hline
 \rowcolor[gray]{0.75}$\CodeblockMode$ & {\bf Description} \\
\hline
0 & Single quantiser per subband, used for all codeblocks\\
\hline
1 & Multiple Quantiser per subband, one for each codeblock \\
\hline
\end{tabular}
\caption{Codeblock modes}\label{codeblockmodes}
\end{table}

The operation of subband codeblock decoding shall be as defined in Section \ref{codeblocks}.

\subsubsection{Slice coding parameters (low delay syntax only)}
\label{sliceparams}

This slice parameters process shall be defined as follows:

\begin{pseudo}{slice\_parameters}{}
\bsITEM{\SlicesX}{uint}
\bsITEM{\SlicesY}{uint}
\bsITEM{\SliceBytesNum}{uint}
\bsITEM{\SliceBytesDenom}{uint}
\end{pseudo}

\subsubsection{Quantisation matrices (low-delay syntax)}
\label{quantmatrix}

The quantization matrix shall be used to modify the slice quantizer for each subband in
 a slice. The quantization matrix shall be encoded in the $\QuantMatrix$ decoder variable. 

The $quant\_matrix()$ process shall be defined as follows:

\begin{pseudo}{quant\_matrix}{}
\bsITEM{custom\_quant\_matrix}{bool}
\bsIF{custom\_quant\_matrix==\true}
    \bsITEM{\QuantMatrix[0][\LL]}{uint}
    \bsFOR{level=1}{\TransformDepth}
        \bsITEM{\QuantMatrix[level][\HL]}{uint}
        \bsITEM{\QuantMatrix[level][\LH]}{uint}
        \bsITEM{\QuantMatrix[level][\HH]}{uint}
    \bsEND
\bsELSE
    \bsCODE{set\_quant\_matrix()}
\bsEND
\end{pseudo}

If $\TransformDepth> 4$ then $custom\_quant\_matrix$  shall be $\true$. 

If $\TransformDepth \leq 4$, then custom quantization matrices may still be transmitted, 
for example to apply a different degree of perceptual weighting (see Appendix \ref{qmatrixdesign}). 

The function $set\_quant\_matrix()$ shall set the quantization matrix based on the wavelet ?lter as per Appendix \ref{defaultquantmatrices}. These are �unweighted� matrices, whose values merely compensate for the differential power gain of the different subband ?lters. For perceptual weighting a custom quantisation matrix must be used.

\clearpage
\section{Motion data unpacking}%%%%%%%%%%%%%%%%%%%%%%%%%%%%%%%%%%%%%%%%%%%%
% - This chapter defines how motion data - %
% - is decoded                           - % 
%%%%%%%%%%%%%%%%%%%%%%%%%%%%%%%%%%%%%%%%%%%%

\label{motiondec}

This section defines the operation of the $block\_motion\_data()$ process for extracting
block motion data from the Dirac stream. 

Block motion data is aggregated into {\em superblocks}, consisting of a 4x4 array of 
blocks. The number of superblocks horizontally and vertically shall be determined so 
that there are sufficient superblocks to cover the picture area. Superblocks 
may overlap the right and bottom edge of the picture.

\begin{informative}
\\
\begin{enumerate}
\item Since superblocks may overlap the right and bottom edge of the picture, blocks in 
such superblocks may also overlap the edges or even fall outside the picture area
 altogether. Motion data for blocks which fall outside the picture area is still decoded, but
 will not be used for motion compensation (Section \ref{motioncompensate}). 

\item Unlike macroblocks in MPEG standards, a superblock does not encapsulate all 
data within a given area of the picture. It is merely an aggregation device for motion data,
and for this reason a different nomenclature has been adopted.
\end{enumerate}
\end{informative}

\subsection{Prediction modes and splitting modes}

\subsubsection{Prediction modes}

Two types of prediction mode shall be defined: a reference prediction mode, indicating
which references are to be used for motion compensation,  and a global motion
 mode flag, indicating how prediction is to be performed (using global motion or block
 motion for a given block).

Four reference prediction modes shall be defined and shall be denoted by integer
constant values: 
\begin{enumerate}
\item \Intra~shall denote value 0, and shall indicate that DC values for a block
shall be decoded and that no motion vectors shall be decoded.
\item  \RefOneOnly~shall denote value 1 and shall indicate that a motion vector
for the first reference picture shall be decoded, but no motion vector for the second
reference picture shall be decoded.
\item  \RefTwoOnly~shall denote value 2 and shall indicate that a motion vector
for the second reference picture shall be decoded, but no motion vector for the first
reference picture shall be decoded.
\item  \RefOneAndTwo~shall denote value 3 and shall indicate that motion vectors
for both the first and second reference picture shall be decoded.
\end{enumerate}

In addition, where global motion is used for a picture (i.e.\ $\PictureUsingGlobal$ is set),
a global motion mode flag shall be encoded for each block. If $\true$, global motion
compensation shall be used for this block, and no block motion vectors or DC values
shall be encoded. If $\false$, block motion compensation shall be employed and
one or more motion vectors shall be encoded.

\subsubsection{Splitting modes}

Block motion data shall be aggregated into superblocks, consisting of a 4x4 array
of blocks, for each block motion data element. 

Three superblock splitting levels shall be defined, numbered 0, 1, and 2. 

When level 0 splitting is used, if a block motion data element is
present for that superblock, only one value shall be coded. This value shall be 
applied to all blocks within the superblock.

When level 1 splitting is used, at most 4 values shall be coded for each
block motion data element. Where present, each of these values
shall be applied to the blocks in within the corresponding 2x2 sub-array of blocks
within the superblock.

When level 2 splitting is used, if a block motion data element is
present for that superblock, only four values shall be coded. Each of these values
shall be applied to all the four blocks in one of the four 2x2 sub-arrays of blocks
within the superblock.

\subsection{Structure of block motion data arrays}

\label{motionconventions}

For the purposes of this specification, block motion data shall be stored in the 
two dimensional array $\BlockData$. Superblock
splitting modes shall be stored in the two dimensional array $\SBSplit$.

For each block with coordinates $(i,j)$, a block motion data element 
$\BlockData[j][i]$ shall be defined. It is a map (Section \ref{datatypes}) and shall
consist of up to five elements:

\begin{enumerate}
\item A motion vector for reference 1, $\BlockData[j][i][\Vect][1]$, consisting of 
integral horizontal and vertical elements $\BlockData[j][i][\Vect][1][0]$ and 
$\BlockData[j][i][\Vect][1][1]$.
\item A motion vector for reference 2, $\BlockData[j][i][\Vect][2]$, consisting of 
integral horizontal and vertical elements $\BlockData[j][i][\Vect][2][0]$ and 
$\BlockData[j][i][\Vect][2][1]$.
\item A set of integral DC values for each component, $\BlockData[j][i][\DC][Y]$,
 $\BlockData[j][i][\DC][C1]$, and $\BlockData[j][i][\DC][C2]$.
\item A reference prediction mode, $\BlockData[j][i][\RMode]$, taking values \Intra, 
\RefOneOnly, \RefTwoOnly, or \RefOneAndTwo~and indicating which references
(if any) are to be used for predicting block $(i,j)$.
\item A global motion mode flag, $\BlockData[j][i][\GMode]$.
\end{enumerate}

\subsubsection{Block motion data initialisation}

\label{motioninit}

This section specifies the operation of the $initialise\_motion\_data()$ process.
 It shall set the dimensions of the block motion parameter arrays according to the numbers
of blocks and superblocks defined in Section {motiondatadimensions}.

The array $\BlockData$ shall be set to have horizontal dimension $\BlocksX$ and 
vertical dimension $\BlocksY$.

The array $\SBSplit$ shall be set to have horizontal dimension $\SuperblocksX$ 
and vertical dimension $\SuperblocksY$.

\subsection{Motion data decoding process}
\label{decodingprocess}

This section defines the $block\_motion\_data()$ process for extracting
block motion data elements.

This process depends upon the picture prediction parameters (Section
 \ref{picpredparams}).

Block motion data elements shall be coded differentially with respect to a spatial
prediction.  The spatial prediction processes for the block motion elements are 
defined in Section \ref{spatialprediction}

The decoding process for the block motion data shall consist of: 
\begin{enumerate}
\item decoding the superblock split modes,
\item decoding the prediction modes in each superblock according to the split mode, and
\item decoding the motion vectors and DC values according to the split mode and the
 decoded mode for each block.
\end{enumerate}

The motion vector elements are further decomposed into horizontal and vertical 
components which are encoded as separate parts. The DC values are further
decomposed into the the components which are encoded as separate parts. 

The coded data for each part (splitting mode, prediction mode, vector component,
or DC component values) shall consist of an entropy coded block
 preceded by a length code. 

\begin{pseudo}{block\_motion\_data}{}
\bsCODE{initialise\_motion\_data()}{\ref{motioninit}}
\bsCODE{superblock\_split\_modes()}{\ref{superblocksplit}}
\bsCODE{prediction\_modes()}{\ref{blockpredmodes}}
\bsCODE{vector\_elements(1, 0)}{\ref{blockmvelements}}
\bsCODE{vector\_elements(1, 1)}{\ref{blockmvelements}}
\bsIF{num\_refs()==2}
    \bsCODE{vector\_elements(2,0)}{\ref{blockmvelements}}
    \bsCODE{vector\_elements(2,1)}{\ref{blockmvelements}}
\bsEND
\bsCODE{dc\_values(Y)}{\ref{DCvalues}}
\bsCODE{dc\_values(C1)}{\ref{DCvalues}}
\bsCODE{dc\_values(C2)}{\ref{DCvalues}}
\end{pseudo}

\begin{informative}
The superblock splitting modes determine the number, and location, of 
prediction mode values to be decoded -- there
must be one for each `prediction unit' (block, 2x2 array or blocks, or 4x4 array or
blocks) within a superblock. Together, the split 
mode and the prediction mode determine the number and location of all other 
motion data parts, which can each then be decoded in parallel. Indeed,
by attempting to decode the maximum possible number of prediction residue 
values for all motion data elements, the first 
two motion data elements may also be decoded in parallel with the others. 
Once all residue values are decoded, excess
values can be discarded, the location of values determined and actual 
values reconstructed by prediction. This approach
may be particularly valuable in hardware. Decoding may proceed in this way, 
as the arithmetic decoding engine allows bits to be read beyond the nominal end of an
 arithmetically-coded chunk by inserting 1s, hence allowing virtual values to be read.
\end{informative}

\subsubsection{Superblock splitting modes}
\label{superblocksplit}

This section defines the decoding of the superblock splitting mode values. 

The superblock splitting mode shall determine the number of prediction modes
coded for each superblock.

$superblock\_split\_modes()$ process shall be defined as follows:

\begin{pseudo}{superblock\_split\_modes}{}
\bsITEM{length}{uint}{}
\bsCODE{\ABitsLeft= 8*length}
\bsCODE{byte\_align()}
\bsCODE{ctx\_labels=[\SBSplitFollowOne,\SBSplitFollowTwo,\SBSplitData]}
\bsCODE{initialise\_arithmetic\_decoding(ctx\_labels)}{\ref{initarith}}
\bsFOR{ysb=0}{\SuperblocksY-1}
    \bsFOR{xsb=0}{\SuperblocksX-1}
        \bsCODE{sb\_split\_residual=read\_uinta(sb\_split\_contexts() ) }{\ref{sbcontexts}}
        \bsCODE{\SBSplit[ysb][xsb] = sb\_split\_residual}
        \bsCODE{\SBSplit[ysb][xsb]+=split\_prediction(xsb, ysb)}{\ref{splitprediction}}
        \bsCODE{\SBSplit[ysb][xsb] \%= 3}
    \bsEND
\bsEND
\bsCODE{flush\_inputb()}{\ref{blockreadbit}}
\end{pseudo}

\subsubsection{Propagating data between blocks}
\label{propagatedata}

The superblock splitting mode determines the maximum number of values to be
decoded for each block motion data element: 0, 4, or 16. If the splitting mode
is 0 or 1 and a value is decoded it applies to all 16 blocks or to one of the 4 2x2 
sub-arrays of blocks within the superblock. So that prediction of values shall
operate correctly, once decoded a value shall be propagated to all blocks to
which it applies.

The $propagate\_data(xtl, ytl, k,idx)$ shall copy decoded block data from the 
top-left-most block $(xtl, ytl)$ of an array of $k\times k$ blocks, where $k$ shall be 4 if the
splitting mode is 0 and $k$ shall be 2 if the splitting mode is 1. It shall be defined
as follows:

\begin{pseudo}{propagate\_data}{xtl, ytl, k,label}
\bsFOR{y=ytl}{ytl+k-1}
    \bsFOR{x=xtl}{xtl+k-1}
        \bsCODE{\BlockData[y][x][label]=\BlockData[ytl][xtl][label]}
    \bsEND
\bsEND
\end{pseudo}

\subsubsection{Block prediction modes}
\label{blockpredmodes}

The prediction mode process shall decode global motion and reference
 prediction modes required for each superblock according to the
the superblock splitting mode.: 16 values shall be decoded for split mode 2, 
4 values shall be decoded for split mode 1, and 1 value for split mode 0. 

For split modes 0 and 1, decoded values shall placed in the top-left corner block 
of the array (4x4 or 2x2) of blocks to which they apply, and then propagated to the 
other blocks.

The $prediction\_modes()$ process shall be defined as follows:

\begin{pseudo}{prediction\_modes}{}
\bsITEM{length}{uint}{}
\bsCODE{\ABitsLeft= 8*length}
\bsCODE{byte\_align()}
\bsCODE{ctx\_labels=[\PredModeOne,\PredModeTwo,\BlockGlobal]}
\bsCODE{initialise\_arithmetic\_decoding(ctx\_labels)}{\ref{initarith}}
\bsFOR{ysb=0}{\SuperblocksY-1}
    \bsFOR{xsb=0}{\SuperblocksX-1}
        \bsCODE{block\_count = 2^{\SBSplit[ysb][xsb]}}
        \bsCODE{step = 4//block\_count }
        \bsFOR{q=0}{block\_count-1}
            \bsFOR{p=0}{block\_count-1}
                \bsCODE{block\_ref\_mode(4*xsb+p*step, 4*ysb+q*step)}{\ref{blockmode}}
                \bsCODE{propagate\_data(4*xsb+p*step, 4*ysb+q*step, step,\RMode)}{\ref{propagatedata}}
                \bsCODE{block\_global\_mode(4*xsb+p*step, 4*ysb+q*step)}{\ref{blockglobal}}
                \bsCODE{propagate\_data(4*xsb+p*step, 4*ysb+q*step, step,\GMode)}{\ref{propagatedata}}
           \bsEND
        \bsEND
    \bsEND
\bsEND
\bsCODE{flush\_inputb()}{\ref{blockreadbit}}
\end{pseudo}

\paragraph{Block prediction mode}
\label{blockmode}

The $block\_ref\_mode()$ process shall be defined as follows:

\begin{pseudo}{block\_ref\_mode}{x, y}
\bsCODE{\BlockData[y][x][\RMode] = 0}
\bsIF{read\_boola(\PredModeOne)==\true}
    \bsCODE{\BlockData[y][x][\RMode] = 1}
\bsEND
\bsIF{num\_refs() == 2}
    \bsIF{read\_boola(\PredModeTwo)==\true}
        \bsCODE{\BlockData[y][x][\RMode] += 2}
    \bsEND
\bsEND
\bsCODE{\BlockData[y][x][\RMode] \wedge=ref\_mode\_prediction(x, y)}{\ref{modeprediction}}
\end{pseudo}

\paragraph{Block global mode}
\label{blockglobal}
$\ $\newline

The $block\_global\_mode()$ process shall be defined as follows:

\begin{pseudo}{block\_global}{x, y}
\bsCODE{\BlockData[y][x][\GMode]=\false}
\bsIF{\PictureUsingGlobal==\true}
    \bsIF{\BlockData[y][x][\RMode]!=\Intra}
        \bsCODE{block\_global\_residue = read\_boola(\BlockGlobal)}
        \bsCODE{\BlockData[y][x][\GMode] = block\_global\_residue}
        \bsCODE{\BlockData[y][x][\GMode] \wedge= block\_global\_prediction(x, y)}{\ref{blockglobalprediction}}
    \bsEND
\bsEND
\end{pseudo}

\subsubsection{Block motion vector elements}
\label{blockmvelements}

The vector element process shall decode the set of horizontal, or the set of vertical  
motion vector elements associated with one of the reference pictures.

$vector\_elements()$ process shall be defined as follows:

\begin{pseudo}{vector\_elements}{ref,dirn}
\bsITEM{length}{uint}{}
\bsCODE{\ABitsLeft= 8*length}
\bsCODE{byte\_align()}
\bsCODE{ctx\_labels=\begin{array}{l}[\VectorFollowOne,\VectorFollowTwo,\VectorFollowThree,
\VectorFollowFour,\VectorFollowFivePlus,\\
\VectorData,\VectorSign]
\end{array}}
\bsCODE{initialise\_arithmetic\_decoding(ctx\_labels)}{\ref{initarith}}
\bsFOR{ysb=0}{\SuperblocksY-1}
    \bsFOR{xsb=0}{\SuperblocksX-1}
        \bsCODE{block\_count = 2^{\SBSplit[ysb][xsb]}}
        \bsCODE{step = 4//block\_count }
        \bsFOR{q=0}{block\_count-1}
            \bsFOR{p=0}{block\_count-1}
                \bsCODE{block\_vector(4*xsb+p*step, 4*ysb+q*step, ref, dirn)}
                \bsCODE{propagate\_data(4*xsb+p*step, 4*ysb+q*step, step,\Vect)}{\ref{propagatedata}}
           \bsEND
        \bsEND
    \bsEND
\bsEND
\bsCODE{flush\_inputb()}{\ref{blockreadbit}}
\end{pseudo}

The block vector proces shall decode an individual motion vector element. 
It shall be defined as follows:

\begin{pseudo}{block\_vector}{x, y, ref, dirn}
\bsIF{\BlockData[y][x][\RMode][ref] == \true}
    \bsIF{\BlockData[y][x][\GMode]==\false}
        \bsCODE{mv\_residual = read\_sinta(mv\_contexts() ) }{\ref{mvcontexts}}
        \bsCODE{\BlockData[y][x][\Vect][ref][dirn] = mv\_residual}
        \bsCODE{\BlockData[y][x][\Vect][ref][dirn] +=mv\_prediction(x, y, ref, dirn)}
    \bsEND
\bsEND
\end{pseudo}

\subsubsection{DC values}
\label{DCvalues}

The DV value process shall decode the DC values for a intra blocks for a given video component (Y, C1 or C2). It shall be defined as follows:

\begin{pseudo}{dc\_values}{c}
\bsITEM{length}{uint}{}
\bsCODE{\ABitsLeft= 8*length}
\bsCODE{byte\_align()}
\bsCODE{ctx\_labels=[\DCFollowOne,\DCFollowTwoPlus,\DCData,\DCSign]}
\bsCODE{initialise\_arithmetic\_decoding(ctx\_labels)}{\ref{initarith}}
\bsFOR{ysb=0}{\SuperblocksY-1}
    \bsFOR{xsb=0}{\SuperblocksX-1}
        \bsCODE{block\_count = 2^{\SBSplit[ysb][xsb]}}
        \bsCODE{step = 4//block\_count }
        \bsFOR{q=0}{block\_count-1}
            \bsFOR{p=0}{block\_count-1}
                \bsCODE{block\_dc(4*xsb+p*step, 4*ysb+q*step, c)}
                \bsCODE{propagate\_data(4*xsb+p*step, 4*ysb+q*step, step,\DC)}{\ref{propagatedata}}
           \bsEND
        \bsEND
    \bsEND
\bsEND
\bsCODE{flush\_inputb()}{\ref{blockreadbit}}
\end{pseudo}

The block DC process shall decode an individual component DC value. It shall
be defined as follows:

\begin{pseudo}{block\_dc}{x, y,c}
\bsIF{\BlockData[y][x][\RMode]=\Intra}
    \bsCODE{dc\_residual = read\_sinta(dc\_contexts()) }{\ref{dcvaluecontexts}}
    \bsCODE{\BlockData[y][x][\DC][c] = dc\_residual}
    \bsCODE{\BlockData[y][x][\DC][c] +=dc\_prediction(x, y, c)}{\ref{dcprediction}}
\bsEND
\end{pseudo}

\subsubsection{Spatial prediction of motion data elements}

\label{spatialprediction}

\paragraph{Prediction apertures\\}

A consistent convention for prediction apertures is used. The nominal prediction 
aperture for block motion data is defined to be the relevant data to the left, top
and top-left of the data element in question (Figure \ref{predaperture}). 
For the superblock split mode of 
the superblock with index $(i,j)$ this means the superblocks with indices $(i-1,j)$,
$(i,j-1)$ and $(i-1,j-1)$. For the block motion data itself, the same applies where these
indices are {\em block} indices. 

\setlength{\unitlength}{1em}
\begin{figure}[!ht]
\centering
\begin{picture}(15,20)
\multiput(0,0)(8,0){3}%
  {\line(0,1){16}}
\multiput(0,0)(0,8){3}%
  {\line(1,0){16}}
  
%Shading  

\multiput(0,0)0.2,0){40}%  
{\multiput(8,0.1)(0,.2){40}%
  {\tiny.}
}

%Arrows
\put(4,12){\vector(1,-1){6}}
\put(12,12){\vector(0,-1){6}}
\put(4,4){\vector(1,0){6}}
\end{picture}
\caption{Basic prediction aperture}\label{predaperture}
\end{figure}

This is the nominal prediction aperture. Not all data elements in this prediction
aperture may be available, either because they would require negative indices, or
 because the data is not available - for example a block to the left of a block with 
reference mode  \RefTwoOnly may have reference mode \RefOneOnly and so 
can furnish no contribution for a prediction to the
Reference 2 motion vector.

When superblocks have split level 1 or 0, block data shall be propagated
(Section \ref{propagatedata}) across 4 or 16 blocks so as to furnish a prediction. The
effect is illustrated in Figure \ref{splitapertures}.

\setlength{\unitlength}{.75em}
\begin{figure}[!ht]
\centering
\begin{picture}(60,17.5)

\multiput(0,0)(20,0){3}%
{

%Main Grid
\multiput(0,0)(8,0){3}%
  {\line(0,1){16}}
\multiput(0,0)(0,8){3}%
  {\line(1,0){16}}
\multiput(0,0)0.2,0){40}%  

%Shading
\multiput(0,0)0.4,0){20}% 
{\multiput(8,0.2)(0,.4){20}%
  {\tiny.}
}

%Dotted Grid
\multiput(0,0)(0,2){3}%
{\multiput(0,2)(0.5,0){16}%
   {\line(1,0){.25}}
}
\multiput(0,0)(2,0){3}%
{\multiput(2,0)(0,0.5){16}%
   {\line(0,1){.25}}
}

\multiput(8,0)(0,8){2}%
{\multiput(0,4)(0.5,0){16}%
   {\line(1,0){.25}}
}

\multiput(12,0)(0,0.5){32}%
   {\line(0,1){.25}}
}
%Arrows
\put(4,12){\vector(1,-1){5}}
\put(7,7) {\vector(2,-1){1.5}}
\put(10,10){\vector(0,-1){3}}

\put(30,6){\vector(0,-1){3.5}}
\put(27,5) {\vector(1,-1){2}}
\put(27,3){\vector(2,-1){2}}

\put(50,2){\vector(1,0){3.5}}
\put(50,6) {\vector(1,-1){3.5}}
\put(54,6){\vector(0,-1){3.5}}

\end{picture}
\caption{Effect of splitting modes on spatial prediction}\label{splitapertures}
\end{figure}

\paragraph{Superblock split prediction}
\label{splitprediction}
$\ $\newline
$split\_prediction$ returns the mean of the the neighbouring split values:

\begin{pseudo}{split\_prediction}{x, y}
\bsIF{ x==0 \text{ \bf and } y==0 }
    \bsRET{0}
\bsELSEIF{y==0}
    \bsRET{\SBSplit[0][x-1]}
\bsELSEIF{x==0}
    \bsRET{\SBSplit[y-1][0]}
\bsELSE
    \bsCODE{ 
    \begin{array}{ll}
    \text{\bf return} & \mean(\SBSplit[y-1][x-1], \\
                       &  \quad\quad\quad \SBSplit[y][x-1],  \\
                       &  \quad\quad\quad \SBSplit[y-1][x])
    \end{array}
    }
\bsEND
\end{pseudo}

\paragraph{Block mode prediction}
\label{modeprediction}
$\ $\newline
The $ref\_mode\_prediction()$ function shall return a value that represents a majority
 verdict for the presence of each of the references individually. It shall be defined
 as follows:

\begin{pseudo}{ref\_mode\_prediction}{x, y}
\bsIF{ x==0 \text{ \bf and } y==0 }
    \bsRET{\Intra}
\bsELSEIF{y==0}
    \bsRET{\BlockData[0][x-1][\RMode]}
\bsELSEIF{x==0}
    \bsRET{\BlockData[y-1][0][\RMode]}
\bsELSE
    \bsCODE{num\_ref1\_nbrs=\BlockData[y-1][x][\RMode] \& 1}
    \bsCODE{num\_ref1\_nbrs +=\BlockData[y-1][x-1][\RMode] \& 1}
    \bsCODE{num\_ref1\_nbrs +=\BlockData[y][x-1][\RMode] \& 1}
    \bsCODE{pred=num\_ref1\_nbrs//2}
    \bsCODE{num\_ref2\_nbrs=(\BlockData[y-1][x][\RMode]\gg 1) \& 1}
    \bsCODE{num\_ref2\_nbrs +=(\BlockData[y-1][x-1][\RMode] \gg 1) \& 1}
    \bsCODE{num\_ref2\_nbrs +=(\BlockData[y][x-1][\RMode] \gg 1) \& 1}
    \bsCODE{pred \wedge= (num\_ref2\_nbrs//2)\ll 1}
    \bsRET{pred}
\bsEND
\end{pseudo}

\paragraph{Block global flag prediction}
\label{blockglobalprediction}
$\ $\newline
The $block\_global\_prediction()$ function shall return a value that represents a majority
 verdict of the neighbouring blocks. It shall be defined as follows:

\begin{pseudo}{block\_global\_prediction}{x, y}
\bsIF{ x==0 \text{ \bf and } y==0 }
    \bsRET{\false}
\bsELSEIF{ y==0 }
    \bsRET{\BlockData[0][x-1][\GMode]}
\bsELSEIF{x==0}
    \bsRET{\BlockData[y-1][0][\GMode]}
\bsELSE
    \bsCODE{
    \begin{array}{ll}
    \text{\bf return} & \majority(\BlockData[y-1][x-1][\GMode], \\
    & \quad\quad\quad \BlockData[y-1][x][\GMode],  \\
    & \quad\quad\quad \BlockData[y][x-1][\GMode]) 
    \end{array}
    }
\bsEND
\end{pseudo}

\paragraph{Motion vector prediction}
\label{mvprediction}
$\ $\newline
Motion vectors shall be predicted using the median of available block vectors 
in the aperture. A vector shall be available for prediction if:
\begin{enumerate}
\item its block falls within the picture area,
\item its prediction mode allows it to be defined, and
\item it is not a global motion block. 
\end{enumerate}

The $mv\_prediction(x, y, ref, dirn)$ shall return motion values according to
the following rules:

{\bf Case 1.}  If $x==0$ and $y==0$, the value $0$ shall be returned.

{\bf Case 2.} If $x>0$ and $y==0$ then:
\begin{enumerate}
   \item If  $\BlockData[0][x-1][\GMode]==\false$  and 
$\BlockData[0][x-1][\RMode][ref]==\true$ then vector element $\BlockData[0][x-1][ref][dirn]$ shall be returned,
   \item otherwise, $0$ shall be returned
\end{enumerate}

{\bf Case 3.} If $x==0$ and $y>0$ then:
\begin{enumerate}
   \item If  $\BlockData[y-1][0][\GMode]==\false$ and 
$\BlockData[y-1][0][\RMode][ref]==\true$ then vector element 
$\BlockData[y-1][0][ref][dirn]$ shall be returned,
   \item otherwise, $0$ shall be returned
\end{enumerate}

{\bf Case 4.} If both $x>0$ and $y>0$ then all 3 blocks in the prediction aperture 
may potentially contribute to the prediction. Define the set $values=\{\}$. The prediction 
shall be the median of the available vector elements, as defined in the following
pseudocode:

\begin{pseudo*}
\bsIF{x>0 \text{ \bf and } y>0 }
    \bsIF{\BlockData[y][x-1][\GMode]==\false}
        \bsIF{\BlockData[y][x-1][\RMode][ref]==\true}
            \bsCODE{values = values\cup\{\BlockData[y][x-1][ref][dirn] \} }
        \bsEND
    \bsEND
    \bsIF{\BlockData[y-1][x][\GMode]==\false}
        \bsIF{\BlockData[y-1][x][\RMode][ref]==\true}
            \bsCODE{values = values\cup\{\BlockData[y-1][x][ref][dirn] \} }
        \bsEND
    \bsEND
    \bsIF{\BlockData[y-1][x-1][\GMode]==\false}
        \bsIF{\BlockData[y-1][x-1][\RMode][ref]==\true}
            \bsCODE{values = values\cup\{\BlockData[y-1][x-1][ref][dirn] \} }
        \bsEND
    \bsEND

    \bsRET{\median(values)}{\ref{integerops}}
\bsEND
\end{pseudo*}

(Note that the median of an empty set is zero.)

\paragraph{DC value prediction}
\label{dcprediction}
$\ $\newline
DC values shall be predicted using the unbiased mean of available values 
in the prediction aperture. 

The process $dc\_prediction(x, y, c)$ shall return values according to
the following rules:

{\bf Case 1.}  If $x==0$ and $y==0$, $0$ shall be returned.

{\bf Case 2.} If $x>0$ and $y==0$ then:
\begin{enumerate}
   \item If $\BlockData[0][x-1][\RMode]==\Intra$, $\BlockData[0][x-1][\DC][c]$ 
shall be returned,
   \item otherwise, $0$ shall be returned.
\end{enumerate}

{\bf Case 3.} If $x==0$ and $y>0$ then:
\begin{enumerate}
   \item If $\BlockData[y-1][0][\RMode]==\Intra$, $\BlockData[y-1][0][\DC][c]$
shall be returned,
   \item otherwise, 0 shall be returned.
\end{enumerate}

{\bf Case 4.} If both $x>0$ and $y>0$ then all 3 blocks in the prediction aperture may
 potentially contribute to the prediction. Define a set $values=\{\}$. The prediction shall
be the unbiased mean of available values, as defined in the following pseudocode:

\begin{pseudo*}
\bsIF{ x>0 \text{ \bf and } y>0 }
    \bsIF{\BlockData[y][x-1][\RMode]==\Intra}
        \bsCODE{values = values\cup\{\BlockData[y][x-1][\DC][c] \} }
    \bsEND
    \bsIF{\BlockData[y-1][x][\RMode]==\Intra}
        \bsCODE{values = valuesx\cup\{\BlockData[y-1][x][ref][\DC][c] \} }
    \bsEND
    \bsIF{\BlockData[y-1][x-1][\RMode]==\Intra}
        \bsCODE{values = values\cup\{\BlockData[y-1][x-1][ref][\DC][c] \} }
    \bsEND
    \bsIF{values!=\{\}}
        \bsRET{pred =\mean(values)}
    \bsELSE
        \bsRET{0}
    \bsEND
\bsEND
\end{pseudo*}

\subsubsection{Block motion parameter contexts}

\paragraph{Superblock splitting mode}\label{sbcontexts}
$\ $\newline
The $sb\_split\_contexts()$ function shall return a context label map $c$ with the 
following values:

\begin{itemize}
\item $c[FOLLOW] = [ \SBSplitFollowOne, \SBSplitFollowTwo ]$
\item $c[DATA] = \SBSplitData$
\end{itemize}

\paragraph{Motion vectors}
\label{mvcontexts}
$\ $\newline
The $mv\_contexts()$ function shall return a context label map $c$ with the 
following values:

\begin{itemize}
\item $c[FOLLOW] = [ \VectorFollowOne, \VectorFollowTwo, \VectorFollowThree, \VectorFollowFour, \VectorFollowFivePlus ]$
\item $c[DATA] = \VectorData$
\item $c[SIGN] = \VectorSign$
\end{itemize}

\paragraph{DC values}
\label{dcvaluecontexts}

The $dc\_contexts()$ function shall return a context label map $c$ with the 
following values:

\begin{itemize}
\item $c[FOLLOW] = [ \DCFollowOne, \DCFollowTwoPlus ]$
\item $c[DATA] = \DCData$
\item $c[SIGN] = \DCSign$
\end{itemize}


\clearpage
\section{Coefficient data unpacking}\label{wltunpacking}
\subsection{Wavelet coefficient unpacking overview}
\label{unapckingoverview}

This section specifies the unpacking (parsing and inverse
quantisation) of wavelet coefficient data
from the Dirac stream. The result of this process is a set
of fully-populated wavelet subband data arrays, as defined in
Section \ref{wltdecodeconventions}, containing coefficient
data with full dynamic range.

Wavelet coefficients are packed in one of two possible formats.
In the core syntax, coefficients are grouped within individual
subbands, representing a range of spatial frequencies, from the
lowest to the highest. A full set of subbands is encoded for each
video component in turn. 

In the low-delay syntax, coefficients
are groups into `slices' which represent coefficient pertaining
to an area of the picture. Each slice contains both luma and chroma
data, and all spatial frequency bands. Unpacking a slice therefore
allows an area of picture to be extracted without extracting (or even
receiving) the remaining picture data.

\subsection{Subband data structures}
\label{wltdecodeconventions}

\subsubsection{Wavelet data initialisation}

\label{wltinit}

This section specifies the $initialise\_wavelet\_data(comp)$ process, which returns a structure which will
contain the wavelet coefficients for the component indicated by $comp$. 

For the purposes of this specification, this is a four-dimensional array $data$,
where individual subbands are two-dimensional arrays accessed by level and orientation:

$band = data[level][orient]$

Valid levels are integers from in the range 0 to $\TransformDepth$ inclusive. 
Level 0 consists of a single subband with orientation \LL. 
All other levels consist of 3 subbands of orientation \LH, \HL, 
and \HH. The orientations correspond to either low- or high-pass filtering
horizontally and vertically: so the \LH band consists of coefficients derived
from horizontal low-pass filtering and vertical high-pass filtering. The subbands
partition the spatial frequency domain by orientation and level as illustrated
in Figure \ref{fig:orientlevel}.

Each subband array is initialised so that:
\begin{eqnarray*}
\width(data[level][orient]) & = & subband\_width(level,comp) \\
\height(data[level][orient]) & = & subband\_height(level,comp)
\end{eqnarray*}

as specified in Section \ref{subbandwidthheight}. These dimensions correspond 
to a wavelet transform being performed on a copy of the component data which 
has been padded (if necessary) so that its
dimensions are a multiple of $2^{\TransformDepth}$.

Individual subband coefficients are signed integers accessed by vertical and 
horizontal coordinates within the subband
\[c = data[level][orient][y][x]\]

where the range of allowable coordinates for a subband coefficient is 
$0\leq x<subband\_width(level,comp)$ and
$0\leq y<subband\_height(level,comp)$.

\setlength{\unitlength}{1em}
\begin{figure}[!ht]
\centering
\begin{picture}(25,35)

%Main Grid
\multiput(0,0)(16,0){3}%
  {\line(0,1){32}}
\multiput(0,0)(0,16){3}%
  {\line(1,0){32}}

%Second Grid
\put(0,24){\line(1,0){16}}
\put(8,16){\line(0,1){16}}

%Third Grid
\put(0,28){\line(1,0){8}}
\put(4,24){\line(0,1){8}}

%Fourth Grid
\put(0,30){\line(1,0){4}}
\put(2,28){\line(0,1){4}}


%Put Levels

\put(24,34){Level 4}
\put(23,36){\vector(0,-1){3}}

\put(13,34){Level 3}
\put(12,36){\vector(0,-1){3}}

\put(7,34){2}
\put(6,36){\vector(0,-1){3}}

\put(4,34){1}
\put(3,36){\vector(0,-1){3}}

%Put decomposition numbers
\put(23,24){4-HL}
\put(7,8){4-LH}
\put(23,8){4-HH}
\put(11,28){3-HL}
\put(3,20){3-LH}
\put(11,20){3-HH}
\put(5,30){2-HL}
\put(1,26){2-LH}
\put(5,26){2-HH}
\put(2.2,31){\tiny 1-HL}
\put(0.2,29){\tiny 1-LH}
\put(2.2,29){\tiny 1-HH}
\put(0.2,31){\tiny 0-LL}

  
\end{picture}
\caption{Subband decomposition of the spatial frequency domain showing subband 
numbering, for a 4-level wavelet decomposition}\label{fig:orientlevel}

\end{figure}

\subsubsection{Dimensions of wavelet subbands}
\label{subbandwidthheight}

This section defines the values of the $subband\_width(level, comp)$ and $subband\_height(level,comp)$
functions, giving the width and height of subbands at a given level for a given component, and hence the range
of subband vertical and horizontal indices. 

If $comp==Y$, set
\begin{eqnarray*}
w & = & \LumaWidth \\
h & = & \LumaHeight
\end{eqnarray*}

Otherwise, set
\begin{eqnarray*}
w & = & \ChromaWidth \\
h & = & \ChromaHeight
\end{eqnarray*}

Define the padded dimensions of the component by
\begin{eqnarray*}
pw = 2^{\TransformDepth}*\left\lceil\frac{w}{2^{\TransformDepth}}\right\rceil\\ 
ph = 2^{\TransformDepth}*\left\lceil\frac{h}{2^{\TransformDepth}}\right\rceil
\end{eqnarray*}

If $level==0$,
\begin{eqnarray*}
subband\_width(level) & = & pw//2^{\TransformDepth} \\
& = & \left\lceil\frac{w}{2^{\TransformDepth}}\right\rceil \\
subband\_height(level) & = & ph//2^{\TransformDepth} \\
& = & \left\lceil\frac{h}{2^{\TransformDepth}}\right\rceil
\end{eqnarray*}

If $level>0$
\begin{eqnarray*}
subband\_width(level) & = & pw//2^{\TransformDepth-level+1} \\
& = & 2^{level-1}*\left\lceil \frac{w}{2^{\TransformDepth}}\right\rceil \\
subband\_height(level) & = & ph//2^{\TransformDepth-level+1} \\
& = & 2^{level-1}*\left\lceil\frac{h}{2^{\TransformDepth}}\right\rceil
\end{eqnarray*}

\begin{informative}
In encoding, these padded dimensions may be achieved by padding the 
component data up to the padded dimensions and applying the forward
Discrete Wavelet Transform (the inverse of the operations specified in
Section \ref{idwt}). Any values may be used for the padded data, although
the choice will affect wavelet coefficients at the right and bottom 
edges of the subbands. Good results, in compression terms, may be obtained
 by using edge extension for intra pictures and zero extension for inter 
pictures. A poor choice of padding may cause visible artefacts near the
bottom and right edges at high levels of compression.
\end{informative}

\subsection{Inverse quantisation}
\label{invquant}

This section specifies the operation of inverse quantisation, which scales the
dynamic range of unpacked wavelet coefficients according to a pre-determined factor.
The inverse quantisation operation is common to both the low-delay and core syntax.

The $inverse\_quant()$ function is defined by:

\begin{pseudo}{inverse\_quant}{quantised\_coeff, quant\_index}
\bsCODE{magnitude = |{quantised\_coeff}|}
\bsCODE{magnitude *= quant\_factor(quant\_index)}{\ref{quantfacs}}
\bsIF{magnitude!=0}
  \bsCODE{magnitude += quant\_offset(quant\_index)}{\ref{quantfacs}}
  \bsCODE{magnitude += 2}
  \bsCODE{magnitude = magnitude//4}
\bsEND
\bsRET{\sign( quantised\_coeff )*magnitude}
\end{pseudo}

\begin{informative}
The pseudocode description separates inverse quantisation from coefficient unpacking. However, 
since dead-zone quantisation is used, the $inverse\_quant()$ function must compute
the magnitude. Hence it is more efficient to first extract the coefficient magnitude,
then inverse quantise, and then extract the coefficient sign. 

Note that an unbiased division by 4 is applied: the additional 2 may be absorbed into
the quantisation offset, but is separated in this specification.
\end{informative}

\subsubsection{Quantisation factors and offsets}
\label{quantfacs}

This section specifies the operation of the $quant\_factor()$ and 
$quant\_offset()$ functions for performing inversion quantisation.

Quantisation factors represent an integer approximation of quarter-bit values 
with two bits of accuracy i.e. of $(2^{\frac{index}{4}+2})$. Note that 64 bits 
of accuracy is required to compute these factors correctly up to $index=128$.

Quantisation factors are set as follows:

\begin{pseudo}{quant\_factor}{index}
\bsCODE{q=\max(index,0)}
\bsCODE{base = 2^{q//4}}
\bsIF{ (q\%4)==0 }
  \bsRET{4*base}
\bsELSEIF{ (q\%4)==1 }
  \bsRET{(503829*base+52958)//105917}
\bsELSEIF{ (q\%4)==2 }
  \bsRET{(665857*q+58854)//117708}
\bsELSEIF{ (q\%4)==3 }
  \bsRET{(440253*base+32722)//65444}
\bsEND
\end{pseudo}

For intra pictures, offsets are approximately $1/2$ of the 
quantisation factors, and for inter pictures they are $3/8$ - these
mark the reconstruction point within the quantisation interval:

\begin{pseudo}{quant\_offset}{index}
\bsIF{index==0}
  \bsCODE{offset = 1}
\bsELSE
  \bsIF{ is\_intra() }
    \bsIF{index==1}
      \bsCODE{offset = 2}
    \bsELSE
      \bsCODE{ offset=(quant\_factor(index)+1)//2 }
    \bsEND
  \bsELSE
    \bsCODE{ offset=(quant\_factor(index)*3+4)//8 }
  \bsEND
\bsEND
\bsRET{offset}
\end{pseudo}

\begin{informative}
The quantisation offsets have been selected so as to make inverse quantisation
and re-quantisation by the same quantisation factor transparent. This requires that
\[3\leq quant\_offset+2<quant\_factor\] -- hence the special conditions for 
quantisation indexes 0 and 1.
\end{informative}

\subsection{Core syntax wavelet coefficient unpacking}%%%%%%%%%%%%%%%%%%%%%%%%%%%%%%%%%%%%%%%%%%%%%%%%%%%%%
% - This chapter defines how wavelet coefficients - %
% - are decoded                                   - % 
%%%%%%%%%%%%%%%%%%%%%%%%%%%%%%%%%%%%%%%%%%%%%%%%%%%%%

\label{transformdec}

This section specifies the overall operation of the $core\_transform\_data(comp)$ process
for unpacking the set of coefficient subbands corresponding
to a video picture component (Y, C1 or C2) in the core Dirac syntax, 
according to the conventions set out in Section \ref{wltdecodeconventions}.

In the Dirac stream core syntax, subband data is arranged by 
level and orientation, from level 0 up to level \TransformDepth. Coefficients may
be VLC or arithmetic coded. Where arithmetic coding is used, the unpacking process
 for each subband is contingent on data from subbands of the same orientation in 
 the next lower level. This is the {\em parent} subband; the subband of the same orientation in the next
higher level is the {\em child} subband. 

Unpacking an individual subband therefore requires prior unpacking of the parent subband,
and of its parent, and so on until level 1 is reached (level 1 subbands do not depend
upon the single level 0 DC band).

\begin{informative}
The data for each subband consists of a subband header and a block of arithmetically
coded coefficient data. The subband header contains a length code giving the number of
bytes of the block of arithmetically-coded data. The transform data can therefore be
parsed without invoking arithmetic decoding at all, since the length codes allow a 
parser to skip from one subband header to the next, similarly to the way that parse unit
offsets allow frame skipping.
\end{informative}


\subsubsection{Overall process}
The overall $core\_transform\_data()$ process is as follows:

\begin{pseudo}{core\_transform\_data}{comp}
\bsCODE{data=initialise\_wavelet\_data(comp)}{\ref{wltinit}}
\bsCODE{subband(data, 0, \LL)}{\ref{subbanddecodeprocess}}    
\bsFOR{level=1}{\TransformDepth}
  \bsFOREACH{orient}{\HL,\LH,\HH}
    \bsCODE{byte\_align()}
    \bsCODE{subband(data, level, orient)}{\ref{subbanddecodeprocess}}
  \bsEND
\bsEND
\bsRET{data}
\end{pseudo}

\subsubsection{Subbands}

\label{subbanddecodeprocess}

This section specifies the process $subband(data, level,orient)$ for unpacking coefficients
within a subband at level $level$ ($0$ to \TransformDepth) and of orientation $orient$
(\LL, \LH, \HL, or \HH). 

The overall process consists of reading a byte-aligned header for each
subband, including a length code for the subsequent arithmetically-coded data.
Subband data is initialised to 0. If the length code is 0, the subband may be 
skipped and all data remains set to zero.
Intra DC bands are predicted, and so must additionally be reconstructed.

\begin{pseudo}{subband}{data, level, orient}
\bsITEM{length}{uint}
\bsCODE{zero\_subband\_data(data[level][orient])}{\ref{zerosubband}}
\bsIF{length == 0}
  \bsCODE{byte\_align()}
\bsELSE
    \bsCODE{quant\_index = read\_uint()}
    \bsCODE{byte\_align()}
    \bsCODE{subband\_coeffs(data,level,orient,length,quant\_index)}{\ref{subbandcoeffs}}
\bsEND 
\bsIF{is\_intra() \text{ and } level==0}
    \bsCODE{intra\_dc\_prediction(data[level][orient])}{\ref{intradcprediction}}
\bsEND
\end{pseudo}

\paragraph{Zero subband}
\label{zerosubband}
$\ $\newline$\ $\newline
The $zero\_subband()$ process sets all subband coefficients to 0. When
a subband or codeblock is skipped, all coefficients within the codeblock
will remain 0. 

\begin{pseudo}{zero\_subband\_data}{band}
\bsFOR{y=0}{\height(band)-1}
  \bsFOR{x=0}{\width(band)-1}
    \bsCODE{band[y][x] = 0}
  \bsEND
\bsEND
\end{pseudo}

\paragraph{Non-skipped subbands}
\label{subbandcoeffs}
$\ $\newline$\ $\newline
Data within subbands is split into one or more rectangular codeblocks (Figure \ref{codeblocks}).
Codeblocks are scanned in raster order across the subband and coefficients are scanned in raster order
within each codeblock. 

\begin{figure}
[Include codeblock figure \label{codeblocks}]
\end{figure}

The overall unpacking process is therefore:

\begin{pseudo}{subband\_coeffs}{data,level,orient,length}
\bsCODE{initialise\_arithmetic\_decoding(length)}{\ref{initarith}}
\bsFOR{y=0}{\Codeblocks[level][v]-1}
    \bsFOR{x=0}{\Codeblocks[level][h]-1}
        \bsCODE{codeblock(data,level,orient,quant\_index,y,x)}{\ref{codeblocks}}
    \bsEND
\bsEND
\bsCODE{flush\_inputb()}{\ref{blockreadbit}}
\end{pseudo}

\begin{informative}
Note that even if arithmetic coding is not being used for coefficient unpacking,
the arithmetic decoding engine is still initialised to ensure that $\ABitsLeft$ is set
to allow the bounded read function $read\_sintb()$ to operate correctly. 
\end{informative}

\paragraph{Intra DC band prediction}
\label{intradcprediction}
$\ $\newline
This section defines the operation of the $intra\_dc\_prediction(band)$ function
for reconstructing values within Intra picture DC bands using spatial prediction.
This function is applied once all coefficients in all codeblocks within the DC
band have been unpacked, although it may be applied progressively to each coefficient
as soon as it has been unpacked.

Intra DC values are derived by spatial prediction using the mean of the
three values to the left, top-left and above a coefficient (if available).

\begin{pseudo}{intra\_dc\_prediction}{band}
\bsCODE{prediction = 0 }
\bsFOR{v=0}{subband\_height(level)-1}
  \bsFOR{h=0}{subband\_width(level)-1}
    \bsIF{h>0}
        \bsIF{v>0}
            \bsCODE{prediction=\mean(band[v][h-1],band[v-1][h-1],band[v-1][h])}
        \bsELSE
            \bsCODE{prediction=band[0][h-1]}
        \bsEND
    \bsELSE
        \bsIF{v>0}
            \bsCODE{prediction=band[v-1][0]}
        \bsELSE
            \bsCODE{prediction = 0}
        \bsEND
    \bsEND
    \bsCODE{band[v][h] += prediction}
  \bsEND
\bsEND
\end{pseudo}

\subsubsection{Subband codeblocks}
\label{codeblocks}

This section defines the operation of the 
$codeblock(band,parent,level, orient,quant\_index,y,x)$ function, which unpacks
coefficients within a 
codeblock in position $(x,y)$.

\begin{comment}
[Include a figure here]
\end{comment}

\paragraph{Codeblock dimensions}
$\ $\newline

The codeblock covers coefficients in the horizontal range $left$ to $right-1$ and in the vertical
range $bottom$ to $top-1$ where these values are defined by:
\begin{eqnarray*}
  left & = & (\width(band)*x)//\Codeblocks[level][horizontal] \\
  right & = & (\width(band)*(x+1))//\Codeblocks[level][horizontal] \\
  bottom & = & (\height(band)*y)//\Codeblocks[level][vertical] \\
  top & = & (\height(band)*(y+1))//\Codeblocks[level][vertical]
\end{eqnarray*}

\paragraph{Codeblock unpacking loop}
$\ $\newline$\ $\newline
The codeblock unpacking process is defined as:

\begin{pseudo}{codeblock}{data,level,orient,quant\_index,y,x}
\bsCODE{skipped=zero\_flag()}{\ref{zeroblockflag}}
\bsIF{skipped==\false}
  \bsCODE{quant\_idx += quant\_offset()}{\ref{blockquantidx}}
  \bsFOR{v=bottom}{top-1}
    \bsFOR{h=left}{right-1}
      \bsIF{using\_ac()==\true}
          \bsCODE{coeff\_unpack(data,level,orient,quant_idx,v,h)}{\ref{wltcoeff}}
      \bsELSE
          \bsCODE{data[level][orient][v][h]=read\_sintb()}{\ref{segol}}
      \bsEND
    \bsEND
  \bsEND
\bsEND

\end{pseudo}

If the codeblock is skipped, then coefficients within that codeblock remain 0.

\begin{informative}
If arithmetic coding is not employed, the use of the bounded read operation ensures
that a subband may, if desired, be terminated early and all remaining coefficients
will be 0.
\end{informative}

\paragraph{Skipped codeblock flag}
\label{zeroblockflag}
$\ $\newline$\ $\newline
If the number of codeblocks is 1, then $zero\_flag()$ is set to $\false$, else
it is unpacked from the stream.

\begin{pseudo}{zero\_flag}{}
\bsCODE{num\_blocks=
\begin{array}{l}
\Codeblocks[level][horizontal]* \\
\Codeblocks[level][vertical]
\end{array} }
\bsIF{num_blocks==1}
    \bsRET{\false}
\bsELSE
    \bsRET{ read\_boola(\ZeroCodeblock) }
\bsEND
\end{pseudo}

\paragraph{Block quantiser offset}
$\ $\newline$\ $\newline
\label{blockquantidx}

If $\CodeblockMode=\SingleQuantiser$,  $quant\_offset()$ shall return 0.

If $\CodeblockMode=\MultipleQuantiser$ then the quantiser index offset
is extracted from the stream -- $read\_sinta(quant\_contexts())$ is returned, where
$quant\_contexts()$ returns the context set:

\begin{itemize}
\item{Follow= \{\QOffsetFollow\}}
\item{Data=\QOffsetData}
\item{Sign=\QOffsetSign}
\end{itemize}

\subsubsection{Subband coefficients}

\label{wltcoeff}

This section describes the operation of the 
$coeff\_unpack(data,level,orient,quant\_idx,v,h)$ process
for unpacking an individual coefficient in position $(h,v)$ 
in the subband $data[level][orient]$.

Unpacking a coefficient makes use of arithmetic decoding, inverse quantisation
and, in the case of DC (level 0) bands of Intra frames, neighbourhood prediction.

Arithmetic coding uses a highly compact set of contexts, 
with magnitudes contextualised on whether parent values
and neighbouring values are zero or non-zero.

\paragraph{Overall process}
$\ $\newline
The overall process for unpacking an individual coefficient is:

\begin{pseudo}{coeff\_unpack}{data,level,orient,quant\_index,v,h}
    \bsCODE{parent = parent\_val(data,level,orient, v, h)}{\ref{parentval}}
    \bsCODE{nhood = zero\_nhood(data[level][orient],v,h)}{\ref{zeronhood}}
    \bsCODE{sign\_pred = sign\_predict(data[level][orient],orient,v,h)}{\ref{signpredict}}
    \bsCODE{context\_set = select\_coeff\_ctxs(nhood, parent, sign\_pred)}{\ref{selectcoeffcontext}}
    \bsCODE{quant\_coeff = read\_sinta( context\_set )}{}
    \bsCODE{data[level][orient][v][h] = inverse\_quant( quant\_coeff, quant\_index )}{\ref{invquant}}
\end{pseudo}

\paragraph{Parent values}
\label{parentval}
$\ $\newline
The function $parent\_val(data,level,orient,v, h)$ returns the parent value of a coefficient in a subband,
which is the co-located coefficient in the parent subband, if there is one. There is deemed to be a 
parent subband if $level\geq 2$:

\begin{pseudo}{parent}{data,level,orient,v,h}
\bsIF{level>=2}
    \bsCODE{parent = data[level-1][orient][v//2][h//2]}
\bsELSE
    \bsCODE{parent = 0}
\bsEND
\bsRET{parent}
\end{pseudo}

\paragraph{Zero neighbourhood}
\label{zeronhood}
$\ $\newline
The $zero\_nhood()$ function returns a boolean indicating whether neighbouring
values are all zero.

\begin{pseudo}{zero\_nhood}{band,v,h}
\bsIF{v>0}
  \bsIF{band[v-1][h]!=0}
    \bsRET{\false}
  \bsEND
  \bsIF{h>0}
    \bsIF{band[v-1][h-1])!=0 \text{ or } band[v][h-1]!=0}
      \bsRET{\false}
    \bsEND
  \bsEND
\bsELSE
  \bsIF{h>0}
    \bsIF{ band[v][h-1] !=0}
      \bsRET{\false}
    \bsEND
  \bsEND
\bsEND
\bsRET{\true}
\end{pseudo}

\paragraph{Sign prediction}
\label{signpredict}
$\ $\newline
The $sign\_predict()$ function returns a prediction for the sign of the 
current pixel. Correlation within subbands depends upon orientation,
and so this is taken into account in forming the prediction.

For vertically-oriented (HL) bands, the predictor is the sign of the
coefficient above the current coefficient; for horizontally-oriented (LH)
bands, the predictor is the sign of the coefficient to the left. 

The predictions are not used for differential encoding of the sign, but for
conditioning of the sign contexts only.

\begin{pseudo}{sign\_predict}{band,orient,v,h}
\bsIF{orient==HL}
  \bsIF{v==0}
    \bsRET{0}
  \bsELSE
    \bsRET{\sign(band[v-1][h])}
  \bsEND
\bsELSEIF{orient==LH}
  \bsIF{h==0}
    \bsRET{0}
  \bsELSE
    \bsRET{\sign(band[v][h-1])}
  \bsEND
\bsELSE
  \bsRET{0}
\bsEND{}
\end{pseudo}

\paragraph{Coefficient context selection}
\label{selectcoeffcontext}
$\ $\newline
This section defines the $select\_coeff\_ctxs(zero\_nhood, parent, sign\_pred)$
function, which chooses a context index set for unpacking a coefficient value.

Twelve possible coefficient index sets are defined, and are returned as specified 
in Table \ref{contexttable}. Note that follow contexts are an array indexed from $0$
as per Section \ref{arithreadint}.

Note that parent values affect the context of all follow bits, and that neighbour
values only affect the context of the first follow bit. A common data context is used
for all coefficients.

%% Table of context sets for signed coefficient extraction %%
\begin{table}[!ht]
\begin{tabular}{|c|c|c||l|l|}
\hline
 $parent$ & $zero\_nhood$ & $sign\_pred$ & \multicolumn{2}{c|}{\bf{Context set}} \\

% Zero parent, zero neighbour, zero sign prediction
\hline
0 & \true & 0 &  Follow & [\ZPZNFollowOne,\ZPFollowTwo,\ZPFollowThree,
                            \ZPFollowFour,\ZPFollowFive,\ZPFollowSixPlus] \\ \cline{4-5}
  &   &   &  Data & \CoeffData \\ \cline{4-5}
  &   &   &  Sign & \SignZero \\

% Zero parent, zero neighbour, -ve sign prediction
\hline
0 & \true & $<0$ &  Follow & [\ZPZNFollowOne,\ZPFollowTwo,\ZPFollowThree,
                               \ZPFollowFour,\ZPFollowFive,\ZPFollowSixPlus] \\ \cline{4-5}
  &   &    &  Data & \CoeffData \\ \cline{4-5}
  &   &    &  Sign & \SignNeg \\

% Zero parent, zero neighbour, +ve sign  prediction
\hline
0 & \true & $>0$ &  Follow & [\ZPZNFollowOne,\ZPFollowTwo,\ZPFollowThree,
                               \ZPFollowFour,\ZPFollowFive,\ZPFollowSixPlus] \\ \cline{4-5}
  &   &    &  Data & \CoeffData \\ \cline{4-5}
  &   &    &  Sign & \SignPos \\

% Zero parent, non-zero neighbour, zero sign prediction
\hline
0 & \false & 0 &  Follow & [\ZPNNFollowOne,\ZPFollowTwo,\ZPFollowThree,
                             \ZPFollowFour,\ZPFollowFive,\ZPFollowSixPlus] \\ \cline{4-5}
  &   &   &  Data & \CoeffData \\ \cline{4-5}
  &   &   &  Sign & \SignZero \\

% Zero parent, non-zero neighbour, -ve sign prediction
\hline
0 & \false & $<0$ &  Follow & [\ZPNNFollowOne,\ZPFollowTwo,\ZPFollowThree,
                                \ZPFollowFour,\ZPFollowFive,\ZPFollowSixPlus] \\ \cline{4-5}
  &        &    &  Data & \CoeffData \\ \cline{4-5}
  &        &    &  Sign & \SignNeg \\

% Zero parent, non-zero neighbour, +ve sign prediction
\hline
0 & \false & $>0$ &  Follow & [\ZPNNFollowOne,\ZPFollowTwo,\ZPFollowThree,
                                \ZPFollowFour,\ZPFollowFive,\ZPFollowSixPlus] \\ \cline{4-5}
  &        &      &  Data & \CoeffData \\ \cline{4-5}
  &        &      &  Sign & \SignPos \\

% Non-zero parent, zero neighbour, zero sign prediction
\hline
$\neq 0$ &  \true & 0 & Follow & [\NPZNFollowOne,\NPFollowTwo,\NPFollowThree,
                                    \NPFollowFour,\NPFollowFive,\NPFollowSixPlus] \\ \cline{4-5}
& &      &  Data & \CoeffData \\ \cline{4-5}
& &      &  Sign & \SignZero \\

% Non-zero parent, zero neighbour, -ve sign prediction
\hline
$\neq 0$ & \true & $<0$ &  Follow & [\NPZNFollowOne,\NPFollowTwo,\NPFollowThree,
                                      \NPFollowFour,\NPFollowFive,\NPFollowSixPlus] \\ \cline{4-5}
& &      &  Data & \CoeffData \\ \cline{4-5}
& &      &  Sign & \SignNeg \\

% Non-zero parent, zero neighbour, +ve sign prediction
\hline
$\neq 0$ & \true & $>0$ &  Follow & [\NPZNFollowOne,\NPFollowTwo,\NPFollowThree,
                                      \NPFollowFour,\NPFollowFive,\NPFollowSixPlus] \\ \cline{4-5}
& &      &  Data & \CoeffData \\ \cline{4-5}
& &      &  Sign & \SignPos \\

% Non-zero parent, non-zero neighbour, zero sign prediction
\hline
$\neq 0$ & \false & 0 &  Follow & [\NPNNFollowOne,\NPFollowTwo,\NPFollowThree,
                                    \NPFollowFour,\NPFollowFive,\NPFollowSixPlus] \\ \cline{4-5}
& &      &  Data & \CoeffData \\ \cline{4-5}
& &      &  Sign & \SignZero \\

% Zero parent, non-zero neighbour, -ve sign prediction
\hline
$\neq 0$ & \false & $<0$ &  Follow & [\NPNNFollowOne,\NPFollowTwo,\NPFollowThree,
                                       \NPFollowFour,\NPFollowFive,\NPFollowSixPlus] \\ \cline{4-5}
& &      &  Data & \CoeffData \\ \cline{4-5}
& &      &  Sign & \SignNeg \\

% Zero parent, non-zero neighbour, +ve sign prediction
\hline
$\neq 0$ & \false  & $>0$ &  Follow & [\NPNNFollowOne,\NPFollowTwo,\NPFollowThree,
                                        \NPFollowFour,\NPFollowFive,\NPFollowSixPlus] \\ \cline{4-5}
& &      &  Data & \CoeffData \\ \cline{4-5}
& &      &  Sign & \SignPos \\
\hline

\end{tabular}
\caption{Subband coefficient context sets}\label{contexttable}
\end{table}

\subsection{Low delay wavelet coefficient unpacking}\label{lowdelayparsing}

This section specifies the $low\_delay\_picture()$ process for unpacking 
wavelet coefficients in the low-delay syntax.

In low-delay operation, the Dirac syntax partitions the wavelet coefficients into a number of sets,
from all subbands, corresponding to a localised area of the picture (Figure \ref{fig:waveletslice}).
Note that it is the coefficients that are divided,
not the picture. These sets are called slices. A single quantiser, weighted by a quantisation matrix, is used for each slice, 
a slice is  of fixed dimensions within a picture and is also of fixed length in bytes. Each slice interleaves all three components, and hence
the transform parameters for the components are placed in the picture header.

In the low-delay syntax, all wavelet coefficients are entropy-coded using variable-length
codes, not arithmetic coding.

\begin{figure}[!ht]
\centering
%\includegraphics[width=0.7\textwidth]{figs/wavelet-slice.eps}
\caption{Partition of wavelet coefficients into slices}
\label{fig:waveletslice}
\end{figure}

This section specifies the syntax and coefficient unpacking operations only: picture decoding operations are as specified in Section \ref{picturedec}. However, the slice
structure implies that in practice incremental picture decoding can be easily achieved without
accumulating an entire picture data set (wavelet coefficients and motion vectors), yielding a decoding delay
proportional to the height of the slices. (The actual achievable delay may be more than one slice height because
of the extended support of wavelet filters and motion block overlaps.)

\begin{informative}
Not allowing slice bit allocations to vary within a picture does impact on
compression efficiency, but on the other hand vastly simplifies both encoder and decoder hardware. Furthermore, by
not allowing load-balancing within a buffer, it assists a chain of multiple encoders and decoders using the same
slice parameters in producing identical coding decisions and hence no cascading loss. These are all factors of great 
significance in a professional environment.
\end{informative}

Although this specification is not concerned with network adaption and carriage over data interfaces, low-delay
applications may have specific requirements for headerless and synchronous operation which may imply that
data carried ``on the wire" should differ from a straight-forward encapsulation of the stream. How this should
be done in particular cases is described in Section \ref{apps}.

\subsubsection{Overall process}
\label{ldpicture}

In the low-delay syntax, each picture is merely a concatenation of slices. 
The number of slices in a picture is determined from the padded picture dimensions
-- the padding being that required for the application of the wavelet transform to depth $\WaveletDepth$.

\begin{pseudo}{low\_delay\_transform\_data()}{}
\bsCODE{\YTransform = initialise\_wavelet\_data(Y)}{\ref{wltinit}}
\bsCODE{\COneTransform = initialise\_wavelet\_data(C1)}{\ref{wltinit}}
\bsCODE{\CTwoTransform = initialise\_wavelet\_data(C2)}{\ref{wltinit}}
\bsCODE{ph=2^{\TransformDepth}*\left\lceil\frac{\LumaHeight}{2^{\TransformDepth}}\right\rceil}
\bsCODE{pw= 2^{\TransformDepth}*\left\lceil\frac{\LumaWidth}{2^{\TransformDepth}}\right\rceil}
\bsCODE{\SlicesX=pw//\SliceWidth}
\bsCODE{\SlicesY=ph//\SliceHeight}
\bsFOR{sy=0}{\SlicesY-1}
  \bsFOR{sx=0}{\SlicesX-1}
    \bsCODE{slice(sy, sx)}{\ref{sliceparsing}}
  \bsEND
\bsEND
\bsIF{is\_intra()}
    \bsCODE{intra\_dc\_prediction(\YTransform[0][\LL])}{\ref{intradcprediction}}
    \bsCODE{intra\_dc\_prediction(\COneTransform[0][\LL])}{\ref{intradcprediction}}
    \bsCODE{intra\_dc\_prediction(\CTwoTransform[0][\LL])}{\ref{intradcprediction}}
\bsEND
\end{pseudo}

Note that the padded dimensions are the padded luma dimensions. Since these are divisible 
by $2^\TransformDepth$ they are {\em a fortiori} divisible by the slice dimensions by 
the constraints of Section \ref{sliceparams}. 

Note also that intra DC coefficient prediction
occurs across all the wavelet coefficients, and in particular DC values on the edge of a slice
are predicted from DC values in previously unpacked slices, which must be cached. 

\subsubsection{Slices}
\label{sliceparsing}

This section specifies the operation of the $slice(sy,sx)$ process for unpacking coefficients
within the slice with coordinates $(sx,sy)$. Each slice
contains coefficients from all subbands and components. Luma data is unpacked first, followed by the chroma data, which
is interleaved. A length code allows the luma and chroma coefficients each to be terminated early, with remaining values
set to zero.

The overall slice unpacking process is as follows:

\begin{pseudo}{slice}{sy,sx}
\bsCODE{slice\_bits\_left = 8*slice\_bytes(sy,sx)}{\ref{slicebytes}}
\bsCODE{qindex=read\_nbits(7)}
\bsCODE{slice\_bits\_left -= 7}
\bsCODE{slice\_quantisers(qindex)}{\ref{slicequantisers}}
\bsCODE{length\_bits = \log_2(8*slice\_bytes(sy,sx))}
\bsCODE{slice\_y\_length=read\_nbits(length\_bits)}
\bsCODE{slice\_bits\_left -= length\_bits}
\bsCODE{\ABitsLeft=slice\_y\_length}
\bsCODE{luma\_slice\_band(0,\LL,sy,sx,q_idx[0][\LL])}{\ref{lumasliceband}}
\bsFOR{level=1}{\WaveletDepth}
  \bsFOREACH{orient}{\HL,\LH,\HH}
    \bsCODE{luma\_slice\_band(level,orient,sy,sx)}{\ref{lumasliceband}}
  \bsEND
\bsEND
\bsCODE{flush\_inputb()}{\ref{blockreadbit}}
\bsCODE{slice\_bits\_left -= slice\_y\_length}
\bsCODE{\ABitsLeft=slice\_bits\_left}
\bsCODE{chroma\_slice\_band(0,\LL,sy,sx,q_idx)}{\ref{chromasliceband}}
\bsFOR{level=1}{\WaveletDepth}
  \bsFOREACH{orient}{\HL,\LH,\HH}
    \bsCODE{chroma\_slice\_band(level,orient,sy,sx)}{\ref{chromasliceband}}
  \bsEND
\bsEND
\bsCODE{flush\_inputb()}{\ref{blockreadbit}}
\end{pseudo}

\subsubsection{Slice bytes}
\label{slicebytes}

This section specifies the $slice\_bytes(sy,sx)$ function for determining the number of bytes to be used for the slice at location $(sx,sy)$:

\begin{pseudo}{slice\_bytes}{sy,sx}
\bsCODE{slice\_num=sy*\SlicesX+sx}
\bsCODE{bytes = ((slice\_num+1)*\SliceBytesNum)//\SliceBytesDenom}
\bsCODE{bytes -= ((slice\_num)*\SliceBytesNum)//\SliceBytesDenom}
\bsRET{bytes}
\end{pseudo}

\subsubsection{Setting slice quantisers}
\label{slicequantisers}

This section specifies how quantisers for individual subbands are determined from the quantisation matrix
and the quantisation index.

\begin{pseudo}{set\_band\_quantisers}{qindex}
\bsCODE{slice\_comp\_quantisers(Y,qindex)}
\bsCODE{slice\_comp\_quantisers(C1,qindex)}
\bsCODE{slice\_comp\_quantisers(C1,qindex)}
\end{pseudo}

\begin{pseudo}{slice\_comp\_quantisers}{c,qindex}
\bsCODE{\Quant[c][0][\LL]=\max(qindex-\QuantMatrix[c][0][\LL],0)}
\bsFOR{level=1}{\WaveletDepth}
    \bsFOREACH{orient}{HL, LH, HH}
        \bsCODE{qval=\max(qindex-\QuantMatrix[c][level][orient],0)}
        \bsCODE{\Quant[c][level][orient]=qval}
    \bsEND
\bsEND
\end{pseudo}

$C1$ and $C2$ quantiser index offsets are determined from the video format defaults and the picture header
as per Appendix \ref{videoformatdefaults}.

\subsubsection{Slice subbands}
\label{sliceband}

This section specifies the operation of the $luma\_slice\_band( level,orient,sy,sx)$ for 
unpacking individual luma slice subbands, and $chroma\_slice\_band( level,orient,sy,sx)$. for 
unpacking individual chroma slice subbands.

\paragraph{Slice subband area}
$\ $\newline
The rectangular set of coefficients covered by a slice component ($Y$, $C1$ and $C2$) 
is demarcated by the values $left, right, top, bottom$, defined as fractions
of the subband dimensions (Section \ref{subbandwidthheight}):
\begin{eqnarray*}
  left & = & (subband\_width(level,c)*sx)//\SlicesX \\
  right & = & (subband\_width(level,c)*(sx+1))//\SlicesX \\
  bottom & = & (subband\_height(level,c)*sy)//\SlicesY \\
  top & = & (subband\_height(level,c)*(sy+1))//\SlicesY
\end{eqnarray*}

where $c=Y$ for luma coefficients, and $C1$ or $C2$ for chroma coefficients.

\paragraph{Luma slice subband data}
\label{lumasliceband}
$\ $\newline
\begin{pseudo}{luma\_slice\_band}{level, orient, sy, sx}
\bsFOR{y=bottom}{top-1}
  \bsFOR{x=left}{right-1}
    \bsITEM{val}{sintb}{\ref{segol}}
    \bsCODE{q=\Quant[Y][level][orient]}
    \bsCODE{\YTransform[level][orient]=inverse\_quant(val,q)}
  \bsEND
\bsEND
\end{pseudo}

\paragraph{Chroma slice subband data}
\label{chromasliceband}
$\ $\newline
\begin{pseudo}{luma\_slice\_band}{level, orinet, sy, sx}
\bsFOR{y=bottom}{top-1}
  \bsFOR{x=left}{right-1}
    \bsITEM{val}{sintb}{\ref{segol}}
    \bsCODE{q1=\Quant[C1][level][orient]}
    \bsCODE{q2=\Quant[C2][level][orient]}
    \bsCODE{\COneTransform[level][orient]=inverse\_quant(val,q1)}{\ref{invquant}}
    \bsITEM{val}{sintb}{\ref{segol}}
    \bsCODE{\CTwoTransform[level][orient]=inverse\_quant(val,q2)}{\ref{invquant}}
  \bsEND
\bsEND
\end{pseudo}

\begin{comment}
\begin{informative*}
\subsection{Data adaption and application requirements}
\label{apps}
\end{informative*}
\end{comment}

\clearpage
\begin{informative*}
\section{Sequence decoding (Informative)}\label{sequencedecoding}

There is no one unique way of describing a Dirac decoder. However the pseudocode 
below is illustrative code for a sample decoder. It emphasizes which parts of the decoding
 process generate decoded output data. Note that the potential presence of padding or auxiliary data is ignored for clarity.

\begin{pseudo}{decode\_sequence}{}
   \bsCODE{\StateName=\{\}}
   \bsCODE{decoded\_pictures = \{\}}
   \bsCODE{\RefBuffer=\{\}}
   \bsCODE{parse\_info()}{\ref{parseinfoheader}}
   \bsCODE{\VideoParams = sequence\_header()}{\ref{sequenceheader}}
   \bsCODE{parse\_info()}{\ref{parseinfoheader}}
   \bsWHILE{is\_end\_of\_sequence() == \false}{\ref{parsecodevalues}}
      \bsIF{is\_seq\_header()==\true}{\ref{parsecodevalues}}
         \bsCODE{\VideoParams = sequence\_header()}{\ref{sequenceheader}}
      \bsELSEIF{is\_picture()==\true}{\ref{parsecodevalues}}
         \bsCODE{picture\_parse()}{\ref{pictureparse}}
         \bsCODE{decoded\_pictures[\PictureNumber] = picture\_decode()}{\ref{picturedec}}
      \bsEND
      \bsCODE{parse\_info()}{\ref{parseinfoheader}}
   \bsEND
   \bsRET{\{\VideoParams, decoded\_pictures\}}
\end{pseudo}

The process returns the video parameters, consisting of the essential metadata required for 
display and interpretation of the video data, and the array of decoded pictures. Each decoded picture contains the three video component data arrays together with a picture number.

The pseudocode describes the decoding process. Decoding starts by clearing the 
decoder state and the decoder output. Thus video sequences may be decoded as independent entities. The first data extracted from the Dirac stream is parse information. The parse info header indicates what type of data unit follows, and this information is stored in the decoder state. The decoder continues to read pairs of data unit and parse info 
headers until the end of the sequence is reached. The end of sequence 
is indicated by data in the final parse info header. If a data unit is a sequence header the 
decoded video parameters are updated with the information contained in the header. If the 
data unit is a picture then:
\begin{itemize}
\item the picture is parsed, then decoded
\item the picture is placed in the correct position in the output array
\end{itemize}

Note that since Dirac supports inter as well as intra picture coding, picture numbers
within the stream may not be sequential, and the decoded output pictures will not be
placed in the output buffer in order. Annex \ref{profilelevel} defines the constraints
which may be placed upon re-ordering depth.

Sequences need not be decoded from the start: decoding can start from any sequence header
according to the provisions of section \ref{randomaccess}, although some pictures might
not be (completely) decodeable due to a chain of references reaching back earlier in the stream
than the sequence header, introducing dependencies on unavailable pictures. 
The behaviour of a decoder when confronted with such pictures is application-specific.

\subsection{Non-sequential picture decoding}

The ability to decode pictures in a non-sequential manner is important for many
 applications, such as video editing. Non-sequential access means decoding a 
stream in any manner other than decoding pictures sequentially from the beginning 
of the stream to the end: this may include decoding only intra pictures, decoding backwards, or decoding pictures in random parts of the stream. 

Stream navigation, including non-sequential access is supported by the information 
in the parse info headers in the stream (section \ref{parseinfoheader}). 

In order to start decoding, other than at the start of a sequence, the decoder 
must first synchronize to the stream. The parse info prefix is present to support such synchronization. A decoder would first search for the parse info prefix to locate 
the start of a parse info header. The parse info prefix is not guaranteed to occur
 uniquely within parse info headers (the entropy coding used in 
Dirac precludes this). However, the probability of a spurious 
prefix occuring is low: 1 in $2^{32}$, since the prefix is 4 bytes long. The probability of finding 
two spurious prefix sequences separated by the value of the next (or previous) parse 
offset is 1 in $2^{64}$.  

Having synchronized with the stream the decoder now needs to locate a sequence header 
in order to find parameters needed to decode pictures. This may be done by moving back (or forward) through the stream, using the parse offsets.

The Dirac stream also supports seeking to a particular picture number, since this
is contained in each picture header.


\end{informative*}
\clearpage
\section{Picture decoding}%%%%%%%%%%%%%%%%%%%%%%%%%%%%%%%%%%%%%%%%%%%%%%%%%
% - This chapter defines the overall process  - %
% - for decoding a picture                    - % 
%%%%%%%%%%%%%%%%%%%%%%%%%%%%%%%%%%%%%%%%%%%%%%%%%

\label{picturedec}

This section specifies the process for decoding pictures from the Dirac stream. Picture decoding depends upon
correctly parsing the Dirac bitstream, and decoding operations are dependent upon the parsing operations
set out in Sections \ref{streamstructure}, \ref{motiondec} and \ref{transformdec}.

This section does not specify how pictures are encoded, nor how pictures are reordered and presented for 
display, which is described in Section \ref{profilelevel}. 

\subsection{Introduction}

Dirac supports both intra and inter picture coding, with forward and backward prediction. This means that
pictures may be encoded in the stream in non-display order: reordering pictures will be required in order
to display them correctly, and  decoded picture buffer will be necessary to store pictures while temporally 
prior pictures are decoded. Note that the core Dirac specification does not encompass the operation of the
decoded picture buffer: this is specified in conjunction with the level and profile values extracted from
the stream (Section \ref{parseparameters}), in Appendix \ref{profilelevel}. 

Decoded pictures may, however, be reference pictures, used for the prediction of subsequent pictures
in the Dirac stream. Reference pictures are stored in a reference picture buffer $\RefBuffer$. The operation
of $\RefBuffer$ does form part of the core Dirac specification, and the rules for management of the
buffer are set out in Section \ref{refbuffer}.

\subsection{Decoding sequences}
\label{sequencedecoding}

The process for decoding a picture sequence (in coded order) is as follows:

\begin{pseudo}{decode\_sequence}{}
   \bsCODE{\SeqStateName = \{\}}
   \bsCODE{video\_params = \{\}}
   \bsCODE{decoded\_pictures = \{\}}
   \bsCODE{\RefBuffer=\{\}}
   \bsCODE{\RetiredPictureList=\{\}}
   \bsCODE{parse\_info()}{\ref{parseinfo}}
   \bsWHILE{is\_end\_of\_sequence() == \false}
      \bsIF{is\_AU()==\true}
         \bsCODE{video\_params = access\_unit\_header()}{\ref{auheader}}
      \bsELSEIF{is\_picture()==\true}
         \bsCODE{picture\_parse()}{\ref{pictureparse}}
         \bsCODE{decoded\_pictures[\PictureNumber] = picture\_decode()}{\ref{picturedecprocess}}
         \bsCODE{ref\_buffer\_remove()}{\ref{refbuffer}}
         \bsIF{is\_ref()}
           \bsCODE{ref\_buffer\_add()}{\ref{refbuffer}}
         \bsEND
         \bsCODE{offset\_output\_data(decoded\_pictures[\PictureNumber])}{\ref{videooutput}}
      \bsEND
      \bsCODE{parse\_info()}{\ref{parseinfo}}
   \bsEND
   \bsRET{\{video\_params, decoded\_pictures\}}
\end{pseudo}

The process returns the video parameters, consisting of the essential metadata required for 
display and interpretation of the video data, and the array of decoded pictures. Each decoded
picture contains the three video component data arrays together with a picture number.

The pseudocode describes the decoding process. Decoding starts by clearing the decoder state 
and the decoder output. Thus video sequences may be decoded as independent entities. The 
first data extracted from the Dirac stream is parse information. Parse Info indicates what type of 
Data Unit follows, and this information is stored in the decoder state. The decoder continues to read
 pairs of Data Unit and Parse Info until the end of the sequence is reached. The end of sequence 
is indicated by data in the final Parse Info header. If a Data Unit is an Access Unit Header the 
decoded video parameters are updated with the information contained in the header. If the 
Data Unit is a Picture then:
\begin{itemize}
\item the picture is parsed, then decoded
\item the picture is placed in the correct position in the output array
\item the reference picture buffer is managed by deleting obsolete pictures and, if the current picture
is a reference, adding it to the reference buffer
\end{itemize}

Note that for clarity this code ignores the presence of Auxiliary Data and Padding Data in the sequence. 
Nor does it illustrate providing the picture numbers, which are coded in the stream, nor details of the 
coding parameters, which may be required by some applications.

Note also that various operations, such as editing, may result in a discontinuity of picture number 
values between sequences within a Dirac stream, even taking into account picture re-ordering.

\subsection{Reference picture buffer management}
\label{refbuffer}

This section specifies how the Dirac stream data is used to manage the reference 
picture buffer $\RefBuffer$. The reference picture buffer has a maximum size of
$\RefBufferSize$ elements, as set in the applicable level (Appendix \ref{profilelevel}).

The $ref\_picture\_remove()$ process operates as
follows:

\begin{pseudo}{ref\_picture\_remove}{}
\bsFOR{i=0}{\length(\RetiredPictureList)-1}
    \bsCODE{n=\RetiredPictureList[i]}
    \bsFOR{k=0}{\RefBufferSize-1}
       \bsIF{\RefBuffer[k][pic\_number]==n}
            \bsFOR{j=k}{\RefBufferSize-1}
                \bsCODE{\RefBuffer[j]=\RefBuffer[j+1]}
            \bsEND
        \bsEND
    \bsEND
\bsEND
\bsCODE{\RetiredPictureList=\emptyset}
\end{pseudo}

The $get\_ref(n)$ function returns the (first) reference picture in the buffer with 
picture number $n$.  

The $ref\_picture\_add()$ process for adding pictures to the reference picture
buffer proceeds according to the following rules:

{\bf Case 1.} If the reference picture buffer is not full i.e. has fewer than $\RefBufferSize$ elements,
then add the $\CurrentPicture$ data to the end of the buffer. 

{\bf Case 2.} If the reference picture is full i.e. it has $\RefBufferSize$ elements, then remove the
first (i.e. oldest) element of the buffer, $\RefBuffer[0]$, set
\[\RefBuffer[i] = \RefBuffer[i+1] \]
for $i=0$ to $\RefBufferSize-2$, and set the last element $\RefBuffer[\RefBufferSize-1]$ equal to
a copy of $\CurrentPicture$.

\subsection{Video output ranges}
\label{videooutput}

Video output data ranges are deemed to be non-negative, so that the offset and excursion values 
may be applied by subsequent processing. Since decoded video data is bipolar, it must be 
suitably offset before output:

\begin{pseudo}{offset\_output\_data}{picture\_data}
\bsFOREACH{c}{Y, C1, C2}
    \bsIF{c==Y}
        \bsCODE{\BitDepth=\LumaDepth}
    \bsELSE
        \bsCODE{\BitDepth=\ChromaDepth}
    \bsEND
    \bsCODE{comp=picture\_data[c]}
    \bsFOR{y=0}{\height{comp}-1}
        \bsFOR{x=0}{\width{comp}-1}
            \bsCODE{comp[y][x]+=2^{\BitDepth-1}}
        \bsEND
    \bsEND
\bsEND
\end{pseudo}

\subsection{Random access}
\label{randomaccess}

Many applications involve random access of a Dirac sequence. In the context of this specification, this
means beginning reading and parsing data from the beginning of an Access Unit header, located
at some point not at the beginning of a sequence. In such circumstances, the process of Section 
\ref{sequencedecoding} is followed, except that parsing begins from the AU header and pictures
that cannot be decoded, since they use as references pictures not previously accessible or decoded, are discarded.
Clearly, what is decodeable depends upon the point at which the sequence has been accessed.

Compliant Dirac streams shall be so constructed that after parsing and partially decoding one whole
access unit, all pictures in subsequent access units may be fully decoded. Hence the placing of
an AU header within the stream constitutes a guarantee of ultimate decodeability.

\begin{informative*}
\subsection{Non-sequential picture decoding (Informative)}

The ability to decode pictures in a non-sequential manner is important for many applications, 
such as video editing. Non-sequential access means decoding a stream in any manner other 
than decoding pictures sequentially from the beginning of the stream to the end: this may
include decoding only intra pictures, decoding backwards, or decoding pictures in random
parts of the stream. Non-sequential 
picture access is outside the scope of this specification. Nevertheless the Dirac stream has 
been designed to support this feature. This section provides informative notes on this aspect 
of the Dirac stream specification.

Stream navigation, including non-sequential access is supported by the information in the Parse 
Info headers in the stream. Details of Parse Info headers are defined in Section \ref{parseinfo}. However, 
in order to discuss stream navigation, it is necessary to indicate the information contained within 
a Parse Info header. A Parse Info headers contains four components: a Parse Info Prefix, a 
Parse Code, a Next Parse Offset and a Previous Parse Offset.  The Parse Info Prefix is a fixed 
4 byte sequence. The Parse Code specifies the type of the following Data Unit. The Next and 
Previous Parse Offsets specify the number of bytes until the next, or from the previous, Parse 
Info header.

In order to start decoding, other than at the start of a stream, the decoder must first synchronize 
to the stream. The Parse Info Prefix is present to support such synchronization. A decoder would 
first search for the Parse Info Prefix to locate the start of a Parse Info header. The Parse Info 
Prefix is not guaranteed to occur uniquely within Parse Info headers (the entropy coding used in 
Dirac precludes this), the Parse Info Prefix may, by chance, occur within a Data Unit. The decoder 
may read the Next or Previous Parse Info Offsets to confirm that an occurrence of the Parse Info 
Prefix corresponds to a Parse Info header.

When the decoder finds a Parse Info Prefix it may skip forward or back by the value of the appropriate offset
 and check whether the next four bytes are again those of the Parse Info Prefix. If so the 
decoder can be reasonably certain that it has found a Parse Info Header. The probability of a spurious 
prefix occuring is low: 1 in $2^{32}$, since the prefix is 4 bytes long. The probability of finding 
two spurious Prefix sequences separated by the value of Next Parse Offset is 1 in $2^{64}$. The test outlined is, therefore, 
more than adequate in practice. 

Having synchronized with the stream the decoder now needs to locate an Access Unit Header in 
order to find parameters needed to decode pictures. This may be done by moving forward (or backward) 
through the stream, between successive Parse Info block, using Next (or Previous) Parse Offsets, 
until a Parse Info header is found containing the Parse Code for an Access Unit Header. Decoding 
may now commence, as if from the beginning of the stream.

The Dirac stream also supports seeking to a particular picture number. The first four bytes of either 
an Access Unit Header or a Picture contain a 4 byte picture number. Picture numbers provide an
 identifier for each picture with a sequence. So, to find a particular picture, the decoder may 
move forward or backward in the stream, using the offsets in the Parse Info headers, until the 
correct picture number is reached. A picture may be decoded once the parameters within an 
Access Unit Header, in the same sequence, have been read. Picture decoding depends upon decoding
all references upon which a picture depends, and all references upon which they {\em they} depend
and so on. The requirements of Section \ref{randomaccess} imply that this will be at most the
contents of two access units, and with MPEG-style Group of Picture structuring will normally
be much less.
\end{informative*}

\subsection{Overall picture decoding process}
\label{overallpicturedec}

\subsubsection{Picture data initialisation}
\label{picdataconventions}

Picture data from the current picture being decoded is stored in the $\CurrentPicture$ state
variable, which is a structure with indices $pic\_num$, $Y$, $C1$ and $C2$ representing
luma and chroma data (typically Y, Cb and Cr, although other formats are supported too -- see
Appendix \ref{vidsys}).


The $init\_picture\_data()$ initialises the current picture data so that:
\begin{itemize}
\item $\CurrentPicture[pic\_num]=\PictureNumber$
\item $\CurrentPicture[Y]$ is a 2-dimensional array of width $\LumaWidth$ and height $\LumaHeight$, 
all values $\CurrentPicture[Y][y][x]$ set to 0
\item $\CurrentPicture[C1]$ and $\CurrentPicture[C2]$ are 2-dimensional arrays of width $\ChromaWidth$ and height $\ChromaHeight$, 
all values $\CurrentPicture[C1][y][x]$ and $\CurrentPicture[C2][y][x]$ set to 0
\end{itemize}

\subsubsection{Decoding process}
\label{picturedecprocess}

The process for decoding a picture within a Dirac sequence can commence 
when: 
\begin{itemize}
\item an Access Unit header has been located, and parsed, and the default parameters set
\item a picture data unit has been located and parsed, overriding default parameters
\item any reference pictures for the current picture have been decoded
\end{itemize}

At this point the reference picture buffer shall be initialised with no 
reference pictures and picture data initialised as per Section \ref{picdataconventions}.

This initialisation process need only occur once within a sequence, since
apart from the AU picture number, all AU headers within a sequence are
identical. 

After this initialisation process, decoding a picture with picture number 
$n$ in a Dirac sequence consists of:

\begin{pseudo}{picture\_decode}{}
\bsCODE{init\_picture\_data()}{\ref{picdataconventions}}
\bsIF{\ZeroResidual==\false}
    \bsCODE{\CurrentPicture[Y]=idwt(\YTransform)}{\ref{idwt}}
    \bsCODE{\CurrentPicture[C1]=idwt(\COneTransform)}{\ref{idwt}}
    \bsCODE{\CurrentPicture[C2]=idwt(\CTwoTransform)}{\ref{idwt}}
\bsEND
\bsIF{is\_inter()}
    \bsCODE{ref1=get\_ref(\RefOneNum)}
    \bsIF{num\_refs()==2}{\ref{parseinfo}}
        \bsCODE{ref2=get\_ref(\RefTwoNum)}
    \bsEND
    \bsCODE{motion\_compensate(ref1[Y], ref2[Y],  \CurrentPicture[Y], c)}{\ref{motioncompensate}}
    \bsCODE{motion\_compensate(ref1[C1], ref2[C1],  \CurrentPicture[C1], c)}{\ref{motioncompensate}}
    \bsCODE{motion\_compensate(ref1[C2], ref2[C2],  \CurrentPicture[C2], c)}{\ref{motioncompensate}}
\bsEND
\bsCODE{clip\_picture()}{\ref{pictureclip}}
\bsRET{\CurrentPicture}
\end{pseudo}

When randomly accessing a sequence, a picture may not be decodeable because reference pictures
may not be available in the buffer. In this case the current picture may be discarded, although some
decoders may be designed to produce an output. 

A Dirac sequence shall be so constructed so that if
pictures are decoding commences from the beginning of the stream and pictures are decoded in 
stream order, there shall be no undecodeable pictures i.e. the reference pictures associated with
any picture in the sequence shall have occurred prior to that picture in the sequence.

Picture numbers within the stream may not be in numerical order, and subsequent reordering may be
required: the size of the decoded picture buffer required to perform any such reordering is specified
as part of the application profile and level (Appendix \ref{profilelevel}).

\subsection{Inverse discrete wavelet transform}
%%%%%%%%%%%%%%%%%%%%%%%%%%%%%%%%%%%%%%%%%%%%%%%%%%%%%
% - This chapter defines how the inverse discrete - %
% - wavelet transform is done                     - % 
%%%%%%%%%%%%%%%%%%%%%%%%%%%%%%%%%%%%%%%%%%%%%%%%%%%%%

\label{idwt}

\input{idwt-intro}
\subsection{IDWT synthesis operation}\input{idwt-synthesis}
\subsection{Removal of IDWT pad values}\input{idwt-padremoval}
\subsection{Vertical and horizontal synthesis}\input{idwt-vhsynthesis}
\subsection{Interleaving}\input{idwt-interleaving}
\subsection{One-dimensional synthesis}\input{idwt-1dsynthesis}
\subsection{Integer lifting}\input{idwt-lifting}
\subsection{avaliable filters}\input{idwt-filters}


\subsection{Motion compensation}
\label{motioncompensate}

This section defines the operation of the process
$motion\_compensate(ref1, ref2,  pic, c)$ for motion-compensating a
picture component array  $pic$ of type $c=Y, U$ or $V$ from reference 
component arrays $ref1$ and $ref2$ of the same type.

This process shall be invoked for each component in a picture, subsequent to the 
decoding of coefficient data, specified in Section \ref{transformdec}, and the Inverse Wavelet Transform (IWT), specified in Section \ref{idwt}. 

Motion compensation shall use the motion block data $\BlockData$ and optionally may use the
global motion parameters $\GlobalParams$.

\begin{informative*}
\subsubsection{Overlapped Block Motion Compensation (OBMC) (Informative)}

Motion compensated prediction methods provide methods for determining 
predictions for pixels in the current picture by using motion vectors to 
define offsets from those pixels to pixels in previously decoded
pictures. Motion compensation techniques vary in how those pixels are grouped
together, and how a prediction is formed for pixels in a given group. In 
conventional  block motion compensation, as used in MPEG2, H.264 and many other
codecs, the picture is divided into {\em disjoint} rectangular blocks and the
motion vector or vectors associated with that block defines the offset(s) into
the reference pictures.

In OBMC, by contrast, the predicted picture is divided into a regular overlapping 
blocks of dimensions $xblen$ by $yblen$ that cover at least the entire picture 
area as shown in figure \ref{fig:blockcoverage}.  Overlapping is ensured by starting
each block at a horizontal separation $xbsep$ and a vertical separation $ybsep$ 
from its neighbours, where these values are less than the corresponding block dimensions.
\end{informative*}

\begin{figure}[!ht]
\centering
\includegraphics[width=0.7\textwidth]{figs/block-coverage.eps}
\caption{Block coverage of the predicted picture}
\label{fig:blockcoverage}
\end{figure}

\begin{informative*}
The overlap (offset) between blocks horizontally is $xblen - xbsep$ and vertically is
$yblen - ybsep$. As a result pixels in the overlapping areas lie in more than
one block, and so more than one motion vector set (and set of associated predictions)
applies to them. Indeed, a pixel may have up to eight predictions, as it may belong to
up to four blocks, each of which may have up to two motion vectors. These are combined
into a single prediction by using weights, which are so constructed so as to sum to 1. In the
 Dirac integer implementation, fractional weights are achieved by insisting that weights sum 
to a power of 2, which is then shifted out once all contributions have been summed.

In Dirac blocks are positioned so that blocks will overspill the left and top edges by 
($xoffset$) and ($yoffset$) pixels.  The number of blocks has been
determined (Section \ref{}) so that the picture area is wholly covered, and the overspill
 on the right hand and bottom edges will be at least the amount on the left and top edges. 
Indeed, the number of blocks has been set so that the blocks divide into whole superblocks
(sets of 4x4 blocks), which mean that some blocks may fall entirely out of the picture area. 
 Any predictions for pixels outside the actual picture area are discarded.

\end{informative*}

\subsubsection{Overall motion compensation process}
\label{mcprocess}

The motion compensation process shall form an integer prediction for each pixel in 
the predicted picture component $pic$, which shall be added to the pixel value, and
 then clipped to keep it in range.

The $motion\_compensate()$ process is defined by means of a temporary data
array $mc\_tmp$ for storing the motion-compensated prediction for the 
current picture. 

The $motion\_compensate()$ shall be defined as follows:

\begin{pseudo}{motion\_compensate}{ref1, ref2,  pic, c}
\bsIF{c==Y}
    \bsCODE{bit\_depth=\LumaDepth}
\bsELSE
    \bsCODE{bit\_depth=\ChromaDepth}
\bsEND
\bsCODE{init\_dimensions(c)}{\ref{mcdimensions}}
\bsCODE{mc\_tmp=init\_temp\_array()}{\ref{mctemparray}}
%\bsCODE{total\_wt\_bits=set\_mc\_wt()}{\ref{wtbits}}
\bsFOR{j=0}{\BlocksY-1}
    \bsFOR{i=0}{\BlocksX-1}
        \bsCODE{block\_mc(mc\_tmp,i,j)}{\ref{blockmc}}
    \bsEND
\bsEND
\bsFOR{y=0}{\LenY-1}
    \bsFOR{x=0}{\LenX-1}
%        \bsCODE{pic[y][x] += (mc\_tmp[y][x]+2^{total\_wt\_bits-1})\gg total\_wt\_bits}
       \bsCODE{pic[y][x] += (mc\_tmp[y][x]+32)\gg 6}
        \bsCODE{pic[y][x] = \clip(pic[y][x], -2^{bit\_depth-1}, 2^{bit\_depth-1}-1)}
    \bsEND
\bsEND
\end{pseudo}

\begin{informative}
Six bits are used for the overlapped-block weighting matrix. This ensures that 10-bit
data may normally be motion compensated using 16-bit words as per Section \ref{blockmc}.
\end{informative}

\subsubsection{Dimensions}
\label{mcdimensions}
Since motion compensation shall apply to both luma and (potentially subsampled)
chroma data, for simplicity a number of variables are defined by the 
$init\_dimensions()$ function, which is as follows:

\begin{pseudo}{init\_dimensions}{c}
\bsIF{c==Y}
   \bsCODE{\LenX=\LumaWidth}
   \bsCODE{\LenY=\LumaHeight}
   \bsCODE{\XBlen=\LumaXBlen}
   \bsCODE{\YBlen=\LumaYBlen}
   \bsCODE{\XBsep=\LumaXBsep}
   \bsCODE{\YBsep=\LumaYBsep}
\bsELSE
   \bsCODE{\LenX=\ChromaWidth}
   \bsCODE{\LenY=\ChromaHeight}
   \bsCODE{\XBlen=\ChromaXBlen}
   \bsCODE{\YBlen=\ChromaYBlen}
   \bsCODE{\XBsep=\ChromaXBsep}
   \bsCODE{\YBsep=\ChromaYBsep}
\bsEND
\bsCODE{\XOffset = (\XBlen-XBsep)//2}
\bsCODE{\YOffset = (\YBlen-YBsep)//2}
\end{pseudo}

\begin{informative}
The subband data that makes up the IWT coefficients is padded in order that the IWT
may function correctly. For simplicity, in this specification, padding data is removed
after the IWT has been performed so that the picture data and reference data arrays have
the same dimensions for motion compensation. However, it may be more efficient to 
perform all operations prior to the output of pictures using padded data, i.e. to discard
 padding values subsequent to motion compensation. Such a course of action is equivalent,
 so long as it is realised that blocks must be regarded as edge blocks if they overlap the
 actual picture area, not the larger area produced by padding.
\end{informative}

\subsubsection{Initialising the motion compensated data array}
\label{mctemparray}

The $init\_temp\_array()$ function shall return a two-dimensional data array with
horizontal size $\LenX$ and vertical size $\LenY$, such that each element of the two dimensional array shall be set to zero.

\begin{comment}
\subsubsection{Weighting bits}
\label{wtbits}


[Can omit this section if spatial weights are just set to 6 bits.]

The function $set\_mc\_wts()$ shall set the total number of extra bits of resolution
that shall be added to motion compensation data as a result of applying weighting
matrices and picture weights. It shall be defined as follows:

\begin{pseudo}{set\_mc\_wts}{}
\bsCODE{hbits  = \intlog2(\XOffset)+2}
\bsCODE{vbits  = \intlog2(\YOffset)+2}
\bsRET{hbits+vbits+\RefsWeightPrecision}
\end{pseudo}

\begin{informative}
This is the number of bits added to pixel values in order to perform OBMC 
reversibly with integer arithmetic using the spatial matrix specified in Sections 
\ref{mcspatialweights} and the reference weights extracted in
parsing the picture prediction header data (Section \ref{refpicweights}).

Note that  the number of motion compensation bits depends upon the 
block sizes -- specifically
the block overlaps - selected. A Dirac decoder level (Section \ref{profilelevel}) 
specifies the maximum block overlaps allowable, and hence 
the word widths necessary for processing OBMC. If we assume that 
the picture weights are complementary (i.e. the weights
for reference 1 and reference 2 sum to $2^\RefsWeightPrecision$, 
then the number of bits required for performing motion compensation 
calculations is \[bit\_depth+total\_wt\_bits+\RefsWeightPrecision\]
unsigned bits. 8 bit video data encoded with block overlaps of 4 
luminance pixels and the standard picture weights therefore
requires 8+3+3+1=15 unsigned bits. The additional bit within 
a 16 bit word could be used to provide additional reference 
weighting.
\end{informative}
\end{comment}

\subsubsection{Motion compensation of a block}
\label{blockmc}

This section defines the $block\_mc()$ process for motion-compensating a single
block.

Each block shall be motion-compensated by applying a weighting matrix to a block prediction and adding the weighted prediction into the motion-compensated 
prediction array. 

The $block\_mc()$ process shall be defined as follows:

\begin{pseudo}{block\_mc}{mc\_pred,i,j}
\bsCODE{xstart = i*\XBsep-\XOffset}
\bsCODE{ystart = j*\XBsep-\XOffset}
\bsCODE{xstop = (i+1)*\XBsep+\XOffset}
\bsCODE{ystop = (j+1)*\YBsep+\YOffset}
\bsCODE{mode=\BlockData[j][i][mode]}
\bsCODE{W=spatial\_wt(i,j)}{\ref{pixelprediction}}
\bsFOR{y=\max(ystart,0)}{\min(ystop,\LenY)-1}
    \bsFOR{x=\max(xstart,0)}{\min(xstop,\LenX)-1}
        \bsCODE{p=x-xstart}
        \bsCODE{q=y-ystart}
        \bsIF{mode==\Intra}
            \bsCODE{val=\BlockData[j][i][dc][c]}
        \bsELSEIF{mode==\RefOneOnly}
            \bsCODE{val=pixel\_pred(ref1, 1, i, j, x, y, c)}{\ref{pixelprediction}}
            \bsCODE{val*=\RefOneWeight+\RefTwoWeight}
            \bsCODE{val=(val+2^{\RefsWeightPrecision-1})\gg\RefsWeightPrecision}
        \bsELSEIF{mode==\RefTwoOnly}
            \bsCODE{val=pixel\_pred(ref2, 2, i, j, x, y, c)}{\ref{pixelprediction}}
            \bsCODE{val*=\RefOneWeight+\RefTwoWeight}
            \bsCODE{val=(val+2^{\RefsWeightPrecision-1})\gg\RefsWeightPrecision}          
        \bsELSEIF{mode==\RefOneAndTwo}
            \bsCODE{val1=pixel\_pred(ref1, 1, i, j, x, y, c)}{\ref{pixelprediction}}
            \bsCODE{val1*=\RefOneWeight}
            \bsCODE{val2=pixel\_pred(ref2, 2, i, j, x, y, c)}{\ref{pixelprediction}}
            \bsCODE{val2*=\RefTwoWeight}
            \bsCODE{val=(val1+val2+2^{\RefsWeightPrecision-1})\gg\RefsWeightPrecision}
        \bsEND
        \bsCODE{val *= W[q][p]}
        \bsCODE{mc\_tmp[y][x]+=val}
    \bsEND
\bsEND
\end{pseudo}

\begin{informative}
Note that if the two reference weights are 1 and $\RefsWeightPrecision$ is 1, then
reference weighting is transparent and
\begin{pseudo*}
\bsCODE{val=pixel\_pred(ref1, 1, i, j, x, y, c)}
\bsCODE{val*=\RefOneWeight+\RefTwoWeight}
\bsCODE{val=(val+2^{\RefsWeightPrecision-1})\gg\RefsWeightPrecision}
\bsCODE{\ldots}
\end{pseudo*}

reduces to

\begin{pseudo*}
\bsCODE{val=pixel\_pred(ref1, 1, i, j, x, y, c)}
\bsCODE{\ldots}
\end{pseudo*}

In this case, therefore, the normal reference weighting produces no additional dynamic
range for internal processing and 10 bit video can be motion compensated with 16 bit
unsigned internal values.

In general, however, the worst case internal bit widths consist of the video bit depth plus the maximum of: 6 (the spatial matrix bit width) and the value of $\RefsWeightPrecision$. 6 bits
should be sufficient for most fading compensation applications, and so 16 bit internals will
suffice for all practical motion compensation scenarios for 8 and 10 bit video.
\end{informative}


\subsubsection{Spatial weighting matrix}

\label{mcspatialweights}

This section specifies the function $spatial\_wt(i,j)$ for deriving the 6-bit spatial weighting 
matrix that shall be applied to the block with coordinates  $(i,j)$. 

Note that other weights shall be applied to the prediction as a result of the 
weights applied to each reference.

The same weighting matrix shall be returned for all blocks within the interior
of the picture component array. Suitably modified weighting matrices shall
be returned for blocks at the edges of the picture component data array.

The function shall return a two-dimensional spatial weighting matrix. This
shall apply a linear roll-off in both horizontal and vertical directions.

The spatial matrix returned shall be the product of a horizontal and a vertical
weighting matrix. It shall be defined as follows:

\begin{pseudo}{spatial\_wt}{i,j}
\bsFOR{y==0}{\YBlen-1}
    \bsFOR{x==0}{\XBlen-1}
        \bsCODE{W=h\_wt(i)[x]*v\_wt(j)[y]}
    \bsEND
\bsEND
\bsRET{W}
\end{pseudo}

The horizontal weighting function shall be defined as follows:

\begin{pseudo}{h\_wt}{i}
\bsFOR{x=0}{2*\XOffset-1}
    \bsCODE{hwt[x]=1+(6*x+\XOffset-1)//(2*\XOffset-1)}
    \bsCODE{hwt[x+\XBsep]=8-hwt[x]}
\bsEND
\bsFOR{2*\XOffset}{\XBsep-1}
    \bsCODE{hwt[x]=8}
\bsEND
\bsIF{i==0}
    \bsFOR{x=0}{2*\XOffset-1}
        \bsCODE{hwt[x]=8}
    \bsEND
\bsELSEIF{i==\BlocksX-1}
    \bsFOR{x=0}{2*\XOffset-1}
        \bsCODE{hwt[x+\XBsep]=8}
    \bsEND
\bsEND
\bsRET{hwt}
\end{pseudo}

The vertical weighting function  shall be defined as follows:

\begin{pseudo}{v\_wt}{j}
\bsFOR{y=0}{2*\YOffset-1}
    \bsCODE{vwt[y]=1+(6*y+\XOffset-1)//(2*\YOffset-1)}
    \bsCODE{vwt[y+\XBsep]=8-vwt[y]}
\bsEND
\bsFOR{2*\YOffset}{\YBsep-1}
    \bsCODE{xwt[y]=8}
\bsEND
\bsIF{j==0}
    \bsFOR{y=0}{2*\YOffset-1}
        \bsCODE{vwt[y]=8}
    \bsEND
\bsELSEIF{j==\BlocksY-1}
    \bsFOR{y=0}{2*\YOffset-1}
        \bsCODE{vwt[x+\YBsep]=8}
    \bsEND
\bsEND
\bsRET{vwt}
\end{pseudo}

\begin{informative}
The horizontal and vertical weighting arrays satisfy the perfect reconstruction property across block overlaps by construction:
\begin{eqnarray*}
hwt[x+\XBsep] & = & 8 - hwt[x] \\ 
vwt[y+\YBsep] & = & 8 - vwt[y]  
\end{eqnarray*}

In addition, it can be shown they are always symmetric (except at picture edges), or
equivalently the leading edges have skew-symmetry about the half-way point:
\begin{eqnarray*}
hwt[\XBlen-1-x] & =  & hwt[x] \\
vwt[\YBlen-1-y] & = & vwt[y] 
\end{eqnarray*}

The horizontal and vertical weighting matrix components for various block
 overlaps are shown in Table \ref{table:leadingedges}. 
These encompass all the default values listed 
in Table \ref{} for both luma and chroma.
\end{informative}
\begin{table}[!ht]
\centering
\begin{tabular}{|c|c|c|}
\hline
\rowcolor[gray]{0.75}\bf{Overlap}  & \bf{Offset} & \bf{Leading edge} \\
\rowcolor[gray]{0.75}\bf{(length-separation)} & & \\
\hline
2 & 1 & 1,7\\
\hline
4 & 2 & 1,3,5,7\\
\hline
8 & 4 & 1,2,3,4,4,5,6,7\\
\hline
16 & 8 & 1,1,2,2,3,3,3,4,4,5,5,5,6,6,7,7 \\
\hline
\end{tabular}
\caption{Leading and trailing edge values for different block overlaps}
\label{table:leadingedges}
\end{table}

\begin{comment}
The profile of the matrix 
for interior blocks is illustrated in Figure \ref{fig:weightprofile}.

\begin{figure}[!ht]
\centering
\includegraphics[width=0.7\textwidth]{figs/obmc-profile}
\caption{Profile of overlapped-block motion compensation matrix}
\label{fig:weightprofile}
\end{figure}

\begin{informative*}
\subsubsection{Reference weights and fade prediction (Informative)}

The reference prediction weights used for each prediction mode for 
block prediction (Section \ref{blockmc}) may appear 
confusing. It is helpful
to think of two cases for using reference picture weighting. The first is interpolative 
prediction, where the picture being predicted is, for example, a cross-fade and is
closely approximated by some mixture of the reference pictures:
 $P\backsimeq\delta R_1+(1-\delta)R_2$. Here the weights we'd like to
use for each frame prediction add up to 1 (or $2^\RefsWeightPrecision$ 
for integer weights). 
The second case is scaling prediction, where 
the weights we'd like to use for the frame predictions don't add up to 1: for example,
a fade to or from black
$P\backsimeq\delta_1 R_1$ and $P\backsimeq\delta_2 R_2$. It is not possible to choose 
weights for each prediction mode which will be optimal both cases. The weighting
factors chosen will give work with interpolative prediction (which is more common) 
but are not perfect for scaling prediction. It would have been possible to create a variety of
prediction modes to cover all cases, however the potential savings do not justify the
additional complexity.

For interpolative prediction, all data in the current picture will be of commensurate scale to
that of the references. In forming the bi-directional prediction, a value 
$W_1 p_1 + W_2 p2_2$ is 
formed, so the prediction has "scale" $W_1+W_2$. $W_1+W_2$ is 
therefore the weighting value used to scale unidirectional prediction, in order to provide
predictions of commensurate order. The unity weighting value 
$2^\RefsWeightPrecision$ is used
for DC blocks as this gives the best prediction, and in the interpolative case 
this equals $W_1+W_2$
so all predictions are of the same order.

The weighting factors we would like to use for unidirectionally 
redicted blocks in the scaling case
are $2W_1$ and $2W_2$ - the factor 2 takes into account that 
we're only adding in one prediction
value as against two for bidirectional prediction. These factors differ f
rom $W_1+W_2$, and hence
unidirectional prediction is incorrect when there are two references. 
Note, however, that we can
still perform prediction with the correct scaling values when we 
only have a single reference. Note
also that the value of $W_1+W_2$ was selected instead of 
$2^\RefsWeightPrecision$, which
would be equivalent in the interpolative case, as it gives a 
better approximation when the
weights do not sum to $2^\RefsWeightPrecision$.
\end{informative*}
\end{comment}

\subsubsection{Pixel prediction}
\label{pixelprediction}

This section defines the operation of the $pixel\_pred(ref, ref\_num, i, j, x, y, c)$ 
process which shall be used for forming the prediction for a pixel 
with coordinates $(x,y)$ in component $c$, belonging to the block with coordinates $(i,j)$.

The pixel prediction process shall consist of two stages. In the first stage, a motion vector
 to be applied to pixel $(x,y)$ shall be derived. For block motion, this shall be a block
 motionvector that shall apply to all pixels in a block. For global motion the motion
vector shall be computed from the global motion parameters and may vary pixel-by-pixel.

In the second stage, the motion vector shall be used to derive coordinates in an reference picture.

\begin{pseudo}{pixel\_pred}{ref,ref\_num,i,j,x,y,c}
\bsIF{\BlockData[j][i][global]==\false}
  \bsCODE{mv= \BlockData[j][i][ref]}
\bsELSE
  \bsCODE{mv=global\_mv(ref, ref\_num, x, y, c)}{\ref{globalmv}}
\bsEND
\bsIF{c!=Y}
  \bsCODE{mv = chroma\_mv\_scale(mv)}{\ref{chromamvscale}}
\bsEND
\bsCODE{px = (x\ll \MotionVectorPrecision)+mv[0]}
\bsCODE{py = (y\ll \MotionVectorPrecision)+mv[1]}
\bsIF{\MotionVectorPrecision>0}
  \bsRET{subpel\_predict(ref, c, px, py))}{\ref{upconvert}}
\bsELSE
  \bsRET{ref[\clip(py,0,\height(ref)-1)][\clip(px,0,\width(ref)-1)]}
\bsEND
\end{pseudo}

\subsubsection{Global motion vector field generation}
\label{globalmv}

This section specifies the operation of the $global\_mv(ref, ref\_num, x,y, c)$ process
for deriving a global motion vector for a pixel at location $(x,y)$, in a component of 
type $c$ from a reference $ref$.

The function shall be defined as follows:

\begin{pseudo}{global\_mv}{ref, ref\_num, x,y, c}
\bsCODE{ez  =  \GlobalParams[ref\_num][ZRS\_exp]}
\bsCODE{ep  =  \GlobalParams[ref\_num][perspective\_exp]}
\bsCODE{b=\GlobalParams[ref\_num][pan\_tilt]}
\bsCODE{A=\GlobalParams[ref\_num][ZRS]}
\bsCODE{m=2^{ep}-(c[0]*x+c[1]*y)}
\bsCODE{v[0]=m*((A[0][0]*x+A[0][1]*y)+2^{ez}*b[0])}
\bsCODE{v[1]=m*((A[1][0]*x+A[1][1]*y)+2^{ez}*b[0])}
\bsCODE{v[0] = (v[0]+(1\ll(ez+ep)) )\gg (ez+ep)}
\bsCODE{v[1] = (v[1]+(1\ll(ez+ep)) )\gg (ez+ep)}
\bsRET{v}
\end{pseudo}

\begin{informative}
Write ${\bf x}=\left( \begin{array}{c} x\\y \end{array}\right)$. 
Mathematically, we wish the global motion vector ${\bf v}$ to be defined by:
\[{\bf v}=\dfrac{{\bf Ax}+{\bf b}}{1+{\bf c}^T{\bf x}}\]
where: ${\bf A}$ is a matrix describing the degree of zoom, rotation or shear; ${\bf b}$
is a translation vector; and ${\bf c}$ is a perspective vector which expresses the
degree to which the global motion is not orthogonal to the axis of view.

In Dirac, this formula is adjusted in two ways in order to get an implementable result.
Firstly, the perspective element is adjusted to remove a division, changing the 
formula to:
\[{\bf v}=(1-{\bf c}^T{\bf x})({\bf Ax}+{\bf b})\]
which is valid for small ${\bf c}$. Secondly, the formula is re-cast in terms of integer
arithmetic by giving the matrix element an accuracy factor $\alpha$ and the perspective
element an accuracy factor $\beta$:
\[{\bf v}=(1-2^{-\beta}{\bf c}^T{\bf x})(2^{-\alpha}{\bf Ax}+{\bf b})\]
where the parameters ${\bf A}, {\bf b},{\bf c}$ are now integral. (No accuracy bits are required for the translation, since it must be an integral number of sub-pixels.) 

This reduces to
\[2^{\alpha+\beta}{\bf v}=(2^\beta-{\bf c}^T{\bf x})({\bf Ax}+2^\alpha{\bf b})\]
and this formula is used for the computation of values.
\end{informative}

\subsubsection{Chroma subsampling}
\label{chromamvscale}

When motion compensating chroma components, motion vectors shall be scaled by the
$chroma\_mv\_scale()$ function. This produces chroma vectors in units of 
$\MotionVectorPrecision$ with respect to the chroma samples, as follows:

\begin{pseudo}{chroma\_mv\_scale}{v}
\bsCODE{sv[0] = v[0]//chroma\_h\_ratio()}{\ref{picturedimensions}}
\bsCODE{sv[1] = v[1]//chroma\_v\_ratio()}{\ref{picturedimensions}}
\bsRET{sv}
\end{pseudo}.

\begin{informative}
Recall that division in this specification rounds towards -infinity. This division can be achieved by a bit-shift in C/C++ as chroma dimension ratios are 1 or 2.
\end{informative}


\subsubsection{Sub-pixel prediction}
\label{upconvert}

This section defines the operation of the $subpel\_predict(ref, c, u, v)$ function
for producing a sub-pixel accurate value at location $(u,v)$ from an upconverted picture reference component of type $c$ (Y, C1 or C2). 

Upconversion shall be defined by means of a half-pixel interpolated reference array
$upref$.  $upref$ shall have dimensions $(2W-1)$x$(2H-1)$ where the original reference 
picture component array has dimensions $W$x$H$, as per Section \ref{halfpel}. 

Motion vectors shall be permitted to extend beyond the edges of reference picture data,
 where values lying outside shall be determined by edge extension. 

If $\MotionVectorPrecision==2$, upconverted values shall be derived directly from the
the half-pixel interpolated array $upref$, which shall be calculated as per Section \ref{halfpel}.

If $\MotionVectorPrecision==2$ or $\MotionVectorPrecision==3$, upconverted values shall be
derived by linear interpolation from the half-pixel interpolated array.

The sub-pixel prediction process shall be defined as follows:

\begin{pseudo}{subpel\_predict}{ref,c,u,v}
\bsCODE{upref=interp2by2(ref,c)}{\ref{halfpel}}
\bsIF{\MotionVectorPrecision==1}
    \bsCODE{xpos=\clip(u,0,\width(upref)-1)}
    \bsCODE{ypos=\clip(v,0,\height(upref)-1)}
    \bsRET{upref[ypos][xpos]}
\bsELSE
    \bsCODE{hu = u \gg (\MotionVectorPrecision-1)}
    \bsCODE{hv = v \gg (\MotionVectorPrecision-1)} 
    \bsCODE{ru = u-(hu\ll (\MotionVectorPrecision-1))}
    \bsCODE{rv = v-(hv\ll (\MotionVectorPrecision-1))}
    \bsCODE{w00 = (2^{\MotionVectorPrecision-1}-rv)*(2^{\MotionVectorPrecision-1}-ru)}
    \bsCODE{w01 = (2^{\MotionVectorPrecision-1}-rv)*ru}
    \bsCODE{w10 = rv*(2^{\MotionVectorPrecision-1}-ru)}
    \bsCODE{w11 = rv*ru}
    \bsCODE{xpos = \clip(hu, 0, \width(upref)-1)}
    \bsCODE{xpos1 = \clip(hu+1, 0,\width(upref)-1)}
    \bsCODE{ypos = \clip(hv, 0, \height(upref)-1)}
    \bsCODE{ypos1 = \clip(hv+1, 0, \height(upref)-1)}
    \bsCODE{\begin{array}{ll} val = & w00*upref[ypos][xpos]+w01*upref[ypos][xpos1]+ \\
                                & w10*upref[ypos1][xpos]+w11*upref[ypos1][xpos1]
            \end{array}}
    \bsRET{(val+2^{\MotionVectorPrecision-1})\gg\MotionVectorPrecision}
\bsEND
\end{pseudo} 

\begin{informative}
$hu$ and $hv$ represent the half-pixel part of the sub-pixel position $(u,v)$.

$ru$ and $rv$ represent the remaining sub-pixel component of the position.
$ru$ and $rv$ satisfy \[0\leq ru,rv <2^{\MotionVectorPrecision-1}\] 

The four weights $w00,w01,w10$ and $w11$ sum to $2^{\MotionVectorPrecision}$, and
hence the upconverted value is returned to pixel ranges in the pseudocode above.

Note that the remainder values $ru$ and $rv$, and hence the four weight values, 
only depend on the motion vectors. This is because
$u$ and $v$ have been computed by scaling the picture coordinates by
$2^{\MotionVectorPrecision}$ and adding the motion vector.

In particular constant linear interpolation weights are applied throughout a 
block when block motion is used. Likewise, the necessity of clipping the ranges of
$xpos$, $ypos$ etc can be determined in advance for each block by checking whether any 
corner of the reference block will fall outside of the reference picture area. In most
cases it will not and clipping will not be required for motion compensating most blocks. 
\end{informative}


\subsubsection{Half-pixel interpolation}
\label{halfpel}

This section defines the $interp2by2(ref,c)$ process for generating
an upconverted reference array $upref$ representing a half-pixel interpolation of 
the reference array $ref$ for component $c$ (Y, C1, or C2). 

$upref$ shall be created in two stages. The first stage shall upconvert vertically. The second stage shall upconvert horizontally. 

$upref$ shall have width $2*\width(ref)-1$ and height $2*\height(ref)-1$, so that all
edge values shall be copied from the original array and not interpolated.

The interpolation filter shall be the 8-tap symmetric filter with taps as defined in Figure \ref{upfilter}.

\begin{figure}[h!]
\begin{centering}
\begin{tabular}{l|ccccc}
Tap & $t[0]$ & $t[1]$ & $t[2]$ & $t[3]$\\
\hline
Value & 21 & -7 & 3 & -1
\end{tabular}
\caption{Interpolation filter coefficients \label{upfilter}}
\end{centering}
\end{figure}

Where coefficients used in the filtering process fall outside the bounds of the 
reference array, values shall be supplied by edge extension. 

The overall process shall be defined as follows:

\begin{pseudo}{interp2by2}{ref,c}
\bsIF{c==Y}
    \bsCODE{bit\_depth=\LumaDepth}
\bsELSE
    \bsCODE{bit\_depth=\ChromaDepth}
\bsEND
\bsFOR{q=0}{2*\height(ref)-2}
    \bsIF{q\%2==0}
        \bsFOR{p=0}{\width(ref)-1}
            \bsCODE{ref2[q][p]=ref[q//2][p]}
        \bsEND
    \bsELSE
        \bsFOR{p=0}{\width(ref)-1}
            \bsCODE{ref2[q][p]=16}
            \bsFOR{i=0}{3}
                \bsCODE{ypos=(q-1)//2-i}
                \bsCODE{ref2[q][p]+=t[i]*ref[\clip(ypos,0,\height(ref)-1)][p]}
                \bsCODE{ypos=(q+1)//2+i}
                \bsCODE{ref2[q][p]+=t[i]*ref[\clip(ypos,0,\height(ref)-1)][p]}
            \bsEND
            \bsCODE{ref2[q][p] \gg=5}
            \bsCODE{ref2[q][p] = \clip(ref2[q][p], -2^{bit\_depth-1}, 2^{bit\_depth-1}-1)}
        \bsEND    
    \bsEND
\bsEND
\bsFOR{q=0}{2*\height(ref)-2}
    \bsFOR{p=0}{2*\width(ref)-2}
        \bsIF{p\%2==0}
            \bsCODE{upref[q][p]=ref2[q][p//2]}
        \bsELSE
            \bsCODE{upref[q][p]=16}
            \bsFOR{i=0}{3}
                \bsCODE{xpos=(p-1)//2-i}
                \bsCODE{upref[q][p]+=t[i]*ref2[q][\clip(xpos,0,\height(ref)-1)]}
                \bsCODE{xpos=(p+1)//2+i}
                \bsCODE{upref[q][p]+=t[i]*ref2[q][\clip(xpos,0,\height(ref)-1)]}
            \bsEND
            \bsCODE{upref[q][p] \gg=5}
            \bsCODE{upref[q][p] = \clip(upref[q][p], -2^{bit\_depth-1}, 2^{bit\_depth-1}-1)}
        \bsEND
    \bsEND
\bsEND
\end{pseudo}

\begin{informative}
While this filter may appear to be variable separable, the integer rounding and 
clipping processes prevent this being so. Note also that the clipping process for
filtering terms implies that the upconversion uses edge-extension at the array
edges, consistent with the edge-extension used in motion-compensation itself.
\end{informative}


\subsection{Clipping}
\label{pictureclip}

Picture data must be clipped prior to being output or being
used as a reference:

\begin{pseudo}{clip\_picture}{}
\bsFOREACH{c}{Y,C1,C2}
    \bsCODE{clip\_component(\CurrentPicture[c])}
\bsEND
\end{pseudo}


\begin{pseudo}{clip\_component}{comp\_data,c}
\bsIF{c==Y}
    \bsCODE{\BitDepth=\LumaDepth}
\bsELSE
    \bsCODE{\BitDepth=\ChromaDepth}
\bsEND
\bsFOR{y=0}{\height(comp\_data)-1}
    \bsFOR{x=0}{\width(comp\_data)-1}
        \bsCODE{data = \clip(comp\_data[y][x], -2^{\BitDepth-1}, 2^{\BitDepth-1}-1)}
     \bsEND
\bsEND
\end{pseudo}

\begin{informative}
Note that clipping is incorporated into motion compensation, so that strictly speaking additional
clipping is only required for intra pictures.
\end{informative}

%%%%%%%%%%%%%%%%%%%%%%%%%%%%%%%%%
% -      Appendices etc       - %
%%%%%%%%%%%%%%%%%%%%%%%%%%%%%%%%%

%TO DO: separate off the appendices also




\appendix
\section{Dirac stream data parsing}%%%%%%%%%%%%%%%%%%%%%%%%%%%%%%%%%%%%%%%%%%%%%%%%%%
% - This chapter defines how raw and VLC data  - %
% - is extracted                               - % 
%%%%%%%%%%%%%%%%%%%%%%%%%%%%%%%%%%%%%%%%%%%%%%%%%%

Data is encoded in the Dirac bitstream in three basic ways: fixed-length
bit-wise and byte-wise encodings; variable-length codes; and arithmetic encoding.

This section defines how data bits are to be extracted from the bitstream and how
sequences of bits are to be interpreted as values of various types using fundamental
data-reading functions, covering encodings of the first two sorts. The extraction
of arithmetic-encoded data is defined in Section \ref{arithdecoding}.

\subsection{Bit-packing and data input}\label{bitpacking}

This section defines the operation of the $read\_bit()$, $read\_byte()$ 
and $byte\_align()$ functions used for direct access to the Dirac stream.

Access to the Dirac stream is bytewise, and a decoder is deemed to maintain
a copy of the current byte, \CurrentByte, and an index to the next bit
to be read, \NextBit. \NextBit is an integer from 0 (least-significant bit) to 7 
(most-significant bit). Bits within bytes are accessed from the msb first to the
lsb.

Each access unit and individual frame is a whole number of bytes. Decoding from the
start of an access unit, \NextBit is set to 7.

The $read\_byte()$ function returns the next byte in the Dirac stream and sets
\NextBit to 7.

The $read\_bit()$ function is defined by

\begin{pseudo}{read\_bit}{}
\bsIF{\NextBit < 0}
\bsCODE{\CurrentByte = read\_byte()}
\bsEND
\bsCODE{bit = ( \CurrentByte \gg \NextBit ) \& 1 }
\bsCODE{\NextBit -= 1}
\bsRET{bit}
\end{pseudo}

The $byte\_align()$ function discards data in the current byte and begins data access
at the next byte, unless input is already at the beginning of a byte. This is used to 
ensure that a whole number of bytes are read before
beginning reading a new stream element.

\begin{pseudo}{byte\_align()}{}
\bsIF{\NextBit != 7}
\bsCODE{\CurrentByte = read\_byte()}
\bsEND
\end{pseudo}

\subsection{Fixed-length code formats}
\subsubsection{Bool}Encoded as a single bit. A value of  1 shall be decoded as TRUE and 0
shall be decoded as false.



\subsubsection{n-bit literal}An $n$-bit number in literal format shall be decoded by extracting $n$ bits
in order, using the $read\_bit()$ function (Section \ref{bitpacking})
 and placing the first bit in the leftmost position, the second
bit in the next position and so on. The resulting value is to be
interpreted according to the conventions of the type to which it is
deemed to belong. For example, signed integers are represented in
two's complement format [ISO ref].


\subsubsection{n-byte literal}An $n$-byte number in literal format shall be decoded by extracting $n$
bytes in order and placing the first byte in the leftmost position, the
second byte in the next position and so on. The resulting value is to be
interpreted according to the conventions of the type to which it is
deemed to belong.

\subsection{Variable-length code formats}Seven different variable-length code formats are used. Of these three,
the unary binarisation formats are not used directly for data encoding
but for binarising values so that integer values may be produced from
strings of bits and vice-versa. Binarisation is used in the context of
arithmetic decoding.


\subsubsection{Unsigned interleaved exp-Golomb}Unsigned exp-Golomb data is decoded to produce unsigned integer values.
The format consists of two parts. A prefix part, consisting of n zeroes
followed by a one, indicates how many further bits to read. A suffix
part, consisting of n bits, is then used to determine the value. The
decoding procedure for extracting a value VALUE is mathematically
equivalent to:

\begin{verbatim}
COUNT= 0
while( !read\_bits(1) ) {
    COUNT++
}
VALUE = (1<<COUNT) -1 + read\_bits( COUNT )
\end{verbatim}

where the value returned by read\_bits( COUNT ) is interpeted as a binary
representation of an unsigned integer with most-significant bit first.
The bit sequences corresponding to some values are shown in  .

\begin{figure}[h]
\begin{tabular}{c|c}
Bit sequence & Decoded value \\
\hline\\
1       &  0\\
010     &  1\\
011     &  2\\
00100   &  3\\
00101   &  4\\
00110   &  5\\
00111   &  6\\
0001000 &  7\\
0001001 &  8\\
0001010 &  9\\
0001011 & 10
\end{tabular}

\caption{Example conversions from uegol-coded values to binary}
\end{figure}




\subsubsection{Signed interleaved exp-Golomb}\label{segol}

This section defines the signed interleaved exp-Golomb data format and the operation
of the $read\_sint()$ function.

The code for the signed interleaved exp-Golomb data format consists of the
unsigned interleaved exp-Golomb code for the magnitude, followed by a sign bit
for non-zero values (figure \ref{segolcodings}).

\begin{figure}[h]
\begin{tabular}{l|c}
Bit sequence & Decoded value \\
\hline\\
1                 &  0\\
0 0 1 0           &  1\\
0 0 1 1           &  -1\\
0 1 1 0            &  2\\
0 1 1 1            &  -2\\
0 0 0 0 1 0         &  3\\
0 0 0 0 1 1         &  -3\\
0 0 0 1 1 0         &  4\\
0 0 0 1 1 1         &  -4\\

\end{tabular}

\caption{Example conversions from signed interleaved exp-Golomb-coded values 
to signed integers \label{segolcodings}}
\end{figure}

The decoding operation is as follows.

\begin{pseudo}{read\_sint}{}
\bsCODE{value = read\_uint()}
\bsIF{read\_bool()}
  \bsCODE{value *= -1}
\bsEND
\bsRET{value}

\end{pseudo}

\subsubsection{Unsigned truncated unary}

If a value lies in a range $0 \leq x \leq n$, then truncated unary
binarisation can be used: that is, for the last value the final 1 can be
omitted, since its presence is inevitable.

\begin{figure}[h]
\begin{tabular}{c|c}
Value & Binarisation \\
\hline\\
0     & 1 \\
1     & 01 \\
\dots & \dots \\
$n-1$ & $\underbrace{0\dots0}_{n-1} 1$ \\
$n$   & $\underbrace{0\dots0}_{n}$
\end{tabular}

\caption{Conversion from unsigned truncated unary to binary}
\end{figure}




\clearpage
\section{Arithmetic Coding}%%%%%%%%%%%%%%%%%%%%%%%%%%%%%%%%%%%%%%%%%%%%%%%%%%%%%
% - This chapter defines the arithmetic decoding  - %
% - engine                                        - % 
%%%%%%%%%%%%%%%%%%%%%%%%%%%%%%%%%%%%%%%%%%%%%%%%%%%%%

\label{arithdecoding}

This section describes the arithmetic decoding engine and
processes for using it to extract data from the Dirac stream.

The arithmetic decoding engine consists of two elements: 

\begin{itemize}
\item a collection of state variables representing the state of the arithmetic 
decoder (Section \ref{initarith})
\item a set of functions for extracting values from the decoder 
and updating the decoder state (Section \ref{extractarith})
\end{itemize}

\subsection{State and contexts}
\label{arithcontexts}

The arithmetic decoder state consists of the following decoder state variables:

\begin{itemize}
\item $\ALow$, an integer representing the beginning of the current coding interval
\item $\AHigh$, an integer representing the end of the current coding interval
\item $\ACode$, an integer within the interval from $\ALow$ to $\AHigh$, determined from the encoded bitstream
\item $\ABitsLeft$, a decrementing count of the number of bits yet to be read in
\item $\AContexts$, a map of all the contexts used in the Dirac decoder
\end{itemize}

A context $context$ is an integer array consisting of two positive values,
$context[0]$, and $context[1]$, representing counts of values $0$ and $1$
respectively. Contexts are accessed by decoding functions
via the indices defined in Section \ref{contextindices}.

\subsubsection{Rescaling contexts}
\label{rescalecontext}

An individual context is rescaled by halving the counts of $0$ and $1$ and ensuring that
these counts do not reach zero:

\begin{pseudo}{rescale\_context}{context}
\bsCODE{context[0] \gg= 1}
\bsCODE{context[0] += 1}
\bsCODE{context[1] \gg= 1}
\bsCODE{context[1] += 1}
\end{pseudo}

\subsection{Initialisation}
\label{initarith}

At the beginning of the decoding of any data unit, the arithmetic
decoding state is initialised as follows:

\begin{pseudo}{initialise\_arithmetic\_decoding}{block\_data\_length}
\bsCODE{\ABitsLeft=8*block\_data\_length}
\bsCODE{\ALow = \text{0x0000}}
\bsCODE{\AHigh =\text{0x0000}}
\bsCODE{\ACode =\text{ 0x0000}}
\bsCODE{init\_contexts()}
\end{pseudo}

Contexts are initialised by the $init\_contexts()$ function as follows:

\begin{pseudo}{init\_contexts}{}
\bsFOR{i=0}{length(\AContexts)-1}
  \bsCODE{\AContexts[i][0]=1}
  \bsCODE{\AContexts[i][1]=1}
\bsEND
\end{pseudo}

\subsection{Data input}
\label{inputarith}

The arithmetic decoding process accesses data in a contiguous block of bytes
whose size is set on initialisation (Section \ref{initarith}). The bits in this
block are sufficient to allow for the
decoding of all coefficients. However, the specification of arithmetic
decoding operations in this section may occasionally cause further bits to be read,
even though they are not required for determining decoded values. For this
reason a read function $read\_bita()$ is defined which returns $0$ if the
bounds of this block of data have been exceeded:

\begin{pseudo}{read\_bita}{}
\bsIF{\ABitsLeft==0}
  \bsRET{0}
\bsELSE
  \bsCODE{\ABitsLeft -= 1}
  \bsRET{read\_bit()}
\bsEND
\end{pseudo}

Since the length of arithmetically coded data elements is given in bytes within the Dirac
stream, there may be bits left unread when all values have been extracted. The $flush\_arith()$
function reads all unread data, and aligns data input with the beginning of the byte 
after the arithmetically coded data i.e. at the end of the
data chunk. $flush\_arith()$ is always called at the end of decoding an arithmetically encoded
data element.

\begin{pseudo}{flush\_arith()}{}
\bsWHILE{\ABitsLeft>0}
    \bsCODE{read\_bit()}
    \bsCODE{\ABitsLeft-=1}
\bsEND
\end{pseudo}

\begin{informative}
The Dirac arithmetic decoding engine uses 16 bit words, and so no more than 16
additional bits can be read beyond the end of the block. Hence it is sufficient
to read in the entire block of data and pad the end with two zero bytes to
avoid a branch condition with each input bit. If this approach is taken, the
$flush\_arith()$ operation at the end of decoding is superfluous.
\end{informative}

\subsection{Decoder functions}
\label{extractarith}
The arithmetic decoding engine is a multi-context, adaptive binary
arithmetic decoder, performing binary renormalisation and producing
binary outputs. For each bit decoded, the semantics of the relevant
calling decoder function determine which contexts are passed to the
arithmetic decoding operations.

\subsubsection{Shifting bits in}

\label{arithshiftin}

This section defines the operation of the $shift\_bit\_in()$ 
and $shift\_all\_bits()$ functions
for reading bits into the arithmetic decoding state variables.

\begin{pseudo}{shift\_bit\_in}{}
\bsCODE{\AHigh \ll= 1}
\bsCODE{\AHigh \&= \text{0xFFFF}}
\bsCODE{\AHigh += 1}
\bsCODE{\ALow \ll= 1}
\bsCODE{\ALow \&= \text{0xFFFF}}
\bsCODE{\ACode \ll= 1}
\bsCODE{\ACode \&= \text{0xFFFF}}
\bsCODE{\ACode += read\_bita()}{\ref{inputarith}}
\end{pseudo}

$shift\_all\_bits()$ expands the interval between $\ALow$ and $\AHigh$
until the msbs (bit 15) differ and the interval no longer
straddles the half-way point 0x8000.

\begin{pseudo}{shift\_all\_bits}{}
\bsWHILE{ \AHigh\&\text{0x8000})==\text{0x0} \&\& (\ALow\&\text{0x8000})==\text{0x0}}
  \bsCODE{shift\_bit\_in()}
\bsEND
\bsWHILE{ (\AHigh\&\text{0x4000})==\text{0x0} \text{and} (\ALow\&\text{0x4000})==\text{0x4000} }
  \bsCODE{\ACode \hat{\text{ }}= \text{0x4000}}
  \bsCODE{\AHigh \hat{\text{ }}= \text{0x4000}}
  \bsCODE{\ALow \hat{\text{ }}= \text{0x4000}}
  \bsCODE{shift\_bit\_in()}
\bsEND
\end{pseudo}

\begin{informative}
Note that if 16-bit words (unsigned shorts) are used for decoder state variables $\ALow$,
 $\AHigh$ and $\ACode$ then there is no need for {\&}-ing with 0xFFFF. However, the 
operations specified here are defined in terms of integers, since intermediate calculations
 require higher dynamic range. In software, the efficiency of using short word lengths may
or may not be offset by the requirement to cast to other data types for these calculations.
\end{informative}

\subsubsection{Decoding boolean values}

\label{arithreadbool}

This section specifies the operation of the $read\_boola()$ function
for extracting a boolean value from the Dirac stream. Before extracting
any values, all possible bits are shifted in to ensure that the decoding
state has maximum information.

\begin{pseudo}{read\_boola}{context\_index}
\bsCODE{shift\_all\_bits()}{\ref{arithshiftin}}
\bsCODE{context=\AContexts[context\_index]}
\bsCODE{weight = context[0] + context[1]}
\bsCODE{scaler = (\text{0x10000}+weight//2)//weight}
\bsCODE{probability0 = context[0]*scaler}
\bsCODE{count = code-low+1}
\bsCODE{range = high-low+1}
\bsCODE{range\_times\_prob = (range * probability0)>>16}
\bsIF{ count > range\_times\_prob }
  \bsCODE{value = \true}
  \bsCODE{low = low + range\_times\_prob}
  \bsCODE{context[1] += 1}
\bsELSE
  \bsCODE{value = False}
  \bsCODE{high = low + range\_times\_prob - 1}
  \bsCODE{context[0] += 1}
\bsEND
\bsIF{ (context[0] + context[1]) > 255 }
  \bsCODE{rescale\_context(\AContexts[context\_index])}{\ref{rescalecontext}}
\bsEND
\bsRET{value}
\end{pseudo}

\begin{informative}
The function scales the probability of $0$ from the decoding context
so that a probability of $1$ is commensurate with the interval between
 $\ALow$ and $\AHigh$. If $\ACode$ is greater than this cut-off, then 1 ($\true$) has
been encoded, else 0 ($\false$) has.
\end{informative}

\subsubsection{Arithmetic decoding of integer values}

\label{arithreadint}

This section defines the operation of the $read\_sinta(context\_set)$ function
 for extracting integer values from a block of arithmetically coded data.

\paragraph{Binarisation and contexts \\}

Signed and unsigned integer values are binarised using interleaved exp-Golomb
 binarisation as per Section \ref{vlc}: the $read\_sinta()$ and $read\_uinta()$
processes are essentially identical to the 
$read\_sint()$ and $read\_uint()$ processes, except that instances of $read\_bool()$ are replaced
by instances of $read\_ba()$ (Section \ref{arithreadbool}) using suitable contextualisation. 

A choice of context depends upon whether the bit is a data bit, follow bit, or sign bit, and the 
position of the bit within the binarisation: $context\_set$ consists of three parts -
\begin{itemize}
\item an array of follow contexts, $context\_set[follow]$ (indexed from 0 to 
$\length(context\_set[follow])-1$)
\item a single data context $context\_set[data]$ 
\item a sign context $context\_set[sign]$ (ignored for unsigned integer decoding)
\end{itemize}

Each follow context is used for decoding the corresponding follow bit, with the
last follow context being used for all subsequent follow bits (if any) also. 
The follow context selection function $follow\_context()$ is defined by:

\begin{pseudo}{follow\_context}{index, context\_set}
\bsCODE{pos= max(index, length(context\_set[follow])-1 }
\bsCODE{ctx_index = context\_set[follow][pos]}
\bsRET{\Contexts[ctx\_index]}
\end{pseudo}

So the last follow context is used for all the remaining follow bits also.

\paragraph{Unsigned integer decoding \\}

Unsigned integers are decoded by:

\begin{pseudo}{read\_uinta}{context\_set}
\bsCODE{value = 1}
\bsCODE{index = 0}
\bsWHILE{read\_ba(follow\_context(index,context\_set) )== \false}
  \bsCODE{value \ll = 1}
  \bsIF{read\_ba(\text{state[contexts]}[context\_set[data])])}
    \bsCODE{value += 1}
  \bsEND
  \bsCODE{index += 1}
\bsEND
\bsCODE{value -= 1}
\bsRET{value}
\end{pseudo}

\paragraph{Signed integer decoding \\}

$read\_sinta()$ decodes first the magnitude then the sign, as necessary:

\begin{pseudo}{read\_sinta}{context\_set}
\bsCODE{value=read\_uinta(context_set)}
\bsIF{value != 0}
  \bsIF{read\_ba(\Contexts[context\_set[sign])]) == \true}
    \bsCODE{value=-value}
  \bsEND
\bsEND
\bsRET{value}
\end{pseudo}

\subsection{Context indices}
\label{contextindices}

The following is a list of all the context indices used in Dirac arithmetic decoding operations:

%\begin{tabular}{l}
\SignZero\\
\SignPos\\
\SignNeg\\
\ZPZNFollowOne\\
\ZPNNFollowOne\\
\ZPFollowTwo\\
\ZPFollowThree\\
\ZPFollowFour\\
\ZPFollowFive\\
\ZPFollowSixPlus\\
\NPZNFollowOne\\
\NPNNFollowOne\\
\NPFollowTwo\\
\NPFollowThree\\
\NPFollowFour\\
\NPFollowFive\\
\NPFollowSixPlus\\
\CoeffData\\
\ZeroCodeblock\\
\QOffsetFollow\\
\QOffsetInfo\\
\QOffsetSign\\
\SBSplitFollowOne\\
\SBSplitFollowTwo\\
\SBSplitData\\
\PredModeOne\\
\PredModeTwo\\
\BlockGlobal\\
\RefOnexFollowOne\\
\RefOnexFollowTwo\\
\RefOnexFollowThree\\
\RefOnexFollowFour\\
\RefOnexFollowFivePlus\\
\RefOnexData\\
\RefOnexSign\\
\RefOneyFollowOne\\
\RefOneyFollowTwo\\
\RefOneyFollowThree\\
\RefOneyFollowFour\\
\RefOneyFollowFivePlus\\
\RefOneyData\\
\RefOneySign\\
\RefTwoxFollowOne\\
\RefTwoxFollowTwo\\
\RefTwoxFollowThree\\
\RefTwoxFollowFour\\
\RefTwoxFollowFivePlus\\
\RefTwoxData\\
\RefTwoxSign\\
\RefTwoyFollowOne\\
\RefTwoyFollowTwo\\
\RefTwoyFollowThree\\
\RefTwoyFollowFour\\
\RefTwoyFollowFivePlus\\
\RefTwoyData\\
\RefTwoySign\\
\YDCFollowOne\\
\YDCFollowTwoPlus\\
\YDCData\\
\YDCSign\\
\UDCFollowOne\\
\UDCFollowTwoPlus\\
\UDCData\\
\UDCSign\\
\VDCFollowOne\\
\VDCFollowTwoPlus\\
\VDCData\\
\VDCSign
%\end{tabular}
%\section{Comments on coders}It is not our intention to be prescriptive about how coders should or
should not be organised. Any coder which produces a compliant bytestream
may have a place in someone's work. This Appendix is provided to show
that there are quite a lot of things to consider.

First of all, it is worth noting the competing factors of quality, bit
rate, complexity and delay.

High quality usually requires a high bit rate. To maintain a consistent
quality may require the bit rate to be variable. This leads to a
potential need for a fair amount of buffer storage in a system which
provides a fixed bit rate.

Any system which provides a lot of compression (i.e. a low bit rate)
requires full use of all the tools. When we explore the capacity used by
each element, we find that the Intra frames tend to require most
capacity. Access Units with many Inter frames and few Intra frames will
therefore seem to be a desirable solution - except for the fact that
this will potentially increase the necessary buffer size and lengthen
the time between access points.

Of the data provided in each frame, the data used for motion vectors and
prediction is roughly the same as the quantity of data used for the
wavelet coefficients. The trade off between the two can be enhanced by
careful rate distortion optimisation.

In some low-complexity implementations, it is possible to simplify the
motion vectors, or even omit them.

For low delay systems, a sequence of Intra frames gives the best
performance.


%\section{Subband inverse quantizer values}
The inverse quantisation process for reconstructing subband coefficients
requires quantisation factors and offsets derived from an index
quant\_index as specified in table \ref{}.

These values may be calulcated as:

\begin{displaymath}
    \texttt{quant\_factor}= round(2^{\frac{\texttt{quant\_index}}{4}})
\end{displaymath}

\begin{displaymath}
    \texttt{offset}       = round(\texttt{quant\_factor} * 0.375)
\end{displaymath}
    

\begin{figure}[h!]
    \centering
    \begin{tabular}{|c|c|c|}
        \hline
        quant\_index & quant\_factor & offset \\\hline
        0            & 1             & 0      \\\hline
        1            & 1             & 0      \\\hline
        2            & 1             & 0      \\\hline
        3            & 2             & 1      \\\hline
        4            & 2             & 1      \\\hline
        5            & 2             & 1      \\\hline
        6            & 3             & 1      \\\hline
        7            & 3             & 1      \\\hline
        8            & 4             & 2      \\\hline
        9            & 5             & 2      \\\hline
        10           & 6             & 2      \\\hline
        11           & 7             & 3      \\\hline
        12           & 8             & 3      \\\hline
        13           & 10            & 4      \\\hline
        14           & 11            & 4      \\\hline
        15           & 13            & 5      \\\hline
        16           & 16            & 6      \\\hline
        17           & 19            & 7      \\\hline
        18           & 23            & 9      \\\hline
        19           & 27            & 10     \\\hline
        20           & 32            & 12     \\\hline
        21           & 38            & 14     \\\hline
        22           & 45            & 17     \\\hline
        23           & 54            & 20     \\\hline
        24           & 64            & 24     \\\hline
        25           & 76            & 29     \\\hline
        26           & 91            & 34     \\\hline
        27           & 108           & 41     \\\hline
        28           & 128           & 48     \\\hline
        29           & 152           & 57     \\\hline
        30           & 181           & 68     \\\hline
        31           & 215           & 81     \\\hline
        32           & 256           & 96     \\\hline
        33           & 304           & 114    \\\hline
        34           & 362           & 136    \\\hline
        35           & 431           & 162    \\\hline
        36           & 512           & 192    \\\hline
        37           & 609           & 228    \\\hline
        38           & 724           & 272    \\\hline
        39           & 861           & 323    \\\hline
        40           & 1024          & 384    \\\hline
        41           & 1218          & 457    \\\hline
        42           & 1448          & 543    \\\hline
        43           & 1722          & 646    \\\hline
        44           & 2048          & 768    \\\hline
        45           & 2435          & 913    \\\hline
        46           & 2896          & 1086   \\\hline
        47           & 3444          & 1292   \\\hline
        48           & 4096          & 1536   \\\hline
        49           & 4871          & 1827   \\\hline
        50           & 5793          & 2172   \\\hline
        51           & 6899          & 2583   \\\hline
        52           & 8192          & 3072   \\\hline
        53           & 9742          & 3653   \\\hline
        54           & 11585         & 4344   \\\hline
        55           & 13777         & 5166   \\\hline
        56           & 16384         & 6144   \\\hline
        57           & 19484         & 7307   \\\hline
        58           & 23170         & 8689   \\\hline
        59           & 27554         & 10333  \\\hline
        60           & 32768         & 12288  \\\hline
    \end{tabular}
\end{figure}


\clearpage
\section{Quantisation matrices and weighting functions}
\label{quantmatrices}

This section specifies the default quantisation matrices to be used
in the low delay syntax and provides an informative description of quantisation
matrix design principles and of quantiser selection in both the core
and low-delay syntax.

\subsection{Quantisation matrices (low delay syntax)}
\label{defaultquantmatrices}

This section defines default quantisation matrices to be used 
for the quantisation of slice coefficients in the low-delay syntax.
The following tables define matrices for $\TransformDepth\leq 4$. 
Values of $\TransformDepth$ not present in the tables
in this section shall require a custom matrix to be encoded, 
as per Section \ref{sliceparams}. Informative advice for 
constructing quantisation matrices based on noise power 
conservation and perceptual weighting is given in 
Appendix \ref{custommatrices}.

\begin{table}[!ht]
\centering
\begin{tabular}{|c|c|c|}
\hline
\multicolumn{3}{|c|}{{$\TransformDepth==1$}} \\
\hline
Level & Orientation & \QuantMatrix[level][orientation] \\
\hline
0 & LL & 5 \\
\hline
1 & HL,LH & 3 \\
1 & HH & 0 \\
\hline
\hline
\multicolumn{3}{|c|}{{$\TransformDepth==2$}} \\
\hline
Level & Orientation & \QuantMatrix[level][orientation] \\
\hline
0 & LL & 6 \\
\hline
1 & HL,LH & 3 \\
1 & HH & 0 \\
\hline
2 & HL,LH & 5 \\
2 & HH & 2 \\
\hline
\hline
\multicolumn{3}{|c|}{{$\TransformDepth==3$}} \\
\hline
Level & Orientation & \QuantMatrix[level][orientation] \\
\hline
0 & LL & 5 \\
\hline
1 & HL,LH & 2 \\
1 & HH & 0 \\
\hline
2 & HL,LH & 4 \\
2 & HH & 1 \\
\hline
3 & HL,LH & 6 \\
3 & HH & 3 \\
\hline
\hline
\multicolumn{3}{|c|}{{$\TransformDepth==4$}} \\
\hline
Level & Orientation & \QuantMatrix[level][orientation] \\
\hline
0 & LL & 5 \\
\hline
1 & LH,LH & 3 \\
1 & HH & 0 \\
\hline
2 & HL,LH & 4 \\
2 & HH & 2 \\
\hline
3 & HL,LH & 6 \\
3 & HH & 3 \\
\hline
4 & HL,LH & 8 \\
4 & HH & 5 \\
\hline
\end{tabular}
\caption{Default quantisation matrices for $\WaveletIndex==0$ (Deslauriers-Debuc (9,5)) 
\label{table:qm0}}
\end{table}

\begin{table}[!ht]
\centering
\begin{tabular}{|c|c|c|}
\hline
\multicolumn{3}{|c|}{{$\TransformDepth==1$}} \\
\hline
Level & Orientation & \QuantMatrix[level][orientation] \\
\hline
0 & LL & 4 \\
\hline
1 & HL,LH & 2 \\
1 & HH & 0 \\
\hline
\hline
\multicolumn{3}{|c|}{{$\TransformDepth==2$}} \\
\hline
Level & Orientation & \QuantMatrix[level][orientation] \\
\hline
0 & LL & 4 \\
\hline
1 & HL,LH & 2 \\
1 & HH & 0 \\
\hline
2 & HL,LH & 4 \\
2 & HH & 2 \\
\hline
\hline
\multicolumn{3}{|c|}{{$\TransformDepth==3$}} \\
\hline
Level & Orientation & \QuantMatrix[level][orientation] \\
\hline
0 & LL & 5 \\
\hline
1 & HL,LH & 3 \\
1 & HH & 0 \\
\hline
2 & HL,LH & 5 \\
2 & HH & 3 \\
\hline
3 & HL,LH & 7 \\
3 & HH & 5 \\
\hline
\hline
\multicolumn{3}{|c|}{{$\TransformDepth==4$}} \\
\hline
Level & Orientation & \QuantMatrix[level][orientation] \\
\hline
0 & LL & 5 \\
\hline
1 & LH,LH & 3 \\
1 & HH & 0 \\
\hline
2 & HL,LH & 5 \\
2 & HH & 2 \\
\hline
3 & HL,LH & 7 \\
3 & HH & 5 \\
\hline
4 & HL,LH & 9 \\
4 & HH & 7 \\
\hline
\end{tabular}
\caption{Default quantisation matrices for $\WaveletIndex==1$ (LeGall (5,3)) 
\label{table:qm1}}
\end{table}

\begin{table}[!ht]
\centering
\begin{tabular}{|c|c|c|}
\hline
\multicolumn{3}{|c|}{{$\TransformDepth==1$}} \\
\hline
Level & Orientation & \QuantMatrix[level][orientation] \\
\hline
0 & LL & 5 \\
\hline
1 & HL,LH & 2 \\
1 & HH & 0 \\
\hline
\hline
\multicolumn{3}{|c|}{{$\TransformDepth==2$}} \\
\hline
Level & Orientation & \QuantMatrix[level][orientation] \\
\hline
0 & LL & 6 \\
\hline
1 & HL,LH & 3 \\
1 & HH & 0 \\
\hline
2 & HL,LH & 4 \\
2 & HH & 2 \\
\hline
\hline
\multicolumn{3}{|c|}{{$\TransformDepth==3$}} \\
\hline
Level & Orientation & \QuantMatrix[level][orientation] \\
\hline
0 & LL & 6 \\
\hline
1 & HL,LH & 3 \\
1 & HH & 0 \\
\hline
2 & HL,LH & 4 \\
2 & HH & 1 \\
\hline
3 & HL,LH & 5 \\
3 & HH & 3 \\
\hline
\hline
\multicolumn{3}{|c|}{{$\TransformDepth==4$}} \\
\hline
Level & Orientation & \QuantMatrix[level][orientation] \\
\hline
0 & LL & 5 \\
\hline
1 & LH,LH & 3 \\
1 & HH & 0 \\
\hline
2 & HL,LH & 4 \\
2 & HH & 1 \\
\hline
3 & HL,LH & 5 \\
3 & HH & 2 \\
\hline
4 & HL,LH & 6 \\
4 & HH & 4 \\
\hline
\end{tabular}
\caption{Default quantisation matrices for $\WaveletIndex==2$ (Deslauriers-Debuc (13,5))) 
\label{table:qm2}}
\end{table}

\begin{table}[!ht]
\centering
\begin{tabular}{|c|c|c|}
\hline
\multicolumn{3}{|c|}{{$\TransformDepth==1$}} \\
\hline
Level & Orientation & \QuantMatrix[level][orientation] \\
\hline
0 & LL & 8 \\
\hline
1 & HL,LH & 4 \\
1 & HH & 0 \\
\hline
\hline
\multicolumn{3}{|c|}{{$\TransformDepth==2$}} \\
\hline
Level & Orientation & \QuantMatrix[level][orientation] \\
\hline
0 & LL & 12 \\
\hline
1 & HL,LH & 8 \\
1 & HH & 4 \\
\hline
2 & HL,LH & 4 \\
2 & HH & 0 \\
\hline
\hline
\multicolumn{3}{|c|}{{$\TransformDepth==3$}} \\
\hline
Level & Orientation & \QuantMatrix[level][orientation] \\
\hline
0 & LL & 16 \\
\hline
1 & HL,LH & 12 \\
1 & HH & 8 \\
\hline
2 & HL,LH & 8 \\
2 & HH & 4 \\
\hline
3 & HL,LH & 4 \\
3 & HH & 0 \\
\hline
\hline
\multicolumn{3}{|c|}{{$\TransformDepth==4$}} \\
\hline
Level & Orientation & \QuantMatrix[level][orientation] \\
\hline
0 & LL & 20 \\
\hline
1 & LH,LH & 16 \\
1 & HH & 12 \\
\hline
2 & HL,LH & 12 \\
2 & HH & 8 \\
\hline
3 & HL,LH & 8 \\
3 & HH & 4 \\
\hline
4 & HL,LH & 4 \\
4 & HH & 0 \\
\hline
\end{tabular}
\caption{Default quantisation matrices for $\WaveletIndex==3$ (Haar with no shift)) 
\label{table:qm3}}
\end{table}

\begin{table}[!ht]
\centering
\begin{tabular}{|c|c|c|}
\hline
\multicolumn{3}{|c|}{{$\TransformDepth==1$}} \\
\hline
Level & Orientation & \QuantMatrix[level][orientation] \\
\hline
0 & LL & 8 \\
\hline
1 & HL,LH & 4 \\
1 & HH & 0 \\
\hline
\hline
\multicolumn{3}{|c|}{{$\TransformDepth==2$}} \\
\hline
Level & Orientation & \QuantMatrix[level][orientation] \\
\hline
0 & LL & 8 \\
\hline
1 & HL,LH & 4 \\
1 & HH & 0 \\
\hline
2 & HL,LH & 4 \\
2 & HH & 0 \\
\hline
\hline
\multicolumn{3}{|c|}{{$\TransformDepth==3$}} \\
\hline
Level & Orientation & \QuantMatrix[level][orientation] \\
\hline
0 & LL & 8 \\
\hline
1 & HL,LH & 4 \\
1 & HH & 0 \\
\hline
2 & HL,LH & 4 \\
2 & HH & 0 \\
\hline
3 & HL,LH & 4 \\
3 & HH & 0 \\
\hline
\hline
\multicolumn{3}{|c|}{{$\TransformDepth==4$}} \\
\hline
Level & Orientation & \QuantMatrix[level][orientation] \\
\hline
0 & LL & 8 \\
\hline
1 & LH,LH & 4 \\
1 & HH & 0 \\
\hline
2 & HL,LH & 4 \\
2 & HH & 0 \\
\hline
3 & HL,LH & 4 \\
3 & HH & 0 \\
\hline
4 & HL,LH & 4 \\
4 & HH & 0 \\
\hline
\end{tabular}
\caption{Default quantisation matrices for $\WaveletIndex==4$ (Haar with single shift per level)) 
\label{table:qm4}}
\end{table}

\begin{table}[!ht]
\centering
\begin{tabular}{|c|c|c|}
\hline
\multicolumn{3}{|c|}{{$\TransformDepth==1$}} \\
\hline
Level & Orientation & \QuantMatrix[level][orientation] \\
\hline
0 & LL & 8 \\
\hline
1 & HL,LH & 4 \\
1 & HH & 0 \\
\hline
\hline
\multicolumn{3}{|c|}{{$\TransformDepth==2$}} \\
\hline
Level & Orientation & \QuantMatrix[level][orientation] \\
\hline
0 & LL & 8 \\
\hline
1 & HL,LH & 4 \\
1 & HH & 0 \\
\hline
2 & HL,LH & 8 \\
2 & HH & 4 \\
\hline
\hline
\multicolumn{3}{|c|}{{$\TransformDepth==3$}} \\
\hline
Level & Orientation & \QuantMatrix[level][orientation] \\
\hline
0 & LL & 8 \\
\hline
1 & HL,LH & 4 \\
1 & HH & 0 \\
\hline
2 & HL,LH & 8 \\
2 & HH & 4 \\
\hline
3 & HL,LH & 12 \\
3 & HH & 8 \\
\hline
\hline
\multicolumn{3}{|c|}{{$\TransformDepth==4$}} \\
\hline
Level & Orientation & \QuantMatrix[level][orientation] \\
\hline
0 & LL & 8 \\
\hline
1 & LH,LH & 4 \\
1 & HH & 0 \\
\hline
2 & HL,LH & 8 \\
2 & HH & 4 \\
\hline
3 & HL,LH & 12 \\
3 & HH & 8 \\
\hline
4 & HL,LH & 16 \\
4 & HH & 12 \\
\hline
\end{tabular}
\caption{Default quantisation matrices for $\WaveletIndex==5$ (Haar with double shift per level)) 
\label{table:qm5}}
\end{table}

\begin{table}[!ht]
\centering
\begin{tabular}{|c|c|c|}
\hline
\multicolumn{3}{|c|}{{$\TransformDepth==1$}} \\
\hline
Level & Orientation & \QuantMatrix[level][orientation] \\
\hline
0 & LL & 0 \\
\hline
1 & HL,LH & 4 \\
1 & HH & 7 \\
\hline
\hline
\multicolumn{3}{|c|}{{$\TransformDepth==2$}} \\
\hline
Level & Orientation & \QuantMatrix[level][orientation] \\
\hline
0 & LL & 0 \\
\hline
1 & HL,LH & 3 \\
1 & HH & 7 \\
\hline
2 & HL,LH & 7 \\
2 & HH & 10 \\
\hline
\hline
\multicolumn{3}{|c|}{{$\TransformDepth==3$}} \\
\hline
Level & Orientation & \QuantMatrix[level][orientation] \\
\hline
0 & LL & 0 \\
\hline
1 & HL,LH & 4 \\
1 & HH & 7 \\
\hline
2 & HL,LH & 7 \\
2 & HH & 11 \\
\hline
3 & HL,LH & 11 \\
3 & HH & 14 \\
\hline
\hline
\multicolumn{3}{|c|}{{$\TransformDepth==4$}} \\
\hline
Level & Orientation & \QuantMatrix[level][orientation] \\
\hline
0 & LL & 0 \\
\hline
1 & LH,LH & 3 \\
1 & HH & 7 \\
\hline
2 & HL,LH & 7 \\
2 & HH & 10 \\
\hline
3 & HL,LH & 10 \\
3 & HH & 14 \\
\hline
4 & HL,LH & 14 \\
4 & HH & 17 \\
\hline
\end{tabular}
\caption{Default quantisation matrices for $\WaveletIndex==6$ (Fidelity)) 
\label{table:qm6}}
\end{table}

\begin{table}[!ht]
\centering
\begin{tabular}{|c|c|c|}
\hline
\multicolumn{3}{|c|}{{$\TransformDepth==1$}} \\
\hline
Level & Orientation & \QuantMatrix[level][orientation] \\
\hline
0 & LL & 4 \\
\hline
1 & HL,LH & 2 \\
1 & HH & 0 \\
\hline
\hline
\multicolumn{3}{|c|}{{$\TransformDepth==2$}} \\
\hline
Level & Orientation & \QuantMatrix[level][orientation] \\
\hline
0 & LL & 3 \\
\hline
1 & HL,LH & 2 \\
1 & HH & 0 \\
\hline
2 & HL,LH & 4 \\
2 & HH & 2 \\
\hline
\hline
\multicolumn{3}{|c|}{{$\TransformDepth==3$}} \\
\hline
Level & Orientation & \QuantMatrix[level][orientation] \\
\hline
0 & LL & 3 \\
\hline
1 & HL,LH & 1 \\
1 & HH & 0 \\
\hline
2 & HL,LH & 4 \\
2 & HH & 2 \\
\hline
3 & HL,LH & 6 \\
3 & HH & 4 \\
\hline
\hline
\multicolumn{3}{|c|}{{$\TransformDepth==4$}} \\
\hline
Level & Orientation & \QuantMatrix[level][orientation] \\
\hline
0 & LL & 3 \\
\hline
1 & LH,LH & 2 \\
1 & HH & 0 \\
\hline
2 & HL,LH & 4 \\
2 & HH & 3 \\
\hline
3 & HL,LH & 7 \\
3 & HH & 5 \\
\hline
4 & HL,LH & 9 \\
4 & HH & 7 \\
\hline
\end{tabular}
\caption{Default quantisation matrices for $\WaveletIndex==7$ (Daubechies (9,7))
\label{table:qm7}}
\end{table}

\clearpage
\begin{informative*}
\subsection{Quantisation matrix design and quantiser selection (Informative)}
\label{qmatrixdesign}

This section provides an informative guide to the principles used to design the default
quantisation matrix 

\subsubsection{Noise power normalisation}
\label{noisenorm}

The quantisation matrices defined in the preceding section are designed to counteract the
differential power gain of the various wavelet filters, so that quantisation noise from 
each subband is weighted equally in terms of its contribution to noise power when transformed
back into the picture domain. Let $\alpha$ and $\beta$ represent the noise gain factors of
the low-pass and high-pass wavelet filters used in wavelet decomposition. In a single level of
wavelet decomposition, quantisation noise in each of the four subbands is therefore weighted by the factors shown in Figure \ref{fig:onelevelweight}.
\end{informative*}
\setlength{\unitlength}{1em}
\begin{figure}[!h]
\centering
\begin{picture}(20,27)
\put(0,5){\line(1,0){20}}
\put(0,5){\line(0,1){20}}
\put(20,5){\line(0,1){20}}
\put(20,25){\line(-1,0){20}}

\put(10,5){\line(0,1){20}}
\put(0,15){\line(1,0){20}}

\put(3,19.5){\text{\Large LL -- $\alpha^2$}}
\put(3,9.5){\text{\Large LH -- $\alpha\beta$}}
\put(13,19.5){\text{\Large HL -- $\alpha\beta$}}
\put(13,9.5){\text{\Large HH -- $\beta^2$}}
\end{picture}
\caption{Subband weights for a 1-level decomposition}\label{fig:onelevelweight}
\end{figure}
\begin{informative*}

For higher levels of decomposition, these subband weighting factors iterate
in the same manner as the wavelet transform itself. For example, with a two-level
decomposition, the first level LL band, with weight $\alpha^2$ is further decomposed
to give four more bands with weights as for the 1-level decomposition, but multiplied
by $\alpha^2$. This yields the weights shown in Figure \ref{fig:twolevelweight}.
\end{informative*}
\setlength{\unitlength}{1em}
\begin{figure}[!h]
\centering
\begin{picture}(30,40)
\put(0,5){\line(1,0){30}}
\put(0,5){\line(0,1){30}}
\put(30,5){\line(0,1){30}}
\put(30,35){\line(-1,0){30}}

\put(15,5){\line(0,1){30}}
\put(0,20){\line(1,0){30}}

\put(5.5,12){\text{\Large LH -- $\alpha\beta$}}
\put(20.5,27){\text{\Large HL -- $\alpha\beta$}}
\put(20.5,12){\text{\Large HH -- $\beta^2$}}

\put(7.5,20){\line(0,1){15}}
\put(0,27.5){\line(1,0){15}}

\put(2,31){\text{\Large LL -- $\alpha^4$}}
\put(2,23.5){\text{\Large LH -- $\alpha^3\beta$}}
\put(9,31){\text{\Large HL -- $\alpha^3\beta$}}
\put(9,23.5){\text{\Large HH -- $\alpha^2\beta^2$}}

\end{picture}
\caption{Subband weights for a 2-level decomposition}\label{fig:twolevelweight}
\end{figure}
\begin{informative*}

In this specification, wavelet synthesis filters have been defined in terms of lifting stages,
which are filters operating on subsampled data. Wavelet filters are more traditionally
represented in terms of an iterated binary polyphase filter bank: the relationship between
these representation is described in Appendix \ref{lifting}. The factors $\alpha$ and $\beta$
are most easily computed from the filter bank representation. In this case $\alpha$ is either
the RMS power gain of the low-pass synthesis filter, or the {\em reciprocal} of the RMS power
gain of the low-pass analysis filter; and $\beta$ is the RMS power gain of the high-pass
synthesis filter of the reciprocal of the RMS power gain of the high-pass analysis filter. 

Thus, in the terminology of Appendix \ref{lifting}, 
$\alpha=\dfrac{1}{(\sum_n h(n)^2)^{\frac{1}{2}}}$ or
$\alpha=(\sum_n \tilde{h}(n)^2)^{\frac{1}{2}}$

and
$\beta=\dfrac{1}{(\sum_n g(n)^2)^{\frac{1}{2}}}$ or
$\beta=(\sum_n \tilde{g}(n)^2)^{\frac{1}{2}}$

These alternative definitions arise because the wavelet filters defined in this specification
are not orthogonal, but technically {\em biorthogonal} and so, strictly speaking, there is
not power addition of the quantisation noise in each subband. The values used for quantisation
matrices have been computed from the analysis rather than the synthesis filters, as this yields
better compression results in practice.

Note also that these factors must also take into account the shift factors used to add accuracy 
bits prior to each wavelet decomposition stage. For a filter shift of $d$, $\alpha$ and 
$\beta$ are each multiplied by $2^{-d/2}$.

Given a subband weighting factor $w$, a quantisation offset for that subband may be defined 
as $4*\log_2(w)$ rounded to the nearest integer. These offsets are then normalised so as
to be non-negative, to produce the tables of the preceding section.

\subsubsection{Custom quantisation matrices}
\label{custommatrices}

Custom matrices may be defined that take into account not only noise power normalisation
but also perceptual weighting based on spatial frequency. Additional multiplicative factors
may be computed for each subband, which produce a matrix of quantisation offsets which may
then be added to the default unweighted quantisation matrices to produce a weighted quantisation
matrix.

An example perceptual weighting may be constructed from the CCIR 959 Contrast Sensitivity
Function (CSF). This is a function $csf(s)$ which produces a value representing the
sensitivity to detail at a given normalised spatial frequency $s$. For luminance, it is defined
by 
\[csf(s)=0.255*(1+0.2561*s^2)^{-0.75}\]

Assuming an isotropic response, we may form a 2-d perceptual weighting function on 
horizontal and vertical spatial frequencies $x_s,y_s$ by
\begin{eqnarray*}
c(x_s,y_s) & = & \dfrac{1}{csf((xs^2+ys^2)^{\frac{1}{2}})} \\
& = & 0.255*(1+0.2561*(x_s^2+y_s^2))^{0.75}
\end{eqnarray*}

Each subband in a wavelet decomposition represents a subset of spatial frequencies according
to level and orientation, partitioning the spatial frequency domain as per Figure \ref{fig:orientlevel}.
Note that this partitioning is un-normalised, since output pictures (and their compression artefacts) may
be viewed at a range of distances. 

Accordingly we may pick a representative, un-normalised horizontal and vertical spatial frequency $(f_x(b),f_y(b))$ -- perhaps the middle frequency of the band. For example, an LH band $b$ at level 1 in a 1-level 
decomposition will have mid frequency at $(pw/4,3*ph/4)$ where $ph$ and $pw$ are the padded
width and height of the picture (Section \ref{subbandwidthheight}). This may be turned into a true
spatial frequency by normalising by the number of horizontal and vertical cycles per degree the output
pictures will subtend at the target viewing distance and aspect ratio:
\[ (f_x(b)/cpd_x,f_y(b)/cpd_y)\]

and this value may be fed into the weighting function to get a value $c(b)$. The appropriate
quantisation offset for that subband is then $4*\log_2(c(b))$, which may be used to define a modified
quantisation matrix.
\subsection{Quantiser selection in the core syntax (Informative)}
\label{qselectcore}

\subsection{Quantiser selection in the low delay syntax (Informative)}
\label{qselectld}



\end{informative*}

\clearpage
\begin{informative*}
\section{Wavelet transform and lifting (Informative)}
This is an informative appendix introducing the fundamentals of wavelet
filtering and the lifting scheme. For a fuller explanation see ?.

\subsection{Wavelet filter banks}

Figure \ref{fig:decimatereconstruct} below illustrates a single stage of a 
generalized wavelet decimation followed by reconstruction. The aim is to 
get perfect reconstruction of the output so that it is identical to the original input. 
\end{informative*}
\setlength{\unitlength}{1em}
\begin{figure}[!ht]
\begin{picture}(50,20)
\put(0,10){\line(1,0){5}}
\put(5,5){\line(0,1){10}}

\put(5,5){\line(1,0){2}}
\put(5,15){\line(1,0){2}}

\put(7,3){\framebox(8,4){\Large $h(-z)$}}
\put(7,13){\framebox(8,4){\Large $g(-z)$}}

\put(15,5){\line(1,0){2}}
\put(15,15){\line(1,0){2}}

\put(18.5,5){\circle{3}}\put(18,5){\Large $\downarrow 2$}
\put(18.5,15){\circle{3}}\put(18,15){\Large $\downarrow 2$}

\put(20,5){\line(1,0){2}}
\put(20,15){\line(1,0){2}}

\put(24,5){\line(1,0){2}}
\put(24,15){\line(1,0){2}}

\put(27.5,5){\circle{3}}\put(27,5){\Large $\uparrow 2$}
\put(27.5,15){\circle{3}}\put(27,15){\Large $\uparrow 2$}

\put(29,5){\line(1,0){2}}
\put(29,15){\line(1,0){2}}

\put(31,3){\framebox(8,4){\Large $\tilde{h}(z)$}}
\put(31,13){\framebox(8,4){\Large $\tilde{g}(z)$}}

\put(39,5){\line(1,0){2}}
\put(39,15){\line(1,0){2}}

\put(39.5,10){\line(1,0){6.5}}
\put(41,5){\line(0,1){10}}

\put(41,10){\circle{3}}
\end{picture}
\caption{Wavelet decimation and reconstruction}\label{fig:decimatereconstruct}
\end{figure}
\begin{informative*}
The filters $h(z)$ and $g(z)$ are the low-pass and high-pass analysis 
filters, whilst $\tilde{h}(z)$ and $\tilde{g}(z)$ are the 
synthesis filters. The filters must satisfy the conditions
\begin{eqnarray*}
h(z)\tilde{h}(z^{-1})+g(z)\tilde{g}(z^{-1}) & = & 2 \text{ (Perfect reconstruction)}\\
h(z)\tilde{h}(-z^{-1})+g(z)\tilde{g}(-z^{-1}) & = & 0 \text{ (Alias cancellation)}
\end{eqnarray*}
These conditions imply that the synthesis filters are derived from the analysis filters and vice-versa:
\begin{eqnarray*}
\tilde{g}(z) & = & z^{-1}h(-z^{-1}) \\
\tilde{h}(z) & = & z^{-1}g(-z^{-1})
\end{eqnarray*}

If we have an {\em orthogonal} wavelet decomposition, then additionally $H=\tilde{h}$ and $g=\tilde{g}$
and there is a single ``mother'' wavelet.
 
The next figure (F.2) illustrates how the frequency components are distributed both during decimation and reconstruction. This figures illustrates how the alias frequencies created during the decimation process are cancelled out during the reconstruction process. This feature of alias cancellation results from the wavelet process and is a specific attribute of wavelet coding. It is important to note that if the decoder receives imperfect signals (caused, for example, by quantisation errors) then the imperfections will result in distortion in the reconstructed output.

 
Figure F.2 -Illustration of the alias frequency generation and cancellation in a wavelet filter bank
A single wavelet stage is insufficient for most video coding applications. The figure below illustrates how only the low-pass path is passed on to the next wavelet decimation step. Because each step of the wavelet decimation is self-contained, the reconstructed output is still identical to the input (barring quantisation errors).
 
Figure F.3 -Two-step wavelet processing filter bank
The application of wavelet filter banks in picture coding results in a two-dimensional decimation process as illustrated in figure F.4 below.
 
Figure F.4 -Decomposition of a single image into 7 wavelet frequency bands
The final figure illustrates how a real image is decimated to produce a low-frequency proxy in the top-left corner and a range of increasing frequency band components extending to the right side for increasing horizontal frequencies and downwards for increasing vertical frequencies.
 
Figure F.5 -Decomposition of the EBU "Boats" picture into 7 wavelet frequency bands

\subsection{Lifting}
\label{lifting}

For any set of filters, the analysis and synthesis filter banks shown 
in Figure \ref{fig:decimatereconstruct} can easily be re-expressed as polyphase
filter banks by means of applying {\em matrices} of filters in the subsampled
domain. This is shown in Figure \ref{fig:polyphase}, where $A(z)$ is the $z-$transform
of the analysis polyphase filter matrix, and $S(z)$ is the $z-$transform
of the synthesis polyphase filter matrix (the entries of both matrices being Laurent
polynomials).
\end{informative*}
\setlength{\unitlength}{1em}
\begin{figure}[!ht]
\begin{picture}(50,20)
% Analysis side
\put(0,9){\line(1,0){2}}

\put(2,5){\line(0,1){8}}

\put(2,5){\line(1,0){1}}\put(3,3.5){\framebox(3,3){\Large $z$}}\put(6,5){\line(1,0){1}}
\put(2,13){\line(1,0){5}}

\put(8.5,5){\circle{3}}\put(8,5){\Large $\downarrow 2$}
\put(8.5,13){\circle{3}}\put(8,13){\Large $\downarrow 2$}

\put(10,5){\line(1,0){2}}
\put(10,13){\line(1,0){2}}

\put(12,4){\framebox(8,10){\Large $A(z)$}}

\put(20,5){\line(1,0){1}}
\put(20,13){\line(1,0){1}}

% Synthesis side
\put(23,5){\line(1,0){1}}
\put(23,13){\line(1,0){1}}

\put(24,4){\framebox(8,10){\Large $S(z)$}}

\put(32,5){\line(1,0){2}}
\put(32,13){\line(1,0){2}}

\put(35.5,5){\circle{3}}\put(35,5){\Large $\uparrow 2$}
\put(35.5,13){\circle{3}}\put(35,13){\Large $\uparrow 2$}

\put(37,5){\line(1,0){1}}\put(38,3.5){\framebox(3,3){\Large $z^{-1}$}}\put(41,5){\line(1,0){1}}
\put(37,13){\line(1,0){5}}


\put(41,9){\line(1,0){4}}
\put(42,5){\line(0,1){8}}

\put(42,9){\circle{2}}

\end{picture}
\caption{Polyphase representation of wavelet filter banks}\label{fig:polyphase}
\end{figure}
\begin{informative*}
In this representation, linear combinations of filters operate on both even and 
odd samples to produce new even and odd samples:
\[
\left( 
\begin{array}{c}
x^{out}_e(z) \\
x^{out}_o(z)
\end{array}
\right) 
= A(z)
\left(
\begin{array}{c}
x^{in}_e(z) \\
x^{in}_o(z)
\end{array}
\right)
\]

Since the filter process is invertible, it can be shown that the analysis and 
synthesis matrices are related by $A(z)=(S(z^{-1})^T)^{-1}$. Hence, in particular
both the analysis and synthesis matrices are invertible. It can be shown that
this means that they are (up to gain factors and delays) factorisable into 
products of upper and lower triangular matrices:
\[A(z)= 
\left(
\begin{array}{cc}
1 & a_1(z) \\
0 & 1
\end{array}
\right)
\left(
\begin{array}{cc}
1 & 0 \\
b_1(z) & 1
\end{array}
\right)
\left(
\begin{array}{cc}
1 & a_2(z) \\
0 & 1
\end{array}
\right)
\ldots
\]

Each upper- or lower-triangular polyphase matrix represents a so-called {\em lifting} stage
whereby either even coefficients are modified solely by odd coefficients or odd coefficients
solely by even coefficients. For example, if 
\[
\left( 
\begin{array}{c}
x^{out}_e(z) \\
x^{out}_o(z)
\end{array}
\right) 
=
\left(
\begin{array}{cc}
1 & a(z) \\
0 & 1
\end{array}
\right)
\left(
\begin{array}{c}
x^{in}_e(z) \\
x^{in}_o(z)
\end{array}
\right) 
\]
then
\begin{eqnarray*}
x^{out}_e(z) & = & x^{in}_e(z) + a(z)x^{in}_o(z) \\
x^{out}_o(z) & = & x^{out}_o(z)\\
\end{eqnarray*}
and the filter $a(z)$ has been applied to the odd coefficients and then used to modify
the even coefficients. Not only is this computationally efficient, breaking long
filters into a number of shorter filter applied successively but the
factorisation into such filter stages allows for all computations to be done in-place,
without additional memory.

\end{informative*}
\clearpage
\section{Video systems model and source parameters}
The interpretation of Display Parameters by a display mechanism
interfacing with a compliant decoder is non-normative. However, it
should where possible follow the recommendations and interpretations
specified in this section. Likewise, encoders should ensure that
accurate display parameter information is encoded to maximise the
quality of displayed video.

[Include discussion of YCgCo here. Different matrixing requirements]




\clearpage
\section{Video format defaults}\label{videoformatdefaults}

[TBC!!]
\begin{comment}

\begin{table}[!ht]
\begin{tabular}{|l|c|c|c|c|c|c|c|}
\hline
& \multicolumn{7}{|c|}{{\bf Video Formats}} \\
\hline
 &0 -- Custom &1 -- QSIF & 2 -- QCIF & 3 -- SIF & 4 -- CIF &	5 -- 4SIF	& 6 -- 4CIF \\
\hline
\SLumaWidth&640&176&176&352&352&704&704\\
\SLumaHeight&480&120&144&240&288&480&576\\
\hline
%\VChromaFormat&4:2:0&4:2:0&4:2:0&4:2:0&4:2:0&4:2:0&4:2:0\\
\hline
\SLumaDepth&8&8&8&8&8&8&8\\
\SChromaDepth&8&8&8&8&8&8&8\\
\hline
\end{tabular}
\caption{Default sequence parameters for video formats 0--6}
\end{table}

\begin{table}[!ht]
\begin{tabular}{|l|c|c|c|c|c|c|}
\hline
& \multicolumn{6}{|c|}{{\bf Video Formats}} \\
\hline
   &7 -- SD480 & 8 -- SD576 & 9 -- HD720 &10 -- HD1080 & 11 -- 2KCinema & 12 -- 4KCinema\\
\hline
\SLumaWidth & 720 & 720 & 1280 & 1920 & 2048 & 4096\\
\SLumaHeight & 480 & 576 & 720 & 1080 & 1556 & 3112\\
\hline
%\VChromaFormat & 4:2:0 & 4:2:0 & 4:2:0 & 4:2:0 &4:4:4 & 4:4:4\\
\hline
\SLumaDepth & 8 & 8 & 8 & 8 & 16 & 16\\
\SChromaDepth & 8 & 8 & 8 & 8 & 16 & 16\\
\hline

\end{tabular}
\caption{Default sequence parameters for video formats 7--12}
\end{table}

\begin{table}[!ht]
\begin{tabular}{|l|c|c|c|c|c|c|c|}
\hline
& \multicolumn{7}{|c|}{{\bf Video Formats}} \\
\hline
 &0 -- Custom &1 -- QSIF & 2 -- QCIF & 3 -- SIF & 4 -- CIF &	5 -- 4SIF	& 6 -- 4CIF \\
\hline
\SInterlaced & \false & \false & \false & \false & \false & \false & \false\\
\STopFieldFirst & \true & \true & \true & \true & \true & \true & \true \\

\hline
\SFrameRateNumerator&30&15000&25&15000&25&15000&25\\
\SFrameRateDenominator&1&1001&2&1001&2&1001&2\\
\hline
\SAspectRatioNumerator&1&10&12&10&12&10&12\\
\SAspectRatioDenominator&1&11&11&11&11&11&11\\
\hline
\SCleanWidth&640&176&176&352&352&704&704\\
\SCleanHeight&480&120&144&240&288&480&576\\
\hline
\SLeftOffset&0&0&0&0&0&0&0\\
\STopOffset&0&0&0&0&0&0&0\\
\hline
\SLumaOffset&128&128&128&128&128&128&128\\
\SLumaExcursion&255&255&255&255&255&255&255\\
\SChromaOffset&0&0&0&0&0&0&0\\
\SChromaExcursion&254&254&254&254&254&254&254\\
\hline
\SColourSpecIndex&0&1&2&1&2&1&2\\
\hline
\SColourPrimariesIndex&ITU709&SMPTE C&EBU3213&SMPTE C&EBU3213&SMPTE C&EBU3213\\
\hline
$K_{R}$ & 0.2126 &0.299&0.299&0.299&0.299&0.299&0.299\\
$K_{B}$ & 0.0722 &0.144&0.144&0.144&0.144&0.144&0.144\\
\hline
\STransferFunction&TV&TV&TV&TV&TV&TV&TV\\
\hline

\end{tabular}
\caption{Default source parameters for video formats 0--6}
\end{table}

\begin{table}[!ht]
\begin{tabular}{|l|c|c|c|c|c|c|}
\hline
& \multicolumn{6}{|c|}{{\bf Video Formats}} \\
\hline
   &7 -- SD480 & 8 -- SD576 & 9 -- HD720 &10 -- HD1080 & 11 -- 2KCinema & 12 -- 4KCinema\\
\hline
\SInterlaced & \false & \false & \false & \false & \false & \false \\
\STopFieldFirst & \true & \true & \true & \true & \true & \true\\

\hline
\SFrameRateNumerator & 24000 & 25 & 24 &24 &24 &24 \\
\SFrameRateDenominator& 1001 & 1 & 1 & 1 & 1 & 1 \\
\hline
\SAspectRatioNumerator & 10 & 12 & 1 & 1 & 1 & 1 \\
\SAspectRatioDenominator& 11 & 11 & 1 & 1 & 1 & 1 \\
\hline
\SCleanWidth & 720 & 720 & 1280 & 1920 & 2048 & 4096\\
\SCleanHeight & 480 & 576 & 720 & 1080 & 1536 & 3072\\
\hline
\SLeftOffset & 0 & 0 & 0 & 0 & 0 & 0 \\
\STopOffset & 0 & 0 & 0 & 0 & 0 & 0 \\
\hline
\SLumaOffset & 128 & 128 & 128 & 128 & 32768 & 32768\\
\SLumaExcursion & 235 & 235 & 235 & 235 & 65535 & 65535\\
\SChromaOffset & 0 & 0 & 0 & 0 & 0 & 0\\
\SChromaExcursion & 224 & 224 & 224 & 224 & 65534 & 65534\\
\hline
\SColourSpecIndex & 1 & 2 & 0 & 0 & 3 & 3\\
\hline
\SColourPrimariesIndex & SMPTE C & EBU3213 & ITU709 & ITU709 & Not defined & Not defined\\
\hline
$K_{R}$ & 0.299 & 0.299 & 0.2126 & 0.2126 & 0.25 & 0.25\\
$K_{B}$ & 0.144 & 0.144 & 0.0722 & 0.0722 & 0.25 & 0.25\\
\hline
\STransferFunction&TV&TV&TV&TV&Linear&Linear \\
\hline
\end{tabular}
\caption{Default source parameters for video formats 7--12}
\end{table}

\end{comment}

\clearpage
\section{Profiles and levels}\label{profilelevel}

A compliant Dirac decoder may support a number of different profiles and levels, which determine 
which tools, syntax elements and structures are to be supported, and what decoder resources 
(computational and memory) are required. Profiles deal largely with the former and levels
with the latter, although in particular in a software implementation, the choice of supported tools
also affects decoder resources. Level constraints are expressed in terms of picture sizes and 
formats 

A particular profile will require that particular syntax/syntax elements be used and that decoder
variables or functions are set to particular values. Currently only two profiles are defined, Main and Low-Delay. Low-delay profile corresponds to a decoder capable of decoding the low-delay syntax only. 
Main profile corresponds to a decoder capable of decoding the both the core syntax and low-delay syntax.

Levels determine the size of the
decoded picture buffer $\DecodedBuffer$ and the stream buffer for data entering the decoder. Levels
apply differently in Low-Delay and Main profile, as buffering requirements are quite different (and much
less demanding) for Low-Delay operation.
The operation of these buffers is described in Sections \ref{decodedbufferop} and \ref{streambufferop}.

\subsection{Decoded picture buffer model operation}
\label{decodedbufferop}

In order to define compliance with level and profile requirements, a model of decoded picture buffer
operation is specified.

The decoded picture buffer (DPB) $\DecodedBuffer$ exists to allow for pictures to be presented in display, or 
picture number, order whilst pictures are coded out of order. No constraints are placed on the order
of pictures within a Dirac stream provided that reordering can be carried out within the given size of
the DPB. The DPB size is the minimum delay required by the decoder, ignoring decoding time and stream
buffering delays (Section \ref{streambufferop}).

Note that the DPB is distinct from the reference picture buffer, and the removal of pictures from,
or placement in, the DPB is wholly independent in this specification, although in practice shared
storage may be employed.

Pictures are deemed to be output from the decoder and placed in the DPB instantaneously at multiples
of the (real number) picture sample interval $\PictureInterval$. This interval is defined as follows:
\begin{itemize}
\item if $\Interlaced$ is $\true$ and $\SequentialFields$ is $\true$ (i.e. pictures are fields) then 
\begin{equation*}
\PictureInterval=\dfrac{\FrameRateDenominator}{2*\FrameRateNumerator}
\end{equation*}
\item otherwise (i.e. pictures are frames) then 
\begin{equation*}
\PictureInterval=\dfrac{\FrameRateDenominator}{\FrameRateNumerator}
\end{equation*}
\end{itemize}

If a picture is output from the DPB, it is deemed to occur at the same instant as a picture is
added, resulting in no net increase in DPB occupancy.

A decoder may access a Dirac stream at an AU header. Let $N_n$ denote
the picture number of the $r$th picture in coded order after
the AU header, with $N_0=\AUPictureNumber$. Picture decoding/parsing 
is deemed to start at this point. 

In case $\DPBSize$ is 0, the picture output at time $t_0+n*\PictureInterval$
shall be the picture decoded at time $n*\PictureInterval$. In this case
the picture numbers of pictures output by the decoder shall increment
by 1 with each picture decoded, and form a contiguous sequence of integers.

In case $\DPBSize>0$, a compliant stream shall satisfy the following
conditions
\begin{enumerate}
\item For $k\geq 0$, there exists some $n$ with $N_n=P_0+k$ with
\[k\leq n\leq k+\DPBSize+2\]
\item For $n\geq\DPBSize+2$, picture $N_n$ is decodeable 
using only data received subsequent to the AU header
\end{enumerate}

These conditions are sufficient to ensure that a buffer of size $\DPBSize$
is sufficient to reorder all pictures to produce a stream of pictures
with contiguous picture numbers, and these pictures shall be capable of being fully 
decoded and displayed after a delay equal to at most $\DPBSize+2$ picture intervals.

\begin{informative*}
\subsubsection{Example operation (Informative)}
Consider a conventional MPEG-2 style `IBBP' Group of Pictures (GOP), consisting
of intra pictures (I), forward-predicted pictures (P) and bi-directionally
predicted pictures (B). In picture order, this is
\[\ldots\ I_{N_0}\ B_{N_0+1}\ B_{N_0+2}\ P_{N_0+3}\ B_{N_0+4}\ B_{N_0+5}\ 
P_{N_0+6}\ B_{N_0+7}\ B_{N_0+8}\ P_{N_0+9}\ B_{N_0+10}\ B_{N_0+11}\ I_{N_0+12}
\ B_{N_0+13}\ \ldots\]

Each P-picture is predicted from the previous P or I picture, and each B-picture
from the previous and subsequent P- or I-picture. Hence coded order is:
\[\ldots\ I_{N_0}\ B_{N_0-2}\ B_{N_0-1}\ P_{N_0+3}\ B_{N_0+1}\ B_{N_0+2}\ 
P_{N_0+6}\ B_{N_0+4}\ B_{N_0+5}\ P_{N_0+9}\ B_{N_0+7}\ B_{N_0+8}\ I_{N_0+12} 
\ B_{N_0+10}\ B_{N_0+11}\ \ldots\]
with random access at each intra picture. A decoder accessing the stream at
$I_{N_0}$ would require a buffer of 2 pictures, since B-pictures can be
displayed immediately. The DPB would operate as follows:
\begin{enumerate}
\item Time $t_0$. \\
Receive $I_{N_0}$ and store in the DPB.\\
DPB=$\{I_{N_0}\}$\\
Displayed picture=
\item Time $t_0+\PictureInterval$. \\
Receive $B_{N_0-2}$ (undecoded) and discard:\\
DPB=$\{I_{N0}\}$\\
Displayed picture=
\item Time $t_0+2*\PictureInterval$. \\
Receive $B_{N_0-1}$ (undecoded) and discard:\\
DPB=$\{I_{N0}\}$\\
Displayed picture=
\item Time $t_0+3*\PictureInterval$. \\
Receive $P_{N_0+3}$ (decoded) and place in DPB, 
display $I_{N_0}$ and remove from DPB:\\
DPB=$\{P_{N_0+3}\}$\\
Displayed picture=$I_{N_0}$
\item Time $t_0+4*\PictureInterval$. \\
Receive $B_{N_0+1}$ (decoded) and display immediately:\\
DPB=$\{P_{N_0+3}\}$\\
Displayed picture=$B_{N_0+1}$
\item Time $t_0+5*\PictureInterval$. \\
Receive $B_{N_0+2}$ (decoded) and display immediately:\\
DPB=$\{P_{N_0+3}\}$\\
Displayed picture=$B_{N_0+2}$
\item Time $t_0+6*\PictureInterval$. \\
Receive $P_{N_0+6}$ (decoded) and place in DPB, display 
$P_{N_0+3}$ and remove from the DPB:\\
DPB=$\{P_{N_0+6}\}$\\
Displayed picture=$P_{N_0+3}$
\item Time $t_0+7*\PictureInterval$. \\
Receive $B_{N_0+4}$ (decoded) and display immediately:\\
DPB=$\{P_{N_0+6}\}$\\
Displayed picture=$B_{N_0+4}$
\item Time $t_0+8*\PictureInterval$. \\
Receive $B_{N_0+5}$ (decoded) and display immediately:\\
DPB=$\{P_{N_0+6}\}$\\
Displayed picture=$B_{N_0+5}$
\item Time $t_0+9*\PictureInterval$. \\
Receive $P_{N_0+9}$ (decoded) and place in DPB, display 
$P_{N_0+6}$ and remove from the DPB:\\
DPB=$\{P_{N_0+9}\}$\\
Displayed picture=$P_{N_0+6}$
\item $\ldots$
\end{enumerate}

Note that a better subjective experience can be gleaned if the
decoder outputs the initial intra frame $I_{N_0}$ whilst buffering
pictures initially.

\end{informative*}

\subsection{Stream buffer model operation}
\label{streambufferop}

In order to define compliance with level and profile requirements, a model of decoded picture buffer
operation is specified.

The decoder stream buffer (DSB) $\StreamBuffer$ exists to allow for data to be transmitted at a constant bit
rate (CBR), yet consist of varying bit allocations to individual pictures. It is well-known that input buffers
operate in mirror-image to output buffers, and so the specification of the ISB implicitly defines
how an identically-sized encoder stream buffer (ESB) would operate to ensure correct CBR operation, although
in practice some encoder-side headroom may be required.

[TBC]

\subsection{Supported levels}
[TBC]


%\clearpage
%\section{Parse diagrams}\label{parsediagrams}

\subsection{Core syntax}

% Stream

\setlength{\unitlength}{1em}
\begin{figure}[!ht]
\centering
\begin{picture}(20,12)
\put(0,5){\vector(1,0){5}}
\put(10,5){\oval(10,4.7)\put(-2,-0.5){Sequence}}
\put(15,5){\vector(1,0){5}}
\put(2.5,5){\line(0,1){5}}
\put(17.5,5){\line(0,1){5}}
\put(17.5,10){\vector(-1,0){10}}
\put(2.5,10){\line(1,0){10}}
\end{picture}
\caption{Stream}\label{fig:stream}
\end{figure}

% Sequence

\setlength{\unitlength}{1em}
\begin{figure}[!ht]
\centering
\begin{picture}(45,12)

\put(0,3){\vector(1,0){5}}
\put(10,3){\oval(10,4.7)\put(-1.75,0.5){Parse} \put(-1.25,-1){Info}}
\put(15,3){\vector(1,0){5}}
\put(25,3){\oval(10,4.7)\put(-1.75,0.5){Access}\put(-1.3,-1){Unit}}
\put(30,3){\vector(1,0){10}}
\put(17.5,3){\line(0,1){5}}
\put(32.5,3){\line(0,1){5}}
\put(32.5,8){\vector(-1,0){10}}
\put(17.5,8){\line(1,0){10}}

\end{picture}
\caption{Sequence}\label{fig:sequence}
\end{figure}

% Parse info header

\setlength{\unitlength}{1em}
\begin{figure}[!ht]
\centering
\begin{picture}(45,8)
\put(0,3){\vector(1,0){3}}
\put(7,3){\oval(8,4.7) \put(-2,.5){Parse Info}\put(-2,-1){Prefix}}
\put(11,3){\vector(1,0){2}}
\put(17,3){\oval(8,4.7)\put(-2,.5){Parse} \put(-2,-1){Code}}
\put(21,3){\vector(1,0){2}}
\put(27,3){\oval(8,4.7)\put(-2,.5){Next Parse} \put(-2,-1){Offset}}
\put(31,3){\vector(1,0){2}}
\put(37,3){\oval(8,4.7)\put(-2,.5){Previous} \put(-2,-1){Parse Offset}}
\put(41,3){\vector(1,0){3}}
\end{picture}
\caption{Parse Info}\label{fig:parseinfo}
\end{figure}

% AU
\setlength{\unitlength}{1em}
\begin{figure}[!ht]
\centering
\begin{picture}(50,8)
\put(0,3){\vector(1,0){4}}
\put(8,3){\oval(8,4.7) \put(-2.1,.5){Access Unit}\put(-1.65,-1){Header}}
\put(12,3){\vector(1,0){2}}
\put(18,3){\oval(8,4.7)\put(-1.6,.5){Parse} \put(-1.2,-1){Info}}
\put(22,3){\vector(1,0){4}}
\put(30,3){\oval(8,4.7)\put(-2,-.5){Picture}}
\put(34,3){\vector(1,0){2}}
\put(40,3){\oval(8,4.7)\put(-1.6,.5){Parse} \put(-1.2,-1){Info}}
\put(44,3){\vector(1,0){4}}
\put(46,3){\line(0,1){5}}
\put(24,3){\line(0,1){5}}
\put(35,8){\line(-1,0){11}}
\put(46,8){\vector(-1,0){11}}

\end{picture}
\caption{Access Unit}\label{fig:accessunit}
\end{figure}

%\clearpage

% AU header
\setlength{\unitlength}{1em}
\begin{figure}[!ht]
\centering
\begin{picture}(40,8)
\put(0,3){\vector(1,0){4}}
\put(8,3){\oval(8,4.7) \put(-2,.5){Parse}\put(-2.2,-1){Parameters}}
\put(12,3){\vector(1,0){4}}
\put(20,3){\oval(8,4.7)\put(-2,.5){Sequence} \put(-2.1,-1){Parameters}}
\put(24,3){\vector(1,0){4}}
\put(32,3){\oval(8,4.7)\put(-2,.5){Source} \put(-2.4,-1){Parameters}}
\put(36,3){\vector(1,0){4}}
\end{picture}
\caption{Access Unit header}\label{fig:auheader}
\end{figure}


%% AU parse params
\setlength{\unitlength}{1em}
\begin{figure}[!ht]
\centering
\begin{picture}(45,5)
\put(0,3){\vector(1,0){3}}
\put(7,3){\oval(8,4.7) \put(-2,1.){Access Unit}\put(-2,-0.50){Picture}\put(-2,-2){Number}}
\put(11,3){\vector(1,0){2}}
\put(17,3){\oval(8,4.7)\put(-2,.5){Version} \put(-2,-1){Number}}
\put(21,3){\vector(1,0){2}}
\put(27,3){\oval(8,4.7)\put(-2,-.5){Profile}} 
\put(31,3){\vector(1,0){2}}
\put(37,3){\oval(8,4.7)\put(-2,-.5){Level}}
\put(41,3){\vector(1,0){3}}
\end{picture}
\caption{Access Unit Parse Parameters}\label{fig:parseparameters}
\end{figure}


% AU sequence params
\setlength{\unitlength}{1em}
\begin{figure}[!ht]
\centering
\begin{picture}(45,8)
\put(0,3){\vector(1,0){3}}
\put(7,3){\oval(8,4.7) \put(-2,.5){Video}\put(-2,-1){Format}}
\put(11,3){\vector(1,0){2}}
\put(17,3){\oval(8,4.7)\put(-2,.5){Image} \put(-2,-1){Dimensions}}
\put(21,3){\vector(1,0){2}}
\put(27,3){\oval(8,4.7)\put(-2,.5){Chroma} \put(-2,-1){Format}}
\put(31,3){\vector(1,0){2}}
\put(37,3){\oval(8,4.7)\put(-2,.5){Video} \put(-2,-1){Depth}}
\put(41,3){\vector(1,0){3}}
\end{picture}
\caption{Sequence Parameters}\label{fig:sequenceparameters}
\end{figure}


% Setting image dimensions
\setlength{\unitlength}{1em}
\begin{figure}[!ht]
\centering
\begin{picture}(35,12)
\put(0,3){\vector(1,0){3}}
\put(7,3){\oval(8,4.7) \put(-2,1){Custom}\put(-2,-.5){Dimensions} \put(-2,-2){Flag}}
\put(12,3){\line(0,1){5}}
\put(12,8){\vector(1,0){1}}
\put(17,8){\oval(8,4.7)\put(-2,.5){Luma} \put(-2,-1){Width}}
\put(21,8){\vector(1,0){2}}
\put(27,8){\oval(8,4.7)\put(-2,.5){Luma} \put(-2,-1){Height}}
\put(32,3){\line(0,1){5}}
\put(31,8){\vector(1,0){1}}
\put(11,3){\vector(1,0){24}}
\end{picture}
\caption{Image dimensions}\label{fig:imagedimensions}
\end{figure}

% Chroma formats

\setlength{\unitlength}{1em}
\begin{figure}[!ht]
\centering
\begin{picture}(30,12)
\put(0,3){\vector(1,0){3}}
\put(7,3){\oval(8,4.7) \put(-2,1){Chroma}\put(-2,-.5){Format} \put(-2,-2){Flag}}
\put(11,3){\line(1,0){2}}
\put(13,3){\line(0,1){5}}
\put(13,8){\vector(1,0){2}}
\put(19,8){\oval(8,4.7)\put(-2,1){Chroma}\put(-2,-.5){Format} \put(-2,-2){Index}}
\put(23,8){\vector(1,0){2}}
\put(25,8){\line(0,-1){5}}
\put(13,3){\vector(1,0){15}}
\end{picture}
\caption{Chroma formats}\label{fig:chromaformats}
\end{figure}

%\clearpage

% Video depth

\setlength{\unitlength}{1em}
\begin{figure}[!ht]
\centering
\begin{picture}(30,12)
\put(0,3){\vector(1,0){3}}
\put(7,3){\oval(8,4.7) \put(-2,1){Video}\put(-2,-.5){Depth} \put(-2,-2){Flag}}
\put(11,3){\line(1,0){2}}
\put(13,3){\line(0,1){5}}
\put(13,8){\vector(1,0){2}}
\put(19,8){\oval(8,4.7)\put(-2,1){Video}\put(-2,-.5){Depth} \put(-2,-2){Value}}
\put(23,8){\vector(1,0){2}}
\put(25,8){\line(0,-1){5}}
\put(13,3){\vector(1,0){15}}
\end{picture}
\caption{Video Depth}\label{fig:videodepth}
\end{figure}

% AU source parameters

\setlength{\unitlength}{1em}
\begin{figure}[!ht]
\centering
\begin{picture}(40,14)
\put(0,11){\vector(1,0){4}}
\put(8,11){\oval(8,4.7) \put(-2,.5){Scan}\put(-2,-1){Format}}
\put(12,11){\vector(1,0){4}}
\put(20,11){\oval(8,4.7)\put(-2,.5){Frame} \put(-2,-1){Rate}}
\put(24,11){\vector(1,0){4}}
\put(32,11){\oval(8,4.7)\put(-2,.5){Aspect} \put(-2,-1){Ratio}}
\put(36,11){\vector(1,0){4}}
\put(40,11){\line(0,-1){4}}
\put(40,7){\vector(-1,0){17.5}} 
\put(22.5,7){\line(-1,0){17.5}}
\put(5,7){\line(0,-1){4}}
\put(5,3){\vector(1,0){4}}
\put(13,3){\oval(8,4.7) \put(-2,.5){Clean}\put(-2,-1){Area}}
\put(17,3){\vector(1,0){4}}
\put(25,3){\oval(8,4.7)\put(-2,.5){Signal} \put(-2,-1){Range}}
\put(29,3){\vector(1,0){4}}
\put(37,3){\oval(8,4.7)\put(-2,.5){Colour} \put(-2,-1){Specification}}
\put(41,3){\vector(1,0){4}}

\end{picture}
\caption{Access Unit Source Parameters}\label{fig:sourceparameters}
\end{figure}

% Scan format

\setlength{\unitlength}{1em}
\begin{figure}[!ht]
\centering
\begin{picture}(30,12)
\put(0,3){\vector(1,0){3}}
\put(7,3){\oval(8,4.7) \put(-2,1){Scan}\put(-2,-.5){Format} \put(-2,-2){Flag}}
\put(11,3){\line(1,0){2}}
\put(13,3){\line(0,1){5}}
\put(13,8){\vector(1,0){2}}
\put(19,8){\oval(8,4.7)\put(-2,-0.5){Interlace}}
\put(23,8){\vector(1,0){2}}
\put(25,8){\line(0,-1){5}}
\put(13,3){\vector(1,0){15}}
\end{picture}
\caption{Scan Format}\label{fig:scanformat}
\end{figure}

% Interlace

\setlength{\unitlength}{1em}
\begin{figure}[!ht]
\centering
\begin{picture}(48,15)
\put(0,3){\vector(1,0){3}}
\put(6.5,3){\oval(7,4.7) \put(-2,0.5){Interlaced}\put(-2,-1){Source}}
\put(10,3){\line(1,0){1}}
\put(11,3){\vector(1,0){38}}
\put(11,3){\line(0,1){5}}
\put(11,8){\vector(1,0){1}}
\put(15.5,8){\oval(7,4.7)\put(-2,1){Field}\put(-2,-.5){Dominance} \put(-2,-2){Flag}}
\put(19,8){\line(1,0){1}}
\put(20,8){\vector(1,0){10}}
\put(20,8){\line(0,1){5}}
\put(20,13){\vector(1,0){1}}
\put(24.5,13){\oval(7,4.7)\put(-2,1){Top}\put(-2,-.5){Field} \put(-2,-2){First}}
\put(28,13){\line(1,0){1}}
\put(29,13){\vector(0,-1){5}}
\put(33.5,8){\oval(7,4.7)\put(-2,1){Field}\put(-2,-.5){Interleaving} \put(-2,-2){Flag}}
\put(37,8){\line(1,0){1}}
\put(38,8){\vector(1,0){10}}
\put(38,8){\line(0,1){5}}
\put(38,13){\vector(1,0){1}}
\put(42.5,13){\oval(7,4.7)\put(-2,0.5){Sequential} \put(-2,-1){Fields}}
\put(46,13){\line(1,0){1}}
\put(47,13){\vector(0,-1){5}}
\put(48,8){\line(0,-1){5}}
\end{picture}
\caption{Interlace}\label{fig:interlace}
\end{figure}

\clearpage
% frame rate
\setlength{\unitlength}{1em}
\begin{figure}[!ht]
\centering
\begin{picture}(48,15)
\put(0,3){\vector(1,0){3}}
\put(6.5,3){\oval(7,4.7) \put(-2,1){Frame}\put(-2,-.5){Rate} \put(-2,-2){Flag}}
\put(10,3){\line(1,0){1}}
\put(11,3){\vector(1,0){35}}
\put(11,3){\line(0,1){5}}
\put(11,8){\vector(1,0){1}}
\put(15.5,8){\oval(7,4.7)\put(-2,1){Frame}\put(-2,-.5){Rate} \put(-2,-2){Index}}
\put(19,8){\line(1,0){1}}
\put(20,8){\vector(1,0){22}}
\put(20,8){\line(0,1){5}}
\put(20,13){\vector(1,0){1}}
\put(24.5,13){\oval(7,4.7)\put(-2.5,1){Frame}\put(-2.5,-.5){Rate} \put(-2.5,-2){Numerator}}

\put(28,13){\vector(1,0){2}}
\put(33.5,13){\oval(7,4.7)\put(-2.5,0.5){Frame Rate}\put(-3,-1){Denominator}}
\put(37,13){\line(1,0){1}}
\put(38,13){\vector(0,-1){5}}

\put(42,8){\line(0,-1){5}}
\end{picture}
\caption{Frame Rate}\label{fig:framerate}
\end{figure}

% Aspect ratio
\setlength{\unitlength}{1em}
\begin{figure}[!ht]
\centering
\begin{picture}(48,15)
\put(0,3){\vector(1,0){3}}
\put(6.5,3){\oval(7,4.7) \put(-2,1){Aspect}\put(-2,-.5){Ratio} \put(-2,-2){Flag}}
\put(10,3){\line(1,0){1}}
\put(11,3){\vector(1,0){35}}
\put(11,3){\line(0,1){5}}
\put(11,8){\vector(1,0){1}}
\put(15.5,8){\oval(7,4.7)\put(-2,1){Aspect}\put(-2,-.5){Ratio} \put(-2,-2){Index}}
\put(19,8){\line(1,0){1}}
\put(20,8){\vector(1,0){22}}
\put(20,8){\line(0,1){5}}
\put(20,13){\vector(1,0){1}}
\put(24.5,13){\oval(7,4.7)\put(-2.5,1){Aspect}\put(-2.5,-.5){Ratio} \put(-2.5,-2){Numerator}}

\put(28,13){\vector(1,0){2}}
\put(33.5,13){\oval(7,4.7)\put(-2.5,0.5){Aspect Ratio}\put(-3,-1){Denominator}}
\put(37,13){\vector(1,0){1}}
\put(38,13){\line(0,-1){5}}

\put(42,8){\line(0,-1){5}}
\end{picture}
\caption{Aspect Ratio}\label{fig:aspectratio}
\end{figure}

% Clean area

\setlength{\unitlength}{1em}
\begin{figure}[!ht]
\centering
\begin{picture}(48,15)
\put(0,3){\vector(1,0){3}}
\put(6.5,3){\oval(7,4.7) \put(-2,1){Clean}\put(-2,-.5){Area} \put(-2,-2){Flag}}
\put(10,3){\line(1,0){1}}
\put(11,3){\vector(1,0){38}}
\put(11,3){\line(0,1){5}}
\put(11,8){\vector(1,0){1}}
\put(15.5,8){\oval(7,4.7)\put(-2,0.5){Clean} \put(-2,-1){Width}}
\put(19,8){\vector(1,0){2}}
\put(24.5,8){\oval(7,4.7)\put(-2,0.5){Clean} \put(-2,-1){Height}}
\put(28,8){\vector(1,0){2}}
\put(33.5,8){\oval(7,4.7)\put(-2,0.5){Left} \put(-2,-1){Offset}}
\put(37,8){\vector(1,0){2}}
\put(42.5,8){\oval(7,4.7)\put(-2,0.5){Top} \put(-2,-1){Offset}}
\put(46,8){\vector(1,0){2}}
\put(48,8){\line(0,-1){5}}
\end{picture}
\caption{Clean Area}\label{fig:cleanarea}
\end{figure}

% Signal range

\setlength{\unitlength}{1em}
\begin{figure}[!ht]
\centering
\begin{picture}(48,26)
\put(0,3){\vector(1,0){3}}
\put(6.5,3){\oval(7,4.7) \put(-2,1){Signal}\put(-2,-.5){Range} \put(-2,-2){Flag}}
\put(10,3){\line(1,0){1}}
\put(11,3){\vector(1,0){35}}
\put(11,3){\line(0,1){5}}
\put(11,8){\vector(1,0){1}}
\put(15.5,8){\oval(7,4.7)\put(-2,1){Signal}\put(-2,-.5){Range} \put(-2,-2){Index}}
\put(19,8){\line(1,0){1}}
\put(20,8){\vector(1,0){23}}
\put(20,8){\line(0,1){14}}
\put(20,22){\vector(1,0){1}}
\put(24.5,22){\oval(7,4.7)\put(-2,.5){Luma}\put(-2,-1){Offset}}
\put(28,22){\vector(1,0){2}}
\put(33.5,22){\oval(7,4.7)\put(-2,0.5){Luma}\put(-2,-1){Excursion}}
\put(37,22){\line(1,0){1}}
\put(38,22){\vector(0,-1){4.5}}
\put(38,17.5){\vector(-1,0){8}}
\put(30,17.5){\line(-1,0){8}}
\put(22,17.5){\line(0,-1){4.5}}
\put(22,13){\vector(1,0){3}}
\put(20,8){\line(0,1){5}}
\put(28.5,13){\oval(7,4.7)\put(-2,.5){Chroma}\put(-2,-1){Offset}}
\put(32,13){\vector(1,0){2}}
\put(37.5,13){\oval(7,4.7)\put(-2,0.5){Chroma}\put(-2,-1){Excursion}}
\put(41,13){\vector(1,0){1}}
\put(42,13){\line(0,-1){5}}
\put(43,8){\line(0,-1){5}}
\end{picture}
\caption{Signal Range}\label{fig:signalrange}
\end{figure}

% Colour specification
\setlength{\unitlength}{1em}
\begin{figure}[!ht]
\centering
\begin{picture}(48,25)
\put(0,3){\vector(1,0){3}}
\put(6.5,3){\oval(7,4.7) \put(-2,1){Colour}\put(-2,-.5){Spec} \put(-2,-2){Flag}}
\put(10,3){\line(1,0){1}}
\put(11,3){\vector(1,0){35}}
\put(11,3){\line(0,1){5}}
\put(11,8){\vector(1,0){1}}
\put(15.5,8){\oval(7,4.7)\put(-2,1){Colour}\put(-2,-.5){Spec} \put(-2,-2){Index}}
\put(19,8){\line(1,0){1}}
\put(20,8){\vector(1,0){23}}
\put(20,8){\line(0,1){14}}
\put(20,22){\vector(1,0){1}}
\put(24.5,22){\oval(7,4.7)\put(-2,.5){Colour}\put(-2,-1){Primaries}}

\put(28,22){\vector(1,0){2}}
\put(33.5,22){\oval(7,4.7)\put(-2,0.5){Colour}\put(-2,-1){Matrix}}
\put(37,22){\line(1,0){1}}
\put(38,22){\vector(0,-1){4.5}}
\put(38,17.5){\vector(-1,0){8}}
\put(30,17.5){\line(-1,0){6}}
\put(24,17.5){\line(0,-1){4.5}}
\put(24,13){\vector(1,0){5.5}}
\put(20,8){\line(0,1){5}}
\put(33,13){\oval(7,4.7)\put(-2,.5){Transfer}\put(-2,-1){Function}}
\put(36.5,13){\vector(1,0){5.5}}
\put(42,13){\line(0,-1){5}}
\put(43,8){\line(0,-1){5}}
\end{picture}
\caption{Colour Specification}\label{fig:colourspec}
\end{figure}
\clearpage

\setlength{\unitlength}{1em}
\begin{figure}[!ht]
\centering
\begin{picture}(30,12)
\put(0,3){\vector(1,0){3}}
\put(7,3){\oval(8,4.7) \put(-2,1){Colour}\put(-2,-.5){Primaries} \put(-2,-2){Flag}}
\put(11,3){\line(1,0){2}}
\put(13,3){\line(0,1){5}}
\put(13,8){\vector(1,0){2}}
\put(19,8){\oval(8,4.7)\put(-2,1){Colour}\put(-2,-.5){Primaries} \put(-2,-2){Index}}
\put(23,8){\vector(1,0){2}}
\put(25,8){\line(0,-1){5}}
\put(13,3){\vector(1,0){15}}
\end{picture}
\caption{Colour Primaries}\label{fig:colourprimaries}
\end{figure}


\setlength{\unitlength}{1em}
\begin{figure}[!ht]
\centering
\begin{picture}(30,12)
\put(0,3){\vector(1,0){3}}
\put(7,3){\oval(8,4.7) \put(-2,1){Colour}\put(-2,-.5){Matrix} \put(-2,-2){Flag}}
\put(11,3){\line(1,0){2}}
\put(13,3){\line(0,1){5}}
\put(13,8){\vector(1,0){2}}
\put(19,8){\oval(8,4.7)\put(-2,1){Colour}\put(-2,-.5){Matrix} \put(-2,-2){Index}}
\put(23,8){\vector(1,0){2}}
\put(25,8){\line(0,-1){5}}
\put(13,3){\vector(1,0){15}}
\end{picture}
\caption{Colour Matrix}\label{fig:colourmatrix}
\end{figure}


\setlength{\unitlength}{1em}
\begin{figure}[!ht]
\centering
\begin{picture}(30,12)
\put(0,3){\vector(1,0){3}}
\put(7,3){\oval(8,4.7) \put(-2,1){Transfer}\put(-2,-.5){Function} \put(-2,-2){Flag}}
\put(11,3){\line(1,0){2}}
\put(13,3){\line(0,1){5}}
\put(13,8){\vector(1,0){2}}
\put(19,8){\oval(8,4.7)\put(-2,1){Transfer}\put(-2,-.5){Function} \put(-2,-2){Index}}
\put(23,8){\vector(1,0){2}}
\put(25,8){\line(0,-1){5}}
\put(13,3){\vector(1,0){15}}
\end{picture}
\caption{Transfer Function}\label{fig:transferfunction}
\end{figure}
%%%%%%%%%%%%%%%%%%%%%%%%%%%%%%%%%%%%
% Picture

\setlength{\unitlength}{1em}
\begin{figure}[!ht]
\centering
\begin{picture}(40,12)
\put(0,3){\vector(1,0){3}}
\put(7,3){\oval(8,4.7) \put(-2,.5){Picture}\put(-2,-1){Header} }
\put(11,3){\line(1,0){2}}
\put(13,3){\line(0,1){5}}
\put(13,8){\vector(1,0){2}}
\put(19,8){\oval(8,4.7)\put(-2,.5){Picture}\put(-2,-1){Prediction}}
\put(23,8){\vector(1,0){2}}
\put(25,8){\line(0,-1){5}}
\put(13,3){\vector(1,0){15}}
\put(32,3){\oval(8,4.7)\put(-2,.5){Wavelet}\put(-2,-1){Transform}}
\put(36,3){\vector(1,0){3}}
\end{picture}
\caption{Picture}\label{fig:picture}
\end{figure}

\clearpage

% Picture header
\setlength{\unitlength}{1em}
\begin{figure}[!ht]
\centering
\begin{picture}(40,12)
\put(0,3){\vector(1,0){3}}
\put(7,3){\oval(8,4.7) \put(-2,1){Picture}\put(-2,-.5){Number} }
\put(11,3){\line(1,0){2}}
\put(13,3){\line(0,1){5}}
\put(13,8){\vector(1,0){2}}
\put(19,8){\oval(8,4.7)\put(-2,1){Reference}\put(-2,-.5){Picture}\put(-2,-2){Numbers}}
\put(23,8){\vector(1,0){2}}
\put(25,8){\line(0,-1){5}}
\put(13,3){\vector(1,0){15}}
\put(32,3){\oval(8,4.7)\put(-2,1){Retired}\put(-2,-.5){Picture} \put(-2,-2){List}}
\put(36,3){\vector(1,0){3}}

\end{picture}
\caption{Picture header}\label{fig:pictureheader}
\end{figure}

\setlength{\unitlength}{1em}
\begin{figure}[!ht]
\centering
\begin{picture}(30,12)
\put(0,3){\vector(1,0){3}}
\put(7,3){\oval(8,4.7) \put(-2,1){Reference 1}\put(-2,-.5){Picture}\put(-2,-2){Offset} }
\put(11,3){\line(1,0){2}}
\put(13,3){\line(0,1){5}}
\put(13,8){\vector(1,0){2}}
\put(19,8){\oval(8,4.7)\put(-2,1){Reference 2}\put(-2,-.5){Picture}\put(-2,-2){Offset}}
\put(23,8){\vector(1,0){2}}
\put(25,8){\line(0,-1){5}}
\put(13,3){\vector(1,0){15}}
\end{picture}
\caption{Reference Picture Numbers}\label{fig:refpicturenumbers}
\end{figure}

\setlength{\unitlength}{1em}
\begin{figure}[!ht]
\centering
\begin{picture}(30,12)
\put(0,3){\vector(1,0){3}}
\put(7,3){\oval(8,4.7) \put(-2,.5){List}\put(-2,-1){Length}}
\put(11,3){\line(1,0){2}}
\put(13,3){\line(0,1){5}}
\put(13,8){\vector(1,0){2}}
\put(14,8){\line(0,1){4}}
\put(24,12){\vector(-1,0){5}}
\put(19,12){\line(-1,0){5}}
\put(19,8){\oval(8,4.7)\put(-2,1){Retired}\put(-2,-.5){Picture}\put(-2,-2){Offset}}
\put(24,8){\line(0,1){4}}
\put(23,8){\vector(1,0){2}}
\put(25,8){\line(0,-1){5}}
\put(13,3){\vector(1,0){15}}
\end{picture}
\caption{Retired Picture List}\label{fig:retiredpicturelist}
\end{figure}

% Picture prediction
\setlength{\unitlength}{1em}
\begin{figure}[!ht]
\centering
\begin{picture}(30,8)
\put(0,3){\vector(1,0){3}}
\put(7,3){\oval(8,4.7)\put(-2.5,1){Picture} \put(-2.5,-.5){Prediction} \put(-2.5,-2){Parameters}}
\put(11,3){\vector(1,0){5}}
\put(20,3){\oval(8,4.7)\put(-2,0.5){Block motion} \put(-2,-1){Data}}
\put(24,3){\vector(1,0){5}}
\end{picture}
\caption{Picture Prediction}\label{fig:pictureprediction}
\end{figure}

% Picture prediction parameters
\setlength{\unitlength}{1em}
\begin{figure}[!ht]
\centering
\begin{picture}(51,8)
\put(0,3){\vector(1,0){2.5}}
\put(6,3){\oval(7,4.7) \put(-2,.5){Block}\put(-2,-1){Parameters}}
\put(9.5,3){\vector(1,0){2}}
\put(15,3){\oval(7,4.7)\put(-2,1){Motion} \put(-2,-.5){Vector}\put(-2,-2){Precision}}
\put(18.5,3){\vector(1,0){2}}
\put(24,3){\oval(7,4.7)\put(-2,.5){Global} \put(-2,-1){Motion}}
\put(27.5,3){\vector(1,0){2}}
\put(33,3){\oval(7,4.7)\put(-2,1){Picture} \put(-2,-.5){Prediction}\put(-2,-2){Mode}}
\put(36.5,3){\vector(1,0){2}}
\put(42,3){\oval(7,4.7)\put(-2,.5){Picture} \put(-2,-1){Weights}}
\put(45.5,3){\vector(1,0){3}}
\end{picture}
\caption{Picture prediction parameters}\label{fig:picpredparams}
\end{figure}

% Block parameters
\setlength{\unitlength}{1em}
\begin{figure}[!ht]
\centering
\begin{picture}(48,25)
\put(0,3){\vector(1,0){3}}
\put(6.5,3){\oval(7,4.7) \put(-2,1){Block}\put(-2,-.5){Parameters} \put(-2,-2){Flag}}
\put(10,3){\line(1,0){1}}
\put(11,3){\vector(1,0){35}}
\put(11,3){\line(0,1){5}}
\put(11,8){\vector(1,0){1}}
\put(15.5,8){\oval(7,4.7)\put(-2,1){Block}\put(-2,-.5){Parameters} \put(-2,-2){Index}}
\put(19,8){\line(1,0){1}}
\put(20,8){\vector(1,0){23}}
\put(20,8){\line(0,1){14}}
\put(20,22){\vector(1,0){1}}
\put(24.5,22){\oval(7,4.7)\put(-2,1){Luma}\put(-2,-.5){Block} \put(-2,-2){Width}}

\put(28,22){\vector(1,0){2}}
\put(33.5,22){\oval(7,4.7)\put(-2,1){Luma}\put(-2,-.5){Block } \put(-2,-2){Height}}
\put(37,22){\line(1,0){1}}
\put(38,22){\vector(0,-1){4.5}}
\put(38,17.5){\vector(-1,0){8}}
\put(30,17.5){\line(-1,0){8}}
\put(22,17.5){\line(0,-1){4.5}}
\put(22,13){\vector(1,0){3}}
\put(20,8){\line(0,1){5}}
\put(28.5,13){\oval(7,4.7)\put(-2.5,1){Horizontal}\put(-2.5,-.5){Luma Block}\put(-2.5,-2){Separation}}

\put(32,13){\vector(1,0){2}}
\put(37.5,13){\oval(7,4.7)\put(-2.5,1){Vertical}\put(-2.5,-.5){Luma Block}\put(-2.5,-2){Separation}}
\put(41,13){\vector(1,0){1}}
\put(42,13){\line(0,-1){5}}
\put(43,8){\line(0,-1){5}}

\end{picture}
\caption{Block Parameters}\label{fig:blockparameters}
\end{figure}

%Motion vector precision
\setlength{\unitlength}{1em}
\begin{figure}[!ht]
\centering
\begin{picture}(40,12)
\put(0,3){\vector(1,0){3}}
\put(7,3){\oval(8,4.7) \put(-2,1){M-Vector}\put(-2,-.5){Precision}\put(-2,-2){Flag}}
\put(11,3){\line(1,0){2}}
\put(13,3){\line(0,1){5}}
\put(13,8){\vector(1,0){2}}
\put(19,8){\oval(8,4.7)\put(-2,1){M-Vector}\put(-2,-.5){Precision}\put(-2,-2){Bits}}
\put(23,8){\vector(1,0){2}}
\put(25,8){\line(0,-1){5}}
\put(13,3){\vector(1,0){15}}
\end{picture}
\caption{Motion Vector Precision}\label{fig:motionvectorprecision}
\end{figure}

% Global motion
\setlength{\unitlength}{1em}
\begin{figure}[!ht]
\centering
\begin{picture}(35,12)
\put(0,3){\vector(1,0){3}}
\put(7,3){\oval(8,4.7) \put(-2.5,1){Using}\put(-2.5,-.5){Global} \put(-2.5,-2){Motion Flag}}
\put(12,3){\line(0,1){5}}
\put(12,8){\vector(1,0){2}}
\put(13,8){\line(0,1){4}}
\put(13,12){\line(1,0){5}}
\put(23,12){\vector(-1,0){5}}
\put(18,8){\oval(8,4.7)\put(-2.5,1){Global}\put(-2.5,-.5){Motion} \put(-2.5,-2){Parameters}}
\put(23,8){\line(0,1){4}}
\put(24,3){\line(0,1){5}}
\put(22,8){\vector(1,0){2}}
\put(11,3){\vector(1,0){24}}
\end{picture}
\caption{Global Motion}\label{fig:globalmotion}
\end{figure}

% Global motion parameters
\setlength{\unitlength}{1em}
\begin{figure}[!ht]
\centering
\begin{picture}(40,10)
\put(0,3){\vector(1,0){4}}
\put(8,3){\oval(8,4.7) \put(-2,-.5){Pan/Tilt}}
\put(12,3){\vector(1,0){4}}
\put(20,3){\oval(8,4.7)\put(-2,1){Zoom} \put(-2,-.5){Rotation}\put(-2,-2){Shear} }
\put(24,3){\vector(1,0){4}}
\put(32,3){\oval(8,4.7)\put(-2,-.5){Perspective}}
\put(36,3){\vector(1,0){4}}
\end{picture}
\caption{Global Motion Parameters}\label{fig:globalmotionparameters}
\end{figure}

% Pan/tilt
\setlength{\unitlength}{1em}
\begin{figure}[!ht]
\centering
\begin{picture}(35,12)
\put(0,3){\vector(1,0){3}}
\put(7,3){\oval(8,4.7) \put(-2,1){Non-Zero}\put(-2,-.5){Pan/Tilt} \put(-2,-2){Flag}}
\put(12,3){\line(0,1){5}}
\put(12,8){\vector(1,0){1}}
\put(17,8){\oval(8,4.7)\put(-2,.5){Horizontal} \put(-2,-1){Pan}}
\put(21,8){\vector(1,0){2}}
\put(27,8){\oval(8,4.7)\put(-2,.5){Vertical} \put(-2,-1){Tilt}}
\put(32,3){\line(0,1){5}}
\put(31,8){\vector(1,0){1}}
\put(11,3){\vector(1,0){24}}
\end{picture}
\caption{Pan/tilt}\label{fig:pantilt}
\end{figure}

% Zoom, rotation and shear matrix
\setlength{\unitlength}{1em}
\begin{figure}[!ht]
\centering
\begin{picture}(48,20)
\put(0,3){\vector(1,0){2}}
\put(6.5,3){\oval(9,4.7) \put(-3,1){Non-Trivial}\put(-3,-.5){Zoom, Rotation} \put(-3,-2){and Shear}}
\put(11,3){\line(1,0){1}}
\put(12,3){\vector(1,0){35}}
\put(12,3){\line(0,1){14}}
\put(12,17){\vector(1,0){1}}
\put(17.5,17){\oval(9,4.7)\put(-3,1){Zoom, Rotation}\put(-3,-.5){and Shear} \put(-3,-2){Exponent}}
\put(22,17){\vector(1,0){2}}
\put(27.5,17){\oval(7,4.7)\put(-1,-0.5){$A_{0,0}$}}

\put(31,17){\vector(1,0){2}}
\put(36.5,17){\oval(7,4.7)\put(-1,-0.5){$A_{0,1}$}}
\put(40,17){\line(1,0){2}}
\put(42,17){\vector(0,-1){4.5}}
\put(42,12.5){\vector(-1,0){10}}
\put(32,12.5){\line(-1,0){10}}
\put(22,12.5){\line(0,-1){4.5}}
\put(22,8){\vector(1,0){3}}
\put(28.5,8){\oval(7,4.7)\put(-1,-0.5){$A_{1,0}$}}
\put(32,8){\vector(1,0){2}}
\put(37.5,8){\oval(7,4.7)\put(-1,-0.5){$A_{1,1}$}}
\put(41,8){\vector(1,0){1}}
\put(42,8){\line(0,-1){5}}
\end{picture}
\caption{Zoom, Rotation and Shear}\label{fig:zoomrotationshear}
\end{figure}

% Perspective
\setlength{\unitlength}{1em}
\begin{figure}[!ht]
\centering
\begin{picture}(45,15)
\put(0,3){\vector(1,0){3}}
\put(6.5,3){\oval(7,4.7) \put(-2,1){Non-zero}\put(-2,-.5){Perspective} \put(-2,-2){Flag}}
\put(10,3){\line(1,0){1}}
\put(11,3){\vector(1,0){38}}
\put(11,3){\line(0,1){5}}
\put(11,8){\vector(1,0){1}}
\put(15.5,8){\oval(7,4.7)\put(-2,0.5){Perspective} \put(-2,-1){Exponent}}
\put(19,8){\vector(1,0){2}}
\put(24.5,8){\oval(7,4.7)\put(-2,0.5){Horizontal} \put(-2,-1){Perspective}}
\put(28,8){\vector(1,0){2}}
\put(33.5,8){\oval(7,4.7)\put(-2,0.5){Vertical} \put(-2,-1){Perspective}}
\put(37,8){\vector(1,0){2}}
\put(39,8){\line(0,-1){5}}
\end{picture}
\caption{Perpective}\label{fig:perpective}
\end{figure}

\clearpage

% Picture prediction mode
\setlength{\unitlength}{1em}
\begin{figure}[!ht]
\centering
\begin{picture}(33,12)
\put(0,3){\vector(1,0){3}}
\put(7,3){\oval(8,4.7)\put(-2,1){Picture}\put(-2,-.5){Prediction} \put(-2,-2){Mode Flag}}
\put(11,3){\vector(1,0){18}}

\put(13,3){\line(0,1){5}}
\put(13,8){\vector(1,0){2}}
\put(19,8){\oval(8,4.7)\put(-2.5,1){Picture}\put(-2.5,-.5){Prediction} \put(-2.5,-2){Mode Index}}
\put(23,8){\vector(1,0){2}}
\put(25,8){\line(0,-1){5}}
\end{picture}
\caption{Picture Prediction Mode}\label{fig:picturepredictionmode}
\end{figure}

% Reference picture weights
\setlength{\unitlength}{1em}
\begin{figure}[!ht]
\centering
\begin{picture}(35,12)
\put(0,3){\vector(1,0){3}}
\put(7,3){\oval(8,4.7) \put(-2,1){Non-default}\put(-2,-.5){Weights} \put(-2,-2){Flag}}
\put(12,3){\line(0,1){5}}
\put(12,8){\vector(1,0){1}}
\put(17,8){\oval(8,4.7)\put(-2,1){Reference}\put(-2,-.5){Weights} \put(-2,-2){Precision}}
\put(21,8){\vector(1,0){2}}
\put(22,8){\line(0,1){4}}
\put(22,12){\line(1,0){5}}
\put(32,12){\vector(-1,0){5}}
\put(27,8){\oval(8,4.7)\put(-2,.5){Reference}\put(-2,-1){Weight}}
\put(32,8){\line(0,1){4}}
\put(33,3){\line(0,1){5}}
\put(31,8){\vector(1,0){2}}
\put(11,3){\vector(1,0){24}}
\end{picture}
\caption{Reference Picture Weights}\label{fig:referencepictureweights}
\end{figure}

% Wavelet transform
\setlength{\unitlength}{1em}
\begin{figure}[!ht]
\centering
\begin{picture}(51,8)
\put(0,3){\vector(1,0){12}}

\put(1,3){\line(0,1){4}}
\put(1,7){\vector(1,0){1}}
\put(5.5,7){\oval(7,4.7) \put(-2,.5){Zero}\put(-2,-1){Residual}}
\put(9,7){\vector(1,0){1}}
\put(10,3){\line(0,1){4}}

\put(15.5,3){\oval(7,4.7)\put(-2,.5){Transform} \put(-2,-1){Parameters}}
\put(19,3){\vector(1,0){2}}
\put(24.5,3){\oval(7,4.7)\put(-2,1){Y} \put(-2,-.5){Transform}\put(-2,-2){Data}}
\put(28,3){\vector(1,0){2}}
\put(33.5,3){\oval(7,4.7)\put(-2,1){U} \put(-2,-.5){Transform}\put(-2,-2){Data}}
\put(37,3){\vector(1,0){2}}
\put(42.5,3){\oval(7,4.7)\put(-2,1){V} \put(-2,-.5){Transform}\put(-2,-2){Data}}
\put(46,3){\vector(1,0){2}}
\put(11,3){\line(0,1){4}}
\put(11,7){\vector(1,0){18.5}}
\put(29.5,7){\line(1,0){17}}
\put(46.5,7){\line(0,-1){4}}
\end{picture}
\caption{Wavelet Transform}\label{fig:wavelettransform}
\end{figure}

% Transform parameters
\setlength{\unitlength}{1em}
\begin{figure}[!ht]
\centering
\begin{picture}(40,10)
\put(0,3){\vector(1,0){4}}
\put(8,3){\oval(8,4.7) \put(-2,.5){Wavelet}\put(-2,-1){Filter}}
\put(12,3){\vector(1,0){4}}
\put(20,3){\oval(8,4.7)\put(-2,.5){Wavelet}\put(-2,-1){Depth}}
\put(24,3){\vector(1,0){4}}
\put(32,3){\oval(8,4.7)\put(-2,.5){Spatial}\put(-2,-1){Partition}}
\put(36,3){\vector(1,0){4}}
\end{picture}
\caption{Transform Parameters}\label{fig:transformparameters}
\end{figure}

% Wavelet filter
\setlength{\unitlength}{1em}
\begin{figure}[!ht]
\centering
\begin{picture}(30,12)
\put(0,3){\vector(1,0){3}}
\put(7,3){\oval(8,4.7)\put(-2,1){Non-default}\put(-2,-.5){Wavelet} \put(-2,-2){Flag}}
\put(11,3){\vector(1,0){18}}
\put(13,3){\line(0,1){5}}
\put(13,8){\vector(1,0){2}}
\put(19,8){\oval(8,4.7)\put(-2,.5){Wavelet}\put(-2,-1){Index}}
\put(23,8){\vector(1,0){2}}
\put(25,8){\line(0,-1){5}}
\end{picture}
\caption{Wavelet Filter}\label{fig:waveletfilter}
\end{figure}

% Wavelet depth
\setlength{\unitlength}{1em}
\begin{figure}[!ht]
\centering
\begin{picture}(30,12)
\put(0,3){\vector(1,0){3}}
\put(7,3){\oval(8,4.7)\put(-2.5,1){Non-default}\put(-2.5,-.5){Wavelet} \put(-2.5,-2){Depth Flag}}
\put(11,3){\vector(1,0){18}}
\put(13,3){\line(0,1){5}}
\put(13,8){\vector(1,0){2}}
\put(19,8){\oval(8,4.7)\put(-2,.5){Transform}\put(-2,-1){Depth}}
\put(23,8){\vector(1,0){2}}
\put(25,8){\line(0,-1){5}}
\end{picture}
\caption{Wavelet Depth}\label{waveletdepth}
\end{figure}

% Spatial partition
\setlength{\unitlength}{1em}
\begin{figure}[!ht]
\centering
\begin{picture}(43,18)
\put(0,3){\vector(1,0){3}}
\put(6.5,3){\oval(7,4.7) \put(-2,1){Spatial}\put(-2,-.5){Partition} \put(-2,-2){Flag}}
\put(10,3){\line(1,0){1}}
\put(11,3){\vector(1,0){32}}
\put(11,3){\line(0,1){5}}
\put(11,8){\vector(1,0){1}}
\put(15.5,8){\oval(7,4.7)\put(-2,1){Non-default}\put(-2,-.5){Partition} \put(-2,-2){Flag}}
\put(19,8){\line(1,0){1}}
\put(20,8){\vector(1,0){12}}
\put(20,8){\line(0,1){5}}
\put(20,13){\vector(1,0){2}}
\put(25.5,13){\oval(7,4.7)\put(-2.5,1){Number}\put(-2.5,-.5){of} \put(-2.5,-2){Codeblocks}}
\put(29,13){\vector(1,0){2}}
\put(21,13){\line(0,1){4}}
\put(21,17){\line(1,0){4}}
\put(30,17){\vector(-1,0){5}}
\put(30,17){\line(0,-1){4}}
\put(31,13){\line(0,-1){5}}
\put(35.5,8){\oval(7,4.7)\put(-2,.5){Codeblock}\put(-2,-1){Mode}}
\put(39,8){\vector(1,0){2}}
\put(41,8){\line(0,-1){5}}
\end{picture}
\caption{Spatial Partition}\label{fig:spatialpartition}
\end{figure}

% Block motion data stuff
%%%%%%%%%%%%%

% Block motion data

\setlength{\unitlength}{1em}
\begin{figure}[!ht]
\centering
\begin{picture}(50,18)
\put(0,3){\vector(1,0){2}}
\put(5.5,3){\oval(7,4.7)\put(-2,.5){Superblock}\put(-2,-1.5){Split Mode}}
\put(9,3){\vector(1,0){2}}
\put(15.5,3){\oval(9,4.7)\put(-3.25,.5){Block}\put(-3.25,-1.5){Prediction Mode}}
\put(20,3){\vector(1,0){2}}
\put(25.5,3){\oval(7,4.7)\put(-2,.5){Motion}\put(-2,-1.5){Vector}}
\put(29,3){\vector(1,0){2}}
\put(34.5,3){\oval(7,4.7)\put(-2,.5){Motion}\put(-2,-1.5){Vector}}
\put(38,3){\vector(1,0){2}}
\put(43.5,3){\oval(7,4.7)\put(-1,-0.5){DC}}
\put(47,3){\vector(1,0){2}}

\put(30,3){\line(0,1){5}}
\put(30,8){\vector(1,0){5}}
\put(35,8){\line(1,0){4}}
\put(39,8){\line(0,-1){5}}

\end{picture}
\caption{Block Motion Data}\label{fig:blockmotiondata}
\end{figure}


% Block motion data

\setlength{\unitlength}{1em}
\begin{figure}[!ht]
\centering
\begin{picture}(40,18)
\put(0,3){\vector(1,0){2}}
\put(5.5,3){\oval(7,4.7)\put(-2,-0.5){Length}}
\put(9,3){\vector(1,0){2}}
\put(14.5,3){\oval(7,4.7)\put(-2,.5){Horizontal}\put(-2,-1.5){Element}}
\put(18,3){\vector(1,0){2}}
\put(23.5,3){\oval(7,4.7)\put(-2,-0.5){Length}}
\put(27,3){\vector(1,0){2}}
\put(32.5,3){\oval(7,4.7)\put(-2,.5){Vertical}\put(-2,-1.5){Element}}
\put(36,3){\vector(1,0){2}}

\put(10,8){\line(0,-1){5}}
\put(14,8){\line(-1,0){4}}
\put(19,8){\vector(-1,0){5}}
\put(19,3){\line(0,1){5}}

\put(28,8){\line(0,-1){5}}
\put(32,8){\line(-1,0){4}}
\put(37,8){\vector(-1,0){5}}
\put(37,3){\line(0,1){5}}

\end{picture}
\caption{Motion Vector}\label{fig:motionvector}
\end{figure}

\setlength{\unitlength}{1em}
\begin{figure}[!ht]
\centering
\begin{picture}(52,18)
\put(0,3){\vector(1,0){2}}
\put(4.5,3){\oval(5,4.7)\put(-1.5,-0.5){Length}}
\put(7,3){\vector(1,0){2}}
\put(12.5,3){\oval(7,4.7)\put(-2.5,.5){Luma DC}\put(-3,-1.5){Residual}}
\put(16,3){\vector(1,0){2}}
\put(20.5,3){\oval(5,4.7)\put(-1.5,-0.5){Length}}
\put(23,3){\vector(1,0){2}}
\put(28.5,3){\oval(7,4.7)\put(-2.5,.5){Chroma1 DC}\put(-2,-1.5){Residual}}
\put(32,3){\vector(1,0){2}}
\put(36.5,3){\oval(5,4.7)\put(-1.5,-0.5){Length}}
\put(39,3){\vector(1,0){2}}
\put(44.5,3){\oval(7,4.7)\put(-2.5,.5){Chroma2 DC}\put(-2,-1.5){Residual}}
\put(48,3){\vector(1,0){2}}

\put(8,8){\line(0,-1){5}}
\put(12,8){\line(-1,0){4}}
\put(17,8){\vector(-1,0){5}}
\put(17,3){\line(0,1){5}}

\put(24,8){\line(0,-1){5}}
\put(28,8){\line(-1,0){4}}
\put(33,8){\vector(-1,0){5}}
\put(33,3){\line(0,1){5}}

\put(40,8){\line(0,-1){5}}
\put(44,8){\line(-1,0){4}}
\put(49,8){\vector(-1,0){5}}
\put(49,3){\line(0,1){5}}

\end{picture}
\caption{DC}\label{fig:dc}
\end{figure}



% Superblock

\setlength{\unitlength}{1em}
\begin{figure}[!ht]
\centering
\begin{picture}(15,10)
\put(0,3){\vector(1,0){4}}
\put(8,3){\oval(8,4.7)\put(-2,.5){SB Split}\put(-2,-1.5){Residual} }
\put(12,3){\vector(1,0){4}}
\put(14,3){\line(0,1){5}}
\put(14,8){\vector(-1,0){6}}
\put(8,8){\line(-1,0){6}}
\put(2,8){\line(0,-1){5}}
\end{picture}
\caption{Superblock Split Mode}\label{fig:superblocksplit}
\end{figure}

% Block

\setlength{\unitlength}{1em}
\begin{figure}[!ht]
\centering
\begin{picture}(30,15)
\put(0,3){\vector(1,0){2}}

\put(6,3){\oval(8,4.7)\put(-3,.5){Prediction}\put(-3,-1.5){Mode Residual}}
\put(10,3){\vector(1,0){3}}


\put(17,3){\oval(8,4.7)\put(-2,.5){Block Global}\put(-2,-1.5){Residual}}
\put(21,3){\vector(1,0){3}}
\put(22,3){\line(0,1){5}}
\put(22,8){\line(-1,0){6}}
\put(12,8){\vector(1,0){5}}
\put(12,8){\line(0,-1){5}}


\end{picture}
\caption{Block Prediction Mode}\label{fig:blockpredmode}
\end{figure}

\setlength{\unitlength}{1em}
\begin{figure}[!ht]
\centering
\begin{picture}(30,12)
\put(0,3){\vector(1,0){3}}
\put(7,3){\oval(8,4.7)\put(-2,1){Prediction}\put(-2,-.5){Mode} \put(-2,-2){Residual 1}}
\put(11,3){\vector(1,0){18}}
\put(13,3){\line(0,1){5}}
\put(13,8){\vector(1,0){2}}
\put(19,8){\oval(8,4.7)\put(-2,1){Prediction}\put(-2,-.5){Mode} \put(-2,-2){Residual 2}}
\put(23,8){\vector(1,0){2}}
\put(25,8){\line(0,-1){5}}
\end{picture}

\caption{Prediction Mode Residual}\label{fig:predmoderesidual}

\end{figure}




% Transform data stuff 
%%%%%%%%%%%

% Transform data

\setlength{\unitlength}{1em}
\begin{figure}[!ht]
\centering
\begin{picture}(20,12)
\put(0,5){\vector(1,0){5}}
\put(10,5){\oval(10,4.7)\put(-2,-0.5){Subband}}
\put(15,5){\vector(1,0){5}}
\put(2.5,5){\line(0,1){5}}
\put(17.5,5){\line(0,1){5}}
\put(17.5,10){\vector(-1,0){10}}
\put(2.5,10){\line(1,0){10}}
\end{picture}
\caption{Transform Data}\label{fig:transformdata}
\end{figure}

% Subband

\setlength{\unitlength}{1em}
\begin{figure}[!ht]
\centering
\begin{picture}(40,10)
\put(0,3){\vector(1,0){4}}
\put(8,3){\oval(8,4.7) \put(-2,1){Subband}\put(-2,-.5){Data} \put(-2,-2){Length}}
\put(12,3){\vector(1,0){4}}
\put(14,3){\line(0,1){6}}
\put(14,9){\vector(1,0){8}}
\put(22,9){\line(1,0){14}}
\put(36,9){\line(0,-1){6}}
\put(20,3){\oval(8,4.7)\put(-2,.5){Quantiser}\put(-2,-1){Index}}
\put(24,3){\vector(1,0){2}}

\put(25,3){\line(0,1){4}}
\put(30,3){\oval(8,4.7)\put(-2,-0.25){Codeblock}}
\put(35,3){\line(0,1){4}}
\put(35,7){\vector(-1,0){5}}
\put(30,7){\line(-1,0){5}}

\put(34,3){\vector(1,0){5}}

\end{picture}
\caption{Subband}\label{fig:subband}
\end{figure}

% Codeblock

\setlength{\unitlength}{1em}
\begin{figure}[!ht]
\centering
\begin{picture}(40,12)
\put(0,3){\vector(1,0){3}}
\put(2,3){\line(0,1){5}}
\put(2,8){\vector(1,0){18}}
\put(20,8){\line(1,0){5}}
\put(25,8){\line(0,-1){5}}
\put(7,3){\oval(8,4.7)\put(-2,.5){Zero Block}\put(-2,-1.5){Flag}}
\put(11,3){\vector(1,0){4}}

\put(19,3){\oval(8,4.7)\put(-2,1){Differential}\put(-2,-.5){Quantiser} \put(-2,-2){Index}}
\put(23,3){\vector(1,0){5}}
\put(13,3){\line(0,1){4}}
\put(13,7){\vector(1,0){6}}
\put(19,7){\line(1,0){5}}
\put(24,7){\line(0,-1){4}}

\put(32,3){\oval(8,4.7)\put(-2,.5){Wavelet}\put(-2,-1.5){Coefficient}}
\put(36,3){\vector(1,0){4}}

\put(37,3){\line(0,1){4}}
\put(37,7){\vector(-1,0){5}}
\put(27,7){\line(1,0){5}}
\put(27,7){\line(0,-1){4}}

\put(26,3){\line(0,1){5}}
\put(26,8){\vector(1,0){6}}
\put(32,8){\line(1,0){6}}
\put(38,8){\line(0,-1){5}}
\end{picture}

\caption{Codeblock}\label{fig:codeblock}
\end{figure}

\clearpage

\subsection{Low delay syntax}

\setlength{\unitlength}{1em}
\begin{figure}[!ht]
\centering
\begin{picture}(45,8)
\put(0,3){\vector(1,0){3}}
\put(7,3){\oval(8,4.7) \put(-2,.5){Transform}\put(-2,-1){Parameters}}
\put(11,3){\vector(1,0){2}}
\put(17.5,3){\oval(9,4.7)\put(-3.5,.5){Picture Prediction} \put(-3.5,-1){Parameters}}
\put(22,3){\vector(1,0){2}}
\put(28,3){\oval(8,4.7)\put(-2,.5){Slice} \put(-2,-1){Parameters}}
\put(32,3){\vector(1,0){2}}
\put(38,3){\oval(8,4.7)\put(-2,.5){Quant} \put(-2,-1){Matrix}}
\put(42,3){\vector(1,0){3}}
\end{picture}
\caption{Low Delay Header}\label{fig:lowdelayheader}
\end{figure}

\setlength{\unitlength}{1em}
\begin{figure}[!ht]
\centering
\begin{picture}(30,8)
\put(0,3){\vector(1,0){2}}
\put(5.5,3){\oval(7,4.7) \put(-1.5,.5){Slice}\put(-1.5,-1){Width}}
\put(9,3){\vector(1,0){2}}
\put(14.5,3){\oval(7,4.7) \put(-1.5,.5){Slice}\put(-1.5,-1){Height}}
\put(18,3){\vector(1,0){3}}
\put(24.5,3){\oval(7,4.7) \put(-1.5,.5){Slice}\put(-1.5,-1){Bits}}
\put(28,3){\vector(1,0){2}}

\put(1,3){\line(0,1){4}}
\put(1,7){\vector(1,0){9}}
\put(10,7){\line(1,0){9}}
\put(19,7){\line(0,-1){4}}

\put(20,3){\line(0,1){4}}
\put(20,7){\vector(1,0){5}}
\put(25,7){\line(1,0){4}}
\put(29,7){\line(0,-1){4}}

\end{picture}
\caption{Slice Parameters}\label{fig:sliceparameters}
\end{figure}

\setlength{\unitlength}{1em}
\begin{figure}[!ht]
\centering
\begin{picture}(22,8)
\put(0,3){\vector(1,0){2}}
\put(6,3){\oval(8,4.7) \put(-2.5,.5){Quant Matrix}\put(-2.5,-1){Index}}
\put(10,3){\vector(1,0){2}}
\put(16,3){\oval(8,4.7) \put(-2.5,.5){Subband}\put(-2.5,-1){Quant Offset}}
\put(20,3){\vector(1,0){3}}

\put(1,3){\line(0,1){4}}
\put(1,7){\vector(1,0){5.5}}
\put(6.5,7){\line(1,0){4.5}}
\put(11,7){\line(0,-1){4}}

\end{picture}	
	\caption{Quant Matrix}
	\label{fig:quantmatrix}
\end{figure}

\begin{figure}[!ht]
\centering
\begin{picture}(13,8)
\put(0,3){\vector(1,0){3}}
\put(6.5,3){\oval(7,4.7)\put(-1.5,-0.5){Slice}}
\put(10,3){\vector(1,0){3}}

\put(2,7){\line(0,-1){4}}
\put(11,3){\line(0,1){4}}
\put(11,7){\vector(-1,0){5}}
\put(6,7){\line(-1,0){4}}
\end{picture}
	
	\caption{Low Delay Picture}
	\label{fig:lowdelaypicture}
\end{figure}

\begin{figure}[!ht]
\centering
\begin{picture}(45,8)
\put(0,3){\vector(1,0){2}}
\put(5.5,3){\oval(7,4.7) \put(-2,.5){Slice Quant}\put(-2,-1){Index}}
\put(9,3){\vector(1,0){2}}
\put(14.5,3){\oval(7,4.7)\put(-2.5,.5){Luma Slice} \put(-2.5,-1){Bits}}
\put(18,3){\vector(1,0){3}}

\put(24.5,3){\oval(7,4.7)\put(-2.5,.5){Luma Slice} \put(-2.5,-1){Subband}}
\put(28,3){\vector(1,0){3}}
\put(34.5,3){\oval(7,4.7)\put(-2.5,.5){Chroma Slice} \put(-2.5,-1){Subband}}
\put(38,3){\vector(1,0){2}}

\put(20,7){\line(0,-1){4}}
\put(29,3){\line(0,1){4}}
\put(29,7){\vector(-1,0){5}}
\put(24,7){\line(-1,0){4}}

\put(30,7){\line(0,-1){4}}
\put(39,3){\line(0,1){4}}
\put(39,7){\vector(-1,0){5}}
\put(34,7){\line(-1,0){4}}

\end{picture}
	
	\caption{Slice}
	\label{fig:slice}
\end{figure}

\begin{figure}[!ht]
\centering
\begin{picture}(13,8)
\put(0,3){\vector(1,0){3}}
\put(6.5,3){\oval(7,4.7)\put(-2.5,.5){Luma} \put(-2.5,-1){Coefficient}}
\put(10,3){\vector(1,0){3}}

\put(2,7){\line(0,-1){4}}
\put(11,3){\line(0,1){4}}
\put(11,7){\vector(-1,0){5}}
\put(6,7){\line(-1,0){4}}

\end{picture}
	
	\caption{Luma Slice Subband}
	\label{fig:lumaslicesubband}
\end{figure}

\begin{figure}[!ht]
\centering
\begin{picture}(13,8)
\put(0,3){\vector(1,0){3}}
\put(6.5,3){\oval(7,4.7)\put(-2.5,.5){Chroma1} \put(-2.5,-1){Coefficient}}
\put(10,3){\vector(1,0){2}}
\put(15.5,3){\oval(7,4.7)\put(-2.5,.5){Chroma2} \put(-2.5,-1){Coefficient}}
\put(19,3){\vector(1,0){3}}

\put(2,7){\line(0,-1){4}}
\put(20,3){\line(0,1){4}}
\put(20,7){\vector(-1,0){9}}
\put(11,7){\line(-1,0){9}}

\end{picture}
	
	\caption{Chroma Slice Subband}
	\label{fig:chromaslicesubband}
\end{figure}

%\clearpage
%\section{References}
%\annotate{shas}{I hope we do not have any references - it would be nice
%to be freestanding.}


\end{document}
