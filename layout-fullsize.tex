\documentclass[a4paper,9pt]{extarticle}
\usepackage{array}
\usepackage{amsmath}
\usepackage{amssymb}
\usepackage{graphicx}
\usepackage{subfigure}
\usepackage{verbatim}
\usepackage{framed}
\usepackage{color}
\usepackage{fancyhdr}
\usepackage{layout}
\usepackage{makeidx}
\DeclareFontShape{OT1}{cmtt}{bx}{n}{<->cmbtt10}{}

\setlength\voffset{-0.5in}
\addtolength\textheight{1.5in}
\setlength\oddsidemargin{0in}
\setlength\marginparwidth{1.25in}
\setlength\textwidth{\paperwidth}
\addtolength\textwidth{-1.25in}
\addtolength\textwidth{-\oddsidemargin}
\addtolength\textwidth{-\marginparsep}
\addtolength\textwidth{-\marginparwidth}

\linespread{1.2}
\setlength\parskip{6pt}
\setlength\parindent{0pt}

\pagestyle{fancy}
\fancyhf{}
\fancyhead[LE,OR]{\bfseries\thepage}
\fancyhead[RE]{\bfseries\leftmark}
\fancyhead[OL]{\bfseries\rightmark}
%\fancyfoot[EC,OC]{\input{.version-date}}
\renewcommand{\headrulewidth}{0.5pt}
\renewcommand{\footrulewidth}{0.5pt}
\addtolength{\headheight}{0.5pt}
\addtolength{\textheight}{-0.5pt}
\fancypagestyle{plain}{% no headers on plain pages
    \fancyhead{}%
    \renewcommand{\headrulewidth}{0pt}%
}

%\bibliographystyle{IEEEtran}
%\renewcommand\bibname{References}

\DeclareMathOperator{\clip}{clip}
\DeclareMathOperator{\sign}{sign}
 \DeclareMathOperator{\row}{row}
\DeclareMathOperator{\rounddivide}{round\_divide}
\DeclareMathOperator{\column}{column}
\DeclareMathOperator{\width}{width}
\DeclareMathOperator{\height}{height}
\DeclareMathOperator{\length}{length}
\DeclareMathOperator{\median}{median}
\DeclareMathOperator{\mean}{mean}
\DeclareMathOperator{\majority}{majority}
\DeclareMathOperator{\args}{args}

\definecolor{shadecolor}{gray}{0.90}
\newenvironment{informative}[0]%
{
\begin{shaded}
\FrameCommand{{\bf Informative:}}
}
{\end{shaded}}

\newenvironment{informative*}[0]%
{
\begin{shaded}
}
{\end{shaded}}

\setcounter{secnumdepth}{4}
\setcounter{tocdepth}{4}
\definecolor{commentcolour}{rgb}{1,0,0}

\makeindex

\begin{document}
\newcommand{\SpecVersion}{\text{1.0.0\_pre7}}

\begin{titlepage}
\vspace*{\stretch{1}} \hspace*{\stretch{1}}
 {\Huge Dirac Specification}
\hspace*{\stretch{1}}\\

\hspace*{\stretch{1}}
 {\huge Version \SpecVersion}
\hspace*{\stretch{1}}\\
\vspace*{\stretch{3}}

 \clearpage { \thispagestyle{empty}
\begin{abstract}

This document is the specification of the Dirac video decoder and stream syntax.

Dirac is a video compression system utilising wavelet transforms and motion
compensation. It is designed to be simple, flexible, yet highly effective. 
It can operate across a wide range of resolutions and
application domains, including: internet and mobile streaming, delivery of 
standard-definition and high-definition
television, digital television and cinema production and distribution,
and low-power devices and embedded applications.

The system offers several key features:
\begin{itemize}
\item lossy and lossless coding using a common tool set
\item intra-coded modes for professional production applications
\item a special low delay mode for link adaption applications, such as the carriage of HDTV
over SDTV infrastructure
 \item motion-compensated (`long-GOP') modes for distribution applications
\item gradual quality reduction with increasing compression 
\end{itemize}



\end{abstract}
}
\end{titlepage}


% This file contains commands for all the state
% variables in the spec. These commands should be
% used for each occurrence of the state variables
% so that they have a consistent style, which can
% be varied at will.

% Arithmetic decoding engine state variables %

\newcommand{\true}{\text{\bf{True}}}
\newcommand{\false}{\text{\bf{False}}}

\newcommand\StateName{\textbf{state}}
\newcommand\SeqStateName{\textbf{default\_state}}

% How to use state macros:
% To insert the raw variable index, type \VNickName
% To insert the default/sequence state variable, type \SNickName
% To inset the state variable, type \NickName

\newcommand{\pdefine}[2]{%
    \label{#1}%
    \expandafter\def\csname #1\endcsname{\text{\StateName[#2]}}}

\newcommand{\sdefine}[2]{%
    \label{#1}%
    \expandafter\def\csname S#1\endcsname{\text{\SeqStateName[#2]}}}

\newcommand{\kdefine}[2]{%
    \expandafter\def\csname S#1\endcsname{\text{\SeqStateName[#2]}\index{#2}}
    \expandafter\def\csname #1\endcsname{\text{\StateName[#2]}\index{#2}}
    \expandafter\def\csname V#1\endcsname{\text{#2}\index{#2}}
}

\kdefine{AHigh}{high}
\kdefine{ARange}{range}
\kdefine{ALow}{low}
\kdefine{ACode}{code}
\kdefine{ABitsLeft}{bits\_left}
\kdefine{AContexts}{contexts}
\kdefine{ALUT}{lut}

\kdefine{RefBuffer}{ref\_pictures}
\kdefine{DecodedBuffer}{decoded\_pictures}
\kdefine{RefBufferSize}{rb\_size}
\kdefine{DPBSize}{dpb\_size}
\kdefine{StreamBuffer}{stream\_buffer}
\kdefine{StreamBufferSize}{stream\_buffer\_size}
\kdefine{PictureInterval}{picture\_interval}
\kdefine{ParseInfoPrefix}{parse\_info\_prefix}
\kdefine{ParseCode}{parse\_code}
\kdefine{NextParseOffset}{next\_parse\_offset}
\kdefine{PrevParseOffset}{previous\_parse\_offset}

\kdefine{AUPictureNumber}{au\_picture\_number}
\kdefine{VersionMajor}{version\_major}
\kdefine{VersionMinor}{version\_minor}
\kdefine{Profile}{profile}
\kdefine{Level}{level}

\kdefine{VideoFormat}{video\_format}
\kdefine{CustomImageSize}{custom\_image\_size}
\kdefine{LumaWidth}{luma\_width}
\kdefine{LumaHeight}{luma\_height}
\kdefine{ChromaWidth}{chroma\_width}
\kdefine{ChromaHeight}{chroma\_height}
\kdefine{ComponentWidth}{component\_width}
\kdefine{LumaXBlen}{luma\_xblen}
\kdefine{LumaYBlen}{luma\_yblen}
\kdefine{LumaXBsep}{luma\_xbsep}
\kdefine{LumaYBsep}{luma\_ybsep}
\kdefine{ChromaXBlen}{chroma\_xblen}
\kdefine{ChromaYBlen}{chroma\_yblen}
\kdefine{ChromaXBsep}{chroma\_xbsep}
\kdefine{ChromaYBsep}{chroma\_ybsep}
\kdefine{SuperblocksX}{superblocks\_x}
\kdefine{SuperblocksY}{superblocks\_y}
\kdefine{BlocksX}{blocks\_x}
\kdefine{BlocksY}{blocks\_y}

\kdefine{ComponentHeight}{component\_height}
\kdefine{ChromaFormatIndex}{chroma\_format\_index}
\kdefine{ChromaFormat}{chroma\_format}
\kdefine{CustomVideoDepth}{custom\_video\_depth}
\kdefine{VideoDepth}{video\_depth}


\kdefine{Interlaced}{interlaced}
\kdefine{TopFieldFirst}{top\_field\_first}
\kdefine{SequentialFields}{sequential\_fields}
\kdefine{CustomFrameRate}{custom\_frame\_rate}
\kdefine{FrameRateIndex}{frame\_rate\_index}
\kdefine{FrameRateNumerator}{frame\_rate\_numer}
\kdefine{FrameRateDenominator}{frame\_rate\_denom}
\kdefine{CustomAspectRatio}{custom\_aspect\_ratio}
\kdefine{AspectRatioIndex}{aspect\_ratio\_index}
\kdefine{AspectRatioNumerator}{aspect\_ratio\_numer}
\kdefine{AspectRatioDenominator}{aspect\_ratio\_denom}

\kdefine{CustomCleanArea}{custom\_clean\_area}
\kdefine{CleanWidth}{clean\_width}
\kdefine{CleanHeight}{clean\_height}
\kdefine{LeftOffset}{left\_offset}
\kdefine{TopOffset}{top\_offset}

\kdefine{CustomSignalRange}{custom\_signal\_range}
\kdefine{SignalRangeIndex}{signal\_range\_index}
\kdefine{LumaOffset}{luma\_offset}
\kdefine{LumaExcursion}{luma\_excursion}
\kdefine{ChromaOffset}{chroma\_offset}
\kdefine{ChromaExcursion}{chroma\_excursion}

\kdefine{ColourSpec}{colour\_spec}
\kdefine{ColourSpecIndex}{colour\_spec\_index}
\kdefine{ColourPrimaries}{colour\_primaries}
\kdefine{ColourPrimariesIndex}{colour\_primaries\_index}
\kdefine{ColourMatrix}{colour\_matrix}
\kdefine{ColourMatrixIndex}{colour\_matrix\_index}
\kdefine{TransferFunction}{transfer\_fn}
\kdefine{TransferFunctionIndex}{transfer\_fn\_index}

\kdefine{PictureNumber}{picture\_number}
\kdefine{RefOneNum}{ref1\_picture\_number}
\kdefine{RefTwoNum}{ref2\_picture\_number}
\kdefine{NumRetiredPictures}{num\_retired\_pictures}
\kdefine{RetiredPictureList}{retired\_picture\_list}

\kdefine{BlockData}{block\_data}
\kdefine{BlockDataLength}{block\_data\_length}
\kdefine{CompressedBlockData}{compressed\_block\_data}
\kdefine{CustomBlockParameters}{custom\_block\_parameters}
\kdefine{BlockParametersIndex}{block\_parameters\_index}
\kdefine{LumaXBLen}{luma\_xblen}
\kdefine{LumaYBLen}{luma\_yblen}
\kdefine{LumaXBSep}{luma\_xbsep}
\kdefine{LumaYBSep}{luma\_ybsep}
\kdefine{MotionVectorPrecision}{mv\_precision}
\kdefine{RefsWeightPrecision}{refs\_weight\_precision}
\kdefine{RefOneWeight}{ref1\_weight}
\kdefine{RefTwoWeight}{ref2\_weight}
\kdefine{PictureUsingGlobal}{using\_global}
\kdefine{GlobalParams}{global\_params}
\kdefine{NumRefs}{num\_refs}

\kdefine{CustomPicturePredictionMode}{custom\_picture\_prediction\_mode}
\kdefine{PicturePredictionModeIndex}{picture\_prediction\_mode}

\kdefine{CustomReferenceWeights}{custom\_reference\_weights}
\kdefine{PictureWeightBits}{picture\_weight\_bits}
\kdefine{PictureWeightRefOne}{picture\_weight\_ref1}
\kdefine{PictureWeightRefTwo}{picture\_weight\_ref2}
\kdefine{NonZeroPanTiltFlag}{nonzero\_pan\_tilt\_flag}
\kdefine{ZRSexponent}{ZRS\_exponent}
\kdefine{NonZeroPerspectiveFlag}{nonzero\_perspective\_flag}
\kdefine{GMperspectiveExponent}{perspective\_exponent}
\kdefine{GMperspectiveX}{perspective\_x}
\kdefine{GMperspectiveY}{perspective\_y}

\kdefine{ZeroResidual}{zero\_residual}
\kdefine{WaveletIndex}{wavelet\_index}
\kdefine{WaveletFilter}{wavelet\_filter}
\kdefine{WaveletDepth}{wavelet\_depth}
\kdefine{TransformDepth}{transform\_depth}
\kdefine{YTransform}{y\_transform}
\kdefine{COneTransform}{c1\_transform}
\kdefine{CTwoTransform}{c2\_transform}
\kdefine{SpatialPartitionFlag}{spatial\_partition\_flag}
\kdefine{Codeblocks}{codeblocks}
\kdefine{CodeblockMode}{codeblock\_mode}
\kdefine{SubbandDataLength}{subband\_data\_length}% [level][band]
\kdefine{QuantIndex}{quantiser\_index}
\kdefine{CompressedSubbandData}{compressed\_subband\_data}%needed?
\kdefine{Contexts}{contexts}
\kdefine{CoefficientCount}{coefficient\_count}
\kdefine{CoefficientReset}{coefficient\_reset}


\kdefine{SBSplitResidue}{sb\_split\_residual}
\kdefine{SBSplit}{sb\_split}
\kdefine{SBCommonResidue}{sb\_common\_residual}
\kdefine{SBCommon}{sb\_common}
\kdefine{PURefinuseAResidual}{ref1\_mode\_residual}
\kdefine{PURefinuseA}{ref1\_mode}
%\kdefine{PURefinuseB}{ref2\_mode\_residual}
\kdefine{PURefinuseB}{ref2\_mode}
\kdefine{PULumaDCResidual}{luma\_dc\_residual}
\kdefine{PULumaDC}{luma\_dc}
\kdefine{PUChromaADCResidual}{chroma1\_dc\_residual}
\kdefine{PUChromaADC}{chroma1\_dc}
\kdefine{PUChromaBDCResidual}{chroma2\_dc\_residual}
\kdefine{PUChromaBDC}{chroma2\_dc}
\kdefine{PUUsingGlobalResidue}{pu\_using\_global\_residual}
\kdefine{PUvectorA}{ref1\_vector}
\kdefine{PUvectorB}{ref2\_vector}
\kdefine{MVhorzontalResidue}{horizontal\_residual}
\kdefine{MVverticalResidue}{vertical\_residual}

\kdefine{CurrentByte}{current\_byte}
\kdefine{NextBit}{next\_bit}
\kdefine{Contexts}{contexts}
\kdefine{Codeblocks}{codeblocks}
\kdefine{CodeblockMode}{codeblock\_mode}

\kdefine{CurrentPicture}{current\_picture}

\kdefine{SliceWidth}{slice\_width}
\kdefine{SliceHeight}{slice\_height}
\kdefine{SliceBits}{slice\_bits}
\kdefine{SliceYLength}{slice\_y\_length}
\kdefine{QuantMatrix}{quant\_matrix}
\kdefine{SlicesX}{slices\_x}
\kdefine{SlicesY}{slices\_y}
\kdefine{UQuantOffset}{U\_quant\_offset}
\kdefine{VQuantOffset}{V\_quant\_offset}

\newcommand{\LL}{\textit{LL}}
\newcommand{\LH}{\textit{LH}}
\newcommand{\HL}{\textit{HL}}
\newcommand{\HH}{\textit{HH}}

\newcommand{\SingleQuantiser}{\text{SINGLE\_QUANT}}
\newcommand{\MultipleQuantiser}{\text{MULTI\_QUANT}}
\newcommand{\Intra}{\text{INTRA}}
\newcommand{\Inter}{\text{INTER}}
\newcommand{\RefOneAndTwo}{\text{REF1AND2}}
\newcommand{\RefOneOnly}{\text{REF1ONLY}}
\newcommand{\RefTwoOnly}{\text{REF2ONLY}}

% Contexts:

\newcommand{\SignZero}{\text{SIGN\_ZERO}}
\newcommand{\SignPos}{\text{SIGN\_POS}}
\newcommand{\SignNeg}{\text{SIGN\_NEG}}

\newcommand{\ZPZNFollowOne}{\text{ZPZN\_F1}}
\newcommand{\ZPNNFollowOne}{\text{ZPNN\_F1}}
\newcommand{\ZPFollowTwo}{\text{ZP\_F2}}
\newcommand{\ZPFollowThree}{\text{ZP\_F3}}
\newcommand{\ZPFollowFour}{\text{ZP\_F4}}
\newcommand{\ZPFollowFive}{\text{ZP\_F5}}
\newcommand{\ZPFollowSixPlus}{\text{ZP\_F6+}}

\newcommand{\NPZNFollowOne}{\text{NPZN\_F1}}
\newcommand{\NPNNFollowOne}{\text{NPNN\_F1}}
\newcommand{\NPFollowTwo}{\text{NP\_F2}}
\newcommand{\NPFollowThree}{\text{NP\_F3}}
\newcommand{\NPFollowFour}{\text{NP\_F4}}
\newcommand{\NPFollowFive}{\text{NP\_F5}}
\newcommand{\NPFollowSixPlus}{\text{NP\_F6+}}

\newcommand{\CoeffData}{\text{COEFF\_DATA}}

\newcommand{\ZeroCodeblock}{\text{ZERO\_BLOCK}}
\newcommand{\QOffsetFollow}{\text{Q\_OFFSET\_FOLLOW}}
\newcommand{\QOffsetData}{\text{Q\_OFFSET\_DATA}}
\newcommand{\QOffsetSign}{\text{Q\_OFFSET\_SIGN}}
\newcommand{\TotalCoeffCtxs}{\text{TOTAL\_COEFF\_CTXs}}

\newcommand{\SBSplitFollowOne}{\text{SB\_F1}}
\newcommand{\SBSplitFollowTwo}{\text{SB\_F2}}
\newcommand{\SBSplitData}{\text{SB\_DATA}}

\newcommand{\PredModeOne}{\text{PMODE\_REF1}}
\newcommand{\PredModeTwo}{\text{PMODE\_REF2}}
\newcommand{\BlockGlobal}{\text{GLOBAL\_BLOCK}}

\newcommand{\VectorFollowOne}{\text{VECTOR\_F1}}
\newcommand{\VectorFollowTwo}{\text{VECTOR\_F2}}
\newcommand{\VectorFollowThree}{\text{VECTOR\_F3}}
\newcommand{\VectorFollowFour}{\text{VECTOR\_F4}}
\newcommand{\VectorFollowFivePlus}{\text{VECTOR\_F5+}}
\newcommand{\VectorData}{\text{VECTOR\_DATA}}
\newcommand{\VectorSign}{\text{VECTOR\_SIGN}}

\newcommand{\DCFollowOne}{\text{DC\_F1}}
\newcommand{\DCFollowTwoPlus}{\text{DC\_F2+}}
\newcommand{\DCData}{\text{DC\_DATA}}
\newcommand{\DCSign}{\text{DC\_SIGN}}


\newcommand{\annotate}[2]{\marginpar{\color{commentcolour} #2 {\bf --- #1}}}

\newcounter{indent}
\newlength{\indentx}
\setlength{\indentx}{1em}

% Environment for pseudocode with a function declaration %
\newenvironment{pseudo}[2]
    {\newcommand{\dfindent}{\\\hline\hspace{\value{indent}\indentx}}
     \newcommand{\bsIF}[1]{\dfindent\text{{\bf if} (\text{$##1$}):}\stepcounter{indent} & &}
     \newcommand{\bsEND}{\addtocounter{indent}{-1}}
     \newcommand{\bsELSE}{\addtocounter{indent}{-1}\dfindent\text{{\bf else}:}\stepcounter{indent} & &}
     \newcommand{\bsELSEIF}[1]{\addtocounter{indent}{-1}\dfindent\text{{\bf else if} (\text{$##1$}):}\stepcounter{indent} & &}
     \newcommand{\bsWHILE}[1]{\dfindent\text{{\bf while} (\text{$##1$}):}\stepcounter{indent} & &}
     \newcommand{\bsFOREACH}[2]{\dfindent\text{{\bf for each} \text{$##1$} {\bf in } \text{$##2$}:}\stepcounter{indent} & &}
     \newcommand{\bsFORSUCH}[2]{\dfindent\text{{\bf for} \text{$##1$} {\bf such that} \text{$##2$}:}\stepcounter{indent} & &}
     \newcommand{\bsFOR}[2]{\dfindent\text{{\bf for} \text{$##1$} {\bf to} \text{$##2$}:}\stepcounter{indent} & &}
%     \newcommand{\bsRET}[1]{\dfindent\text{return \text{$##1$}}\addtocounter{indent}{-1} & &}
     \newcommand{\bsRET}[1]{\dfindent\text{{\bf return} \text{$##1$}} & &}
     \newcommand{\bsCODE}[1]{\dfindent\text{$##1$} & &}
     \newcommand{\bsITEM}[3]{\dfindent\text{$##1 = read\_##2()$} & & ##3}
     \setcounter{indent}{1}
     \hspace{0.5in}
     \begin{tabular}{|m{4.2in}m{0.0in}|m{0.4in}|}
         % firstline is function definition
         \hline
         \text{$#1(#2)$} : &  & %\textbf{Ref}
    }
    {    % last line is end of function
         \\\hline
         \end{tabular}
         }

% Environment for pseudocode without a function declaration %
%\newenvironment{pseudo*}[0]
%    {\newcommand{\dfindent}{\\\hline\hspace{\value{indent}\indentx}}
%     \newcommand{\bsIF}[1]{\dfindent\text{if (\text{$##1$}):}\stepcounter{indent} & &}
%     \newcommand{\bsEND}{\addtocounter{indent}{-1}}
%     \newcommand{\bsELSE}{\addtocounter{indent}{-1}\dfindent\text{else:}\stepcounter{indent} & &}
%     \newcommand{\bsELSEIF}[1]{\addtocounter{indent}{-1}\dfindent\text{else if (\text{$##1$}):}\stepcounter{indent} & &}
%     \newcommand{\bsWHILE}[1]{\dfindent\text{while (\text{$##1$}):}\stepcounter{indent} & &}
%     \newcommand{\bsFOREACH}[2]{\dfindent\text{for each \text{$##1$} in \text{$##2$}:}\stepcounter{indent} & &}
%     \newcommand{\bsFORSUCH}[2]{\dfindent\text{for \text{$##1$} such that \text{$##2$}:}\stepcounter{indent} & &}
%     \newcommand{\bsFOR}[2]{\dfindent\text{for \text{$##1$} to \text{$##2$}:}\stepcounter{indent} & &}
%%     \newcommand{\bsRET}[1]{\dfindent\text{return \text{$##1$}}\addtocounter{indent}{-1} & &}
%     \newcommand{\bsRET}[1]{\dfindent\text{return \text{$##1$}} & &}
%     \newcommand{\bsCODE}[1]{\dfindent\text{$##1$} & &}
%     \newcommand{\bsITEM}[3]{\dfindent\text{$##1 = read\_##2()$} & & ##3}
%     \setcounter{indent}{1}
%     \hspace{0.5in}
%     \begin{tabular}{|m{4.2in}m{0.0in}|m{0.4in}|}
%
%         \hline
%          &  &
%    }
%    {    % last line is end of function part
%         \\\hline
%         \end{tabular}
%         }

\newenvironment{pseudo*}[0]
    {\newcommand{\dfindent}{\\\hline\hspace{\value{indent}\indentx}}
     \newcommand{\bsIF}[1]{\dfindent\text{{\bf if} (\text{$##1$}):}\stepcounter{indent} & &}
     \newcommand{\bsEND}{\addtocounter{indent}{-1}}
     \newcommand{\bsELSE}{\addtocounter{indent}{-1}\dfindent\text{{\bf else}:}\stepcounter{indent} & &}
     \newcommand{\bsELSEIF}[1]{\addtocounter{indent}{-1}\dfindent\text{{\bf else if} (\text{$##1$}):}\stepcounter{indent} & &}
     \newcommand{\bsWHILE}[1]{\dfindent\text{{\bf while} (\text{$##1$}):}\stepcounter{indent} & &}
     \newcommand{\bsFOREACH}[2]{\dfindent\text{{\bf for each} \text{$##1$} {\bf in} \text{$##2$}:}\stepcounter{indent} & &}
     \newcommand{\bsFORSUCH}[2]{\dfindent\text{{\bf for} \text{$##1$} {\bf such that} \text{$##2$}:}\stepcounter{indent} & &}
     \newcommand{\bsFOR}[2]{\dfindent\text{{\bf for} \text{$##1$} {\bf to} \text{$##2$}:}\stepcounter{indent} & &}
%     \newcommand{\bsRET}[1]{\dfindent\text{return \text{$##1$}}\addtocounter{indent}{-1} & &}
     \newcommand{\bsRET}[1]{\dfindent\text{{\bf return} \text{$##1$}} & &}
     \newcommand{\bsCODE}[1]{\dfindent\text{$##1$} & &}
     \newcommand{\bsITEM}[3]{\dfindent\text{$##1 = read\_##2()$} & & ##3}
     \setcounter{indent}{1}
     \hspace{0.5in}
     \begin{tabular}{|m{4.2in}m{0.0in}|m{0.4in}|}

      \hline
      $\hdots$  &  &
    }
    {    % last line is end of function part
         \\\hline
         \end{tabular}
         }
\tableofcontents \clearpage

\section{Abstract}
This document is the specification of the Dirac video decoder and stream syntax.

Dirac is a video compression system utilising wavelet transforms and motion
compensation. It is designed to be simple, flexible, yet highly effective. 
It can operate across a wide range of resolutions and
application domains, including: internet and mobile streaming, delivery of 
standard-definition and high-definition
television, digital television and cinema production and distribution,
and low-power devices and embedded applications.

The system offers several key features:
\begin{itemize}
\item lossy and lossless coding using a common tool set
\item intra-coded modes for professional production applications
\item a special low delay mode for link adaption applications, such as the carriage of HDTV
over SDTV infrastructure
 \item motion-compensated (`long-GOP') modes for distribution applications
\item gradual quality reduction with increasing compression 
\end{itemize}




\section{Introduction}
\subsection{Purpose}
Dirac was developed to address the growing complexity and cost of current video
compression standards, which provide greater compression efficiency
at the expense of implementing a very large number of tools. Dirac is
a powerful and flexible compression system, yet uses only a small number
of core tools. A key element of its flexibility is its use of the wavelet
multi-resolution transform for compressing pictures and motion-compensated 
residuals, which allows Dirac to be used across a very wide range of resolutions
without defining additional profiles.




\subsection{Scope}This document specifies normative decoder operations and stream syntax.A 
number of other elements are also included for informative purposes, covering:

\begin{itemize}
    \item encoder operations
    \item encoder and decoder design and implementation issues, including error
recovery
	\item high-level design characteristics and operation
	\item potential application scenarios 
\end{itemize}

In particular we are well aware that many users of this document may wish
to make both encoders and decoders. There are many sources of information
on how to design efficient compression algorithms, for example for entropy coding,
motion estimation, frame-dropping, rate control, motion estimation and 
rate-distortion optimisation. This document does not attempt to address these
issues in detail, but to provide supplementary information
to allow those reasonably "skilled in the art" to develop a Dirac encoder
rapidly and accurately, and approach design compromises knowledgably.

\subsection{Document structure}
This document seeks to provide a wide range of information about Dirac.

XXX This document needs reordering!

Section \ref{outline} provides an overview of the algorithms used. This
is intended to give an outline of video compression techniques, and
describes Dirac as a simple evolution of the art, enabled by technology
advances.

Section \ref{conventions} gives an introduction to the conventions used
in the rest of the document. Sometimes we need to be clear about the
meanings of our words.

Section \ref{bytestream} gives a description of the Bytestream

Section \ref{arith-coding} explains how to unpack the
arithmetically-coded data.
XXX Why is this here???

Sections \ref{idwt-1} and \ref{idwt-2} give details about how to derive
the wavelet coefficients.

Sections \ref{mc-1} and \ref{mc-2} give details about how to use the
motion vectors together with the restored wavelet coefficients to give
the restored video sequence.

Finally, the appendices contain tables of standard settings, guidance on
the options available in the coder and other useful data.



%\section{Overview of Dirac video coding (Informative)}
%\subsection{A general introduction to video compression for beginners}This is a simple guide to video coding. You can miss this subsection out
if you already know the basics.

Video signals are made up of a succession of still pictures, displayed
one after the other. Each picture is made up from a series of elementary
units called pixels, arranged in a raster of lines.

The raw total capacity required for the transmission of moving pictures
is the product of the number of pixels per picture, pictures per second,
colours in use and the quantising accuracy adopted. In nearly every
field of application, the resultant raw bandwidth required exceeds that
available by more than an order of magnitude.  If we keep the same size
pictures, then the only variable we have with the simple system is to
change the quantisation accuracy and lowering the accuracy rapidly leads
to poor quality pictures.

The solution which has been used from before 1970 is to make a
prediction of the value of each pixel using information that should be
available in the decoder. In general, if the prediction is good, then
the difference between reality and the prediction is small. The entropy
of the difference signal is low, and so we need less capacity to deliver
it. One simple solution was to recognise that the quantisation accuracy
required for the difference signal could be coarser than that required
for the original picture, and so fewer bits are needed to deliver the
difference signal than for the original.

This has formed the basis for many of the early video compression
systems.  The elements that have changed over the years have been

\begin{itemize}
    \item the algorithms used for making the prediction, and
    \item the formats used for delivering the difference signal
    efficiently to the receiver.
\end{itemize}

Many of the changes have been enabled by the availability of better
electronic circuitry. Real-time operation puts an upper limit on the
time available for processing.  We can only use more sophisticated
algorithms if we can carry out the calculations in the time available.
Fortunately, the improvement in the speed of hardware has almost matched
the development of algorithms.

Early algorithms made use of simple predictions predicting one pixel
from others nearby in the picture, or (spatial prediction in the
jargon).

Predictions were helped when field-stores became affordable (and smaller
than a small garden shed) and it became possible to use information from
the previous field (temporal prediction in the jargon). This process
works well when the picture is still, but is less effective when the
pictures depict a lot of movement.

This led to a whole raft of developments, seeking to identify motion in
the picture. Knowing how parts of the scene are moving (motion
estimation in the jargon) allowed much more accurate prediction than had
been possible hitherto. These developments led eventually to mature
products such as the MPEG 2 compression system.

Having created a good prediction, it is inevitable that it will not be
perfect. The error signal is the signal we wish to convey to the
decoder.  In any system, the raw error signal is at least as
bandwidth-hungry as the original. In most cases it is slightly more
hungry.

However, the error signal has some physical properties which help us to
reduce its bandwidth. In information theory terms, the error signal is
rarely totally random, real pictures have properties which distinguish
them from random noise.  We can therefore use some of these properties
to reduce the bit-rate further.

The spectrum of the error signal is usually heavily biased towards the
low frequency end. This is a direct consequence of the prediction
process usually being reasonably good. It also turns out that the eye is
less sensitive to small inaccuracies in the high frequency components of
the error signal.

Taken together, these observations provide the potential for coarser
quantisation or omission of some of the components of the error signal.

There are several methods available if we wish to translate the temporal
signal we started with into the frequency domain. Although many coding
systems use Discrete Cosine Transforms, we preferred to use the Discrete
Wavelet Transform. This approach divides the signal into higher and
lower frequency sub-bands. By quantising the different sub bands
appropriately, we achieve significant reduction in bandwidth.

Finally, when all the information for this system is packed together,
there is still a structure it is not statistically random. Information
theory says that we can use entropy coding to further increase the
randomness, and thereby reduce the bandwidth.  One of the more powerful
of these is arithmetic coding the system adopted in Dirac.  So we have
now outlined some of the key elements of a modern coder.


Fig XX. An outline of a typical modern video coder.

It is worth a quick look at the simple representation of the encoder in
Fig. XX as it leads us to understand the receiver topology.

The input signal $V_{in}$ is compared with the prediction $P$ to produce
an error signal $e$.  This is then compressed and passed (with the
various elements of compression metadata) to the arithmetic coder for
transmission.

The prediction is created from a local version of the signal for a very
good reason.  This is the signal that the decoder is able to recreate
with the information available to it. The signal delivered to the
receiver  $e_{TQ}$ allows the receiver to recreate a close copy of the
prediction error $e$. If we compare the two signals,

\begin{align*}
  V_{in} &= P + e \\
  V_{local}	&= P + e'
\end{align*}

As the difference between the two error signals $e$ and $e'$ is the
distortion introduced by the compression algorithm, the local signal is
a close approximation to the input signal. Looking carefully at Fig. XX,
we can see that we can create a decoder using a subset of the encoder.
This is shown in Fig. XXX.

It is quite clear that the main elements of the receiver are duplicated
in the encoder.  The encoder has to maintain a local decoder within it,
in part so that the result of the compression can be monitored at the
time of compression, but mainly because compressed pictures must be used
as reference frames for subsequent motion compensation else the encoder
and the decoder will not remain in synchronism.

The motion vectors are delivered from the encoder as metadata. This
avoids the need to analyse motion vectors in the receiver, and allows
the encoder considerable flexibility in the choice of appropriate motion
vectors.  We will now move on to consider the application of this
technology to Dirac.


Fig. YY An outline of the decoder


%\subsection{Outline of the compressino methods used in Dirac}The Dirac video codec uses three main techniques to compress the signal:

\begin{itemize}
    \item Prediction using motion compensation to remove temporal redundancy
    \item Wavelet transformation and quantisation to remove spatial redundancy
    \item Arithmetic coding of the resulting data to maximise efficiency
\end{itemize}


Initially, similarities between frames (temporal redundancy) are
exploited to predict one frame from another. The process is aided by
motion vectors - metadata detailing where a particular pixel in the
predicted frame might have been in the reference frame. It is a bit
wasteful to assign a motion vector to each pixel, so pixels are
aggregated in blocks. In Dirac, blocks are overlapping. This reduces
some of the artefacts found at the boundaries of blocks in some earlier
systems. The motion is calculated to sub pixel accuracy.

Once the prediction has been made, it is compared with the actual image
to be transmitted.  The resulting difference (error) is potentially
greater in range than the original signal. To reduce it the difference
signal is then transformed using the discrete wavelet transform. This
process, for the majority of video sequences, produces coefficients
which are largely zero or near zero, and most of the non-zero
coefficients are concentrated near at the lower frequency end of the
range. The properties of the eye allow us to coarsely quantise the high
frequency coefficients.

Both the motion vectors and the quantised coefficients are then further
compressed using arithmetic coding.

%\subsection{Prediction using motion compensation}The object of motion compensation is to try to predict the picture in
one image from others in the sequence. In television, the picture
usually contains moving objects, so a key part of the process is to
identify the moving objects and their the details of the motion.

In Dirac, the motion vector is merely an indication of which pixel in
the reference picture can be used as a good prediction for a particular
pixel in the current picture.

%\subsubsection{Types of picture}Three types of picture are defined.

Intra pictures are coded without reference to other pictures in the
sequence. These pictures form a useful point to start the decoding
process. They also have uses if low-delay coding is required - a
sequence of Intra pictures has mimimum delay, at the expense of
potentially greater bandwidth requirements. There is a third use for
Intra pictures: when the sequence is highly dynamic, prediction using
motion compensation may be impractical. In such instances, a coder may
choose to default to a sequence of Intra frames.

Inter pictures are coded with reference to other pictures in the
sequence, and are split into two types:
\begin{itemize}
	\item references for other pictures
	\item not references for other pictures
\end{itemize}

Within the receiver, we have to arrange that all the references for
pictures are available at the right time. This means that pictures are
often delivered in a different sequence from the display sequence, to
ensure that the buffer memory in the receiver has the necessary
information to decode pictures when it is needed.


%\subsubsection{Blocks}In theory, we could define a motion vector for every pixel in the image.
This would be extremely data intensive and of no practical value.

Instead, pixels are grouped into small regions or blocks, with a single
motion vector assigned to each block. Ideally these blocks are large -
thus minimising the amount of information we have to transmit. However,
the larger the block, the greater the chance that we have more than one
object or region, and so a single motion vector might be inappropriate.

Although there are several standard block sizes identified within the
specification, you can choose your own, and that could be as large as
the original picture. This would not be unreasonable for a picture which
is a static scene, with the camera being panned across it - although
there are other ways of identifying this in Dirac.

One of the hard tasks the coder has is to identify which is the optimum
size of block, and the best compromise for the motion vector in the
block.

When we have created the prediction, we then calculate the error. The
error is often greatest at the block boundaries, as this is where the
motion vector is often least accurate.

To overcome this, Dirac overlaps blocks. Each pixel in the overlap
regions uses a weighted prediction, incorporating information from all
the blocks it may lie in. This smooths the error signal, and making the
following wavelet transformation much more effective.

%\subsubsection{Global motion parameters}Although we said that it is of no practical value to specify a motion
vector for each pixel, there are exceptions.

Some types of motion are more likely to be features of the camera than
the scene. For example, a camera may pan, tilt, rotate, zoom, sheer, or
change of perspective through, say tracking.

If the coder can identify such motion, then Dirac permits the motion to
be signalled by a small set of parameters, which in effect assign a
motion vector to each pixel.

%\subsubsection{Accuracy}Often, we find that the motion vectors required to match a pixel in a
reference frame to the predicted frame are not integer values.

In Dirac, we can specify motion vectors to 1/8 pixel accuracy.

This means that the coder and decoder have to carry out a process that
is effectively an upconversion of the signal.

%\subsubsection{Intra frames}The Intra fields are not predicted using motion compensation.

This leaves us with a potential problem. The wavelet transform process
assumes that we have a signal which tends to a zero average.

The Intra field is processed to give it a zero mean by removing the DC
component (average value) of each block from the signal. The DC
components are signalled separately. In effect the DC components are a
local spatial prediction, rather than a temporal prediction as applied
in motion compensation.

%\subsection{Wavelet tranforms}The consequence of processing the Intra frames by removing the DC
values, and the Inter frames by removing the predicted values gives us a
difference signal. This difference signal is hopefully largely zero, but
can have some large peaks, of either polarity. The signal usually has a
large amount of low-frequency energy, but with occasional elements of
high frequency as the prediction process gets it wrong.

Conventional theory says that we can manipulate this signal in the
frequency domain to reduce the amount of information we need to
transmit. The properties of the eye are such that many of the higher
frequency components are less sensitive to coarse quantisation.

We can use wavelet transforms in Dirac. The wavelet transform
decorrelates the data in a roughly frequency-sensitive way, and
preserves the fine details of images better than the ubiquitous Discrete
Cosine Transform.

%%\subsubsection{Wavelet analysis}Put simply, wavelet analysis splits a signal into a low and a high
frequency component, and then subsamples the two partial streams by a
factor of two.

The two filters used to split the signal are called the analysis
filters. It is not possible to use just any pair of half-band filters to
do this. There is an extensive mathematical theory of wavelet filter
banks. Appropriately chosen filters allow us to undo the aliasing
introduced by the critical sampling in the down conversion process and
perfectly reconstruct the signal.

Iterative use of a wavelet process allows us to decompose the
low-frequency component into successively lower components.

For pictures, we apply wavelet filters in both the horizontal and
vertical directions. This results in four so-called subbands, termed
Low-Low (LL), Low-High (LH) High-Low (HL), and High-High (HH). The LL
band can be iteratively decomposed to create a wavelet transformation of
many levels.

%\subsubsection{Parent-child relationships}In wavelet analysis, each subband represents a filtered and subsampled
version of the picture. Because all the subbands are derived from a
single source image, there is likely to be some form of relationship
between the images in the different subbands.

The coefficients of each subband relate to specific areas in the image.
We find that there is often a correlation between these specific areas
in the different subbands.

The subsampling structure means that a coefficient in the lowest level
corresponds to  2 by 2 block of coefficients in the next level, and so
on up the levels. In the jargon, the low-level component is referred to
as the parent and the higher-level component is referred to as the
child.

When coding picture features (edges on objects especially), significant
coefficients are found distributed across subbands, in position related
by the parent-child structure.

A coefficient of a child is more likely to be significant if its parent
is also significant. Children with zero or small parents seem to have
different statistics from children with large parents or ancestors.

These features allow us to entropy code the wavelet coefficients after
they have been quantised.

%\subsection{Entropy coding}Transmission of the raw motion vectors and wavelet coefficients is
inefficient. There are still many redundancies in the data, and the form
of the data is itself suboptimum.

Coding of the motion vectors is especially important for codecs with a
high level of motion accuracy (quarter or eighth pixel say). Motion
vector coding and decoding is quite complicated, since significant gains
in efficiency can be made by choosing a good prediction and entropy
coding structure.

The processes of removing the redundancies is probably one of the most
complicated part of the codec. Similar processes are used for coding the
motion vectors and the wavelet coefficients. In various ways they use a
combination of

\begin{itemize}
	\item prediction,
	\item binarisation,
	\item context modelling and
	\item adaptive arithmetic coding.
\end{itemize}

%\subsubsection{Entropy coding of wavelets}The entropy coding used by Dirac in wavelet subband coefficient coding
is based on three stages: binarisation, context modelling and adaptive
arithmetic coding.

Figure: Entropy coding block diagram

The purpose of the first stage is to provide a bitstream with easily
analysable statistics that can be encoded using arithmetic coding which
can adapt to those statistics, reflecting any local statistical
features.

Binarisation is the process of transforming the multi-valued coefficient
symbols into bits. The resulting bitstream can then be arithmetic coded.

Transform coefficients tend to have a roughly Laplacian distribution,
which decays exponentially with magnitude. This suits so-called unary
binarization. Unary codes are simple variable-length codes in which
every non-negative number $N$ is mapped to $N$ zeros followed by a 1:


\begin{verbatim}
U(0)    =   1
U(1)    =   0   1
U(2)    =   0   0   1
U(3)    =   0   0   0   1
U(4)    =   0   0   0   0   1
U(5)    =   0   0   0   0   0   1
U(6)    =   0   0   0   0   0   0   1
Bins:       1   2   3   4   5   6   7
\end{verbatim}

For Laplacian distributed values, the probability of $N$ occurring is
$2-(|N|+1)$, so the probability of a zero or a 1 occurring in any unary
bin is constant. So for an ideal only one context would be needed for
all the bins, leading to a very compact and reliable description of the
statistics. In practice, the coefficients do deviate from the Laplacian
ideal and so the lower bins are modelled separately and the larger bins
lumped into one context.

The process is best explained by example. Suppose one wished to encode
the sequence:

\begin{verbatim}
-3 0 1 0 -1
\end{verbatim}

When binarized, the sequence to be encoded is:

\begin{verbatim}
0 0 0 1 | 0 | 1 | 0 1 | 1 | 1 | 0 1 | 0
\end{verbatim}

The first 4 bits encode the magnitude, 3. The first bit is encoded using
the statistics for Bin1, the second using those for Bin 2 and so on.
When a 1 is detected, the magnitude is decoded and a sign bit is
expected. This is encoded using the sign context statistics; here it is
0 to signify a negative sign. The next bit must be a magnitude bit and
is encoded using the Bin 1 contexts; since it is 1 the value is 0 and
there is no need for a subsequent sign bit. And so on.

The context modelling in Dirac is based on the principle that whether a
coefficient is small (or zero, in particular) or not is well-predicted
by its neighbours and its parents. Therefore the codec conditions the
probabilities used by the arithmetic coder for coding bins 1 and 2 on
the size of the neighbouring coefficients and the parent coefficient.

The reason for this approach is that, whereas the wavelet transform
largely removes correlation between a coefficient and its neighbours,
they may not be statistically independent even if they are uncorrelated.
The main reason for this is that small and especially zero coefficients
in wavelet subbands tend to clump together, located at points
corresponding to smooth areas in the image, and as discussed elsewhere,
are grouped together across subbands in the parent-child relationship.

Conceptually, an arithmetic coder can be thought of a progressive way of
producing variable-length codes for entire sequences of symbols based on
the probabilities of their constituent symbols.

For example, if we know the probability of 0 and 1 in a binary sequence,
we also know the probability of the sequence itself occurring. So if

$P(0)=0.2, $

$P(1)=0.8$

then

$P(11101111111011110101)=(0.2)*3*(0.8)*17=1.8 * 10^{-4}$ (assuming
independent occurrences).

Information theory then says that optimal entropy coding of this
sequence requires $log_2 (\frac{1}{p})=12.4$ bits. Arithmetic coding
produces a code word very close to this optimal length, and
implementations can do so progressively, outputting bits when possible
as more arrive.

All arithmetic coding requires are estimates of the probabilities of
symbols as they occur, and this is where context modelling fits in.
Since arithmetic coding can, in effect, assign a fractional number of
bits to a symbol, it is very efficient for coding symbols with
probabilities very close to 1, without the additional complication of
run-length coding. The aim of context modelling within Dirac is to use
information about the symbol stream to be encoded to produce accurate
probabilities as close to 1 as possible.

Dirac computes these estimates for each context simply by counting their
occurrences. In order for the decoder to be in the same state as the
encoder, these statistics cannot be updated until after a binary symbol
has been encoded. This means that the contexts must be initialised with
a count for both 0 and 1, which is used for encoding the first symbol in
that context.

An additional source of redundancy lies in the local nature of the
statistics. If the contexts are not refreshed periodically then later
data has less influence in shaping the statistics than earlier data,
resulting in bias, and local statistics are not exploited. Dirac adopts
a simple way of refreshing the contexts by halving the the counts of 0
and 1 for that context at regular intervals. The effect is to maintain
the probabilities to a reasonable level of accuracy, but to keep the
influence of all coefficients roughly constant.

%\subsubsection{Entropy coding of motion vectors}The basic format of the coding the motion vectors is similar to the
coding of wavelet data: it consists of prediction, followed by
binarisation, context modelling and adaptive arithmetic coding.


Figure: motion vector entropy coding architecture

All the motion vector data are predicted from previously encoded data
from nearest neighbours.

%\subsection{Bytestream}The bytestream of Dirac is the complete, compressed video stream, ready
for file storage or transmission.

The features are

\begin{itemize}
    \item A parse structure which gives ready access to the video, even
    in mid file. This identifies all the static features of the stream.
    Intelligent systems can use the redundancy in this structure to
    recover from errors in the stream.

    \item Blocks of data comprising the arithmetically-coded motion
    vector information

    \item Blocks of data comprising the arithmetically-coded wavelet
    coefficients
\end{itemize}

Because of the arithmetic coding, it is difficult to provide a simple
specification which gives meaningful details of the bytestream without
some reference to either the early coding processes, or the later
decoding process. The simple description of the bytestream would only
refer to compressed motion vectors, and compressed wavelet coefficients.
We have therefore had to provide extra information to explain how to
unpack this data, and how to use the resulting structured data.


%\clearpage
%\section{The conventions used in the specification}In this specification we have adopted three specific processes to define
what we mean.

\begin{itemize}
    \item The structure of the system is described in a series of parse
    diagrams, which are represented pictorially.

    \item When reading the data from the bytestream, the receiver
    follows the process identified by the parse diagrams to allocate the
    data to the appropriate variables in the receiver's memory. The
    variables are described in a pseudocode, which is loosely based on
    the Python programming language.

	\item The path through the tree, is again described in the pseudocode.
\end{itemize}

This section provides the necessary information for understanding the
normative specification.

%\subsection{Pictorial representation}Hierarchical bitstream elements are defined in terms of their component
data elements, which are presented as block parse diagrams.  A
blockparse diagram of the form

XX Include diagram here

specifies that Data Element 1 is always followed in the bitstream by
Data Element 2.

A blockparse diagram of the form

XX Include diagram here

specifies that one or more data elements of type Data Element 2 always
follow Data Element 1. The Data Element 2 elements may be different,
i.e. may be differently instantiated according to their own
specification.

A blockparse diagram of the form

XX Include diagram here

specifies that Data Element 2 conditionally follows Data Element 1.
Whether or not it does will depend on previously derived decoder
variable values (not necessarily always contained in Data Element 1).


%\subsection{Pseudocode}The specification is written in a style of pseudocode. The intention is
to make it easy to understand, but at the same time to be reasonably
rigorous. We chose to base this loosely on the syntax used in Python. We
have to emphasise that this is pseudocode: it only resembles Python. In
some aspects the pseudocode differs from the conventional Python syntax.

The pseudocode is intended to achieve maximum clarity and hence it may
contain redundant code and other inefficiencies.

As we are doing little more than read some data and assign that data to
variables, only a small subset of the Python syntax is used. The
elements used are introduced in the next subsection for the benefit of
those who are not familiar with Python. The interested reader is
recommended to pick up one of the vast wealth of books on Python for
further information.

In the pseudocode, variable assignments and function calls that
represent data in the byte stream are emphasised by bold text.

%\subsection{The syntax of the pseudocode}This description is a simple presentation of the syntax used in this
specification.

\textbf{Variables}

Variables are strongly typed - but the choice of type is implicit rather
than being explicitly defined.

Variables are usually local to the procedure in which they are defined.
Thus the same variable name may refer to different variables in
different scopes.

Global variables are identified in Appendix XX. There is no formal
declaration of global variables in the pseudocode as would normally be
the case in Python.

Variables are passed to procedures by reference. That means they may be
modified within the procedure and the modifications will be seen by
subsequent procedures and lines of code.

\textbf{Lists}

Lists in the pseudocode are one dimensional arrays, defined in the
manner: [1,2,3,4]. Elements of the list are counted from the left,
starting with zero as the index. E.g:

\begin{verbatim}
	x 		= [1,2,3,4]
	x[2] 	= 3
\end{verbatim}

\textbf{Dictionaries}

Dictionaries in the pseudocode are look-up tables defined in the manner:

$\lbrace a: \alpha , b: \beta, c: \gamma \rbrace$

The look-up parameters a, b and c return values of $\alpha$, $\beta$,
and $\gamma$  respectively.

\textbf{Flow}

The pseudocode comprises a series of statements, linked by functions and
flow control statements such as if, while, and for.

The statements do not have a termination character, unlike the ; in C
for example.  Blocks of statements are indicated by indentation:
indenting in begins a block, indenting out ends one.

Statements that expect a block (and hence a following indentation) end
in a colon.

\textbf{if}

The if control evaluates a function, and if true, passes the flow to the
block of following statement or block of statements. If the control
evaluates as false, then there is an option to include one or more elif
controls which offer alternative responses if some other condition is
true.  If none of the preceding controls evaluate to true, then there is
the option to include an else control. E.g.


\begin{verbatim}
if control:
    statement1
elif anothercontrol:
    statement2
else:
    statement3
\end{verbatim}

\textbf{for}

The for control repeats a loop for each instance of a variable in a list. E.g.

\begin{verbatim}

for variable in list:
    statement1
    statement2
\end{verbatim}

\textbf{while}

The while control repeats a loop until the switch variable is true. When
it is false, the loop breaks to the next statement outside the block.
E.g.

\begin{verbatim}
while (switch):
    statement1
    statement2        # changes value of switch to false
NextStatement
\end{verbatim}

\textbf{Functions}

Functions are identified by a function name, and the parameters that are
passed to the function when it is evaluated.

The function is defined using pseudocode in a block (which does not
require the def or return  statements which are required in Python. The
return  statement is optional, depending on functionality). E.g.

\begin{verbatim}
function (parameter):
    variable = 2*parameter
    return variable
\end{verbatim}

\textbf{Arithmetic}

The following operators are used for arithmetic:

$=$	equals

$+$	addition

$-$	subtraction

$*$	multiplication

$**$	raising preceding variable to the power of the following variable

$//$	truncating division

$/$	simple division

$\%$	indicates modulo division - the remainder after the preceding
variable has been divided by the following variable

\textbf{Logic}

The following operators are used in logic:

$= =$ 	test of equality of two variables

$<$	less than

$<=$	less than or equal to

$>$	greater than

$>=$	greater than or equal to

\textbf{Bitwise operations}

$>>$	shift all the elements in a bit sequence one step right

$<<$	shift all the elements in a bit sequence one step left

\textbf{Comments}

In many places, we have felt it useful to add comments to the code.
These are preceded with the \# symbol and do not form part of the main
code.

%\subsection{Reading data}When reading the data from the bytestream, we choose to read in an
appropriate format, not just in bytes. The different formats used in the
pseudocode are:

\textbf{read\_bool()}

reads a single bit from the byte stream and returns a Boolean value.

\textbf{fread\_uint(number\_of\_bytes)}

reads and returns the unsigned integer that occupies a fixed length
format of number\_of\_bytes in the byte stream.

\textbf{read\_uegol()}

reads and returns an unsigned integer in the variable length exp Golomb
format

\textbf{read\_segol()}

reads and returns an signed integer in the variable length exp Golomb
format

\textbf{read\_ba(context)}

reads a single arithmetic coded bit from the bytestream and returns a
Boolean value.

For definition see section XX (Arithmetic Decoding) below.

\textbf{read\_uua(context\_list)}

reads and returns an unsigned integer encoded in the bytestream as an
arithmetic coded unary binarisation. context\_list is a list of contexts
for each bin. If the number of contexts in the list is less than the bin
number then the last context on the list is used. That is a common
context is used for all the higher bins.

\textbf{Read Arithmetic Coded Unsigned Integer }

\begin{verbatim}
read\_uua(context_list):
    context_index = 0 #Bin Number (numbered from zero)
    max_index = len(context_list) - 1
    value = 0
    more = True
    while (more):
        if ( read_ba(context_list[context_index]) ):
            more = False
        else:
            value += 1
            if (context_index < max_index):
                context_index += 1
    return value
\end{verbatim}

\textbf{read\_uuta(context\_list)}

reads and returns an unsigned integer encoded in the bytestream as an
arithmetic coded truncated unary binarisation. context\_list is a list
of contexts for each bin. If the number of contexts in the list is less
than the bin number then the content of the bin is assumed to be 1 (i.e.
the conditional probability of that bin, the context, is exactly 1).

\textbf{Read Truncated Arithmetic Coded Unsigned Integer }
\begin{verbatim}

read\_uuta(context_list):
    context_index = 0 #Bin Number (numbered from zero)
    max_index = len(context_list) - 1
    value = 0
    more = True
    while (more):
        if ( read_ba(context_list[context_index]) ):
            more = False
        else:
            value += 1
            if (context_index < max_index):
                context_index += 1
            else:
                more = False
    return value
\end{verbatim}

\textbf{read\_sua(context\_list)}

reads and returns a signed integer encoded in the bytestream as an
arithmetic coded unary binarisation.  context\_list is a two element
list. The first element is a  context list for reading the magnitude of
the signed integer.  The second element is the context for the sign bit.
The magnitude context list is a list of contexts for each bin. If the
number of contexts in the magnitude context list is less than the bin
number then the last context on the list is used. That is a common
context is used for all the higher bins.

\textbf{Read Arithmetic Coded Signed Integer }

\begin{verbatim}
read\_sua(context\_list):
# context_list contains a magnitude context list followed
# by a sign context
    #Read magnitude
    magnitude = read_uua(context_list[0])
    if ( magnitude==0 ):
        value = 0
    else:
        #Read sign
        sign = read_ba(context_list[1])
        #Determine value
        if ( sign = False):
            value = magnitude
        else:
            value = -magnitude
    return value
\end{verbatim}

\textbf{next\_byte()}

In order to define the decoding process it is necessary to assume an
implementation dependent function for reading bytes from the bitstream.
This function hereafter is referred to as next\_byte.

When called next\_byte returns the next byte from the bitstream.

\textbf{next\_parse\_code()}

Reads bytes from the stream until it encounters a parse code prefix,
then reads the parse code and returns its value.

The input to the decoder is a stream of bytes. But to achieve
compression individual bits are used to pack the data more efficiently.
So the decoder needs a way to read a sequence of individual bits from
the byte stream. The procedure to read bits must retain information
between invocations because sometimes a new byte must be read from the
stream and sometime not. The process of reading bits from the byte
stream is, therefore, a state machine.

\textbf{next\_bit()}

Reads and returns the next bit in a sequence of bits

\textbf{read\_sint()}

Reads and returns a signed variable length integer

%\subsection{Byte alignment}If a structure is byte aligned the first bit of the structure coincides
with the first bit of the next byte in the bytestream.

If a structure is byte aligned the previous structure may need to be
padded with up to 7 bits. These seven padding bits are undefined.
Decoding the bytestream does not depend on the value of these undefined
bits.

%\subsection{Default parameters}The bytestream implements a policy for default parameters designed to
simplify the use of Dirac and enable easy parsing of the bytestream.

All Sequence Parameters and Source Parameters and some decoding
parameters have default values. The default values are determined by the
Video Format. The default values are used unless an appropriate flag is
set to explicitly signal otherwise. Typically this means that few
parameters have to be signalled in the bytestream and a user does not
have to understand all the parameter to be able to use the codec.

Sequence and source parameters remain constant throughout the entire
sequence.

The default decoding parameters, corresponding to the chosen Video
Format, are loaded at the beginning of each Picture. Therefore any
changes to the default parameters only affect the current picture. This
supports random access to individual pictures, using the Parse Unit
Offsets, without having to parse any header information.

%\subsection{Numbering of parse diagrams}The bytestream can be considered as a process, in which various
groupings or structures of data are passed from encoder to decoder.
Within the bytestream, there is a natural association of data in
structures. When the parse diagram is written out in full, it has the
form of a tree and branch map.

In this text, parse diagrams are numbered in the form X.Y.

X is the level of the diagram, with the highest level of X being 0.

Y is the relevant branch, with all levels in the same branch having the
same value of Y. The same value of Y can only be found in the same
branch. Sub-branches lead to a new branch number.


%\clearpage
%\section{Semantics (Meaning of terms)}The parameters in use in Dirac are divided into three general categories

\begin{itemize}
	\item Largely static parameters and synchronisation parameters
	\item Parameters describing the prediction mechanism, including the motion vectors
	\item Parameters describing the wavelet analysis of the residue after prediction
\end{itemize}

%\subsection{Static and synchronisation parameters}These parameters are used to enable the decoder to synchronise with the
incoming bytestream, and to indicate the parameters of the decoded
signal.

\textbf{Access Unit}

The Access Unit Header marks a point in the Bytestream from which
decoding may commence.

Sometimes sequences are simply played from start to finish. Often,
however, it is necessary to start playing a sequence part way through a
stream (for example if a viewer has just connected to a
broadcast/multicast transmission or following transmission errors). To
achieve this the player must be able to start decoding at some point in
the middle of a stream without requiring prior information.

Access Unit Header are points in a Dirac bit stream at which a player
can start decoding. An Access Unit Header provides the sequence and
decoding parameters with which to configure the decoder. The Access Unit
Header should be interpreted as giving an access point to the data in
display order.

An Access Unit Header does NOT imply that all subsequent frames in coded
order can be decoded. An Access Unit Header is followed immediately by
an Intra frame. However, typically, some Inter frames that are earlier
in display order, will be transmitted following an Intra frame. Such
Inter frames might be predicted from frames prior to the Access Unit
Header. Clearly such Inter frames cannot be decoded from only the
information following the Access Unit Header. However subsequent frames
can be decoded. Given a limited reordering depth in a prediction
structure, eventually all frames after a Access Unit Header will be
decodeable.

The parameters specified in the Access Unit Header remain the same
throughout a Video Sequence. That is, the parameters in later Access
Unit Headers simply repeat those in earlier headers. The repetitions are
included to provide entry points to start decoding the Bytestream.

See also Video Sequence

\textbf{Access Unit Parse Parameters}

The Access Unit Parse Parameters indicate the Picture Number which can
be decoded using the available information, and the Version Number,
Profile and Level of Dirac coding that has been used.

\textbf{Access Unit Picture Number}

The Access Unit Picture Number indicates the picture number of the first
picture that may be decoded following this Access Unit header.

\textbf{Aspect Ratios}

Aspect Ratios refers to the Pixel Aspect Ratios, not the image aspect
ratios. The Pixel Aspect Ratio value of an image is the ratio of the
intended spacing of horizontal samples (pixels) to the spacing of
vertical samples (picture lines) on the display device. Historically,
the aspect ratio was not square, and differed between 525-line and
625-line systems.

The shape of the elementary pixels is signalled as the quotient of two
variables, the numerator or dividend and denominator or divisor. The
numerator refers to the horizontal dimension and the denominator to the
vertical dimension.

Pixel Aspect Ratios are fundamental properties of sampled images because
they determine the displayed shape of objects in the image. Failure to
use the right parameters will result in distorted images, for example
circles will be displayed as ellipses etc. In spite of their fundamental
nature, pixel Aspect Ratios are not rigidly defined in many video
standards.

Some HDTV standards and computer image formats are defined to have Pixel
Aspect Ratios that are exactly 1:1.

Some video processing tools require an Image Aspect Ratio. This may be
derived from the pixel Aspect Ratio. The image aspect ratio is the ratio
of horizontal to vertical pixels multiplied by the pixel aspect ratio.
So, for example, for a 525 line 704 x 480 line picture the image aspect
ratio is $\frac{704 * 10}{480 * 11}$ which is exactly 4:3.

You are strongly advised to use one of the default Aspect Ratios.
However, if you know what you are doing and don't like the default
values you can, in principle, define your own. But our advice is don't
do it, just say no.

\textbf{Bytestream}

The Bytestream is the complete flow of transmitted Dirac data, after
coding.

\textbf{Chroma Excursion}

The dynamic range of the chrominance components.

\textbf{Chroma Format}

In sampled television signals, the signal is usually split into three
component: luminance and two colour components. The eye has around 20
times more receptors sensitive to the luminance of a scene than to the
chrominance. As a consequence, it is possible to provide less colour
detail than luminance detail and still satisfy the eye of the beholder.

The format for this subsampling is usually specified in the style 4:2:2
or 4:2:0 for example. This nomenclature may be written

luminance samples: chroma 1 samples : chroma 2 samples,

where chroma 1 and chroma 2 are the two colour components, usually $U$
and $V$ or $C_b$ and $C_r$. The figures refer to the number of samples
of each component along a small segment of the video line,
conventionally taking the number of luminance samples as 4. In the case
of 4:2:2, the signal would be $Y U Y V Y U Y V ...$., whilst in the case
of 4:2:0, it does not mean that there are no samples of $V$ at all; it
is just that on one line it is $YUYYUY ...$, and on the next it is
$YVYYVY ...$.

\textbf{Chroma Offset}

The zero level of the chrominance components.

\textbf{Clean Area}

Clean area is pure metadata, not used in any way to influence or control
the decoding process.

The Clean Area defines an area within which picture information is
subjectively uncontaminated by all edge effects. These may be transient
(and other) distortions. These may be coding artefacts, deliberate
introduction of elements such as time codes, etc.

Clean area is useful metadata for D-Cinema in which the 2K or 4K image
size is used as a container for a variety of picture formats with
different aspect ratios.  It is appropriate to display the Clean Area
rather than the whole picture.  The clean area is defined by the
position of the top left corner, and its horizontal and vertical
dimensions.

Clean area is desirable for ordinary SD video with a 720 pixel width
because only about 702 of the pixels represent active video, the
remainder allowing for overscan and filters falling off the edge of the
picture. The analogue PAL specification could be interpreted to say that
the clean width is 704 pixels. If you then use a pixel aspect ratio of
12:11 you end up with a picture aspect ratio of exactly 4:3. If you
could not specify the clean area then you would have big problems
specifying either the picture aspect ratio or the pixel aspect ratio. It
is simply wrong to say that the whole of the 720 pixels represent the
4:3 picture.

See also Clean Width, Clean Height, Left Offset, Top Offset.

\textbf{Clean Height}

The height of the Clean Area in pixels.

\textbf{Clean Width}

The width of the Clean Area in pixels.

\textbf{Colour Matrix}

We use the conventional relationships between the different
representations of the colour matrixes linking the different portrayals
of the colour channels. The $E_{Y}$, $E_{Cb}$, $E_{Cr}$ values are
derived from the $E_R$, $E_G$, $E_B$ values by the following equations.


The inverse transform, which would be used by a player, is the following.

Conversion between $E_R$, $E_G$, $E_B$ and $E_Y$, $E_{Cb}$, $E_{Cr}$ are
matrix operations. They can be specified by two of $K_R$, $K_G$, $K_B$.
The remaining $K$, say $K_Y$ can be calculated as $K_Y=1- K_R - K_B$.

\textbf{Colour Primaries}

The Colour Primaries define the assumed colours of the phosphors. The
choice offered is between three sets - the SMPTE C set, the EBU Tech
3213 set and the ITU Rec 709 set.

\textbf{Colour Space}

Dirac defines the subsets of colour primaries for the phosphors, the
colour matrixes and the transfer function used for the relationship
between signal level and light output.

We have deliberately eschewed provisions for a custom colour space.

Knowing the Source Parameters for the colour space, if the available
display has different parameters, it is possible to convert the received
signal to optimise the colour in the display.

All current video system use the following model for YUV coding of the
RGB values (computer systems often omit coding to and from YUV).

The R, G \& B are tristimulus values $(e.g. candelas/meter^{2})$. Their
relationship to CIE XYZ tristimulus values can be derived from the set
of primaries and white point defined in the colour primaries part of the
colour specification using the method described in SMPTE RP 177-1993. In
this document the RGB values are normalised to the range [0,1], so that
RGB=1,1,1 represents the peak white of the display device and RGB=0,0,0
represents black.

The $E_R$, $E_G$, $E_B$ values, also in the range [0,1], are related to
the RGB values by non-linear transfer functions $f()$ and $g()$. The
transfer function $f()$ is typically performed in the camera and is
specified in the Transfer Characteristic part of the Colour
Specification. For aesthetic and psychovisual reasons the transfer
function $g()$ is not quite the inverse of $f()$. In fact the combined
effect of $f()$ and $g()$ is such that

Where $r$ is the rendering intent or end to end gamma of the system,
which may vary between about 1.1 and 1.6 depending on viewing
conditions. The non-linear $E_R$, $E_G$, $E_B$ values are subject to a
matrix operation (known as non-constant luminance coding), which
transforms them into the $E_Y$, $E_{Cb}$, $E_{Cr}$ values. $E_Y$ is in
the range [0,1] and the $E_{Cb}$ and $E_{Cr}$ values are in the range
[-0.5, 0.5]. This is YUV coding and sometimes the U and V components are
subsampled, either horizontally or both horizontally and vertically. UV
sampling is specified by Colour Format.

Sometimes the matrix operation is omitted (or, equivalently, a unit
matrix is used). In this case $E_Y$, $E_{Cb}$, $E_{Cr}$ values are
simply the $E_Y$, $E_{Cb}$, $E_{Cr}$ values. This is signified by Colour
Format = RGB.

The $E_Y$, $E_{Cb}$, $E_{Cr}$ values are mapped to a range of integers
$Y$, $C_{b}$, $C_{r}$. Typically they are mapped to an 8 bit range [0,
255]. The way in which $E_Y$, $E_{Cb}$, $E_{Cr}$ values are mapped to
the integer values that are actually compressed is specified by Signal
Range.

\textbf{End of Sequence}

The End of Sequence code marks the end of a Video Sequence. It is not
necessarily the end of a complete work. The editing process may require
concatenation of Video Sequences. The decoder should therefore be
prepared to respond to data which follows the End of Sequence code.

\textbf{Field Dominance }

This is a flag which indicates that, for interlaced scanning, the top
field is first or second.

See also Interlace

\textbf{Frame Rate}

The Frame Rate is the rate at which frames of video are displayed.
Historically these have been close to the frequencies of the main
electricity supply: hence the global variations.

In Dirac, the frame rate is signalled as the quotient of two variables,
the numerator or dividend and denominator or divisor.

\textbf{Image Aspect Ratio}

The ratio of the horizontal to vertical dimensions of the Clean Area of
the image.

\textbf{Image Size}

The Image Size is a measure of the segmentation of the image. It is the
size of the image, measured in pixels of luminance. It is signalled as
the number of pixels along a line (the Luma Width) and the number of
lines in a frame (the Luma Height).

\textbf{Interlace}

If we have an interlaced source, each frame of the Video Sequence is
built up from two fields. These fields are spatially offset, with each
field delivering alternate lines of the final frame. For transmission,
the two fields can be combined, with field lines interleaved line by
line (pseudo-progressive format, the default for Dirac). Alteratively
they can be transmitted sequentially - interleaved field by field as in
conventional analogue broadcasting (giving low delay and low resource
coding).

The field interleaving flag indicates non-default field interleaving,
and the Sequential Fields (Boolean) parameter indicates whether the
fields are interleaved as pseudo-progressive or sequential fields.

With field sequential coding the picture sequence is a sequence of
fields rather than frames. So, for example, for interlaced 625 line
video we would have picture size 720 pixels by 288 lines, frame rate
25~Hz and Sequential Fields True.

Since Interlace is pure metadata there is no change to the coding
algorithm. It may seem unusual that the picture size refers to a field
rather than the whole frame.

However this preserves the separation of the Source Parameters, which
are pure metadata, from point of view of the decoding process.

When two fields are interlaced to make a frame, the top field (the odd
lines in the image) is usually transmitted first and the bottom field
(the even lines) comes second for images sourced in 625-line PAL. The
order is reversed for signals sourced from 525-line NTSC sources. The
difference arises because the image information starts on an odd line
number in PAL systems and an even line number in NTSC.

As an example, the default settings for the standard definition of SD
576 (the parameters which would be used as a basis for conventional PAL
in Europe) assume a pseudo-progressive or film mode, based on 25~Hz
progressive scan. If we wish to signal an interlaced signal, then it is
only necessary to modify the interlace flag. There needs to be a
sympathetic handling of the decoded signal when it is displayed. Whether
this is done by signalling or re-formatting is not a matter of
specification.

See also Scan Formats

\textbf{Left Offset}

The offset of the top left hand corner of the Clean Area from the left
hand side of the full image.

See also Clean Area

\textbf{Level}

The Level is an indication that the transmission is intended for a
decoder which may not necessarily decode the complete range of Video
Formats enabled by Dirac. Together with the Profile, it describes the
subset. A level is a set of decoder resource requirements that must be
satisfied in order to decode a bitstream, together with a set of
constraints on the bitstream that ensures that these requirements are
not exceeded.

This version of the specification does not define distinct levels and
profiles.  In future, we expect it to define the processing power of the
decoder - and hence the likely range of Video Formats which can be
handled by the device. One particular element we expect to be included
is the number of reference frames which can be stored.

Level is also used as a label for a parameter in the wavelet transform.
Hopefully there is no ambiguity caused by the use of the same name for
two different functions. The context should be clear.

See also Profile and Level in Section \ref{waveletparameters}

\textbf{Luma Excursion}

The dynamic range between black and white levels.

\textbf{Luma Offset}

The signal level corresponding to black level.

\textbf{Next Parse Offset }

Next Parse Offset is added to the Bytestream to simplify parsing. It
represents the offset in bytes from the start of the current Parse Info
to the start of the next Parse Info. So counting forward Next Parse
Offset bytes from the first byte (0x42 equivalent to B) of the current
Parse Info should yield a byte of value 0x42 or B corresponding to the
start of the next Parse Info. The Previous Parse Offset of the current
Parse Info equals the Next Parse Offset of the previous Parse Info.

\textbf{Parse Info}

Information which identifies the structure of the bytestream.

\textbf{Parse Info Prefix}

The Parse Info Prefix is the sequence of bytes 0x42  0x42  0x43  0x44,
which are the ASCII codes for BBCD.  This identifies the stream as a
Video Sequence coded using Dirac compression.

The Parse Info Prefix is present to allow an application to find a point
from which to start decoding. That is, the function of Parse Prefix
Header is to synchronise the decoder with the Bytestream. Decoding can
start from any Access Unit Header. The decoder first needs to find a
Parse Info structure. It should then check the Parse Code in the Parse
Info. If the following Parse Unit is an Access Unit Header then the
decoder can start decoding. If the Parse Unit is a Picture then the
decoder should skip forward by Next Parse Unit bytes (from the start of
the Parse Info Prefix) to the next Parse Info. The decoder would
continue skipping forward unit it locates an Access Unit Header. Note
that the decoder does not need to parse any Parse Units in order to
navigate through the stream to find an Access Unit Header. The Previous
Parse Offset is provided to allow searching backwards through the
Bytestream.

Any particular instance of the Parse Info Prefix in the Bytestream may
not, necessarily, indicate the start of a Parse Info structure. This is
because other parts of the Bytestream may, by chance, introduce these
bytes into the Bytestream. The use of arithmetic coding in Dirac means
that it is impossible to directly avoid accidentally introducing the
Parse Info Prefix.

When encoding a bytestream it is not necessary to avoid accidentally
introducing Parse Info Prefix sequences. They are present to allow
synchronisation of the bytes stream with the decoder and this can be
ensured, even in the presence of spurious Parse Info Prefixes, as
follows. When the decoder finds a Parse Info Prefix it should skip
forward by Next Parse Offset (or back by Previous Parse Offset) and
check whether the next three bytes are a Parse Info Prefix. If so the
decoder can be reasonably certain that it has found a genuine Parse Info
Prefix. If it does not find another Parse Info Prefix it was probably
unlucky enough to have found a spurious Parse Info Prefix. In this case
it should search for the next Prefix and repeat the test.

The probability of a spurious Parse Info Prefix is low; 1 in $2^{32}$
since the prefix is 4 bytes long. This is the probability of finding two
Parse Info Prefix sequences separated by Next Parse Offset. The test
outlined in the previous paragraph is, therefore, adequate in practice.
For the paranoid the test may be extended to find three Parse Info
Prefixes separated by the indicated Next Parse Offsets. This extended
test probably reduces the chance of failure to less than once in the
lifetime of the universe and should be sufficient for all but the
extremely cautious.

The test for two appropriately separated Parse Info Prefixes is, anyway,
prudent in any channel subject to bit errors even in the absence of
spurious Prefixes.

\textbf{Parse Unit}

The fundamental aggregation of data within the bytestream.

This definition of Dirac only includes two sorts of Parse Unit, Access
Unit Headers and Pictures. It is envisaged that other types of Parse
Unit may be introduced in future to carry data such as user data or
extension data.

\textbf{Picture Number}

Each picture has a unique Picture Number.

The Picture Number is a unique label (within the stream) indicating the
presentation/display ordering of the pictures. In a valid sequence the
Picture Numbers increment by one between consecutive frames (modulo
$2^{32}$, so 0x000 follows 0xFFFF).

If the Picture Number is even, and the picture is a field, then that
field has even parity. It is the first field of a pair of fields in a
frame. An interleaved Video Sequence coded sequentially (as opposed to
pseudo-progressively) starts with an even Picture Number. It does not
have to start with the Picture Number 0x000.

\textbf{Pixel Aspect Ratio}

See Aspect Ratio

\textbf{Previous Parse Offset}

The Previous Parse Offset is added to the Bytestream to simplify
parsing. It is the number of bytes backwards to the start of the
previous parse unit. The Previous Parse Offset of the current Parse Info
equals the Next Parse Offset of the previous

\textbf{Parse Info.}

See Next Parse Offset

\textbf{Profile}

The Profile is an indication that the transmission is intended for a
decoder which may not necessarily decode the complete range of Video
Formats enabled by Dirac. A profile is a set of decoding tools necessary
to decode a bit stream. A level is a set of decoder resource
requirements that must be satisfied in order to decode a bitstream,
together with a set of constraints on the bitstream that ensures that
these requirements are not exceeded.

This version of the specification does not define levels and profiles.

In future, we expect it to define the subset of tools that will be
accessible to the relevant decoder.

See also Level

\textbf{Scan Format}

In television, the picture is usually created by scanning the image as a
series of horizontal lines (some early formats used vertical lines, and
old electron beam devices had lines which were nearly, but not quite,
horizontal. We will ignore these in the Dirac specification).

A format in which all the lines in the frame are scanned in sequence is
called progressive format.

A format in which every other line in the frame is scanned, and then the
other half is called interleaved format. Each group of half lines is
called a field.  In old analogue broadcasting the two fields would be
broadcast one after the other: this is sequential field transmission.

It is also possible to combine the two fields to make a frame. This is
called pseudo-progressive format.

See also Interlace, Frame Rate

\textbf{Sequence Parameters}

A description of the parameters of the picture which are necessary to
decode the image. These include the Luma Width, Luma Height, Chroma
Format and Video Depth.  These parameters are essential to decoding and
displaying the bitstream.  The Sequence Parameters are intended to
change rarely if ever. If a change is necessary, it is recommended that
the bitstream is terminated by a Stop Sequence Parse Code and a new
bitstream is initiated.

See also Source Parameters

\textbf{Signal Range}

The Signal Range defines how the signal is scaled and clipped prior to
matrixing and display within the bits available, and is merely metadata
describing the source. It provides information to allow a bi-polar
signal such as $U$ and $V$ to be restored for display. Signal Range
embraces the set of parameters Luma Offset, Luma Excursion, Chroma
Offset and Chroma Excursion.

The offset and excursion values should be used to convert the
integer-valued decoded luma and chroma data $Y$, $C_{b}$, $C_{r}$ to
intermediate values $E_Y$, $E_{Cr}$, and $E_{Cb}$ by the recipe

$E_Y$, is normally clipped to the range [0,1], and $E_{Cr}$, and
$E_{Cb}$ to the range [-0.5,0.5].

This effectively clips

$Y$ to [LUMA\_OFFSET, LUMA\_OFFSET+LUMA\_EXCURSION]

and

$C_{b}$, $C_{r}$ to [CHROMA\_OFFSET-LUMA\_EXCURSION/2,
LUMA\_OFFSET+LUMA\_EXCURSION/2]

However, maintaining an extended RGB gamut may mean that either such
clipping is not done, or non-standard offset and excursion values are
used to extract the extended gamut from the non-negative decoded $Y$,
$C_{r}$, and $C_{b}$ values.

See also Luma Offset, Luma Excursion, Chroma Offset and Chroma Excursion.

\textbf{Source Parameters}

Source Parameters are a description of the parameters of the picture
which are not necessary to decode the image, but which may be desirable
to ensure accurate display of the decoded images. These include elements
such as Frame Rate, interlace information, Pixel Aspect Ratios,
information about the clean area, luma and chroma parameters and the
colour system being used.

The interpretation of Source Parameters by a display mechanism
interfacing with a compliant decoder is not specified. However, it would
make jolly good sense to follow the recommendations and interpretations
signalled if at all possible.

Likewise, encoders should ensure that accurate Source Parameter
information is encoded to maximise the potential quality of displayed
video.

See also Sequence Parameters

\textbf{Stream}

A Stream is a concatenation of Video Sequences.

See also Video Sequence

\textbf{Top Offset}

The Top Offset is the offset of the top left hand corner of the Clean
Area from the top of the full image.

See also Clean Area

\textbf{Transfer Function}

The Transfer Function defines the non-linear processing used in the
camera when converting the received light flux into an electrical
signal.

\textbf{Version Number}

The Version Number is coded as an unsigned integer with the first minor
version starting at zero.

Dirac is expected to be released in different versions at different
times. Version numbering will have two elements: the major version
number and the minor version number. Later versions will have the higher
numbers. Small changes in specification will be indicated by increases
in the minor version number.

The major version number defines the version of the syntax with which
the bit steam complies. Decoders that comply with a version of the spec
must be able to parse all previous versions too. Decoders that comply
with a version of the spec may not be able to parse the bit stream
corresponding to a later spec.

The first major version starts at one. Major version zero is a draft.
All minor versions of a spec should be functionally compatible with
earlier minor versions with the same major version number. Later minor
versions may contain corrections, clarifications, disambiguations etc;
they must not contain new features.

The first minor version starts at zero.

\textbf{Video Depth}

Video Depth is the number of bits used to represent the video data, i.e.
the video word width (typically 8  or 10 bits).

Video Depth is different from the Signal Range. The former defines how
many bits are used to contain the signal, and is used in the decoding
process. The input data, be it $Y$, $U$, $V$, or $R$, $G$ and $B$ are
all coded as if they were unsigned integers.

Note, this is separate from what the bits represent. It would be
possible, for example, to have an 8 bit signal represented in a 10 bit
word (in which case either the upper two, or lower two, bits of the word
would always be zero). The meaning of the bits is defined in the Signal
Range (the Luma and Chroma Offsets and Excursions) part of the Source
Parameters. Video Depth relates to how the video is coded. The Signal
Range relates to what the numbers mean and how the video should be
displayed.

\textbf{Video Format}

The Video Format indicates whether the signal conforms to something
close to one of the conventional video formats (such as High Definition,
PAL, NTSC, QCIF etc) or whether it is a custom format, with all
parameters available for setting independently. The video format
embraces a range of Source Parameters and Sequence

\textbf{Parameters.}

See also Source Parameters, Sequence Parameters and Appendix XXX [the
default format settings]

\textbf{Video Sequence}

A Video Sequence is a collection of images which can be of any length,
which have constant Source Parameters (e.g. picture size, aspect ratio
etc.) If the parameters need to change the only way to do it is to
signal the end of a Video Sequence and start a new Video Sequence.

The process of editing two coded sequences together might introduce
presentation order picture numbers which are not contiguous. This is
accommodated by introducing an End of Sequence Parse Code before a cut
so that the decoder would restart after a cut.

%\subsection{Motion vector parameters}
Dirac uses motion-adaptive prediction to reduce the bit rate of a
sequence. The prediction parameters define the prediction tools up to
the point that the error signal or residual is calculated.

\textbf{Block Parameters Index}

Block Parameters Index defines the default settings for the sizes and
separation of the blocks used in the prediction of pictures. The
parameters defined are the Chroma Block Width, Chroma Block Height,
Horizontal Chroma Block Separation, Vertical Chroma Block Separation,
Luma Block Width, Luma Block Height, Horizontal Luma Block Separation,
Vertical Luma Block Separation.

\textbf{Common Mode}

Common Mode indicates that the same values of Motion Vector and
Prediction Mode are being used within a Superblock.

\textbf{Chroma Block Width}

The width of a chrominance prediction block in pixels.

\textbf{Chroma Block Height}

The height of a chrominance prediction block in pixels.

\textbf{Chroma DC Residual}

In Intra Frames, this is the residual of the predicted mean level of
each Block.

See also DC Value

\textbf{DC Value}

When coding an Intra Reference Picture or an Intra Non-reference
Picture, the picture is not predicted, but simply wavelet transformed.
Whereas the mean value of predicted images is usually zero, the mean
value of intra picture is not. To avoid the block artefacts that this
would produce, the mean level of the Blocks is removed from the signal
before the wavelet transform process. This mean level is used as the
prediction.

Instead of transmitting the absolute value of the mean level for each
Block, the DC Value is predicted by reference to other Blocks in the
Superblock, and the difference signalled. The same process is applied to
luminance and the two chrominance signals.

See also Luma DC Residual, Chroma DC Residual.

\textbf{Global Motion}

The Global Motion Flag indicates the presence of global motion data. If
TRUE, the Global Motion Only Flag indicates whether only global motion
data is present.

Global Motion is a technique which works well when the whole, or at
least a large proportion, of the image is being subjected to the same
transformation. Dirac uses an eight-parameter model that allows for pan,
zoom, rotation, shear and change of perspective.

\textbf{Horizontal Chroma Block Separation}

The separation of chrominance prediction blocks horizontally.

\textbf{Horizontal Luma Block Separation}

The separation of luminance prediction blocks horizontally.

\textbf{Horizontal Perspective}

Metadata which describes the Motion Vector element caused by changes in
perspective.

See also Appendix XXX [Global Motion Compensation]

\textbf{Horizontal Offset Residual}

When the Motion Vector has been calculated in the encoder, it comprises
an indication of the  horizontal and a vertical offsets of the reference
pixel from the pixel to be calculated. This is transmitted as a
difference from a value that is predicted using information from
surrounding Blocks - i.e. as a residual (or difference) from the
predicted value.

\textbf{Horizontal Perspective}

XX I am not sure what physical features this conveys.

\textbf{Inter Reference Picture}

A picture which is coded by prediction with reference to other pictures,
and which itself is used as a reference for prediction when coding other
pictures.

There are two different types of Inter Reference Pictures. One uses only
one other picture as a reference. The other type can use two other
pictures as references.

\textbf{Inter Non Reference Picture}

A picture which is coded by prediction with reference to other pictures,
but which itself is not used as a reference for prediction when coding
other pictures.

There are two different types of Inter Non Reference Pictures. One uses
only one other picture as a reference. The other type can use two other
pictures as references.

\textbf{Intra Non Reference Picture}

A picture which is coded without prediction with reference to other
pictures, and which itself is not used as a reference for prediction
when coding other pictures.

\textbf{Intra Reference Picture}

A picture which is coded without prediction with reference to other
pictures, but which itself is used as a reference for prediction when
coding other pictures.

\textbf{Luma Block Width}

The width of a luminance prediction block in pixels.

\textbf{Luma Block Height}

The height of a luminance prediction block in pixels.

\textbf{Luma DC Residual}

In Intra Frames, this is the residual of the predicted mean level of
each Block.

See also DC Value

\textbf{Motion Vector}

A Motion Vector is an indication of which pixels in the reference frame
can be used as predictors for a particular pixel in the predicted frame.
It is the offset of the predicted pixel from the reference pixel (and is
conveyed, either as a global field, or on a block by block basis).

The observant will notice that this vector is a spatial vector, being a
spatial offset. Only by weighting by the temporal separation between the
reference and predicted fields, with due note of the sign, can a true
indication of motion be deduced. We would have called it prediction
vector, but there is a weight of existing custom and practice against us
- so Motion Vector it remains.

Two types of Motion Vector information are used: global and block motion
vectors. Global motion is intended to describe the motion of the
background, using a parametric model. The block motion vectors are
intended to describe the more varied motion of the foreground. The two
type of motion information are used together to define the overall
motion vectors.



XX Could introduce elements here about Motion Vector Prediction process
from TD 0.9

\textbf{Motion Vector Precision}

Motion Vector Precision is what it says on the tin. It is an important
factor in achieving efficient video compression. If the Motion Vector
Precision is too low the motion-compensated prediction residuals will be
larger than necessary and require more bits to be coded. However if the
Motion Vector Precision is too high the motion vectors themselves will
require a disproportionately large number of bits to be coded.

Empirically, motion vector precisions of � or � of a pixel have been
found to work well. However the optimum motion vector precision depends
on many factors, such as the nature of the sequence to be coded and the
desired compression quality. Dirac defaults to � pixel motion vector
precision but provides the flexibility to use a non-default value of
precision.

XX In the range ?????

\textbf{Pan Tilt}

A measure of the degree of horizontal movement of the whole image (Pan)
and vertical movement of the whole image (Tilt). This is used as part of
the information provided for Global Motion Compensation.

See also Appendix XXX [Global Motion Compensation]

\textbf{Perspective}

A measure of the change in perspective of the image. This is used as
part of the information provided for Global Motion Compensation.

See also Appendix XXX [Global Motion Compensation]

\textbf{Perspective Exponent}

A multiplier which allows us to use integers in place of floating point
in the delivery of the Perspective metadata.

\textbf{Picture Prediction Parameters}

Dirac predicts Inter frames from one or two reference frames using
motion compensation. Frame prediction parameters and data are not
included for Intra frames, which are indicated by the Start Code in the
Parse Information.

Dirac's motion model is overlapping block motion compensation. The block
sizes can be adapted to match the requirements of the Sequence. The
spatial displacement of each block, from the corresponding position in a
reference frame, is coded in the bitstream. These displacements are
known as motion vectors. Motion vectors are the displacement that should
be applied to the reference frame to predict the current frame. The name
motion vector is, therefore, a misnomer because they are actually
prediction displacement vectors. Nevertheless the term motion vector is
used for consistency with industry practice

Two types of motion vector information are used: global and block motion
vectors. Global motion is intended to describe the motion of the
background, using a parametric model. The block motion vectors are
intended to describe the more varied motion of the foreground. The two
types of motion information are used together to define the overall
motion vectors.

Global Motion parameters are included, or not, depending on the value of
the Prediction Mode flags. One or two sets of Global Motion parameters
are included in the Frame Prediction, one for each reference frame.

When predicting an image from two different pictures, we are able to
weight the contributions. This aids, amongst other predictions, the
ability of Dirac to handle cross-fades between images and fades to black
or white.

\textbf{Prediction Mode}

Prediction Mode is a pair of bits, each of which indicates whether one
of the references is used to form a motion compensated prediction of the
picture. If a Prediction Mode bit is asserted (True) then the
corresponding reference picture is used in the prediction. An intra
prediction block is indicated when both bits are not asserted (False).

\textbf{Prediction Unit}

The Prediction Unit is the process which identifies whether Blocks
within a Superblock can use Common Modes - using the same Motion Vectors
and Prediction Modes - or whether different Motion Vectors and
Prediction Modes are appropriate.

\textbf{Picture Weights}

When using more than one reference image for prediction, the default is
to weight the two equally relevant. At times of cuts or fades in the
sequence, it may be that the default setting is invalid, so there is an
option to set the weighting pragmatically.

\textbf{Reference Picture Number}

Dirac predicts Inter frames from one or two reference frames using
motion compensation. The Reference Picture Numbers are the offset(s) of
the one or two Reference Pictures from the Picture Number of the picture
being decoded.

\textbf{Retired Picture List}

In Dirac, reference frames need to be retained for a while to act as
references for succeeding frames. Non-reference frames are always
discarded when the current output frame number exceeds their frame
number.

The Dirac decoder contains a small buffer of previously decoded frames.

An important aspect of the way the decoder works is how it manages which
frames to retain in its frame buffer and which frames it discards.

The encoder may explicitly specify frames that are no longer required in
the buffer.  To support this method of frame management each frame may
contain a list of frames that should be discarded from the buffer.

If no frames are specified to be retired then Dirac discards the oldest
frame first, but only as necessary. That is, when space is needed in the
frame buffer the frame with the lowest frame number is discarded. It is
up to the encoder to ensure that this default process does not discard
frames that are needed by the decoder. This default discard procedure is
only invoked when the frame buffer size would otherwise exceed that
specified by the decoder level and no signalling is present to indicate
which frames should be discarded.

It is the responsibility of the encoder to ensure that the correct
frames are retained in the buffer. The decoder may assume that the
reference frames it requires will be available in the buffer.

\textbf{Split Mode}

Split Mode indicates that different values of Motion Vector and
Prediction Mode are being used within a Superblock.

\textbf{Superblock}

In many instances, the Motion Vectors of adjoining blocks are similar.
Dirac therefore aggregates Blocks into Superblocks to enable Prediction
Modes and Motion Vectors to be transmitted more efficiently.

The metadata describing the Prediction Modes and Motion Vectors can be
the same (Common Mode) or different (Split Mode).

Superblocks are four Blocks wide and four Blocks high. They are arranged
such that the overlapped part of the Blocks at the edge of the image
falls outside the image. There is always an integer number of
Superblocks, so extra Blocks (with no source content) have to be added
to ensure complete population of the space.

See also Common Mode, Split Mode, Prediction Mode Residual 1, Prediction
Mode Residual 2, DC Value, Motion Data, Luma DC Residual, Chroma DC
Residual

\textbf{Vertical Chroma Block Separation}

The separation of chrominance prediction blocks vertically.

\textbf{Vertical Luma Block Separation}

The separation of luminance prediction blocks vertically.

\textbf{Horizontal Offset Residual}

When the Motion Vector has been calculated in the encoder, it comprises
an indication of the  horizontal and a vertical offsets of the reference
pixel from the pixel to be calculated. This is transmitted as a
difference from a value that is predicted using information from
surrounding Blocks - i.e. as a residual (or difference) from the
predicted value.

\textbf{Vertical Perspective}

Metadata which describes the Motion Vector element caused by changes in
perspective.

See also Appendix XXX [Global Motion Compensation]

\textbf{Zoom Rotate Sheer}

A measure of the amount of zoom, rotation and sheer in the image. This
is used as part of the information provided for Global Motion
Compensation. The parameters in the bytestream are presented as a
combination of these parameters in a matrix representation.

See also Appendix XXX [Global Motion Compensation]

\textbf{Zoom Rotation and Sheer Exponent}

A multiplier which allows us to use integers in place of floating point
in the delivery of the Zoom Rotate Sheer metadata.

%\subsection{Wavelet parameters}
\textbf{Coefficient Context}

The wavelet coefficients are decoded using a probability model derived
from previously decoded coefficients. The probability model determines
the arithmetic decoding contexts that are used. So, before the quantised
coefficient can be read, it is first necessary to select the right
context to use, based on previously decoded coefficients. This is done
by using the Coefficient Context

After the quantised coefficient has been read the contexts, must be
updated to reflect the new probabilities.

\textbf{Codeblock Mode}

A flag which indicates the functionality of the quantisers.

XXX Needs better description.

\textbf{Chroma Height}

The height of the chrominance blocks input to the wavelet transform.

\textbf{Chroma Transform Data}

The Chroma Transform Data is the data after applying the wavelet
transform to the residual.

XXX is it before or after arithmetic coding?

\textbf{Chroma Width}

The width of the chrominance blocks input to the wavelet transform.

\textbf{Horizontal Codeblocks}

The number of Codeblocks across the width of the image. This allows us
to calculate the width of the Codeblocks in pixels.

\textbf{Luma Height}

The height of the luminance blocks input to the wavelet transform.

\textbf{Luma Transform Data}

The Luma Transform Data is the data after applying the wavelet transform
to the residual.

XXX is it before or after arithmetic coding?

\textbf{Luma Width}

The width of the luminance blocks input to the wavelet transform.

\textbf{Quantiser Codeblock}

XX Is this the definition of the codeblocks subject to the spatial
partition??????

\textbf{Spatial Partition}

The Spatial Partition enables us to chose whether to use separate
quantisers for each code block in a subband or a single quantiser to be
used for the whole subband.

These different spatial partition modes can be used to support
region-of-interest coding.



\textbf{Subband}

The subband names, LL, LH, HL, HH, correspond to the frequencies in the
subband.

Wavelet analysis results in a filtered and subsampled version of the
picture. When the picture is subject to wavelet analysis in two
directions this results in the four so-called subbands termed Low-Low
(LL), Low-High (LH), High-Low(HL) and High-High (HH). The first
descriptor refers to horizontal frequencies, the second to vertical
frequencies. L represents low frequencies and H represents high
frequencies. The order of these letters (horizontal, vertical) is
consistent with names used in the literature.

However this is the opposite order from with the array indices used in
this document.  XX Do we need to say the so what factor.

When several levels of wavelet are used, the LL band (only) is
iteratively decomposed. As a consequence, there are Low-High (LH),
High-Low(HL) and High-High subbands for each iteration, plus an extra
subband for the outstanding Low-Low information. This latter is
effectively a low frequency, sub-sampled version of the original.

In the processing, Subband is the metadata describing the transformed
data, which component it is, i.e. which level and which band.

\textbf{Transform Data}

Transform Data is the data after applying the wavelet transform to the
residual.

XXX is it before or after arithmetic coding?

\textbf{Transform Depth}

Same as Wavelet Depth: the total number of wavelet levels.

XX Do not understand why there are two names for it.

\textbf{Transform Parameters}

The transform parameters are the framework used for delivering the
Transform Data. The Transform Parameters describe the Wavelet Filter,
the Wavelet Depth and the Spatial Partition.

See also Wavelet Filter, Wavelet Depth and Spatial Partition

\textbf{Vertical Codeblocks}

The number of codeblocks across the height of the image. This allows us
to calculate the height of the Codeblocks in pixels.

\textbf{Wavelet Coefficients}

The Wavelet Coefficients are decoded using a probability model derived
from previously decoded coefficients. The probability model determines
the contexts that are use. Different sets of contexts are used for the
XXX"follow" bits, but a single context is used for all the XXX"data".

The magnitude and the sign of the coefficient are modelled separately.

In coding the magnitude, different sets of contexts are used for the
"follow" bits, but a single context is used for all the "data". The
probability distribution of coefficient magnitude is modelled as
depending on whether the parent coefficient was zero or non zero. The
probability distribution is also assumed to depend on the sum of
(previously decoded) neighbouring coefficients. Because the probability
distribution is usually peaked around zero it is only the context that
models the probability of zero/non-zero coefficient that changes with
the neighbourhood sum.

The probability distribution of the sign of the coefficient is assumed
to depend on an appropriate neighbouring coefficient.

\textbf{Wavelet Depth }

Wavelet Depth is the total number of wavelet levels. Usually this is
four, i.e. we do 4 splits both horizontally and vertically.

\textbf{Wavelet Filter}

XXX [filter pairs??] The definition of the filter used in the wavelet
analysis (and this, as a consequence, defines the filter used in the
synthesis process too).

\textbf{Wavelet index}

The Wavelet index is a parameter which allows us to identify which of
the wavelet filters we are using for the lifting filters.

\textbf{Zero Residual}

XXX Zero Residual is a flag which signals that there is no Residual to
decode.


%\clearpage
%\section{Logical constructs used in Dirac}This section describes the conceptual structures that exist within the
Dirac codec.

%\subsection{Sequence}
A consecutive list of frames that all share the exact same sequence and
display parameters is a sequence.  Otherwise put, the sequence and
display parameters must not change during a sequence.  A sequence may be
terminated by an end of sequence notification.

The frames within a sequence are numbered consecutively in display
order.

\begin{figure}
    \centering
    \includegraphics[width=0.7\textwidth]{figs/sequence}
    \caption{Sequences, Access Units and Frames}
    \label{fig:sequence}
\end{figure}


%\subsection{Parse unit}
Seekable constructs within the bitstream are called parse units.  Parse
units start with a synchronisation word, type identifier and pointers to
the start of the next/previous parse unit.

Synchronisation words allow for easy stream synchronisation.

There is no deliberate attempt to prevent the arithmetic coding of data
in the bytestream from containing accidental repetitions of the
synchronisation word. A decoder should check that the next/previous
pointers in the parse unit point to a valid synchronisation word,
especially on start up.

There are only two types of seekable construct within the bitstream:
Access Unit Headers and Frames.

No synchronisation words are used within a video frame to regain
synchronisation.

%\subsection{Access unit}Access units group frames together. The existance of multiple Access
Units in a sequence allows the sequence to be randomly accessed to the
granularity of an Access Unit.

Access Units commence with an Access Unit Header, containing parameters
that stay in effect for the whole Access Unit (ie, until the next Access
Unit starts).

Access units are unbounded in length.  The end of an Access Unit is
signalled by the start of another.

Access units are logical constructs.  Only an Access Unit Header is
transmitted at the start of the access unit in the bitstream.

Access unit headers contain three categories of information:

\begin{itemize}
    \item Parse parameters -- provide the decoder with information
        required to parse all subsequent data.

    \item Sequence parameters -- provide the video parameters for
        correct decoding.

    \item Display parameters -- provide metadata about the video that is
        not required in the decode process, but may be required by a
        display divice to present the video correctly.
\end{itemize}

The first frame in an Access Unit must be an Intra coded frame.

%\subsection{Frames}Frames are the fundamental components of a television image. In Dirac,
each transmitted frame comprises picture data in the form of wavelet
transformed residuals and optional motion vector data.  Each frame may
be Inter of Intra coded, the former using a motion compensated
prediction based upon one or two reference frames, the latter not.

A frame may be stored for later use as a reference frame.  The choice
depends on the properties of the frame type: whether it is an Inter or
Intra frame,  the reference status (Reference/NonReference) and the
number of reference frames used in the prediction.

Reference frames are stored in the decoder until they are either
explicitly signalled as expired by the encoder, or implicitly retired
according to the level and profile settings.

%\subsection{Coordinate systems}
\ldots

%\subsection{Frame ordering}Each frame is given a unique number, consecutively increasing for each
frame in the original source material; this ordering is known as display
order, the order in which frames must be displayed.  In order to reduce
the codec's complexity, each frame the decoder receives must be received
after any reference frames it uses; this is the coded order, the order
in which coded frames are transmitted. See
figure~\ref{fig:frame-ordering}. Note that this means that the
transmitted order of the frames is not the display order. There will
therefore have to be a buffer of sufficient size to accommodate the
reordering.

\begin{figure}
    \centering
    \includegraphics[width=0.9\textwidth]{figs/frame-ordering}
    \caption{Frame reordering for transmission}
    \label{fig:frame-ordering}
\end{figure}

References may not persist across Video Sequence boundaries.

\paragraph{Frame ordering across Access Unit boundaries}
Consider figure x, with two Access Units $A$ and $B$.  While frame
x occurs in display order before the start of access unit $B$, it
references frames x1 (the first frame of $B$) and therefore must be
coded and transmitted after x1.  If decoding were to commence at the
start of $B$, frame x would not be decoded.



%\subsection{Wavelet transform}The Discrete Wavelet Transform (DWT) takes an image and recursively
decomposes it into sets of horizontal/vertical frequency subbands of low
and high frequency. The result is a set of data which occupies the same
total memory space as the original image, however energy is concentrated
in the components corresponding to the lower frequency subbands.

\begin{figure}
    \centering
    \includegraphics[width=0.9\textwidth]{figs/dwt}
    \caption{Logical structure of DWT transformed frame}
    \label{fig:dwt}
\end{figure}

The inverse transform recursively builds up the image from the lowest
frequency subbands to the highest.  The forward and inverse transforms
are capable of perfectly reconstructing the original image.

The DWT used in Dirac has been optimised using a method known as
lifting.  Lifted wavelet filters have some useful properties:
\begin{itemize}
    \item they may filter data in place (i.e. using the same memory for
    the image, the transformed image and the partially processed
    subsets).

    \item the filtering operations are shorter than convolutional
        filters.

    \item the same filters may be used for the inverse and forward
        transforms.
\end{itemize}

Part of the forward transform process involves padding the image to
allow the wavelet transform to opperate correctly.  The padded data is
also required in the inverse transform, but is discarded afterwards.



%\subsection{Subband coding}Using the subband numbering in figure~\ref{fig:dwt}, subbands are
transmitted from 10 downto 1, ie from the lowest frequency to the
highest frequency.  Since the wavelet transform and its inverse are
lossless, coding gains are made by quantizing the subbands for
transmission.

Further coding gains may be made by exploting the correlation in
frequency components, whereby each pixel in a subband is predicted from
some of its neighbours.  Correlation accross different frequency bands
is also exploited.

An extension avaliable to encoders is to use multiple quantizers per
subband.  This is achieved by subdividing a subband into a number of
codeblocks, each having its own quantizer.  As an aside, this means that
we can identify parts of the picture which require most accurate coding
(for example we often take most notice of errors in the region of a
person's eyes when looking at images of faces) and enhance the coding
accuracy in this area. This process is sometimes referred to as region
of interest coding.

%\subsection{Blocks}For the purpose of local Motion Compensation, the whole image is divided
up into a number of overlapping blocks that covers the entire image.

\begin{figure}
    \centering
    \subfigure[A block]{
        \label{fig:}
        \includegraphics[width=1in]{figs/block}
    }
    \hspace{0.5in}
    \subfigure[Positioning of blocks in a frame]{
        \label{fig:}
        \includegraphics[width=0.25\textwidth]{figs/block-offset}
   }
   \caption{block \ldots}
   \label{fig:blocks}
\end{figure}

Figure~\ref{fig:blocks} shows a block and how they overlap at the
pictures left and top edges.

For frames that have their chroma components downsampled, the block
sizes are reduced by the same factor.


%\subsection{Local motion compensation}
For each block, a vector is provided that gives the relative position of
a block of equal dimension in a reference frame.  This referenced block
is weighted according to a global weight for that frame and weighted
again using a weighting matrix designed to make blocks overlap nicely
and reduce high reduce high frequency components entering the residual
at block edges.

Dirac allows Motion Vectors to be specified to subpixel precision.  When
using a Motion Vector to subpixel precision, the reference image is
upconverted over the referenced area and a sampled upconverted reference
is used.

The maximum motionvector precision is an eighth of a pixel.

Motion Vectors are predicted


%\subsection{Global motion compensation}
\ldots



%\subsection{Superblocks}
To allow a more efficient representation of blocks and Motion Vectors,
and to allow further explotation of the spatial redundancy in motion
vectors and additionally allow a tradeoff between Motion Vector accuracy
to bytestream symbols required, blocks are transmitted in a 4 by 4
arrangement called a Superblock (figure~\ref{fig:superblock-pu-16of1x1}).

\begin{figure}
    \centering
    \subfigure[16 perdiction units, each a single block]{
        \label{fig:superblock-pu-16of1x1}
        \includegraphics[width=0.25\textwidth]{figs/superblock-16pu}
    }
    \hspace{0.5in}
    \subfigure[4 prediction units, each covering 2x2 blocks]{
        \label{fig:superblock-pu-4of2x2}
        \includegraphics[width=0.25\textwidth]{figs/superblock-4pu}
    }
    \hspace{0.5in}
    \subfigure[1 prediction unit, covering 4x4 blocks]{
        \label{fig:superblock-pu-1of4x4}
        \includegraphics[width=0.25\textwidth]{figs/superblock-1pu}
    }
    \caption{Illustration of grouping blocks into prediction units
    (numbered) and their relationship to superblocks}
    \label{fig:superblock-pu}
\end{figure}

Blocks can be merged inside the superblock into arrangements of 4 of 2 by 2
blocks, or 1 of 4 by 4 blocks (figures~\ref{fig:superblock-pu-4of2x2} and
\ref{fig:superblock-pu-1of4x4} respectively).  In each
case, only one motion vector is transmitted per merged set of blocks
(also known as a prediction unit).

Each prediction unit has a prediction mode associated with it, which may
be one of:
\begin{itemize}
    \item Intra -- for blocks with no Motion Vector associated with
        them.
    \item Ref1Only -- the Motion Vector applies to the first reference frame
        only.
    \item Ref2Only -- the Motion Vector applies to the second reference
        frame only.
    \item Ref1and2 -- two Motion Vectors are sent, each using one
        reference frame.
\end{itemize}

Global motion compensation is opted in/out of at the Superblock level.
If opted in, all prediction units within the super block will be
globally motion compensated, unless they signal that a particular
prediction unit should be intra coded instead.



%\subsection{Arithmetic coding}All the data for Motion Vectors and the transformed residuals are
transmitted after having been arithmetically coded.

The arithmetic coder is a black box device that is given an array of
data to code. It codes each symbol in context.

Arithmetic coding achieves good compression through maintaining accurate
counts of the symbol probabilities.  Each set of symbol probabilities is
held in a context (in effect recognising which other data in the array
may be a good match to the symbol to be coded).

Different contexts are used depending upon where the symbol originated
or depending upon some aprori information about the symbol.



\clearpage
\section{preparing decoded data for idwt}
\subsection{Subband Coefficient Data decoding}This Section specifies the $transform_data()$ process for decoding wavelet subband
coefficients.
elements of the bitstream. This process is invoked in decoding Subband
Data elements.

Inputs to this process are: IS\_INTRA, SUBBAND\_NUM, NUM\_SUBBANDS,
SPATIAL\_PARTITION, PARTITION\_INDEX, MULTI\_QUANT, SUBBAND\_WIDTH,
SUBBAND\_HEIGHT; previously decoded subband data arrays within the same
Transform Data element.

Outputs of this process are: A two-dimensional array SUBBAND[][] of
decoded subband coefficients.

 The Subband Coefficient Data unit is only present is ZERO\_SUBBAND\_FLAG is
FALSE. It is byte-aligned and occupies a whole number of bytes, padded
with zero bits as necessary. It consists of a byte-aligned block of raw
arithmetically-coded bytes of length SUBBAND\_LENGTH.

Subband decoding conventions are specified in Section  and the overall
subband decoding process is specified in Section .

\begin{verbatim}
transform_data():
    subband(0,LL)
    for level in range(1,transform_depth+1):
        for band in [LH, HL, HH]:
            subband(level,band)
\end{verbatim}

\begin{verbatim}
subband(level,band):
    subband_data_length[level][band] = read_uegol()
    if (subband_data_length[level][band] != 0):
        quantiser_index[level][band] = read_uegol()
        BYTEALIGN()
        compressed_subband_data[level][band] = read_chunk(subband_data_length[level][band])
\end{verbatim}



\subsection{Decoded subband data conventions}\label{wltdecodeconventions}

\subsubsection{Wavelet data initialisation}

\label{wltinit}

The $transform\_data()$ process begins with an initialisation process, 
$initialise\_data()$, which returns a structure which will
contain the decoded wavelet coefficients for the component. 

For the purposes of this specification, this is a four-dimensional array $data$,
where individual subbands are two-dimensional arrays accessed by level and depth:

$band = data[level][orientation]$

Valid levels are integers from in the range 0 to \TransformDepth inclusive. 
Level 0 consists of a single subband with orientation \LL. 
All other levels consist of 3 subbands of orientation \LH, \HL, 
and \HH.

Individual subband coefficients are signed integers accessed by vertical and 
horizontal coordinates within the subband:

$c = band[y][x], x\in\{0, ... , subband\_width(level)-1\}, 
y\in\{0, ... , subband\_height(level)-1\}$

where the dimensions $subband\_width(level)$ and $subband\_height(level)$ of the subband
are as defined in Section 
\ref{subbandwidthheight}. These dimensions correspond to a wavelet transform
being performed on a copy of the component data which has been padded (if necessary) so that its
dimensions are a multiple of $2^{\TransformDepth}$.


\subsubsection{Dimensions of wavelet subbands}
\label{subbandwidthheight}

This section defines the values of the $subband\_width(level)$ and $subband\_height(level)$
functions, giving the width and height of subbands at a given level, and hence the range
of subband vertical and horizontal indices. 

Define the padded dimensions of the component by

\begin{eqnarray*}
ph = 2^{\TransformDepth}*\lceil\frac{\ComponentHeight}{2^{\TransformDepth}}\rceil \\
pw = 2^{\TransformDepth}*\lceil\frac{\ComponentWidth}{2^{\TransformDepth}}\rceil
\end{eqnarray*}

If $level==0$,

\begin{eqnarray*}
subband\_height(level)=ph//2^{\TransformDepth} \\
=\lceil\frac{\ComponentHeight}{2^{\TransformDepth}}\rceil \\
subband\_width(level)=pw//2^{\TransformDepth} \\
=\lceil\frac{\ComponentWidth}{2^{\TransformDepth}}\rceil
\end{eqnarray*}

If $level>0$
\begin{eqnarray*}
subband\_height(level)=ph//2^{\TransformDepth-level+1} \\
=2^{level-1}*\lceil\frac{\ComponentHeight}{2^{\TransformDepth}}\rceil \\
subband\_width(level)=pw//2^{\TransformDepth-level+1} \\
=2^{level-1}*\lceil \frac{\ComponentWidth}{2^{\TransformDepth}}\rceil
\end{eqnarray*}

\begin{informative}
In encoding, these padded dimensions may be achieved by padding the 
component data up to the padded dimensions and applying the forward
Discrete Wavelet Transform (the inverse of the operations specified in
Section \ref{idwt}). Any values may be used for the padded data, although
the choice will affect wavelet coefficients at the right and bottom 
edges of the subbands. Good results, in compression terms, may be obtained
 by using edge extension for intra pictures and zero extension for inter 
pictures. A poor choice of padding may cause visible artefacts near the
bottom and right edges at high levels of compression.
\end{informative}

\subsection{Overall subband decoding process}\label{subbanddecodeprocess}

This section specifies the process $subband\_decode(level,orient)$ for coefficients
within a subband at level $level$ ($0$ to \TransformDepth) and of orientation $orient$
(\LL, \LH, \HL, or \HH). 

\subsubsection{Subband header and codeblock loop}

This section specifies the operation of the $subband\_decode(data, level, orient)$
function for decoding a subband at level $level$ and of orientation $orient$.

\paragraph{Initialisation\newline}

The $subband\_decode()$ process begins by reading a length code. If this length is
zero, then the subband is deemed to be skipped and all coefficients are set to zero
before exiting.


\begin{pseudo*}
\bsITEM{length}{uint}
\bsIF{length == 0}
  \bsFOR{y=0}{subband\_height(level)-1}
    \bsFOR{x=0}{subband\_width(level)-1}
    \bsCODE{band[y][x] = 0}
    \bsEND
  \bsEND
  \bsRET{}
\bsEND
\end{pseudo*}

If $length!=0$ then the subband coefficient decoding process is initialised by
setting up arithmetic decoding, initialising the coefficient count, reading
the quantisation index and byte-aligning the subsequent read operations.

\begin{pseudo*}
\bsCODE{initialise\_arithmetic\_decoding(length)}{\ref{initarith}}
\bsCODE{\CoefficientCount=0}
\bsCODE{quant\_index = read\_uint()}
\bsCODE{byte\_align()}
\end{pseudo*}

Note that byte alignment only occurs if $length$ is non-zero: a skipped 
subband is not byte-aligned.

\paragraph{Codeblocks\newline}

Data within a subband is divided into code blocks,
representing rectangular blocks of coefficients. The numbers of codeblocks
in each subband are determined in decoding the transform header, as specified
in Section \ref{transformheader} [NB: Tim's spec is wrong in that the limit on
the number of code blocks depends on the *chroma* dimensions ie so that the
chroma code blocks size is >=4. His call to $subband\_width/height$ doesn't first 
set the component dimensions to be the chroma dimensions. Could just make this a 
matter of compliancy rather than setting the numbers].

The overall subband decoding process loops over all the code blocks after initialising the
arithmetic decoding engine, and setting quantisers. There is a different code block decoding
process for Intra DC bands, since values are coded using spatial prediction.

\begin{pseudo*}
  \bsCODE{band = data[level][orient]}
  \bsIF{level>1}
  \bsCODE{parent = data[level-1][orient]}
  \bsELSE
    \bsCODE{parent = []}
  \bsEND
  \bsFOR{y=0}{\Codeblocks[level][vertical]-1}
    \bsFOR{x=0}{\Codeblocks[level][horizontal]-1}
      \bsCODE{codeblock(band,parent,orient,quant\_index,y,x)}{\ref{codeblocks}}
    \bsEND
  \bsEND

  \bsIF{IsIntra() \&\& level==0}
    \bsCODE{intra\_dc\_prediction(band,level)}{\ref{intradcprediction}}
  \bsEND

\bsEND

\end{pseudo*}


\subsubsection{Decoding subband codeblocks}

\label{codeblocks}

This section defines the operation of the 
$codeblock(band,parent,orient,quant\_index,y,x)$ function, which decodes a 
codeblock in position $(x,y)$ and populates it with reconstructed 
wavelet coefficients (figure ??).

[Include a figure here]

\paragraph{Codeblock dimensions\newline}

The codeblock covers coefficients in the horizontal range $left$ to $right-1$ and in the vertical
range $bottom$ to $top-1$ where these values are defined by:

\begin{eqnarray*}
  left & = & (subband\_width(level)*x)//\Codeblocks[level][horizontal] \\
  right & = & (subband\_width(level)*(x+1)//\Codeblocks[level][horizontal] \\
  bottom & = & (subband\_height(level)*y)//\Codeblocks[level][vertical] \\
  top & = & (subband\_height(level)*(y+1)//\Codeblocks[level][vertical]
\end{eqnarray*}

\paragraph{Codeblock decode process\newline}

The codeblock decoding process is defined as:

\begin{pseudo}{codeblock}{band, parent,orient,quant\_index,y,x}
\bsCODE{zero\_block\_flag=set\_zero\_flag()}{\ref{zeroblockflag}}
\bsIF{zero\_block\_flag == \true}
  \bsFOR{v=bottom}{top-1}
    \bsFOR{h=left}{right-1}
      \bsCODE{band[v][h] = 0}
    \bsEND
  \bsEND
\bsELSE
  \bsCODE{quant\_idx += quant\_offset()}{\ref{blockquantidx}}
  \bsFOR{v=bottom}{top-1}
    \bsFOR{h=left}{right-1}
      \bsCODE{coeff\_decode(band,parent,orient,quant_idx,v,h)}{\ref{wltcoeff}}
    \bsEND
  \bsEND

\bsEND

\end{pseudo}

\paragraph{Zero block flag\newline}
\label{zeroblockflag}

We may set the number of codeblocks in the subband as

$num\_blocks = \Codeblocks[level][horizontal]*\Codeblocks[level][vertical]$

If $num\_blocks$ is 1 or $level=0$, then $zero\_block\_flag()$ returns \false.

Otherwise, the flag is decoded from the stream: $read\_boola(\ZeroCodeblock)$
is returned.

\paragraph{Block quantiser offset\newline}
\label{blockquantidx}

If single quantisers per subband are used, the codeblock quantiser is set to the 
subband quantiser. Otherwise, the codeblock quantiser is encoded in the stream
differentially. Thus, if \CodeblockMode is set to SingleQuantiser then $quant\_offset()$
returns $0$.

If \CodeblockMode is set to MultipleQuantiser then the quantiser index offset
is decoded from the stream: $read\_sinta(quant\_contexts())$ is returned, where
$quant\_contexts()$ returns the context set:

\begin{itemize}
\item{Follow= \{\QOffsetFollow\}}
\item{Data=\QOffsetInfo}
\item{Sign=\QOffsetSign}
\end{itemize}

\subsubsection{Intra DC band prediction}
\label{intradcprediction}

This section defines the operation of the $intra\_dc\_prediction(band,level)$ function
for reconstructing values within Intra picture DC bands using spatial prediction.

\begin{pseudo}{intra\_dc\_prediction}{band,level}

\bsCODE{prediction = 0 }
\bsFOR{v=0}{subband\_height(level)-1}
  \bsFOR{h=0}{subband\_width(level)-1}
    \bsCODE{prediction = 0 }
    \bsIF{v>0)}
      \bsCODE{prediction += band[v-1][h]}
      \bsIF{h>0)}
        \bsCODE{prediction += band[v-1][h-1]+band[v][h-1]}
      \bsEND
    \bsELSE
      \bsIF{h>0)}
        \bsCODE{prediction += band[0][h-1]}
      \bsEND
    \bsEND
    \bsCODE{band[v][h] += prediction}
  \bsEND
\bsEND

\end{pseudo}

\subsection{Number and dimension of code blocks}This section specifies the number and dimensions of code blocks used in
decoding subband data.

Inputs to this process are: PARTITION\_INDEX, IS\_INTRA, SUBBAND\_NUM,
MAX\_XBLOCKS, MAX\_YBLOCKS

Outputs to this process are: XNUM\_CODEBLOCKS, YNUM\_CODEBLOCKS

If SPATIAL\_PARTITION is FALSE, then

XNUM\_CODEBLOCKS=1
YNUM\_CODEBLOCKS=1

If SPATIAL\_PARTITION\_FLAG is TRUE then XNUM\_CODEBLOCKS and
YNUM\_CODEBLOCKS are derived from the values of PARTITION\_INDEX and
SUBBAND\_NUM as follows.

Default

If PARTITION\_INDEX is 1 then the default spatial partition is used. The
minimum dimension of a code block is set by

MIN\_BLOCK\_DIM=4

The partition is derived from the subband number and whether the frame
is intra or not. Initial values of XNUM\_CODEBLOCKS and YNUM\_CODEBLOCKS
are derived from .




IS\_INTRA



TRUE

FALSE

SUBBAND\_NUM

<NUM\_SUBBANDS-6

XNUM\_CODEBLOCKS=4

YNUM\_CODEBLOCKS=3

XNUM\_CODEBLOCKS=12

YNUM\_CODEBLOCKS=8



≥NUM\_SUBBANDS-6

<NUM\_SUBBANDS-3

XNUM\_CODEBLOCKS=1

YNUM\_CODEBLOCKS=1

XNUM\_CODEBLOCKS=8

YNUM\_CODEBLOCKS=6



≥NUM\_SUBBANDS-3

XNUM\_CODEBLOCKS=1

YNUM\_CODEBLOCKS=1

XNUM\_CODEBLOCKS=1

YNUM\_CODEBLOCKS=1


Table   Matrix of code block numbers

XNUM\_CODEBLOCKS and YNUM\_CODEBLOCKS are then adjusted to ensure that
minimum codeblock sizes are respected:

XNUM\_CODEBLOCKS=min(XNUM\_CODEBLOCKS , SUBBAND\_WIDTH//MIN\_BLOCK\_DIM)

YNUM\_CODEBLOCKS=min(YNUM\_CODEBLOCKS , SUBBAND\_HEIGHT//MIN\_BLOCK\_DIM)

Custom

If PREDICTION\_INDEX is 0 then the number of codeblocks is configurable
on a frame basis by setting MAX\_XBLOCKS and MAX\_YBLOCKS.
XNUM\_CODEBLOCKS. The minimum dimension of a code block is set by

MIN\_BLOCK\_DIM=4

and

XNUM\_CODEBLOCKS=min(MAX\_XBLOCKS , SUBBAND\_WIDTH//MIN\_BLOCK\_DIM)

YNUM\_CODEBLOCKS=min(MAX\_YBLOCKS , SUBBAND\_HEIGHT//MIN\_BLOCK\_DIM)



\subsection{Code block decoding process}This section specifies the operation of the decode\_code\_block(u,v)
process for a code block in position (u,v).

Code block parameters

The code block in position (u,v) consists of coefficients in positions
(i,j) such that

XSTART<=i<XSTOP
YSTART<=j<YSTOP

where

XSTART=(u*SUBBAND\_WIDTH)//XNUM\_CODEBLOCKS
XSTOP=((u+1)*SUBBAND\_WIDTH)//XNUM\_CODEBLOCKS
YSTART=(v*SUBBAND\_HEIGHT)//YNUM\_CODEBLOCKS
YSTOP=((v+1)*SUBBAND\_HEIGHT)//YNUM\_CODEBLOCKS

Skip flag

If the number of code blocks in the subband is greater than 1 i.e.
XNUM\_CODEBLOCKS>1 or YNUM\_CODEBLOCKS>1, a skip flag is decoded by the
recipe

SKIP[j][i]=binary\_arithdecode()

If XNUM\_CODEBLOCKS=1 and YNUM\_CODEBLOCKS=1 then SKIP[j][i] is set to
FALSE.

If SKIP[j][i] is TRUE, then all coefficients within the code block are
set to zero and code block decoding terminates.

Quantisation parameters

If SKIP[j][i] is FALSE and if MULTI\_QUANT is TRUE, then a quantisation
index is decoded by the recipe

QF[j][i]=su\_arith\_decode()+QUANT\_INDEX

If there is only one code block or MULTI\_QUANT is FALSE then

QF[j][i]=QUANT\_INDEX

The quantisation factor QUANT\_FACTOR[j][i] and offset value OFFSET[j][i]
are derived as follows:

QUANT\_FACTOR[j][i]= [NB this differs from software]

OFFSET[j][i]=round( QUANT\_FACTOR[j][i] * 0.375 )

A table of QUANT\_FACTOR and OFFSET values is given in  in Appendix .

Coefficient data

Coefficients within a code block are decoded in raster order according
to the specification of Section .

If SKIP[j][i] is FALSE then the remaining code block data is decoded by
the following process

for (y=YSTART ; y<YSTOP ; ++y)

{

    for (x=XSTART ; x<XSTOP ; ++x)

        decode\_coeff(x,y)

}



\subsection{Subband coefficient decoding process}\label{wltcoeff}

This section describes the operation of the 
$coeff_decode(band,parent,orient,quant\_idx,v,h)$ process
for decoding a coefficient in position $(h,v)$ in the subband $band$ with
parent band $parent$ and orientation $orient$.

Decoding a coefficient makes use of arithmetic decoding, inverse quantisation
and, in the case of DC (level 0) bands of Intra frames, neighbourhood prediction.
The decoding process periodically refreshes the contexts, by
halving the context counts every time a count is reached.

Arithmetic coding uses a highly compact set of contexts, 
with magnitudes contextualised on whether parent values
and neighbouring values are zero or non-zero.

\subsubsection{Overall coefficient decoding process}

Two different processes are used for decoding coefficients, depending
upon whether spatial prediction is required or not.

\begin{pseudo}{coeff\_decode}{band,parent,orient,quant\_index,v,h}
    \bsCODE{parent = parent\_val(parent, v, h)}{\ref{parentval}}
    \bsCODE{nhood = zero_nhood(band,v,h)}{\ref{zeronhood}}
    \bsCODE{sign\_pred = sign\_predict(band,orient,v,h)}{\ref{signpredict}}
    \bsCODE{context\_set = select\_coeff\_ctxs(parent, nhood, sign\_pred)}{\ref{selectcoeffcontext}}
    \bsCODE{quant\_coeff = read\_sina( context\_set )}{}
    \bsCODE{band[v][h] = inverse\_quant( quant\_coeff, quant\_index )}{\ref{invquant}}
    \bsCODE{update\_count()}{\ref{updatecounts}}
\end{pseudo}


\subsubsection{Parent values}
\label{parentval}
The function $parent\_val()$ returns the parent value of a coefficient in a subband,
which is the co-located coefficient in the parent subband. If there is no parent,
$0$ is returned:

\begin{pseudo}{parent\_val}{parent,v,h}
  \bsIF{parent == []}
    \bsRET{0}
  \bsELSE
    \bsRET{parent[v//2][h//2]}
  \bsEND
\end{pseudo}

\subsubsection{Zero neighbourhood}
\label{zeronhood}

The $zero\_nhood()$ function returns a boolean indicating whether neighbouring
values are all zero.

\begin{pseudo}{zero\_nhood}{band,v,h}
\bsIF{v>0}
  \bsIF{band[v-1][h]!=0}
    \bsRET{\false}
  \bsEND
  \bsIF{h>0}
    \bsIF{band[v-1][h-1])!=0 || band[v][h-1]!=0}
      \bsRET{\false}
    \bsEND
  \bsEND
\bsELSE
  \bsIF{h>0}
    \bsIF{ band[v][h-1] !=0}
      \bsRET{\false}
    \bsEND
  \bsEND
\bsEND
\bsRET{\true}
\end{pseudo}

\subsubsection{Sign prediction}
\label{signpredict}

The $sign\_predict()$ function returns a prediction for the sign of the 
current pixel. Correlation within subbands depends upon orientation,
and so this is taken into account in forming the prediction.

For vertically-oriented (HL) bands, the predictor is the sign of the
coefficient above the current coefficient; for horizontally-oriented (LH)
bands, the predictor is the sign of the coefficient to the left. 

The predictions are not used for differential encoding of the sign, but for
conditioning of the sign contexts only.

\begin{pseudo}{sign\_predict}{band,orient,v,h}
\bsIF{orient==HL}
  \bsIF{v==0}
    \bsRET{0}
  \bsELSE
    \bsRET{sign(band[v-1][h])}
  \bsEND
\bsELSEIF{orient==LH}
  \bsIF{h==0}
    \bsRET{0}
  \bsELSE
    \bsRET{sign(band[v][h-1])}
  \bsEND
\bsELSE
  \bsRET{0}
\bsEND{}
\end{pseudo}

\subsubsection{Coefficient context selection}
\label{selectcoeffcontext}

This section defines the $select\_context(zero\_nhood, parent, sign)$
function, which chooses a context index set for decoding a coefficient value.

Twelve possible coefficient index sets are defined, and are returned as specified 
in Table \ref{contexttable}. Note that follow contexts are an array indexed from $0$
as per Section \ref{arithreadint}.

Note that parent values affect the context of all follow bits, and that neighbour
values only affect the context of the first follow bit. A common data context is used
for all coefficients.

%% Table of context sets for signed coefficient extraction %%
\begin{table}[h!]
\begin{tabular}{|c|c|c||l|l|}
\hline
 $parent$ & $zero\_nhood$ & $sign$ & \multicolumn{2}{c|}{\bf{Context set}} \\

% Zero parent, zero neighbour, zero sign prediction
\hline
0 & \true & 0 &  follow[] & \ZPZNFollowOne,\ZPFollowTwo,\ZPFollowThree,
                            \ZPFollowFour,\ZPFollowFive,\ZPFollowSixPlus \\ \cline{4-5}
  &   &   &  data & \CoeffData \\ \cline{4-5}
  &   &   &  sign & \SignZero \\

% Zero parent, zero neighbour, -ve sign prediction
\hline
0 & \true & $<0$ &  follow[] & \ZPZNFollowOne,\ZPFollowTwo,\ZPFollowThree,
                               \ZPFollowFour,\ZPFollowFive,\ZPFollowSixPlus \\ \cline{4-5}
  &   &    &  data & \CoeffData \\ \cline{4-5}
  &   &    &  sign & \SignNeg \\

% Zero parent, zero neighbour, +ve sign  prediction
\hline
0 & \true & $>0$ &  follow[] & \ZPZNFollowOne,\ZPFollowTwo,\ZPFollowThree,
                               \ZPFollowFour,\ZPFollowFive,\ZPFollowSixPlus \\ \cline{4-5}
  &   &    &  data & \CoeffData \\ \cline{4-5}
  &   &    &  sign & \SignPos \\

% Zero parent, non-zero neighbour, zero sign prediction
\hline
0 & \false & 0 &  follow[] & \ZPNNFollowOne,\ZPFollowTwo,\ZPFollowThree,
                             \ZPFollowFour,\ZPFollowFive,\ZPFollowSixPlus \\ \cline{4-5}
  &   &   &  data & \CoeffData \\ \cline{4-5}
  &   &   &  sign & \SignZero \\

% Zero parent, non-zero neighbour, -ve sign prediction
\hline
0 & \false & $<0$ &  follow[] & \ZPNNFollowOne,\ZPFollowTwo,\ZPFollowThree,
                                \ZPFollowFour,\ZPFollowFive,\ZPFollowSixPlus \\ \cline{4-5}
  &        &    &  data & \CoeffData \\ \cline{4-5}
  &        &    &  sign & \SignNeg \\

% Zero parent, non-zero neighbour, +ve sign prediction
\hline
0 & \false & $>0$ &  follow[] & \ZPNNFollowOne,\ZPFollowTwo,\ZPFollowThree,
                                \ZPFollowFour,\ZPFollowFive,\ZPFollowSixPlus \\ \cline{4-5}
  &        &      &  data & \CoeffData \\ \cline{4-5}
  &        &      &  sign & \SignPos \\

% Non-zero parent, zero neighbour, zero sign prediction
\hline
$\neq 0$ &  \true & 0 &  follow[] & \NPZNFollowOne,\NPFollowTwo,\NPFollowThree,
                                    \NPFollowFour,\NPFollowFive,\NPFollowSixPlus \\ \cline{4-5}
& &      &  data & \CoeffData \\ \cline{4-5}
& &      &  sign & \SignZero \\

% Non-zero parent, zero neighbour, -ve sign prediction
\hline
$\neq 0$ & \true & $<0$ &  follow[] & \NPZNFollowOne,\NPFollowTwo,\NPFollowThree,
                                      \NPFollowFour,\NPFollowFive,\NPFollowSixPlus \\ \cline{4-5}
& &      &  data & \CoeffData \\ \cline{4-5}
& &      &  sign & \SignNeg \\

% Non-zero parent, zero neighbour, +ve sign prediction
\hline
$\neq 0$ & \true & $>0$ &  follow[] & \NPZNFollowOne,\NPFollowTwo,\NPFollowThree,
                                      \NPFollowFour,\NPFollowFive,\NPFollowSixPlus \\ \cline{4-5}
& &      &  data & \CoeffData \\ \cline{4-5}
& &      &  sign & \SignPos \\

% Non-zero parent, non-zero neighbour, zero sign prediction
\hline
$\neq 0$ & \false & 0 &  follow[] & \NPNNFollowOne,\NPFollowTwo,\NPFollowThree,
                                    \NPFollowFour,\NPFollowFive,\NPFollowSixPlus \\ \cline{4-5}
& &      &  data & \CoeffData \\ \cline{4-5}
& &      &  sign & \SignZero \\

% Zero parent, non-zero neighbour, -ve sign prediction
\hline
$\neq 0$ & \false & $<0$ &  follow[] & \NPNNFollowOne,\NPFollowTwo,\NPFollowThree,
                                       \NPFollowFour,\NPFollowFive,\NPFollowSixPlus \\ \cline{4-5}
& &      &  data & \CoeffData \\ \cline{4-5}
& &      &  sign & \SignNeg \\

% Zero parent, non-zero neighbour, +ve sign prediction
\hline
$\neq 0$ & \false  & $>0$ &  follow[] & \NPNNFollowOne,\NPFollowTwo,\NPFollowThree,
                                        \NPFollowFour,\NPFollowFive,\NPFollowSixPlus \\ \cline{4-5}
& &      &  data & \CoeffData \\ \cline{4-5}
& &      &  sign & \SignPos \\
\hline

\end{tabular}
\caption{Subband coefficient context sets}\label{contexttable}
\end{table}


\subsubsection{Inverse quantisation}
\label{invquant}

The inverse\_quant() function is defined by:

\begin{pseudo}{inverse\_quantise}{quantised\_coeff, quant\_index}
\bsCODE{magnitude = |{quantised\_coeff}|}
\bsCODE{magnitude *= quant\_factor(quant\_index)}{\ref{quantfacs}}
\bsCODE{magnitude += quant\_offset(quant\_index)}{\ref{quantfacs}}
\bsCODE{magnitude = magnitude//4}
\bsRET{sign( quantised\_coeff )*magnitude}
\end{pseudo}

\begin{informative}
The pseudocode description separates inverse quantisation from decoding. However, 
since dead-zone quantisation is used, the $inverse\_quant()$ function must compute
the magnitude. Hence it is more efficient to first decode the coefficient magnitude,
then inverse quantise, and then decode the coefficient sign. 
\end{informative}


\subsubsection{Quantisation factors and offsets}
\label{quantfacs}

This section specifies the operation of the $quant\_factor()$ and 
$quant\_offset()$ functions for performing inversion quantisation.

Quantisation factors represent an approximation of quarter-bit values 
with two bits of accuracy (i.e. $round(2^{\frac{index}{4}+2})$:

\begin{pseudo}{quant\_factor}{index}
\bsCODE{base = 2**(index//4)}
\bsIF{ (x\%4)==0 }
  \bsRET{4*base}
\bsELSEIF{ (x\%4)==1 }
  \bsRET{78892*base+8292)//16585}
\bsELSEIF{ (x\%4)==2 }
  \bsRET{228486*base+20195)//40391}
\bsELSEIF{ (x\%4)==3 }
  \bsRET{440253*base+32722)//65444}
\bsEND
\end{pseudo}

Offsets are approximately $3/8$ of the quantisation factors - these
mark the reconstruction point within the quantisation interval:

\begin{pseudo}{quant\_offset}{index}
\bsRET{ (quant\_factor(index)*3+4)//8 }
\end{pseudo}

\subsubsection{Updating counts and resetting contexts}
\label{updatecounts}

The $update\_count()$ function updates a periodic count of subband 
coefficients and rescales arithmetic decoding contexts if \CoefficientReset has been reached.

\begin{pseudo}{update\_count}{}
\bsCODE{\CoefficientCount += 1}
\bsIF{\CoefficientCount == \CoefficientReset}
  \bsCODE{\CoefficientCount = 0}
  \bsFOR{i=0}{len(\AContexts)-1}
    \bsCODE{reset\_context(i)}{\ref{arithcontexts}}
  \bsEND
\bsEND
\end{pseudo}

\subsection{Magnitude context modelling}This section specifies the procedure for choosing contexts for decoding
the magnitude of a coefficient in position (x,y) in the subband.

Neighbourhood sum

A value NHOOD\_SUM is defined as the sum of the absolute values of
previously decoded and reconstructed coefficients:



A value NTOP is defined by the recipe

NTOP=(SCALE\_FACTOR>>1)*QF[j][i]

Parent flag

A Boolean flag PARENT\_ZERO is set to TRUE if the subband has a parent
subband and the parent coefficient at position (x>>1,y>>1) in that
subband is zero. 

PARENT\_ZERO is set to FALSE if the subband has a parent subband and the
parent coefficient at position (x>>1,y>>1) in that subband is not zero.

If the subband does not have a parent subband then if PARENT\_ZERO is FALSE for all
coefficients in the subband. 

Context selection

Magnitude contexts are selected by bin and, PARENT\_ZERO value and
NHOOD\_SUM value, according to .


Subband coefficient magnitude contexts

PARENT\_ZERO=TRUE

PARENT\_ZERO=FALSE

Bin       NHOOD\_SUM     Context

  1                  0                         Z\_BIN1z\_CTX

  1                 >0                        Z\_BIN1nz\_CTX

  2                 -                           Z\_BIN2\_CTX

  3                 -                           Z\_BIN3\_CTX

  4                 -                           Z\_BIN4\_CTX

  ≥5               -                           Z\_BIN5+\_CTX

Bin       NHOOD\_SUM     Context

  1                  0                         NZ\_BIN1z\_CTX

  1                 >0,≤NTOP          NZ\_BIN1a\_CTX

  1                 >NTOP               NZ\_BIN1b\_CTX

  2                 -                          NZ\_BIN2\_CTX

  3                 -                          NZ\_BIN3\_CTX

  4                 -                          NZ\_BIN4\_CTX

  ≥5               -                          NZ\_BIN5+\_CTX 


Table   Contexts for coefficient magnitude decoding



\subsection{Sign context modelling}This section specifies the procedure for choosing contexts for decoding
the sign of a coefficient in position (x,y) in the subband. 

Sign contexts are selected based on the sign of previously decoded
coefficients. The choice of the coefficients used for the decision
depends on the orientation of the subband.

A value PREVIOUS\_VALUE is defined to be

    SUBBAND[y][x-1] if x>0 and the subband is horizontally oriented

    SUBBAND[y-1][x] if y>0 and the subband is vertically oriented

    0 otherwise

The sign context used is specified by .


Subband coefficient sign contexts

PREVIOUS\_VALUE

Context

>0

<0

0

SIGN\_POS\_CTX

SIGN\_NEG\_CTX

SIGN\_ZERO\_CTX


Table   Contexts for coefficient signs




\subsection{Skip parameter context modelling}The skip parameter statistics are modelled by a single context
BLOCK\_SKIP\_CTX.



\subsection{Quantisation index context modelling}The quantisation index magnitude statistics are modelled by a single
context QUANT\_MAG\_CTX for all bins. The quantisation index sign
statistics are modelled by a single context QUANT\_SIGN\_CTX.


\subsection{Context initialisation}Prior to decoding the subband, contexts are initialised according to .


Context

COUNT0

COUNT1

BLOCK\_SKIP\_CTX

1

1

QUANT\_MAG\_CTX

1

1

QUANT\_SIGN\_CTX

1

1

SIGN\_POS\_CTX

1

1

SIGN\_NEG\_CTX

1

1

SIGN\_ZERO\_CTX

1

1

Z\_BIN1z\_CTX

1

1

Z\_BIN1nz\_CTX

1

1

Z\_BIN2\_CTX

1

1

Z\_BIN3\_CTX

1

1

Z\_BIN4\_CTX

1

1

Z\_BIN5+\_CTX

1

1

NZ\_BIN1z\_CTX

1

1

NZ\_BIN1a\_CTX

1

1

NZ\_BIN1b\_CTX

1

1

NZ\_BIN2\_CTX

1

1

NZ\_BIN3\_CTX

1

1

NZ\_BIN4\_CTX

1

1

NZ\_BIN5+\_CTX

1

1




Table   Subband decoding context initialisation [NB: these context
initialisations are subject to change]



\subsection{Intra DC subband prediction}This section specifies the procedure intra\_dc\_prediction() used for
predicting coefficients in the DC band of Intra frames.

The prediction value to be returned is denoted PRED\_VALUE and is derived
based on previously decoded and reconstructed coefficients, by the
recipe:




\clearpage
\section{preparing decoded data for mc}
\subsection{Block Motion Data decoding}This section specifies the process for decoding the Block Motion data
unit. This process is invoked for Inter frames when decoding the Frame
Prediction data unit. The Block Motion data unit is byte aligned and
occupies a whole number of bytes, padded with zero bits as necessary.
Its size in bytes must be equal to BLOCK\_DATA\_LENGTH.

Inputs to this process are: global motion field GLOBAL[][][];
GLOBAL\_MOTION\_FLAG, X\_NUM\_MB, Y\_NUM\_MB, X\_NUM\_BLOCKS, Y\_NUM\_BLOCKS,
NUM\_REFS, 

Outputs from this process are: a value of MB\_SPLIT and MB\_COMMON for
each MB; a value of PRED\_MODE for each prediction unit in each MB; a
motion vector MV1 for each prediction unit with PRED\_MODE equal to
REF1ONLY or REF1AND2; a motion vector MV2 for each prediction unit with
PRED\_MODE equal to REF2ONLY or REF1AND2.

The Block Motion data unit is a single block of arithmetically coded
binary data. This section specifies the decoding operations to be used
in conjunction with the arithmetic decoding engine specified in Section
with the contexts and initialisation defined in Section .

Block motion data is used in predicting Inter frames using either global
or block motion or both. When block motion is used it encodes the motion
vectors to be used. When two reference frames are used it encodes motion
vectors for both references. With two references it also encodes which
reference, or both references, are to be used and which blocks are to be
coded Intra (i.e. without using motion compensated prediction). The
Block Motion data also encodes any other information that is needed to
perform motion compensated prediction, such as macroblock  splitting.

Motion vector data is organised into macroblock s, which are 4x4 arrays
of blocks. The motion data is decoded by decoding the data in each MB,
scanning in raster order from the top-left corner. 

Numbering

For the purposes of this specification, macroblocks  are numbered by x-
and y- coordinates in raster order, from the top-left MB. Blocks are
also numbered from by x- and y-coordinates in raster order from the
top-left block. The indices of blocks within a macroblock with
coordinates (x,y) therefore run from (4x,4y) (top-left) through to
(4x+3,4y+3) (bottom right). 

Overall decoding process

The decoding process iterates across all macroblocks, first decoding the
macroblock data and then decoding the prediction unit data pertaining to
each macroblock. The macroblock data constrains how much block data
there is and how it is interpreted. The decoder maintains a value
MB\_COUNT which is used to reset statistics periodically.

In pseudocode the decoding process is:

MB\_COUNT=0

for (y=0;  y<Y\_NUM\_MB;  y++)

{

    for (x=0;  x<X\_NUM\_MB;  x++)

    {

          decode\_MB\_data (x,y)

          decode\_MB\_block\_data(x,y)

          MB\_COUNT++

          if (MB\_COUNT>32)

              halve\_all\_counts()

    }

}

The arithmetic decoding engine function halve\_all\_counts() is specified
in Section .



\subsection{Motion vector data prediction apertures}This section specifies the aperture used for predicting motion vector
data, namely those previously decoded motion vector data elements (if
any) which may be used for prediction.

All motion vector data is differentially decoded. The differential
element is performed by forming a prediction from previously decoded
values. Various methods of prediction are used for predicting different
elements of motion vector data for differential, however in each case
the data elements used to form the prediction are defined in the same
way.  Data elements which may be available for prediction are termed the
nominal prediction aperture. 

Not all values in the nominal prediction aperture may exist, and so will
not be used to form any prediction. 

MB data prediction aperture

The nominal prediction aperture for macroblock data is defined to be the
existing macroblocks  to the top, left and above the current MB. For
macroblocks  on the left-hand side of the frame, but not the top-left
MB, the aperture consists of only the macroblock above for macroblocks
on the top of the frame, but not the top-left macroblock the aperture
consists solely of the macroblock to the left for the top-left
macroblock the aperture is empty. The nominal aperture for macroblock
data is illustrated in  below.




Figure  Macroblock prediction aperture

Block motion data prediction aperture

The nominal prediction aperture for block motion data is defined to be:
the prediction units containing the blocks to the left, top and top-left
of the top-left block in the current prediction unit. An example
aperture for different macroblock splittings is shown in . Prediction
units at the top and left of the frame have prediction aperture
restricted to those prediction units in the nominal aperture that lie
within the frame. 




Figure  Prediction aperture for prediction units within a MB.  

Where a prediction unit in the prediction aperture has a value of the
same type as that being decoded, it forms part of the prediction.
However, not all prediction units in the prediction aperture may have
such a value. For example if Reference 2 motion vectors are being
decoded but the top-left prediction unit only has Reference 1 motion
vectors, then it will not provide values in the prediction.  



\subsection{MB data}This section specifies the decode\_MB\_data(x,y) process.

Inputs to this process are: none. 

Outputs of this process are: a value of MB\_USING\_GLOBAL, MB\_SPLIT and
MB\_COMMON for the macroblock at position (x,y).

MB data is only present if GLOBAL\_ONLY\_FLAG is FALSE.

The structure of the macroblock data is shown in .


Name

Type

Signed

Size (bits)

Value restrictions

if (!GLOBAL\_ONLY\_FLAG)

{

    if (GLOBAL\_MOTION\_FLAG)

    {

        MB\_USING\_GLOBAL

Bool

-

1

-

    }

    if (!MB\_USING\_GLOBAL)

    {

        MB\_SPLIT

Integer

No

2

0,1,2
    }

    if ( MB\_SPLIT!=0 )
    {
        MB\_COMMON

Bool

-

1

-

    }

}


Figure      Macroblock Structure

The decoding process for a macroblock at coordinates (x,y) is as
follows, in pseudocode

if (!GLOBAL\_ONLY\_FLAG)

{

    MB\_SPLIT=2

    MB\_COMMON=TRUE

}

else

{

     if (GLOBAL\_MOTION\_FLAG)

         decode\_mb\_using\_global(x,y)

     else

         MB\_USING\_GLOBAL=FALSE



     if (!MB\_USING\_GLOBAL)

         decode\_mb\_split(x,y)

     else

         MB\_SPLIT=2



     if (MB\_SPLIT!=0)

         decode\_mb\_common(x,y)

     else

         MB\_COMMON=TRUE

}

The subsidiary decoding functions decode\_mb\_using\_global(),
decode\_mb\_split(), decode\_mb\_common() are specified in subsequent
sections. 

The MB\_SPLIT parameter determines to what level the macroblock shall be
split i.e. how many Prediction Units it contains and what size they are.
If MB\_SPLIT=2, then the macroblock is split into a 4x4 array of
macroblock s. If MB\_SPLIT=1, then the macroblock is split into a 4x4
array of sub-macroblocks. If MB\_SPLIT=0, then the macroblock is not
split. For each prediction unit in the MB, motion vectors and possibly
prediction modes will be extracted from the bitstream, so MB\_SPLIT
determines how many values must be extracted for the MB.

The MB\_COMMON parameter determines whether a common prediction mode is
to be decoded for the whole macroblock contained within the bitstream,
or whether different prediction modes are to be decoded for each
prediction unit. If MB\_SPLIT=0, MB\_COMMON is not necessary, since in
this case there is only one prediction unit and the two cases are
identical. 



\subsection{MB\_USING\_GLOBAL decoding}This section specifies the function decode\_mb\_using\_global(x,y) for a
macroblock at coordinates (x,y).

MB\_USING\_GLOBAL is decoded differentially with respect to a prediction.
The reconstruction procedure is 

MB\_USING\_GLOBAL(x,y)=binary\_arith\_decode()+mb\_using\_global\_prediction(x,y)
( mod 2 )

A single context is used: MB\_USING\_GLOBAL\_CTX.

The prediction function mb\_using\_global\_prediction(x,y) is defined as
the mean of available previously-decoded predictors in the prediction
arrangement defined in Section:

if (x>0 \&\& y>0 )

    mb\_using\_global\_prediction(x,y)=( MB\_USING\_GLOBAL (x-1,y)+
                                                              MB\_USING\_GLOBAL (x-1,y-1)+
                                                              MB\_USING\_GLOBAL (x,y-1) )/3

It returns a value 0 or 1.


\subsection{MB\_SPLIT decoding}This section specifies the function decode\_mb\_split(x,y) for a
macroblock at coordinates (x,y).

MB\_SPLIT is decoded differentially with respect to a prediction. The
MB\_SPLIT prediction residue is encoded in the bitstream using truncated
unary binarisation, since the values are constrained to lie in the range
0-2. The reconstruction procedure is therefore

MB\_SPLIT(x,y)=ut\_arith\_decode()+mb\_split\_prediction(x,y) ( mod 3 )

Two contexts are used for bin 1 and bin 2: MB\_SPLIT\_CTX\_BIN1 and
MB\_SPLIT\_CTX\_BIN2.

The prediction function mb\_split\_prediction(x,y) is defined as the mean
of available previously-decoded MB\_SPLIT values in the neighbouring
predictor macroblocks, as per Section :

if (x>0 \&\& y>0 )

    mb\_split\_prediction(x,y)=( MB\_SPLIT(x-1,y)+ MB\_SPLIT(x-1,y-1)+
MB\_SPLIT(x,y-1) )/3

 It returns a value 0,1 or 2.



\subsection{MB\_COMMON decoding}This section specifies the function decode\_mb\_split(x,y) for a
macroblock at coordinates (x,y).

MB\_COMMON is decoded differentially with respect to a prediction. The
reconstruction procedure is 

MB\_COMMON(x,y)=binary\_arith\_decode()+mb\_split\_prediction(x,y) ( mod 2 )

The MB\_COMMON prediction residue is encoded in the bitstream as a single
binary bit, with context MB\_COMMON\_CTX.

The prediction function mb\_common\_prediction(x,y) is defined as the mean
of available previously-decoded predictors in the prediction arrangement
defined in Section :

if (x>0 \&\& y>0 )

    mb\_common\_prediction(x,y)=( MB\_COMMON (x-1,y)+MB\_COMMON(x-1,y-1)+
                                                        MB\_COMMON(x,y-1)
)/3

It returns a value 0 or 1.



\subsection{Prediction modes}Four prediction modes shall be supported by the decoder:

INTRA

REF1ONLY

REF2ONLY

REF1AND2.

These correspond to using DC prediction, using a motion compensated
prediction from Reference 1 only, using a motion compensated prediction
from Reference 2 and using a motion compensated prediction composed from
data from both Reference 1 and Reference 2.

The prediction modes shall be identified with two-bit words as follows: 

INTRA=b00

REF1ONLY=b01

REF2ONLY=b10

REF1AND2=b11

In this way, Reference 1 is used for prediction if and only if
(PRED\_MODE\&b01) and  Reference 2 is used for prediction if and only if
(PRED\_MODE\&b10).  



\subsection{Block motion data decoding}This section specifies the operation of decode\_MB\_block\_data(x,y). This
process is invoked in the decoding of each MB.

Block motion data consists of decoding the prediction mode and motion
vectors for all the prediction units in the MB. The process depends upon
the values of MB\_SPLIT and MB\_COMMON. The decoding process is as
follows: scan the prediction units within a macroblock in raster order
and for each prediction unit:

1. If MB\_COMMON is TRUE, decode a mode PRED\_MODE for the top-left
prediction unit in the macroblock as per Section . This mode shall also
be used for all prediction units in the MB. 

2. Loop over all the prediction units in the macroblock in raster order,
and for each prediction unit:

a) If MB\_COMMON is FALSE decode a mode to be used for the prediction
unit as per Section  and set as PRED\_MODE

b) If  PRED\_MODE is REF1ONLY or REF1AND2 decode a motion vector MV1 as
per Section 

c) If  PRED\_MODE is REF2ONLY or REF1AND2 decode a motion vector MV2 as
per Section 

d) If PRED\_MODE is INTRA, decode DC values PU\_DC\_Y, PU\_DC\_C1 and
PU\_DC\_C2 as per Section 



\begin{informative}
The identification of the PRED\_MODE value for the macroblock with that
for the top-left prediction unit is so that the prediction aperture
defined in Section  can be consistently applied.
\end{informative}


\subsection{Block prediction mode decoding}PRED\_MODE consists of two bits of information (for reference 1 and
reference 2) and is decoded differentially with respect to predictions
applying to each bit.

Two contexts are used: PRED\_MODE\_BIT1\_CTX and PRED\_MODE\_BIT2\_CTX. 

The decoding process is:

PRED\_MODE=binary\_arith\_decode(PRED\_MODE\_BIT1\_CTX)

PRED\_MODE|= (binary\_arith\_decode(PRED\_MODE\_BIT2\_CTX)<<1)

PRED\_MODE\^= block\_mode\_pred()

PRED\_MODE prediction

PRED\_MODE is predicted by predicting the first and second bit
independently as the mode of the corresponding bits of prediction modes
for prediction units in the prediction aperture.

I.e. for prediction units not at the top or left side of the frame:

block\_mode\_pred()\&1=mode( PRED\_MODEL\&1, PRED\_MODET\&1, PRED\_MODETL\&1 )

block\_mode\_pred()\&2=mode( PRED\_MODEL\&2, PRED\_MODET\&2, PRED\_MODETL\&2 )

For prediction units on the left side of the frame other than the
top-left prediction unit in the frame:

block\_mode\_pred()=PRED\_MODET

For prediction units on the top side of the frame other than the
top-left prediction unit in the frame:

block\_mode\_pred()=PRED\_MODEL

For the top-left prediction unit:

block\_mode\_pred()=INTRA[NB: this is inconsistent with current software]


\subsection{Block motion vector decoding}This section specifies the decoding process for motion vectors in a
prediction unit whose top-left block has coordinates (x,y).

If MB\_USING\_GLOBAL is TRUE then: if PRED\_MODE\&1 is TRUE, MV1 is equal to
GLOBAL[0][y][x]; if PRED\_MODE\&2 is TRUE, MV2 is GLOBAL[1][y][x].

If MB\_USING\_GLOBAL is FALSE then block motion vectors are differentially
decoded with respect to a prediction.  The decoding process is as
follows.

Motion vector prediction residues are decoded horizontal component
first, followed by the vertical component. The values use signed unary
binarisation. The decoding process for each motion vector is:

1.      MV\_x=mv\_pred()+su\_arith\_decode()

2.      MV\_y=mv\_pred()+su\_arith\_decode()

Different contexts are used for Reference 1 and Reference 2 vectors, for
horizontal and vertical components, for different bins and for sign and
magnitude data, as per . 


MV contexts

Reference 1

Horizontal

Vertical

Magnitude Contexts

Sign Context

Magnitude

Sign Context

Bin   Context

1        REF1x\_BIN1\_CTX

2        REF1x\_BIN2\_CTX

3        REF1x\_BIN3\_CTX

4        REF1x\_BIN4\_CTX

≥5      REF1x\_BIN5+\_CTX

REF1x\_SIGN\_CTX          



Bin   Context

1        REF1y\_BIN1\_CTX

2        REF1y\_BIN2\_CTX

3        REF1y\_BIN3\_CTX

4        REF1y\_BIN4\_CTX

≥5      REF1y\_BIN5+\_CTX

REF1y\_SIGN\_CTX

Reference 2

Horizontal

Vertical

Magnitude

Sign

Magnitude

Sign Context

Bin   Context

2        REF2x\_BIN1\_CTX

2        REF2x\_BIN2\_CTX

3        REF2x\_BIN3\_CTX

4        REF2x\_BIN4\_CTX

≥5      REF2x\_BIN5+\_CTX

REF2y\_SIGN\_CTX         



Bin   Context

2        REF2y\_BIN1\_CTX

2        REF2y\_BIN2\_CTX

3        REF2y\_BIN3\_CTX

4        REF2y\_BIN4\_CTX

≥5      REF2y\_BIN5+\_CTX

REF2y\_SIGN\_CTX         


Table   Contexts for Reference1 and Reference2 Vectors

Prediction

The prediction process mv\_pred() predicts Reference 1 motion as the
median of available Reference 1 motion vectors from prediction units in
the prediction aperture, and Reference 2 motion vectors of available
Reference 2 motion vectors from prediction units in the prediction
aperture. The prediction aperture is as specified in Section .

Reference 1 motion vectors in a prediction unit are available if
PRED\_MODE\&1 is non-zero.

Reference 2 motion vectors in a prediction unit are available if
PRED\_MODE\&2 is non-zero.



\subsection{Decoding of DC values}This section specifies the decoding process for a DC value PU\_DC in a
prediction unit whose top-left block has coordinates (x,y).

DC values are decoded differentially with respect to a prediction. The
decoding process is

PU\_DC\_Y = dc\_pred()+su\_arith\_decode()

If CHROMA\_FORMAT is not YONLY then the chroma DC values are also
decoded:

PU\_DC\_C1 = dc\_pred()+su\_arith\_decode()

PU\_DC\_C2 = dc\_pred()+su\_arith\_decode()

Different contexts are used depending upon the binarisation bins and
component as per 


Component

Bin

Context

Y

1

Y\_DC\_BIN1\_CTX

Y

≥2

Y\_DC\_BIN2+\_CTX

C1

1

C1\_DC\_BIN1\_CTX

C1

≥2

C2\_DC\_BIN2+\_CTX

C2

1

C1\_DC\_BIN1\_CTX

C2

≥2

C2\_DC\_BIN2+\_CTX


Table   Contexts for DC value decoding

Prediction 

The prediction process dc\_pred() predicts component DC values for a
prediction unit as the unbiased mean of available DC values for the
corresponding component from prediction units in the prediction
aperture. The prediction aperture is as specified in Section .

DC values in a prediction unit are available if PRED\_MODE is INTRA.



\subsection{Motion vector data context initialisation}Motion vector data contexts are initialised prior to decoding according
to .


Context

COUNT0

COUNT1

REF1x\_BIN1\_CTX

1

1

REF1x\_BIN2\_CTX

1

1

REF1x\_BIN3\_CTX

1

1

REF1x\_BIN4\_CTX

1

1

REF1x\_BIN5+\_CTX

1

1

REF1y\_BIN1\_CTX

1

1

REF1y\_BIN2\_CTX

1

1

REF1y\_BIN3\_CTX

1

1

REF1y\_BIN4\_CTX

1

1

REF1y\_BIN5+\_CTX

1

1

REF2x\_BIN1\_CTX

1

1

REF2x\_BIN2\_CTX

1

1

REF2x\_BIN3\_CTX

1

1

REF2x\_BIN4\_CTX

1

1

REF2x\_BIN5+\_CTX

1

1

REF2y\_BIN1\_CTX

1

1

REF2y\_BIN2\_CTX

1

1

REF2y\_BIN3\_CTX

1

1

REF2y\_BIN4\_CTX

1

1

REF2y\_BIN5+\_CTX

1

1

REF1x\_SIGN\_CTX

1

1

REF1y\_SIGN\_CTX

1

1

REF2x\_SIGN\_CTX

1

1

REF2y\_SIGN\_CTX

1

1

MB\_SPLIT\_BIN1\_CTX

1

1

MB\_SPLIT\_BIN2\_CTX

1

1

MB\_COMMON\_CTX

1

1

PRED\_MODE\_BIT1\_CTX

1

1

PRED\_MODE\_BIT2\_CTX

1

1

Y\_DC\_BIN1\_CTX

1

1

Y\_DC\_BIN2+\_CTX

1

1

C1\_DC\_BIN1\_CTX

1

1

C2\_DC\_BIN2+\_CTX

1

1

C1\_DC\_BIN1\_CTX

1

1

C2\_DC\_BIN2+\_CTX

1

1


Table   Motion vector data context initialisation [True values TBD]




\clearpage
\section{Inverse discrete wavelet transform}%%%%%%%%%%%%%%%%%%%%%%%%%%%%%%%%%%%%%%%%%%%%%%%%%%%%%
% - This chapter defines how the inverse discrete - %
% - wavelet transform is done                     - %
%%%%%%%%%%%%%%%%%%%%%%%%%%%%%%%%%%%%%%%%%%%%%%%%%%%%%


\subsection{Picture IDWT}

The inverse discrete wavelet transform process shall consist of transforming the 
wavelet coefficients for each of the video components. It shall be defined as follows:

\begin{pseudo}{inverse\_wavelet\_transform}{}
\bsCODE{\CurrentPicture[Y]=idwt(\YTransform)}{\ref{idwt}}
\bsCODE{\CurrentPicture[C1]=idwt(\COneTransform)}{\ref{idwt}}
\bsCODE{\CurrentPicture[C2]=idwt(\CTwoTransform)}{\ref{idwt}}
\bsFOREACH{c}{Y,C1,C2}
    \bsCODE{idwt\_pad\_removal(\CurrentPicture[c],c)}{\ref{padremoval}}
\end{pseudo}

\subsection{Component IDWT}
\label{idwt}

This section defines the $idwt(coeff\_data)$ process for reconstructing picture 
component data from decoded subband data $coeff\_data$ using the inverse discrete wavelet transform (IDWT). The IDWT shall be invoked in the picture decoding process only after successful unpacking of the subband coefficient data (Section \ref{transformdec}).

The IDWT process shall return a pixel array from the subband wavelet coefficients representing a reconstructed video component (Y, C1 or C2) for a single picture.


Since wavelet filtering operates on both rows and columns of two-dimensional arrays
 independently it is useful to define operators $\row(a,k)$ and $\column(a,k)$ for 
extracting rows and columns with index $k$ from a 2-dimensional array $a$:

If $b=\row(a,k)$ then $b[r]$ is a {\em reference} to the value of $a[k][r]$. This means that modifying the
value of $b[r]$ modifies the value of $a[k][r]$.

If $b=\column(a,k)$ then $b[r]$ is a {\em reference} the value of $a[r][k]$. This means that modifying the
value of $b[r]$ modifies the value of $a[r][k]$.

The $idwt()$ process shall be an iterative procedure operating on four subbands 
(\LL, \HL,\LH and \HH) at each iteration stage to produce a new subband. The procedure
shall be as follows

\begin{pseudo}{idwt}{coeff\_data}
\bsCODE{LL\_band = coeff\_data[0][\LL]}
\bsFOR{n=1}{\TransformDepth}
   \bsCODE{ 
                   new\_LL= vh\_synth(LL\_band, coeff\_data[n][\HL], coeff\_data[n][\LH], 
                   coeff\_data[n][\HH])
                   }{\ref{vhsynth}}
   \bsCODE{LL\_band=new\_LL}
\bsEND
\bsRET{LL\_band}
\end{pseudo}

Note that at each stage, the input dimensions of the input $LL\_band$ will be the same 
as those of the other input bands, whereas the output dimensions are double those of the input bands.

\subsubsection{Vertical and horizontal synthesis}
\label{vhsynth}

This section specifies the operation of the vertical and horizontal
synthesis process:

$vh\_synth(LL\_data, HL\_data, LH\_data, HH\_data)$

$vh\_synth$ shall return an array of twice the dimensions of each of the input
argument arrays. 

$vh\_synth$ is repeatedly invoked by the IDWT synthesis process and operates on four subband data arrays of identical dimensions to produce a new array $synth$, 
which shall be returned as the result of the process.

{\bf Step 1.} $synth$ is a temporary two-dimensional array that shall be initialised so that:
\begin{eqnarray*}
\width(synth) & = & 2*\width(LL\_data) \\
\height(synth) & = & 2*\height(LL\_data)
\end{eqnarray*}

{\bf Step 2.} The data from the four arrays shall be interleaved as follows:

\begin{pseudo*}
\bsFOR{y=0}{(\height(synth)//2)-1}
    \bsFOR{x=0}{(\width(synth)//2)-1}
        \bsCODE{synth[2*y][2*x] = LL\_data[y][x]}
        \bsCODE{synth[2*y][2*x+1] = HL\_data[y][x]}
        \bsCODE{synth[2*y+1][2*x] = LH\_data[y][x]}
        \bsCODE{synth[2*y+1][2*x+1] = HH\_data[y][x]}
    \bsEND
\bsEND
\bsCODE{\hdots}
\end{pseudo*}

Note: This enables in-place calculation during the inverse filter process.

{\bf Step 3.} Data shall be synthesised vertically by operating on each column
of data using a one-dimensional filter, and then horizontally by operating
on each row. The one-dimensional filters used shall be determined by
the value of $\WaveletIndex$ according to Tables \ref{filtertable0}--\ref{filtertable6}. 
The process shall be as follows:

\begin{pseudo*}
\bsFOR{x=0}{\width(synth)-1}
    \bsCODE{1d\_synthesis(\column(synth, x) )}{\ref{onedsynth}}
\bsEND
\bsFOR{y=0}{\height(synth)-1}
    \bsCODE{1d\_synthesis(\row(synth, y) )}{\ref{onedsynth}}
\bsEND
\bsCODE{\hdots}
\end{pseudo*}

{\bf Step 4.} Finally, the synthesised subband data shall implement a bitshift to
remove any accuracy bits. The bit shift value $filtershift()$ used shall be determined 
by the value of  $\WaveletIndex$ according to Tables \ref{filtertable0}--\ref{filtertable6}. 
The process shall be as follows:

\begin{pseudo*}
\bsCODE{shift = filtershift()}
\bsIF{shift>0}
    \bsFOR{y=0}{\height(synth)-1}
        \bsFOR{x=0}{\width(synth)-1}
            \bsCODE{synth[y][x] = (synth[y][x] + (1<<(shift-1)))\gg shift}
         \bsEND
    \bsEND
\bsEND
\end{pseudo*}

\begin{informative}
Accuracy bits are added in the encoder by shifting up all coefficients in the LL band 
prior to applying any filtering (this includes an initial shift of all values in the component
 data). Adding a small shift before each decomposition stage is the most efficient way of
 providing additional resolution mitigating aliasing through non-linear rounding effects.
\end{informative}

\subsubsection{One-dimensional synthesis}
\label{onedsynth}

This section specifies the one-dimensional synthesis process
$1d\_synthesis()$, which shall apply to a 1-dimensional array of coefficients 
of even length, consisting of either a row or a column of a 2-dimensional integral data array.

The one-dimensional synthesis process shall comprise the application of a
number of reversible integer lifting filter operations. 

Lifting filtering operations shall be one of four types, Type 1, Type 2, Type 3 and 
Type 4. Each type shall be characterised by four elements:
\begin{itemize}
\item a filter length value $L$
\item a filter offset value $D$
\item an array of taps of length $L$: $taps[0],\ldots,taps[L-1]$
\item a scale factor $S$
\end{itemize}

The four types of lifting operations shall be defined by the functions:
\begin{description}
\item[] $lift1(A,L,D,taps,S)$,
\item[] $lift2(A,L,D,taps,S)$,
\item[] $lift3(A,L,D,taps,S)$, and
\item[] $lift4(A,L,D,taps,S)$.
\end{description}
respectively and shall act upon the values in a one-dimensional array $A$.

The Type 1 lifting process $lift1(A,L,D,taps,S)$ shall be defined as follows:

\begin{pseudo}{lift1}{A,L,D,taps,S}
\bsFOR{n=0}{(\length(A)//2)-1}
    \bsCODE{sum=0}
    \bsFOR{i=D}{L+D-1}
        \bsCODE{pos=2*(n+i)-1}
        \bsCODE{pos=\min(pos, \length(A)-1)}
        \bsCODE{pos=\max(pos, 1)}
        \bsCODE{sum+=taps[i-D]*A[pos]}
    \bsEND
    \bsIF{S>0}
        \bsCODE{sum+=(1\ll (S-1))}
    \bsEND
    \bsCODE{A[2*n]+=(sum\gg S)}
\bsEND
\end{pseudo}

The Type 2 lifting process $lift2(A,L,D,taps,S)$ shall be defined as follows:

\begin{pseudo}{lift2}{A,L,D,taps,S}
\bsFOR{n=0}{(\length(A)//2)-1}
    \bsCODE{sum=0}
    \bsFOR{i=D}{L+D-1}
        \bsCODE{pos=2*(n+i)-1}
        \bsCODE{pos=\min(pos, \length(A)-1)}
        \bsCODE{pos=\max(pos, 1)}
        \bsCODE{sum+=taps[i-D]*A[pos]}
    \bsEND
    \bsIF{S>0}
        \bsCODE{sum+=(1\ll (S-1))}
    \bsEND
    \bsCODE{A[2*n]-=(sum\gg S)}
\bsEND
\end{pseudo}

The Type 3 lifting process $lift3(A,L,D,taps,S)$ shall be defined as follows:

\begin{pseudo}{lift3}{A,L,D,taps,S}
\bsFOR{n=0}{(\length(A)//2)-1}
    \bsCODE{sum=0}
    \bsFOR{i=D}{L+D-1}
        \bsCODE{pos=2*(n+i)}
        \bsCODE{pos=\min(pos, \length(A)-2)}
        \bsCODE{pos=\max(pos, 0)}
        \bsCODE{sum+=taps[i-D]*A[pos]}
    \bsEND
    \bsIF{S>0}
        \bsCODE{sum+=(1\ll (S-1))}
    \bsEND
    \bsCODE{A[2*n+1]+=(sum\gg S)}
\bsEND
\end{pseudo}

The Type 4 lifting process $lift4(A,L,D,taps,S)$ shall be defined as follows:

\begin{pseudo}{lift4}{A,L,D,taps,S}
\bsFOR{n=0}{(\length(A)//2)-1}
    \bsCODE{sum=0}
    \bsFOR{i=D}{L+D-1}
        \bsCODE{pos=2*(n+i)}
        \bsCODE{pos=\min(pos, \length(A)-2)}
        \bsCODE{pos=\max(pos, 0)}
        \bsCODE{sum+=taps[i-D]*A[pos]}
    \bsEND
    \bsIF{S>0}
        \bsCODE{sum+=(1\ll (S-1))}
    \bsEND
    \bsCODE{A[2*n+1]-=(sum\gg S)}
\bsEND
\end{pseudo}

$1d\_synthesis$ shall apply the sequence of lifting filters specified in Section \ref{wltfilters}
corresponding to the value of $\WaveletIndex$,and shall invoke the corresponding lifting processes with the parameters defined.

\paragraph{Mathematical formulation of lifting processes (Informative)}
$\ $\newline

The lifting processes defined in the previous section are extremely similar, and 
careful attention should be paid to the detail of their operation in any implementation. 
The four different variants arise from two factors: the �phase� (odd or
even) of the lifting operation, and their implementation using integer-only 
operations, which introduces rounding errors and makes addition and subtraction 
subtly different. 

A lifting operation either modifies the odd coefficients by a linear combination of the 
even coefficients, or vice-versa. Mathematically, the four types of filter may be 
described as follows.

Type 1 and Type 2 lifting filtering operations modify the even coefficients
by the odd coefficients:
\begin{eqnarray*}
  A[2*n]& +=& \left( \sum^M_{i=-N} t_i *A[2*(n+i) + 1] +(1\ll (s-1))\right) \gg s \mbox{ (Type 1)} \\
  A[2*n]& -=& \left( \sum^M_{i=-N} t_i *A[2*(n+i) + 1] +(1\ll (s-1))\right) \gg s \mbox{ (Type 2)}
\end{eqnarray*}

Type 3 and Type 4 lifting filtering operation modify the odd coefficients
 by the even coefficients:
\begin{eqnarray*}
  A[2*n+1]& +=&  \left( \sum^M_{i=-N} t_i A[2*(n+i)]+(1\ll (s-1)) \right) \gg s \mbox{ (Type 3)} \\
  A[2*n+1]& -=&  \left( \sum^M_{i=-N} t_i A[2*(n+i)] +(1\ll (s-1))\right) \gg s \mbox{ (Type 4)} \\
\end{eqnarray*}

The distinctions between Type 1 and Type 2 and between 
Type 3 and Type 4 filters are necessary
because integer division (bit-shifting) is being used, and so the filters are non-linear:
a Type 1 or Type 3 filter with taps $t_i$ is {\em not} equivalent to 
an Type 2 or Type 4 filter with taps $-t_i$.

Edge extension is used where the filter would otherwise extend beyond the 
boundaries of the array. This is slightly different between Types 1 and 2 on the 
one hand and Types 3 and 4 on the other. This is because even values and
odd values must be extended separately to maintain the correct phase (and hence invertibility of the filter). For example, a Type 1 filter must use the values 1 and $\length(A)-1$ at the edges because (as the length is even) these are the odd values nearest the edges.

Further information on wavelet filtering and lifting is provided in Annex 
\ref{transformandlifting}.

\subsubsection{Lifting filter parameters}
\label{wltfilters}

The lifting filters and filter bit-shift operations that apply for each value
$\WaveletIndex$ shall be as specified in Tables  \ref{filtertable0} to \ref{filtertable6}
below.

\begin{table}[h]
\begin{tabular}{|l|}

\hline
Lifting steps:  \\
\begin{tabular}{l}
1. Type 2, $L=2, D=0, taps=[1,1], S=2$ i.e. \\
\quad $ A[2*n]     -=  (A[2*n-1]+A[2*n+1]+2)\gg 2$ \\
2. Type 3, $L=4, D=-1, taps=[-1,9,9,-1], S=4$ i.e. \\
\quad $ A[2*n+1]  +=  (-A[2*n-2]+9*A[2*n]+9*A[2*n+2]-A[2*n+4]+8)\gg 4$
\end{tabular}
\\
\\
$filtershift()$ returns 1\\
\hline

\end{tabular}
\caption{$\WaveletIndex==0$: Deslauriers-Dubuc (9,7) lifting stages and shift values}\label{filtertable0}
\end{table}

\begin{table}[h]

\begin{tabular}{|l|}
\hline
Lifting steps: \\

\begin{tabular}{l}
1. Type 2, $L=2, D=0, taps=[1,1], S=2$ i.e. \\
\quad $ A[2*n]  -= (A[2*n-1]+A[2*n+1]+2)\gg 2$ \\
2. Type 3, $L=2, D=0, taps=[1,1], S=1$ i.e. \\
\quad $ A[2*n+1]  += (A[2*n]+A[2*n+2]+1)\gg 1$
\end{tabular}
\\
\\
$filtershift()$ returns 1\\
\hline

\end{tabular}
\caption{$\WaveletIndex==1$: LeGall (5,3) lifting stages and shift values}
\end{table}

\begin{table}[h]
\begin{tabular}{|l|}

\hline
Lifting steps: \\
\begin{tabular}{l}
1. Type 2, $L=4, D=-1, taps=[-1,9,9,-1], S=5$ i.e. \\
\quad $ A[2*n]  -= (-A[2*n-3]+9*A[2*n-1]+9*A[2*n+1]-A[2*n+3]+16)\gg 5$ \\
2. Type 3, $L=4, D=-1, taps=[-1,9,9,-1], S=4$ i.e. \\
\quad $ A[2*n+1]  += (-A[2*n-2]+9*A[2*n]+9*A[2*n+2]-A[2*n+4]+8)\gg 4$
\end{tabular}
\\
\\
$filtershift()$ returns 1\\
\hline

\end{tabular}
\caption{$\WaveletIndex==2$: Deslauriers-Dubuc (13,7) lifting stages and shift values}
\end{table}

\begin{table}[h]
\begin{tabular}{|l|}

\hline
Lifting steps:\\
\begin{tabular}{l}
1. Type 2, $L=1, D=1, taps=[1], S=1$ i.e. \\
\quad $A[2*n] -=  (A[2*n+1]+1)\gg 1$ \\
2. Type 3, $L=1, D=0, taps=[1], S=0$ i.e. \\
\quad $ A[2*n+1] += A[2*n]$
\end{tabular}
\\
\\
$filtershift()$ returns 0\\
\hline

\end{tabular}
\caption{$\WaveletIndex==3$: Haar filter with no shift}
\end{table}

\begin{table}[h]
\begin{tabular}{|l|}

\hline
Lifting steps:\\
\begin{tabular}{l}
1. Type 2, $L=1, D=1, taps=[1], S=1$ i.e. \\
\quad $A[2*n] -=  (A[2*n+1]+1)\gg 1$ \\
2. Type 3, $L=1, D=0, taps=[1], S=0$ i.e. \\
\quad $ A[2*n+1] += A[2*n]$
\end{tabular}
\\
\\
$filtershift()$ returns 1\\
\hline

\end{tabular}
\caption{$\WaveletIndex==4$: Haar filter with single shift}
\end{table}

\begin{table}[h]
\begin{tabular}{|l|}
\hline
Lifting steps:\\
\begin{tabular}{l}
1. Type 3, $L=8, D=-3, taps=[]-2,10,-25,81,81,-25,10,-2], S=8$ i.e. \\
$ \begin{array}{rcl} A[2*n+1] & += & (-2*(A[2*n-6]+A[2*n+8])+10*(A[2*n-4]+A[2*n+6])\\
                                             & &       -25*(A[2*n-2]+A[2*n+4])+81*(A[2*n]+A[2*n+2])+128)\gg 8 \end{array}$ \\
1. Type 2, $L=8, D=-3, taps=[-8,21,-46,161,161,-46,21,-8], S=8$ i.e. \\
 $ \begin{array}{rcl}A[2*n]  & -= & (-8*(A[2*n-7]+A[2*n+7])+21*(A[2*n-5]+A[2*n+5]) \\
                                          & &      -46*(A[2*n-3]+A[2*n+3])+161*(A[2*n-1]+A[2*n+1])+128)\gg 8 \end{array}$ 
\end{tabular}
\\
\\
$filtershift()$ returns 0\\
\hline

\end{tabular}
\caption{$\WaveletIndex==5$: Fidelity filter for improved downconversion and anti-aliasing}
\end{table}

\begin{table}[h]
\begin{tabular}{|l|}

\hline
Lifting steps:\\
\begin{tabular}{l}
1. Type 2, $L=2, D=0, taps=[1817,1817], S=12$ i.e. \\
\quad $A[2*n] -=  (1817*A[2*n-1]+1817*A[2*n+1]+2048)\gg 12$ \\
2. Type 4, $L=2, D=0, taps=[3616,3616], S=12$ i.e. \\
\quad $A[2*n+1] -= (3616*A[2*n]+3616*A[2*n+2]+2048)\gg 12$ \\
3. Type 1, $L=2, D=0, taps=[217,217], S=12$ i.e. \\
\quad $A[2*n] +=  (217*A[2*n-1]+217*A[2*n+1]+2048)\gg 12$ \\
4. Type 3, $L=2, D=0, taps=[6497,6497], S=12$ i.e. \\
\quad $ A[2*n+1] += (6497*A[2*n]+6497*A[2*n+2]+2048)\gg 12$
\end{tabular}
\\
\\
$filtershift()$ returns 1\\
\hline

\end{tabular}
\caption{$\WaveletIndex==6$: Integer lifting approximation to Daubechies (9,7)}\label{filtertable6}
\end{table}
\clearpage
\subsection{Removal of IDWT pad values}
\label{padremoval}

This section defines the decoding process $idwt\_pad\_removal(pic, c)$
for removing extraneous values after performing the IDWT.

Section \ref{wltinit} requires that subband coefficient data arrays are padded to ensure that 
the reconstructed data array $pic$ has dimensions divisible by $2^\TransformDepth$.

Values $width$ and $height$ are defined to be the appropriate dimensions
of the component data:

\begin{itemize}
\item If $c=Y$, then
\begin{eqnarray*}
width & =& \LumaWidth \\
height & =& \LumaHeight
\end{eqnarray*}
\item else if $c=C1$ or $c=C2$,
\begin{eqnarray*}
width & =& \ChromaWidth \\
height & =& \ChromaHeight
\end{eqnarray*}
\end{itemize}

All component data $pic[j][i]$ with

\begin{itemize}
\item $i\geq width$, or
\item $j\geq height$
\end{itemize}

shall be discarded and $pic$ shall be resized to have width $width$ and height $height$.


\subsection{IWT synthesis operation}
This section defines the process iwt\_synthesis() invoked by iwt().

This is an iterative procedure operating on four subbands at each
iteration stage to produce a new subband. In pseudocode the procedure
is:

\begin{verbatim}
LL=SUBBAND[NUM\_SUBBANDS]
for (n=0 , k=NUM\_SUBBANDS-1; n<TRANSFORM\_DEPTH ; ++n , k-=3)
{
    LL=vh\_synthesis( LL , SUBBAND[k-2] ,  SUBBAND[k-1] , SUBBAND[k] )
}
\end{verbatim}

The decoded component data values will comprise the subband coefficients
LL[][], which are identified with the two-dimensional array DATA[][].



\subsection{Removal of IWT pad values}This section defines the decoding process iwt\_pad\_removal().

When carrying out the forward transform in the coder, it is desirable
that the data is in blocks which permit the iterative subsampling which
is required as part of the lifting process. For some pictures, the
natural size of the image may not give rise to an appropriate set of
values, so padding is inserted. The padding has no useful data and so
should be discarded.

This process is invoked after iwt\_synthesis(). In this process values
not used subsequently are discarded.

Values WIDTH and HEIGHT are defined as follows. For Intra frame
components, if the component is the luma component, the values are set
as

WIDTH = LUMA\_WIDTH

HEIGHT = LUMA\_HEIGHT

If the component is a chroma component, then

WIDTH = CHROMA\_WIDTH

HEIGHT = CHROMA\_HEIGHT

For Inter frame component data, the values are set as follows. For luma
components,

WIDTH = MC\_LUMA\_WIDTH

HEIGHT = MC\_LUMA\_HEIGHT

If the component is a chroma component, then

WIDTH = MC\_CHROMA\_WIDTH

HEIGHT = MC\_CHROMA\_HEIGHT

All component data with horizontal index greater than or equal to WIDTH
or with vertical index greater than or equal to HEIGHT is discarded
(Figure ??).

[Figure ?? Discarded data values]


\subsection{Vertical and horizontal synthesis}This section specifies the operation of the vertical and horizontal
synthesis process vh\_synthesis().

vh\_synthesis( LL , HL , LH , HH ) is repeatedly invoked by
iwt\_synthesis(). It operates on four subbands labelled LL, HL, LH and HH
to produce a new subband SYNTH\_BAND, which is returned as the result of
the process. The subbands LL, HL, LH and HH shall all have identical
dimensions.

First, the data from the four subbands are interleaved to form a single
two-dimensional array A[][] whose vertical and horizontal dimensions are
twice that of each of the original subbands, using the interleave()
process specified in Section :

A=interleave( LL , HL, LH , HH )

Vertical synthesis is performed second. For each column of coefficients
in the array A[][], the 1d\_synthesis() procedure is applied.

Horizontal synthesis is performed third. For each row of coefficients in
the array A[][], the 1d\_synthesis procedure is applied.

The new subband SYNTH\_BAND comprises the data in the processed 2D array
A[][].



\subsection{Interleaving}This section specifies the interleaving process interleave( LL , HL , LH
, HH ) by which the data elements belonging to four subband data arrays
are combined into a single data array A. The elements of A are defined
by:

A[2*j][2*i] = LL[j][i]

A[2*j][2*i+1] = HL[j][i]

A[2*j+1][2*i] = LH[j][i]

A[2*j+1][2*i+1] = HH[j][i]

for all $i$ such that $0 \leq i < width(LL)$ and all $j$ such that
$0 \leq j < height(LL)$.

\begin{informative}
Note that the interleaving process always creates an array with even
dimensions.
\end{informative}


\subsection{One-dimensional synthesis}This section specifies the one-dimensional synthesis process
1d\_synthesis() applied either to rows or columns of the matrix A[][].

1d\_synthesis() applies filtering to a one dimensional array C[] of
coefficients.

Where 1d\_synthesis() is applied to a row of data, then a value C[k]
shall be identified with a value A[j][k] where j is the index of the row
being processed.

Where 1d\_synthesis() is applied to a column of data, then a value C[k]
shall be identified with a value A[k][i] where i is the index of the
column being processed.

The one-dimensional synthesis process comprises the application of a
number of reversible integer lifting filter operations lift1() to
liftN().

The number of lifting operations and their definition depends upon the
choice of wavelet filter derived from the WAVELET\_FILTER\_INDEX value
defined for the frame data unit. The definition of lifting and of
lifting operations for particular wavelet filters are given in
subsequent sections.


\subsection{Integer lifting}This section defines the general operation of a single integer lifting
operation.

Each lifting process applies either to odd or to even coefficients. If
it applies to even coefficients, then the even coefficients are modified
using the values of odd coefficients only. If it applies to odd
coefficients, then the odd coefficients are modified using the values of
even coefficients only. Specifically, a single lifting filtering
operation is of the form:

\begin{displaymath}
  C(2n) = C(2n) + \big( \sum^M_{i=-N} t_i C(2(n+i) - 1) \big) >> s
\end{displaymath}

if it operates on even coefficients and of the form

\begin{displaymath}
  C(2n+1) = C(2n+1) + \big( \sum^M_{i=-N} t_i C(2(n+i)) \big) >> s
\end{displaymath}

if it operates on odd coefficients. The values ti are the lifting filter
tap values; the value s is the lifting filter scale factor.

The lifting operation is applied for all the even, or all the odd,
coefficients in the row or column array C. A decoder may perform the
individual coefficient filtering operations in any order, as this does
not affect the result.

The filtering process uses edge extension.  Where filtering requires
values which fall out of the range of values of the array C, then the
value selected is determined as follows:

- C[2k] is identified with C[0] if $2k < 0$
- C[2k] is identified with C[length(C)-2] if $2k > length(C)$
- C[2k+1] is identified with C[1] if $2k+1 < 0$
- C[2k+1] is identified with C[length(C)-1] if $2k + 1 > length(C)$

\begin{informative}
This specification defines the lifting process on the basis of lifting
procedures applied to an entire row or column consecutively. It is
possible to implement lifting filtering operations so that a filtering
operation associated with one lifting filter is followed by a filtering
operation associated with another lifting filter. I.e. the order of
iteration is changed. In this case, the order in which filtering is
applied to coefficients does affect the outcome of the process as even
lifting operations may be followed by odd ones, and care must be taken
that values are not modified in the wrong order. Nevertheless such an
implementation may be more efficient, and complies with this
specification if it produces identical results.
\end{informative}


\subsection{avaliable filters}     Daubechies (9,7) lifting filters

This section specifies the lifting filters that must be used if
WAVELET\_FILTER\_INDEX=0.

There are four lifting filters lift1, lift2, lift3, and lift4. Their
filtering operations are given in .


Filter

Parity

Filter equation

lift1

Even

C[2n] -= ( 1817 * ( C[2n-1] + C[2n+1] ) )>>12

lift2

Odd

C[2n+1] -= ( 3616 * ( C[2n] + C[2n+2] ) )>>12

lift3

Even

C[2n] += ( 217 * ( C[2n-1] + C[2n+1] ) )>>12

lift4

Odd

C[2n+1] += ( 6497 * ( C[2n] + C[2n+2] ) )>>12


Table   Lifting filters for WAVELET\_FILTER\_INDEX=0



    Approximate Daubechies (9,7) lifting filters

This section specifies the lifting filters that must be used if
WAVELET\_FILTER\_INDEX=1.

There are two lifting filters lift1 and lift2. Their filtering
operations are given in .


Filter

Parity

Filter equation

lift1

Even

C[2n] -=  ( C[2n-1] + C[2n+1] )>>2

lift2

Odd

C[2n+1] += ( 9*( C[2n] + C[2n+2] ) - ( C[2n-2]+C[2n+4] ) )>>4


Table   Lifting filters for WAVELET\_FILTER\_INDEX=1


    (5,3) lifting filters

This section specifies the lifting filters that must be used if
WAVELET\_FILTER\_INDEX=2.

There are two lifting filters lift1 and lift2. Their filtering
operations are given in .


Filter

Parity

Filter equation

lift1

Even

C[2n] -=  ( C[2n-1] + C[2n+1] )>>2

lift2

Odd

C[2n+1] += ( C[2n] + C[2n+2]  )>>1


Table   Lifting filters for WAVELET\_FILTER\_INDEX=2


    (13,5) lifting filters

This section specifies the lifting filters that must be used if
WAVELET\_FILTER\_INDEX=3.

There are two lifting filters lift1 and lift2. Their filtering
operations are given in .


Filter

Parity

Filter equation

lift1

Even

C[2n] -= ( 9*( C[2n-1] + C[2n+1] ) - ( C[2n-3]+C[2n+3] ) )>>4

lift2

Odd

C[2n+1] += ( 9*( C[2n] + C[2n+2] ) - ( C[2n-2]+C[2n+4] ) )>>5


Table   Lifting filters for WAVELET\_FILTER\_INDEX=3






\clearpage
\section{Motion compensation}\label{motioncompensate}

This section defines the operation of the process
$motion\_compensate(ref1, ref2,  pic, c)$ for motion-compensating a
picture component array  $pic$ of type $c=Y, U$ or $V$ from reference 
component arrays $ref1$ and $ref2$ of the same type.

This process is invoked for each component in a picture, subsequent to the 
decoding of coefficient data, specified in Section \ref{transformdec}, and the Inverse Wavelet 
Transform (IWT), specified in Section \ref{idwt}. 

\subsection{Definitions and conventions}

Motion compensation uses the motion block data $\BlockData$ and (optionally) the
global motion parameters $\GlobalParams$.

Since $motion\_compensate()$ applies to both luma and (potentially subsampled)
chroma data, for simplicity a number of local variables are defined. If $c=Y$ then:
\begin{eqnarray*}
lenX & = & \LumaWidth \\
lenY & = & \LumaHeight \\
xblen & = & \LumaXBlen \\
yblen & = & \LumaYBlen \\
xbsep & = & \LumaXBsep \\
ybsep & = & \LumaYBsep
\end{eqnarray*}

If $c=U$ or $c=V$, then likewise:
\begin{eqnarray*}
lenX & = & \ChromaWidth \\
lenY & = & \ChromaHeight \\
xblen & = & \ChromaXBlen \\
yblen & = & \ChromaYBlen \\
xbsep & = & \ChromaXBsep \\
ybsep & = & \ChromaYBsep
\end{eqnarray*}

Define the offsets $xoffset, yoffset$ by
\begin{eqnarray*}
xoffset & = & (xblen-xbsep)//2 \\
yoffset & = & (yblen-ybsep)//2
\end{eqnarray*}

\begin{informative}
The subband data that makes up the IWT coefficients is padded in order that the IWT
may function correctly. For simplicity, in this specification, padding data is removed
after the IWT has been performed so that the picture data and reference data arrays have
the same dimensions for motion compensation. However, it may be more efficient to perform
all operations prior to the output of pictures using padded data, i.e. to discard padding values
subsequent to motion compensation. Such a course of action is equivalent, so long as it is realised
that blocks must be regarded as edge blocks if they overlap the actual picture area, not the
larger area produced by padding. The specification of this section fully supports such an 
interpretation.
\end{informative}

Throughout this Section, the following conventions are used.

\begin{itemize}
\item $x,y$ are co-ordinates in the predicted picture component
\item $u,v$ are co-ordinates in a potentially upconverted reference picture component
\end{itemize}

\begin{comment}
[Are they??]
\end{comment}

\begin{informative*}
\subsection{Overlapped Block Motion Compensation (OBMC) (Informative)}

Motion compensated prediction methods provide methods for determining 
predictions for pixels in the current picture by using motion vectors to 
define offsets from those pixels to pixels in previously decoded
pictures. Motion compensation techniques vary in how those pixels are grouped
together, and how a prediction is formed for pixels in a given group. In 
conventional  block motion compensation, as used in MPEG2, H.264 and many other
codecs, the picture is divided into {\em disjoint} rectangular blocks and the
motion vector or vectors associated with that block defines the offset(s) into
the reference pictures.

In OBMC, by contrast, the predicted picture is divided into a regular overlapping 
blocks of dimensions $xblen$ by $yblen$ that cover at least the entire picture 
area as shown in figure \ref{fig:blockcoverage}.  Overlapping is ensured by starting
each block at a horizontal separation $xbsep$ and a vertical separation $ybsep$ 
from its neighbours, where these values are less than the corresponding block dimensions.
\end{informative*}

\begin{figure}[!h]
\centering
\includegraphics[width=0.7\textwidth]{figs/block-coverage.eps}
\caption{Block coverage of the predicted picture}
\label{fig:blockcoverage}
\end{figure}

\begin{informative*}
The overlap between blocks horizontally is $xblen - xbsep$ and vertically is
$yblen - ybsep$. As a result pixels in the overlapping areas lie in more than
one block, and so more than one motion vector set (and set of associated predictions)
applies to them. Indeed, a pixel may have up to eight predictions, as it may belong to
up to four blocks, each of which may have up to two motion vectors. These are combined
into a single prediction by using weights, which are so constructed so as to sum to 1. In the
 Dirac integer implementation, fractional weights are achieved by insisting that weights sum 
to a power of 2, which is then shifted out once all contributions have been summed.

In Dirac blocks are positioned so that blocks will overspill the left and top edges by 
($xoffset$) and ($yoffset$) pixels.  The number of blocks has been
determined (Section ??) so that the picture area is wholly covered, and the overspill
 on the right hand and bottom edges will be at least the amount on the left and top edges. 
Indeed, the number of blocks has been set so that the blocks divide into whole superblocks
(sets of 4x4 blocks), which mean that some blocks may fall entirely out of the picture area. 
 Any predictions for pixels outside the picture area defined by $0 \leq x < lenX, 0 \leq y <lenY$
are discarded.

\end{informative*}

\subsection{Overall motion compensation process}
\label{mcprocess}

The motion compensation process forms an integer prediction $p[y][x]$ for each pixel in the predicted
picture component $pic$, and adds it to the component data. This is then clipped to keep it in range.
Note that this clipping is {\em in addition} to clipping performed on the output picture after motion
 compensation and/or the inverse wavelet transform (Section \ref{pictureclip}).

\begin{pseudo*}
\bsFOR{y=0}{lenY-1}
    \bsFOR{x=0}{lenX-1}
        \bsCODE{pic[y][x] += pixel\_predict(y, x, pic, ref1, ref2, c)}{\ref{pixelpredict}}
        \bsCODE{pic[y][x] = \clip(pic[y][x], 0, 2^\VideoDepth-1)}
    \bsEND
\bsEND
\end{pseudo*}

\subsection{Pixel prediction}
\label{pixelpredict}

In order to specify the $pixel\_predict()$ process, some definitions are required. For block indices $(i,j)$, 
define the set of elements $B(i,j)$ in the corresponding
block by:
\begin{eqnarray*}
xstart & = & \max(i*xbsep-xoffset, 0) \\
ystart & = & \max(j*xbsep-xoffset, 0) \\
xstop & = & \min\left( xstart+xblen, lenX\right)= \min\left( (i+1)*xbsep+xoffset, lenX\right)\\
ystop & = & \min\left( ystart+yblen, lenY\right)= \min\left( (j+1)*ybsep+yoffset, lenY\right)\\
B(i,j) & = & \{(x,y): xstart\leq x<xstop, ystart\leq y<ystop\}
\end{eqnarray*}

Define the total weight resolution $total\_wt\_bits$ as follows: 
\begin{eqnarray*}
hbits  & = & \log_2(xblen-xbsep)+1 = \log_2(xoffset)+2 \\
vbits  & = & \log_2(yblen-ybsep)+1 = \log_2(yoffset)+2 \\
total\_wt\_bits & = & hbits+vbits+\RefsWeightPrecision
\end{eqnarray*}

This is the number of bits added to pixel values in order to perform OBMC reversibly with integer arithmetic
using the spatial specified in Sections \ref{mcspatialweights} and the reference weights extracted in
parsing the picture prediction header data (Section \ref{refpicweights}).

The $pixel\_predict(y, x, ref1, ref2, c)$ function forms a prediction by adding together weighted predictions
for all blocks containing the pixel $(x,y)$. Weight contributions come both from a spatial matrix and from the
weights assigned to references:

\begin{pseudo}{pixel\_pred}{y, x, pic, ref1, ref2, c}
\bsCODE{p=0}
\bsFORSUCH{(i,j)}{(x,y)\in B(i,j)}
    \bsCODE{m=\BlockData[j][i][mode]}
    \bsIF{m==\Intra}
        \bsCODE{val=\BlockData[j][i][dc][c]}
        \bsCODE{val=val*2^\RefsWeightPrecision}
    \bsELSEIF{m==\RefOneOnly}
        \bsCODE{val=block\_pred(ref1, 1, i, j, x, y, c)}
        \bsCODE{val=val*(\RefOneWeight+\RefTwoWeight)}
    \bsELSEIF{m==\RefTwoOnly}
        \bsCODE{val=block\_pred(ref2, 2, i, j, x, y, c)}
        \bsCODE{val=val*(\RefOneWeight+\RefTwoWeight)}
   \bsELSE
        \bsCODE{val1=block\_pred(ref1, 1, i, j, x, y, c)}
        \bsCODE{val1=val1*\RefOneWeight}
        \bsCODE{val2=block\_pred(ref2, 2, i, j, x, y, c)}
        \bsCODE{val2=va2l*\RefTwoWeight}
    \bsEND{}
    \bsCODE{val=val*spatial\_wt(i,j,x,y)}{\ref{mcspatialweights}}
    \bsCODE{p=p+val}
\bsEND
\bsCODE{p=(p+2^{total\_wt\_bits-1})\gg total\_wt\_bits}
\bsRET{p}
\end{pseudo}


\begin{informative}

{\bf 1.} Note that the number of bits $total\_wt\_bits$ used for the OBMC weighting matrix depends upon the block sizes - specifically
the block overlaps - selected. A Dirac decoder level (\ref{levels}) specifies the maximum block overlaps allowable, and hence 
the word widths necessary for processing OBMC. If we assume that the picture weights are complementary (i.e. the weights
for reference 1 and reference 2 sum to $2^\RefsWeightPrecision$, then the number of bits required for performing 
motion compensation 
calculations is
\[\VideoDepth+total\_wt\_bits+\RefsWeightPrecision\]
unsigned bits. 8 bit video data encoded with block overlaps of 4 luminance pixels and the standard picture weights therefore
requires 8+3+3+1=15 unsigned bits. The additional bit within a 16 bit word could be used to provide additional reference 
weighting.


{\bf 2.} The motion compensation process has been presented as double loop: first over all 
pixels in a given component and second over all blocks of which the pixel is a member. However, if an intermediate
buffer is allowed, the loop order can be reversed. In this case one sets a picture buffer, consisting of 
a component of data (or a strip of component data lines) and add in the weighted predictions for each block. As the
blocks overlap, the contributions sum and form a prediction for every pixel. This is the most natural implementation
strategy:

\begin{pseudo*}
\bsCODE{b[\quad][\quad]=0}
\bsFOR{j=0}{\BlocksY-1}
    \bsFOR{i=0}{\BlocksX-1}
        \bsCODE{m=\BlockData[j][i][mode]}
        \bsFOREACH{(x,y)}{B(i,j)}
            \bsIF{m==\Intra}
                \bsCODE{wt=2^\RefsWeightPrecision * spatial\_wt(i,j,x,y)}
                \bsCODE{b[y][x]+=\BlockData[j][i][dc][c]*wt}
            \bsELSEIF{m==\RefOneOnly}
                \bsCODE{wt=(\RefOneWeight+\RefTwoWeight) * spatial\_wt(i,j,x,y)}
                \bsCODE{b[y][x]+=block\_pred(ref1, 1, i, j, x, y, c)*wt}
            \bsELSEIF{m==\RefTwoOnly}
                \bsCODE{wt=(\RefOneWeight+\RefTwoWeight) * spatial\_wt(i,j,x,y)}
                \bsCODE{b[y][x]+=block\_pred(ref2, 2, i, j, x, y, c)*wt}
            \bsELSE
                \bsCODE{wt=\RefOneWeight * spatial\_wt(i,j,x,y)}
                \bsCODE{b[y][x]+=block\_pred(ref1, 1, i, j, x, y, c)*wt}
                \bsCODE{wt=\RefTwoWeight * spatial\_wt(i,j,x,y)}
                \bsCODE{b[y][x]+=block\_pred(ref2, 2, i, j, x, y, c)*wt}
            \bsEND
        \bsEND
    \bsEND
\bsEND
\bsFOR{y=0}{lenY-1}
    \bsFOR{x=0}{lenX-1}
        \bsCODE{pic[y][x] += (b[y][x]+2^{total\_wt\_bits-1})\gg total\_wt\_bits}
        \bsCODE{pic[y][x] = \clip(pic[y][x], 0, 2^\VideoDepth-1)}
    \bsEND
\bsEND
\end{pseudo*}

The double multiplication by a spatial and by a reference weight can be avoided by using a set of spatial weighting matrices
pre-multiplied by the applicable reference weights according to prediction mode. In the default case where the reference
picture weights are one, as is the picture weight precision, this means a double-size spatial matrix for all modes other than
$\RefOneAndTwo$.

{\bf 3.} The reference prediction weights used for each prediction mode may appear confusing. It is helpful
to think of two cases for using reference picture weighting. The first is interpolative 
prediction, where the picture being predicted is, for example, a cross-fade and is
closely approximated by some mixture of the reference pictures:
 $P\backsimeq\delta R_1+(1-\delta)R_2$. Here the weights we'd like to
use for each frame prediction add up to 1 (or $2^\RefsWeightPrecision$ for integer weights). 
The second case is scaling prediction, where 
the weights we'd like to use for the frame predictions don't add up to 1: for example,
a fade to or from black
$P\backsimeq\delta_1 R_1$ and $P\backsimeq\delta_2 R_2$. It is not possible to choose 
weights for each prediction mode which will be optimal both cases. The weighting
factors chosen will give work with interpolative prediction (which is more common) 
but are not perfect for scaling prediction. It would have been possible to create a variety of
prediction modes to cover all cases, however the potential savings do not justify the
additional complexity.

For interpolative prediction, all data in the current picture will be of commensurate scale to
that of the references. In forming the bi-directional prediction, a value $W_1 p_1 + W_2 p2_2$ is 
formed, so the prediction has "scale" $W_1+W_2$. $W_1+W_2$ is 
therefore the weighting value used to scale unidirectional prediction, in order to provide
predictions of commensurate order. The unity weighting value $2^\RefsWeightPrecision$ is used
for DC blocks as this gives the best prediction, and in the interpolative case this equals $W_1+W_2$
so all predictions are of the same order.

The weighting factors we would like to use for unidirectionally predicted blocks in the scaling case
are $2W_1$ and $2W_2$ - the factor 2 takes into account that we're only adding in one prediction
value as against two for bidirectional prediction. These factors differ from $W_1+W_2$, and hence
unidirectional prediction is incorrect when there are two references. Note, however, that we can
still perform prediction with the correct scaling values when we only have a single reference. Note
also that the value of $W_1+W_2$ was selected instead of $2^\RefsWeightPrecision$, which
would be equivalent in the interpolative case, as it gives a better approximation when the
weights do not sum to $2^\RefsWeightPrecision$.
\end{informative}

\subsection{Spatial weighting matrix}

\label{mcspatialweights}

This section specifies the process $wt(i,j,x,y)$ for deriving a spatial weighting value for a pixel with
coordinates $(x,y)$ in the block with coordinates $(i,j)$. Note that other weights are applied
to the prediction as a result of the weights applied to each reference.

The two-dimensional spatial weighting matrix $W$ applies a linear roll-off in both horizontal and vertical directions
based on the position of the pixel $(x,y)$ within the block $B(i,j)$. Define $xpos, ypos$ as the relative
pixel coordinates from the top-left corner of the block:

\begin{eqnarray*}
xpos & = & x-(i*xbsep-xoffset) \\
ypos & = & x-(i*ybsep-yoffset) 
\end{eqnarray*}

Define a horizontal weighting array $WH$ by the recipe:

\providecommand{\abs}[1]{\left\lvert#1\right\rvert}
\begin{pseudo*}
\bsCODE{max\_x=2(xblen-xbsep)}
\bsIF{i==0 \text{ and } xpos< xblen//2}
    \bsCODE{WH[xpos]=max\_x}
\bsELSEIF{i==\BlocksX \text{ and } xpos\geq xblen//2}
    \bsCODE{WH[xpos]=max\_x}
\bsELSE
    \bsCODE{WH[xpos]  =  \clip(xblen-2\abs{xpos-\dfrac{(xblen-1)}{2}}, max\_x ) }
\bsEND
\end{pseudo*}

Likewise define $WV$ by

\begin{pseudo*}
\bsCODE{max\_y=2(yblen-ybsep)}
\bsIF{j==0 \text{ and } xpos< yblen//2}
    \bsCODE{WH[ypos]=max\_y}
\bsELSEIF{j==\BlocksY \text{ and } ypos\geq yblen//2}
    \bsCODE{WH[ypos]=max\_y}
\bsELSE
    \bsCODE{WV[ypos]  =  \clip(yblen - 2\abs{ypos-\dfrac{(yblen-1)}{2}},0, max\_y) }
\bsEND
\end{pseudo*}

The overall spatial weighting matrix $W$ is given by

\begin{equation*}
W[ypos][xpos] = WH[xpos][WV[ypos]
\end{equation*}

and this is the value returned.

Note that blocks at the extremities of the block set receive maximum weight around their outward-facing edges.
This is to compensate for the lack of blocks making weight contributions on these edges, and ensures that
the total contribution for the pixels in the blocks is $2^alpha$. In section ($i=0$), the profile of the matrix 
for interior blocks is:

\begin{figure}[!h]
\centering
\includegraphics[width=0.7\textwidth]{figs/obmc-profile}
\caption{Profile of overlapped-block motion compensation matrix}
\label{fig:weightprofile}
\end{figure}

\subsection{Block prediction}
\label{blockprediction}

This section specifies the operation of the $block\_pred(ref, ref\_num, i, j, x, y, c)$ 
process for forming a prediction for a pixel 
with coordinates $(x,y)$ in component $c$, belonging to the block with coordinates $(i,j)$.

{\bf Case 1: $\BlockData[j][i][global]=\false$}. In this case, the block motion vectors are used to form a prediction.
Motion vectors for chroma components must be scaled according to the chroma scale factors. If $c=Y$, set

\begin{equation*}
mv= \BlockData[j][i][ref]
\end{equation*}

whereas if $c=U$ or $c=V$, set

\begin{eqnarray*}
mv.x & = & \rounddivide(\BlockData[j][i][ref].x, chroma\_h\_factor() ) \\
mv.y & = & \rounddivide(\BlockData[j][i][ref].y, chroma\_v\_factor() )
\end{eqnarray*}

(chroma subsampling factors are as specified in Section \ref{chromaformats}.)

{\bf Case 2: $\BlockData[j][i][global]=\true$}. In this case, a motion vector is determined from the global
motion parameters as per Section \ref{globalmv}:
\begin{equation*}
mv=global\_mv(ref, ref\_num, x, y, c)
\end{equation*}

In both cases the value 

$upconvert(ref, (x\ll \MotionVectorPrecision)+mv.x, (y\ll \MotionVectorPrecision)+mv.y)$ 

where $upconvert$ is defined in Section \ref{upconvert} is returned.
 
\subsection{Global motion vector field generation}
\label{globalmv}

This section specifies the operation of the $global\_pred(ref, ref\_num, x,y, c)$ process
for deriving a global motion vector for a pixel at location $(x,y)$, in a component of 
type $c$ from a reference $ref$.

\subsubsection{Chroma scaling}
\label{chromascaling}

The global motion parameters are extracted from the state data. If the component is a chroma
component, the parameters must be scaled appropriately. Set:

\begin{itemize}
\item $\tilde{\bf A}=\GlobalParams[ref\_num].{\bf A}$
\item $\tilde{\bf b}=\GlobalParams[ref\_num].{\bf b}$
\item $\tilde{\bf c}=\GlobalParams[ref\_num].{\bf c}$
\end{itemize}

If $c=Y$, set ${\bf A}=\tilde{\bf A}$, ${\bf b}=\tilde{\bf b}$ and ${\bf c}=\tilde{\bf c}$.

If $c=U$ or $c=V$, ${\bf A}$, ${\bf b}$ and ${\bf c}$ are defined as follows. Scale 
${\bf b}$ according to the chroma scale factors:
\begin{eqnarray*}
b_0 & = & \left(\tilde{b}_0+chroma\_h\_factor()//2\right)//chroma\_h\_factor() \\
b_1 & = & \left(\tilde{b}_1+chroma\_h\_factor()//2\right)//chroma\_h\_factor()
\end{eqnarray*}

Scale ${\bf A}$, taking into account vertical and horizontal factors:
\begin{eqnarray*}
A_{0,0} & = & \tilde{A}_{0,0}\\
A_{1,1} & = & \tilde{A}_{1,1}\\
A_{0,1} & = & \left(\tilde{A}_{0,1}*chroma\_h\_factor()+chroma\_v\_factor()//2 \right) // chroma\_v\_factor() \\
A_{1,0} & = & \left(\tilde{A}_{1,0}*chroma\_v\_factor()+chroma\_h\_factor()//2\right)//chroma\_h\_factor()
\end{eqnarray*}

and give ${\bf c}$ the inverse scaling to ${\bf b}$:
\begin{eqnarray*}
c_0 & = & \tilde{c}_0*chroma\_h\_factor \\
c_1 & = & \tilde{c}_1*chroma\_h\_factor
\end{eqnarray*}

\subsubsection{Field generation}

Set $\alpha = \GlobalParams[ref\_num][ZRS\_exp]$ and 
$\beta =\GlobalParams[ref\_num][perspective\_exp]$.

Writing ${\bf x}=\left( \begin{array}{c} x\\y \end{array}\right)$, set ${\bf y}$ to be
the integer vector defined by:
\begin{equation*}
{\bf y}=\left(2^{\alpha+\beta}-2^\alpha * {\bf c}^T {\bf x}\right)\left(2^\beta*{\bf Ax}+2^{\alpha+\beta}*{\bf b}\right)
\end{equation*}

Set
\begin{eqnarray*}
mv.x & = & y_0\gg (\alpha+\beta) \\
mv.x & = & y_1\gg (\alpha+\beta)
\end{eqnarray*}

and return the motion vector $mv$.

\subsection{Upconversion}
\label{upconvert}

This section specifies the operation of the $upconvert(ref, u, v)$ function
for producing a value from an upconverted picture reference. This
allows for sub-pixel precision in motion compensation.

Motion vectors are allowed to extend beyond the edges of the 
upconverted reference picture component and values lying outside the range
of the component are determined by edge extension, using the values:
\begin{eqnarray*}
cu & = &\clip(u, 0, 2^\MotionVectorPrecision*width(ref)) \\
cv & = &\clip(v, 0, 2^\MotionVectorPrecision*height(ref))
\end{eqnarray*}

There are four cases, depending upon the motion vector precision selected.

\subsubsection{Pixel-accurate motion vectors}

If $\MotionVectorPrecision==0$, no upconversion is actually required and the value
$ref[cv][cu]$ is returned.

\subsubsection{Half-pixel accurate motion vectors}
\label{halfpel}

If $\MotionVectorPrecision==1$ then the reference picture component $ref$ is
upconverted by a factor of 2 in each dimension to create an
array $upref$. The value returned is $upref[cv][cu]$.

$upref$ is created in two stages, first upconverting vertically by
a factor of 2, then horizontally. Define the interpolation filter $h$
to be the 10-tap symmetric filter with taps as defined in figure \ref{upfilter}.

\begin{figure}[h!]
\begin{centering}
\begin{tabular}{l|ccccc}
Tap & $t_0$ & $t_1$ & $t_2$ & $t_3$ & $t_4$\\
\hline
Value & 167 & -56 & 25 & -11 & 3
\end{tabular}
\caption{Interpolation filter coefficients \label{upfilter}}
\end{centering}
\end{figure}

Define an array $ref2$ of height $2*height(ref)$ and width $width(ref)$ by
the recipe, for $0\leq p<width(ref)$ and $0\leq q<2*height(ref)$:

{\bf Case 1.} If $q\%2==0$, set
\begin{equation*}
ref2[q][p]=ref2[q//2][p]
\end{equation*}

{\bf Case 2.} If $q\%2!=0$, $ref2[q][p]$ is set by

\begin{pseudo*}
\bsCODE{ref2[q][p]=  \sum_{i=0}^{4} t_i *(ref[\clip( (q-1)//2-i, 0, height(ref)-1)][p]+
ref[\clip((q+1)//2+i, 0,height(ref)-1)) ][p])}
\bsCODE{ref2[q][p] = (ref2[q][p]+128)\gg 8}
\bsCODE{ref2[q][p] = \clip(ref2[q][p], 0, 2^\VideoDepth-1)}
\end{pseudo*}

The full upconverted array is constructed from $ref2$ in the same way.
 For $0\leq p<2*width(ref)$ and $0\leq q<2*height(ref)$ we have:

{\bf Case 1.} If $p\%2==0$, set
\begin{equation*}
upref[q][p]=ref2[q][p//2]
\end{equation*}

{\bf Case 2.} If $p\%2!=0$, $upref[q][p]$ is set by

\begin{pseudo*}
\bsCODE{upref[q][p]=  \sum_{i=0}^{4} t_i *\left(ref2[q][\clip( (p-1)//2-i, 0, width(ref2)-1)]+
ref2[q][\clip((p+1)//2+i, 0, width(ref2)-1)]\right)}
\bsCODE{upref[q][p] = (upref[q][p]+128)\gg 8}
\bsCODE{upref[q][p] = \clip(upref[q][p], 0, 2^\VideoDepth-1)}
\end{pseudo*}

\begin{informative}
While this filter may appear to be variable separable, the integer rounding and 
clipping processes prevent this being so. Note also that the clipping process for
filtering terms implies that the upconversion uses edge-extension at the array
edges, consistent with the edge-extension used in motion-compensation itself.
\end{informative}

\subsubsection{Quarter- and eighth-pixel accurate motion vectors}

If $\MotionVectorPrecision==2$ or $\MotionVectorPrecision==3$, upconverted values are derived by linear
interpolation from the half-pixel interpolation values $upref$, which is calculated as 
per Section \ref{halfpel}. Given coordinates $(u,v)$, their half-pixel part
is extracted by:

\begin{eqnarray*}
hu & = & u \gg (\MotionVectorPrecision-1) \\
hv & = & v \gg (\MotionVectorPrecision-1) 
\end{eqnarray*}

and their remainder (giving the residual subpixel accuracy) by

\begin{eqnarray*}
ru & = & u-(hu\ll (\MotionVectorPrecision-1)) \\
rv & = & v-(hv\ll (\MotionVectorPrecision-1)) 
\end{eqnarray*}

$ru$ and $rv$ satisfy $0\leq ru,rv <2^{\MotionVectorPrecision-1}$. Then define four weighting
values by:

\begin{eqnarray*}
w00 & = & (2^{\MotionVectorPrecision-1}-rv)*(2^{\MotionVectorPrecision-1}-ru)\\
w01 & = & (2^{\MotionVectorPrecision-1}-rv)*ru\\
w10 & = & rv*(2^{\MotionVectorPrecision-1}-ru)\\
w11 & = & rv*ru
\end{eqnarray*}

and also define the clipped coordinates that we shall use for interpolation by:

\begin{eqnarray*}
cu & = \clip(hu, 0, width(upref)-1) \\
cu1 & = \clip(hu+1, 0, width(upref)-1) \\
cv & = \clip(hv, 0, width(upref)-1) \\
cv1 & = \clip(hv+1, 0, width(upref)-1)
\end{eqnarray*}

The value returned is:
\[
\left(
\begin{array}{l}
w00*upref[cv][cu]+\\
w01*upref[cv][cu1]+\\
w10*upref[cv1][cu]+\\
w11*upref[cv1][cu1]+2^{\MotionVectorPrecision-2}
\end{array}
\right)\gg(\MotionVectorPrecision-1)\]

\begin{informative}
Note that the remainder values $ru$, $rv$ are not determined from the clipped
half-pixel values $cu$, $cv$, $cu1$, and $cv1$. This ensures the remainder
values depend only on the motion vector, and hence are constant across
each block, and allows a block-wise implementation. If the clipped values 
had been used, blocks whose reference
block straddled the edge of a picture would use different remainders in
different parts of the block. See Section \ref{mcimplementation}.
\end{informative}

\begin{informative*}
\subsection{Implementation}
\label{mcimplementation}
[TBC]

The motion compensation process is defined in this specification in
terms of the prediction determined for each pixel. Typically, an
implementation will motion compensate all pixels within a prediction
unit or a block together, as they share motion parameters and hence will
have a contiguous prediction data set in the reference frames.


%The weighting matrix is then applied directly
%to the prediction block as a whole.
%
%It is not necessary for the prediction buffer to consist of the whole video
%component -- it could be a single row of blocks, with values being overwritten
%successively.
%
%The process for a single video component is effectively:
%
%-- Starting with an empty prediction buffer,
%pred(x,y) = 0, forall x,y
%
%for r = 0 to numrefs:
%    -- upconvert the reference frame and clip the results
%    ref'_r(x,y) = clip(upconvert(ref_r,x,y),0,2^{video_depth})
%
%    -- perform block prediction, OBMC weighting and reference weighting,
%    -- accumumulating the results into the prediction buffer.
%    foreach b in blocks:
%        b'pred(x,y) = block_predict(b,x,y)
%
%        -- choosereference weight contribution
%        W <- ...
%
%        for i = 0 to yblen:
%            for j = 0 to xblen:
%                -- accumulate predicted pixels into prediction buffer
%                pred
%
%%\begin{lstlisting}
%%    upconvertB_1d(f,r,x) = let $x' = x mod f$
%%                           in $%
%%    \lfloor\frac{(16-fx')r_{\frac{x-x'}{f}} + fx'r_{\frac{x+f-x'}{f}} + 8}{16}\rfloor%
%%    $
%%\end{lstlisting}
%
%
%\annotate{df}{Contrasts to the rest of the
%section that uses absolute pixel position (not vector)}
%\begin{informative}
%A note about global motion calculation.
%
%It is possible to calculate the entire global motion field using
%approximately four additions per pixel.
%
%The global transformation may be expanded as:
%
%\begin{equation*}
%\pmb{z}_{ref} = -\pmb{A}\left(\pmb{z}\pmb{z}^\text{T}\right)\pmb{c}
%                + \left(\pmb{A} - \pmb{b}\pmb{c}^\text{T}\right)\pmb{z}
%                + \pmb{b}
%\end{equation*}
%
%Then $\pmb{z}_{ref} - \pmb{z}$ may be further expanded to give the individual
%components of the motion vector:
%
%\begin{align*}
%  v_x &= \alpha{}x^2 + \beta{}xy + \gamma{}y^2 + (\delta - 1)x + \epsilon{}y + b_1 \\
%  v_y &= \zeta{}x^2 + \eta{}xy + \theta{}y^2 + \kappa{}x + (\mu - 1)y + b_2 \\
%\end{align*}
%where,
%\begin{align*}
%\alpha &= -a_{11}c_1 &
%\beta &= -(a_{11}c_2 + a_{12}c_1) &
%\gamma &= -a_{12}c_2 \\
%\delta &= a_{11} - b_1c_1 &
%\epsilon &= a_{12} - b_1c_2 &
%\zeta &= -a_{21}c_1 \\
%\eta &= -(a_{21}c_2 + a_{22}c_1) &
%\theta &= -a_{22}c_2 &
%\kappa &= a_{21} - b_2c_1 &
%\mu &= a_{22} - b_2c_2
%\end{align*}
%
%That is the components of the global motion vector are given by
%two-dimensional quadratic functions.
%
%Given a two dimensional quadratic function:
%
%\begin{equation*}
%z(x,y) = px^2 + qxy + ry^2 + sx + ty + u
%\end{equation*}
%
%We can calculate z(y,x) for successive values of $x$ and $y$ by merely
%incremeting the values for $z(y, x-1)$ or $z(y-1, x)$.  The following
%algorithm for generating a two dimensional array containing the quadratic
%values $z(y,x)$.
%
%\begin{verbatim}
%quadratic(p,q,r,s,t,u):
%    zed = array2D(ylength, xlength)
%    g = 2*p
%    h0 = s - p - q
%    k = 2*r
%    m0 = t - r
%    m = m0 - k
%    z0 = u - m0
%    for y in range(ylength):
%        h0 += q
%        h = h0 - g
%        m += k
%        z0 += m
%        z = z0 - h0
%        for x in range(xlength):
%            h += g
%            z += h
%            zed[y][x] = z
%    return zed
%\end{verbatim}
%

\end{informative*}
\subsection{Block coverage}The predicted picture is divided into a regular overlapping blocks that
cover at least the entire picture area as shown in
figure~\ref{fig:blockcoverage}.  Each block (figure~\ref{fig:}) has a
width and height, $\xblen$ and $\yblen$ respectively.  A value of
$\xbsep$ and
$\ybsep$ repsectively define the horizontal and vertical distances between
blocks.

\begin{figure}[h]
\centering
\includegraphics[width=0.7\textwidth]{figs/block-coverage}
\caption{Block coverage of the predicted picture}
\label{fig:blockcoverage}
\end{figure}

Blocks will overspill the left and top edges by ($\xblen - \xbsep$) and
($\yblen - \ybsep$) pixels.  The overspill on the right hand and bottom edges
will be at least the amount on the left and top edges; in order to
maintain the overlapping property of Overlapped-block motion
compensation and satisfy bytestream requirements, the bottom and right
edge overspill may be multiple blocks.  Any predictions for pixels
outside the picture $(0 \leq x < \picWidth, 0 \leq y < \picHeight)$
are discarded.

\subsection{Algorithm}



The pixel $(x,y)$ of the predicted picture $P$ is formed by summing the
weighted contributions from all the blocks ($\B$) that cover $(x,y)$, for
each reference.  The pixel $(x,y)$ is covered by no more than four
blocks in the overlapping region of the block structure, leading to a
maximum of eight contributions (four from each reference).


The motion compensation process is defined as:

\begin{multline*}
P(x,y) =
 \left[
    2^{\tau+\sigma-1} +
    \displaystyle\sum^{\text{numrefs}}_{r}\, \sum^{\forall \B \text{ at } (x,y)}_{\B}
      \left\bracevert\begin{aligned}
        \text{let}\quad& W = \begin{cases}
                                \W_r &\scriptstyle\text{if numrefs = 1 or } \Brefsinuse = \lbrace0,1\rbrace,\\
                                2^{\tau-1}   &\scriptstyle\text{if numrefs = 2 and } \Bpredmode = \predIntra,\\
                                2^\tau &\scriptstyle\text{otherwise.}
                              \end{cases} \\%
        \text{in}\quad & W*\text{block\_predict}(r,\B,x,y)*\omega_{x-\Bx, y-\By}
        \end{aligned}\right.
 \right]
   \gg (\tau + \sigma)
\end{multline*}

\annotate{df}{sigma not defined}
\annotate{df}{too wide?}



\subsection{Upconversion}Motion compensation is performed against reference pictures that have
been upconverted by a factor of $2^\lambda$, allowing sub-pixel
precision in motion estimation.

\begin{equation*}
\URef_{r,u,v} = \text{upconvert}(\text{Ref}_r,2^\lambda,u,v)
\end{equation*}
\annotate{df}{no function}

\subsubsection{Method A: Half-pixel interpolation using ten-tap filter}
Upconversion by a factor of two is performed using half-pixel
interpolation with a symmetric ten-tap filter.

\begin{figure}[h!]
\begin{centering}
\begin{tabular}{l|ccccc}
Tap & $t_0$ & $t_1$ & $t_2$ & $t_3$ & $t_4$\\
\hline
Value & 167 & -56 & 25 & -11 & 3
\end{tabular}
\caption{Interpolation filter coefficients}
\end{centering}
\end{figure}

While this filter may appear to be variable separable, due to integer
rounding, filtering must be performed vertically first, then horizontally.

\begin{equation*}
\text{upconvertA\_1d}(p,u) =
  \begin{cases}
    \displaystyle
    \clip\left(\left[128
          + \sum_{i=0}^{4} t_i (p_{\frac{u-1}{2}-i} + p_{\frac{u+1}{2}+i})
          \right] \gg 8, 0, 2^{\text{\texttt{bit\_depth}}-1} \right) & \text{if $u$ is odd},\\
    p_{\frac{u}{2}} & \text{otherwise}.
  \end{cases}
\end{equation*}
\annotate{df}{no 2d version}


\subsubsection{Method B: Quarter- and eighth-pixel linear interpolation}
Upconversion by factors of four or eight is achieved by first
upconverting by a factor of two using method A, then linearly interpolating by a
factor of two or four respectively:

\begin{equation*}
\text{upconvertB\_2d}(n,p,u,v) =
  \left\bracevert\begin{aligned}
    &\text{let } u' = u \bmod n \text{ in} \\
    &\text{let } v' = v \bmod n \text{ in} \\
    &\text{let } a = n^2 (n-u')(n-v')  p_{\frac{u-u'}{n},\frac{v-v'}{n}} \text{ in} \\
    &\text{let } c = n^2 u'(n-u')  p_{\frac{u-u'}{n},\frac{v+n-v'}{n}} \text{ in} \\
    &\text{let } b = n^2 v'(n-v')  p_{\frac{u+n-u'}{n},\frac{v-v'}{n}} \text{ in} \\
    &\text{let } d = n^2 v'u'  p_{\frac{u+n-u'}{n},\frac{v+n-v'}{n}} \text{ in} \\
    &\Bigl[ a + b + c + d + 2^{2n} \Bigr] \gg 2n
    \end{aligned}\right.
\end{equation*}
\annotate{df}{$\frac{u-u'}{n}$ is $x$}

\subsection{Forming an unweighted block prediction}
Given a single block \verb|b|, there are three different methods for
predicting the contents of the block; depending upon the block's
prediction mode \verb|b.prediction_mode|:

\begin{equation*}
\text{block\_predict}(r,\B,x,y) =
  \begin{cases}
    \text{intra\_predict}(\B) & \text{when }\Bpredmode = \predIntra,\\
    \text{inter\_predict}(r,\B,x,y) & \text{when }\Bpredmode = \predInter \text{ and } r \in \Brefsinuse,\\
    \text{global\_predict}(r,x,y) & \text{when }\Bpredmode = \predGlobal \text{ and } r \in \Brefsinuse,\\
    0 & \text{otherwise}
  \end{cases}
\end{equation*}

\subsubsection{Intra blocks}
Intra coded blocks contain a dc residue.\annotate{df}{something about selecting
which dc component (chroma)}

\begin{equation*}
\text{intra\_predict}(\B) = \Bdc
\end{equation*}

\subsubsection{Inter blocks}
Inter coded blocks contain one or two vectors $\Bv_r$

\providecommand{\vv}[0]{\bigl[\begin{smallmatrix}u\\v\end{smallmatrix}\bigr]}
\providecommand{\V}[0]{\bigl[\begin{smallmatrix}x\\y\end{smallmatrix}\bigr]}
\begin{equation*}
\text{inter\_predict}(r,\B,x,y) =
  \left\bracevert\begin{aligned}
    &\text{let } \vv = 2^\lambda\V + \Bv_r \text{ in}\\
    &\URef_{r,\clip(u, 0, \picWidth-1), \clip(v, 0, \picHeight-1)}
    \end{aligned}\right.
\end{equation*}
\annotate{df}{discuss clipping}

\subsubsection{Global blocks}
Blocks signalled to use the global motion parameters. Global motion
compensation is performed on the basis of a dense motion field generated
using a parameterised model of motion.  A dense motion field is one in
which there may be a different motion vector for each pixel in the
predicted picture.

Dirac uses an eight parameter model that allows for pan, zoom, rotation,
shear and change of perspective.


\begin{equation*}
\text{global\_predict}(r,x,y) =
  \left\bracevert\begin{aligned}
    &\text{let } \vv = \Bigl[(2^\lambda\gmA_r\V + 2^\mu\gmB_r)(2^\psi - \gmC_r^\text{T}\V) + 2^{\mu\psi-1} \Bigr] \gg (\mu\psi) \text{ in}\\
    &\URef_{r, \clip(u, 0, \picWidth-1), \clip(v, 0, \picHeight-1)}
    \end{aligned}\right.
\end{equation*}

A more optimal method of calculating the global motion field is shown in
section~\ref{mc:}.  Any method must be compliant with the above.

\subsection{Overlapped-block motion compensation weighting matrix}
For purposes of overlapped block motion compensation, it is necessary to
weight each block by a weighting matrix of size $(\xblen,\yblen)$ using:

\providecommand{\abs}[1]{\left\lvert#1\right\rvert}
\begin{equation*}
\begin{split}
\omega_{i,j} &= \text{clip}\Bigl(\left( \xblen - 2\abs{j-\tfrac{\xblen-1}{2}} \right), 0, 2(\xblen-\xbsep)\Bigr) \\
        &\quad* \text{clip}\Bigl(\left( \yblen - 2\abs{i-\tfrac{\yblen-1}{2}} \right), 0, 2(\yblen-\ybsep)\Bigr)
\end{split}
\end{equation*}

NB: len - sep must be a power of 2.

In section ($i=0$), the profile of the matrix is:

\begin{figure}[h]
\centering
\includegraphics[width=0.7\textwidth]{figs/obmc-profile}
\caption{Profile of overlapped-block motion compensation matrix}
\label{fig:blockcoverage}
\end{figure}


\subsection{Chominance components}\label{mc:scale}
When motion compensating chrominance components of the picture, some
scaling of parameters is required when the chroma components
do not have the same dimensions as the luminance component in order to
take into account different chroma sampling formats.

The horizontal and vertical scaling factors are:

\begin{align*}
\text{\texttt{chroma\_h\_scale}} =
  \begin{cases}
    1 &\text{when Cr4:4:4},\\
    2 &\text{when Cr4:2:2},\\
    2 &\text{when Cr4:2:0}
  \end{cases}
&\qquad
\text{\texttt{chroma\_v\_scale}} =
  \begin{cases}
    1 &\text{when Cr4:4:4},\\
    1 &\text{when Cr4:2:2},\\
    2 &\text{when Cr4:2:0}
  \end{cases}
\end{align*}

The scaling of parameters is then:
\begin{itemize}
\item Each block is reduced in size horizontally and vertically by
factors of \texttt{chroma\_h\_scale} and \texttt{chroma\_v\_scale}.

\item Each block motion vector is reduced in magnitude horizontally
and vertically by factors of \texttt{chroma\_h\_scale} and
\texttt{chroma\_v\_scale}.

\end{itemize}
\annotate{df}{what about global motion?}
\annotate{df}{needs rounding not truncation}

\subsection{Notes on precision}The encoder is required to ensure that all calculations are performed by
a decoder within the bounds set by the current level in force.

Typical values for $\psi$ and $\mu$ in the global motion calculations
may be quite large.  Typically the perspective exponent ($\psi$) may be
of the order $2\lg(\picWidth) + \lambda$ bits; the zoom-rotation-shear
exponent ($\mu$) would similary be of the order $\lg(\picHeight) +
\lambda$ bits.

\subsection{Example implementation}
\begin{informative}
The motion compensation process is defined in this specification in
terms of the prediction determined for each pixel. Typically, an
implementation will motion compensate all pixels within a prediction
unit or a block together, as they share motion parameters and hence will
have a contiguous prediction data set in the reference frames.


The weighting matrix is then applied directly
to the prediction block as a whole.

It is not necessary for the prediction buffer to consist of the whole video
component -- it could be a single row of blocks, with values being overwritten
successively.
\end{informative}

\begin{comment}
The process for a single video component is effectively:

-- Starting with an empty prediction buffer,
pred(x,y) = 0, forall x,y

for r = 0 to numrefs:
    -- upconvert the reference frame and clip the results
    ref'_r(x,y) = clip(upconvert(ref_r,x,y),0,2^{video_depth})

    -- perform block prediction, OBMC weighting and reference weighting,
    -- accumumulating the results into the prediction buffer.
    foreach b in blocks:
        b'pred(x,y) = block_predict(b,x,y)

        -- choosereference weight contribution
        W <- ...

        for i = 0 to yblen:
            for j = 0 to xblen:
                -- accumulate predicted pixels into prediction buffer
                pred

%\begin{lstlisting}
%    upconvertB_1d(f,r,x) = let $x' = x mod f$
%                           in $%
%    \lfloor\frac{(16-fx')r_{\frac{x-x'}{f}} + fx'r_{\frac{x+f-x'}{f}} + 8}{16}\rfloor%
%    $
%\end{lstlisting}
\end{comment}

\annotate{df}{Contrasts to the rest of the
section that uses absolute pixel position (not vector)}
\begin{informative}
A note about global motion calculation.

It is possible to calculate the entire global motion field using
approximately four additions per pixel.

The global transformation may be expanded as:

\begin{equation*}
\pmb{z}_{ref} = -\pmb{A}\left(\pmb{z}\pmb{z}^\text{T}\right)\pmb{c}
                + \left(\pmb{A} - \pmb{b}\pmb{c}^\text{T}\right)\pmb{z}
                + \pmb{b}
\end{equation*}

Then $\pmb{z}_{ref} - \pmb{z}$ may be further expanded to give the individual
components of the motion vector:

\begin{align*}
  v_x &= \alpha{}x^2 + \beta{}xy + \gamma{}y^2 + (\delta - 1)x + \epsilon{}y + b_1 \\
  v_y &= \zeta{}x^2 + \eta{}xy + \theta{}y^2 + \kappa{}x + (\mu - 1)y + b_2 \\
\end{align*}
where,
\begin{align*}
\alpha &= -a_{11}c_1 &
\beta &= -(a_{11}c_2 + a_{12}c_1) &
\gamma &= -a_{12}c_2 \\
\delta &= a_{11} - b_1c_1 &
\epsilon &= a_{12} - b_1c_2 &
\zeta &= -a_{21}c_1 \\
\eta &= -(a_{21}c_2 + a_{22}c_1) &
\theta &= -a_{22}c_2 &
\kappa &= a_{21} - b_2c_1 &
\mu &= a_{22} - b_2c_2
\end{align*}

That is the components of the global motion vector are given by
two-dimensional quadratic functions.

Given a two dimensional quadratic function:

\begin{equation*}
z(x,y) = px^2 + qxy + ry^2 + sx + ty + u
\end{equation*}

We can calculate z(y,x) for successive values of $x$ and $y$ by merely
incremeting the values for $z(y, x-1)$ or $z(y-1, x)$.  The following
algorithm for generating a two dimensional array containing the quadratic
values $z(y,x)$.

\begin{verbatim}
quadratic(p,q,r,s,t,u):
    zed = array2D(ylength, xlength)
    g = 2*p
    h0 = s - p - q
    k = 2*r
    m0 = t - r
    m = m0 - k
    z0 = u - m0
    for y in range(ylength):
        h0 += q
        h = h0 - g
        m += k
        z0 += m
        z = z0 - h0
        for x in range(xlength):
            h += g
            z += h
            zed[y][x] = z
    return zed
\end{verbatim}

\end{informative}


\clearpage
\section{Arithmetic decoding}\label{arithcoding}

This appendix provides three things:
\begin{itemize}
\item a description of the principles of arithmetic
coding
\item a specification of the arithmetic decoding
engine used in Dirac
\item a description of a compatible arithmetic encoder
\end{itemize}

\begin{informative*}
\subsection{Arithmetic coding principles (Informative)}

This section provides an introduction to the principles underlying arithmetic
coding. Since arithmetic coding is very extensively described in published literature,
this section is necessarily brief: for more information, Alasdair Moffat's
article ''Arithmetic coding revisited'' (ACM Transactions on Information Systems,
Vol. 16 \#3, July 1998) is recommended. 

Arithmetic coding is an extremely powerful form of entropy coding, which closely
approximates the Shannon information limit for given data. Arithmetic 
encoding consists of an asynchronous state machine, 
in which data
is fed to an arithmetic encoding engine, together with an estimate of its
probability, and the encoder outputs bits (Figure \ref{fig:arithencoder}. 
It is asynchronous because data
input does not trigger any output directly, but changes the state of the 
engine. When a certain state is reached a variable amount of output is produced.
\end{informative*}
\setlength{\unitlength}{1em}
\begin{figure}[!ht]
\centering
\begin{picture}(45,12)

%\put(0,3){\vector(1,0){5}}
%\put(10,3){\oval(10,4.7)\put(-1.75,0.5){Parse} \put(-1.25,-1){Info}}
%\put(15,3){\vector(1,0){5}}
%\put(25,3){\oval(10,4.7)\put(-1.75,0.5){Access}\put(-1.3,-1){Unit}}
%\put(30,3){\vector(1,0){10}}
%\put(17.5,3){\line(0,1){5}}
%\put(32.5,3){\line(0,1){5}}
%\put(32.5,8){\vector(-1,0){10}}
%\put(17.5,8){\line(1,0){10}}

\end{picture}
\caption{Arithmetic encoder}\label{fig:arithencoder}
\end{figure}

\begin{informative*}
This asynchronous nature makes arithmetic coding trickier to implement 
than Variable-Length Codes (VLCs) but is essential to its optimal nature. Consider
a binary symbol $b$, with $p(b=0)=p_0$ and $p(b=1)=1-p_0$. The entropy of $b$
is the expected number of bits required to encode $b$, and is equal to
\[e(p_0)=p_0\log_2(1/p_0)+(1-p_0)\log_2(1/(1-p_0))\]

If $e(p_0)$ is plotted against $p_0$, it can be seen that if $p_0$ is not equal
to 0.5 exactly, $e(p_0)<1$. This means that an optimal binary entropy encoder
that operates symbol by symbol, cannot produce an output for every symbol - i.e.
it must operate asynchronously.

\subsubsection{Interval division and scaling}
The fundamental idea of arithmetic coding is interval division and scaling. An
arithmetic code can be thought of as a single number lying in an interval 
determined by the sequence of values being coded. For simplicity, this discussion
describes binary arithmetic coding, but larger symbol alphabets can be treated
in an analogous manner.

Let us begin with the interval $[0,1)$, and suppose that we know (or have some 
estimate of) the probability of zero, $p_0$. Conceptually we divide the interval
into the intervals $[0,p0)$ and $[p_0,1)$. Suppose we code a 0 as the
first symbol. In this case the interval is changed to $[0,p0)$. If we code a 1,
then the interval is changed to $[p0,1)$. After coding 1 or more symbols we 
arrive at an interval $[low,high)$. To code the next symbol we partition this interval
into $[low,low+p_0(high-low))$ and $[low+p_0(high-low),b)$, and if there is a 0 we choose the
first interval, and if a 1 the second.

For any integer $N$, this process clearly partitions the interval $[0,1)$ into 
a set of disjoint intervals that correspond to all the sequences of $N$ bits.
Identifying such a bit sequence is equivalent to choosing a value in the 
corresponding interval, and for an interval width $w$ that in general requires 
\[\left\lceil\log_2(1/w)\right\rceil\]
bits. With static probabilities, on average,
\[w=p_0^{Np_0}(1-p_0)^{N(1-p_0)}\]
resulting in 
\[\left\lceil Ne(p_0)\right\rceil\]
being used, demonstrating the near-optimality of arithmetic coding.
Moreover, it is clearly possible to create an adaptive arithmetic code by
changing the estimate of $p_0$ based on previously coded data.

\subsubsection{Finite precision arithmetic}
As it stands, the procedure outlined in the previous section has a number of
drawbacks for practical application. Firstly, it requires unlimited precision
to scale the interval, which is not available in real hardware or software.
Secondly, it only produces an output when all values have been coded. These
problems are addressed by renormalisation and progressive output: periodically rescaling the
interval, and outputting the most significant bits of $low$ and $high$ whenever they agree.

For example, if we know that $low=b0xyz...$ and $high=b0pqr...$ then we can
output $0$, since this must prefix any value lying in the interval, 
and shift $low$ and $high$ to get $low=bxyz...$ and $high=bpqr...$.
This has the effect of doubling the interval from 0 ($x\mapsto 2x$). Likewise
if $low=b1xyz...$ and $high=b1pqr...$ we can output $1$ and shift to get
$low=bxyz...$ and $high=bpqr...$ again: this is equivalent to doubling the interval
from 1 ($x\mapsto 2x-1$).

One problem remains: suppose the interval resolutely sits on the fence, straddling
1/2 whilst getting smaller and smaller, with the most 
significant bits of low and high staying as 0 and 1 respectively. 
In this case, when the straddle is finally resolved, $low$ and $high$ will
both be of the form $b10000...xyz$ or $b01111...pqr$. 

The resolution strategy is to again rescale $low$ and $high$, but this time
double from 1/2 ($x\mapsto 2x-\frac{1}{2}$), and keep a count of the number $k$
of times this is done, as this is the number of carry bits that are
required. When the straddle is resolved as 1, then 1 followed by $k$ zero bits is 
output, otherwise 0 followed by k 1s is output. This ensures that the
output exactly represents the small straddling interval.

A decoder can determine a symbol as soon as it has sufficient bits to distinguish
whether a value lies in one interval or another. If constraints are placed on the
 size of the smallest interval before
renormalisation (for example, by renormalising often enough and by having a fixed
smallest allowable probability), then this can be accomplished within a fixed word width.

\end{informative*}
\begin{informative*}
\subsection{Arithmetic encoding (Informative)}

This document only normatively specifies the decoding of arithmetic coded data. 
However whilst it is clearly vital that an encoding process matches the decoding
process, it is not entirely straightforward to derive an implementation of the
encoder by only looking only at the decoder specification. Therefore this
informative section describes a possible implementation for an
arithmetic encoder that will produce output decodeable by
the Dirac arithmetic decoder. This section should be read in conjunction with
Section \ref{arithengine}.

\subsubsection{Encoder variables}

An arithmetic encoder would require the following unsigned integer variables:
\begin{itemize}
\item $low$, a value indicating the bottom of the encoding interval
\item $range$, a value indicating the width of the encoding interval
\item $carry$, a value tracking the number of unresolved ``straddle" conditions 
(described below)
\item a set of 16-bit probability contexts, as described in Section \ref{arithdecoding}
\end{itemize}

Boolean values are encoded using an estimate $prob0$ of the probability of zero
for that value: $prob0$ is 16 bits, i.e. the probability of zero is $prob0/2^{16}$. 
The process for updating this probability is described in Section \ref{contextupdate}

A Dirac binary arithmetic encoder implementation will code a set of data in three stages:
\begin{enumerate}
\item Initialisation
\item Processing of all values
\item Flushing
\end{enumerate}

\subsubsection{Initialisation}

Initialisation of the arithmetic encoder is very simple -- the internal variables are
set as:
\begin{eqnarray*}
low&=&\text{0x0} \\
range&=&\text{0x10000} \\
carry&=&0
\end{eqnarray*}

With 16 bit accuracy, $0x10000$ corresponds to an interval width value of 1. All
context probabilities are initialised to probability $1/2$ (0x8000).

\subsubsection{Encoding binary values}
\label{arithwritebool}
The encoding process for a binary value must precisely mirror
that for the decoding process (Section \ref{arithreadbool}), in
particular the interval variables $low$ and $range$ must be
updated in the same way.

Coding a boolean value consists of three sub-stages (in order): 
\begin{enumerate}
\item{scaling the interval $[low,low+range)$}
\item{updating contexts}
\item{renormalising and outputting data}
\end{enumerate}

\paragraph*{Scaling the interval\\}
The integer interval $[low,low+range)$ represents the real interval
$[l,h)=[low/2^{16},(low+range)/2^{16})$. In a given context with index $i$,
the probability of zero can be extracted as 
\[prob\_zero=\AContexts[i][prob0]\]

If $0$ is to be encoded, the real interval $[l,h)$
should be rescaled so that $l$ is unchanged and the
width $r=h-l=range/2^{16}$ is scaled to $r*p0$ where $p0=prob\_zero/2^{16}$.
This operation is approximated by setting
\[range=(range*prob\_zero)>>16\]

If 1 is to be encoded, $[l,h)$ should be rescaled so that $h$ is
unchanged and $r$ is scaled to $(1-p0)*r$. This operation is
approximated by setting
\begin{eqnarray*}
range & -= & (range*prob\_zero)>>16 \\
low &+= & (range*prob\_zero)>>16 
\end{eqnarray*}

\paragraph*{Updating contexts\\}
Contexts are updated in exactly the same way as the 
decoder (Section \ref{contextupdate}).


\paragraph*{Renormalisation and output\\}
Renormalisation must cause $low$ and $range$ to be modified exactly
as in the decoder (Section \ref{renormalisation}). In addition, 
during renormalisation bits are output when $low$ and $low+range$ 
agree in their msbs, taking into account carries accumulated when a
straddle condition is accumulated. 
In pseudocode, this is as follows:

\begin{pseudo*}
\bsWHILE{range<=\text{0x4000}}
    \bsIF{(low+range-1)\wedge low>=\text{0x8000}}
        \bsCODE{low \wedge= \text{0x4000}}
        \bsCODE{carry+=1}
    \bsELSE
        \bsCODE{write\_bit( low\&\text{0x8000} )}
        \bsWHILE{ carry>0}
            \bsCODE{write\_bit( ~(low\&\text{0x8000}) )}
            \bsCODE{carry -= 1}
        \bsEND
    \bsEND
    \bsCODE{low  <<=  1}
    \bsCODE{range  <<=  1}
    \bsCODE{low \&= \text{0xFFFF}}
\bsEND
\end{pseudo*}

\paragraph*{Flushing the encoder\\}
After encoding, there may still be insufficient bits for a decoder
to determine the final few encoded symbols, partly because further 
renormalisation is required -- for example, msbs may agree but the range
may still be larger than 0x4000) -- and partly because there may be 
unresolved carries.

A four-stage process will adequately flush the encoder:
\begin{enumerate}
\item{output remaining resolved msbs}
\item{resolve remaining straddle conditions}
\item{flush carry bits}
\item{byte align the output with padding bits}
\end{enumerate}

The remaining msbs are output as follows:

\begin{pseudo*}
\bsWHILE{range<=\text{0x8000}}
    \bsIF{(low+range-1)\wedge low<\text{0x8000}}
        \bsCODE{write\_bit( low\&\text{0x8000} )}
        \bsWHILE{ carry>0}
            \bsCODE{write\_bit( ~(low\&\text{0x8000}) )}
            \bsCODE{carry -= 1}
        \bsEND
    \bsEND
    \bsCODE{low  <<=  1}
    \bsCODE{range  <<=  1}
    \bsCODE{low \&= \text{0xFFFF}}
\bsEND
\end{pseudo*}

Note that this renormalisation is invoked if the range is less than
or equal to 1/2 the total range. Remaining straddles can then
be resolved in the same way:

\begin{pseudo*}
\bsWHILE{range<=\text{0x8000}}
    \bsIF{(low+range-1)\wedge low>=\text{0x8000}}
        \bsCODE{low \wedge= \text{0x4000}}
        \bsCODE{carry+=1}
    \bsEND
    \bsCODE{low  <<=  1}
    \bsCODE{range  <<=  1}
    \bsCODE{low \&= \text{0xFFFF}}
\bsEND
\end{pseudo*}

Carry bits can be discharged by picking a resolution of
the final straddles:

\begin{pseudo*}
\bsCODE{write\_bit(m_low_code \& \text{0x4000})}
\bsWHILE{carry >= 0}
    \bsCODE{write\_bit(~(low \& \text{0x4000}))}
    \bsCODE{carry-=1}
\bsEND
\end{pseudo*}

Finally, 0-7 padding bits are added to the encoded output to make
a whole number of bytes. These are not necessary for decoding, but
for stream compliance.

\end{informative*}
\subsection{Arithmetic decoding engine}
\label{arithengine}

This section is a normative specification of the operation of the arithmetic
decoding engine and the processes for using it to extract binary values from coded streams.

The arithmetic decoding engine consists of two elements: 
\begin{itemize}
\item a collection of state variables representing the state of the arithmetic 
decoder (Section \ref{initarith})
\item a function for extracting binary values from the decoder 
and updating the decoder state (Section \ref{arithreadbool})
\end{itemize}

\subsubsection{State and contexts}
\label{arithcontexts}

The arithmetic decoder state consists of the following decoder state variables:

\begin{itemize}
\item $\ALow$, an integer representing the beginning of the current coding interval
\item $\ARange$, an integer representing the size of the current coding interval
\item $\ACode$, an integer within the interval from $\ALow$ to $\AHigh$, determined from the encoded bitstream
\item $\ABitsLeft$, a decrementing count of the number of bits yet to be read in
\item $\AContexts$, a map of all the contexts used in the Dirac decoder
\end{itemize}

A context $context$ is an integer array with a single value which encapsulates
the probability of zero in that context represented as a 16 bit number, such that
\[0<context[prob0]<\text{0xFFFF}\]

Contexts are accessed by decoding functions via the indices defined in Section \ref{contextindices}. 

\subsubsection{Initialisation}
\label{initarith}

At the beginning of the decoding of any data unit, the arithmetic
decoding state is initialised as follows:

\begin{pseudo}{initialise\_arithmetic\_decoding}{block\_data\_length}
\bsCODE{\ABitsLeft=8*block\_data\_length}
\bsCODE{\ALow = \text{0x0}}
\bsCODE{\ARange =\text{0x10000}}
\bsCODE{\ACode =\text{ 0x0}}
\bsFOR{i=0}{15}
    \bsCODE{\ACode <<= 1}
    \bsCODE{\ACode+= read\_bitb()} 
\bsEND
\bsCODE{init\_contexts()}
\end{pseudo}

Contexts are initialised by the $init\_contexts()$ function as follows:

\begin{pseudo}{init\_contexts}{}
\bsFOR{i=0}{\length(\AContexts)-1}
  \bsCODE{\AContexts[i][prob0]=\text{0x8000}}
\bsEND
\end{pseudo}

\subsubsection{Data input}
\label{inputarith}

The arithmetic decoding process accesses data in a contiguous block of bytes
whose size is set on initialisation (Section \ref{initarith}). The bits in this
block are sufficient to allow for the
decoding of all coefficients. However, the specification of arithmetic
decoding operations in this section may occasionally cause further bits to be read,
even though they are not required for determining decoded values. For this
reason the bounded-block read function $read\_bitb()$ (Section \ref{blockreadbit}) is 
used for data access.

Since the length of arithmetically coded data elements is given in bytes within the Dirac
stream, there may be bits left unread when all values have been extracted. These
are flushed as desribed in Section \ref{blockreadbit}. Since arithmetically coded blocks
are byte-aligned and a whole number of bytes, this aligns data input with the beginning of the byte 
after the arithmetically coded data i.e. at the end of the
data chunk. $flush\_inputb()$ is always called at the end of decoding an arithmetically encoded
data element.

\begin{informative}
The Dirac arithmetic decoding engine uses 16 bit words, and so if all coefficients are
coded no more than 16 additional bits should be read beyond the end of the block. Hence it 
would seem sufficient to read in the entire block of data and pad the end with two bytes of value 0xFF,
to avoid a branch condition on inputting data
However, an arithmetically encoded block may end with a string of 1s, which an encoder could
conceivably strip out to save bits, in the knowledge that $read\_bitb()$ will re-insert them. Terminating
strings of 1s can occur (but not exclusively) in coding many zero wavelet subband coefficients at the end
of a subband. So a larger number of pad bytes may be required in practice.
\end{informative}

\subsubsection{Decoding boolean values}
\label{arithreadbool}

The arithmetic decoding engine is a multi-context, adaptive binary
arithmetic decoder, performing binary renormalisation and producing
binary outputs. For each bit decoded, the semantics of the relevant
calling decoder function determine which contexts are passed to the
arithmetic decoding operations. 

This section specifies the operation of the $read\_boola()$ function
for extracting a boolean value from the Dirac stream. The overall decoding
process for extracting a symbol is as defined by the following
pseudocode:

\begin{pseudo}{read\_boola}{context\_index}
\bsCODE{prob\_zero=\AContexts[context\_index][prob0]}
\bsCODE{count = \ACode-\ALow}
\bsCODE{range\_times\_prob = (\ARange * prob\_zero)\gg 16}
\bsIF{ count >= range\_times\_prob }
  \bsCODE{value = \true}
  \bsCODE{\ALow +=  range\_times\_prob}
  \bsCODE{\ARange -= range\_time\_prob}
\bsELSE
  \bsCODE{value = \false}
  \bsCODE{\ARange = range\_times\_prob}
\bsEND
\bsCODE{update\_context(\AContexts[context\_index],value)}{\ref{contextupdate}}
\bsWHILE{\ARange<=\text{0x4000}}
    \bsCODE{renormalise()}{\ref{renormalisation}}
\bsEND
\bsRET{value}
\end{pseudo}

\begin{informative}
The function scales the probability of the symbol $0$ from the decoding context
so that if this probability were $1$, then the interval would equal that between
 $\ALow$ and 
 \[high=\ALow+\ARange-1\]
and $count$ is set to the normalised cut-off between 0 and 1 within this range.
\end{informative}

\subsubsection{Renormalisation}
\label{renormalisation}

Renormalisation is applied to stop the arithmetic decoding 
engine from losing accuracy: the range must not get too small,
 in order that 0 and 1 may be distinguished. Renormalisation is
 applied while the range is less than or equal to a quarter of 
 the total available 16-bit range ($\text{0x4000}$). 

For convenience let $low=\ALow$ and $high=\ALow+\ARange-1$ 
represent the upper and lower bounds of the interval. If the
range is $<=\text{0x4000}$ then
one of three possibilities must obtain:
\begin{enumerate}
\item the msbs of $low$ and $high$ are both 0
\item the msbs of $low$ and $high$ are both 1
\item $low=b01...$, $high=b10....$,  and the interval straddles the half-way point 0x8000. 
\end{enumerate}

Renormalisation doubles the interval and reads a bit into the codeword
as follows:

\begin{pseudo}{renormalise}{}
\bsIF{(\ALow+\ARange-1)\wedge\ALow>=\text{0x8000}}
    \bsCODE{\ACode \wedge= \text{0x4000}}
    \bsCODE{\ALow \wedge= \text{0x4000}}
\bsEND
\bsCODE{\ALow  <<=  1}
\bsCODE{\ARange  <<=  1}
\bsCODE{\ALow \&= \text{0xFFFF}}
\bsCODE{\ACode <<= 1}
\bsCODE{\ACode+= read\_bitb()}
\bsCODE{\ACode \&= \text{0xFFFF}}
\end{pseudo}

The second bit (0x4000) is flipped if there is a straddle condition (case 3). The renormalisation
process has the effect that: in case 1, the interval $[low,high]$ is doubled from 0 ($x\mapsto 2*x$); 
in case 2 it is doubled from 1 ($x\mapsto 2*x-1$); and in case 3 it is doubled from 1/2 ($x\mapsto 2x-0.5$).
 
\begin{informative}
Note that if 16-bit words (unsigned shorts) are used for decoder state variables $\ALow$,
 and $\ACode$ then there is no need for {\&}-ing with 0xFFFF. However, the 
operations specified here are defined in terms of integers, since intermediate calculations
 require higher dynamic range. In software, the efficiency of using short word lengths may
or may not be offset by the requirement to cast to other data types for these calculations.
\end{informative}

\subsubsection{Updating contexts}
\label{contextupdate}

Contexts are updated according to a probability look-up table
$\ALUT$ (Table \ref{table:lut}), which supplies a value for decrementing
or incrementing the probability of zero based on the first 
8 bits of its current value, according to Table \ref{table:lut}.

\begin{pseudo}{update\_context}{ctx,value}
\bsIF{value==\true}
    \bsCODE{ctx[prob0] -= \ALUT[ctx[prob0]>>8]}{Table \ref{table:lut}}
\bsELSE
    \bsCODE{ctx[prob0] += \ALUT[255-(ctx[prob0]>>8)]}{Table \ref{table:lut}}
\bsEND
\end{pseudo}


\begin{table}[!ht]
\begin{tabular}{|cccccccc|}
\hline
\multicolumn{8}{|c|}{{\bf \ALUT[] (indexes 0 to 255)}} \\
\hline
             0,&    2,&    5,&    8,&   11,&   15,&   20,&   24,\\
            29,&   35,&   41,&   47,&   53,&   60,&   67,&   74,\\
            82,&   89,&   97,&  106,&  114,&  123,&  132,&  141,\\
           150,&  160,&  170,&  180,&  190,&  201,&  211,&  222,\\
           233,&  244,&  256,&  267,&  279,&  291,&  303,&  315,\\
           327,&  340,&  353,&  366,&  379,&  392,&  405,&  419,\\
           433,&  447,&  461,&  475,&  489,&  504,&  518,&  533,\\
           548,&  563,&  578,&  593,&  609,&  624,&  640,&  656,\\
           672,&  688,&  705,&  721,&  738,&  754,&  771,&  788,\\
           805,&  822,&  840,&  857,&  875,&  892,&  910,&  928,\\
           946,&  964,&  983,& 1001,& 1020,& 1038,& 1057,& 1076,\\
          1095,& 1114,& 1133,& 1153,& 1172,& 1192,& 1211,& 1231,\\
          1251,& 1271,& 1291,& 1311,& 1332,& 1352,& 1373,& 1393,\\
          1414,& 1435,& 1456,& 1477,& 1498,& 1520,& 1541,& 1562,\\
          1584,& 1606,& 1628,& 1649,& 1671,& 1694,& 1716,& 1738,\\
          1760,& 1783,& 1806,& 1828,& 1851,& 1874,& 1897,& 1920,\\
          1935,& 1942,& 1949,& 1955,& 1961,& 1968,& 1974,& 1980,\\
          1985,& 1991,& 1996,& 2001,& 2006,& 2011,& 2016,& 2021,\\
          2025,& 2029,& 2033,& 2037,& 2040,& 2044,& 2047,& 2050,\\
          2053,& 2056,& 2058,& 2061,& 2063,& 2065,& 2066,& 2068,\\
          2069,& 2070,& 2071,& 2072,& 2072,& 2072,& 2072,& 2072,\\
          2072,& 2071,& 2070,& 2069,& 2068,& 2066,& 2065,& 2063,\\
          2060,& 2058,& 2055,& 2052,& 2049,& 2045,& 2042,& 2038,\\
          2033,& 2029,& 2024,& 2019,& 2013,& 2008,& 2002,& 1996,\\
          1989,& 1982,& 1975,& 1968,& 1960,& 1952,& 1943,& 1934,\\
          1925,& 1916,& 1906,& 1896,& 1885,& 1874,& 1863,& 1851,\\
          1839,& 1827,& 1814,& 1800,& 1786,& 1772,& 1757,& 1742,\\
          1727,& 1710,& 1694,& 1676,& 1659,& 1640,& 1622,& 1602,\\
          1582,& 1561,& 1540,& 1518,& 1495,& 1471,& 1447,& 1422,\\
          1396,& 1369,& 1341,& 1312,& 1282,& 1251,& 1219,& 1186,\\
          1151,& 1114,& 1077,& 1037,&  995,&  952,&  906,&  857,\\
           805,&  750,&  690,&  625,&  553,&  471,&  376,&  255\\
\hline
\end{tabular}
\caption{Look-up table for context probability adaptation}
\label{table:lut}
\end{table}

\begin{informative}
The look-up table approximates a count-based adaption mechanism. In
a count-based system, a context would maintain a weight $w$ and a
count of zeroes $n_0$, such that the probability of zero would
be
\[prob0=\dfrac{n_0}{w}\]
If 0 is then coded, this probability becomes $\dfrac{n_0+1}{w+1}$,
i.e. and if 1 is coded, $prob0=\dfrac{n_0}{w+1}$. In other words,
\[prob0+=\left(\dfrac{1-prob0}{w+1}\right)\]
if a 0 is coded, and 
\[prob0-=\left(\dfrac{prob0}{w+1}\right)\]
if a 1 is coded.
The LUT updates $prob0$ by implicitly defining $w$ as a function of
$prob0$ in the formulae above. The values have been chosen so that if $prob0$ is close
to 0, $w=256$, and if $prob0$ is close to $0.5$, $w=32$. This allows
for fast initial adaptation and good representation of highly biased
probabilities. 

Note that a single 512-element LUT taking $value$ as an argument can
avoid the branch in $update\_contexts()$.
\end{informative}

\begin{comment}
\subsection{Alternative arithmetic decoding engines}

to
16 consisting of three positive values:
\begin{itemize}
\item $context[count0]$ is a count of the number of zeroes
\item $context[count1]$ is a count of the number of ones
\item $context[prob0]$ is an estimate of the probability of zero to 16 bit accuracy
\end{itemize}

Contexts are accessed by decoding functions
via the indices defined in Section \ref{contextindices}. 

Although counts are updated with each symbol decoded, the probability is only updated occasionally, as it is computationally
expensive (see Section \ref{rescalecontext} below).

\subsubsection{Rescaling contexts and probabilities}
\label{rescalecontext}

Contexts maintain counts to 8 bit accuracy, and contexts are rescaled when the total count (count0+count1) reaches 256. In addtion, prob0 is recalculated every
8 symbols in each context. A context is rescaled by halving the counts of $0$ and $1$.

\begin{pseudo}{update\_context}{context,value}
\bsIF{value==\true}
    \bsCODE{context[count1] += 1}
\bsELSE
    \bsCODE{context[count0] += 1}
\bsEND
\bsIF( (context[count0]+context[count1])\%8==0)
    \bsIF{context[0]+context[1]== 256}
      \bsCODE{context[0] += 1}
      \bsCODE{context[0] \gg= 1}
      \bsCODE{context[1] += 1}
      \bsCODE{context[1] \gg= 1}
    \bsEND
    \bsCODE{calc\_prob0(context)}
\bsEND
\end{pseudo}

The probability of zero is recalculated to approximate
\[ \frac{context[count0]*2^{16}}{context[count0]+context[count1]}\]
to 16 bit accuracy:

\begin{pseudo}{calc\_prob0}{context}
\bsCODE{weight = context[count0]+context[count1]}
\bsCODE{inverse\_weight=(2^{16}+(weight//2))//weight}
\bsCODE{context[prob0]=context[count0]*inverse\_weight}
\end{pseudo}

\begin{informative}
Note that since $context[count0]<weight$, $context[prob0]$ is always a 16 bit unsigned quantity.
The inverse weight may easily be stored within a look-up table.
\end{informative}

\subsection{Initialisation}
\label{initarith}

At the beginning of the decoding of any data unit, the arithmetic
decoding state is initialised as follows:

\begin{pseudo}{initialise\_arithmetic\_decoding}{block\_data\_length}
\bsCODE{\ABitsLeft=8*block\_data\_length}
\bsCODE{\ALow = \text{0x0}}
\bsCODE{\ARange =\text{0x10000}}
\bsCODE{\ACode =\text{ 0x0}}
\bsFOR{i=0}{15}
    \bsCODE{\ACode <<= 1}
    \bsCODE{\ACode+= read\_bitb()} 
\bsEND
\bsCODE{init\_contexts()}
\end{pseudo}

Contexts are initialised by the $init\_contexts()$ function as follows:

\begin{pseudo}{init\_contexts}{}
\bsFOR{i=0}{length(\AContexts)-1}
  \bsCODE{\AContexts[i][count0]=1}
  \bsCODE{\AContexts[i][count1]=1}
  \bsCODE{\AContexts[i][prob0]=0x8000}
\bsEND
\end{pseudo}

\subsection{Data input}
\label{inputarith}

The arithmetic decoding process accesses data in a contiguous block of bytes
whose size is set on initialisation (Section \ref{initarith}). The bits in this
block are sufficient to allow for the
decoding of all coefficients. However, the specification of arithmetic
decoding operations in this section may occasionally cause further bits to be read,
even though they are not required for determining decoded values. For this
reason the bounded-block read function $read\_bitb()$ (Section \ref{blockreadbit}) is 
used for data access.

Since the length of arithmetically coded data elements is given in bytes within the Dirac
stream, there may be bits left unread when all values have been extracted. These
are flushed as desribed in Section \ref{blockreadbit}. Since arithmetically coded blocks
are byte-aligned and a whole number of bytes, this aligns data input with the beginning of the byte 
after the arithmetically coded data i.e. at the end of the
data chunk. $flush\_inputb()$ is always called at the end of decoding an arithmetically encoded
data element.

\begin{informative}
The Dirac arithmetic decoding engine uses 16 bit words, and so if all coefficients are
coded no more than 16 additional bits should be read beyond the end of the block. Hence it 
would seem sufficient to read in the entire block of data and pad the end with two bytes of value 0xFF,
to avoid a branch condition on inputting data
However, an arithmetically encoded block may end with a string of 1s, which an encoder could
conceivably strip out to save bits, in the knowledge that $read\_bitb()$ will re-insert them. Terminating
strings of 1s can occur (but not exclusively) in coding many zero wavelet subband coefficients at the end
of a subband. So a much larger number of pad bytes may be required in practice.
\end{informative}

\subsection{Decoder functions}
\label{extractarith}
The arithmetic decoding engine is a multi-context, adaptive binary
arithmetic decoder, performing binary renormalisation and producing
binary outputs. For each bit decoded, the semantics of the relevant
calling decoder function determine which contexts are passed to the
arithmetic decoding operations.

\subsubsection{Decoding boolean values}

\label{arithreadbool}

This section specifies the operation of the $read\_boola()$ function
for extracting a boolean value from the Dirac stream. The overall decoding
process is defined for extracting a symbol is as defined by the following
pseudocode:

\begin{pseudo}{read\_boola}{context\_index}
\bsCODE{context=\AContexts[context\_index]}
\bsCODE{count = \ACode-\ALow+1}
\bsCODE{range\_times\_prob = (\ARange * context[prob0])\gg 16}
\bsIF{ count > range\_times\_prob }
  \bsCODE{value = \true}
  \bsCODE{\ALow +=  range\_times\_prob}
  \bsCODE{\ARange -= range\_time\_prob}
\bsELSE
  \bsCODE{value = \false}
  \bsCODE{\ARange = range\_times\_prob}
\bsEND
\bsCODE{update\_context(\AContexts[context\_index],value)}
\bsCODE{renormalise()}{\ref{renormalisation}}
\bsEND
\bsRET(value)
\end{pseudo}

\begin{informative}
[Describe what's going on here]
\end{informative}

\subsubsection{Renormalisation}
\label{renormalisation}

Renormalisation is applied to stop the arithmetic decoding engine from losing accuracy: the range
must not get too small to allow 0 and 1 to be distinguished. Renormalisation is applied while the
range is less than or equal to a quarter of the total available 16-bit  range:

\begin{pseudo}{renormalise}{}
\bsWHILE{\ARange<=\text{0x4000}}
    \bsIF{(\ALow+\ARange-1)^\ALow>=\text{0x8000}}
        \bsCODE{resolve\_straddle()}
    \bsEND
    \bsCODE{shift\_bits()}
\bsEND
\end{pseudo}

\begin{informative}
Let the bottom of the arithmetic coding interval is represented by $low=\ALow$ and the top by $high=\ALow+\ARange-1$.
When the range is one quarter or less of the original range ($2^{16}$), then one of three possibilities occurs:
\begin{enumerate}
\item the top bits of $low$ and $high$ are both 0
\item the top bits of $low$ and $high$ are both 1
\item $low=b01...$, $high=b10....$,  and the interval straddles the half-way point 0x8000. 
\end{enumerate}

In all of these cases the interval can be doubled in size to increase accuracy. In the first case, it is doubled from zero ($x\mapsto 2*x$); 
in the second it is doubled from 1 ($x\mapsto 2*x-1$); and in the third it is doubled from 1/2 ($x\mapsto 2x-0.5$).
 
\end{informative}

\begin{pseudo}{resolve\_staddle}{}
\bsCODE{\ACode ^= \text{0x4000}}
\bsCODE{\ALow ^= \text{0x4000}}
\end{pseudo}

\begin{pseudo}{shift\_bits}{}
\bsCODE{\ALow  <<=  1}
\bsCODE{\ARange  <<=  1}
\bsCODE{\ALow \&= \text{0xFFFF}}
\bsCODE{\ACode <<= 1}
\bsCODE{\ACode+= read\_bitb()}
\bsCODE{\ACode \&= \text{0xFFFF}}
\end{pseudo}

\begin{comment}

\begin{informative}
The function scales the probability of the symbol $0$ from the decoding context
so that if this probability were $1$, then the interval would equal that between
 $\ALow$ and $\AHigh$. If $\ACode$ is greater than this cut-off, then 1 ($\true$) has
been encoded, else 0 ($\false$) has.
\end{informative}

\subsubsection{Shifting bits in}

\label{arithshiftin}

This section defines the operation of the $shift\_bit\_in()$ 
and $shift\_all\_bits()$ functions
for reading bits into the arithmetic decoding state variables.

\begin{pseudo}{shift\_bit\_in}{}
\bsCODE{\AHigh \ll= 1}
\bsCODE{\AHigh \&= \text{0xFFFF}}
\bsCODE{\AHigh += 1}
\bsCODE{\ALow \ll= 1}
\bsCODE{\ALow \&= \text{0xFFFF}}
\bsCODE{\ACode \ll= 1}
\bsCODE{\ACode \&= \text{0xFFFF}}
\bsCODE{\ACode += read\_bitb()}{\ref{blockreadbit}}
\end{pseudo}

$shift\_all\_bits()$ expands the interval between $\ALow$ and $\AHigh$
until the msbs (bit 15) differ and the interval no longer
straddles the half-way point 0x8000.

\begin{pseudo}{shift\_all\_bits}{}
\bsWHILE{ \AHigh\&\text{0x8000})==\text{0x0} \text{ and } (\ALow\&\text{0x8000})==\text{0x0}}
  \bsCODE{shift\_bit\_in()}
\bsEND
\bsWHILE{ (\AHigh\&\text{0x4000})==\text{0x0} \text{ and } (\ALow\&\text{0x4000})==\text{0x4000} }
  \bsCODE{\ACode \wedge= \text{0x4000}}
  \bsCODE{\AHigh \wedge= \text{0x4000}}
  \bsCODE{\ALow \wedge= \text{0x4000}}
  \bsCODE{shift\_bit\_in()}
\bsEND
\end{pseudo}

\begin{informative}
Note that if 16-bit words (unsigned shorts) are used for decoder state variables $\ALow$,
 $\AHigh$ and $\ACode$ then there is no need for {\&}-ing with 0xFFFF. However, the 
operations specified here are defined in terms of integers, since intermediate calculations
 require higher dynamic range. In software, the efficiency of using short word lengths may
or may not be offset by the requirement to cast to other data types for these calculations.
\end{informative}



\end{comment}
\subsection{Arithmetic decoding engine}
Global Variables used for arithmetic decoding
\begin{verbatim}
bytes_left      #integer
bits_left       #integer
bit_buffer      #integer
low             #integer
high            #integer
code            #integer
\end{verbatim}

\begin{informative}
The use of global variables for implementation code is
considered poor programming practice. It should be remembered that this
document is a specification and not implementation code. Global
variables are used in this specification for two reasons. Firstly they
avoid the need to pass the arithmetic coding state to each function that
reads arithmetic coded data, thereby (we hope) clarifying the specification.
Secondly a specific implementation is likely to associate these
variables with some sort of output stream object. For example in a
software implementation written in C++ these variables may be associated
with an output stream in which output is implemented using custom stream
manipulators. Data output is highly implementation specific. The use of
global variables is a way of specifying output in a generic way. Their
use should not be taken to indicate that they should (or should not) be
used in any specific implementation.
\end{informative}

\subsection{Data input}\label{inputarith}

The arithmetic decoding process accesses data in a contiguous block of bytes
whose size is set on initialisation (Section \ref{initarith}). The bits in this
block are sufficient to allow for the
decoding of all coefficients. However, the specification of arithmetic
decoding operations in this section may occasionally cause further bits to be read,
even though they are not required for determining decoded values. For this
reason a read function $read\_bita()$ is defined which returns $0$ if the
bounds of this block of data have been exceeded:

\begin{pseudo}{read\_bita}{}
\bsIF{\ABitsLeft==0}
  \bsRET{0}
\bsELSE
  \bsCODE{\ABitsLeft -= 1}
  \bsRET{read\_bit()}
\bsEND
\end{pseudo}

\begin{informative}
The Dirac arithmetic decoding engine uses 16 bit words, and so no more than 16
additional bits can be read beyond the end of the block. Hence it is sufficient
to read in the entire block of data and pad the end with two zero bytes to
avoid a branch condition with each input bit.
\end{informative}

\subsection{Initialisation}\label{initarith}

At the beginning of the decoding of any data unit, the arithmetic
decoding state is initialised as follows:

\begin{pseudo}{initialise\_arithmetic\_decoding}{(block\_data\_length}
\bsCODE{\ABitsLeft=8*block\_data\_length}
\bsCODE{\ALow = 0x0000}
\bsCODE{\AHigh = 0x0000}
\bsCODE{\ACode = 0x0000}
\bsCODE{init\_contexts()}
\end{pseudo}

Contexts are initialised by the $init\_contexts()$ function as follows:
\begin{pseudo}{init\_contexts}{}
\bsFOR{i=0}{len(\AContexts)-1}
  \bsCODE{\AContexts[i][0]=1}
  \bsCODE{\AContexts[i][1]=1}
\bsEND
\end{pseudo}

\begin{informative}
[Move to a general section on spec methodology??]
The use of global variables for implementation code is
considered poor programming practice. It should be remembered that this
document is a specification and not implementation code. Global
variables are used in this specification for two reasons. Firstly they
avoid the need to pass the arithmetic coding state to each function that
reads arithmetic coded data, thereby (we hope) clarifying the specification.
Secondly a specific implementation is likely to associate these
variables with some sort of output stream object. For example in a
software implementation written in C++ these variables may be associated
with an output stream in which output is implemented using custom stream
manipulators. Data output is highly implementation specific. The use of
global variables is a way of specifying output in a generic way. Their
use should not be taken to indicate that they should (or should not) be
used in any specific implementation.
\end{informative}

\subsection{Contexts}\label{arithcontexts}

The global arithmetic decoder state consists of the following variables:

\begin{pseudo*}
\bsCODE{\ArithState:}
\bsCODE{\ALow}{\#integer}
\bsCODE{\AHigh}{\#integer}
\bsCODE{\ACode}{\#integer}
\bsCODE{\ABitsLeft}{\#integer}
\bsCODE{\AContexts[]}{\#context array}
\end{pseudo*}

The state variable \AContexts is an array of all contexts
used for entropy decoding in the Dirac decoder. 
A context $context$ is an integer array consisting of two positive values,
$context[0]$, and $context[1]$, representing counts of values $0$ and $1$
respectively. Contexts are accessed by decoding functions
via the indices defined in Section ??.

\subsubsection{Resetting contexts}

An individual context is reset by halving the counts of $0$ and $1$ and ensuring that
these counts do not reach zero:

\begin{pseudo}{reset\_context}{context}
\bsCODE{context[0] \gg= 1}
\bsCODE{context[0] += 1}
\bsCODE{context[1] \gg= 1}
\bsCODE{context[1] += 1}
\end{pseudo}

Periodic resetting of all contexts occurs in entropy decoding 

\subsection{Decoder functions}
%src: 0.9

The arithmetic decoding engine is a multi-context, adaptive binary
arithmetic decoder, performing binary renormalisation and producing
binary outputs. For each bit decoded, the semantics of the relevant
decoder function determine a statistical context (probability model) to
be used for the internal rescaling functions. Assuming this context, the
raw arithmetic decoding function is written

\begin{comment}
binary\_arith\_decode()

Its operation is defined in Section .

Signed and unsigned integer outputs can be derived by the use of
binarisation schemes which turn a symbol into a string of bits. There
are two basic binarisations employed for arithmetic coding: unary and
truncated unary binarisation. These are defined in Appendix , giving
rise to four decoder functions 

Unsigned unary arithmetic decoding: uu\_arith\_decode()

Signed unary arithmetic decoding:   su\_arith\_decode()

Unsigned truncated unary arithmetic decoding:   ut\_arith\_decode()

In addition a decoding process may invoke halve\_all\_counts() to reset
context statistics.

\begin{informative}
There is no signed truncated unary arithmetic decoding function in
Dirac, since truncated unary values have been derived from modulo
arithmetic and have no sign.
\end{informative}
\end{comment}

\subsubsection{Unsigned unary arithmetic decoding}\begin{comment}
Pseudo-code for unsigned unary arithmetic decoding uu\_arith\_decode() is
as follows:

VALUE=0

while ( !binary\_arith\_decode( choose\_context() ) )

    VALUE++

choose\_context() is a function that produces a context with which the
binary bit shall be decoded. The value it returns can depend on any
values known to the decoder at the time it is called, especially
including the binarisation bin (the bin number is equal to VALUE+1
according to the conventions of Appendix).
\end{comment}

%src: tim 0.9.1.48

reads and returns an unsigned integer encoded in the bytestream as an
arithmetic coded unary binarisation. ``context\_list'' is a list of
contexts for each bin. If the number of contexts in the list is less
than the bin number then the last context on the list is used. In other
words, a
common context is used for all the higher bins.

Read Arithmetic Coded Unsigned Integer
\begin{verbatim}
read_uua(context_list):

    value = 0
    context_index = 0 #Bin Number (numbered from zero)
    max_index = len(context_list) - 1
    # Read first bit
    context = context_list[context_index]
    if ( read_ba(context) ):
        more = False
    else:
        value += 1
        more = True
    context_list[context_index] = context
    # Read remaining bits
    while ( more ):
        if ( context_index < max_index):
            context_index += 1
        context = context_list[context_index]

        if ( read_ba(context) ):
            more = False
        else:
            value += 1
            more = True
        context_list[context_index] = context
    return value
\end{verbatim}

\subsubsection{Signed unary arithmetic decoding}\begin{comment}
Signed unary arithmetic decoding su\_arith\_decode() has the following
pseudocode representation:

VALUE= uu\_arith\_decode()

if ( VALUE!=0 )

{

    if ( !binary\_arith\_decode( choose\_context() ) )

    VALUE=-VALUE

}

\end{comment}

%src: tim-0.9.1.48

reads and returns a signed integer encoded in the bytestream as an
arithmetic coded truncated unary binarisation.  ``context\_list'' is a two
element list. The second element is the context for the sign bit. The
first element is a  context list for reading the magitude of the signed
integer.  The magnitude context list is a list of contexts for each bin.
If the number of contexts in the magnitude context list is less than the
bin number then the last context on the list is used. That is a common
context is used for all the higher bins.

Read Arithmetic Coded Signed Integer
\begin{verbatim}
read_sua(context_list):
    #Read magnitude
    magnitude_context_list = context_list[0]
    magnitude = read_uua(magnitude_context_list)
    context_list[0] = magnitude_context_list
    if ( magnitude==0 ):
        value = 0
    else:
        #Read sign
        sign_context = context_list[1]
        sign = read_ba(sign_context)
        context_list[1] = sign_context
        #Determine value
        if ( sign = False):
            value = magnitude
        else:
            value = -magnitude
    return value
\end{verbatim}

\subsubsection{Truncated unary arithmetic decoding}\begin{comment}
This section specifies the operation of the unsigned truncated unary
arithmetic decoding function ut\_arith\_decode() in terms of binary
arithmetic coding operations.

Pseudo-code for ut\_arith\_decode() is as follows, for values known to be
in the range :

\begin{verbatim}
VALUE=0

while ( !binary\_arith\_decode( choose\_context() ) && VALUE<N )

    VALUE++
\end{verbatim}

choose\_context() is a function that produces a context with which the
binary bit shall be decoded. The value it returns can depend on any
values known to the decoder at the time it is called, especially
including the binarisation bin (the bin number is equal to VALUE+1
according to the conventions of Appendix ).

\end{comment}

%src tim 0.9.1.48

reads and returns an unsigned integer encoded in the bytestream as an
arithmetic coded truncated unary binarisation. ``context\_list'' is a list
of contexts for each bin. If the number of contexts in the list is less
than the bin number then the content of the bin is assumed to be 1 (i.e.
the conditional probability of that bin, the context, is exactly 1).

Read Truncated Arithmetic Coded Unsigned Integer
\begin{verbatim}
read_uua(context_list):

    value = 0
    context_index = 0 #Bin Number (numbered from zero)
    max_index = len(context_list) - 1
    # Read first bit
    context = context_list[context_index]
    if ( read_ba(context) ):
        more = False
    else:
        value += 1
        more = True
    context_list[context_index] = context
    # Read remaining bits
    while ( (context_index < max_index) and more ):
        context_index += 1
        context = context_list[context_index]
        if ( read_ba(context) ):
            more = False
        else:
            value += 1
            more = True
        context_list[context_index] = context
    return value
\end{verbatim}

\subsubsection{Binary arithmetic decoding}\begin{comment}
%from 0.9.1

This function is fundamental to all the arithmetic decoding functions.
It consists of three stages:

1.  Determine the binary output value VALUE as per Section 

2.  Modify the decoder state as per Section 

3.  Update the context statistics by invoking update( CONTEXT , VALUE )
as per Section 
\end{comment}


%src: tim-0.9.1.48

reads a single arithmetic coded bit from the bytestream and returns a
Boolean value.

Read Binary Arithmetic Coded Bit
\begin{verbatim}
read_ba(context):
    while (((high&0x8000)==0x0) and ((low&0x8000)==0x0)):
        shift_bit_in()
    while ( ((high&0x4000)==0x0) and
           ((low&0x4000)==0x4000) ):
        code ^= 0x4000
        high ^= 0x4000
        low ^= 0x4000
        shift_bit_in()
    weight = context[0] + context[1]
    scaler = (0x10000+weight//2)//weight   #lookup table

    probability0 = context[0]*scaler
    count = code-low+1
    range = high-low+1
    range_x_prob = (range * probability0)>>16
    if ( count > range_x_prob ):
        value = True
        low = low + range_x_prob
        context[1] += 1
    else
        value = False
        high = low + range_x_prob - 1
        context[0] += 1
    if ( (context[0] + context[1]) > 255 ):
        #Halve counts in the context
        context[0] >> 1
        context[0] += 1
        context[1] >> 1
        context[1] += 1
    return value
\end{verbatim}

Shift Bit In
\begin{verbatim}
shift_bit_in():

    high << 1
    high &= 0xFFFF
    high += 1
    low << 1
    low &= 0xFFFF
    code << 1
    code &= 0xFFFF
    code += read_bita()
\end{verbatim}

\subsection{Implementation}
%src: tim-0.9.1.48

\begin{informative}

Arithmetic decoding is specified in terms of fixed point arithmetic.
Word widths are defined in the table below.

\begin{tabular}{l|c}
Variable & Width (bits)\\
\hline
bits\_left & 3\\
bit\_buffer & 8\\
low & 16\\
high & 16\\
code & 16\\
context[0] & 8\\
context[1] & 8\\
weight & 8\\
scaler & 16\\
probability0 & 16\\
count & 17 (See informative text for 16 bit systems)\\
range & 17 (See informative text for 16 bit systems)\\
range\_x\_prob & 16
\end{tabular}

probability0 is an estimate of the probability of 0 (Boolean False)
occurring. It is scaled by $2^{16}$ so that probability 0 is represented by
0 and probability 1 is represented by 65536.  In this implementation of
arithmetic coding the probabilities can never take the values 0 or 1
(because contexts are initialised so that count0=count1=1), Ideally the
probability estimate would be calculated from the ratio of the counts of
previous zeros and ones, i.e.:

\begin{displaymath}
probability of zero \approx \frac{count0}{count0 + count1}
\end{displaymath}

Implementing this would require a division per decoded bit, which is
computationally expensive. Instead we multiply by the inverse of
(count0+count1), i.e. we multiply by 1/weight (suitably scaled for fixed
point arithmetic). The integer division can be pre-calculated as a
lookup table. So one possible implementation for calculating
probability0 might be:


Possible implementation for calculating "probability0"
\begin{verbatim}
#Pre-calculate the lookup table
#Create an empty list with 256 elements

scaler = [None]*256

for weight in range(2,255):

    scaler[index] = (0x10000+weight//2)//weight
.
.
.
.
probability0 = context[0]*scaler[weight])
\end{verbatim}

Note: scaler is a lookup table with 256 entries each 16 bits wide. The
minimum weight is 2 (because of context initialisation) and the maximum
weight is 255 (beyond which the counts are halved). Hence entries for
weight 0 and 1 are undefined.

In the specification of "read\_ba(context)" first bits are read in to the
variable "code" until the MSB of "high" is 1 and the MSB of "low" is 0.
For efficiency, because high is always > low,  the test:

\begin{verbatim}
if (((high\&0x8000)==0x0) and ((low\&0x8000)==0x0)):
\end{verbatim}

may be replaced by:

\begin{verbatim}
if ( (high\^low)<0x8000 ):
\end{verbatim}

After  sufficient bits have been read in the 2nd MSB (bit 14) is
sometimes deleted, but the mechanism is not obvious. The way this works
is as follows:

The 2nd MSB should be deleted when the current range, defined by high
and low, straddles the halfway point, a condition known as underflow.
Under these circumstances the values of code, high and low are (with a
and b being specific Boolean values, $\bar{a}$ being not a and x being any
value):

\begin{verbatim}
code    =       a$\bar{a}$xxxxxxxxxxxxxx
high    =       10xxxxxxxxxxxxxx
low     =       01xxxxxxxxxxxxxx
In the specification the 2nd MSBs are inverted yielding:
code    =       aaxxxxxxxxxxxxxx
high    =       11xxxxxxxxxxxxxx
low     =       00xxxxxxxxxxxxxx
This makes the 2nd MSB the same as the MSB. The MSB is now discarded by
shifting bits left giving:
code    =       axxxxxxxxxxxxxxb
high    =       1xxxxxxxxxxxxxx1
low     =       0xxxxxxxxxxxxxx0
The overall effect is to delete the 2nd MSB.
\end{verbatim}

The word widths for variables count and range are specified as 17
bits. This may be inefficient in 16 bit implementations. The 17th bit is
only required in the special case when "code"/"high" equals 0xFFFF and
"low" equals 0. The algorithm may be refactored as follows so that 16
bit arithmetic may be used without overflow:


Read Binary Arithmetic Coded Bit (alternative 16 bit version)
\begin{verbatim}
read_ba(context):
    .
    .
    .
    count = code - low
    range = high - low
    range_x_prob = (range*probability0 + probability0)>>16
    if ( count > (range_x_prob-1) ):
    .
    .
    .
\end{verbatim}

The function "read\_ba(context)" has not been specified in this way for
clarity. Although this alternative would work for word widths of 16 bits
or more, to specify it this way would obfuscate the algorithm.

\end{informative}


\clearpage
\section{Data encodings}%%%%%%%%%%%%%%%%%%%%%%%%%%%%%%%%%%%%%%%%%%%%%%%%
% - This chapter defines how raw and VLC data  - %
% - is extracted                               - % 
%%%%%%%%%%%%%%%%%%%%%%%%%%%%%%%%%%%%%%%%%%%%%%%%

Data is encoded in the Dirac bitstream in four basic ways: fixed-length
bit-wise and byte-wise encodings; variable-length codes; and arithmetic encoding.

This section defines how data bits are extracted from the bitstream and how
sequences of bits are interpreted as values of various types using fundamental
data-reading functions. The extraction of arithmetic-encoded data makes use of
an arithmetic decoding engine which is specified in Appendix \ref{arithengine}.

\subsection{Bit-packing and data input}
\label{bitpacking}

This section defines the operation of the $read\_bit()$, $read\_byte()$ 
and $byte\_align()$ functions used for direct access to the Dirac stream.

Access to the Dirac stream is bytewise, and a decoder is deemed to maintain
a copy of the current byte, $\CurrentByte$, and an index to the next bit
to be read, $\NextBit$. $\NextBit$ is an integer from 0 (least-significant bit) to 7 
(most-significant bit). Bits within bytes are accessed from the msb first to the
lsb.

Each access unit and individual picture is a whole number of bytes. Decoding from the
start of an access unit, $\NextBit$ is set to 7.

The $read\_byte()$ function performs the following steps --
\begin{enumerate}
\item Sets $\NextBit=7$ 
\item Sets $\CurrentByte$ to the next unread byte in the Dirac stream
\end{enumerate}

The $read\_bit()$ function is defined by

\begin{pseudo}{read\_bit}{}
\bsCODE{bit = ( \CurrentByte \gg \NextBit ) \& 1 }
\bsCODE{\NextBit -= 1}
\bsIF{\NextBit<0}
    \bsCODE{\NextBit = 7}
    \bsCODE{read\_byte()}
\bsEND
\bsRET{bit}
\end{pseudo}

The $byte\_align()$ function discards data in the current byte and begins data access
at the next byte, unless input is already at the beginning of a byte: 

\begin{pseudo}{byte\_align}{}
\bsIF{\NextBit != 7}
    \bsCODE{read\_byte()}
\bsEND
\end{pseudo}

This is used to ensure that a whole number of bytes are read before
beginning reading a new stream element.

\subsection{Parsing of fixed-length data}
\subsubsection{Bool}

The $read\_bool()$ function identifies 1 with $\true$ and 0 with $\false$:

\begin{pseudo}{read\_bool}{}
\bsIF{read\_bit()==1}
    \bsRET{\true}
\bsELSE
    \bsRET{\false}
\bsEND
\end{pseudo}

\subsubsection{n-bit literal}
An $n$-bit number in literal format shall be decoded by extracting $n$ bits
in order, using the $read\_bit()$ function (Section \ref{bitpacking})
 and placing the first bit in the leftmost position, the second
bit in the next position and so on. The resulting value is to be
interpreted as an unsigned integer:

\begin{pseudo}{read\_nbits}{n}
\bsCODE{val=0}
\bsFOR{i=0}{n-1}
    \bsCODE{val += read\_bit()}
    \bsCODE{val \ll = 1}
\bsEND
\bsRET{val}
\end{pseudo}

\subsubsection{$n$-byte unsigned integer literal}
A single byte may be interpreted as an unsigned integer value from 0 to 255.

An $n$-byte number in literal format shall be decoded by the recipe:

\begin{pseudo}{read\_uint\_lit}{n}
\bsCODE{val=0}
\bsFOR{i=0}{n-1}
    \bsCODE{val += read\_byte()}{\ref{bitpacking}}
    \bsCODE{val \ll = 8}
\bsEND
\bsRET{val}
\end{pseudo}

\subsection{Parsing of VLC coded data}
\label{vlc}
Variable-length codes are used in three ways in the Dirac stream. The first
use is for direct encoding of header values into the stream. The second use
is for entropy coding of motion data and coefficients, where arithmetic decoding
is not in use.

The third use 
is for binarisation in the arithmetic encoding/decoding process so that integer 
values may be coded and decoded using a binary arithmetic coding engine. This is
described in Section \ref{arithdecoding}.

When used for coding motion data and coefficients, VLCs are employed within
a data block of known length. It is possible to gain additional compression by early termination:
maintaining a count of remaining bits, and returning default values when this length
is exceeded. This is achieved by use of the $read\_bitb()$, $read\_boolb()$, 
$read\_uintb()$ and $read\_sintb()$ for reading values from data blocks. 
(A similar early termination facility is a used for arithmetic decoding.)

\subsubsection{Data input for bounded block operation}
\label{blockreadbit}

This section specifies the operation of the $read\_bitb()$ process for reading bits from
a block of known size. In this case, the decoder state variable $\ABitsLeft$ has been
set to the block length, and is decremented on reading a bit. When all bits in the block
have been read, then $1$ is returned by default:

\begin{pseudo}{read\_bitb}{}
\bsIF{\ABitsLeft==0}
  \bsRET{1}
\bsELSE
  \bsCODE{\ABitsLeft -= 1}
  \bsRET{read\_bit()}
\bsEND
\end{pseudo}

It is possible that not all data in a block is exhausted after a sequence of read operations.
At the end of a sequence of read operations, the decoder will flush the block. This assists
with recovery in the event of errors in the stream, and enforces byte-alignment where required 
(for example in arithmetic coded blocks).

\begin{pseudo}{flush\_inputb}{}
\bsWHILE{\ABitsLeft>0}
    \bsCODE{read\_bit()}
    \bsCODE{\ABitsLeft-=1}
\bsEND
\end{pseudo}


\subsubsection{Unsigned interleaved exp-Golomb codes}
This section defines the unsigned interleaved exp-Golomb data format and the operation
of the $read\_uint()$ and the $read\_uintb()$ functions. 

Unsigned interleaved exp-Golomb data is decoded to produce unsigned
 integer values.The format consists of two interleaved parts, 
and each code is an odd number, $2K+1$ bits in length.

The $K+1$ bits in the even positions (counting from zero) are the ``follow" bits, and 
the $K$ bits in the odd positions are the ``data" bits $b_i$ which are used to construct
the decoded value itself. A follow bit value of $0$ indicates a subsequent data bit,
whereas a follow bit value of $1$ terminates the code:
\begin{equation*}
0\quad b_{K-1}\quad 0\quad b_{K-2}\hdots 0\quad b_{0}\quad 1
\end{equation*}

The data bits $b_{K-1}, b_{K-2}, \hdots, b_0$ are the binary representation 
 of the first $K$ bits of the $(K+1)$-bit number 
$N+1$, where $N$ is the number to be decoded:
\begin{equation*}
N+1=1 b_{K-1} b_{K-2}\hdots b_0 \text{(base $2$)}
\end{equation*}

A table of encodings of the first 10 values is shown in Figure \ref{uegolcodings}.

\begin{figure}[!ht]]
\centering
\begin{tabular}{l|c}
Bit sequence & Decoded value \\
\hline\\
1                 &  0\\
0 0 1             &  1\\
0 1 1             &  2\\
0 0 0 0 1         &  3\\
0 0 0 1 1         &  4\\
0 1 0 0 1         &  5\\
0 1 0 1 1         &  6\\
0 0 0 0 0 0 1     &  7\\
0 0 0 0 0 1 1     &  8\\
0 0 0 1 0 0 1     &  9\\
\end{tabular}

\caption{Example conversions from unsigned interleaved exp-Golomb-coded 
values to unsigned integers \label{uegolcodings}}
\end{figure}

Although apparently complex, the interleaving ensures that the code has a very simple decoding loop. 
The $read\_uint()$ function returns an unsigned integer value and is defined by the recipe:

\begin{pseudo}{read\_uint}{}
\bsCODE{value = 1}
\bsWHILE{read\_bit() == 0}
  \bsCODE{value \ll = 1}
  \bsIF{read\_bit()==1}
    \bsCODE{value += 1}
  \bsEND
\bsEND
\bsCODE{value -= 1}
\bsRET{value}
\end{pseudo}

\begin{informative}
Conventional exp-Golomb coding places all follow bits at the beginning as a prefix. This is
easier to read, but requires that a count of the prefix length be maintained. Values can only
be decoded in two loops -- the prefix followed by the data bits. Interleaved exp-Golomb 
coding allows values to be decoded in a single loop, without the need for a length count.
\end{informative}

The $read\_uintb()$ function is identical to $read\_uint()$ except that the block-bounded read
operation is employed:

\begin{pseudo}{read\_uintb}{}
\bsCODE{value = 1}
\bsWHILE{read\_bitb() == 0}
  \bsCODE{value \ll = 1}
  \bsIF{read\_bitb()==\true}
    \bsCODE{value += 1}
  \bsEND
\bsEND
\bsCODE{value -= 1}
\bsRET{value}
\end{pseudo}

Note that when $\ABitsLeft==0$, all subsequent values read by $read\_uintb()$ will be 0.

\subsubsection{Signed interleaved exp-Golomb}
\label{segol}

This section defines the signed interleaved exp-Golomb data format and the operation
of the $read\_sint()$ and $read\_sintb()$ functions.

The code for the signed interleaved exp-Golomb data format consists of the
unsigned interleaved exp-Golomb code for the magnitude, followed by a sign bit
for non-zero values (Figure \ref{segolcodings}).

\begin{figure}[!ht]
\centering
\begin{tabular}{l|c}
Bit sequence & Decoded value \\
\hline\\
0 0 0 1 1 1         &  -4\\
0 0 0 0 1 1         &  -3\\
0 1 1 1            &  -2\\
0 0 1 1           &  -1\\
1                 &  0\\
0 0 1 0           &  1\\
0 1 1 0            &  2\\
0 0 0 0 1 0         &  3\\
0 0 0 1 1 0         &  4\\
\end{tabular}

\caption{Example conversions from signed interleaved exp-Golomb-coded values 
to signed integers \label{segolcodings}}
\end{figure}

The decoding operation is as follows.

\begin{pseudo}{read\_sint}{}
\bsCODE{value = read\_uint()}
\bsIF{ value!= 0}
  \bsIF{read\_bit()==1}
    \bsCODE{value = -value}
  \bsEND
\bsEND
\bsRET{value}
\end{pseudo}

The $read\_sintb()$ function is identical to $read\_sint()$ except that the block-bounded read
operation is employed:

\begin{pseudo}{read\_sintb}{}
\bsCODE{value = read\_uintb()}
\bsIF{value != 0}
  \bsIF{read\_bitb()==1}
    \bsCODE{value = -value}
  \bsEND
\bsEND
\bsRET{value}
\end{pseudo}

Note that when $\ABitsLeft==0$, all subsequent values read by $read\_sintb()$ will be 0.

\subsection{Parsing of arithmetic-coded data}

\label{arithdecoding}

This section defines the operations for reading and writing arithmetic-coded
data, and may be thought of as specifying an API for
extracting data elements using the arithmetic decoding engine, making use
of elementary arithmetic coding functions defined by the arithmetic decoding engine 
specified in Appendix \ref{arithengine}.

Arithmetically-coded data is present in the Dirac stream in data blocks which consist of
a whole number of bytes and are byte aligned. Where arithmetic coding is used, each such
block is preceded by data which includes a length code $length$, which is equal to the length in
bytes of the data block. The function $initialise\_arithmetic\_decoding(length)$
(Section \ref{initarith}) then initialises the arithmetic decoder. Once the arithmetic
decoder is initialised, boolean and integer values may be extracted.

After all values in a particular arithmetic coded block have been parsed, and remaining data
is purged by calling $flush\_inputb()$ (Section \ref{blockreadbit}).



\subsubsection{Contexts}
\label{arithcontexts}

Values are extracted by using binary {\em contexts}, which are estimates of the 
probability of the binary values extracted being 0 ($\false$) or 1 ($\true$).
Contexts are determined prior to values being extracted, based on previously
decoded data, and are updated by the arithmetic decoding engine to track 
statistics. The structure of contexts and how they are initialised and updated 
is specified in Section \ref{arithcontexts}.

The set of contexts is $\AContexts$, and an individual context is accessed via
an index $\AContexts[i]$. 

\paragraph{Context indices}
\label{contextindices}
$\ $\newline
The following context indices are used within the Dirac decoder:

Wavelet coefficient contexts:

\SignZero\\
\SignPos\\
\SignNeg\\
\ZPZNFollowOne\\
\ZPNNFollowOne\\
\ZPFollowTwo\\
\ZPFollowThree\\
\ZPFollowFour\\
\ZPFollowFive\\
\ZPFollowSixPlus\\
\NPZNFollowOne\\
\NPNNFollowOne\\
\NPFollowTwo\\
\NPFollowThree\\
\NPFollowFour\\
\NPFollowFive\\
\NPFollowSixPlus\\
\CoeffData\\
\ZeroCodeblock\\
\QOffsetFollow\\
\QOffsetData\\
\QOffsetSign\\

Motion data contexts:

\SBSplitFollowOne\\
\SBSplitFollowTwo\\
\SBSplitData\\
\PredModeOne\\
\PredModeTwo\\
\BlockGlobal\\
\VectorFollowOne\\
\VectorFollowTwo\\
\VectorFollowThree\\
\VectorFollowFour\\
\VectorFollowFivePlus\\
\VectorData\\
\VectorSign\\
\DCFollowOne\\
\DCFollowTwoPlus\\
\DCData\\
\DCSign\\

\subsubsection{Arithmetic decoding of boolean values}

Given a context index $i$, the arithmetic decoding engine supports a function
$read\_boola(i)$, specified in Section \ref{arithreadbool}, which returns the 
boolean value $\true$ or $\false$.

\subsubsection{Arithmetic decoding of integer values}

\label{arithreadint}

This section defines the operation of the $read\_sinta(context\_set)$ and
$read\_uinta(context\_set)$ function
 for extracting integer values from a block of arithmetically coded data.

\paragraph{Binarisation and contexts}
$\ $\newline
Signed and unsigned integer values are binarised using interleaved exp-Golomb
 binarisation as per Section \ref{vlc}: the $read\_sinta()$ and $read\_uinta()$
processes are essentially identical to the 
$read\_sint()$ and $read\_uint()$ processes, except that instances of $read\_bool()$ are replaced
by instances of $read\_boola()$ (Section \ref{arithreadbool}) using suitable contextualisation. 

A choice of context depends upon whether the bit is a data bit, follow bit, 
or sign bit, and upon the position of the bit within the binarisation: 
$context\_set$ consists of three parts -
\begin{itemize}
\item an array of follow context indices, $context\_set[follow]$ (indexed from 0 to 
$\length(context\_set[follow])-1$)
\item a single data context index, $context\_set[data]$ 
\item a sign context index, $context\_set[sign]$ (ignored for unsigned integer decoding)
\end{itemize}

Each follow context is used for decoding the corresponding follow bit, with the
last follow context being used for all subsequent follow bits (if any) also. 
The follow context selection function $follow\_context()$ is defined by:

\begin{pseudo}{follow\_context}{index, context\_set}
\bsCODE{pos= \min(index, length(context\_set[follow])-1) }
\bsCODE{ctx\_index = context\_set[follow][pos]}
\bsRET{\AContexts[ctx\_index]}
\end{pseudo}

So the last follow context is used for all the remaining follow bits also.

\paragraph{Unsigned integer decoding}
$\ $\newline
Unsigned integers are decoded by:

\begin{pseudo}{read\_uinta}{context\_set}
\bsCODE{value = 1}
\bsCODE{index = 0}
\bsWHILE{read\_boola(follow\_context(index,context\_set) )== \false}
  \bsCODE{value \ll = 1}
  \bsIF{read\_boola(\text{state[contexts]}[context\_set[data])])}
    \bsCODE{value += 1}
  \bsEND
  \bsCODE{index += 1}
\bsEND
\bsCODE{value -= 1}
\bsRET{value}
\end{pseudo}

\paragraph{Signed integer decoding}
$\ $\newline
$read\_sinta()$ decodes first the magnitude then the sign, as necessary:

\begin{pseudo}{read\_sinta}{context\_set}
\bsCODE{value=read\_uinta(context\_set)}
\bsIF{value != 0}
  \bsIF{read\_boola(\AContexts[context\_set[sign])]) == \true}
    \bsCODE{value=-value}
  \bsEND
\bsEND
\bsRET{value}
\end{pseudo}
\subsection{Fixed-length code formats}
\subsubsection{Bool}Encoded as a single bit. A value of  1 shall be decoded as TRUE and 0
shall be decoded as false.



\subsubsection{n-bit literal}An n-bit number in literal format shall be decoded by extracting n bits
in order and placing the first bit in the leftmost position, the second
bit in the next position and so on. The resulting value is to be
interpreted according to the conventions of the type to which it is
deemed to belong. For example, signed integers are represented in
two's complement format [ISO ref].



\subsubsection{n-byte literal}An $n$-byte number in literal format shall be decoded by extracting $n$
bytes in order and placing the first byte in the leftmost position, the
second byte in the next position and so on. The resulting value is to be
interpreted according to the conventions of the type to which it is
deemed to belong.

\subsection{Variable-length code formats}%Seven different variable-length code formats are used. Of these three,
%the unary binarisation formats are not used directly for data encoding
%but for binarising values so that integer values may be produced from
%strings of bits and vice-versa. Binarisation is used in the context of
%arithmetic decoding. [Change this]

Variable-length codes are used in two ways in the Dirac stream. The first
use is for direct encoding into the stream. The second use is for binarisation
in the arithmetic encoding/decoding process so that integer values may be 
coded and decoded using a binary arithmetic coding engine. 

\subsubsection{Unsigned unary uu(n)}
Unsigned unary data represents unsigned integer values. The value
$n \geq 0$ is represented as $n$ zeroes followed by a 1:

\begin{figure}[h]
\begin{tabular}{c|c}
Value & Binarisation\\
\hline\\
0  & 1 \\
1  & 01 \\
\dots & \dots\\
$n$   & $\underbrace{0\dots0}_n 1$
\end{tabular}

\caption{Conversion from unsigned unary to binary}
\end{figure}

The bit values in the unary representation of a value are referred to by
bin numbers: the first bit lies in bin 1, the second in bin 2 and so on.
A 0 in bin $n$ therefore indicates that the value is greater than or equal
to $n$, where a 1 indicates the value equals $n-1$.


\subsubsection{Signed unary su(n)}


\subsubsection{Unsigned truncated unary ut(n)}

If a value lies in a range $0 \leq x \leq n$, then truncated unary
binarisation can be used: that is, for the last value the final 1 can be
omitted, since its presence is inevitable.

\begin{figure}[h]
\begin{tabular}{c|c}
Value & Binarisation \\
\hline\\
0     & 1 \\
1     & 01 \\
\dots & \dots \\
$n-1$ & $\underbrace{0\dots0}_{n-1} 1$ \\
$n$   & $\underbrace{0\dots0}_{n}$
\end{tabular}

\caption{Conversion from unsigned truncated unary to binary}
\end{figure}


\subsubsection{Unsigned exp-Golomb uegol(n)}Unsigned exp-Golomb data is decoded to produce unsigned integer values.
The format consists of two parts. A prefix part, consisting of n zeroes
followed by a one, indicates how many further bits to read. A suffix
part, consisting of n bits, is then used to determine the value. The
decoding procedure for extracting a value VALUE is mathematically
equivalent to:

\begin{verbatim}
COUNT= 0
while( !read\_bits(1) ) {
    COUNT++
}
VALUE = (1<<COUNT) -1 + read\_bits( COUNT )
\end{verbatim}

where the value returned by read\_bits( COUNT ) is interpeted as a binary
representation of an unsigned integer with most-significant bit first.
The bit sequences corresponding to some values are shown in  .

\begin{figure}[h]
\begin{tabular}{c|c}
Bit sequence & Decoded value \\
\hline\\
1       &  0\\
010     &  1\\
011     &  2\\
00100   &  3\\
00101   &  4\\
00110   &  5\\
00111   &  6\\
0001000 &  7\\
0001001 &  8\\
0001010 &  9\\
0001011 & 10
\end{tabular}

\caption{Example conversions from uegol-coded values to binary}
\end{figure}




\subsubsection{Signed exp-Golomb segol(n)}This section defines the signed interleaved exp-Golomb data format and the operation
of the $read\_sint()$ function.

The code for the signed interleaved exp-Golomb data format consists of the
unsigned interleaved exp-Golomb code for the magnitude, followed by a sign bit
for non-zero values (figure \ref{segolcodings}).

\begin{figure}[h]
\begin{tabular}{l|c}
Bit sequence & Decoded value \\
\hline\\
1                 &  0\\
0 0 1 0           &  1\\
0 0 1 1           &  -1\\
0 1 1 0            &  2\\
0 1 1 1            &  -2\\
0 0 0 0 1 0         &  3\\
0 0 0 0 1 1         &  -3\\
0 0 0 1 1 0         &  4\\
0 0 0 1 1 1         &  -4\\

\end{tabular}

\caption{Example conversions from signed interleaved exp-Golomb-coded values 
to signed integers \label{segolcodings}}
\end{figure}

The decoding operation is as follows.

\begin{pseudo}{read\_sint}{}
\bsCODE{value = read\_uint()}
\bsIF{read\_bool()}
  \bsCODE{value *= -1}
\bsEND
\bsRET{value}

\end{pseudo}


%\clearpage
%\section{Bytestream specification}This section defines what the Bytestream. As the Bytestream contains a
lot of metadata which is in arithmetically-coded format, there is a vast
volume of data which cannot be instantly be assigned to particular data
structures. This data will require processing (as described in Sections
XXXXXX) before the Bytestream can be fully decoded.

A Stream is a concatenation of Sequences, which are Video Sequences
that have constant Source Parameters (eg, picture size, aspect ratio,
etc).

The parameters specified in the Access Unit Header reamain the same
throughout a Sequence.  That is, the parameters in later Access Unit
Headers simply repeat those in earlier headers and are provided to
provide entry points to start decoding the byte stream.

Presentation order picture numbers within a sequence must be contiguous.

If the parameters need to change the only way to do this is to signal
the end of sequence and start a new sequence.

The process of editing two sequences together would introduce
discontiguous presentation order picture numbers.  This is accommodated
by introducing End of Sequence parse codes before a cut so that the
decoder would restart after a cut.
\annotate{shas}{May need to explain that the stream is presented as a
series of bytes. When being read in a bitwise fashion, bytes are
presented with the most significant bit first/last?}


%\subsection{Syntactic specification}
\newcommand{\fdefine}[2]{%
    \label{#1}%
    \expandafter\def\csname #1\endcsname{{#2}}}

\newcommand{\pdefine}[2]{%
    \label{#1}%
    \expandafter\def\csname #1\endcsname{{\texttt{#2}}}}


\pdefine{ParseInfoPrefix}{parse\_info\_prefix}
\pdefine{ParseCode}{parse\_code}
\pdefine{NextParseOffset}{next\_parse\_offset}
\pdefine{PrevParseOffset}{previous\_parse\_offset}

\pdefine{AUFrameNumber}{au\_frame\_number}
\pdefine{VersionMajor}{version\_major}
\pdefine{VersionMinor}{version\_minor}
\pdefine{Profile}{profile}
\pdefine{Level}{level}

\pdefine{VideoFormat}{video\_format}
\pdefine{CustomImageSize}{custom\_image\_size}
\pdefine{LumaWidth}{luma\_width}
\pdefine{LumaHeight}{luma\_height}
\pdefine{UnusualChromaFormat}{unusual\_chroma\_format}
\pdefine{ChromaFormatIndex}{chroma\_format\_index}
\pdefine{CustomVideoDepth}{custom\_video\_depth}
\pdefine{VideoDepthValue}{video\_depth\_value}

\pdefine{UnusualScanFormat}{unusual\_scan\_format}
\pdefine{Interlaced}{interlaced}
\pdefine{UnusualFieldDominance}{unusual\_field\_dominance}
\pdefine{TopFieldFirst}{top\_field\_first}
\pdefine{UnusualFieldInterleaving}{unusual\_field\_interleaving}
\pdefine{SequentialFields}{sequential\_fields}
\pdefine{CustomFrameRate}{custom\_frame\_rate}
\pdefine{FrameRateIndex}{frame\_rate\_index}
\pdefine{FrameRateNumerator}{frame\_rate\_numerator}
\pdefine{FrameRateDenominator}{frame\_rate\_denominator}
\pdefine{CustomAspectRatio}{custom\_aspect\_ratio}
\pdefine{AspectRatioIndex}{aspect\_ratio\_index}
\pdefine{AspectRatioNumerator}{aspect\_ratio\_numerator}
\pdefine{AspectRatioDenominator}{aspect\_ratio\_denominator}

\pdefine{CustomCleanArea}{custom\_clean\_area}
\pdefine{CleanWidth}{clean\_width}
\pdefine{CleanHeight}{clean\_height}
\pdefine{LeftOffset}{left\_offset}
\pdefine{TopOffset}{top\_offset}

\pdefine{CustomSignalRange}{custom\_signal\_range}
\pdefine{SignalRangeIndex}{signal\_range\_index}
\pdefine{LumaOffset}{luma\_offset}
\pdefine{LumaExcursion}{luma\_excursion}
\pdefine{ChromaOffset}{chroma\_offset}
\pdefine{ChromaExcursion}{chroma\_excursion}

\pdefine{UnusualColourSpec}{unusual\_colour\_spec}
\pdefine{ColourSpecIndex}{colour\_spec\_index}
\pdefine{UnusualColourPrimaries}{unusual\_colour\_primaries}
\pdefine{ColourPrimariesIndex}{colour\_primaries\_index}
\pdefine{UnusualColourMatrix}{unusual\_colour\_matrix}
\pdefine{ColourMatrixIndex}{colour\_matrix\_index}
\pdefine{UnusualTransferFunction}{unusual\_transfer\_function}
\pdefine{TransferFunctionIndex}{transfer\_function\_index}

\pdefine{PictureNumberOffset}{picture\_number\_offset}
\pdefine{RefOffsetA}{ref1\_offset}
\pdefine{RefOffsetB}{ref2\_offset}
\pdefine{NumRetiredPictures}{num\_retired\_pictures}
\pdefine{RetiredPictureOffset}{retired\_picture\_offset}

\pdefine{BlockDataLength}{block\_data\_length}
\pdefine{CompressedBlockData}{compressed\_block\_data}
\pdefine{CustomBlockParameters}{custom\_block\_parameters}
\pdefine{BlockParametersIndex}{block\_parameters\_index}
\pdefine{LumaXBLen}{luma\_xblen}
\pdefine{LumaYBLen}{luma\_yblen}
\pdefine{LumaXBSep}{luma\_xbsep}
\pdefine{LumaYBSep}{luma\_ybsep}
\pdefine{CustomMotionVectorPrecision}{custom\_motion\_vector\_precision}
\pdefine{MotionVectorPrecision}{motion\_vector\_precision}
\pdefine{UsingGlobalMotionFlag}{using\_global\_motion\_flag}
\pdefine{GlobalMotionOnlyFlag}{global\_motion\_only\_flag}
\pdefine{CustomPicturePredictionMode}{custom\_picture\_prediction\_mode}
\pdefine{PicturePredictionModeIndex}{picture\_prediction\_mode\_index}


\pdefine{CustomReferenceWeights}{custom\_reference\_weights}
\pdefine{PictureWeightBits}{picture\_weight\_bits}
\pdefine{PictureWeightRefA}{picture\_weight\_ref1}
\pdefine{PictureWeightRefB}{picture\_weight\_ref2}
\pdefine{NonZeroPanTiltFlag}{nonzero\_pan\_tilt\_flag}
\pdefine{GMpan}{pan}
\pdefine{GMtilt}{tilt}
\pdefine{NonZeroZRSFlag}{nonzero\_zoom\_rotation\_shear\_flag}
\pdefine{ZRSexponent}{ZRS\_exponent}
\pdefine{GMaTL}{a11}
\pdefine{GMaTR}{a12}
\pdefine{GMaBL}{a21}
\pdefine{GMaBR}{a22}
\pdefine{NonZeroPerspectiveFlag}{nonzero\_perspective\_flag}
\pdefine{GMperspectiveExponent}{perspective\_exponent}
\pdefine{GMperspectiveX}{perspective\_x}
\pdefine{GMperspectiveY}{perspective\_y}


%\lstset{language=Python,basicstyle=\ttfamily,keywordstyle=\rmfamily\underline}


\newcounter{indent}
\newlength{\indentx}
\setlength{\indentx}{1em}
\newenvironment{streamsyntax}[2]
    {\newcommand{\dfindent}{\\\hline\hspace{\value{indent}\indentx}}
     \newcommand{\bsIF}[1]{\dfindent\textrm{if (\texttt{##1}):}\stepcounter{indent} & &}
     \newcommand{\bsEND}{\addtocounter{indent}{-1}}
     \newcommand{\bsELSE}{\addtocounter{indent}{-1}\dfindent\textrm{else:}\stepcounter{indent} & &}
     \newcommand{\bsELSEIF}[1]{\addtocounter{indent}{-1}\dfindent\textrm{else if (\texttt{##1}):}\stepcounter{indent} & &} 
     \newcommand{\bsWHILE}[1]{\dfindent\textrm{while (\texttt{##1}):}\stepcounter{indent} & &}
     \newcommand{\bsFOREACH}[2]{\dfindent\textrm{foreach \texttt{##1} in \texttt{##2}:}\stepcounter{indent} & &}
     \newcommand{\bsFOR}[2]{\dfindent\textrm{for \texttt{##1} to \texttt{##2}:}\stepcounter{indent} & &}
     \newcommand{\bsRET}[1]{\dfindent\textrm{return \texttt{##1}}\addtocounter{indent}{-1} & &}
     \newcommand{\bsCODE}[1]{\dfindent\texttt{##1} & &}
     %\newcommand{\bsITEM}[3]{\dfindent\textbf{\texttt{##1}} & ##2 & ##3}
     \newcommand{\bsITEM}[3]{\dfindent\textbf{\texttt{##1 = read\_##2()}} & & ##3}
     \setcounter{indent}{1}
     \hspace{0.5in}
%     \begin{tabular}{|m{3.75in}|m{0.6in}|m{0.25in}|}
     \begin{tabular}{|m{4.35in}m{0.0in}|m{0.25in}|}
         % firstline is function definition
         \hline
%         \texttt{#1(#2)} : & \textbf{Type} & \textbf{Ref}
         \texttt{#1(#2)} : &  & \textbf{Ref}
    }
    {    % last line is endof function
         \\\hline
         \end{tabular}
         }


%%%%%%%%%%%%%%%%%%%%%%%%%%
\subsubsection{Sequence}


%\bsIF{}
%\bsEND
%\bsELSE
%\bsELSEIF{}
%\bsWHILE{}
%\bsFOREACH{}{}
%\bsFOR{}{}
%\bsRET{}

%\bsCODE{}
%\item{}{}

\begin{streamsyntax}{sequence}{}
\bsCODE{i = 0}
\bsWHILE{true}
 \bsCODE{seq[i] = parse\_unit()}
 \bsIF{seq[i].type == EndOfSequence}
  \bsRET{seq}%\bsEND
 \bsCODE{i = i + 1}
 \bsEND
\end{streamsyntax}


%%%%%%%%%%%%%%%%%%%%%%%%%%
\subsubsection{Parse unit}

\begin{streamsyntax}{parse\_unit}{}
\bsCODE{pu[type] = read\_parse\_info()}
\bsIF{pu[type] == AccessUnitHdr}
 \bsCODE{pu[data] = read\_access\_unit\_header()}
\bsELSEIF{pu[type] == Picture}
 \bsCODE{pu[data] = read\_picture()}
 \bsEND
\bsRET{pu}
\end{streamsyntax}
 

\begin{streamsyntax}{parse\_info}{}
\bsCODE{bytealign()}
\bsITEM{\ParseInfoPrefix}{be(4)}{}
\bsITEM{\ParseCode}{be(1)}{}
\bsITEM{\NextParseOffset}{be(3)}{}
\bsITEM{\PrevParseOffset}{be(3)}{}
\bsRET{\ParseCode}
\end{streamsyntax}


%%%%%%%%%%%%%%%%%%%%%%%%%%
\subsubsection{Access unit}

\begin{streamsyntax}{access\_unit\_header}{}
\bsCODE{bytealign()}
\bsCODE{auh[parse\_params] = read\_parse\_parameters()}
\bsCODE{auh[seq\_params]   = read\_sequence\_parameters()}
\bsCODE{auh[display\_params] = read\_display\_parameters()}
\bsRET{auh}
\end{streamsyntax}


%%%%%%%%%%%%%%%%%%%%%%%%%%
\subsubsection{Access unit parse parameters}

\begin{streamsyntax}{parse\_parameters}{}
\bsITEM{pp[\AUFrameNumber]}{be(4)}{}
\bsITEM{pp[\VersionMajor]}{unsigned}{}
\bsITEM{pp[\VersionMinor]}{unsigned}{}
\bsITEM{pp[\Profile]}{unsigned}{}
\bsITEM{pp[\Level]}{unsigned}{}
\bsRET{pp}
\end{streamsyntax}


%%%%%%%%%%%%%%%%%%%%%%%%%%
\subsubsection{Access unit sequence parameters}

\begin{streamsyntax}{sequence\_parameters}{}
\bsITEM{sp[\VideoFormat]}{unsigned}{}
\bsCODE{sp[custom\_image\_size] = custom\_image\_size()}
\bsCODE{sp[unusual\_chroma\_format] = unusual\_chroma\_format()}
\bsCODE{sp[custom\_video\_depth] = custom\_video\_depth()}
\bsRET{sp}
\end{streamsyntax}

\begin{streamsyntax}{custom\_image\_size}{}
\bsITEM{cis[\CustomImageSize]}{boolean}{}
\bsIF{cis[\CustomImageSize]}
 \bsITEM{cis[\LumaWidth]}{unsigned}{}
 \bsITEM{cis[\LumaHeight]}{unsigned}{}
 \bsEND
\bsRET{cis}
\end{streamsyntax}

\begin{streamsyntax}{unusual\_chroma\_format}{}
\bsITEM{ucf[\UnusualChromaFormat]}{boolean}{}
\bsIF{ucf[\UnusualChromaFormat]}
 \bsITEM{ucf[\ChromaFormatIndex]}{unsigned}{}
 \bsEND
\bsRET{ucf}
\end{streamsyntax}

\begin{streamsyntax}{custom\_video\_depth}{}
\bsITEM{cvd[\CustomVideoDepth]}{boolean}{}
\bsIF{cvd[\CustomVideoDepth]}
 \bsITEM{cvd[\VideoDepthValue]}{unsigned}{}
 \bsEND
\bsRET{cvd}
\end{streamsyntax}


%%%%%%%%%%%%%%%%%%%%%%%%%%
\subsubsection{Access unit source parameters}

\begin{streamsyntax}{source\_parameters}{}
\bsCODE{sp[unusual\_scan\_format] = unusual\_scan\_format()}
\bsCODE{sp[custom\_frame\_rate] = custom\_frame\_rate()}
\bsCODE{sp[custom\_aspect\_ratio] = custom\_aspect\_ratio()}
\bsCODE{sp[custom\_clean\_area] = custom\_clean\_area()}
\bsCODE{sp[custom\_signal\_range] = custom\_signal\_range()}
\bsCODE{sp[unusual\_colour\_spec] = unusual\_colour\_spec()}
\bsRET{sp}
\end{streamsyntax}
\annotate{df}{replace colour\_spec with colourimetry?}

\begin{streamsyntax}{unusual\_scan\_format}{}
\bsITEM{usf[\UnusualScanFormat]}{boolean}{}
\bsIF{usf[\UnusualScanFormat]}
 \bsITEM{usf[\Interlaced]}{boolean}{}
 \bsIF{usf[\Interlaced]}
  \bsITEM{usf[\UnusualFieldDominance]}{boolean}{}
  \bsIF{usf[\UnusualFieldDominance]}
   \bsITEM{usf[\TopFieldFirst]}{boolean}{}
   \bsEND
  \bsITEM{usf[\UnusualFieldInterleaving]}{boolean}{}
  \bsIF{usf[\UnusualFieldInterleaving]}
   \bsITEM{usf[\SequentialFields]}{boolean}{}
   \bsEND
  \bsEND
 \bsEND
\bsRET{usf}
\end{streamsyntax}

\begin{streamsyntax}{custom\_frame\_rate}{}
\bsITEM{cfr[\CustomFrameRate]}{boolean}{}
\bsIF{cfr[\CustomFrameRate]}
 \bsITEM{cfr[\FrameRateIndex]}{unsigned}{}
 \bsIF{cfr[\FrameRateIndex] == 0}
  \bsITEM{cfr[\FrameRateNumerator]}{unsigned}{}
  \bsITEM{cfr[\FrameRateDenominator]}{unsigned}{}
  \bsEND
 \bsEND
\bsRET{cfr}
\end{streamsyntax}

\begin{streamsyntax}{custom\_aspect\_ratio}{}
\bsITEM{car[\CustomAspectRatio]}{boolean}{}
\bsIF{car[\CustomAspectRatio]}
 \bsITEM{car[\AspectRatioIndex]}{unsigned}{}
 \bsIF{car[\AspectRatioIndex] == 0}
  \bsITEM{car[\AspectRatioNumerator]}{unsigned}{}
  \bsITEM{car[\AspectRatioDenominator]}{unsigned}{}
  \bsEND
 \bsEND
\bsRET{car}
\end{streamsyntax}

\begin{streamsyntax}{custom\_clean\_area}{}
\bsITEM{cca[\CustomCleanArea]}{boolean}{}
\bsIF{cca[\CustomCleanArea]}
 \bsITEM{cca[\CleanWidth]}{unsigned}{}
 \bsITEM{cca[\CleanHeight]}{unsigned}{}
 \bsITEM{cca[\LeftOffset]}{unsigned}{}
 \bsITEM{cca[\TopOffset]}{unsigned}{}
 \bsEND
\bsRET{cca}
\end{streamsyntax}

\begin{streamsyntax}{custom\_signal\_range}{}
\bsITEM{csr[\CustomSignalRange]}{boolean}{}
\bsIF{csr[\CustomSignalRange]}
 \bsITEM{csr[\SignalRangeIndex]}{unsigned}{}
 \bsIF{csr[\SignalRangeIndex] == 0}
  \bsITEM{csr[\LumaOffset]}{unsigned}{}
  \bsITEM{csr[\LumaExcursion]}{unsigned}{}
  \bsITEM{csr[\ChromaOffset]}{unsigned}{}
  \bsITEM{csr[\ChromaExcursion]}{unsigned}{}
  \bsEND
 \bsEND
\bsRET{csr}
\end{streamsyntax}

\begin{streamsyntax}{unusual\_colour\_spec}{}
\bsITEM{ucs[\UnusualColourSpec]}{boolean}{}
\bsIF{ucs[\UnusualColourSpec]}
 \bsITEM{ucs[\ColourSpecIndex]}{unsigned}{}
 \bsIF{ucs[\ColourSpecIndex] == 0}
  \bsITEM{ucs[\UnusualColourPrimaries]}{boolean}{}
  \bsIF{ucs[\UnusualColourPrimaries]}
   \bsITEM{ucs[\ColourPrimariesIndex]}{unsigned}{}
   \bsEND
  \bsITEM{ucs[\UnusualColourMatrix]}{boolean}{}
  \bsIF{ucs[\UnusualColourMatrix]}
   \bsITEM{ucs[\ColourMatrixIndex]}{unsigned}{}
   \bsEND
  \bsITEM{ucs[\UnusualTransferFunction]}{boolean}{}
  \bsIF{ucs[\UnusualTransferFunction]}
   \bsITEM{ucs[\TransferFunctionIndex]}{unsigned}{}
   \bsEND
  \bsEND
 \bsEND
\bsRET{ucs}
\end{streamsyntax}


%%%%%%%%%%%%%%%%%%%%%%%%%%
\subsubsection{Picture}

\begin{streamsyntax}{picture\_header}{}
\bsCODE{bytealign()}
\bsITEM{ph[\PictureNumberOffset]}{signed}{}
\bsIF{\textit{is\_inter}}
 \bsCODE{ph[reference\_pictures\_numbers] = reference\_picture\_numbers()}
 \bsEND
\bsCODE{ph[retired\_picture\_list] = retired\_picture\_list()}
\bsRET{ph}
\end{streamsyntax}

\begin{streamsyntax}{reference\_picture\_numbers}{}
\bsITEM{rpn[\RefOffsetA]}{signed}{}
\bsIF{\textit{num\_refs} == 2}
 \bsITEM{rpn[\RefOffsetB]}{signed}{}
 \bsEND
\bsRET{rpn}
\end{streamsyntax}

\begin{streamsyntax}{retired\_picture\_list}{}
\bsITEM{rpl[\NumRetiredPictures]}{unsigned}{}
\bsFOR{i = 0}{rpl[\NumRetiredPictures] - 1}
 \bsITEM{rpl[\RetiredPictureOffset][i]}{signed}{}
 \bsEND
\bsRET{rpl}
\end{streamsyntax}


%%%%%%%%%%%%%%%%%%%%%%%%%%
\subsubsection{Picture prediction}

\begin{streamsyntax}{picture\_prediction}{}
\bsCODE{pp[picture\_prediction\_parameters] = picture\_prediction\_parameters()}
\bsITEM{pp[\BlockDataLength]}{unsigned}{}
\bsCODE{bytealign()}
\bsITEM{pp[\CompressedBlockData]}{chunk}{}
%    compressed\_block\_data = read\_chunk(block\_data\_length)
\bsRET{pp}
\end{streamsyntax}

\begin{streamsyntax}{picture\_prediction\_parameters}{}
\bsCODE{ppp[custom\_block\_parameters] = custom\_block\_parameters()}
\bsCODE{ppp[motion\_vector\_precision] = motion\_vector\_precision()}
\bsCODE{ppp[global\_motion] = global\_motion()}
\bsCODE{ppp[custom\_picture\_prediction\_mode] = custom\_picture\_prediction\_mode()}
\bsCODE{ppp[picture\_weights] = picture\_weights()}
\bsRET{ppp}
\end{streamsyntax}

\begin{streamsyntax}{block\_parameters}{}
\bsITEM{bp[\CustomBlockParameters]}{boolean}{}
\bsIF{bp[\CustomBlockParameters]}
 \bsITEM{bp[\BlockParametersIndex]}{unsigned}{}
 \bsIF{bp[\BlockParametersIndex] == 0}
  \bsITEM{bp[\LumaXBLen]}{unsigned}{}
  \bsITEM{bp[\LumaYBLen]}{unsigned}{}
  \bsITEM{bp[\LumaXBSep]}{unsigned}{}
  \bsITEM{bp[\LumaYBSep]}{unsigned}{}
  \bsEND
 \bsEND
\bsRET{bp}
\end{streamsyntax}

\begin{streamsyntax}{custom\_motion\_vector\_precision}{}
\bsITEM{cmvp[\CustomMotionVectorPrecision]}{boolean}{}
\bsIF{cmvp[\CustomMotionVectorPrecision]}
 \bsITEM{cmvp[\MotionVectorPrecision]}{unsigned}{}
 \bsEND
\bsRET{cmvp}
\end{streamsyntax}

\begin{streamsyntax}{global\_motion}{}
\bsITEM{gm[\UsingGlobalMotionFlag]}{boolean}{}
\bsIF{gm[\UsingGlobalMotionFlag]}
 \bsITEM{gm[\GlobalMotionOnlyFlag]}{boolean}{}
 \bsCODE{gm[gm\_params\_ref1] = global\_motion\_parameters()}
 \bsIF{\textit{num\_refs} == 2}
  \bsCODE{gm[gm\_params\_ref2] = global\_motion\_parameters()}
  \bsEND
 \bsEND
\bsRET{gm}
\end{streamsyntax}

\begin{streamsyntax}{custom\_picture\_prediction\_mode}{}
\bsITEM{cppm[\CustomPicturePredictionMode]}{boolean}{}
\bsIF{cppm[\CustomPicturePredictionMode]}
 \bsITEM{cppm[\PicturePredictionModeIndex]}{unsigned}{}
 \bsEND
\bsRET{cppm}
\end{streamsyntax}

\begin{streamsyntax}{custom\_picture\_weights}{}
\bsITEM{cpw[\CustomReferenceWeights]}{boolean}{}
\bsIF{cpw[\CustomReferenceWeights]}
 \bsITEM{cpw[\PictureWeightBits]}{unsigned}{}
 \bsITEM{cpw[\PictureWeightRefA]}{signed}{}
 \bsIF{\textit{num\_refs} == 2}
  \bsITEM{cpw[\PictureWeightRefB]}{signed}{}
  \bsEND
 \bsEND
\bsRET{cpw}
\end{streamsyntax}


%%%%%%%%%%%%%%%%%%%%%%%%%%
\subsubsection{Global motion parameters}

\begin{streamsyntax}{global\_motion\_parameters}{}
\bsCODE{gmp[pan\_tilt] = pan\_tilt()}
\bsCODE{gmp[zoom\_rotate\_sheer] = zoom\_rotate\_sheer()}
\bsCODE{gmp[perspective] = perspective()}
\end{streamsyntax}

\begin{streamsyntax}{pan\_tilt}{}
\bsITEM{pt[\NonZeroPanTiltFlag]}{boolean}{}
\bsIF{pt[\NonZeroPanTiltFlag]}
 \bsITEM{pt[\GMpan]}{signed}{}
 \bsITEM{pt[\GMtilt]}{signed}{}
 \bsEND
\bsRET{pt}
\end{streamsyntax}

\begin{streamsyntax}{zoom\_rotation\_shear}{}
\bsITEM{zrs[\NonZeroZRSFlag]}{boolean}{}
\bsIF{zrs[\NonZeroZRSFlag]}
 \bsITEM{zrs[\ZRSexponent]}{unsigned}{}
 \bsITEM{zrs[\GMaTL]}{signed}{}
 \bsITEM{zrs[\GMaTR]}{signed}{}
 \bsITEM{zrs[\GMaBL]}{signed}{}
 \bsITEM{zrs[\GMaTR]}{signed}{}
 \bsEND
\bsRET{zrs}
\end{streamsyntax}

\begin{streamsyntax}{perspective}{}
\bsITEM{p[\NonZeroPerspectiveFlag]}{boolean}{}
\bsIF{p[\NonZeroPerspectiveFlag]}
 \bsITEM{p[\GMperspectiveExponent]}{unsigned}{}
 \bsITEM{p[\GMperspectiveX]}{signed}{}
 \bsITEM{p[\GMperspectiveY]}{signed}{}
 \bsEND
\bsRET{p}
\end{streamsyntax}

%%%%%%%%%%%%%%%%%%%%%%%%%%
\subsubsection{Wavelet transform}

\begin{verbatim}
wavelet_transform():
    if (is_inter()):
        zero_residual = read_bool()
    if (is_intra || !zero_residual):
        transform_parameters()
        transform_data()
        transform_data()
        transform_data()
\end{verbatim}

\begin{verbatim}
transform_parameters():
    unusual_wavelet_filter()
    unusual_wavelet_depth()
    spatial_partition()
\end{verbatim}

\begin{verbatim}
unusual_wavelet_filter():
    unusual_wavelet = read_bool()
    if (unusual_wavelet):
        wavelet_index = read_uint()
\end{verbatim}

\begin{verbatim}
unusual_wavelet_depth():
    unusual_wavelet_depth = read_bool()
    if (unusual_wavelet_depth):
        transform_depth = read_uegol()
\end{verbatim}

\begin{verbatim}
spatial_partition():
    spatial_partition_flag = read_bool()
    if (spatial_partition_flag):
        custom_partition_flag = read_bool()
        if (custom_partition_flag):
            for level in range(0,transform_depth+1):
                codeblock_size[level] = codeblock_size()
        codeblock_mode_index = read_uegol()
\end{verbatim}
\annotate{df}{should line6 of spatial\_partition be transform\_depth?}

\begin{verbatim}
codeblock_size():
    horizontal_codeblock_size = read_uegol()
    vertical_codeblock_size = read_uegol()
\end{verbatim}
\annotate{df}{i'd flatten codeblock\_size too}

\begin{verbatim}
transform_data():
    subband(0,LL)
    for level in range(1,transform_depth+1):
        for band in [LH, HL, HH]:
            subband(level,band)
\end{verbatim}

\begin{verbatim}
subband(level,band):
    subband_data_length[level][band] = read_uegol()
    if (subband_data_length[level][band] != 0):
        quantiser_index[level][band] = read_uegol()
        BYTEALIGN()
        compressed_subband_data[level][band] = read_chunk(subband_data_length[level][band])
\end{verbatim}


%%%%%%%%%%%%%%%%%%%%%%%%%%
\subsubsection{Syntax of block\_data}

The following is for structural reference only, unmarshalling is
described in section X.

inside the blockdata:

\begin{verbatim}
superblock(xpos,ypos):
    superblock_header(xpos,ypos)
    for y in range (0,block_count,step):
        for x in range (0,block_count,step):
            prediction_unit(xpos+x, ypos+x)
\end{verbatim}

\begin{verbatim}
superblock_header():
    split_residual = read_uegola(sb_split_contexts)
    if (split != 0):
        common_residual = read_boola(sb_common_contexts)
\end{verbatim}

\begin{verbatim}
prediction_unit(xpos,ypos):
    if ((xpos%4==0 && ypos%4==0) || (!common)):
        mode = prediction_mode()
    if (mode == [False,False]): # IntraBlock
        dc = DC_value()
    else:
        if (!using_global || !global_only):
            motion_data(xpos,ypos)
\end{verbatim}
\annotate{df}{use DMT to simplify the using global expression. Also,
using\_global may only be true when global\_only is true}

\begin{verbatim}
prediction_mode(xpos,ypos):
    ref1_mode_residual = read_boola(block_mode_ref1_context)
    if (num_of_refs == 2):
        ref2_mode_residual = read_boola(block_mode_ref2_context)
\end{verbatim}

\begin{verbatim}
DC_value(xpos,ypos):
    luma_dc_residual = read_segola(luma_dc_contexts)
    chroma1_dc_residual = read_segola(chroma1_dc_contexts)
    chroma2_dc_residual = read_segola(chroma2_dc_contexts)
\end{verbatim}

\begin{verbatim}
motion_data(xpos,ypos):
    pu_using_global_residual = read_boola(global_block_context)
    if (mode[0] == True && !pu_using_global):
        ref1_vector = motion_vector_ref1(xpos,ypos)
    if (mode[1] == True && !pu_using_global):
        ref2_vector = motion_vector_ref2(xpos,ypos)
\end{verbatim}

\begin{verbatim}
motion_vector_ref1(xpos,ypos):
    horizontal_residual = read_segola(mv_ref1_horiz_contexts)
    vertical_residual = read_segola(mv_ref1_vert_contexts)
\end{verbatim}

\begin{verbatim}
motion_vector_ref2(xpos,ypos):
    horizontal_residual = read_segola(mv_ref2_horiz_contexts)
    vertical_residual = read_segola(mv_ref2_vert_contexts)
\end{verbatim}


%%%%%%%%%%%%%%%%%%%%%%%%%%
\subsubsection{Syntax of compressed subband data}
inside the compressed subband data:

\begin{verbatim}
subband_data():
    if (num_vertical_codeblocks(level) == 1 && num_horizontal_codeblocks(level) == 1):
        for some loop:
            for some loop:
                wavelet_coefficient()
    else:
        multicodeblock stuff, please see section whatever for details.
\end{verbatim}

\begin{verbatim}
codeblock():
    if (codeblock_mode == SingleQuantiser):
        sq_codeblock()
    if (codeblock_mode == MultipleQuantiser):
        mq_codeblock()
\end{verbatim}

\begin{verbatim}
sq_codeblock():
    if (!single_codeblock(level)):
        zero_block_flag = read_boola(zero_codeblock_context)
    TODO
\end{verbatim}

\begin{verbatim}
wavelet_coefficient():
    data[level][band][v][h] = decode_coefficient()
\end{verbatim}

\begin{verbatim}
decode_coefficient():
    read_coeff(contexts)
\end{verbatim}
\annotate{df}{contexts selection is defined in a different section}

%\subsection{Semantic specification}
The \verb|parse_info_prefix| is the sequence of bytes
``\verb|0x42, 0x42, 0x43, 0x44|''\annotate{df}{tim-77 has the last byte incorrectly as
0x43}, which are the ASCII codes for ``\verb|BBCD|''.

The \verb|parse_code| is a bit field:
bits 7, 6 \& 5 (MSBs) unused, always zero.
bits 4 \& 3 indicate parse unit type (i.e. Access Unit Header, Picture
or End of Sequence), (value 11 undefined).
bits 2, 1 \& 0(LSBs) specify picture properties, (zero for non
pictures).\annotate{df}{do we want to specify these as zero?}
bit 2 indicatesa reference picture (else non reference picture).
bits 1 \& 0 indicate the number of reference pictures (value 11(binary)
undefined).

\verb|video_depth| signifies the number of bits used to code the
uncompressed input signal, typically eight or ten bits.  It would be
possible, for example, to have an 8 bit signal represented in a 10 bit
word (in which case either the upper two, or lower two bits of the word
would always be zero).  The meaning of the bits defined in the Signal
Range part of the Source Parameters.  Video Depth relates to how the
coded.  The Signal Range relates to what the numbers mean and how the
video should be displayed.

The \verb|field_dominance_flag| if asserted means that you do not use
the default field dominance.  If we have an interlaced source, the field
lines can either be interleaved line by line (pseudo-progressive
formate, the default for Dirac), or interleaved field by field
(sequential fields, required for low delay and low resource coding).

The \verb|field_interleaving_flag| indicates non-default field
interleaving, and the \verb|sequential_fields| parameter indicates
whether the fields are interleaved as pseudo-progressive or sequential
fields.  With field sequential coding the picture sequence is a sequence
of fields rather than frames. So for example, for interlaced 625 line
video we would have picture size 720x288, frame rate 25Hz and
\verb|sequential_fields| true.

Field parity is the same as picture parity.  Each picture has a unique
picture number.  If the picture number is even and the picture is a
field, then that field has even parity, i.e. it is the first field of a
pair of fields in a frame.\annotate{df}{Parity, dominance,
topfieldfirst, so many ways to say the same thing}

\verb|aspect_ratio| refers to the pixel aspect ratio, not the image
aspect ratio.\annotate{df}{then lets rename it to
pixel\_aspect\_ratio!}










\appendix
\section{Arithmetic Encoding}
%src: tim-0.9.1.48

This document only specifies the decoding of arithmetic coded data.
However it is important that the encoding process matches the decoding
process. It is not straightforward to derive an implementation of the
encoder by only looking at the decoder specification. Therefore this
informative section describes a possible implementation for the
arithmetic encoder.

An arithmetic encoder would require the following variables.

\begin{verbatim}
Global Variables used for arithmetic encoding
low        #integer
high       #integer
underflow  #integer
\end{verbatim}

In an encoder implementation the following operations must be performed.

\begin{verbatim}
Arirhmetic Encoding Process
initialise_arithmetic_encoding()
.
.
perform arithmetic encoding
.
.
flush_encoder()
\end{verbatim}

Initialisation of arithmetic encoding is straightforward.

\begin{verbatim}
Initialise Arithmetic Encoding Engine

low       = 0x0000
high      = 0xFFFF
underflow = 0x0000
\end{verbatim}

Bits are encoded one at a time based on an estimated probability
embodied in a ``context''. As with decoding, the context simply relates to
counts of the number of zeros or ones that have been previously encoded.

A possible algorithm for encoding a single bit, given a context, is:

Encode Binary Arithmetic Coded Bit
\begin{verbatim}
write_ba(symbol, context):
    while ( ((high&0x4000)==0x0) and
           ((low&0x4000)==0x4000) ):
        code ^= 0x4000
        high ^= 0x4000
        low ^= 0x4000
        shift_bit_out()
    weight = context[0] + context[1]
    scaler = (0x10000+weight//2)//weight

    probability0 = context[0]*scaler
    range = high-low+1
    range_x_prob = (range * probability0)>>16
    if ( symbol )
        context[1] += 1
        low = low + range_x_prob
    else
        context[0] += 1
        high = low  + range_x_prob - 1
    if ( (context[0] + context[1]) > 255 ):
        #Halve counts in the context
        context[0] >> 1
        context[0] += 1
        context[1] >> 1
        context[1] += 1
    while (((high&0x8000)==0x0) and ((low&0x8000)==0x0)):
        output_bits()
        shift_bit_out()
\end{verbatim}

Shift Bit Out
\begin{verbatim}
shift_bit_out():
    high << 1
    high &= 0xFFFF
    high += 1
    low << 1
    low &= 0xFFFF
\end{verbatim}

Output Bits
\begin{verbatim}
output_bits():
    write_bit( (high&0x8000)==0x8000 )
    while ( underflow > 0 ):
        write_bit( high&0x8000)==0x0 )
        underflow -= 1
\end{verbatim}

Flush Encoder
\begin{verbatim}
flush_encoder():

    write_bit( (high&0x4000)==0x4000 )
    while ( underflow >= 0 ):
        write_bit( high&0x4000)==0x0 )
        underflow -= 1
\end{verbatim}

Where the ``write\_bit(bit)'' function outputs a single bit.

\section{Comments on coders}It is not our intention to be prescriptive about how coders should or
should not be organised. Any coder which produces a compliant bytestream
may have a place in someone's work. This Appendix is provided to show
that there are quite a lot of things to consider.

First of all, it is worth noting the competing factors of quality, bit
rate, complexity and delay.

High quality usually requires a high bit rate. To maintain a consistent
quality may require the bit rate to be variable. This leads to a
potential need for a fair amount of buffer storage in a system which
provides a fixed bit rate.

Any system which provides a lot of compression (i.e. a low bit rate)
requires full use of all the tools. When we explore the capacity used by
each element, we find that the Intra frames tend to require most
capacity. Access Units with many Inter frames and few Intra frames will
therefore seem to be a desirable solution - except for the fact that
this will potentially increase the necessary buffer size and lengthen
the time between access points.

Of the data provided in each frame, the data used for motion vectors and
prediction is roughly the same as the quantity of data used for the
wavelet coefficients. The trade off between the two can be enhanced by
careful rate distortion optimisation.

In some low-complexity implementations, it is possible to simplify the
motion vectors, or even omit them.

For low delay systems, a sequence of Intra frames gives the best
performance.


\section{Subband inverse quantizer values}
The inverse quantisation process for reconstructing subband coefficients
requires quantisation factors and offsets derived from an index
quant\_index as specified in table \ref{}.

These values may be calulcated as:

\begin{displaymath}
    \texttt{quant\_factor}= round(2^{\frac{\texttt{quant\_index}}{4}})
\end{displaymath}

\begin{displaymath}
    \texttt{offset}       = round(\texttt{quant\_factor} * 0.375)
\end{displaymath}
    

\begin{figure}[h!]
    \centering
    \begin{tabular}{|c|c|c|}
        \hline
        quant\_index & quant\_factor & offset \\\hline
        0            & 1             & 0      \\\hline
        1            & 1             & 0      \\\hline
        2            & 1             & 0      \\\hline
        3            & 2             & 1      \\\hline
        4            & 2             & 1      \\\hline
        5            & 2             & 1      \\\hline
        6            & 3             & 1      \\\hline
        7            & 3             & 1      \\\hline
        8            & 4             & 2      \\\hline
        9            & 5             & 2      \\\hline
        10           & 6             & 2      \\\hline
        11           & 7             & 3      \\\hline
        12           & 8             & 3      \\\hline
        13           & 10            & 4      \\\hline
        14           & 11            & 4      \\\hline
        15           & 13            & 5      \\\hline
        16           & 16            & 6      \\\hline
        17           & 19            & 7      \\\hline
        18           & 23            & 9      \\\hline
        19           & 27            & 10     \\\hline
        20           & 32            & 12     \\\hline
        21           & 38            & 14     \\\hline
        22           & 45            & 17     \\\hline
        23           & 54            & 20     \\\hline
        24           & 64            & 24     \\\hline
        25           & 76            & 29     \\\hline
        26           & 91            & 34     \\\hline
        27           & 108           & 41     \\\hline
        28           & 128           & 48     \\\hline
        29           & 152           & 57     \\\hline
        30           & 181           & 68     \\\hline
        31           & 215           & 81     \\\hline
        32           & 256           & 96     \\\hline
        33           & 304           & 114    \\\hline
        34           & 362           & 136    \\\hline
        35           & 431           & 162    \\\hline
        36           & 512           & 192    \\\hline
        37           & 609           & 228    \\\hline
        38           & 724           & 272    \\\hline
        39           & 861           & 323    \\\hline
        40           & 1024          & 384    \\\hline
        41           & 1218          & 457    \\\hline
        42           & 1448          & 543    \\\hline
        43           & 1722          & 646    \\\hline
        44           & 2048          & 768    \\\hline
        45           & 2435          & 913    \\\hline
        46           & 2896          & 1086   \\\hline
        47           & 3444          & 1292   \\\hline
        48           & 4096          & 1536   \\\hline
        49           & 4871          & 1827   \\\hline
        50           & 5793          & 2172   \\\hline
        51           & 6899          & 2583   \\\hline
        52           & 8192          & 3072   \\\hline
        53           & 9742          & 3653   \\\hline
        54           & 11585         & 4344   \\\hline
        55           & 13777         & 5166   \\\hline
        56           & 16384         & 6144   \\\hline
        57           & 19484         & 7307   \\\hline
        58           & 23170         & 8689   \\\hline
        59           & 27554         & 10333  \\\hline
        60           & 32768         & 12288  \\\hline
    \end{tabular}
\end{figure}

\clearpage
\section{Default video parameters}\label{videoformatdefaults}
This annex defines the default values of video parameters that are determined by the 
value of the base video format. These defaults reduce overhead by allowing a large 
number of parameters to be set without explicit signaling.

The collection of default values for each value of the base video format constitutes 
a map, which shall be returned by the $set\_source\_defaults(base\_video\_format)$
function and used as a basis for defining the source video
format in the sequence header as per Section \ref{setsourcedefaults}.

All source parameters for any of the predefined video formats may be overridden as required in the sequence header.


\begin{sidewaystable}[!ht]
\begin{tabular}{|r|c|c|c|c|c|c|c|c|c|}
\hline
\multicolumn{10}{|c|}{\cellcolor[gray]{0.75}\bf Video Formats}\\
\hline
{\bf
\begin{tabular}{r}
Base video format\\
index value
\end{tabular}} 
& 0 & 1 & 2 & 3 & 4 &	5 & 6 & 7 & 8 \\
\hline
{\bf Name (informative)}
&Custom & QSIF525 & QCIF & SIF525 & 4CIF &	4SIF525 & 4CIF  & 
\begin{tabular}{c}SD480\\-60I\end{tabular} & \begin{tabular}{c}SD576\\-50I\end{tabular}\\
\hline
{\bf Frame Width:}&640&176&176&352&352&704&704&720&720\\
{\bf Frame Height:}&480&120&144&240&288&480&576&480&576\\
\hline
{\bf Chroma Sampling Format:}&4:2:0&4:2:0&4:2:0&4:2:0&4:2:0&4:2:0&4:2:0&4:2:2&4:2:2\\
\hline
{\bf Source Sampling:} & 0 & 0 & 0 & 0 & 0 & 0 & 0 & 1 & 1 \\
{\bf Top Field First:} & \false & \false & \true & \false & \true & \false & \true & \false & \true \\
\hline
{\bf Frame Rate Index} &1 &9 &10 &9 &10 &9 &10 &4 &3\\
{\bf Numerator} & 24000   & 15000 & 25 & 15000 & 25 & 15000 & 25 & 30000 & 25\\
{\bf Denominator} & 1001 & 1001   &  2   & 1001  &  2 & 1001  & 2     & 1001  & 1\\
\hline
{\bf Aspect Ratio Index} &1 &2 &3 &2 &3 &2 &3 &2 &3\\
{\bf Numerator} & 1   & 10 & 12 & 10 & 12 & 10 & 12 & 10 & 12\\
{\bf Denominator}& 1 & 11 &11 & 11 &  11 & 11 & 11 & 11  & 11\\
\hline
{\bf Clean Width:}&640&176&176&352&352&704&704&704&704\\
{\bf Clean Height:}&480&120&144&240&288&480&576&480&576\\
{\bf Clean Left Offset} & 0 & 0 &0 &0 &0 &0 &0 &8 &8\\
{\bf Clean Top Offset} & 0 &0 &0 &0 &0 &0 &0 &0 &0\\
\hline
{\bf Signal Range Index} &1 &1 &1 &1 &1 &1 &1 &3 & 3\\
{\bf Luma Offset} &0 &0 &0 &0 &0 &0 &0 &64 &64\\
{\bf Luma Excursion} &255 &255 &255 &255 &255 &255 &255 &876 & 876\\
{\bf Chroma Offset} &128 &128 &128 &128 &128 &128 &128 &512 & 512\\
{\bf Chroma Excursion} &255 &255 &255 &255 &255 &255 &255 & 896 & 896\\
\hline
{\bf Colour Specification Index} &0 & 1 & 2& 1&	2&1 & 2& 1&	2\\
&Custom&SDTV 525&SDTV 625&SDTV 525&SDTV 625&SDTV 525&SDTV 625&SDTV 525&SDTV 625\\
\hline
{\bf Colour Primaries Index} &0&	1&2&1&2&1&2&1&2\\
&HDTV&SDTV 525&SDTV 625&SDTV 525&SDTV 625&SDTV 525&SDTV 625&SDTV 525&SDTV 625\\
\hline
{\bf Colour Matrix Index} &0&1&1&1&1&1&1&1&1\\
&HDTV&SDTV&SDTV&SDTV&SDTV&SDTV&SDTV&SDTV&SDTV\\
\hline
{\bf Transfer Function Index} &0&0&0&0&0&0&0&0&0\\
&TV gamma&TV gamma&TV gamma&TV gamma&TV gamma&TV gamma&TV gamma&TV gamma&TV gamma\\
\hline
\end{tabular}
\caption{Predefined video format parameters for video formats 0--8}
\end{sidewaystable}

\begin{sidewaystable}[!ht]
\begin{tabular}{|r|c|c|c|c|c|c|c|c|}
\hline
\multicolumn{9}{|c|}{\cellcolor[gray]{0.75}\bf Video Formats}\\
\hline
{\bf\begin{tabular}{r}
Base video format\\
index value
\end{tabular}} 
& 9 & 10 & 11 & 12 & 13 &	14 & 15 & 16 \\
\hline
{\bf Name (informative)} & HD720P-60& HD720P-50 & HD1080I-60& HD1080I-50&
HD1080P-60 &HD1080P-50& DC2K &DC4K \\
\hline
{\bf Frame Width:}&1280&1280&1920&1920&1920&1920&2048&4096\\
{\bf Frame Height:}&720&720&1080&1080&1080&1080&1080&2160\\
\hline
{\bf Chroma Sampling Format:}&4:2:2&4:2:2&4:2:2&4:2:2&4:2:2&4:2:2&4:4:4&4:4:4\\
\hline
{\bf Source Sampling:} & 0 & 0 & 1 & 1 & 0 & 0 & 0 & 0 \\
{\bf Top Field First:} & \true & \true & \true & \true & \true & \true & \true & \true \\
\hline
{\bf Frame Rate Index} &7 &6 &4&3 &7 &6 &2&2 \\
{\bf Numerator} & 60000   & 50 & 30000 & 25 & 60000 & 50 & 24 & 24\\
{\bf Denominator} & 1001 & 1   &  1001   & 1  &  1001 & 1  & 1     &  1\\
\hline
{\bf Pixel Aspect Ratio Index} &1 &1 &1 &1 &1 &1 &1 &1\\
{\bf Numerator} & 1   & 1 & 1 & 1 & 1 & 1 & 1 & 1\\
{\bf Denominator}& 1 & 1 &1 & 1 &  1 & 1 & 1 & 1\\
\hline
{\bf Clean Width}&1280&1280&1920&1920&1920&1920&2048&4096\\
{\bf Clean Height}&720&720&1080&1080&1080&1080&1080&2160\\
{\bf Clean Left Offset} & 0 & 0 &0 &0 &0 &0 &0 &0 \\
{\bf Clean Top Offset} & 0 &0 &0 &0 &0 &0 &0 &0 \\
\hline
{\bf Signal Range Index} &3 &3 &3 &3 &3 &3 &4 &4 \\
{\bf Luma Offset} &64 &64 &64 &64 &64 &64 &256 &256\\
{\bf Luma Excursion} &876 &876 &876 &876 &876 &876 &3504 &3504\\
{\bf Chroma Offset} &512 &512 &512 &512 &512 &512 &2048 &2048\\
{\bf Chroma Excursion} &896 &896 &896 &896 &896 &896 &3584 & 3584\\
\hline
{\bf Colour Specification Index} &3 & 3 & 3& 3&	3&3 & 4& 4\\
&HDTV&HDTV&HDTV&HDTV&HDTV&HDTV&Cinema&Cinema\\
\hline
{\bf Colour Primaries Index} &0&	0&0&0&0&0&3&3\\
&HDTV&HDTV&HDTV&HDTV&HDTV&HDTV&Cinema&Cinema\\
\hline
{\bf Colour Matrix Index} &0&0&0&0&0&0&0&0\\
&HDTV&HDTV&HDTV&HDTV&HDTV&HDTV&HDTV&HDTV\\
\hline
{\bf Transfer Function Index} &0&0&0&0&0&0&0&0\\
&TV gamma&TV gamma&TV gamma&TV gamma&TV gamma&TV gamma&TV gamma&TV gamma\\
\hline
\end{tabular}
\caption{Predefined video format parameters for video formats 9--16}
\end{sidewaystable}

\begin{sidewaystable}[!ht]
\begin{tabular}{|r|c|c|c|c|c|c|c|c|}
\hline
\multicolumn{5}{|c|}{\cellcolor[gray]{0.75}\bf Video Formats}\\
\hline
{\bf\begin{tabular}{r}
Base video format\\
index value
\end{tabular}} 
& 17 & 18 & 19 & 20 \\
\hline
{\bf Name (informative)} & UHDTV 4K-60& UHDTV 4K-50 & UHDTV 8K-60& UHDTV 8K-50\\
\hline
{\bf Frame Width:}&3840&3840&7680&7680\\
{\bf Frame Height:}&2160&2160&4320&4320\\
\hline
{\bf Chroma Sampling Format:}&4:2:2&4:2:2&4:2:2&4:2:2\\
\hline
{\bf Source Sampling:} & 0 & 0 & 0 & 0 \\
{\bf Top Field First:} & \true & \true & \true & \true \\
\hline
{\bf Frame Rate Index} &7 &6 &7 &6 \\
{\bf Numerator} & 60000   & 50 & 60000 & 50\\
{\bf Denominator} & 1001 & 1   &  1001   & 1\\
\hline
{\bf Pixel Aspect Ratio Index} &1 &1 &1 &1\\
{\bf Numerator} & 1   & 1 & 1 & 1\\
{\bf Denominator}& 1 & 1 &1 & 1\\
\hline
{\bf Clean Width}&3840&3840&7680&7680\\
{\bf Clean Height}&2160&2160&4320&4320\\
{\bf Clean Left Offset} & 0 & 0 &0 &0\\
{\bf Clean Top Offset} & 0 &0 &0 &0\\
\hline
{\bf Signal Range Index} &3 &3 &3 &3 \\
{\bf Luma Offset} &64 &64 &64 &64\\
{\bf Luma Excursion} &876 &876 &876 &876\\
{\bf Chroma Offset} &512 &512 &512 &512\\
{\bf Chroma Excursion} &896 &896 &896 &896\\
\hline
{\bf Colour Specification Index} &3 & 3 & 3& 3\\
&HDTV&HDTV&HDTV&HDTV\\
\hline
{\bf Colour Primaries Index} &0&0&0&0\\
&HDTV&HDTV&HDTV&HDTV\\
\hline
{\bf Colour Matrix Index} &0&0&0&0\\
&HDTV&HDTV&HDTV&HDTV\\
\hline
{\bf Transfer Function Index} &0&0&0&0\\
&TV gamma&TV gamma&TV gamma&TV gamma\\
\hline
\end{tabular}
\caption{Predefined video format parameters for video formats 17--20}
\end{sidewaystable}





\clearpage
\section{Video systems model}\label{vidsys}

\begin{informative*}
\subsection{Video systems model for interpreting source parameters (Informative)}
\label{vidsysmodel}

The interpretation of sequence source parameters (Section \ref{sourceparameters})
by a display mechanism interfacing with a compliant decoder is non-normative. However, it
should where possible follow the recommendations and interpretations
specified in this section. Likewise, encoders should ensure that
accurate display parameter information is encoded to maximise the
quality of displayed video.
\subsubsection{Colour}
All current video systems use the following model for luminance/chrominance 
(`YUV')coding of RGB values (computer systems often omit coding to and from YUV). 

The R, G and B are tristimulus values (e.g. candelas/$m^2$). Their
relationship to CIE XYZ tristimulus values can be derived from the set
of primaries and white point defined in the colour primaries part of the
colour specification below using the method described in SMPTE RP
177-1993. In this discussion the RGB values are normalised to the range
[0,1], so that RGB=[1,1,1] represents the peak white of the display device
and RGB=0,0,0 represents black.

Values $E_R$, $E_G$ and $E_B$ are are defined which are related to the RGB 
values by non-linear forward and inverse transfer functions $f()$ and $g()$ respectively
(Figure \ref{fig:transferchar}).

[figure \label{fig:transferchar}]

Normally, $E_R$, $E_G$ and $E_B$ also fall in the range $[0,1]$, but in the
case of extended gamut, negative values may be allowed also. The
transfer function $f()$ is typically performed in the camera and
is specified in the Transfer Characteristic part of the
Colour Specification. For aesthetic and psychovisual reasons
the transfer function $g()$ is not quite the inverse of
$f()$. In fact the combined effect of $f()$ and
$g()$ is such that:
\[g(f(x))=x^\gamma\]

where $\gamma$ is the `rendering intent' or end to end gamma of the
system, which may vary between about 1.1 and 1.6 depending on viewing
conditions. The rationale for this is given in `Digital Video and HDTV',
Charles Poynton 2003, Morgan Kaufmann Publishers, ISBN 1-55860-792-7.

The non-linear $E_R$, $E_G$ and $E_B$ values are subject to a matrix operation
(known as `non-constant luminance coding'), which transforms
them into luma ($E_Y$) and chroma (normally $E_{Cb}$ and $E_{Cr}$) values. 
$E_Y$ is normally limited to the 
range $[0,1]$ and the chroma
values to the range $[-0.5, 0.5]$. This is YUV coding and
sometimes the chroma components are subsampled, either horizontally or
both horizontally and vertically. UV sampling is specified by the
chroma format value. 

\paragraph{Conventional YUV coding}
$\ $\newline
In conventional YUV coding, the $E_Y$, $E_{Cb}$ and $E_{Cr}$ values are
mapped to a range of integers denoted $Y$, $C1$ and $C2$ within this
specification. In this way, $C1$ typically corresponds to $Cb$ and
$C2$ to $Cr$. The way this mapping occurs is defined by the signal
range parameters (Section \ref{signalranges}). It is these integer values 
that are actually output from the decoder. In order to display video, the inverse to the above
operations must be performed to convert this data to $E_Y$, $E_{Cb}$, $E_Cr$, then
to $E_R$, $E_G$, $E_B$ and thence to R, G and B.  

\paragraph{YCoCg coding}
$\ $\newline
The E values can be viewed as something of a mathematical abstraction.
For example in digital display devices, R, G and B values are specified
in terms of integer levels which are derived from the integral luma and
chroma values by direct operations subsuming and approximating all the
real-number operations described here. Generally, these approximations
cause loss through quantisation of intermediate values, and the
restriction of values to particular ranges also restricts the colour
gamut. 

In the case of YCoCg coding, the $E_R$, $E_G$ and $E_B$ values are directly
linearly scaled to integer ranges $[0,2^\LumaDepth-1]$ before a lossless 
direct integer transform is applied to convert this data to $Y$, $C1$ (representing
Co) and $C2$ (representing $Cg$) data. This transform is described in Section
\ref{colourmatrix}. This supports efficient lossless RGB coding.

\subsubsection{Signal range}
\label{signalranges}

Video signals within the decoder are bi-polar. Offset  and excursion values 
$\SLumaOffset$, $\SLumaExcursion$, $\SChromaOffset$ and
$\SChromaExcursion$ should be used to convert the
integer-valued decoded luma and chroma data $Y$, $C1$ ($Cb$) and $C2$ ($Cr$) 
to intermediate values $E_Y$, $E_{Cb}$, and $E_{Cr}$ by the recipe
\begin{eqnarray*}
E_Y & = & \dfrac{Y}{\SLumaExcursion}+dfrac{\SLumaOffset}{2^\LumaDepth} = \dfrac{Y}{\SLumaExcursion}+0.5\\
E_{Cb} & = & \dfrac{C1+\SChromaOffset}{\SChromaExcursion} \\
E_{Cr} & = & \dfrac{C2+\SChromaOffset}{\SChromaExcursion}
\end{eqnarray*}

$E_Y$ is normally clipped to the range $[0,1]$ and $E_{Cb}$, $E_{Cr}$
to the range $[-0.5,0.5]$. This effectively clips integer $Y$ values within
the codec to 
the interval
\[ [-\SLumaExcursion/2, \SLumaExcursion/2] \]
and $C1$, $C2$ values to
\[ [-\SChromaExcursion/2,\SChromaExcursion/2] \]

However, maintaining an extended RGB gamut may mean that either such
clipping is not done, or non-standard offset and excursion values are
used to extract the extended gamut from the non-negative $Y$, $C1$,
and $C2$ values.

In the case of $YCoCg$ coding, [tbc]

\begin{comment}
Non-default offset and excursion values cannot be coded if the chroma
format is YCoCg: default parameters should be used. However, even in
this case, EY, ECo, and ECg should not be calculated. Instead, direct
integer conversion to RGB should be done as described in Section . (In
fact, excursion values will be ignored in this integer conversion.)
\end{comment}

\subsubsection{Primaries}
\label{primaries}
The colour primaries allow device dependent linear RGB colour
co-ordinates to be mapped to device independent linear CIE XYZ space.
The primaries specified below are the CIE (1931) XYZ chromaticity
co-ordinates of the primaries and the white point of the device. The
maths required to convert between RGB and XYZ is reproduced below. $V_X$,
$V_Y$ and $V_Z$ are the XYZ coordinates of value $V$, for $V$ equal to
the device-dependent red, green, blue or white value.

\begin{eqnarray*}
F & = &
\left(
    \begin{array}{ccc}
    \dfrac{R_X}{R_Y} & \dfrac{G_X}{G_Y} & \dfrac{B_X}{B_Y} \\
    1 & 1 & 1 \\
    \dfrac{1-R_X-R_Y}{R_Y} & \dfrac{1-G_X-G_Y}{G_Y} & \dfrac{1-B_X-B_Y}{B_Y}
    \end{array}
\right)
\\
\left(
    \begin{array}{c}
    s_r \\
    s_g \\
    s_b
    \end{array}
\right ) & = & F^{-1}
\left(
    \begin{array}{c}
    \dfrac{W_X}{W_Y} \\
    1 \\
    \dfrac{1-W_X-W_Y}{W_Y}
    \end{array}
\right) 
\\
\left(
    \begin{array}{c}
    X \\
    Y \\
    Z
    \end{array}
\right) & = & 
\left(
    \begin{array}{c}
    s_r*R \\
    s_g*G \\
    s_b*B
    \end{array}
\right)
\end{eqnarray*}

The colour primary specification therefore allows exact colour
reproduction of decoded RGB values on different displays with different
display primaries. (Although it has to be said that often conversion between
encoded primaries and display primaries is not done.)

\subsubsection{Matrix}
\label{matrix}
\paragraph{Conventional YUV coding}
$\ $\newline
Unit-scale luma and chroma values $E_Y$, $E_{Cb}$ and $E_{Cr}$ should be
derived from decoded $Y$, $C1$ and $C2$ values using the signal range parameters
as per Section \ref{signalranges}. Given these values, $E_R$, $E_G$ and $E_B$ are
determined as follows:
\begin{eqnarray*}
E_R & = & E_Y + 2*(1-K_R)*E_{Cr} \\
E_G & = & E_Y - \dfrac{2*K_R*(1-K_R)*E_{Cr}}{K_G}-\dfrac{2*K_B*(1-K_B)*E_{Cb}}{K_G} \\
E_B & = & E_Y + 2*(1-K_R)*E_{Cb} 
\end{eqnarray*}
where $K_G=1-K_R-K_B$.
This follows by inverting the equations 
\begin{eqnarray*}
K_R+K_G+K_B & = & 1 \\
E_Y & = & K_R*E_R+K_G*E_G+K_B*E_B \\
E_{Cb} & = & \dfrac{E_B - E_Y}{2*(1-K_B)} \\
E_{Cr} & = & \dfrac{E_R - E_Y}{2*(1-K_R)} \\
\end{eqnarray*}

\paragraph{YCoCg coding}
$\ $\newline
In the case of YCoCg coding, integer $I_R$, $I_G$, $I_B$ should be directly computed from
the decoded $Y$, $C1$ ($Co$) and $C2$ ($Cg$) values by
\begin{eqnarray*}
Y & -= & \SLumaOffset \\
Co=C1 & -= & \SChromaOffset \\
Cg=C2 & -= & \SChromaOffset \\
t & = & Y-(Cg\gg1) \\
I_G & = & t+Cg \\
I_B & = & t-(Co\gg1) \\
I_R & = & I_B+Co
\end{eqnarray*}
The integer values are converted to unit-scale $E_R$, $E_G$, $E_B$ by dividing by 
$2^\LumaDepth$ and clipping to $[0,1]$.
If the inverse transform has been correctly
applied prior to coding and lossless coding employed, then clipping will
be unnecessary, and reversing the above operations will reproduce $Y$, $Co$ and $Cg$
losslessly from $I_R$, $I_G$ and $I_R$ yielding a transparent RGB to RGB coding system:
\begin{eqnarray*}
Co & = & I_R-I_B \\
t & = & I_B+(I_R-I_B)\gg1 \approx (I_R+I_B)/2\\
Cg & = & I_G-t = \approx I_G-(I_R+I_B)/2\\
Y & = & t+(Cg\gg1) \approx I_G/2-(I_R+I_B)/4+(I_R+I_B)/2=I_R/4+I_G/2+I_B/4
\end{eqnarray*}

Note that these matrix operations give that the chroma data requires an additional bit, due to the subtractions used to create chroma components. 
So for 8-bit RGB ($I_R$, $I_G$, $I_B$) values, $Y$ will be 8 bits and $Co$ and
$Cg$ will be 9 bits. 

Hence $\ChromaDepth$ should be set to $1+\LumaDepth$.

\subsubsection{Transfer characteristics}
\paragraph{TV transfer characteristic}
$\ $\newline

Denoting R or G or B as `$L$' (light) and $E_R$, $E_G$, $E_B$ as
`$E$' then $E=f(L)$ such that
\[
E=\left\{
        \begin{array}{ll}
        4.5L & 0\leq L<0.18\\
        1.099*L^{0.45} &0.18\leq L \leq 1
        \end{array}
  \right.
\]

All modern TV systems use this transfer characteristic at present. ITU-R
BT 470 (`Conventional Television systems PAL, NTSC and SECAM') specifies
an ``assumed gamma value of the receiver for which the primary
signals are pre-corrected'' as 2.2 for NTSC and 2.8 for PAL. This
specification is incomplete, incorrect and obsolete and modern PAL and
NTSC systems use the `TV' transfer characteristic above.

\paragraph{Extended Colour Gamut}
$\ $\newline

ITU-R BT 1361, `Worldwide unified colorimetry of future TV systems'
defines a transfer characteristic for systems with an extended colour
gamut as follows.

Denoting R or G or B as `$L$' (light) and $E_R$, $E_G$, $E_B$ as
`$E$' then $E=f(L)$ such that
\[
E=\left\{
        \begin{array}{ll}
        -\dfrac{1.099*(-4*L)^{0.45}-0.099}{4} & -0.25\leq L<-0.0045\\
        4.5L & -0.0045\leq L<0.18\\
        1.099*L^{0.45} &0.18\leq L \leq 1.33
        
        
        \end{array}
  \right.
\]

This transfer characteristic is intended to be used with systems using
an extended colour gamut.

\paragraph{Linear}
$\ $\newline
A linear transfer characteristic has $f(x)=x$. 

\subsubsection{Frame rate}
The ratio of the frame rate values $\SFrameRateNumer$ and $\SFrameRateDenom$
 encodes the intended rate at which frames should be
displayed subsequent to decoding. If $\SSourceSampling$ is 1 (interlaced sampling),
 then fields are displayed at double the frame rate, in the order specified by the
$\STopFieldFirst$ flag.

\subsubsection{Aspect ratios and clean area}
The aspect ratio of an image is the ratio of the intended
spacing of horizontal samples (pixels) to the spacing of vertical
samples (picture lines) on the display device. Pixel aspect ratios are
fundamental properties of sampled images because they determine the
displayed shape of objects in the image. Failure to use the right value
of will result in distorted images for example,
circles will be displayed as ellipses and so forth. HDTV standards and 
computer image formats are generally defined to have pixel
aspect ratios that are exactly 1:1.

The clean area defines an area of pixels within the picture which
should be displayed -- other pixels outside the area should not be
displayed. In particular, the clean area can be used to suppress artefacts 
near the picture edges: at high levels of compression it may only be appropriate
 to display the clean area rather than the whole picture. 

The top-left corner of the clean area has coordinates
\[(\SLeftOffset,\STopOffset)\]
counting from the top-left corner of the picture data, and
dimensions $\SCleanWidth$ by $\SCleanHeight$.

Note that these dimensions refer to pixels within a picture, not a frame,
so a change from interlaced to progressive picture coding will
necessitate a change of clean area if a custom clean area is used.

The clean area and the pixel aspect ratio together determine the
aspect ratio of the displayed image which is the ratio of the width of the intended
display area to the height of the intended display area:
\[\dfrac{\SCleanWidth*\SAspectRatioNumer}{\SCleanHeight*\SAspectRatioDenom}\]

Given two separate sequences, with identical image aspect ratio, if the
top left corner and bottom right corners of their clean apertures are
coincident when displayed, then the images as a whole should be exactly
coincident. This is regardless of the actual pixel dimensions of the
images or their clean areas. This allows sequences to be combined
together appropriately if they are appropriately scaled.

The defined pixel aspect ratios are designed to give standard image
aspect ratios for typical TV broadcasts. For example, for a 525 line
(American) 704 x 480 (clean area)  picture the image aspect ratio is
(704 x 10)/(480 x 11) which is exactly 4:3.

For 625 line systems the 12:11 pixel aspect ratio means (less
conveniently) that a 704x576 image would have an exact 4:3 image
aspect ratio. It might be argued that the pixel aspect ratio for 625
line systems should be such that a 702x576 image would have an exact 4:3
image aspect ratio, corresponding to the nominal sample rate for 625 line TV.
This requirement would lead to a
pixel aspect ratio of 128:117. However, the tolerance of the analogue
line length is $702\pm 3$ pixels, which does not really seem to justify a
ratio of exactly 128:117.

The values specified here are generally agreed to be the
`correct' values. Then again not everyone agrees with this
consensus. These arise from the `industry standard' sampling
frequencies used for square pixels, which were originally designed for
digitising composite analogue video signals. These `industry
standard' sampling frequencies are 11+3/11 MHz for 525 line systems
and 14.75MHz for 625 line systems. The ratio of these frequencies to the
(standardised) 13.5MHz sampling frequency used for broadcasting (approximately) 
yields the pixel aspect ratios given.

You are strongly advised to use one of the default pixel aspect ratios.
However, if you know what you are doing and don't like the default
values you can define your own ratio. You should be aware that many
display devices may ignore your decision and may use different and
unsuitable values. 

\end{informative*}
\subsection{Colour}

\subsection{Frame rate}

\subsection{Aspect ratios and clean area}

\subsection{Signal range}

\subsection{Colour primaries}

\subsection{Colour matrix}

\subsection{Transfer charactaristics}


\clearpage
\section{genralized decoder}
\begin{verbatim}
mainloop:
  sync()
  pi = parse_info()
  case pi.parse_code of:
      AccessUnit -> state <- access_unit()
      Frame      -> frame <- frame_()
                    process_frame(frame, state)


process_frame:
  
\end{verbatim}


\clearpage
\section{References}
\annotate{shas}{I hope we do not have any references - it would be nice
to be freestanding.}

\clearpage
\printindex
\end{document}
