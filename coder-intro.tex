It is not our intention to be prescriptive about how coders should or
should not be organised. Any coder which produces a compliant bytestream
may have a place in someone's work. This Appendix is provided to show
that there are quite a lot of things to consider.

First of all, it is worth noting the competing factors of quality, bit
rate, complexity and delay.

High quality usually requires a high bit rate. To maintain a consistent
quality may require the bit rate to be variable. This leads to a
potential need for a fair amount of buffer storage in a system which
provides a fixed bit rate.

Any system which provides a lot of compression (i.e. a low bit rate)
requires full use of all the tools. When we explore the capacity used by
each element, we find that the Intra frames tend to require most
capacity. Access Units with many Inter frames and few Intra frames will
therefore seem to be a desirable solution - except for the fact that
this will potentially increase the necessary buffer size and lengthen
the time between access points.

Of the data provided in each frame, the data used for motion vectors and
prediction is roughly the same as the quantity of data used for the
wavelet coefficients. The trade off between the two can be enhanced by
careful rate distortion optimisation.

In some low-complexity implementations, it is possible to simplify the
motion vectors, or even omit them.

For low delay systems, a sequence of Intra frames gives the best
performance.

