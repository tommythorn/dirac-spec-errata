This sections specifies the overall decoding process for coefficients
within a subband. Data within a subband is divided into code blocks,
representing rectangular blocks of coefficients. If SPATIAL\_PARTITION is
TRUE, then subbands may be divided into more than one code block,
otherwise only a single block is used. The horizontal number of code
blocks is XNUM\_CODEBLOCKS. The vertical number of code blocks is
YNUM\_CODEBLOCKS. 

The value of XNUM\_CODEBLOCKS and YNUM\_CODEBLOCKS is derived as per
Section . 

Before any coefficient data is decoded, a count of coefficients is
initialised by

COEFF\_COUNT=0

A reset interval COUNT\_RESET is set by 

COUNT\_RESET=max( 25, min( (SUBBAND\_WIDTH*SUBBAND\_HEIGHT)//32, 800 ) )

Code blocks are decoded in raster order. In pseudocode,

for (v=0 ; v<YNUM\_CODEBLOCKS ; ++v)

{

    for (u=0 ; u<XNUM\_CODEBLOCKS ; ++u)

        decode\_code\_block(u,v)

}

The state of the arithmetic decoder, including all contexts is
maintained between codeblocks.



