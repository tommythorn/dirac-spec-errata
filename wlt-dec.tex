%%%%%%%%%%%%%%%%%%%%%%%%%%%%%%%%%%%%%%%%%%%%%%%%%%%%%
% - This chapter defines how wavelet coefficients - %
% - are decoded                                   - % 
%%%%%%%%%%%%%%%%%%%%%%%%%%%%%%%%%%%%%%%%%%%%%%%%%%%%%

\label{transformdec}

This section specifies how wavelet transform coefficients are parsed.


\subsection{Decoded subband data conventions}
\label{wltdecodeconventions}

\subsubsection{Wavelet data initialisation}

\label{wltinit}

This section specifies the $initialise\_wavelet\_data(comp)$ process, which returns a structure which will
contain the decoded wavelet coefficients for the component indicated by $comp$. 

For the purposes of this specification, this is a four-dimensional array $data$,
where individual subbands are two-dimensional arrays accessed by level and orientation:

$band = data[level][orient]$

Valid levels are integers from in the range 0 to $\TransformDepth$ inclusive. 
Level 0 consists of a single subband with orientation \LL. 
All other levels consist of 3 subbands of orientation \LH, \HL, 
and \HH. The orientations correspond to either low- or high-pass filtering
horizontally and vertically: so the \LH band consists of coefficients derived
from horizontal low-pass filtering and vertical high-pass filtering. The subbands
partition the spatial frequency domain by orientation and level as illustrated
in Figure \ref{fig:orientlevel}.

Each subband array is initialised so that:
\begin{eqnarray*}
\width(data[level][orient]) & = & subband\_width(level,comp) \\
\height(data[level][orient]) & = & subband\_height(level,comp)
\end{eqnarray*}

as specified in Section \ref{subbandwidthheight}. These dimensions correspond 
to a wavelet transform being performed on a copy of the component data which 
has been padded (if necessary) so that its
dimensions are a multiple of $2^{\TransformDepth}$.

Individual subband coefficients are signed integers accessed by vertical and 
horizontal coordinates within the subband
\[c = data[level][orient][y][x]\]

where the range of allowable coordinates for a subband coefficient is $0\leq x<subband\_width(level,comp)$ and
$0\leq y<subband\_height(level,comp)$.

\setlength{\unitlength}{1em}
\begin{figure}[!h]
\centering
\begin{picture}(25,35)

%Main Grid
\multiput(0,0)(16,0){3}%
  {\line(0,1){32}}
\multiput(0,0)(0,16){3}%
  {\line(1,0){32}}

%Second Grid
\put(0,24){\line(1,0){16}}
\put(8,16){\line(0,1){16}}

%Third Grid
\put(0,28){\line(1,0){8}}
\put(4,24){\line(0,1){8}}

%Fourth Grid
\put(0,30){\line(1,0){4}}
\put(2,28){\line(0,1){4}}


%Put Levels

\put(24,34){Level 4}
\put(23,36){\vector(0,-1){3}}

\put(13,34){Level 3}
\put(12,36){\vector(0,-1){3}}

\put(7,34){2}
\put(6,36){\vector(0,-1){3}}

\put(4,34){1}
\put(3,36){\vector(0,-1){3}}

%Put decomposition numbers
\put(23,24){4-HL}
\put(7,8){4-LH}
\put(23,8){4-HH}
\put(11,28){3-HL}
\put(3,20){3-LH}
\put(11,20){3-HH}
\put(5,30){2-HL}
\put(1,26){2-LH}
\put(5,26){2-HH}
\put(2.2,31){\tiny 1-HL}
\put(0.2,29){\tiny 1-LH}
\put(2.2,29){\tiny 1-HH}
\put(0.2,31){\tiny 0-LL}

  
\end{picture}
\caption{Subband decomposition of the spatial frequency domain showing subband 
numbering, for a 4-level wavelet decomposition}\label{fig:orientlevel}

\end{figure}

\subsubsection{Dimensions of wavelet subbands}
\label{subbandwidthheight}

This section defines the values of the $subband\_width(level, comp)$ and $subband\_height(level,comp)$
functions, giving the width and height of subbands at a given level for a given component, and hence the range
of subband vertical and horizontal indices. 

If $comp==Y$, set
\begin{eqnarray*}
w & = & \LumaWidth \\
h & = & \LumaHeight
\end{eqnarray*}

Otherwise, set
\begin{eqnarray*}
w & = & \ChromaWidth \\
h & = & \ChromaHeight
\end{eqnarray*}

Define the padded dimensions of the component by
\begin{eqnarray*}
pw = 2^{\TransformDepth}*\left\lceil\frac{w}{2^{\TransformDepth}}\right\rceil\\ 
ph = 2^{\TransformDepth}*\left\lceil\frac{h}{2^{\TransformDepth}}\right\rceil
\end{eqnarray*}

If $level==0$,
\begin{eqnarray*}
subband\_height(level) & = & ph//2^{\TransformDepth} \\
& = & \left\lceil\frac{\ComponentHeight}{2^{\TransformDepth}}\right\rceil \\
subband\_width(level) & = & pw//2^{\TransformDepth} \\
& = & \left\lceil\frac{\ComponentWidth}{2^{\TransformDepth}}\right\rceil
\end{eqnarray*}

If $level>0$
\begin{eqnarray*}
subband\_height(level) & = & ph//2^{\TransformDepth-level+1} \\
& = & 2^{level-1}*\left\lceil\frac{\ComponentHeight}{2^{\TransformDepth}}\right\rceil \\
subband\_width(level) & = & pw//2^{\TransformDepth-level+1} \\
& = & 2^{level-1}*\left\lceil \frac{\ComponentWidth}{2^{\TransformDepth}}\right\rceil
\end{eqnarray*}

\begin{informative}
In encoding, these padded dimensions may be achieved by padding the 
component data up to the padded dimensions and applying the forward
Discrete Wavelet Transform (the inverse of the operations specified in
Section \ref{idwt}). Any values may be used for the padded data, although
the choice will affect wavelet coefficients at the right and bottom 
edges of the subbands. Good results, in compression terms, may be obtained
 by using edge extension for intra pictures and zero extension for inter 
pictures. A poor choice of padding may cause visible artefacts near the
bottom and right edges at high levels of compression.
\end{informative}

\subsection{Wavelet data decoding process}

\subsubsection{Summary}

\label{transformdata}

This section specifies the overall operation of the $transform\_data(comp)$ process
for parsing and decoding the set of coefficient subbands corresponding
to a video picture component (Y, U or V), according to the conventions set out
in Section \ref{wltdecodeconventions}.

In the Dirac stream, subband data is arranged by level and orientation, from
level 0 up to level \TransformDepth. The decoding process for each subband is
contingent on data from subbands of the same orientation in the next lower level.
This is the {\em parent} subband; the subband of the same orientation in the next
higher level is the {\em child} subband. 

Decoding an individual subband therefore requires prior decoding of the parent subband,
and of its parent, and so on until level 1 is reached (level 1 subbands do not depend
upon the single level 0 DC band).

\begin{informative}
The data for each subband consists of a subband header and a block of arithmetically
coded coefficient data. The subband header contains a length code giving the number of
bytes of the block of arithmetically-coded data. The transform data can therefore be
parsed without invoking arithmetic decoding at all, since the length codes allow a 
parser to skip from one subband header to the next, similarly to the way that parse unit
offsets allow frame skipping.
\end{informative}

\subsubsection{Decoding loop}

The overall $transform\_data()$ decoding process is as follows:

\begin{pseudo}{transform\_data}{comp}
\bsCODE{data=initialise\_wavelet\_data(comp)}{\ref{wltinit}}
\bsCODE{subband(data, 0, \LL)}{\ref{subbanddecodeprocess}}    
\bsFOR{level=1}{\TransformDepth}
  \bsFOREACH{orient}{\LH,\HL,\HH}
    \bsCODE{byte\_align()}
    \bsCODE{subband(data, level, orient)}{\ref{subbanddecodeprocess}}
  \bsEND
\bsEND
\bsRET{data}
\end{pseudo}

\subsection{Subband decoding process}

\label{subbanddecodeprocess}

This section specifies the process $subband(level,orient)$ for coefficients
within a subband at level $level$ ($0$ to \TransformDepth) and of orientation $orient$
(\LL, \LH, \HL, or \HH). 

\subsubsection{Subband header and codeblock loop}

This section specifies the operation of the $subband(data, level, orient)$
function for decoding a subband at level $level$ and of orientation $orient$.

\paragraph{Initialisation\newline}

The $subband()$ process begins by reading a length code. If this length is
zero, then the subband is deemed to be skipped and all coefficients are set to zero
before the process exits:

\begin{pseudo}{subband}{data, level, orient}
\bsITEM{length}{uint}
\bsIF{length == 0}
  \bsFOR{y=0}{\height(data[level][orient])-1}
    \bsFOR{x=0}{\width(data[level][orient])-1}
    \bsCODE{data[level][orient][y][x] = 0}
    \bsEND
  \bsEND
  \bsRET{}
\bsEND
\bsCODE{\hdots}
\end{pseudo}

If $length>0$ then the subband coefficient decoding process is initialised by
setting up arithmetic decoding, initialising the coefficient count, reading
the quantisation index and byte-aligning the subsequent read operations.

\begin{pseudo*}
\bsIF{length>0}
    \bsCODE{initialise\_arithmetic\_decoding(length)}{\ref{initarith}}
    \bsCODE{\CoefficientCount=0}
    \bsCODE{vol=\height(data[level][orient])*\width(data[level][orient])}
    \bsCODE{\CoefficientReset=max(min(vol//32, 800), 25)}
    \bsCODE{quant\_index = read\_uint()}
\bsEND
\bsCODE{byte\_align()}
\bsCODE{\hdots}
\end{pseudo*}

\paragraph{Codeblocks\newline}

Data within a subband is divided into code blocks,
representing rectangular blocks of coefficients. The numbers of codeblocks
in each subband are determined in decoding the transform header, as specified
in Section \ref{spatialpartition}.

The overall subband decoding process loops over all the code blocks after initialising the
arithmetic decoding engine, and setting quantisers. There is a different code block decoding
process for Intra DC bands, since values are coded using spatial prediction.

\begin{pseudo*}
\bsIF{level>1}
\bsCODE{parent = data[level-1][orient]}
\bsELSE
    \bsCODE{parent = \emptyset}
\bsEND
\bsFOR{y=0}{\Codeblocks[level][v]-1}
    \bsFOR{x=0}{\Codeblocks[level][h]-1}
        \bsCODE{codeblock(data[level][orient],parent,level, orient,quant\_index,y,x)}{\ref{codeblocks}}
    \bsEND
\bsEND

\bsIF{is\_intra() \text{ and } level==0}
    \bsCODE{intra\_dc\_prediction(band)}{\ref{intradcprediction}}
\bsEND
\bsCODE{flush\_inputb()}{\ref{blockreadbit}}
\end{pseudo*}

\subsubsection{Decoding subband codeblocks}

\label{codeblocks}

This section defines the operation of the 
$codeblock(band,parent,level, orient,quant\_index,y,x)$ function, which decodes a 
codeblock in position $(x,y)$ and populates it with reconstructed 
wavelet coefficients.

\begin{comment}
[Include a figure here]
\end{comment}

\paragraph{Codeblock dimensions\newline}

The codeblock covers coefficients in the horizontal range $left$ to $right-1$ and in the vertical
range $bottom$ to $top-1$ where these values are defined by:
\begin{eqnarray*}
  left & = & (\width(band)*x)//\Codeblocks[level][horizontal] \\
  right & = & (\width(band)*(x+1))//\Codeblocks[level][horizontal] \\
  bottom & = & (\height(band)*y)//\Codeblocks[level][vertical] \\
  top & = & (\height(band)*(y+1))//\Codeblocks[level][vertical]
\end{eqnarray*}

\paragraph{Codeblock decode process\newline}

The codeblock decoding process is defined as:

\begin{pseudo}{codeblock}{band, parent,orient,quant\_index,y,x}
\bsIF{zero\_flag()}{\ref{zeroblockflag}}
  \bsFOR{v=bottom}{top-1}
    \bsFOR{h=left}{right-1}
      \bsCODE{band[v][h] = 0}
    \bsEND
  \bsEND
\bsELSE
  \bsCODE{quant\_idx += quant\_offset()}{\ref{blockquantidx}}
  \bsFOR{v=bottom}{top-1}
    \bsFOR{h=left}{right-1}
      \bsCODE{coeff\_decode(band,parent,orient,quant_idx,v,h)}{\ref{wltcoeff}}
    \bsEND
  \bsEND

\bsEND

\end{pseudo}

\paragraph{Zero block flag\newline}
\label{zeroblockflag}

We may set the number of codeblocks in the subband as

$num\_blocks = \Codeblocks[level][horizontal]*\Codeblocks[level][vertical]$

If $num\_blocks$ is 1 or $level=0$, then $zero\_flag()$ returns \false.

Otherwise, the flag is decoded from the stream: $read\_boola(\ZeroCodeblock)$
is returned.

\paragraph{Block quantiser offset\newline}
\label{blockquantidx}

If $\CodeblockMode=\SingleQuantiser$,  $quant\_offset()$ shall return 0.

If $\CodeblockMode=\MultipleQuantiser$ then the quantiser index offset
is decoded from the stream -- $read\_sinta(quant\_contexts())$ is returned, where
$quant\_contexts()$ returns the context set:

\begin{itemize}
\item{Follow= \{\QOffsetFollow\}}
\item{Data=\QOffsetInfo}
\item{Sign=\QOffsetSign}
\end{itemize}

\subsubsection{Intra DC band prediction}
\label{intradcprediction}

This section defines the operation of the $intra\_dc\_prediction(band)$ function
for reconstructing values within Intra picture DC bands using spatial prediction.
Intra DC values are derived by spatial prediction using the mean of the
three values to the left, top-left and above a coefficient (if available).

\begin{comment}
\begin{pseudo}{intra\_dc\_prediction}{band}
\bsFOR{v=0}{\height(band)-1}
  \bsFOR{h=0}{\width(band)-1}
    \bsCODE{prediction = 0 }
    \bsCODE{N = 0 }
    \bsIF{v>0}
      \bsCODE{prediction += band[v-1][h]}
      \bsCODE{N+=1}
      \bsIF{h>0}
        \bsCODE{prediction += band[v-1][h-1]+band[v][h-1]}
        \bsCODE{N+=1}
      \bsEND
    \bsELSE
      \bsIF{h>0}
        \bsCODE{prediction += band[0][h-1]}
        \bsCODE{N+=1}
      \bsEND
    \bsEND
    \bsCODE{prediction = prediction//N}
    \bsCODE{band[v][h] += prediction}
  \bsEND
\bsEND
\end{pseudo}
\end{comment}

\begin{pseudo}{intra\_dc\_prediction}{band}
\bsCODE{prediction = 0 }
\bsFOR{v=0}{subband\_height(level)-1}
  \bsFOR{h=0}{subband\_width(level)-1}
    \bsIF{h>0}
        \bsIF{v>0}
            \bsCODE{prediction=\mean(band[v][h-1],band[v-1][h-1],band[v-1][h])}
        \bsELSE
            \bsCODE{prediction=band[0][h-1]}
        \bsEND
    \bsELSE
        \bsIF{v>0}
            \bsCODE{prediction=band[v-1][0]}
        \bsELSE
            \bsCODE{prediction = 0}
        \bsEND
    \bsEND
    \bsCODE{band[v][h] += prediction}
  \bsEND
\bsEND
\end{pseudo}
\subsection{Subband coefficient decoding process}

\label{wltcoeff}

This section describes the operation of the 
$coeff\_decode(band,parent,orient,quant\_idx,v,h)$ process
for decoding a coefficient in position $(h,v)$ in the subband $band$ with
parent band $parent$ and orientation $orient$.

Decoding a coefficient makes use of arithmetic decoding, inverse quantisation
and, in the case of DC (level 0) bands of Intra frames, neighbourhood prediction.
The decoding process periodically refreshes the contexts, by
halving the context counts every time a count is reached.

Arithmetic coding uses a highly compact set of contexts, 
with magnitudes contextualised on whether parent values
and neighbouring values are zero or non-zero.

\subsubsection{Overall coefficient decoding process}

Two different processes are used for decoding coefficients, depending
upon whether spatial prediction is required or not.

\begin{pseudo}{coeff\_decode}{band,parent,orient,quant\_index,v,h}
    \bsCODE{parent = parent\_val(parent, v, h)}{\ref{parentval}}
    \bsCODE{nhood = zero\_nhood(band,v,h)}{\ref{zeronhood}}
    \bsCODE{sign\_pred = sign\_predict(band,orient,v,h)}{\ref{signpredict}}
    \bsCODE{context\_set = select\_coeff\_ctxs(parent, nhood, sign\_pred)}{\ref{selectcoeffcontext}}
    \bsCODE{quant\_coeff = read\_sina( context\_set )}{}
    \bsCODE{band[v][h] = inverse\_quant( quant\_coeff, quant\_index )}{\ref{invquant}}
    \bsCODE{update\_count()}{\ref{updatecounts}}
\end{pseudo}


\subsubsection{Parent values}
\label{parentval}
The function $parent\_val(v, h)$ returns the parent value of a coefficient in a subband,
which is the co-located coefficient in the parent subband. If there is a parent band 
($parent!=\emptyset$), then

$parent[v//2][h//2]$

is returned. If there is no parent ($parent=\emptyset$), then 0 is returned.

\subsubsection{Zero neighbourhood}
\label{zeronhood}

The $zero\_nhood()$ function returns a boolean indicating whether neighbouring
values are all zero.

\begin{pseudo}{zero\_nhood}{band,v,h}
\bsIF{v>0}
  \bsIF{band[v-1][h]!=0}
    \bsRET{\false}
  \bsEND
  \bsIF{h>0}
    \bsIF{band[v-1][h-1])!=0 \text{ or } band[v][h-1]!=0}
      \bsRET{\false}
    \bsEND
  \bsEND
\bsELSE
  \bsIF{h>0}
    \bsIF{ band[v][h-1] !=0}
      \bsRET{\false}
    \bsEND
  \bsEND
\bsEND
\bsRET{\true}
\end{pseudo}

\subsubsection{Sign prediction}
\label{signpredict}

The $sign\_predict()$ function returns a prediction for the sign of the 
current pixel. Correlation within subbands depends upon orientation,
and so this is taken into account in forming the prediction.

For vertically-oriented (HL) bands, the predictor is the sign of the
coefficient above the current coefficient; for horizontally-oriented (LH)
bands, the predictor is the sign of the coefficient to the left. 

The predictions are not used for differential encoding of the sign, but for
conditioning of the sign contexts only.

\begin{pseudo}{sign\_predict}{band,orient,v,h}
\bsIF{orient==HL}
  \bsIF{v==0}
    \bsRET{0}
  \bsELSE
    \bsRET{sign(band[v-1][h])}
  \bsEND
\bsELSEIF{orient==LH}
  \bsIF{h==0}
    \bsRET{0}
  \bsELSE
    \bsRET{sign(band[v][h-1])}
  \bsEND
\bsELSE
  \bsRET{0}
\bsEND{}
\end{pseudo}

\subsubsection{Coefficient context selection}
\label{selectcoeffcontext}

This section defines the $select\_context(zero\_nhood, parent, sign)$
function, which chooses a context index set for decoding a coefficient value.

Twelve possible coefficient index sets are defined, and are returned as specified 
in Table \ref{contexttable}. Note that follow contexts are an array indexed from $0$
as per Section \ref{arithreadint}.

Note that parent values affect the context of all follow bits, and that neighbour
values only affect the context of the first follow bit. A common data context is used
for all coefficients.

%% Table of context sets for signed coefficient extraction %%
\begin{table}[h!]
\begin{tabular}{|c|c|c||l|l|}
\hline
 $parent$ & $zero\_nhood$ & $sign$ & \multicolumn{2}{c|}{\bf{Context set}} \\

% Zero parent, zero neighbour, zero sign prediction
\hline
0 & \true & 0 &  Follow & [\ZPZNFollowOne,\ZPFollowTwo,\ZPFollowThree,
                            \ZPFollowFour,\ZPFollowFive,\ZPFollowSixPlus] \\ \cline{4-5}
  &   &   &  Data & \CoeffData \\ \cline{4-5}
  &   &   &  Sign & \SignZero \\

% Zero parent, zero neighbour, -ve sign prediction
\hline
0 & \true & $<0$ &  Follow & [\ZPZNFollowOne,\ZPFollowTwo,\ZPFollowThree,
                               \ZPFollowFour,\ZPFollowFive,\ZPFollowSixPlus] \\ \cline{4-5}
  &   &    &  Data & \CoeffData \\ \cline{4-5}
  &   &    &  Sign & \SignNeg \\

% Zero parent, zero neighbour, +ve sign  prediction
\hline
0 & \true & $>0$ &  Follow & [\ZPZNFollowOne,\ZPFollowTwo,\ZPFollowThree,
                               \ZPFollowFour,\ZPFollowFive,\ZPFollowSixPlus] \\ \cline{4-5}
  &   &    &  Data & \CoeffData \\ \cline{4-5}
  &   &    &  Sign & \SignPos \\

% Zero parent, non-zero neighbour, zero sign prediction
\hline
0 & \false & 0 &  Follow & [\ZPNNFollowOne,\ZPFollowTwo,\ZPFollowThree,
                             \ZPFollowFour,\ZPFollowFive,\ZPFollowSixPlus] \\ \cline{4-5}
  &   &   &  Data & \CoeffData \\ \cline{4-5}
  &   &   &  Sign & \SignZero \\

% Zero parent, non-zero neighbour, -ve sign prediction
\hline
0 & \false & $<0$ &  Follow & [\ZPNNFollowOne,\ZPFollowTwo,\ZPFollowThree,
                                \ZPFollowFour,\ZPFollowFive,\ZPFollowSixPlus] \\ \cline{4-5}
  &        &    &  Data & \CoeffData \\ \cline{4-5}
  &        &    &  Sign & \SignNeg \\

% Zero parent, non-zero neighbour, +ve sign prediction
\hline
0 & \false & $>0$ &  Follow & [\ZPNNFollowOne,\ZPFollowTwo,\ZPFollowThree,
                                \ZPFollowFour,\ZPFollowFive,\ZPFollowSixPlus] \\ \cline{4-5}
  &        &      &  Data & \CoeffData \\ \cline{4-5}
  &        &      &  Sign & \SignPos \\

% Non-zero parent, zero neighbour, zero sign prediction
\hline
$\neq 0$ &  \true & 0 & Follow & [\NPZNFollowOne,\NPFollowTwo,\NPFollowThree,
                                    \NPFollowFour,\NPFollowFive,\NPFollowSixPlus] \\ \cline{4-5}
& &      &  Data & \CoeffData \\ \cline{4-5}
& &      &  Sign & \SignZero \\

% Non-zero parent, zero neighbour, -ve sign prediction
\hline
$\neq 0$ & \true & $<0$ &  Follow & [\NPZNFollowOne,\NPFollowTwo,\NPFollowThree,
                                      \NPFollowFour,\NPFollowFive,\NPFollowSixPlus] \\ \cline{4-5}
& &      &  Data & \CoeffData \\ \cline{4-5}
& &      &  Sign & \SignNeg \\

% Non-zero parent, zero neighbour, +ve sign prediction
\hline
$\neq 0$ & \true & $>0$ &  Follow & [\NPZNFollowOne,\NPFollowTwo,\NPFollowThree,
                                      \NPFollowFour,\NPFollowFive,\NPFollowSixPlus] \\ \cline{4-5}
& &      &  Data & \CoeffData \\ \cline{4-5}
& &      &  Sign & \SignPos \\

% Non-zero parent, non-zero neighbour, zero sign prediction
\hline
$\neq 0$ & \false & 0 &  Follow & [\NPNNFollowOne,\NPFollowTwo,\NPFollowThree,
                                    \NPFollowFour,\NPFollowFive,\NPFollowSixPlus] \\ \cline{4-5}
& &      &  Data & \CoeffData \\ \cline{4-5}
& &      &  Sign & \SignZero \\

% Zero parent, non-zero neighbour, -ve sign prediction
\hline
$\neq 0$ & \false & $<0$ &  Follow & [\NPNNFollowOne,\NPFollowTwo,\NPFollowThree,
                                       \NPFollowFour,\NPFollowFive,\NPFollowSixPlus] \\ \cline{4-5}
& &      &  Data & \CoeffData \\ \cline{4-5}
& &      &  Sign & \SignNeg \\

% Zero parent, non-zero neighbour, +ve sign prediction
\hline
$\neq 0$ & \false  & $>0$ &  Follow & [\NPNNFollowOne,\NPFollowTwo,\NPFollowThree,
                                        \NPFollowFour,\NPFollowFive,\NPFollowSixPlus] \\ \cline{4-5}
& &      &  Data & \CoeffData \\ \cline{4-5}
& &      &  Sign & \SignPos \\
\hline

\end{tabular}
\caption{Subband coefficient context sets}\label{contexttable}
\end{table}

\subsubsection{Inverse quantisation}
\label{invquant}

The inverse\_quant() function is defined by:

\begin{pseudo}{inverse\_quant}{quantised\_coeff, quant\_index}
\bsCODE{magnitude = |{quantised\_coeff}|}
\bsCODE{magnitude *= quant\_factor(quant\_index)}{\ref{quantfacs}}
\bsIF{magnitude!=0}
  \bsCODE{magnitude += quant\_offset(quant\_index)}{\ref{quantfacs}}
  \bsCODE{magnitude += 2}
  \bsCODE{magnitude = magnitude//4}
\bsEND
\bsRET{sign( quantised\_coeff )*magnitude}
\end{pseudo}

\begin{informative}
The pseudocode description separates inverse quantisation from decoding. However, 
since dead-zone quantisation is used, the $inverse\_quant()$ function must compute
the magnitude. Hence it is more efficient to first decode the coefficient magnitude,
then inverse quantise, and then decode the coefficient sign. 

Note that an unbiased division by 4 is applied: the additional 2 may be absorbed into
the quantisation offset, but is separated in this specification.
\end{informative}

\subsubsection{Quantisation factors and offsets}
\label{quantfacs}

This section specifies the operation of the $quant\_factor()$ and 
$quant\_offset()$ functions for performing inversion quantisation.

Quantisation factors represent an approximation of quarter-bit values 
with two bits of accuracy (i.e. $round(2^{\frac{index}{4}+2})$:

\begin{pseudo}{quant\_factor}{index}
\bsCODE{base = 2^{index//4}}
\bsIF{ (x\%4)==0 }
  \bsRET{4*base}
\bsELSEIF{ (x\%4)==1 }
  \bsRET{78892*base+8292)//16585}
\bsELSEIF{ (x\%4)==2 }
  \bsRET{228486*base+20195)//40391}
\bsELSEIF{ (x\%4)==3 }
  \bsRET{440253*base+32722)//65444}
\bsEND
\end{pseudo}

For intra pictures, offsets are approximately $1/2$ of the 
quantisation factors, and for inter pictures they are $3/8$ - these
mark the reconstruction point within the quantisation interval:

\begin{pseudo}{quant\_offset}{index}
\bsIF{index==0}
  \bsCODE{offset = 1}
\bsELSE
  \bsIF{ is\_intra() }
    \bsIF{index==1}
      \bsCODE{offset = 2}
    \bsELSE
      \bsCODE{ offset=(quant\_factor(index)+1)//2 }
    \bsEND
  \bsELSE
    \bsCODE{ offset=(quant\_factor(index)*3+4)//8 }
  \bsEND
\bsEND
\bsRET{offset}
\end{pseudo}

\begin{informative}
The quantisation offsets have been selected so as to make inverse quantisation
and quantisation by the same quantisation factor transparent. This requires that
\[3\leq quant\_offset+2<quant\_factor\]

-- hence the special conditions for quantisation index 0 and 1.
\end{informative}

\subsubsection{Updating counts and resetting contexts}
\label{updatecounts}

The $update\_count()$ function updates a periodic count of subband 
coefficients and rescales arithmetic decoding contexts if \CoefficientReset has been reached.

\begin{pseudo}{update\_count}{}
\bsCODE{\CoefficientCount += 1}
\bsIF{\CoefficientCount == \CoefficientReset}
  \bsCODE{\CoefficientCount = 0}
  \bsFOR{i=0}{len(\AContexts)-1}
    \bsCODE{rescale\_context(i)}{\ref{arithcontexts}}
  \bsEND
\bsEND
\end{pseudo}
