\begin{comment}
%from 0.9.1

This function is fundamental to all the arithmetic decoding functions.
It consists of three stages:

1.  Determine the binary output value VALUE as per Section 

2.  Modify the decoder state as per Section 

3.  Update the context statistics by invoking update( CONTEXT , VALUE )
as per Section 
\end{comment}


%src: tim-0.9.1.48

reads a single arithmetic coded bit from the bytestream and returns a
Boolean value.

Read Binary Arithmetic Coded Bit
\begin{verbatim}
read_ba(context):
    while (((high&0x8000)==0x0) and ((low&0x8000)==0x0)):
        shift_bit_in()
    while ( ((high&0x4000)==0x0) and
           ((low&0x4000)==0x4000) ):
        code ^= 0x4000
        high ^= 0x4000
        low ^= 0x4000
        shift_bit_in()
    weight = context[0] + context[1]
    scaler = (0x10000+weight//2)//weight   #lookup table

    probability0 = context[0]*scaler
    count = code-low+1
    range = high-low+1
    range_x_prob = (range * probability0)>>16
    if ( count > range_x_prob ):
        value = True
        low = low + range_x_prob
        context[1] += 1
    else
        value = False
        high = low + range_x_prob - 1
        context[0] += 1
    if ( (context[0] + context[1]) > 255 ):
        #Halve counts in the context
        context[0] >> 1
        context[0] += 1
        context[1] >> 1
        context[1] += 1
    return value
\end{verbatim}

Shift Bit In
\begin{verbatim}
shift_bit_in():

    high << 1
    high &= 0xFFFF
    high += 1
    low << 1
    low &= 0xFFFF
    code << 1
    code &= 0xFFFF
    code += read_bita()
\end{verbatim}
