\label{bitpacking}

This section defines the operation of the $read\_bit()$, $read\_byte()$ 
and $byte\_align()$ functions used for direct access to the Dirac stream.

Access to the Dirac stream is bytewise, and a decoder is deemed to maintain
a copy of the current byte, \CurrentByte, and an index to the next bit
to be read, \NextBit. \NextBit is an integer from 0 (least-significant bit) to 7 
(most-significant bit). Bits within bytes are accessed from the msb first to the
lsb.

Each access unit and individual frame is a whole number of bytes. Decoding from the
start of an access unit, \NextBit is set to 7.

The $read\_byte()$ function returns the next byte in the Dirac stream and sets
\NextBit to 7.

The $read\_bit()$ function is defined by

\begin{pseudo}{read\_bit}{}
\bsIF{\NextBit < 0}
\bsCODE{\CurrentByte = read\_byte()}
\bsEND
\bsCODE{bit = ( \CurrentByte \gg \NextBit ) \& 1 }
\bsCODE{\NextBit -= 1}
\bsRET{bit}
\end{pseudo}

The $byte\_align()$ function discards data in the current byte and begins data access
at the next byte, unless input is already at the beginning of a byte. This is used to 
ensure that a whole number of bytes are read before
beginning reading a new stream element.

\begin{pseudo}{byte\_align()}{}
\bsIF{\NextBit != 7}
\bsCODE{\CurrentByte = read\_byte()}
\bsEND
\end{pseudo}
