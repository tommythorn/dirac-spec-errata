\label{arithcontexts}

[Contexts will exist in a state array. Will be accessed via integer indices]


An individual context is reset by halving the counts of $0$ and $1$ and ensuring that
these counts do not reach zero:

\begin{pseudo}{reset\_context}{index}
\bsCODE{\Contexts[0] = \Contexts[0]//2}
\bsCODE{\Contexts[0] += 1}
\bsCODE{\Contexts[1] = \Contexts[1]//2}
\bsCODE{\Contexts[1] += 1}
\end{pseudo}

\begin{comment}
The decoder shall maintain the statistics relating to the set of
contexts that may be used in decoding a data element. How these shall be
initialised is set out in Sections  and . A context is modelled as
consisting of the following unsigned integer values, which constitute
its state:

COUNT0

COUNT1

COUNT0 represents a count of the number of zeroes that have occurred in
the context, COUNT1 a count of the number of ones that have occurred in
the context. 

From these values WEIGHT, SCALED\_COUNT0 and SCALED\_COUNT1 are derived.
WEIGHT shall always be equal to COUNT0+COUNT1. WEIGHT shall always be
less than 1024

SCALED\_COUNT0 is a count of zeroes scaled to a total count of 1024.
SCALED\_COUNT1 is a count of ones, likewise scaled to a total count of
1024. The derivation of the scaled counts is specified in Section .
\end{comment}
