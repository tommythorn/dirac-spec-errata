Each frame is given a unique number, consecutively increasing for each
frame in the original source material; this ordering is known as display
order, the order in which frames must be displayed.  In order to reduce
the codec's complexity, each frame the decoder receives must be received
after any reference frames it uses; this is the coded order, the order
in which coded frames are transmitted. See
figure~\ref{fig:frame-ordering}. Note that this means that the
transmitted order of the frames is not the display order. There will
therefore have to be a buffer of sufficient size to accommodate the
reordering.

\begin{figure}
    \centering
    \includegraphics[width=0.9\textwidth]{figs/frame-ordering}
    \caption{Frame reordering for transmission}
    \label{fig:frame-ordering}
\end{figure}

References may not persist across Video Sequence boundaries.

\paragraph{Frame ordering across Access Unit boundaries}
Consider figure x, with two Access Units $A$ and $B$.  While frame
x occurs in display order before the start of access unit $B$, it
references frames x1 (the first frame of $B$) and therefore must be
coded and transmitted after x1.  If decoding were to commence at the
start of $B$, frame x would not be decoded.


