%%%%%%%%%%%%%%%%%%%%%%%%%%%%%%%%%%%%%%%%%%%%%%%%%%%%%
% - This chapter defines how the inverse discrete - %
% - wavelet transform is done                     - %
%%%%%%%%%%%%%%%%%%%%%%%%%%%%%%%%%%%%%%%%%%%%%%%%%%%%%

\label{idwt}

This section defines the process $idwt(data)$ for reconstructing picture component data
from decoded subband data $data$ using the Inverse Discrete Wavelet Transform (IDWT). $idwt()$
can be invoked in the picture decoding process only after successful parsing of the
subband coefficient data (Section \ref{transformdec}).

The $idwt()$ process consists of two sub-processes:

\begin{enumerate}
\item Synthesis, which returns a pixel array from the subband wavelet
coefficients: 

$pic=idwt\_synthesis(data)$ (Section \ref{idwtsynthesis})
\item Pad-removal, which removes padding values from the synthesised pixel array $pic$ (Section \ref{paddingremoval})
\end{enumerate}

The output returned by this process is a two-dimensional array $pic$ of pixel data representing
Y, U or V component data.

\begin{informative}
The IDWT can operate with a number of different wavelet filters, whose identity has
been signalling in the transform data header. These allow trade-offs to be made
between compression efficiency, implementation complexity and perceptual quality.

Different filters are applicable in different scenarios. Shorter filters are generally more
suitable for motion-compensated residual data, and longer filters for intra picture data.
The default filter set includes a ``Fidelity" filter optimised for downconversion, so that a lower-resolution,
but high-quality, proxy layer may be decoded for viewing.
\end{informative}

Since wavelet filtering operates on both rows and columns of two-dimensional arrays independently
it is useful to define operators $\row(a,k)$ and $\column(a,k)$ for extracting rows and columns
with index $k$ from a 2-dimensional array $a$:

If $b=\row(a,k)$ then $b[r]$ is a $reference to$ the value of $a[k][r]$. This means that modifying the
value of $b[r]$ modifies the value of $a[k][r]$.

If $b=\column(a,k)$ then $b[r]$ is a $reference to$ the value of $a[r][k]$. This means that modifying the
value of $b[r]$ modifies the value of $a[r][k]$.

\begin{informative}
These definitions allow us to express the important feature that all filtering operations are specified
as {\em in place} calculations, not involving data being copied at any stage.
\end{informative}

\subsection{IDWT synthesis operation}
\label{idwtsynthesis}

This section defines the process $idwt\_synthesis(pic, data)$ invoked by $idwt()$.

This is an iterative procedure operating on four subbands at each
iteration stage to produce a new subband. The procedure
is:

\begin{pseudo}{idwt\_synthesis}{data}
\bsCODE{LL\_band = data[0][\LL]}
\bsFOR{n=0}{\WaveletDepth-1}
   \bsCODE{ LL\_band = vh\_synthesis(LL\_band, data[n][\HL], data[n][\LH], data[n][\HH])}{\ref{vhsynth}}
\bsEND
\bsRET{LL\_band}
\end{pseudo}

Note that at each stage, the input dimensions of $LL\_band$ will be the same as those of the
other input bands, whereas the output dimensions are double those of the input bands.

\subsection{Vertical and horizontal synthesis}
\label{vhsynth}

This section specifies the operation of the vertical and horizontal
synthesis process $vh\_synthesis(LL\_data, HL\_data, LH\_data, HH\_data)$.

$vh\_synthesis$ is repeatedly invoked by $idwt\_synthesis()$. It operates on four subband
data arrays of identical dimensions to produce a new array $synth$, which is returned as the result of
the process.

{\bf Step 1.} $synth$ is initialised so that:
\begin{eqnarray*}
\width(synth) & = & 2*\width(LL\_data) \\
\height(synth) & = & 2*\height(LL\_data)
\end{eqnarray*}

{\bf Step 2.} The data from the four arrays is interleaved as follows:

\begin{pseudo*}
\bsFOR{y=0}{(\height(synth)//2)-1}
    \bsFOR{x=0}{(\width(synth)//2)-1}
        \bsCODE{synth[2*y][2*x] = LL\_data[y][x]}
        \bsCODE{synth[2*y][2*x+1] = HL\_data[y][x]}
        \bsCODE{synth[2*y+1][2*x] = LH\_data[y][x]}
        \bsCODE{synth[2*y+1][2*x+1] = HH\_data[y][x]}
    \bsEND
\bsEND
\bsCODE{\hdots}
\end{pseudo*}

This enables in-place calculation during the inverse filter process.

{\bf Step 3.} Data is next synthesised vertically by operating on each column
of data using a one-dimensional filter, and then horizontally by operating
on each row.

\begin{pseudo*}
\bsFOR{x=0}{\width(synth)-1}
    \bsCODE{1d\_synthesis(\column(synth, x) )}{\ref{onedsynth}}
\bsEND
\bsFOR{y=0}{\height(synth)-1}
    \bsCODE{1d\_synthesis(\row(synth, y) )}{\ref{onedsynth}}
\bsEND
\bsCODE{\hdots}
\end{pseudo*}

{\bf Step 4.} Finally, the synthesised subband data receives a bitshift to
remove any accuracy bits. The shift value $filtershift()$ used is as determined in Section \ref{wltfilters}
from the wavelet index $\WaveletIndex$.

\begin{pseudo*}
\bsCODE{shift = filtershift()}
\bsFOR{y=0}{\height(synth)-1}
    \bsFOR{x=0}{\width(synth)-1}
        \bsCODE{synth[y][x] = synth[y][x]\gg shift}
    \bsEND
\bsEND
\end{pseudo*}

\begin{informative}
Accuracy bits are added in the encoder by shifting up all coefficients in the 
$\LL$ band prior to applying any filtering (this includes an initial shift of all 
values in the component data). Adding a small shift before each decomposition
is the most efficient way of providing additional resolution where it is needed:
the result is that the shift varies with the level to which a subband belongs.

Accuracy bits are required because the rounding stages in integer lifting
introduce non-linearities that can aliase data between subbands. Quantising
data in non-DC bands can then (for example) produce artefacts at DC 
band frequencies. This increases bit rate and the perceptual impact of quantisation, 
especially for 8-bit video. The accuracy bits have been computed so as to virtually
eliminate these effects (with the exception of Haar0 which is included as it
is suitable for low-delay low-complexity implementations, especially lossless coding).
For example, Fidelity does not require a shift value since its filter gain is such
as to add a bit of resolution with each level of decomposition.
\end{informative}

\subsection{One-dimensional synthesis}
\label{onedsynth}

This section specifies the one-dimensional synthesis process
$1d\_synthesis()$ applied to a 1-dimensional array of coefficients of
even length, consisting
of either a row or a column of a 2-dimensional integral data array.

The one-dimensional synthesis process comprises the application of a
number of reversible integer lifting filter operations. An integral lifting filter
 is characterised by four elements:
\begin{itemize}
\item a set of taps $t_{-N}, \hdots,t_M$
\item a scale factor $s$
\item a parity (odd or even)
\item whether it is an ``update`` or ``predict`` filter
\end{itemize}

An even lifting filtering operation modifies the even coefficients
by the odd coefficients:

\begin{eqnarray*}
  A[2*n]& +=& \big( \sum^M_{i=-N} t_i *A[2*(n+i) + 1] +(1\ll (s-1))\big) \gg s \mbox{ (Update)} \\
  A[2*n]& -=& \big( \sum^M_{i=-N} t_i *A[2*(n+i) + 1] +(1\ll (s-1))\big) \gg s \mbox{ (Predict)}
\end{eqnarray*}

An odd lifting filtering operation modifies the odd coefficients
 by the even coefficients:

\begin{eqnarray*}
  A[2*n+1]& +=&  \big( \sum^M_{i=-N} t_i A[2*(n+i)]+(1\ll (s-1)) \big) \gg s \mbox{ (Update)} \\
  A[2*n+1]& -=&  \big( \sum^M_{i=-N} t_i A[2*(n+i)] +(1\ll (s-1))\big) \gg s \mbox{ (Predict)} \\
\end{eqnarray*}

\begin{informative}
Note that the distinction between update and predict filters is necessary
because integer rounding is being used, and the filters are non-linear.
A predict filter with taps $t_i$ is not equivalent to an update filter with taps $-t_i$.
\end{informative}

These formulae are deemed to be applied for all applicable $n$
before the next filtering stage is applied and to use edge extension (of even and
odd parts separately) at the array
edges where the filter would otherwise overlap. In pseudo-code, two
functions $lifteven(A, t_i, s)$ and $liftodd(A, t_i, s)$ are specified by:

\begin{pseudo}{lifteven}{A,t_i, s,fsort}
\bsFOR{n=0}{(\length(A)//2)-1}
    \bsCODE{sum=0}
    \bsFOR{i=-N}{M}
        \bsCODE{pos=2*(n+i)-1}
        \bsCODE{pos=\min(pos, \length(A)-1)}        
        \bsCODE{pos=\max(pos, 1)}
        \bsCODE{sum+=t_i*A[pos]}
    \bsEND
    \bsCODE{sum+=(1\ll (s-1))}
    \bsIF{fsort==\textit{PREDICT}}
        \bsCODE{A[2*n]-=(sum\gg s)}
    \bsELSE
        \bsCODE{A[2*n]+=(sum\gg s)}
\bsEND
\end{pseudo}

and

\begin{pseudo}{liftodd}{A,t_i, s,fsort}
\bsFOR{n=0}{(\length(A)//2)-1}
    \bsCODE{sum=0}
    \bsFOR{i=-N}{M}
        \bsCODE{pos=2*(n+i)}
        \bsCODE{pos=\min(pos, \length(A)-2)}        
        \bsCODE{pos=\max(pos, 0)}
        \bsCODE{sum+=t_i*A[pos]}
    \bsEND
    \bsCODE{sum+=(1\ll (s-1))}
    \bsIF{fsort==\textit{PREDICT}}
        \bsCODE{A[2*n+1]-=(sum\gg s)}
    \bsELSE
        \bsCODE{A[2*n+1]+=(sum\gg s)}
\bsEND\end{pseudo}

$1d\_synthesis$ applies the sequence of lifting filters specified in Section \ref{wltfilters},
invoking $lifteven$ for even parity filters and $liftodd$ for odd parity filters.

\begin{informative}
This specification defines the lifting process on the basis of lifting
procedures applied to an entire row or column consecutively. It is
possible to implement lifting filtering operations so that a filtering
operation associated with one lifting filter is followed by a filtering
operation associated with another lifting filter. I.e. the order of
iteration is changed. In this case, the order in which filtering is
applied to coefficients does affect the outcome of the process as even
lifting operations may be followed by odd ones, and care must be taken
that values are not modified in the wrong order. Nevertheless such an
implementation may be more efficient, and complies with this
specification if it produces identical results.
\end{informative}

\subsection{Filters and shift values}
\label{wltfilters}

The following the lifting filters and bitshift operations that
apply for each value of $\WaveletIndex$ are specified in Tables \ref{filtertable0} to
\ref{filtertable7}.

\begin{informative}
This specification contains an implementation of the Daubechies (9,7) filter 
($\WaveletIndex==7$). Daubechies (9,7), like any other FIR biorthogonal wavelet, possesses a lifting
implementation. However, to produce a perfect reconstruction filter, the lifting stages 
require real-valued filter taps, and a final coefficient scaling stage. Any realisable implementation
is therefore an approximation. The implementation specified here is fully integral, yet (other than omitting
the final scaling stages), it is a very close approximation. These filters can therefore
(unlike the JPEG2000 implementation) be used for lossless as well as lossy compression, whilst lossy
compression performance is near-indentical to the real-valued filter. The integer lifting implementaiton
also allows for much more efficient implementation.
\end{informative}


\begin{table}[h!]
\begin{tabular}{|l|l|}

\hline
{\bf Lifting steps} & $filtershift()$ \\
\hline
\begin{tabular}{l}
1. Even, Predict, $s=2, t_0=1, t_1=1$ i.e. \\
\quad $ A[2*n]     -=  (A[2*n-1]+A[2*n+1]+2)\gg 2$ \\
2. Odd, Update, $s=4$, $t_{-1}=-1, t_0=9, t_1=9, t_2=-1$ i.e. \\
\quad $ A[2*n+1]  +=  (-A[2*n-2]+9*A[2*n]+9*A[2*n+2]-A[2*n+4]+8)\gg 4$
\end{tabular}
& 1 \\
\hline

\end{tabular}
\caption{$\WaveletIndex==0$: Deslauriers-Debuc (9,3) lifting stages and shift values}\label{filtertable0}
\end{table}

\begin{table}[h!]

\begin{tabular}{|l|l|}
\hline
{\bf Lifting steps} & $filtershift()$ \\
\hline

\begin{tabular}{l}
1. Even, Predict, $s=2, t_0=1, t_1=1$ i.e. \\
\quad $ A[2*n]  -= (A[2*n-1]+A[2*n+1]+2)\gg 2$ \\
2. Odd, Update, $s=1$, $t_0=1, t_1=1$ i.e. \\
\quad $ A[2*n+1]  += (A[2*n]+A[2*n+2]+1)\gg 1$
\end{tabular}
& 1 \\ 
\hline

\end{tabular}
\caption{$\WaveletIndex==1$: LeGall (5,3) lifting stages and shift values}
\end{table}

\begin{table}[h!]
\begin{tabular}{|l|l|}

\hline
{\bf Lifting steps} & $filtershift()$ \\
\hline
\begin{tabular}{l}
1. Even, Predict, $s=5, t_{-1}=-1, t_0=9, t_1=9, t_2=-1$ i.e. \\
\quad $ A[2*n]  -= (-A[2*n-3]+9*A[2*n-1]+9*A[2*n+1]+A[2*n+3]+16)\gg 5$ \\
2. Odd, Update, $s=4, t_{-1}=-1, t_0=9, t_1=9, t_2=-1$ i.e. \\
\quad $ A[2*n+1]  += (-A[2*n-2]+9*A[2*n]+9*A[2*n+2]+A[2*n+4]+8)\gg 4$
\end{tabular}

& 1 \\
\hline

\end{tabular}
\caption{$\WaveletIndex==2$: Deslauriers-Debuc (13,5) lifting stages and shift values}
\end{table}

\begin{table}[h!]
\begin{tabular}{|l|l|}

\hline
{\bf Lifting steps} & $filtershift()$ \\
\hline
\begin{tabular}{l}
1. Even, Predict, $s=1, t_1=1$ i.e. \\
\quad $A[2*n] -=  (A[2*n+1]+1)\gg 1$ \\
2. Odd, Update, $s=0, t_0=1$ i.e. \\
\quad $ A[2*n+1] += A[2*n]$
\end{tabular}

& 0 \\
\hline

\end{tabular}
\caption{$\WaveletIndex==3$: Haar filter with no shift}
\end{table}

\begin{table}[h!]
\begin{tabular}{|l|l|}

\hline
{\bf Lifting steps} & $filtershift()$ \\
\hline
\begin{tabular}{l}
1. Even, Predict, $s=1, t_1=1$ i.e. \\
\quad $A[2*n] -=  (A[2*n+1]+1)\gg 1$ \\
2. Odd, Update, $s=0, t_0=1$ i.e. \\
\quad $ A[2*n+1] += A[2*n]$
\end{tabular}

& 1 \\
\hline

\end{tabular}
\caption{$\WaveletIndex==4$: Haar filter with single shift}
\end{table}

\begin{table}[h!]
\begin{tabular}{|l|l|}

\hline
{\bf Lifting steps} & $filtershift()$ \\
\hline
\begin{tabular}{l}
1. Even, Predict, $s=1, t_1=1$ i.e. \\
\quad $A[2*n] -=  (A[2*n+1]+1)\gg 1$ \\
2. Odd, Update, $s=0, t_0=1$ i.e. \\
\quad $ A[2*n+1] += A[2*n]$
\end{tabular}

& 2\\
\hline

\end{tabular}
\caption{$\WaveletIndex==5$: Haar filter with double shift}
\end{table}

\begin{table}[h!]
\begin{tabular}{|l|l|}
\hline
{\bf Lifting steps} & $filtershift()$ \\
\hline
\begin{tabular}{l}
1. Odd, Update, $s=8, t_{-3}=-2, t_{-2}=10, t_{-1}=-25, t_0=81, t_1=81, t_2=-25, t_3=10, t_4=-2$ i.e. \\
$ \begin{array}{rcl} A[2*n+1] & += & (-2*(A[2*n-6]+A[2*n+8])+10*(A[2*n-4]+A[2*n+6])\\
                                             & &       -25*(A[2*n-2]+A[2*n+4])+81*(A[2*n]+A[2*n+2])+128)\gg 8 \end{array}$ \\
1. Even, Predict,  $s=8, t_{-3}=-8, t_{-2}=21, t_{-1}=-46, t_0=161, t_1=161, t_2=-46, t_3=21, t_4=-8$ i.e. \\
 $ \begin{array}{rcl}A[2*n]  & += & (-8*(A[2*n-7]+A[2*n+7])+21*(A[2*n-5]+A[2*n+5]) \\
                                          & &      -46*(A[2*n-3]+A[2*n+3])+161*(A[2*n-1]+A[2*n+1])+128)\gg 8 \end{array}$ 
\end{tabular}
& 0 \\
\hline

\end{tabular}
\caption{$\WaveletIndex==6$: Fidelity filter for improved downconversion and anti-aliasing}
\end{table}

\begin{table}[h!]
\begin{tabular}{|l|l|}

\hline
{\bf Lifting steps} & $filtershift()$ \\
\hline
\begin{tabular}{l}
1. Even, Predict, $s=12, t_0=1817, t_1=1817$ i.e. \\
\quad $A[2*n] -=  (1817*A[2*n-1]+1817*A[2*n+1]+2048)\gg 12$ \\
2. Odd, Predict, $s=12, t_0=3616, t_1=3616$ i.e. \\
\quad $ A[2*n+1] -= (3616*A[2*n]+3616*A[2*n+2]+2048)\gg 12$ \\
3. Even, Update, $s=12, t_0=217, t_1=217$ i.e. \\
\quad $A[2*n] +=  (217*A[2*n-1]+217*A[2*n+1]+2048)\gg 12$ \\
4. Odd, Update, $s=12, t_0=6497, t_1=6497$ i.e. \\
\quad $ A[2*n+1] += (6497*A[2*n]+6497*A[2*n+2]+2048)\gg 12$
\end{tabular}
& 1 \\
\hline

\end{tabular}
\caption{$\WaveletIndex==7$: Integer lifting approximation to Daubechies (9,7)}\label{filtertable7}
\end{table}

\clearpage
\subsection{Removal of IDWT pad values}
\label{paddingremoval}

This section defines the decoding process $idwt\_pad\_removal(pic, c)$. This
is invoked subsequent to $idwt\_synthesis$.

Subband data elements have been padded to ensure that the reconstructed 
data array $pic$ has dimensions divisible by $2^\WaveletDepth$.

Values $width$ and $height$ are defined to be the appropriate dimensions
of the component data:

\begin{itemize}
\item If $c=Y$, then
\begin{eqnarray*}
width & =& \LumaWidth \\
height & =& \LumaHeight
\end{eqnarray*}
\item else if $c=U$ or $c=V$,
\begin{eqnarray*}
width & =& \ChromaWidth \\
height & =& \ChromaHeight
\end{eqnarray*}
\end{itemize}

All component data $pic[j][i]$ with

\begin{itemize}
\item $i\geq width$, or
\item $j\geq height$
\end{itemize}

are discarded and $pic$ is resized to have width $width$ and height $height$.

