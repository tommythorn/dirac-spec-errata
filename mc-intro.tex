
This section defines the operation of the motion compensation function
$\mcfunc$.

The arguments to this function are:

\begin{itemize}
    \item The picture being motion-compensated $\pic$
    \item Reference pictures (one or two) $\refs$
    \item Motion compensation parameters $\mcparams$, as produced by the motion
    vector decoding process defined in Section ??.
[Consist of: a set of prediction modes, motion vectors, picture weights, block dimensions etc]
\end{itemize}

The current picture is motion compensated by compensating each component in turn:

\begin{verbatim}
motion_compensate(refs, mc_params):
    for c=0 to 2:
        component_compensate(pic[c], refs[c], mc_params )
\end{verbatim}

The remainder of this section specifies the operation of $\compmc$ for luma and
chroma component data.

The $\compmc$ function proceeds by first initialising an array of blocks,
a block being a set. 

\subsubsection{Blocks}

For convenience we will define a $\block$ data structure, which is a set
of pixel coordinates $(x,y)$ covering a block in the picture component,
together with its associated motion vector data:

\begin{itemize}
    \item A position of the blocks' top left pixel in picture
    co-ordinates, $(\Bx, \By)$.

    \item A prediction mode, $\Bpredmode$ that gives the prediction mode
    in use for the block.  Valid values are: $\predIntra$, $\predInter$,
    $\predGlobal$.

    \item A set of references that the block uses, $\Brefsinuse$.
    If $\Bpredmode = \predIntra$, then $\Brefsinuse$ is empty, otherwise
    it is a valid combination of references in use: $\lbrace0\rbrace$,
    $\lbrace1\rbrace$, $\lbrace0,1\rbrace$.

    \item If $\Bpredmode = \predIntra$ then the block contains three dc
    components (one for each component of the predicted picture),
    refered to as $\Bdc$.

    \item If $\Bpredmode = \predInter$ then the block contains a motion
    vector for each reference the block uses ($\Brefsinuse$), refered to
    as $\Bv_r, \forall r \in \Brefsinuse$.

    \item If $\Bpredmode = \predGlobal$ then the block contains no
    additional data, and uses the global motion parameters for each
    reference the block uses ($\Brefsinuse$).
  \end{itemize}
\end{itemize}


 Weighting values for each reference picture, $\W_0$ and $\W_1$,
and a weighting precision, $\tau$ (in bits).

\item General parameters consisting of block dimensions $(\xblen,
\yblen)$, block seperations $(\xbsep, \ybsep)$ and motion vector
precision, $\lambda$ (in bits).

\item Optional global motion parameters for each reference consisting of
a $2\times2$ zoom-rotation-shear matrix $\gmA_0$, $\gmA_1$; a
$2\times1$ translation vector $\gmB_0$, $\gmB_1$; a $2\times1$
perspective vector $\gmC_0$, $\gmC_1$; precision values for the $\gmA$
matricies $\mu_0$, $\mu_1$ (in bits); and precision values for the
$\gmC$ matricies $\psi_0$, $\psi_1$ (in bits).

\item A set of blocks, each block $\B$ consisting of:
  \begin{itemize}
    \item A position of the blocks' top left pixel in picture
    co-ordinates, $(\Bx, \By)$.

    \item A prediction mode, $\Bpredmode$ that gives the prediction mode
    in use for the block.  Valid values are: $\predIntra$, $\predInter$,
    $\predGlobal$.

    \item A set of references that the block uses, $\Brefsinuse$.
    If $\Bpredmode = \predIntra$, then $\Brefsinuse$ is empty, otherwise
    it is a valid combination of references in use: $\lbrace0\rbrace$,
    $\lbrace1\rbrace$, $\lbrace0,1\rbrace$.

    \item If $\Bpredmode = \predIntra$ then the block contains three dc
    components (one for each component of the predicted picture),
    refered to as $\Bdc$.

    \item If $\Bpredmode = \predInter$ then the block contains a motion
    vector for each reference the block uses ($\Brefsinuse$), refered to
    as $\Bv_r, \forall r \in \Brefsinuse$.

    \item If $\Bpredmode = \predGlobal$ then the block contains no
    additional data, and uses the global motion parameters for each
    reference the block uses ($\Brefsinuse$).
  \end{itemize}
\end{itemize}

Motion compensation is performed independently on each video
component\footnote{With scaling of parameters in the cases of
downsampled chroma components as per section~\ref{mc:scale}.}, is
defined pixel-wise; Section~\ref{mc:impl} provides some notes for
performing the process in a block-wise manner.

Throughout this section, the following conventions are used.

\begin{itemize}
\item $(x,y)$ are co-ordinates in the predicted picture.
\item $(u,v)$ are co-ordinates in an upconverted reference picture.
\end{itemize}
\annotate{df}{coordinate convention is shown in
fig~\ref{fig:blockcoverage}, but is defined earlier in the spec}

\begin{comment}
For each component, motion compensation is effected by forming a
prediction for each pixel within the component and adding it to the
pixel value. Specifically, for luma components, set

The use of OBMC implies that each pixel may fall into between 1 and 4
Prediction Units (), and the prediction value is a weighted combination
of predictions from each prediction unit of which the pixel is a member.
\end{comment}
