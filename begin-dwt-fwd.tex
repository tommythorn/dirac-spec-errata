Put simply, wavelet analysis splits a signal into a low and a high
frequency component, and then subsamples the two partial streams by a
factor of two.

The two filters used to split the signal are called the analysis
filters. It is not possible to use just any pair of half-band filters to
do this. There is an extensive mathematical theory of wavelet filter
banks. Appropriately chosen filters allow us to undo the aliasing
introduced by the critical sampling in the down conversion process and
perfectly reconstruct the signal.

Iterative use of a wavelet process allows us to decompose the
low-frequency component into successively lower components.

For pictures, we apply wavelet filters in both the horizontal and
vertical directions. This results in four so-called subbands, termed
Low-Low (LL), Low-High (LH) High-Low (HL), and High-High (HH). The LL
band can be iteratively decomposed to create a wavelet transformation of
many levels.
