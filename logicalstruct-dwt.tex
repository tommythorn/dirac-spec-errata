The Discrete Wavelet Transform (DWT) takes an image and recursively
decomposes it into sets of horizontal/vertical frequency subbands of low
and high frequency. The result is a set of data which occupies the same
total memory space as the original image, however energy is concentrated
in the components corresponding to the lower frequency subbands.

\begin{figure}
    \centering
    \includegraphics[width=0.9\textwidth]{figs/dwt}
    \caption{Logical structure of DWT transformed frame}
    \label{fig:dwt}
\end{figure}

The inverse transform recursively builds up the image from the lowest
frequency subbands to the highest.  The forward and inverse transforms
are capable of perfectly reconstructing the original image.

The DWT used in Dirac has been optimised using a method known as
lifting.  Lifted wavelet filters have some useful properties:
\begin{itemize}
    \item they may filter data in place (i.e. using the same memory for
    the image, the transformed image and the partially processed
    subsets).

    \item the filtering operations are shorter than convolutional
        filters.

    \item the same filters may be used for the inverse and forward
        transforms.
\end{itemize}

Part of the forward transform process involves padding the image to
allow the wavelet transform to opperate correctly.  The padded data is
also required in the inverse transform, but is discarded afterwards.


