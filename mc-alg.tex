



The pixel $(x,y)$ of the predicted picture $P$ is formed by summing the
weighted contributions from all the blocks ($\B$) that cover $(x,y)$, for
each reference.  The pixel $(x,y)$ is covered by no more than four
blocks in the overlapping region of the block structure, leading to a
maximum of eight contributions (four from each reference).


The motion compensation process is defined as:

\begin{multline*}
P(x,y) =
 \left[
    2^{\tau+\sigma-1} +
    \displaystyle\sum^{\text{numrefs}}_{r}\, \sum^{\forall \B \text{ at } (x,y)}_{\B}
      \left\bracevert\begin{aligned}
        \text{let}\quad& W = \begin{cases}
                                \W_r &\scriptstyle\text{if numrefs = 1 or } \Brefsinuse = \lbrace0,1\rbrace,\\
                                2^{\tau-1}   &\scriptstyle\text{if numrefs = 2 and } \Bpredmode = \predIntra,\\
                                2^\tau &\scriptstyle\text{otherwise.}
                              \end{cases} \\%
        \text{in}\quad & W*\text{block\_predict}(r,\B,x,y)*\omega_{x-\Bx, y-\By}
        \end{aligned}\right.
 \right]
   \gg (\tau + \sigma)
\end{multline*}

\annotate{df}{sigma not defined}
\annotate{df}{too wide?}


