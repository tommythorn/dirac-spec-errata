
Dirac uses motion-adaptive prediction to reduce the bit rate of a
sequence. The prediction parameters define the prediction tools up to
the point that the error signal or residual is calculated.

\textbf{Block Parameters Index}

Block Parameters Index defines the default settings for the sizes and
separation of the blocks used in the prediction of pictures. The
parameters defined are the Chroma Block Width, Chroma Block Height,
Horizontal Chroma Block Separation, Vertical Chroma Block Separation,
Luma Block Width, Luma Block Height, Horizontal Luma Block Separation,
Vertical Luma Block Separation.

\textbf{Common Mode}

Common Mode indicates that the same values of Motion Vector and
Prediction Mode are being used within a Superblock.

\textbf{Chroma Block Width}

The width of a chrominance prediction block in pixels.

\textbf{Chroma Block Height}

The height of a chrominance prediction block in pixels.

\textbf{Chroma DC Residual}

In Intra Frames, this is the residual of the predicted mean level of
each Block.

See also DC Value

\textbf{DC Value}

When coding an Intra Reference Picture or an Intra Non-reference
Picture, the picture is not predicted, but simply wavelet transformed.
Whereas the mean value of predicted images is usually zero, the mean
value of intra picture is not. To avoid the block artefacts that this
would produce, the mean level of the Blocks is removed from the signal
before the wavelet transform process. This mean level is used as the
prediction.

Instead of transmitting the absolute value of the mean level for each
Block, the DC Value is predicted by reference to other Blocks in the
Superblock, and the difference signalled. The same process is applied to
luminance and the two chrominance signals.

See also Luma DC Residual, Chroma DC Residual.

\textbf{Global Motion}

The Global Motion Flag indicates the presence of global motion data. If
TRUE, the Global Motion Only Flag indicates whether only global motion
data is present.

Global Motion is a technique which works well when the whole, or at
least a large proportion, of the image is being subjected to the same
transformation. Dirac uses an eight-parameter model that allows for pan,
zoom, rotation, shear and change of perspective.

\textbf{Horizontal Chroma Block Separation}

The separation of chrominance prediction blocks horizontally.

\textbf{Horizontal Luma Block Separation}

The separation of luminance prediction blocks horizontally.

\textbf{Horizontal Perspective}

Metadata which describes the Motion Vector element caused by changes in
perspective.

See also Appendix XXX [Global Motion Compensation]

\textbf{Horizontal Offset Residual}

When the Motion Vector has been calculated in the encoder, it comprises
an indication of the  horizontal and a vertical offsets of the reference
pixel from the pixel to be calculated. This is transmitted as a
difference from a value that is predicted using information from
surrounding Blocks - i.e. as a residual (or difference) from the
predicted value.

\textbf{Horizontal Perspective}

XX I am not sure what physical features this conveys.

\textbf{Inter Reference Picture}

A picture which is coded by prediction with reference to other pictures,
and which itself is used as a reference for prediction when coding other
pictures.

There are two different types of Inter Reference Pictures. One uses only
one other picture as a reference. The other type can use two other
pictures as references.

\textbf{Inter Non Reference Picture}

A picture which is coded by prediction with reference to other pictures,
but which itself is not used as a reference for prediction when coding
other pictures.

There are two different types of Inter Non Reference Pictures. One uses
only one other picture as a reference. The other type can use two other
pictures as references.

\textbf{Intra Non Reference Picture}

A picture which is coded without prediction with reference to other
pictures, and which itself is not used as a reference for prediction
when coding other pictures.

\textbf{Intra Reference Picture}

A picture which is coded without prediction with reference to other
pictures, but which itself is used as a reference for prediction when
coding other pictures.

\textbf{Luma Block Width}

The width of a luminance prediction block in pixels.

\textbf{Luma Block Height}

The height of a luminance prediction block in pixels.

\textbf{Luma DC Residual}

In Intra Frames, this is the residual of the predicted mean level of
each Block.

See also DC Value

\textbf{Motion Vector}

A Motion Vector is an indication of which pixels in the reference frame
can be used as predictors for a particular pixel in the predicted frame.
It is the offset of the predicted pixel from the reference pixel (and is
conveyed, either as a global field, or on a block by block basis).

The observant will notice that this vector is a spatial vector, being a
spatial offset. Only by weighting by the temporal separation between the
reference and predicted fields, with due note of the sign, can a true
indication of motion be deduced. We would have called it prediction
vector, but there is a weight of existing custom and practice against us
- so Motion Vector it remains.

Two types of Motion Vector information are used: global and block motion
vectors. Global motion is intended to describe the motion of the
background, using a parametric model. The block motion vectors are
intended to describe the more varied motion of the foreground. The two
type of motion information are used together to define the overall
motion vectors.



XX Could introduce elements here about Motion Vector Prediction process
from TD 0.9

\textbf{Motion Vector Precision}

Motion Vector Precision is what it says on the tin. It is an important
factor in achieving efficient video compression. If the Motion Vector
Precision is too low the motion-compensated prediction residuals will be
larger than necessary and require more bits to be coded. However if the
Motion Vector Precision is too high the motion vectors themselves will
require a disproportionately large number of bits to be coded.

Empirically, motion vector precisions of � or � of a pixel have been
found to work well. However the optimum motion vector precision depends
on many factors, such as the nature of the sequence to be coded and the
desired compression quality. Dirac defaults to � pixel motion vector
precision but provides the flexibility to use a non-default value of
precision.

XX In the range ?????

\textbf{Pan Tilt}

A measure of the degree of horizontal movement of the whole image (Pan)
and vertical movement of the whole image (Tilt). This is used as part of
the information provided for Global Motion Compensation.

See also Appendix XXX [Global Motion Compensation]

\textbf{Perspective}

A measure of the change in perspective of the image. This is used as
part of the information provided for Global Motion Compensation.

See also Appendix XXX [Global Motion Compensation]

\textbf{Perspective Exponent}

A multiplier which allows us to use integers in place of floating point
in the delivery of the Perspective metadata.

\textbf{Picture Prediction Parameters}

Dirac predicts Inter frames from one or two reference frames using
motion compensation. Frame prediction parameters and data are not
included for Intra frames, which are indicated by the Start Code in the
Parse Information.

Dirac's motion model is overlapping block motion compensation. The block
sizes can be adapted to match the requirements of the Sequence. The
spatial displacement of each block, from the corresponding position in a
reference frame, is coded in the bitstream. These displacements are
known as motion vectors. Motion vectors are the displacement that should
be applied to the reference frame to predict the current frame. The name
motion vector is, therefore, a misnomer because they are actually
prediction displacement vectors. Nevertheless the term motion vector is
used for consistency with industry practice

Two types of motion vector information are used: global and block motion
vectors. Global motion is intended to describe the motion of the
background, using a parametric model. The block motion vectors are
intended to describe the more varied motion of the foreground. The two
types of motion information are used together to define the overall
motion vectors.

Global Motion parameters are included, or not, depending on the value of
the Prediction Mode flags. One or two sets of Global Motion parameters
are included in the Frame Prediction, one for each reference frame.

When predicting an image from two different pictures, we are able to
weight the contributions. This aids, amongst other predictions, the
ability of Dirac to handle cross-fades between images and fades to black
or white.

\textbf{Prediction Mode}

Prediction Mode is a pair of bits, each of which indicates whether one
of the references is used to form a motion compensated prediction of the
picture. If a Prediction Mode bit is asserted (True) then the
corresponding reference picture is used in the prediction. An intra
prediction block is indicated when both bits are not asserted (False).

\textbf{Prediction Unit}

The Prediction Unit is the process which identifies whether Blocks
within a Superblock can use Common Modes - using the same Motion Vectors
and Prediction Modes - or whether different Motion Vectors and
Prediction Modes are appropriate.

\textbf{Picture Weights}

When using more than one reference image for prediction, the default is
to weight the two equally relevant. At times of cuts or fades in the
sequence, it may be that the default setting is invalid, so there is an
option to set the weighting pragmatically.

\textbf{Reference Picture Number}

Dirac predicts Inter frames from one or two reference frames using
motion compensation. The Reference Picture Numbers are the offset(s) of
the one or two Reference Pictures from the Picture Number of the picture
being decoded.

\textbf{Retired Picture List}

In Dirac, reference frames need to be retained for a while to act as
references for succeeding frames. Non-reference frames are always
discarded when the current output frame number exceeds their frame
number.

The Dirac decoder contains a small buffer of previously decoded frames.

An important aspect of the way the decoder works is how it manages which
frames to retain in its frame buffer and which frames it discards.

The encoder may explicitly specify frames that are no longer required in
the buffer.  To support this method of frame management each frame may
contain a list of frames that should be discarded from the buffer.

If no frames are specified to be retired then Dirac discards the oldest
frame first, but only as necessary. That is, when space is needed in the
frame buffer the frame with the lowest frame number is discarded. It is
up to the encoder to ensure that this default process does not discard
frames that are needed by the decoder. This default discard procedure is
only invoked when the frame buffer size would otherwise exceed that
specified by the decoder level and no signalling is present to indicate
which frames should be discarded.

It is the responsibility of the encoder to ensure that the correct
frames are retained in the buffer. The decoder may assume that the
reference frames it requires will be available in the buffer.

\textbf{Split Mode}

Split Mode indicates that different values of Motion Vector and
Prediction Mode are being used within a Superblock.

\textbf{Superblock}

In many instances, the Motion Vectors of adjoining blocks are similar.
Dirac therefore aggregates Blocks into Superblocks to enable Prediction
Modes and Motion Vectors to be transmitted more efficiently.

The metadata describing the Prediction Modes and Motion Vectors can be
the same (Common Mode) or different (Split Mode).

Superblocks are four Blocks wide and four Blocks high. They are arranged
such that the overlapped part of the Blocks at the edge of the image
falls outside the image. There is always an integer number of
Superblocks, so extra Blocks (with no source content) have to be added
to ensure complete population of the space.

See also Common Mode, Split Mode, Prediction Mode Residual 1, Prediction
Mode Residual 2, DC Value, Motion Data, Luma DC Residual, Chroma DC
Residual

\textbf{Vertical Chroma Block Separation}

The separation of chrominance prediction blocks vertically.

\textbf{Vertical Luma Block Separation}

The separation of luminance prediction blocks vertically.

\textbf{Horizontal Offset Residual}

When the Motion Vector has been calculated in the encoder, it comprises
an indication of the  horizontal and a vertical offsets of the reference
pixel from the pixel to be calculated. This is transmitted as a
difference from a value that is predicted using information from
surrounding Blocks - i.e. as a residual (or difference) from the
predicted value.

\textbf{Vertical Perspective}

Metadata which describes the Motion Vector element caused by changes in
perspective.

See also Appendix XXX [Global Motion Compensation]

\textbf{Zoom Rotate Sheer}

A measure of the amount of zoom, rotation and sheer in the image. This
is used as part of the information provided for Global Motion
Compensation. The parameters in the bytestream are presented as a
combination of these parameters in a matrix representation.

See also Appendix XXX [Global Motion Compensation]

\textbf{Zoom Rotation and Sheer Exponent}

A multiplier which allows us to use integers in place of floating point
in the delivery of the Zoom Rotate Sheer metadata.
