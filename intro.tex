\subsection{Purpose}
\label{intropurpose}
Dirac was developed to address the growing complexity and cost of current video
compression technologies, which provide greater compression efficiency
at the expense of implementing a very large number of tools. Dirac is
a powerful and flexible compression system, yet uses only a small number
of core tools. A key element of its flexibility is its use of the wavelet
multi-resolution transform for compressing pictures and motion-compensated 
residuals, which allows Dirac to be used across a very wide range of resolutions
without enlarging the toolset.

Dirac began as an Open Source software project, and reference implementations
of the decoder and encoder are available at \underline{http://sourceforge.net/projects/dirac}.

\subsection{Scope}
\label{introscope}

This document specifies normative decoder operations (``semantics") and 
stream syntax. The stream syntax is primarily specified by means of
pseudocode, the conventions of which are described in Section \ref{pseudocode}.
The decoder semantics are specified by means of a mixture of pseudocode
and conventional mathematical symbolism.

A number of other elements are also included for informative purposes.
The specification is not an implementation guide, and in the interests
of clarity many of the operations are specified in a way that would not be 
efficient to implement. However, we have attempted
to indicate where this is so, and to suggest ways in which an efficient implementation
may be achieved, but these are by no means exhaustive. An optimised Open Source
software Dirac encoder and decoder system, named Schr\"odinger, is available at
\underline{http://sourceforge.net/projects/schrodinger}, and may be studied to aid implementation.

In addition, we are well aware that many users of this document may wish
to make both encoders and decoders. There are many sources of information
on how to design efficient compression algorithms, for example for entropy coding,
motion estimation, frame-dropping, rate control, motion estimation and 
rate-distortion optimisation. This document does not attempt to address these
issues in detail, but to provide supplementary information where appropriate
to allow those reasonably ``skilled in the art" to develop a Dirac encoder
rapidly and accurately, and approach design compromises knowledgably.

\subsection{Status}
\label{introstatus}

This is version \SpecVersion of the Dirac specification. The document includes
a full description of the core Dirac stream syntax and decoder operations, together
with compatible extensions to support low-delay operation (professional profiles). It does
not yet contain a full specification of profiles and levels supported by Dirac.

\subsection{Document structure}
\label{introdocstruct}

This document specifies the Dirac decoder and stream structure in terms of
a layered model:

\begin{enumerate}
    \item Stream data access
    \item Parsing and interpretation of the Dirac stream
    \item Picture decoding operations
\end{enumerate}

Stream data access consists of the operations used to extract data values
(of boolean and integer type) from a raw Dirac bitstream. These include
data that has been encoded ``literally" (i.e. according to conventional bit-wise
representations), variable-length codes, and data entropy coded using arithmetic
encoding. Stream data access methods are used both by parsing and decoding operations.

Parsing and interpretation defines the structure of Dirac streams, and defines
intermediate decoder data structures in which extracted data is stored, 
used to control picture decoding processes.

Picture decoding operations produce decoded pictures from these populated
data structures by applying specified functions to them. These operations
are logically distinct from those for navigating the stream or reading the stream data.

Note in particular that the distinction between parsing and picture decoding is
{\em not} exactly that between syntax and semantics: complex semantics are
required for correct parsing of the stream as well as for decoding pictures. 

It is perhaps unusual in a specification to separate these layers quite so distinctly, 
and our purpose in doing so is to provide much greater clarity. For implementors,
we hope that the decoupling of the stream structure from the (computationally intensive)
picture decoding processes will avoid imposing
implicit design decisions merely through the style of the specification. Many
other users of the specification will not be interested in the precise format
of stream elements but in how the underlying algorithm works - or vice-versa.
It should be possible to construct a Dirac parsing engine, for example for
frame skipping in video playback applications, extremely simply and without
requiring comprehension of the entire specification.

This layered structure is reflected in the structure of the specification,
which, after defining conventions used, is divided into three
corresponding parts: stream data access, defining functions for data types; 
accessing and parsing the Dirac bitstream and populating data
structures (including the wavelet coefficients and motion data); and 
high-level decoder operations and picture output, specifically the
inverse wavelet transform and motion compensation.

In addition to these parts, appendices deal with standard settings, parameter
presets and levels and profiles.