%%%%%%%%%%%%%%%%%%%%%%%%%%%%%%%%%%%%%%%%%%%%%%%%%%%%%%
% - This part relates to informative and normative - %
% - concepts for Dirac parsing and decoding        - % 
%%%%%%%%%%%%%%%%%%%%%%%%%%%%%%%%%%%%%%%%%%%%%%%%%%%%%%

%\section{Overview of Dirac video coding (Informative)}%%%%%%%%%%%%%%%%%%%%%%%%%%%%%%%%%%%%%%%%%%%%%%%%%%
% - This chapter gives an informative overview - %
% - Dirac video coding concepts                - % 
%%%%%%%%%%%%%%%%%%%%%%%%%%%%%%%%%%%%%%%%%%%%%%%%%%

\subsection{A general introduction to video compression for beginners}This is a simple guide to video coding. You can miss this subsection out
if you already know the basics.

Video signals are made up of a succession of still pictures, displayed
one after the other. Each picture is made up from a series of elementary
units called pixels, arranged in a raster of lines.

The raw total capacity required for the transmission of moving pictures
is the product of the number of pixels per picture, pictures per second,
colours in use and the quantising accuracy adopted. In nearly every
field of application, the resultant raw bandwidth required exceeds that
available by more than an order of magnitude.  If we keep the same size
pictures, then the only variable we have with the simple system is to
change the quantisation accuracy and lowering the accuracy rapidly leads
to poor quality pictures.

The solution which has been used from before 1970 is to make a
prediction of the value of each pixel using information that should be
available in the decoder. In general, if the prediction is good, then
the difference between reality and the prediction is small. The entropy
of the difference signal is low, and so we need less capacity to deliver
it. One simple solution was to recognise that the quantisation accuracy
required for the difference signal could be coarser than that required
for the original picture, and so fewer bits are needed to deliver the
difference signal than for the original.

This has formed the basis for many of the early video compression
systems.  The elements that have changed over the years have been

\begin{itemize}
    \item the algorithms used for making the prediction, and
    \item the formats used for delivering the difference signal
    efficiently to the receiver.
\end{itemize}

Many of the changes have been enabled by the availability of better
electronic circuitry. Real-time operation puts an upper limit on the
time available for processing.  We can only use more sophisticated
algorithms if we can carry out the calculations in the time available.
Fortunately, the improvement in the speed of hardware has almost matched
the development of algorithms.

Early algorithms made use of simple predictions predicting one pixel
from others nearby in the picture, or (spatial prediction in the
jargon).

Predictions were helped when field-stores became affordable (and smaller
than a small garden shed) and it became possible to use information from
the previous field (temporal prediction in the jargon). This process
works well when the picture is still, but is less effective when the
pictures depict a lot of movement.

This led to a whole raft of developments, seeking to identify motion in
the picture. Knowing how parts of the scene are moving (motion
estimation in the jargon) allowed much more accurate prediction than had
been possible hitherto. These developments led eventually to mature
products such as the MPEG 2 compression system.

Having created a good prediction, it is inevitable that it will not be
perfect. The error signal is the signal we wish to convey to the
decoder.  In any system, the raw error signal is at least as
bandwidth-hungry as the original. In most cases it is slightly more
hungry.

However, the error signal has some physical properties which help us to
reduce its bandwidth. In information theory terms, the error signal is
rarely totally random, real pictures have properties which distinguish
them from random noise.  We can therefore use some of these properties
to reduce the bit-rate further.

The spectrum of the error signal is usually heavily biased towards the
low frequency end. This is a direct consequence of the prediction
process usually being reasonably good. It also turns out that the eye is
less sensitive to small inaccuracies in the high frequency components of
the error signal.

Taken together, these observations provide the potential for coarser
quantisation or omission of some of the components of the error signal.

There are several methods available if we wish to translate the temporal
signal we started with into the frequency domain. Although many coding
systems use Discrete Cosine Transforms, we preferred to use the Discrete
Wavelet Transform. This approach divides the signal into higher and
lower frequency sub-bands. By quantising the different sub bands
appropriately, we achieve significant reduction in bandwidth.

Finally, when all the information for this system is packed together,
there is still a structure it is not statistically random. Information
theory says that we can use entropy coding to further increase the
randomness, and thereby reduce the bandwidth.  One of the more powerful
of these is arithmetic coding the system adopted in Dirac.  So we have
now outlined some of the key elements of a modern coder.


Fig XX. An outline of a typical modern video coder.

It is worth a quick look at the simple representation of the encoder in
Fig. XX as it leads us to understand the receiver topology.

The input signal $V_{in}$ is compared with the prediction $P$ to produce
an error signal $e$.  This is then compressed and passed (with the
various elements of compression metadata) to the arithmetic coder for
transmission.

The prediction is created from a local version of the signal for a very
good reason.  This is the signal that the decoder is able to recreate
with the information available to it. The signal delivered to the
receiver  $e_{TQ}$ allows the receiver to recreate a close copy of the
prediction error $e$. If we compare the two signals,

\begin{align*}
  V_{in} &= P + e \\
  V_{local}	&= P + e'
\end{align*}

As the difference between the two error signals $e$ and $e'$ is the
distortion introduced by the compression algorithm, the local signal is
a close approximation to the input signal. Looking carefully at Fig. XX,
we can see that we can create a decoder using a subset of the encoder.
This is shown in Fig. XXX.

It is quite clear that the main elements of the receiver are duplicated
in the encoder.  The encoder has to maintain a local decoder within it,
in part so that the result of the compression can be monitored at the
time of compression, but mainly because compressed pictures must be used
as reference frames for subsequent motion compensation else the encoder
and the decoder will not remain in synchronism.

The motion vectors are delivered from the encoder as metadata. This
avoids the need to analyse motion vectors in the receiver, and allows
the encoder considerable flexibility in the choice of appropriate motion
vectors.  We will now move on to consider the application of this
technology to Dirac.


Fig. YY An outline of the decoder


\subsection{Outline of the compressino methods used in Dirac}The Dirac video codec uses three main techniques to compress the signal:

\begin{itemize}
    \item Prediction using motion compensation to remove temporal redundancy
    \item Wavelet transformation and quantisation to remove spatial redundancy
    \item Arithmetic coding of the resulting data to maximise efficiency
\end{itemize}


Initially, similarities between frames (temporal redundancy) are
exploited to predict one frame from another. The process is aided by
motion vectors - metadata detailing where a particular pixel in the
predicted frame might have been in the reference frame. It is a bit
wasteful to assign a motion vector to each pixel, so pixels are
aggregated in blocks. In Dirac, blocks are overlapping. This reduces
some of the artefacts found at the boundaries of blocks in some earlier
systems. The motion is calculated to sub pixel accuracy.

Once the prediction has been made, it is compared with the actual image
to be transmitted.  The resulting difference (error) is potentially
greater in range than the original signal. To reduce it the difference
signal is then transformed using the discrete wavelet transform. This
process, for the majority of video sequences, produces coefficients
which are largely zero or near zero, and most of the non-zero
coefficients are concentrated near at the lower frequency end of the
range. The properties of the eye allow us to coarsely quantise the high
frequency coefficients.

Both the motion vectors and the quantised coefficients are then further
compressed using arithmetic coding.

\subsection{Prediction using motion compensation}The object of motion compensation is to try to predict the picture in
one image from others in the sequence. In television, the picture
usually contains moving objects, so a key part of the process is to
identify the moving objects and their the details of the motion.

In Dirac, the motion vector is merely an indication of which pixel in
the reference picture can be used as a good prediction for a particular
pixel in the current picture.

\subsubsection{Types of picture}Three types of picture are defined.

Intra pictures are coded without reference to other pictures in the
sequence. These pictures form a useful point to start the decoding
process. They also have uses if low-delay coding is required - a
sequence of Intra pictures has mimimum delay, at the expense of
potentially greater bandwidth requirements. There is a third use for
Intra pictures: when the sequence is highly dynamic, prediction using
motion compensation may be impractical. In such instances, a coder may
choose to default to a sequence of Intra frames.

Inter pictures are coded with reference to other pictures in the
sequence, and are split into two types:
\begin{itemize}
	\item references for other pictures
	\item not references for other pictures
\end{itemize}

Within the receiver, we have to arrange that all the references for
pictures are available at the right time. This means that pictures are
often delivered in a different sequence from the display sequence, to
ensure that the buffer memory in the receiver has the necessary
information to decode pictures when it is needed.


\subsubsection{Blocks}In theory, we could define a motion vector for every pixel in the image.
This would be extremely data intensive and of no practical value.

Instead, pixels are grouped into small regions or blocks, with a single
motion vector assigned to each block. Ideally these blocks are large -
thus minimising the amount of information we have to transmit. However,
the larger the block, the greater the chance that we have more than one
object or region, and so a single motion vector might be inappropriate.

Although there are several standard block sizes identified within the
specification, you can choose your own, and that could be as large as
the original picture. This would not be unreasonable for a picture which
is a static scene, with the camera being panned across it - although
there are other ways of identifying this in Dirac.

One of the hard tasks the coder has is to identify which is the optimum
size of block, and the best compromise for the motion vector in the
block.

When we have created the prediction, we then calculate the error. The
error is often greatest at the block boundaries, as this is where the
motion vector is often least accurate.

To overcome this, Dirac overlaps blocks. Each pixel in the overlap
regions uses a weighted prediction, incorporating information from all
the blocks it may lie in. This smooths the error signal, and making the
following wavelet transformation much more effective.

\subsubsection{Global motion parameters}Although we said that it is of no practical value to specify a motion
vector for each pixel, there are exceptions.

Some types of motion are more likely to be features of the camera than
the scene. For example, a camera may pan, tilt, rotate, zoom, sheer, or
change of perspective through, say tracking.

If the coder can identify such motion, then Dirac permits the motion to
be signalled by a small set of parameters, which in effect assign a
motion vector to each pixel.

\subsubsection{Accuracy}Often, we find that the motion vectors required to match a pixel in a
reference frame to the predicted frame are not integer values.

In Dirac, we can specify motion vectors to 1/8 pixel accuracy.

This means that the coder and decoder have to carry out a process that
is effectively an upconversion of the signal.

\subsubsection{Intra frames}The Intra fields are not predicted using motion compensation.

This leaves us with a potential problem. The wavelet transform process
assumes that we have a signal which tends to a zero average.

The Intra field is processed to give it a zero mean by removing the DC
component (average value) of each block from the signal. The DC
components are signalled separately. In effect the DC components are a
local spatial prediction, rather than a temporal prediction as applied
in motion compensation.

\subsection{Wavelet tranforms}The consequence of processing the Intra frames by removing the DC
values, and the Inter frames by removing the predicted values gives us a
difference signal. This difference signal is hopefully largely zero, but
can have some large peaks, of either polarity. The signal usually has a
large amount of low-frequency energy, but with occasional elements of
high frequency as the prediction process gets it wrong.

Conventional theory says that we can manipulate this signal in the
frequency domain to reduce the amount of information we need to
transmit. The properties of the eye are such that many of the higher
frequency components are less sensitive to coarse quantisation.

We can use wavelet transforms in Dirac. The wavelet transform
decorrelates the data in a roughly frequency-sensitive way, and
preserves the fine details of images better than the ubiquitous Discrete
Cosine Transform.

\subsubsection{Wavelet analysis}Put simply, wavelet analysis splits a signal into a low and a high
frequency component, and then subsamples the two partial streams by a
factor of two.

The two filters used to split the signal are called the analysis
filters. It is not possible to use just any pair of half-band filters to
do this. There is an extensive mathematical theory of wavelet filter
banks. Appropriately chosen filters allow us to undo the aliasing
introduced by the critical sampling in the down conversion process and
perfectly reconstruct the signal.

Iterative use of a wavelet process allows us to decompose the
low-frequency component into successively lower components.

For pictures, we apply wavelet filters in both the horizontal and
vertical directions. This results in four so-called subbands, termed
Low-Low (LL), Low-High (LH) High-Low (HL), and High-High (HH). The LL
band can be iteratively decomposed to create a wavelet transformation of
many levels.

\subsubsection{Parent-child relationships}In wavelet analysis, each subband represents a filtered and subsampled
version of the picture. Because all the subbands are derived from a
single source image, there is likely to be some form of relationship
between the images in the different subbands.

The coefficients of each subband relate to specific areas in the image.
We find that there is often a correlation between these specific areas
in the different subbands.

The subsampling structure means that a coefficient in the lowest level
corresponds to  2 by 2 block of coefficients in the next level, and so
on up the levels. In the jargon, the low-level component is referred to
as the parent and the higher-level component is referred to as the
child.

When coding picture features (edges on objects especially), significant
coefficients are found distributed across subbands, in position related
by the parent-child structure.

A coefficient of a child is more likely to be significant if its parent
is also significant. Children with zero or small parents seem to have
different statistics from children with large parents or ancestors.

These features allow us to entropy code the wavelet coefficients after
they have been quantised.

\subsection{Entropy coding}Transmission of the raw motion vectors and wavelet coefficients is
inefficient. There are still many redundancies in the data, and the form
of the data is itself suboptimum.

Coding of the motion vectors is especially important for codecs with a
high level of motion accuracy (quarter or eighth pixel say). Motion
vector coding and decoding is quite complicated, since significant gains
in efficiency can be made by choosing a good prediction and entropy
coding structure.

The processes of removing the redundancies is probably one of the most
complicated part of the codec. Similar processes are used for coding the
motion vectors and the wavelet coefficients. In various ways they use a
combination of

\begin{itemize}
	\item prediction,
	\item binarisation,
	\item context modelling and
	\item adaptive arithmetic coding.
\end{itemize}

\subsubsection{Entropy coding of wavelets}The entropy coding used by Dirac in wavelet subband coefficient coding
is based on three stages: binarisation, context modelling and adaptive
arithmetic coding.

Figure: Entropy coding block diagram

The purpose of the first stage is to provide a bitstream with easily
analysable statistics that can be encoded using arithmetic coding which
can adapt to those statistics, reflecting any local statistical
features.

Binarisation is the process of transforming the multi-valued coefficient
symbols into bits. The resulting bitstream can then be arithmetic coded.

Transform coefficients tend to have a roughly Laplacian distribution,
which decays exponentially with magnitude. This suits so-called unary
binarization. Unary codes are simple variable-length codes in which
every non-negative number $N$ is mapped to $N$ zeros followed by a 1:


\begin{verbatim}
U(0)    =   1
U(1)    =   0   1
U(2)    =   0   0   1
U(3)    =   0   0   0   1
U(4)    =   0   0   0   0   1
U(5)    =   0   0   0   0   0   1
U(6)    =   0   0   0   0   0   0   1
Bins:       1   2   3   4   5   6   7
\end{verbatim}

For Laplacian distributed values, the probability of $N$ occurring is
$2-(|N|+1)$, so the probability of a zero or a 1 occurring in any unary
bin is constant. So for an ideal only one context would be needed for
all the bins, leading to a very compact and reliable description of the
statistics. In practice, the coefficients do deviate from the Laplacian
ideal and so the lower bins are modelled separately and the larger bins
lumped into one context.

The process is best explained by example. Suppose one wished to encode
the sequence:

\begin{verbatim}
-3 0 1 0 -1
\end{verbatim}

When binarized, the sequence to be encoded is:

\begin{verbatim}
0 0 0 1 | 0 | 1 | 0 1 | 1 | 1 | 0 1 | 0
\end{verbatim}

The first 4 bits encode the magnitude, 3. The first bit is encoded using
the statistics for Bin1, the second using those for Bin 2 and so on.
When a 1 is detected, the magnitude is decoded and a sign bit is
expected. This is encoded using the sign context statistics; here it is
0 to signify a negative sign. The next bit must be a magnitude bit and
is encoded using the Bin 1 contexts; since it is 1 the value is 0 and
there is no need for a subsequent sign bit. And so on.

The context modelling in Dirac is based on the principle that whether a
coefficient is small (or zero, in particular) or not is well-predicted
by its neighbours and its parents. Therefore the codec conditions the
probabilities used by the arithmetic coder for coding bins 1 and 2 on
the size of the neighbouring coefficients and the parent coefficient.

The reason for this approach is that, whereas the wavelet transform
largely removes correlation between a coefficient and its neighbours,
they may not be statistically independent even if they are uncorrelated.
The main reason for this is that small and especially zero coefficients
in wavelet subbands tend to clump together, located at points
corresponding to smooth areas in the image, and as discussed elsewhere,
are grouped together across subbands in the parent-child relationship.

Conceptually, an arithmetic coder can be thought of a progressive way of
producing variable-length codes for entire sequences of symbols based on
the probabilities of their constituent symbols.

For example, if we know the probability of 0 and 1 in a binary sequence,
we also know the probability of the sequence itself occurring. So if

$P(0)=0.2, $

$P(1)=0.8$

then

$P(11101111111011110101)=(0.2)*3*(0.8)*17=1.8 * 10^{-4}$ (assuming
independent occurrences).

Information theory then says that optimal entropy coding of this
sequence requires $log_2 (\frac{1}{p})=12.4$ bits. Arithmetic coding
produces a code word very close to this optimal length, and
implementations can do so progressively, outputting bits when possible
as more arrive.

All arithmetic coding requires are estimates of the probabilities of
symbols as they occur, and this is where context modelling fits in.
Since arithmetic coding can, in effect, assign a fractional number of
bits to a symbol, it is very efficient for coding symbols with
probabilities very close to 1, without the additional complication of
run-length coding. The aim of context modelling within Dirac is to use
information about the symbol stream to be encoded to produce accurate
probabilities as close to 1 as possible.

Dirac computes these estimates for each context simply by counting their
occurrences. In order for the decoder to be in the same state as the
encoder, these statistics cannot be updated until after a binary symbol
has been encoded. This means that the contexts must be initialised with
a count for both 0 and 1, which is used for encoding the first symbol in
that context.

An additional source of redundancy lies in the local nature of the
statistics. If the contexts are not refreshed periodically then later
data has less influence in shaping the statistics than earlier data,
resulting in bias, and local statistics are not exploited. Dirac adopts
a simple way of refreshing the contexts by halving the the counts of 0
and 1 for that context at regular intervals. The effect is to maintain
the probabilities to a reasonable level of accuracy, but to keep the
influence of all coefficients roughly constant.

\subsubsection{Entropy coding of motion vectors}The basic format of the coding the motion vectors is similar to the
coding of wavelet data: it consists of prediction, followed by
binarisation, context modelling and adaptive arithmetic coding.


Figure: motion vector entropy coding architecture

All the motion vector data are predicted from previously encoded data
from nearest neighbours.

\subsection{Bytestream}The bytestream of Dirac is the complete, compressed video stream, ready
for file storage or transmission.

The features are

\begin{itemize}
    \item A parse structure which gives ready access to the video, even
    in mid file. This identifies all the static features of the stream.
    Intelligent systems can use the redundancy in this structure to
    recover from errors in the stream.

    \item Blocks of data comprising the arithmetically-coded motion
    vector information

    \item Blocks of data comprising the arithmetically-coded wavelet
    coefficients
\end{itemize}

Because of the arithmetic coding, it is difficult to provide a simple
specification which gives meaningful details of the bytestream without
some reference to either the early coding processes, or the later
decoding process. The simple description of the bytestream would only
refer to compressed motion vectors, and compressed wavelet coefficients.
We have therefore had to provide extra information to explain how to
unpack this data, and how to use the resulting structured data.



%\clearpage
%\section{Logical constructs used in Dirac}%%%%%%%%%%%%%%%%%%%%%%%%%%%%%%%%%%%%%%%%%%%%%%%%%%
% - This chapter defines normative             - %
% - Dirac structures                           - % 
%%%%%%%%%%%%%%%%%%%%%%%%%%%%%%%%%%%%%%%%%%%%%%%%%%

% Needs a fair bit of work - structures are all of different 
% type: semantic, syntactic, functional

This section describes the conceptual structures that exist within the
Dirac codec.

\subsection{Sequence}
A consecutive list of frames that all share the exact same sequence and
display parameters is a sequence.  Otherwise put, the sequence and
display parameters must not change during a sequence.  A sequence may be
terminated by an end of sequence notification.

The frames within a sequence are numbered consecutively in display
order.

\begin{figure}
    \centering
    \includegraphics[width=0.7\textwidth]{figs/sequence}
    \caption{Sequences, Access Units and Frames}
    \label{fig:sequence}
\end{figure}


\subsection{Parse unit}
Seekable constructs within the bitstream are called parse units.  Parse
units start with a synchronisation word, type identifier and pointers to
the start of the next/previous parse unit.

Synchronisation words allow for easy stream synchronisation.

There is no deliberate attempt to prevent the arithmetic coding of data
in the bytestream from containing accidental repetitions of the
synchronisation word. A decoder should check that the next/previous
pointers in the parse unit point to a valid synchronisation word,
especially on start up.

There are only two types of seekable construct within the bitstream:
Access Unit Headers and Frames.

No synchronisation words are used within a video frame to regain
synchronisation.

\subsection{Access unit}Access units group frames together. The existance of multiple Access
Units in a sequence allows the sequence to be randomly accessed to the
granularity of an Access Unit.

Access Units commence with an Access Unit Header, containing parameters
that stay in effect for the whole Access Unit (ie, until the next Access
Unit starts).

Access units are unbounded in length.  The end of an Access Unit is
signalled by the start of another.

Access units are logical constructs.  Only an Access Unit Header is
transmitted at the start of the access unit in the bitstream.

Access unit headers contain three categories of information:

\begin{itemize}
    \item Parse parameters -- provide the decoder with information
        required to parse all subsequent data.

    \item Sequence parameters -- provide the video parameters for
        correct decoding.

    \item Display parameters -- provide metadata about the video that is
        not required in the decode process, but may be required by a
        display divice to present the video correctly.
\end{itemize}

The first frame in an Access Unit must be an Intra coded frame.

\subsection{Frames}Frames are the fundamental components of a television image. In Dirac,
each transmitted frame comprises picture data in the form of wavelet
transformed residuals and optional motion vector data.  Each frame may
be Inter of Intra coded, the former using a motion compensated
prediction based upon one or two reference frames, the latter not.

A frame may be stored for later use as a reference frame.  The choice
depends on the properties of the frame type: whether it is an Inter or
Intra frame,  the reference status (Reference/NonReference) and the
number of reference frames used in the prediction.

Reference frames are stored in the decoder until they are either
explicitly signalled as expired by the encoder, or implicitly retired
according to the level and profile settings.

\subsection{Coordinate systems}\input{logicalstruct-coordinates}
\subsection{Frame ordering}Each frame is given a unique number, consecutively increasing for each
frame in the original source material; this ordering is known as display
order, the order in which frames must be displayed.  In order to reduce
the codec's complexity, each frame the decoder receives must be received
after any reference frames it uses; this is the coded order, the order
in which coded frames are transmitted. See
figure~\ref{fig:frame-ordering}. Note that this means that the
transmitted order of the frames is not the display order. There will
therefore have to be a buffer of sufficient size to accommodate the
reordering.

\begin{figure}
    \centering
    \includegraphics[width=0.9\textwidth]{figs/frame-ordering}
    \caption{Frame reordering for transmission}
    \label{fig:frame-ordering}
\end{figure}

References may not persist across Video Sequence boundaries.

\paragraph{Frame ordering across Access Unit boundaries}
Consider figure x, with two Access Units $A$ and $B$.  While frame
x occurs in display order before the start of access unit $B$, it
references frames x1 (the first frame of $B$) and therefore must be
coded and transmitted after x1.  If decoding were to commence at the
start of $B$, frame x would not be decoded.



\subsection{Wavelet transform}The Discrete Wavelet Transform (DWT) takes an image and recursively
decomposes it into sets of horizontal/vertical frequency subbands of low
and high frequency. The result is a set of data which occupies the same
total memory space as the original image, however energy is concentrated
in the components corresponding to the lower frequency subbands.

\begin{figure}
    \centering
    \includegraphics[width=0.9\textwidth]{figs/dwt}
    \caption{Logical structure of DWT transformed frame}
    \label{fig:dwt}
\end{figure}

The inverse transform recursively builds up the image from the lowest
frequency subbands to the highest.  The forward and inverse transforms
are capable of perfectly reconstructing the original image.

The DWT used in Dirac has been optimised using a method known as
lifting.  Lifted wavelet filters have some useful properties:
\begin{itemize}
    \item they may filter data in place (i.e. using the same memory for
    the image, the transformed image and the partially processed
    subsets).

    \item the filtering operations are shorter than convolutional
        filters.

    \item the same filters may be used for the inverse and forward
        transforms.
\end{itemize}

Part of the forward transform process involves padding the image to
allow the wavelet transform to opperate correctly.  The padded data is
also required in the inverse transform, but is discarded afterwards.



\subsection{Subband coding}Using the subband numbering in figure~\ref{fig:dwt}, subbands are
transmitted from 10 downto 1, ie from the lowest frequency to the
highest frequency.  Since the wavelet transform and its inverse are
lossless, coding gains are made by quantizing the subbands for
transmission.

Further coding gains may be made by exploting the correlation in
frequency components, whereby each pixel in a subband is predicted from
some of its neighbours.  Correlation accross different frequency bands
is also exploited.

An extension avaliable to encoders is to use multiple quantizers per
subband.  This is achieved by subdividing a subband into a number of
codeblocks, each having its own quantizer.  As an aside, this means that
we can identify parts of the picture which require most accurate coding
(for example we often take most notice of errors in the region of a
person's eyes when looking at images of faces) and enhance the coding
accuracy in this area. This process is sometimes referred to as region
of interest coding.

\subsection{Blocks}For the purpose of local Motion Compensation, the whole image is divided
up into a number of overlapping blocks that covers the entire image.

\begin{figure}
    \centering
    \subfigure[A block]{
        \label{fig:}
        \includegraphics[width=1in]{figs/block}
    }
    \hspace{0.5in}
    \subfigure[Positioning of blocks in a frame]{
        \label{fig:}
        \includegraphics[width=0.25\textwidth]{figs/block-offset}
   }
   \caption{block \ldots}
   \label{fig:blocks}
\end{figure}

Figure~\ref{fig:blocks} shows a block and how they overlap at the
pictures left and top edges.

For frames that have their chroma components downsampled, the block
sizes are reduced by the same factor.


\subsection{Local motion compensation}
For each block, a vector is provided that gives the relative position of
a block of equal dimension in a reference frame.  This referenced block
is weighted according to a global weight for that frame and weighted
again using a weighting matrix designed to make blocks overlap nicely
and reduce high reduce high frequency components entering the residual
at block edges.

Dirac allows Motion Vectors to be specified to subpixel precision.  When
using a Motion Vector to subpixel precision, the reference image is
upconverted over the referenced area and a sampled upconverted reference
is used.

The maximum motionvector precision is an eighth of a pixel.

Motion Vectors are predicted


\subsection{Global motion compensation}\input{logicalstruct-globalmc}
\subsection{Superblocks}
To allow a more efficient representation of blocks and Motion Vectors,
and to allow further explotation of the spatial redundancy in motion
vectors and additionally allow a tradeoff between Motion Vector accuracy
to bytestream symbols required, blocks are transmitted in a 4 by 4
arrangement called a Superblock (figure~\ref{fig:superblock-pu-16of1x1}).

\begin{figure}
    \centering
    \subfigure[16 perdiction units, each a single block]{
        \label{fig:superblock-pu-16of1x1}
        \includegraphics[width=0.25\textwidth]{figs/superblock-16pu}
    }
    \hspace{0.5in}
    \subfigure[4 prediction units, each covering 2x2 blocks]{
        \label{fig:superblock-pu-4of2x2}
        \includegraphics[width=0.25\textwidth]{figs/superblock-4pu}
    }
    \hspace{0.5in}
    \subfigure[1 prediction unit, covering 4x4 blocks]{
        \label{fig:superblock-pu-1of4x4}
        \includegraphics[width=0.25\textwidth]{figs/superblock-1pu}
    }
    \caption{Illustration of grouping blocks into prediction units
    (numbered) and their relationship to superblocks}
    \label{fig:superblock-pu}
\end{figure}

Blocks can be merged inside the superblock into arrangements of 4 of 2 by 2
blocks, or 1 of 4 by 4 blocks (figures~\ref{fig:superblock-pu-4of2x2} and
\ref{fig:superblock-pu-1of4x4} respectively).  In each
case, only one motion vector is transmitted per merged set of blocks
(also known as a prediction unit).

Each prediction unit has a prediction mode associated with it, which may
be one of:
\begin{itemize}
    \item Intra -- for blocks with no Motion Vector associated with
        them.
    \item Ref1Only -- the Motion Vector applies to the first reference frame
        only.
    \item Ref2Only -- the Motion Vector applies to the second reference
        frame only.
    \item Ref1and2 -- two Motion Vectors are sent, each using one
        reference frame.
\end{itemize}

Global motion compensation is opted in/out of at the Superblock level.
If opted in, all prediction units within the super block will be
globally motion compensated, unless they signal that a particular
prediction unit should be intra coded instead.



\subsection{Arithmetic coding}All the data for Motion Vectors and the transformed residuals are
transmitted after having been arithmetically coded.

The arithmetic coder is a black box device that is given an array of
data to code. It codes each symbol in context.

Arithmetic coding achieves good compression through maintaining accurate
counts of the symbol probabilities.  Each set of symbol probabilities is
held in a context (in effect recognising which other data in the array
may be a good match to the symbol to be coded).

Different contexts are used depending upon where the symbol originated
or depending upon some aprori information about the symbol.




\clearpage
\section{The conventions used in the specification}%%%%%%%%%%%%%%%%%%%%%%%%%%%%%%%%%%%%%%%%%%%%%%%%%%
% - This chapter defines specification         - %
% - conventions                                - % 
%%%%%%%%%%%%%%%%%%%%%%%%%%%%%%%%%%%%%%%%%%%%%%%%%%
\label{spec-conventions}
\subsection{State representation}

This standard uses a state model to express parsing and decoding operations. 
The state of the decoder/parser shall be stored in the variable state. Individual elements of the decoder $\StateName$ (state variables) may be accessed
 by means of 
named labels, e.g. $\StateName[\text{VAR\_NAME}]$ (i.e. state is a map, as defined in Section \ref{datatypes}). 

The decoder state shall be globally accessible within the decoder. Other 
variables, not declared as inputs to a process, shall be local to that process.
The parsing and decoding operations are defined in terms of modifying the 
decoder state. The state variables need not directly correspond to elements 
of the stream, but may be calculated from them taking into account the decoder 
state as a whole. For example, a state variable value may be differentially 
encoded with respect to another value, with the difference, not the variable 
itself, encoded in the stream. 

\begin{comment}
The Dirac stream syntax structure is illustrated with informative parse diagrams,
contained in Annex \ref{parsediagrams},
 that complement the normative stream syntax definitions.
\end{comment}
 The parsing process 
is defined by means of pseudocode and/or mathematical formulae. The 
conventions for these elements are described in the following sections.

\subsection{Number formats}
\label{mathnotation}

Numbers without a prefix shall be interpreted as decimal numbers.

The prefix b indicates that the following value shall be interpreted as a binary
natural number (non-negative integer). 

{\bf Example} The value b1110100 is equal to the decimal value 116. 

The prefix 0x shall indicate that the following value is to be interpreted as a hexadecimal (base 16) natural number. 

{\bf Example} The value 0x7A is equal to the decimal value 122. 

\subsection{Data types}
\label{datatypes}

\subsubsection{Elementary data types}

Three basic types are used in the pseudo code:
\begin{description}
\item[Boolean] - A Boolean variable that has only two possible values: $\true$ and $\false$.
\item[Integer] - A positive or negative whole number or zero.
\item[Label] - a unique immutable value used in control structures and to 
access maps (see below).
\end{description}

\subsubsection{Compound data types}

Elementary and compound data types may be aggregated into a single 
compound data type.
There are three compound data type:

\begin{description}
\item[Set]
 A collection of data types. A set is indicated by enclosing the elements within 
curly braces, for example $\{a,b,c\}$ represents a set containing the values $a,b$
 and $c$. An empty set may be indicated by $\{\}$. The usual set-theoretic 
operations such as: $\cup$ (union), $\cap$ (intersection), $\in$ (membership) 
apply to sets and the other compound data types. 

\item[Map] A set of data types whose elements are accessed by their 
corresponding label. For example, $p[Y] ,p[C1] ,p[C2] $ might be the values 
of the different video components of a pixel. The set of  labels corresponding to 
the elements of a map $m$ can be accessed by $\args(m)$, so that, for example, $args(p)$ returns$\{Y,C1,C2\}$.
\item[Array]  A collection of data types accessed by a non-negative integer index.
 This compound data type is typically used to represent an array of variables. Elements of a 1-dimensional array $a$ are accessed by $a[n]$ for $n$
 in the range 0 to $\length(a) - 1$. 
\end{description}

A compound data type may contain other compound data types. For example, 
a two dimensional array is an array of one dimensional arrays. Elements of a 2-dimensional array are accessed by $a[n][m]$ for $0\leq m\leq(\width(a)-1)$, 
and $0\leq n\leq (\height(a)-1)$. Compound data types may be more complex. 
For example, picture data, pic, may be considered to be a map of arrays, where $pic[Y]$ is a 2-dimensional array storing luma data, and $pic[C1]$ and $pic[C2]$
 are two-dimensional arrays storing chroma data.

Elements may be added to a map or array by assignment using the appropriate
 index (label or integer). For example, $a[7]=2$, adds element 7 to the array $a$,
 if a does not already contain element 7, then this element is assigned the value 2.

\subsection{Functions and operators}
\label{functionoperators}

This section defines the functions and operators used 
in the pseudocode in this specification. Functions and operators
are similar but functions use the syntax, $(arg1, arg2,\ldots)$ 
whereas operators are simply placed before or between operands, 
e.g. $a+b$. The difference is purely syntactic and is to 
correspond with conventional mathematical notation.

\subsubsection{Assignment}

The assignment operation  = applies to all variable types. After performing 
\[a=b\]
the value of $a$ shall become equal to that of $b$, and the value of $b$ shall remain unchanged. For a set (or map or array) this constitutes an element-wise copy
 i.e.
\[a[x]=b[x]\]
for all valid values of $x$.

\subsubsection{Boolean functions and operators}
\label{booleanops}
The following functions and operators are defined for one or more Boolean arguments:
\begin{description}
\item[not] 		(not a) or returns $\true$ for a boolean value $a$ if and only if 
$a$ is $\false$
\item[and] 		(a and b) returns $\true$ if and only if a and b are both $\true$. Operator "and" may be used in pseudo-code conditions to denote the logical AND between Boolean values, for example: if (condition1 and condition2): �etc.
\item[or] 		(a or b) returns True if either a or b are True, else it returns False.  Operator "or" may be used in pseudo-code conditions to denote the logical OR between Boolean values, for example: if (condition1 or condition2): � etc.
\item[majority]		Given a set, $S =\{s_0,�, s_{n-1}\}$ of Boolean values, $\majority(S)$ returns the majority condition. That is, if the number of $\true$ 
values is greater than or equal to the number of $\false$ values, $majority(S)$
 returns $\true$, otherwise it returns $\false$.
\end{description}

Boolean operations are to be distinguished from bitwise operations which operate on non-negative 
integer values, and are defined in Section \ref{integerops}.

\subsubsection{Integer functions and operators}
\label{integerops}
The following functions and operators are defined on integer values:

\begin{description}
\item[Absolute value] $|a|=\left\lbrace\begin{array}{l} a \text{ if $a\geq 0$}\\ 
                                                                                   -a \text{ otherwise} \end{array}\right.$.

\item[Sign] $\sign(a)$ is defined by
\[\sign(a)=
\left\{\begin{array}{l} 
1 \text{ if $a>0$} \\
0 \text{ if $a==0$}\\
-1 \text{ if $a<0$} 
\end{array}\right.\]

\item[Addition] The sum of $a$ and $b$ is represented by $a+b$.

\item[Subtraction] $a$ minus $b$ is represented by $a-b$.

\item[Multiplication] $a$ times $b$ is represented, for clarity, by $a*b$.

\item[Integer division] Integer division is defined for integer values $a,b$ with 
$b>0$ where: $n=a//b$ is defined to be the largest integer $n$ such that
\[n*b\leq a\] 

i.e. numbers are rounded towards -infinity. N.B. this differs from C/C++ conventions
of round towards 0.

\item[Remainder] For integers $a,b$, with $b>0$, the remainder $a\%b$ is 
equal to 

\[a\%b = a-(a//b)*b \]

 $a\%b$ always lies between 0 and $b-1$.

\item[Exponentiation] For integers $a, b$, $b>0$ $a^b$ is defined as $a*a*\hdots *a$ ($b$ times). $a^0$ is 1.

\item[Maximum] $\max(a,b)$ returns the largest of $a$ and $b$.

\item[Minimum] $\min(a,b)$ returns the smallest of values $a$ and $b$.

\item[Clip] $\clip(a,b,t)$ clips the value $a$ to the range defined by $b$ (bottom)
and $t$ (top):
\[\clip(a,b,t)=\min(\max(a,b),t)\]

\item[Shift down] For integers $a,b$, with $b\geq 0$, $a\gg b$ is defined as 
$a//2^b$.

\item[Shift up] For integers $a,b$, with $b\geq 0$, $a\ll b$ is defined as $a*2^b$.

\item[Integer logarithm] $m=\intlog2(n)$, for $n>0$, $m$ is the integer such that
$2^{m-1}<n\leq 2^m$.


\item[Mean] Given a set  $S=\{s_0, s_1, \hdots, s_{n-1}\}$ of integer values, the integer unbiased mean, $\mean(S)$, is defined
to be

\[(s_0+s_1+\hdots +s_{n-1}+(n//2))//n\]

\item[Median] Given a set $S=\{s_0, s_1, \hdots, s_{n-1} \}$ of integer values the median, $\median(S)$, 
returns the middle value. If $t_0\leq t_1\leq \hdots \leq t_{n-1}$ are the values $s_i$ placed in ascending order, this
is 

$t_{(n-1)/2}$ 

if $n$ is odd and

$\mean(\{ t_{(n-2)/2},t_{n/2}\})$ if $n$ is even. If $S=\emptyset$, $\median(S)$ returns 0.
\end{description}
The following bitwise operations are defined on non-negative integer values:
\begin{description}
\item[\&] Logical AND is applied between the corresponding bits in the binary representation of two numbers, e.g.
$13\&6$ is $\text{b1101}\&\text{b110}$, which equals $\text{b100}$, or 4.

\item[${\mathbf |}$] Logical OR is applied between the corresponding bits in the binary representation of two numbers, e.g.
$13|6$ is $\text{b1101}\text{|}\text{b110}$, which equals $\text{b1111}$, or 15.

\item[${\mathbf \wedge}$] Logical XOR is applied between the corresponding bits in the binary representation of two numbers, e.g.
$13\wedge 6$ is $\text{b1101}\wedge\text{b110}$, which equals $\text{b1011}$, or 11.

\item[$\mathbf{\&=}$]	 $a \&= b$ is equivalent to $a = (a \& b)$.
\item[$\mathbf{|=}$]	 $a |= b$ is equivalent to $a = (a | b)$.
\item[$\mathbf{\wedge=}$]	$a\wedge^= b$ is equivalent to $a = (a \wedge b)$.
\end{description}

Bitwise-not is not defined for integers to avoid ambiguity concerning leading 
zeroes

The following logical operators are defined for integer and boolean arguments:
\begin{description}
\item[==] Test of equality of two variables. $a == b$ is $\true$ if and only if the 
value of a equals the value of b.
\item[!=] Not equal to. $a != b$ is equivalent to not $(a == b)$
\end{description}

The following logical operators are defined for integer arguments only:
\begin{description}
\item[$\mathbf{<}$] 	Less than
\item[$\mathbf{<=}$]	Less than or equal to
\item[$\mathbf{>}$]	Greater than
\item[$\mathbf{>=}$]	Greater than or equal to

\end{description}

The following combined assignment operators are defined for integer 
arguments:
Operators $+, -, *, //, \%, \gg, \ll, \&, |, \wedge$, may be combined with the assignment operator (as for the Boolean operators $\&$, $|$, and $\wedge$
 above). For example $a += b$ is equivalent to $a = (a + b)$.

\subsubsection{Array and map functions and operators}

The following functions and operators are defined for sets, maps and arrays. 
\begin{description}
\item[Indexing]		For an array $a$, $a[index]$ returns an element of $a$. If $a$ is a map the index shall be a label, else if $a$ is an array the index shall be an integer.
\item[Scalar Assignment]	Where the notation $a = 0$ is used for an array of integer values, it means "set all elements of the array to zero".
\item[Insertion]		$a[index] = b$ inserts a copy of $b$ into set $a$ 
if the element does not already exist.
\item[Tokens]		for a map $a$, $\args(a)$ returns the set of the indexing tokens.
\item[Length]		for a one dimensional array $a$, $\length(a)$ 
returns the number of elements in the array.
\item[Width]		for a two dimensional array $a$, $\width(a)$ returns the 
width the array. The width is the number of scalar elements corresponding to the right most array index.
\item[Height]					for a two dimensional array $a$, $\height(a)$ returns 
the height the array. The height is the number of one dimensional arrays in the two dimensional array and the "height" dimension corresponds to the left most array index.
\end{description}
\subsubsection{Precedence and associativity of operators}
\label{operatorprecedence}
To avoid any confusion over the order of operator precedence, every equation makes extensive use of the expression operators "(" and ")". All operations recursively execute the innermost expression(s) first until the calculation has been completed. In cases where the expression operators do not make clear the order of precedence, the following table defines the descending order of operator precedence and the associativity of each operator.
[Table tbc]
\begin{comment}
Operator Precedence	Associativity
( ) [ ]	left to right
* // %	left to right
+ -	left to right
<< >>	left to right
< <= > >=	left to right
== !=	left to right
! (not)	right to left
& (and)	left to right
^ (xor)	left to right
|	left to right
= += -= *= //= %= &= ^= |= <<= >>=	right to left
\end{comment}

\subsection{Pseudocode}
\label{pseudocode}

Most of the normative specification is defined by means of pseudocode. 
The syntax is intended to be both precise and descriptive; the pseudocode is 
not intended to form the basis for the implementation of a Dirac decoder.

All processing defined by this standard is precise and the entire specification
can be implemented using only the data types, functions and operators 
defined herein. That is, no operations on "real" or "floating point" numbers
 are required. All operations shall be implemented with sufficiently large 
integers so that overflow cannot occur.

The type of variables in the pseudocode is not explicitly declared. 
A variable assumes a type when it is assigned a value, which shall always 
have a defined type.

\subsubsection{Processes and functions}
\label{functionsprocesses}

Decoding and parsing operations are specified by means of processes
 -- a series of operations acting on input data and global variable data. 
A process can also be a function, which means it returns a value, but
it need not do so. So a process
taking in variables $in1$ and $in2$ looks like:

\begin{pseudo}{foo}{in1, in2}
\bsCODE{op1(in1)}
\bsCODE{op2(in2)}
\bsCODE{\hdots}
\end{pseudo}

Whilst a function process looks like:

\begin{pseudo}{bar}{in1, in2}
\bsCODE{op1(in1)}
\bsCODE{foo(in1,in2)}{\ref{functionsprocesses}}
\bsCODE{\hdots}
\bsRET{out1}
\end{pseudo}

The right-hand column in the pseudocode representation contains a cross-reference to the 
section in the specification containing the definition of other processes used at that line.

\subsubsection{Variables}

All input variables are deemed to be passed {\em by reference} in this
specification. This means that any modification to a variable value that
occurs within a process also applies to that variable within the calling process
{\em even if it has a different name} in the calling process. One way to understand
this is to envisage variable names as pointers to workspace memory.

For example, if we define $foo$ and $bar$ by

\begin{pseudo}{foo}{}
\bsCODE{num=0}
\bsCODE{bar(num)}
\bsCODE{\StateName[var\_name]=num}
\end{pseudo}

and 

\begin{pseudo}{bar}{val}
\bsCODE{val=val+1}
\end{pseudo}

then at the end of $foo$, $\StateName[var\_name]$ has been set to 1.

The only global variables are the state variables encapsulated in $\StateName$.
 If a variable is not declared as an input to
the process and is not a state variable, then it is local to the function.

If a process is particularly complex, it may be broken into a number of steps with 
intermediate discussion. This is signalled by appending  and prepending ``$\hdots$" to
the parts of the pseudocode specification:

\begin{pseudo}{foo}{}
\bsCODE{code}
\bsCODE{\hdots}
\end{pseudo}

[text]

\begin{pseudo*}
\bsCODE{more code}
\bsCODE{\hdots}
\end{pseudo*}

[text]

\begin{pseudo*}
\bsCODE{even more code}
\end{pseudo*}

The intervening text may define or modify variables used in the succeeding
pseudocode, and must be considered as a normative part of the specification of the process.
This is done as it is sometimes much more clear to split up a long and complicated process
into a number of steps.

\subsubsection{Control flow}
\label{controlflow}

The pseudocode comprises a series of statements, linked by functions and
flow control statements such as {\bf if}, {\bf while}, and {\bf for}.

The statements do not have a termination character, unlike the ; in C
for example.  Blocks of statements are indicated by indentation:
indenting in begins a block, indenting out ends one.

Statements that expect a block (and hence a following indentation) end
in a colon.

\paragraph*{if}

The if control evaluates a boolean or boolean function, and if true, passes the 
flow to the block of following statement or block of statements. If the control
evaluates as false, then there is an option to include one or more else if
controls which offer alternative responses if some other condition is
true.  If none of the preceding controls evaluate to true, then there is
the option to include an else control which catches remaining cases.

\begin{pseudo*}
\bsIF{control1}
    \bsCODE{block1}
\bsELSEIF{control2}
    \bsCODE{block2}
\bsELSEIF{control3}
    \bsCODE{block3}
\bsELSE
    \bsCODE{block4}
\bsEND
\end{pseudo*}

The if and else if conditions are evaluated in the order in which they
are presented. In particular, if $control1$ or $control2$ is true in
the preceding example, $block3$ will not be executed
even if $control3$ is true; neither will $block4$.

\paragraph*{for}

The for control repeats a loop over an integer range of values. For example,

\begin{pseudo*}
\bsFOR{i=0}{n-1}
    \bsCODE{foo(i)}
\bsEND
\end{pseudo*}

calls $foo()$ with value $i$, as $i$ steps through from 0 to $n-1$ inclusive.


\paragraph*{for each} The for each control loops over the elements in
a list:

\begin{pseudo*}
\bsFOREACH{c}{Y,C1,C2}
    \bsCODE{block}
\bsEND
\end{pseudo*}

\paragraph*{for such that} The for such that control loops over elements in
a set which satisfy some condition:

\begin{pseudo*}
\bsFORSUCH{a\in A}{control}
    \bsCODE{block}
\bsEND
\end{pseudo*}

This may only be used when the order in which elements are processed is 
immaterial.

\paragraph*{while}

The while control repeats a loop so long as a switch variable is true. 
When it is false, the loop breaks to the next statement(s) outside the block.

\begin{pseudo*}
\bsWHILE{condition}
    \bsCODE{block1}
\bsEND
\bsCODE{block2}
\end{pseudo*}


%\clearpage
%\section{Semantics}%%%%%%%%%%%%%%%%%%%%%%%%%%%%%%%%%%%%%%%%%%%%%%%%%%
% - This chapter defines specification         - %
% - conventions                                - % 
%%%%%%%%%%%%%%%%%%%%%%%%%%%%%%%%%%%%%%%%%%%%%%%%%%

% Some sort of explanation along these lines needed but
% semantics not exactly the right word?

The parameters in use in Dirac are divided into three general categories

\begin{itemize}
	\item Largely static parameters and synchronisation parameters
	\item Parameters describing the prediction mechanism, including the motion vectors
	\item Parameters describing the wavelet analysis of the residue after prediction
\end{itemize}

\subsection{Static and synchronisation parameters}These parameters are used to enable the decoder to synchronise with the
incoming bytestream, and to indicate the parameters of the decoded
signal.

\textbf{Access Unit}

The Access Unit Header marks a point in the Bytestream from which
decoding may commence.

Sometimes sequences are simply played from start to finish. Often,
however, it is necessary to start playing a sequence part way through a
stream (for example if a viewer has just connected to a
broadcast/multicast transmission or following transmission errors). To
achieve this the player must be able to start decoding at some point in
the middle of a stream without requiring prior information.

Access Unit Header are points in a Dirac bit stream at which a player
can start decoding. An Access Unit Header provides the sequence and
decoding parameters with which to configure the decoder. The Access Unit
Header should be interpreted as giving an access point to the data in
display order.

An Access Unit Header does NOT imply that all subsequent frames in coded
order can be decoded. An Access Unit Header is followed immediately by
an Intra frame. However, typically, some Inter frames that are earlier
in display order, will be transmitted following an Intra frame. Such
Inter frames might be predicted from frames prior to the Access Unit
Header. Clearly such Inter frames cannot be decoded from only the
information following the Access Unit Header. However subsequent frames
can be decoded. Given a limited reordering depth in a prediction
structure, eventually all frames after a Access Unit Header will be
decodeable.

The parameters specified in the Access Unit Header remain the same
throughout a Video Sequence. That is, the parameters in later Access
Unit Headers simply repeat those in earlier headers. The repetitions are
included to provide entry points to start decoding the Bytestream.

See also Video Sequence

\textbf{Access Unit Parse Parameters}

The Access Unit Parse Parameters indicate the Picture Number which can
be decoded using the available information, and the Version Number,
Profile and Level of Dirac coding that has been used.

\textbf{Access Unit Picture Number}

The Access Unit Picture Number indicates the picture number of the first
picture that may be decoded following this Access Unit header.

\textbf{Aspect Ratios}

Aspect Ratios refers to the Pixel Aspect Ratios, not the image aspect
ratios. The Pixel Aspect Ratio value of an image is the ratio of the
intended spacing of horizontal samples (pixels) to the spacing of
vertical samples (picture lines) on the display device. Historically,
the aspect ratio was not square, and differed between 525-line and
625-line systems.

The shape of the elementary pixels is signalled as the quotient of two
variables, the numerator or dividend and denominator or divisor. The
numerator refers to the horizontal dimension and the denominator to the
vertical dimension.

Pixel Aspect Ratios are fundamental properties of sampled images because
they determine the displayed shape of objects in the image. Failure to
use the right parameters will result in distorted images, for example
circles will be displayed as ellipses etc. In spite of their fundamental
nature, pixel Aspect Ratios are not rigidly defined in many video
standards.

Some HDTV standards and computer image formats are defined to have Pixel
Aspect Ratios that are exactly 1:1.

Some video processing tools require an Image Aspect Ratio. This may be
derived from the pixel Aspect Ratio. The image aspect ratio is the ratio
of horizontal to vertical pixels multiplied by the pixel aspect ratio.
So, for example, for a 525 line 704 x 480 line picture the image aspect
ratio is $\frac{704 * 10}{480 * 11}$ which is exactly 4:3.

You are strongly advised to use one of the default Aspect Ratios.
However, if you know what you are doing and don't like the default
values you can, in principle, define your own. But our advice is don't
do it, just say no.

\textbf{Bytestream}

The Bytestream is the complete flow of transmitted Dirac data, after
coding.

\textbf{Chroma Excursion}

The dynamic range of the chrominance components.

\textbf{Chroma Format}

In sampled television signals, the signal is usually split into three
component: luminance and two colour components. The eye has around 20
times more receptors sensitive to the luminance of a scene than to the
chrominance. As a consequence, it is possible to provide less colour
detail than luminance detail and still satisfy the eye of the beholder.

The format for this subsampling is usually specified in the style 4:2:2
or 4:2:0 for example. This nomenclature may be written

luminance samples: chroma 1 samples : chroma 2 samples,

where chroma 1 and chroma 2 are the two colour components, usually $U$
and $V$ or $C_b$ and $C_r$. The figures refer to the number of samples
of each component along a small segment of the video line,
conventionally taking the number of luminance samples as 4. In the case
of 4:2:2, the signal would be $Y U Y V Y U Y V ...$., whilst in the case
of 4:2:0, it does not mean that there are no samples of $V$ at all; it
is just that on one line it is $YUYYUY ...$, and on the next it is
$YVYYVY ...$.

\textbf{Chroma Offset}

The zero level of the chrominance components.

\textbf{Clean Area}

Clean area is pure metadata, not used in any way to influence or control
the decoding process.

The Clean Area defines an area within which picture information is
subjectively uncontaminated by all edge effects. These may be transient
(and other) distortions. These may be coding artefacts, deliberate
introduction of elements such as time codes, etc.

Clean area is useful metadata for D-Cinema in which the 2K or 4K image
size is used as a container for a variety of picture formats with
different aspect ratios.  It is appropriate to display the Clean Area
rather than the whole picture.  The clean area is defined by the
position of the top left corner, and its horizontal and vertical
dimensions.

Clean area is desirable for ordinary SD video with a 720 pixel width
because only about 702 of the pixels represent active video, the
remainder allowing for overscan and filters falling off the edge of the
picture. The analogue PAL specification could be interpreted to say that
the clean width is 704 pixels. If you then use a pixel aspect ratio of
12:11 you end up with a picture aspect ratio of exactly 4:3. If you
could not specify the clean area then you would have big problems
specifying either the picture aspect ratio or the pixel aspect ratio. It
is simply wrong to say that the whole of the 720 pixels represent the
4:3 picture.

See also Clean Width, Clean Height, Left Offset, Top Offset.

\textbf{Clean Height}

The height of the Clean Area in pixels.

\textbf{Clean Width}

The width of the Clean Area in pixels.

\textbf{Colour Matrix}

We use the conventional relationships between the different
representations of the colour matrixes linking the different portrayals
of the colour channels. The $E_{Y}$, $E_{Cb}$, $E_{Cr}$ values are
derived from the $E_R$, $E_G$, $E_B$ values by the following equations.


The inverse transform, which would be used by a player, is the following.

Conversion between $E_R$, $E_G$, $E_B$ and $E_Y$, $E_{Cb}$, $E_{Cr}$ are
matrix operations. They can be specified by two of $K_R$, $K_G$, $K_B$.
The remaining $K$, say $K_Y$ can be calculated as $K_Y=1- K_R - K_B$.

\textbf{Colour Primaries}

The Colour Primaries define the assumed colours of the phosphors. The
choice offered is between three sets - the SMPTE C set, the EBU Tech
3213 set and the ITU Rec 709 set.

\textbf{Colour Space}

Dirac defines the subsets of colour primaries for the phosphors, the
colour matrixes and the transfer function used for the relationship
between signal level and light output.

We have deliberately eschewed provisions for a custom colour space.

Knowing the Source Parameters for the colour space, if the available
display has different parameters, it is possible to convert the received
signal to optimise the colour in the display.

All current video system use the following model for YUV coding of the
RGB values (computer systems often omit coding to and from YUV).

The R, G \& B are tristimulus values $(e.g. candelas/meter^{2})$. Their
relationship to CIE XYZ tristimulus values can be derived from the set
of primaries and white point defined in the colour primaries part of the
colour specification using the method described in SMPTE RP 177-1993. In
this document the RGB values are normalised to the range [0,1], so that
RGB=1,1,1 represents the peak white of the display device and RGB=0,0,0
represents black.

The $E_R$, $E_G$, $E_B$ values, also in the range [0,1], are related to
the RGB values by non-linear transfer functions $f()$ and $g()$. The
transfer function $f()$ is typically performed in the camera and is
specified in the Transfer Characteristic part of the Colour
Specification. For aesthetic and psychovisual reasons the transfer
function $g()$ is not quite the inverse of $f()$. In fact the combined
effect of $f()$ and $g()$ is such that

Where $r$ is the rendering intent or end to end gamma of the system,
which may vary between about 1.1 and 1.6 depending on viewing
conditions. The non-linear $E_R$, $E_G$, $E_B$ values are subject to a
matrix operation (known as non-constant luminance coding), which
transforms them into the $E_Y$, $E_{Cb}$, $E_{Cr}$ values. $E_Y$ is in
the range [0,1] and the $E_{Cb}$ and $E_{Cr}$ values are in the range
[-0.5, 0.5]. This is YUV coding and sometimes the U and V components are
subsampled, either horizontally or both horizontally and vertically. UV
sampling is specified by Colour Format.

Sometimes the matrix operation is omitted (or, equivalently, a unit
matrix is used). In this case $E_Y$, $E_{Cb}$, $E_{Cr}$ values are
simply the $E_Y$, $E_{Cb}$, $E_{Cr}$ values. This is signified by Colour
Format = RGB.

The $E_Y$, $E_{Cb}$, $E_{Cr}$ values are mapped to a range of integers
$Y$, $C_{b}$, $C_{r}$. Typically they are mapped to an 8 bit range [0,
255]. The way in which $E_Y$, $E_{Cb}$, $E_{Cr}$ values are mapped to
the integer values that are actually compressed is specified by Signal
Range.

\textbf{End of Sequence}

The End of Sequence code marks the end of a Video Sequence. It is not
necessarily the end of a complete work. The editing process may require
concatenation of Video Sequences. The decoder should therefore be
prepared to respond to data which follows the End of Sequence code.

\textbf{Field Dominance }

This is a flag which indicates that, for interlaced scanning, the top
field is first or second.

See also Interlace

\textbf{Frame Rate}

The Frame Rate is the rate at which frames of video are displayed.
Historically these have been close to the frequencies of the main
electricity supply: hence the global variations.

In Dirac, the frame rate is signalled as the quotient of two variables,
the numerator or dividend and denominator or divisor.

\textbf{Image Aspect Ratio}

The ratio of the horizontal to vertical dimensions of the Clean Area of
the image.

\textbf{Image Size}

The Image Size is a measure of the segmentation of the image. It is the
size of the image, measured in pixels of luminance. It is signalled as
the number of pixels along a line (the Luma Width) and the number of
lines in a frame (the Luma Height).

\textbf{Interlace}

If we have an interlaced source, each frame of the Video Sequence is
built up from two fields. These fields are spatially offset, with each
field delivering alternate lines of the final frame. For transmission,
the two fields can be combined, with field lines interleaved line by
line (pseudo-progressive format, the default for Dirac). Alteratively
they can be transmitted sequentially - interleaved field by field as in
conventional analogue broadcasting (giving low delay and low resource
coding).

The field interleaving flag indicates non-default field interleaving,
and the Sequential Fields (Boolean) parameter indicates whether the
fields are interleaved as pseudo-progressive or sequential fields.

With field sequential coding the picture sequence is a sequence of
fields rather than frames. So, for example, for interlaced 625 line
video we would have picture size 720 pixels by 288 lines, frame rate
25~Hz and Sequential Fields True.

Since Interlace is pure metadata there is no change to the coding
algorithm. It may seem unusual that the picture size refers to a field
rather than the whole frame.

However this preserves the separation of the Source Parameters, which
are pure metadata, from point of view of the decoding process.

When two fields are interlaced to make a frame, the top field (the odd
lines in the image) is usually transmitted first and the bottom field
(the even lines) comes second for images sourced in 625-line PAL. The
order is reversed for signals sourced from 525-line NTSC sources. The
difference arises because the image information starts on an odd line
number in PAL systems and an even line number in NTSC.

As an example, the default settings for the standard definition of SD
576 (the parameters which would be used as a basis for conventional PAL
in Europe) assume a pseudo-progressive or film mode, based on 25~Hz
progressive scan. If we wish to signal an interlaced signal, then it is
only necessary to modify the interlace flag. There needs to be a
sympathetic handling of the decoded signal when it is displayed. Whether
this is done by signalling or re-formatting is not a matter of
specification.

See also Scan Formats

\textbf{Left Offset}

The offset of the top left hand corner of the Clean Area from the left
hand side of the full image.

See also Clean Area

\textbf{Level}

The Level is an indication that the transmission is intended for a
decoder which may not necessarily decode the complete range of Video
Formats enabled by Dirac. Together with the Profile, it describes the
subset. A level is a set of decoder resource requirements that must be
satisfied in order to decode a bitstream, together with a set of
constraints on the bitstream that ensures that these requirements are
not exceeded.

This version of the specification does not define distinct levels and
profiles.  In future, we expect it to define the processing power of the
decoder - and hence the likely range of Video Formats which can be
handled by the device. One particular element we expect to be included
is the number of reference frames which can be stored.

Level is also used as a label for a parameter in the wavelet transform.
Hopefully there is no ambiguity caused by the use of the same name for
two different functions. The context should be clear.

See also Profile and Level in Section \ref{waveletparameters}

\textbf{Luma Excursion}

The dynamic range between black and white levels.

\textbf{Luma Offset}

The signal level corresponding to black level.

\textbf{Next Parse Offset }

Next Parse Offset is added to the Bytestream to simplify parsing. It
represents the offset in bytes from the start of the current Parse Info
to the start of the next Parse Info. So counting forward Next Parse
Offset bytes from the first byte (0x42 equivalent to B) of the current
Parse Info should yield a byte of value 0x42 or B corresponding to the
start of the next Parse Info. The Previous Parse Offset of the current
Parse Info equals the Next Parse Offset of the previous Parse Info.

\textbf{Parse Info}

Information which identifies the structure of the bytestream.

\textbf{Parse Info Prefix}

The Parse Info Prefix is the sequence of bytes 0x42  0x42  0x43  0x44,
which are the ASCII codes for BBCD.  This identifies the stream as a
Video Sequence coded using Dirac compression.

The Parse Info Prefix is present to allow an application to find a point
from which to start decoding. That is, the function of Parse Prefix
Header is to synchronise the decoder with the Bytestream. Decoding can
start from any Access Unit Header. The decoder first needs to find a
Parse Info structure. It should then check the Parse Code in the Parse
Info. If the following Parse Unit is an Access Unit Header then the
decoder can start decoding. If the Parse Unit is a Picture then the
decoder should skip forward by Next Parse Unit bytes (from the start of
the Parse Info Prefix) to the next Parse Info. The decoder would
continue skipping forward unit it locates an Access Unit Header. Note
that the decoder does not need to parse any Parse Units in order to
navigate through the stream to find an Access Unit Header. The Previous
Parse Offset is provided to allow searching backwards through the
Bytestream.

Any particular instance of the Parse Info Prefix in the Bytestream may
not, necessarily, indicate the start of a Parse Info structure. This is
because other parts of the Bytestream may, by chance, introduce these
bytes into the Bytestream. The use of arithmetic coding in Dirac means
that it is impossible to directly avoid accidentally introducing the
Parse Info Prefix.

When encoding a bytestream it is not necessary to avoid accidentally
introducing Parse Info Prefix sequences. They are present to allow
synchronisation of the bytes stream with the decoder and this can be
ensured, even in the presence of spurious Parse Info Prefixes, as
follows. When the decoder finds a Parse Info Prefix it should skip
forward by Next Parse Offset (or back by Previous Parse Offset) and
check whether the next three bytes are a Parse Info Prefix. If so the
decoder can be reasonably certain that it has found a genuine Parse Info
Prefix. If it does not find another Parse Info Prefix it was probably
unlucky enough to have found a spurious Parse Info Prefix. In this case
it should search for the next Prefix and repeat the test.

The probability of a spurious Parse Info Prefix is low; 1 in $2^{32}$
since the prefix is 4 bytes long. This is the probability of finding two
Parse Info Prefix sequences separated by Next Parse Offset. The test
outlined in the previous paragraph is, therefore, adequate in practice.
For the paranoid the test may be extended to find three Parse Info
Prefixes separated by the indicated Next Parse Offsets. This extended
test probably reduces the chance of failure to less than once in the
lifetime of the universe and should be sufficient for all but the
extremely cautious.

The test for two appropriately separated Parse Info Prefixes is, anyway,
prudent in any channel subject to bit errors even in the absence of
spurious Prefixes.

\textbf{Parse Unit}

The fundamental aggregation of data within the bytestream.

This definition of Dirac only includes two sorts of Parse Unit, Access
Unit Headers and Pictures. It is envisaged that other types of Parse
Unit may be introduced in future to carry data such as user data or
extension data.

\textbf{Picture Number}

Each picture has a unique Picture Number.

The Picture Number is a unique label (within the stream) indicating the
presentation/display ordering of the pictures. In a valid sequence the
Picture Numbers increment by one between consecutive frames (modulo
$2^{32}$, so 0x000 follows 0xFFFF).

If the Picture Number is even, and the picture is a field, then that
field has even parity. It is the first field of a pair of fields in a
frame. An interleaved Video Sequence coded sequentially (as opposed to
pseudo-progressively) starts with an even Picture Number. It does not
have to start with the Picture Number 0x000.

\textbf{Pixel Aspect Ratio}

See Aspect Ratio

\textbf{Previous Parse Offset}

The Previous Parse Offset is added to the Bytestream to simplify
parsing. It is the number of bytes backwards to the start of the
previous parse unit. The Previous Parse Offset of the current Parse Info
equals the Next Parse Offset of the previous

\textbf{Parse Info.}

See Next Parse Offset

\textbf{Profile}

The Profile is an indication that the transmission is intended for a
decoder which may not necessarily decode the complete range of Video
Formats enabled by Dirac. A profile is a set of decoding tools necessary
to decode a bit stream. A level is a set of decoder resource
requirements that must be satisfied in order to decode a bitstream,
together with a set of constraints on the bitstream that ensures that
these requirements are not exceeded.

This version of the specification does not define levels and profiles.

In future, we expect it to define the subset of tools that will be
accessible to the relevant decoder.

See also Level

\textbf{Scan Format}

In television, the picture is usually created by scanning the image as a
series of horizontal lines (some early formats used vertical lines, and
old electron beam devices had lines which were nearly, but not quite,
horizontal. We will ignore these in the Dirac specification).

A format in which all the lines in the frame are scanned in sequence is
called progressive format.

A format in which every other line in the frame is scanned, and then the
other half is called interleaved format. Each group of half lines is
called a field.  In old analogue broadcasting the two fields would be
broadcast one after the other: this is sequential field transmission.

It is also possible to combine the two fields to make a frame. This is
called pseudo-progressive format.

See also Interlace, Frame Rate

\textbf{Sequence Parameters}

A description of the parameters of the picture which are necessary to
decode the image. These include the Luma Width, Luma Height, Chroma
Format and Video Depth.  These parameters are essential to decoding and
displaying the bitstream.  The Sequence Parameters are intended to
change rarely if ever. If a change is necessary, it is recommended that
the bitstream is terminated by a Stop Sequence Parse Code and a new
bitstream is initiated.

See also Source Parameters

\textbf{Signal Range}

The Signal Range defines how the signal is scaled and clipped prior to
matrixing and display within the bits available, and is merely metadata
describing the source. It provides information to allow a bi-polar
signal such as $U$ and $V$ to be restored for display. Signal Range
embraces the set of parameters Luma Offset, Luma Excursion, Chroma
Offset and Chroma Excursion.

The offset and excursion values should be used to convert the
integer-valued decoded luma and chroma data $Y$, $C_{b}$, $C_{r}$ to
intermediate values $E_Y$, $E_{Cr}$, and $E_{Cb}$ by the recipe

$E_Y$, is normally clipped to the range [0,1], and $E_{Cr}$, and
$E_{Cb}$ to the range [-0.5,0.5].

This effectively clips

$Y$ to [LUMA\_OFFSET, LUMA\_OFFSET+LUMA\_EXCURSION]

and

$C_{b}$, $C_{r}$ to [CHROMA\_OFFSET-LUMA\_EXCURSION/2,
LUMA\_OFFSET+LUMA\_EXCURSION/2]

However, maintaining an extended RGB gamut may mean that either such
clipping is not done, or non-standard offset and excursion values are
used to extract the extended gamut from the non-negative decoded $Y$,
$C_{r}$, and $C_{b}$ values.

See also Luma Offset, Luma Excursion, Chroma Offset and Chroma Excursion.

\textbf{Source Parameters}

Source Parameters are a description of the parameters of the picture
which are not necessary to decode the image, but which may be desirable
to ensure accurate display of the decoded images. These include elements
such as Frame Rate, interlace information, Pixel Aspect Ratios,
information about the clean area, luma and chroma parameters and the
colour system being used.

The interpretation of Source Parameters by a display mechanism
interfacing with a compliant decoder is not specified. However, it would
make jolly good sense to follow the recommendations and interpretations
signalled if at all possible.

Likewise, encoders should ensure that accurate Source Parameter
information is encoded to maximise the potential quality of displayed
video.

See also Sequence Parameters

\textbf{Stream}

A Stream is a concatenation of Video Sequences.

See also Video Sequence

\textbf{Top Offset}

The Top Offset is the offset of the top left hand corner of the Clean
Area from the top of the full image.

See also Clean Area

\textbf{Transfer Function}

The Transfer Function defines the non-linear processing used in the
camera when converting the received light flux into an electrical
signal.

\textbf{Version Number}

The Version Number is coded as an unsigned integer with the first minor
version starting at zero.

Dirac is expected to be released in different versions at different
times. Version numbering will have two elements: the major version
number and the minor version number. Later versions will have the higher
numbers. Small changes in specification will be indicated by increases
in the minor version number.

The major version number defines the version of the syntax with which
the bit steam complies. Decoders that comply with a version of the spec
must be able to parse all previous versions too. Decoders that comply
with a version of the spec may not be able to parse the bit stream
corresponding to a later spec.

The first major version starts at one. Major version zero is a draft.
All minor versions of a spec should be functionally compatible with
earlier minor versions with the same major version number. Later minor
versions may contain corrections, clarifications, disambiguations etc;
they must not contain new features.

The first minor version starts at zero.

\textbf{Video Depth}

Video Depth is the number of bits used to represent the video data, i.e.
the video word width (typically 8  or 10 bits).

Video Depth is different from the Signal Range. The former defines how
many bits are used to contain the signal, and is used in the decoding
process. The input data, be it $Y$, $U$, $V$, or $R$, $G$ and $B$ are
all coded as if they were unsigned integers.

Note, this is separate from what the bits represent. It would be
possible, for example, to have an 8 bit signal represented in a 10 bit
word (in which case either the upper two, or lower two, bits of the word
would always be zero). The meaning of the bits is defined in the Signal
Range (the Luma and Chroma Offsets and Excursions) part of the Source
Parameters. Video Depth relates to how the video is coded. The Signal
Range relates to what the numbers mean and how the video should be
displayed.

\textbf{Video Format}

The Video Format indicates whether the signal conforms to something
close to one of the conventional video formats (such as High Definition,
PAL, NTSC, QCIF etc) or whether it is a custom format, with all
parameters available for setting independently. The video format
embraces a range of Source Parameters and Sequence

\textbf{Parameters.}

See also Source Parameters, Sequence Parameters and Appendix XXX [the
default format settings]

\textbf{Video Sequence}

A Video Sequence is a collection of images which can be of any length,
which have constant Source Parameters (e.g. picture size, aspect ratio
etc.) If the parameters need to change the only way to do it is to
signal the end of a Video Sequence and start a new Video Sequence.

The process of editing two coded sequences together might introduce
presentation order picture numbers which are not contiguous. This is
accommodated by introducing an End of Sequence Parse Code before a cut
so that the decoder would restart after a cut.

\subsection{Motion vector parameters}
Dirac uses motion-adaptive prediction to reduce the bit rate of a
sequence. The prediction parameters define the prediction tools up to
the point that the error signal or residual is calculated.

\textbf{Block Parameters Index}

Block Parameters Index defines the default settings for the sizes and
separation of the blocks used in the prediction of pictures. The
parameters defined are the Chroma Block Width, Chroma Block Height,
Horizontal Chroma Block Separation, Vertical Chroma Block Separation,
Luma Block Width, Luma Block Height, Horizontal Luma Block Separation,
Vertical Luma Block Separation.

\textbf{Common Mode}

Common Mode indicates that the same values of Motion Vector and
Prediction Mode are being used within a Superblock.

\textbf{Chroma Block Width}

The width of a chrominance prediction block in pixels.

\textbf{Chroma Block Height}

The height of a chrominance prediction block in pixels.

\textbf{Chroma DC Residual}

In Intra Frames, this is the residual of the predicted mean level of
each Block.

See also DC Value

\textbf{DC Value}

When coding an Intra Reference Picture or an Intra Non-reference
Picture, the picture is not predicted, but simply wavelet transformed.
Whereas the mean value of predicted images is usually zero, the mean
value of intra picture is not. To avoid the block artefacts that this
would produce, the mean level of the Blocks is removed from the signal
before the wavelet transform process. This mean level is used as the
prediction.

Instead of transmitting the absolute value of the mean level for each
Block, the DC Value is predicted by reference to other Blocks in the
Superblock, and the difference signalled. The same process is applied to
luminance and the two chrominance signals.

See also Luma DC Residual, Chroma DC Residual.

\textbf{Global Motion}

The Global Motion Flag indicates the presence of global motion data. If
TRUE, the Global Motion Only Flag indicates whether only global motion
data is present.

Global Motion is a technique which works well when the whole, or at
least a large proportion, of the image is being subjected to the same
transformation. Dirac uses an eight-parameter model that allows for pan,
zoom, rotation, shear and change of perspective.

\textbf{Horizontal Chroma Block Separation}

The separation of chrominance prediction blocks horizontally.

\textbf{Horizontal Luma Block Separation}

The separation of luminance prediction blocks horizontally.

\textbf{Horizontal Perspective}

Metadata which describes the Motion Vector element caused by changes in
perspective.

See also Appendix XXX [Global Motion Compensation]

\textbf{Horizontal Offset Residual}

When the Motion Vector has been calculated in the encoder, it comprises
an indication of the  horizontal and a vertical offsets of the reference
pixel from the pixel to be calculated. This is transmitted as a
difference from a value that is predicted using information from
surrounding Blocks - i.e. as a residual (or difference) from the
predicted value.

\textbf{Horizontal Perspective}

XX I am not sure what physical features this conveys.

\textbf{Inter Reference Picture}

A picture which is coded by prediction with reference to other pictures,
and which itself is used as a reference for prediction when coding other
pictures.

There are two different types of Inter Reference Pictures. One uses only
one other picture as a reference. The other type can use two other
pictures as references.

\textbf{Inter Non Reference Picture}

A picture which is coded by prediction with reference to other pictures,
but which itself is not used as a reference for prediction when coding
other pictures.

There are two different types of Inter Non Reference Pictures. One uses
only one other picture as a reference. The other type can use two other
pictures as references.

\textbf{Intra Non Reference Picture}

A picture which is coded without prediction with reference to other
pictures, and which itself is not used as a reference for prediction
when coding other pictures.

\textbf{Intra Reference Picture}

A picture which is coded without prediction with reference to other
pictures, but which itself is used as a reference for prediction when
coding other pictures.

\textbf{Luma Block Width}

The width of a luminance prediction block in pixels.

\textbf{Luma Block Height}

The height of a luminance prediction block in pixels.

\textbf{Luma DC Residual}

In Intra Frames, this is the residual of the predicted mean level of
each Block.

See also DC Value

\textbf{Motion Vector}

A Motion Vector is an indication of which pixels in the reference frame
can be used as predictors for a particular pixel in the predicted frame.
It is the offset of the predicted pixel from the reference pixel (and is
conveyed, either as a global field, or on a block by block basis).

The observant will notice that this vector is a spatial vector, being a
spatial offset. Only by weighting by the temporal separation between the
reference and predicted fields, with due note of the sign, can a true
indication of motion be deduced. We would have called it prediction
vector, but there is a weight of existing custom and practice against us
- so Motion Vector it remains.

Two types of Motion Vector information are used: global and block motion
vectors. Global motion is intended to describe the motion of the
background, using a parametric model. The block motion vectors are
intended to describe the more varied motion of the foreground. The two
type of motion information are used together to define the overall
motion vectors.



XX Could introduce elements here about Motion Vector Prediction process
from TD 0.9

\textbf{Motion Vector Precision}

Motion Vector Precision is what it says on the tin. It is an important
factor in achieving efficient video compression. If the Motion Vector
Precision is too low the motion-compensated prediction residuals will be
larger than necessary and require more bits to be coded. However if the
Motion Vector Precision is too high the motion vectors themselves will
require a disproportionately large number of bits to be coded.

Empirically, motion vector precisions of � or � of a pixel have been
found to work well. However the optimum motion vector precision depends
on many factors, such as the nature of the sequence to be coded and the
desired compression quality. Dirac defaults to � pixel motion vector
precision but provides the flexibility to use a non-default value of
precision.

XX In the range ?????

\textbf{Pan Tilt}

A measure of the degree of horizontal movement of the whole image (Pan)
and vertical movement of the whole image (Tilt). This is used as part of
the information provided for Global Motion Compensation.

See also Appendix XXX [Global Motion Compensation]

\textbf{Perspective}

A measure of the change in perspective of the image. This is used as
part of the information provided for Global Motion Compensation.

See also Appendix XXX [Global Motion Compensation]

\textbf{Perspective Exponent}

A multiplier which allows us to use integers in place of floating point
in the delivery of the Perspective metadata.

\textbf{Picture Prediction Parameters}

Dirac predicts Inter frames from one or two reference frames using
motion compensation. Frame prediction parameters and data are not
included for Intra frames, which are indicated by the Start Code in the
Parse Information.

Dirac's motion model is overlapping block motion compensation. The block
sizes can be adapted to match the requirements of the Sequence. The
spatial displacement of each block, from the corresponding position in a
reference frame, is coded in the bitstream. These displacements are
known as motion vectors. Motion vectors are the displacement that should
be applied to the reference frame to predict the current frame. The name
motion vector is, therefore, a misnomer because they are actually
prediction displacement vectors. Nevertheless the term motion vector is
used for consistency with industry practice

Two types of motion vector information are used: global and block motion
vectors. Global motion is intended to describe the motion of the
background, using a parametric model. The block motion vectors are
intended to describe the more varied motion of the foreground. The two
types of motion information are used together to define the overall
motion vectors.

Global Motion parameters are included, or not, depending on the value of
the Prediction Mode flags. One or two sets of Global Motion parameters
are included in the Frame Prediction, one for each reference frame.

When predicting an image from two different pictures, we are able to
weight the contributions. This aids, amongst other predictions, the
ability of Dirac to handle cross-fades between images and fades to black
or white.

\textbf{Prediction Mode}

Prediction Mode is a pair of bits, each of which indicates whether one
of the references is used to form a motion compensated prediction of the
picture. If a Prediction Mode bit is asserted (True) then the
corresponding reference picture is used in the prediction. An intra
prediction block is indicated when both bits are not asserted (False).

\textbf{Prediction Unit}

The Prediction Unit is the process which identifies whether Blocks
within a Superblock can use Common Modes - using the same Motion Vectors
and Prediction Modes - or whether different Motion Vectors and
Prediction Modes are appropriate.

\textbf{Picture Weights}

When using more than one reference image for prediction, the default is
to weight the two equally relevant. At times of cuts or fades in the
sequence, it may be that the default setting is invalid, so there is an
option to set the weighting pragmatically.

\textbf{Reference Picture Number}

Dirac predicts Inter frames from one or two reference frames using
motion compensation. The Reference Picture Numbers are the offset(s) of
the one or two Reference Pictures from the Picture Number of the picture
being decoded.

\textbf{Retired Picture List}

In Dirac, reference frames need to be retained for a while to act as
references for succeeding frames. Non-reference frames are always
discarded when the current output frame number exceeds their frame
number.

The Dirac decoder contains a small buffer of previously decoded frames.

An important aspect of the way the decoder works is how it manages which
frames to retain in its frame buffer and which frames it discards.

The encoder may explicitly specify frames that are no longer required in
the buffer.  To support this method of frame management each frame may
contain a list of frames that should be discarded from the buffer.

If no frames are specified to be retired then Dirac discards the oldest
frame first, but only as necessary. That is, when space is needed in the
frame buffer the frame with the lowest frame number is discarded. It is
up to the encoder to ensure that this default process does not discard
frames that are needed by the decoder. This default discard procedure is
only invoked when the frame buffer size would otherwise exceed that
specified by the decoder level and no signalling is present to indicate
which frames should be discarded.

It is the responsibility of the encoder to ensure that the correct
frames are retained in the buffer. The decoder may assume that the
reference frames it requires will be available in the buffer.

\textbf{Split Mode}

Split Mode indicates that different values of Motion Vector and
Prediction Mode are being used within a Superblock.

\textbf{Superblock}

In many instances, the Motion Vectors of adjoining blocks are similar.
Dirac therefore aggregates Blocks into Superblocks to enable Prediction
Modes and Motion Vectors to be transmitted more efficiently.

The metadata describing the Prediction Modes and Motion Vectors can be
the same (Common Mode) or different (Split Mode).

Superblocks are four Blocks wide and four Blocks high. They are arranged
such that the overlapped part of the Blocks at the edge of the image
falls outside the image. There is always an integer number of
Superblocks, so extra Blocks (with no source content) have to be added
to ensure complete population of the space.

See also Common Mode, Split Mode, Prediction Mode Residual 1, Prediction
Mode Residual 2, DC Value, Motion Data, Luma DC Residual, Chroma DC
Residual

\textbf{Vertical Chroma Block Separation}

The separation of chrominance prediction blocks vertically.

\textbf{Vertical Luma Block Separation}

The separation of luminance prediction blocks vertically.

\textbf{Horizontal Offset Residual}

When the Motion Vector has been calculated in the encoder, it comprises
an indication of the  horizontal and a vertical offsets of the reference
pixel from the pixel to be calculated. This is transmitted as a
difference from a value that is predicted using information from
surrounding Blocks - i.e. as a residual (or difference) from the
predicted value.

\textbf{Vertical Perspective}

Metadata which describes the Motion Vector element caused by changes in
perspective.

See also Appendix XXX [Global Motion Compensation]

\textbf{Zoom Rotate Sheer}

A measure of the amount of zoom, rotation and sheer in the image. This
is used as part of the information provided for Global Motion
Compensation. The parameters in the bytestream are presented as a
combination of these parameters in a matrix representation.

See also Appendix XXX [Global Motion Compensation]

\textbf{Zoom Rotation and Sheer Exponent}

A multiplier which allows us to use integers in place of floating point
in the delivery of the Zoom Rotate Sheer metadata.

\subsection{Wavelet parameters}
\textbf{Coefficient Context}

The wavelet coefficients are decoded using a probability model derived
from previously decoded coefficients. The probability model determines
the arithmetic decoding contexts that are used. So, before the quantised
coefficient can be read, it is first necessary to select the right
context to use, based on previously decoded coefficients. This is done
by using the Coefficient Context

After the quantised coefficient has been read the contexts, must be
updated to reflect the new probabilities.

\textbf{Codeblock Mode}

A flag which indicates the functionality of the quantisers.

XXX Needs better description.

\textbf{Chroma Height}

The height of the chrominance blocks input to the wavelet transform.

\textbf{Chroma Transform Data}

The Chroma Transform Data is the data after applying the wavelet
transform to the residual.

XXX is it before or after arithmetic coding?

\textbf{Chroma Width}

The width of the chrominance blocks input to the wavelet transform.

\textbf{Horizontal Codeblocks}

The number of Codeblocks across the width of the image. This allows us
to calculate the width of the Codeblocks in pixels.

\textbf{Luma Height}

The height of the luminance blocks input to the wavelet transform.

\textbf{Luma Transform Data}

The Luma Transform Data is the data after applying the wavelet transform
to the residual.

XXX is it before or after arithmetic coding?

\textbf{Luma Width}

The width of the luminance blocks input to the wavelet transform.

\textbf{Quantiser Codeblock}

XX Is this the definition of the codeblocks subject to the spatial
partition??????

\textbf{Spatial Partition}

The Spatial Partition enables us to chose whether to use separate
quantisers for each code block in a subband or a single quantiser to be
used for the whole subband.

These different spatial partition modes can be used to support
region-of-interest coding.



\textbf{Subband}

The subband names, LL, LH, HL, HH, correspond to the frequencies in the
subband.

Wavelet analysis results in a filtered and subsampled version of the
picture. When the picture is subject to wavelet analysis in two
directions this results in the four so-called subbands termed Low-Low
(LL), Low-High (LH), High-Low(HL) and High-High (HH). The first
descriptor refers to horizontal frequencies, the second to vertical
frequencies. L represents low frequencies and H represents high
frequencies. The order of these letters (horizontal, vertical) is
consistent with names used in the literature.

However this is the opposite order from with the array indices used in
this document.  XX Do we need to say the so what factor.

When several levels of wavelet are used, the LL band (only) is
iteratively decomposed. As a consequence, there are Low-High (LH),
High-Low(HL) and High-High subbands for each iteration, plus an extra
subband for the outstanding Low-Low information. This latter is
effectively a low frequency, sub-sampled version of the original.

In the processing, Subband is the metadata describing the transformed
data, which component it is, i.e. which level and which band.

\textbf{Transform Data}

Transform Data is the data after applying the wavelet transform to the
residual.

XXX is it before or after arithmetic coding?

\textbf{Transform Depth}

Same as Wavelet Depth: the total number of wavelet levels.

XX Do not understand why there are two names for it.

\textbf{Transform Parameters}

The transform parameters are the framework used for delivering the
Transform Data. The Transform Parameters describe the Wavelet Filter,
the Wavelet Depth and the Spatial Partition.

See also Wavelet Filter, Wavelet Depth and Spatial Partition

\textbf{Vertical Codeblocks}

The number of codeblocks across the height of the image. This allows us
to calculate the height of the Codeblocks in pixels.

\textbf{Wavelet Coefficients}

The Wavelet Coefficients are decoded using a probability model derived
from previously decoded coefficients. The probability model determines
the contexts that are use. Different sets of contexts are used for the
XXX"follow" bits, but a single context is used for all the XXX"data".

The magnitude and the sign of the coefficient are modelled separately.

In coding the magnitude, different sets of contexts are used for the
"follow" bits, but a single context is used for all the "data". The
probability distribution of coefficient magnitude is modelled as
depending on whether the parent coefficient was zero or non zero. The
probability distribution is also assumed to depend on the sum of
(previously decoded) neighbouring coefficients. Because the probability
distribution is usually peaked around zero it is only the context that
models the probability of zero/non-zero coefficient that changes with
the neighbourhood sum.

The probability distribution of the sign of the coefficient is assumed
to depend on an appropriate neighbouring coefficient.

\textbf{Wavelet Depth }

Wavelet Depth is the total number of wavelet levels. Usually this is
four, i.e. we do 4 splits both horizontally and vertically.

\textbf{Wavelet Filter}

XXX [filter pairs??] The definition of the filter used in the wavelet
analysis (and this, as a consequence, defines the filter used in the
synthesis process too).

\textbf{Wavelet index}

The Wavelet index is a parameter which allows us to identify which of
the wavelet filters we are using for the lifting filters.

\textbf{Zero Residual}

XXX Zero Residual is a flag which signals that there is no Residual to
decode.




\subsection{Purpose}
\label{intropurpose}
Dirac was developed to address the growing complexity and cost of current video
compression technologies, which provide greater compression efficiency
at the expense of implementing a very large number of tools. Dirac is
a powerful and flexible compression system, yet uses only a small number
of core tools. A key element of its flexibility is its use of the wavelet
multi-resolution transform for compressing pictures and motion-compensated 
residuals, which allows Dirac to be used across a very wide range of resolutions
without enlarging the toolset.

Dirac is an Open Source software project, and reference implementations
of the decoder and encoder are available at \underline{http://sourceforge.net/projects/dirac}.
A high-performance implementation, called Schrodinger, is also available
open source at \underline{http://schrodinger.sourceforge.net}.

\subsection{Scope}
\label{introscope}

This document specifies normative decoder operations and 
stream syntax. The stream syntax is primarily specified by means of
pseudocode, the conventions of which are described in Section \ref{pseudocode}.
The decoder operations are specified by means of a mixture of pseudocode
and conventional mathematical symbolism.

A number of other elements are also included for informative purposes.
The specification is not an implementation guide, and in the interests
of clarity many of the operations are specified in a way that would not be 
efficient to implement. However, we have attempted
to indicate where this is so, and to suggest ways in which an efficient implementation
may be achieved, but these are by no means exhaustive. An optimised Open Source
software Dirac encoder and decoder system, named Schr\"odinger, is available at
\underline{http://sourceforge.net/projects/schrodinger}, and may be studied to aid implementation.

In addition, we are well aware that many users of this document may wish
to make both encoders and decoders. There are many sources of information
on how to design efficient compression algorithms, for example for entropy coding,
motion estimation, frame-dropping, rate control, motion estimation and 
rate-distortion optimisation. This document does not attempt to address these
issues in detail, but to provide supplementary information where appropriate
to allow those reasonably `skilled in the art' to develop a Dirac encoder
rapidly and accurately, and approach design compromises knowledgably.

\subsection{Status}
\label{introstatus}

This is version $\SpecVersion$ of the Dirac specification. The document includes
a full description of the core Dirac stream syntax and decoder operations, together
with compatible extensions to support low-delay operation. It does
not yet contain a full specification of profiles and levels supported by Dirac.

\subsection{Document structure}
\label{introdocstruct}

This document specifies the Dirac decoder and stream structure in terms of
a layered model:

\begin{enumerate}
    \item Stream data access
    \item Parsing and interpretation of the Dirac stream metadata
    \item Unpacking of Dirac data -- coefficients and motion data
    \item Picture decoding operations
\end{enumerate}

Stream data access consists of the operations used to extract data values
(of boolean and integer type) from a raw Dirac bitstream. These include
data that has been encoded `literally' (i.e. according to conventional bit-wise
representations), variable-length codes, and data entropy coded using arithmetic
encoding. Stream data access methods are used both by parsing and unpacking operations.

Parsing and interpretation defines the structure of Dirac streams, including
random access points and navigation. 

Unpacking comprises parsing transform and motion data, and performing
inverse quantisation to reconstruct wavelet coefficients, and defines
intermediate decoder data structures in which extracted data is stored. This
step may be seen either as a stream parsing operation or as a decoding
operation, but it is separated here for clarity, and also because it
deals with intermediate data structures (transform coefficients and motion vectors), 
neither directly present in the bitstream nor output by the decoder.

Picture decoding operations produce decoded pictures from these populated
data structures by applying specified functions to them. These operations
are logically distinct from those for navigating the stream, reading the stream data
or reconstructing coefficients.

Note in particular that the distinction between parsing and picture decoding is
{\em not} exactly that between syntax and semantics: complex semantics are
required for correct parsing of the stream as well as for decoding pictures. 

It is perhaps unusual in a specification to separate these layers quite so distinctly, 
and our purpose in doing so is to provide much greater clarity. For implementors,
we hope that the decoupling of the stream structure from the (computationally intensive)
picture decoding processes will avoid imposing
implicit design decisions merely through the style of the specification. Many
other users of the specification will not be interested in the precise format
of stream elements but in how the underlying algorithm works - or vice-versa.
It should be possible to construct a Dirac parsing engine, for example for
frame skipping in video playback applications, extremely simply and without
requiring comprehension of the entire specification.

\begin{comment}
This layered structure is reflected in the structure of the specification,
which, after defining conventions used, is divided into three
corresponding parts: stream data access, defining functions for data types; 
accessing and parsing the Dirac bitstream and populating data
structures (including the wavelet coefficients and motion data); and 
high-level decoder operations and picture output, specifically the
inverse wavelet transform and motion compensation.

In addition to these parts, appendices deal with standard settings, parameter
presets and levels and profiles.
\end{comment}