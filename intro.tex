Dirac is an open video codec developed by the BBC. It has been 
developed to address the growing complexity and cost of current video
compression technologies, which provide greater compression efficiency
at the expense of implementing a very large number of tools. Dirac is
a powerful and flexible compression system, yet uses only a small number
of core tools. A key element of its flexibility is its use of the wavelet
multi-resolution transform for compressing pictures and motion-compensated 
residuals, which allows Dirac to be used across a very wide range of resolutions
without enlarging the toolset.

Dirac is an Open Source software project, and reference implementations
of the decoder and encoder are available at \url{http://sourceforge.net/projects/dirac}.
A high-performance implementation, called Schrodinger, is also available
open source at \url{http://schrodinger.sourceforge.net}.

Dirac offers the following features:
\begin{description}
\item[Multi-resolution transforms] Data is encoded using the wavelet transform, 
and packed into the bitstream subband by subband. High compression ratios result 
in a gradual loss of resolution. Lower resolution output can be obtained 
for low complexity decoding by extracting only the lower resolution data.
\item[Inter and intra frame coding] Pictures can be encoded using motion compensation
for low bit rate, or without reference to other pictures for editing, archive
and other professional applications.
\item[Frame and field coding] Both frames and fields can be coded.
\item[Dual syntax] A low delay syntax is available for applications requiring 
very low, fixed, latency. This can be of the order of a few lines of input 
or output video. The low delay syntax is suitable for light compression 
for the re-use of low bandwidth infrastructure, for example carrying 
HDTV over SD-SDI links. The low delay syntax uses 
intra coding and simple Variable Length 
Codes for entropy codes. The core syntax provides much greater compression efficiency 
at the expense of a whole picture delay. The core syntax can use a highly efficient 
form of binary adaptive arithmetic coding, as well as motion compensation, 
for maximum performance.
\item[CBR and VBR operation] Dirac supports both constant bit rate and 
variable bit rate operation.When the low delay syntax is used, the bit 
rate will be constant for each area (Dirac slice) in a picture to 
ensure constant latency.
\item[Variable bit depths] 8, 10, 12 and 16 bit formats are supported.
\item[Multiple chroma sampling formats] 444, 422 and 420 video are all supported.
\item[Lossless and RGB coding] A common toolset is used for both lossy and lossless 
coding. RGB coding is supported via the YCoCg integer color transform.
\item[Choice of wavelet filters] A wide range of wavelet filters can be used 
to trade off performance against complexity. The Daubechies (9,7) filter is 
supported for compatibility with JPEG2000. A Fidelity filter is provided for 
improved resolution scalability.
\item[Simple stream navigation] The encoded stream contains picture numbers and 
forms a doubly-linked list with each picture header indicating an offset to 
the previous and next picture, to support field-accurate high-speed navigation
with no parsing or decoding required.
\end{description}
\section{Scope}
\label{introscope}

This specification defines the Dirac video compression system through the stream syntax,
 entropy coding, coefficient unpacking and picture decoding processes. The decoder
 operations are defined by means of a mixture of pseudocode and mathematical 
operations.
 
This is version $\SpecVersion$ of the Dirac specification. The document includes
a full description of the Dirac stream syntax and decoder operations.

Dirac is a long-GOP video codec that uses wavelet transforms and motion compensation
 together with entropy coding, that can be readily implemented in hardware or software.
Dirac is a superset of the proposed SMPTE VC-2 video codec standard which 
comprises the intra coding parts of this specification.

This version is compatible with and extends Version 1 by the addition of motion
 compensated coding. Version 1 corresponds exactly to the proposed SMPTE VC-2
 video codec standard.

Subsequent versions of this specification may contain additional tools.

