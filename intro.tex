\subsection{Purpose}
\label{intropurpose}
Dirac is an open video codec developed by the BBC. It has been 
developed to address the growing complexity and cost of current video
compression technologies, which provide greater compression efficiency
at the expense of implementing a very large number of tools. Dirac is
a powerful and flexible compression system, yet uses only a small number
of core tools. A key element of its flexibility is its use of the wavelet
multi-resolution transform for compressing pictures and motion-compensated 
residuals, which allows Dirac to be used across a very wide range of resolutions
without enlarging the toolset.

Dirac is an Open Source software project, and reference implementations
of the decoder and encoder are available at \underline{http://sourceforge.net/projects/dirac}.
A high-performance implementation, called Schrodinger, is also available
open source at \underline{http://schrodinger.sourceforge.net}.

\subsection{Scope}
\label{introscope}

This standard defines the Dirac video compression codec through the stream syntax,
 entropy coding, coefficient unpacking and picture decoding processes. The decoder
 operations are defined by means of a mixture of pseudocode and mathematical 
operations.
 
Dirac is a long-GOP video codec that uses wavelet transforms and motion compensation
 together with entropy coding, that can be readily implemented in hardware or software.
Dirac is a superset of the proposed SMPTE VC-2 video codec standard which 
comprises the intra coding parts of this specification.

\subsection{Status}
\label{introstatus}

This is version $\SpecVersion$ of the Dirac specification. The document includes
a full description of the Dirac stream syntax and decoder operations.

This version is compatible with and extends Version 1.0 by the addition of motion
 compensated coding. Version 1.0 corresponds exactly to the proposed SMPTE VC-2
 video codec standard.

Subsequent versions of this specification may contain additional tools.

\subsection{Conformance notation}

Normative text is text that describes elements of the design that are indispensable or
 contains the conformance language keywords: "shall", "should", or "may". Informative 
text is text that is potentially helpful to the user, but not indispensable, and can be 
removed, changed, or added editorially without affecting interoperability. Informative 
text does not contain any conformance keywords.

All text in this document is, by default, normative, except: the Introduction, any section
 explicitly labeled as "Informative" or individual paragraphs that start with "Note:�
The keywords "shall" and "shall not" indicate requirements strictly to be followed in order
 to conform to the document and from which no deviation is permitted

The keywords, "should" and "should not" indicate that, among several possibilities, one 
is recommended as particularly suitable, without mentioning or excluding others; or that a
 certain course of action is preferred but not necessarily required; or that (in the negative
 form) a certain possibility or course of action is deprecated but not prohibited.

The keywords "may" and "need not" indicate courses of action permissible within the 
limits of the document.

The keyword �reserved� indicates a provision that is not defined at this time, shall not 
be used, and may be defined in the future. The keyword �forbidden� indicates �reserved�
 and in addition indicates that the provision will never be defined in the future.
A conformant implementation according to this document is one that includes all
 mandatory provisions ("shall") and, if implemented, all recommended provisions ("should") 
as described. A conformant implementation need not implement optional provisions 
("may") and need not implement them as described.

Unless otherwise specified the order of precedence of the types of normative 
information in this document shall be as follows. Normative prose shall be the authoritative 
definition. Tables shall be next, followed by formal languages, then figures, and then any
 other language forms.

